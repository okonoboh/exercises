\begin{enumerate}
%%%%%%%%%%%%%%%%%%%%%%%%%%%%%%%%%%%Problem 1%%%%%%%%%%%%%%%%%%%%%%%%%%%%%%%%%%%%
\item Prove the generalized version of the basic counting principle.

Suppose there are $r$ experiments such that the first one may result in any of
$n_1$ possible outcomes, and if for each of these $n_1$ possible outcomes there
are $n_2$ possible outcomes of the second experiment, and if for each of the
possible outcomes of the first two experiments there are $n_3$ possible outcomes
of the third experiment, and if $\ldots$, let $P(r)$ be the statement that
there is a total of $n_1 \cdot n_2 \cdots n_r$ possible outcomes of the $r$
experiments.

\underline{Solution}

$P(1)$ is vacuously true and $P(2)$ is the basic counting principle. Suppose
that $P(k)$ is true for some $k \ge 1$. We can regard $k + 1$ experiments as
two experiments, wherein the first experiment is the first $k$
experiments(with $P(k)$ outcomes) and the second experiment is the last
experiment(with $n_{k + 1}$ outcomes). So by the basic counting principle we
have
\[P(k + 1) = P(k) \cdot n_{k + 1} = n_1 \cdot n_2 \cdots n_k \cdot n_{k + 1}.\]

%%%%%%%%%%%%%%%%%%%%%%%%%%%%%%%%%%%Problem 2%%%%%%%%%%%%%%%%%%%%%%%%%%%%%%%%%%%%
\item Two experiments are to be performed. The first can result in any one of
$m$ possible outcomes. If the first experiment results in outcome number $i$,
then the second experiment can result in any of $n_i$ possible outcomes,
$i = 1, 2, \ldots, m$. What is the number of possible outcomes of the two
experiments?

\underline{Solution}

We enumerate the possible outcomes as follows:
\begin{equation*}
\begin{array}{cccc}
(1, 1), & (1, 2), & \ldots, & (1, n_1) \\
(2, 1), & (2, 2), & \ldots, & (2, n_2) \\
\vdots & \vdots & \ldots, & \ddots \\
(m, 1), & (m, 2), & \ldots, & (m, n_m)
\end{array}
\end{equation*}

So the total outcome is $\sum_{i=1}^mn_k$.

%%%%%%%%%%%%%%%%%%%%%%%%%%%%%%%%%%%Problem 3%%%%%%%%%%%%%%%%%%%%%%%%%%%%%%%%%%%%
\item In how many ways can $r$ objects be selected from a set of $n$ if the
order of selection is considered relevant?

\underline{Solution}

$n \cdot (n - 1) \cdot (n - 2) \cdots (n - r + 1)$.

%%%%%%%%%%%%%%%%%%%%%%%%%%%%%%%%%%%Problem 4%%%%%%%%%%%%%%%%%%%%%%%%%%%%%%%%%%%%
\item There are $\dbinom{n}{r}$ different linear arrangements of $n$ balls of
which $r$ are black and $n - r$ are white. Give a combinatorial explanation of
this fact.

\underline{Solution}

There are $n!$ permutations of the balls. By permuting the white and black balls
amongst themselves we see that there are $r!(n - r)!$ permutations corresponding
to a particular arrangement. Hence we have 
\[\frac{n!}{r!(n - r)!} = \binom{n}{r}\] linear arrangements.

%%%%%%%%%%%%%%%%%%%%%%%%%%%%%%%%%%%Problem 5%%%%%%%%%%%%%%%%%%%%%%%%%%%%%%%%%%%%
\item Determine the number of vectors ($x_1,\ldots, x_n$), such that each $x_i$
is either $0$ or $1$ and
\[\sum_{i=1}^nx_i \ge k.\]

\underline{Solution}

Let $k \in \mathbb{Z}$, $P = \{(x_1, \dots, x_n): \sum_{i=1}^nx_i \ge k\}$.
If $k$ is negative, then each $x_i$ can be either $0$ or $1$. For nonnegative
$k$, where $k \le n$, at least $k$ of the $n$ components must be $1$.
The remaining $(n - k)$ components may take on either $0$ or $1$.

\begin{equation*}
|P| = \left\{
\begin{array}{rl}
2^{n} & \text{if } k < 0,\\
\displaystyle{\sum_k^n\binom{n}{k}} & \text{if } 0 \le k \le n,\\
0 & \text{if } k > n.
\end{array} \right.
\end{equation*}

%%%%%%%%%%%%%%%%%%%%%%%%%%%%%%%%%%%Problem 6%%%%%%%%%%%%%%%%%%%%%%%%%%%%%%%%%%%%
\item How many vectors $x_1, \ldots, x_k$ are there for which each $x_i$ is a
positive integer such that $1 \le x_i \le n$ and $x_1 < x_2 < \cdots < x_k$?

\underline{Solution}

Let $n$ and $k$ be postive integers, $k \le n$, and let
\[P = \{(x_1, \dots, x_k): x_i \in \mathbb{Z}^+, 1 \le x_i \le n,
\mbox{ and } x_1 < x_2 < \cdots < x_k\}.\]


Also let $Q$ be the set of $k$-subsets of $I_n = \{1, 2, \ldots, n\}$. That is.
\[Q = \{r : r \subseteq I_n, |r| = k\}.\]
$P$ is nonempty since $\{1, 2, \ldots, k\} \in P$. So for any $p \in P$, we must
have $p \subseteq I_n$ and $|p| = k$; that is, $p \in Q$. Thus $P \subseteq Q$.
Since $P$ is embedded in $Q$, the latter is clearly nonempty. Take any
$q \in Q$. Since the order of a set does not matter and since a set contains
unique elements, we can rewrite the elements of $q$ in strictly ascending order.
That is, $q \in P$. Thus $Q \subseteq P$, or $Q = P$. But $|Q| = \dbinom{n}{k}$.
So $|P| = \dbinom{n}{k}$.

%%%%%%%%%%%%%%%%%%%%%%%%%%%%%%%%%%%Problem 7%%%%%%%%%%%%%%%%%%%%%%%%%%%%%%%%%%%%
\item Give an analytic proof of
\[\binom{n}{r} = \binom{n - 1}{r - 1} + \binom{n - 1}{r} \quad 1 \le r \le n\]

\underline{Solution}

We shall proceed from the RHS.

\begin{align*}
\binom{n - 1}{r - 1} + \binom{n - 1}{r} &=
\frac{(n - 1)!}{(r - 1)!(n - r)!} + \frac{(n - 1)!}{r!(n - 1 - r)!} \\
&= \frac{(n - 1)!}{(n - r)(r - 1)!(n - r - 1)!} +
\frac{(n - 1)!}{r(r - 1)!(n - 1 - r)!} \\
&= \frac{r(n - 1)! + (n - r)(n - 1)!}{r(n - r)(r - 1)!(n - 1 - r)!} \\
&= \frac{(n - 1)!(r + n - r)}{r(r - 1)!(n - r)(n - r - 1)!} \\
&= \frac{n(n - 1)!}{r(r - 1)!(n - r)(n - r - 1)!} \\
&= \frac{n!}{r!(n - r)!} \\
&= \binom{n}{r}.
\end{align*}

%%%%%%%%%%%%%%%%%%%%%%%%%%%%%%%%%%%Problem 8%%%%%%%%%%%%%%%%%%%%%%%%%%%%%%%%%%%%
\item Prove that
\[\binom{n + m}{r} = \binom{n}{0}\binom{m}{r} + \binom{n}{1}\binom{m}{r - 1} +
\cdots + \binom{n}{r}\binom{m}{0}\]

\underline{Solution}

Suppose that a committee of $r$ people are to be formed from a group of $n$
men and $m$ women. All possible $r$-people committees can be divided into
$r + 1$ groups such that a committee is in group $i$, $0 \le i \le r$, if and
only if it has $i$ men and $r - i$ women. Since the number of $r$-people
committee with $i$ men and $r - i$ women = $\binom{n}{i}\binom{m}{r - i}$,
we must have
\[\binom{n + m}{r} = \sum_{i=0}^r\binom{n}{i}\binom{m}{r - i}.\]

%%%%%%%%%%%%%%%%%%%%%%%%%%%%%%%%%%%Problem 9%%%%%%%%%%%%%%%%%%%%%%%%%%%%%%%%%%%%
\item Prove that
\[\binom{2n}{n} = \sum_{k=0}^n\binom{n}{k}^2\]

\underline{Solution}

Setting $n = m$ in the previous problem and using the fact that 
\[\binom{n}{i} = \binom{n}{r - i}\]
gives us the desired result.

%%%%%%%%%%%%%%%%%%%%%%%%%%%%%%%%%%%Problem 10%%%%%%%%%%%%%%%%%%%%%%%%%%%%%%%%%%%
\item From a group of $n$ people, suppose that we want to choose a committee of
$k$, $k \le n$, one of whom is to be designated as chairperson.

\begin{enumerate}
\item By focusing first on the choice of the committee and then on the choice of
the chair, argue that there are $\dbinom{n}{k}k$ possible choices.

\item By focusing first on the choice of the nonchair committee members and then
on the choice of the chair, argue that there are $\dbinom{n}{k - 1}(n - k + 1)$
possible choices.

\item By focusing first on the choice of the chair committee members and then
on the choice of the other committee members, argue that there are
$n\dbinom{n - 1}{k - 1}$ possible choices.

\item Conclude from parts $(a), (b)$, and $(c)$ that
\[k\binom{n}{k} = (n - k + 1)\binom{n}{k - 1} = n\binom{n - 1}{k - 1}\]

\item Use the factorial definition of $\dbinom{m}{r}$ to verify the identity in
part $(d)$.
\end{enumerate}

\underline{Solution}

\begin{enumerate}
\item There are $\dbinom{n}{k}$ ways of first choosing the committee and then
$k$ ways to choose the chairperson so that we will have $\dbinom{n}{k}k$ ways of
forming the committee.

\item There are $\dbinom{n}{k - 1}$ ways of first choosing the non-committee and
then $n - (k - 1)$ ways to choose the chairperson so that we will have
$\dbinom{n}{k - 1}(n - k + 1)$ ways of forming the committee.

\item There are $n$ ways of first choosing the chairperson and then
$\dbinom{n - 1}{k - 1}$ ways of choosing the non-committee so that we will have
$n\dbinom{n - 1}{k - 1}$ ways of forming the committee.

\item It follows from (a), (b), and (c) that
\[k\binom{n}{k} = (n - k + 1)\binom{n}{k - 1} = n\binom{n - 1}{k - 1}.\]

\item
\begin{align*}
k\binom{n}{k} &= \frac{k \cdot n!}{k!(n - k)!} \\
&= \frac{\cancel{k} \cdot n!}{\cancel{k}(k - 1)!(n - k)!} \\
&= \frac{n!}{(k - 1)!(n - k)!} \cdot \frac{n - k + 1}{n - k + 1} \\
&= \frac{(n - k + 1) \cdot n!}{(k - 1)!(n - k + 1)!} \\
& = (n - k + 1)\binom{n}{k - 1} \\
&= \frac{(n - k + 1) \cdot n!}{(k - 1)!(n - k + 1)!} \\
&= \frac{n(n - 1)!}{(k - 1)!(n - k)!} \\
&= n\binom{n - 1}{k - 1}.
\end{align*}
\end{enumerate}

%%%%%%%%%%%%%%%%%%%%%%%%%%%%%%%%%%%Problem 11%%%%%%%%%%%%%%%%%%%%%%%%%%%%%%%%%%%
\item The following identity is known as Fermat's combinatorial identity
\[\binom{n}{k} = \sum_{i=k}^n\binom{i - 1}{k - 1} \quad n \ge k\]

Give a combinatorial argument (no computations are needed) to establish this
identity.

\underline{Solution}

For any $k \le n$, where $k$ is a nonnegative integer and $n$ is a positive
integers. Consider the set $I_n = \{1, 2, \ldots, n\}$. Notice that the maximum
element in any $k$-subset of $I_n$ must be $\ge k.$ The $k$-subsets of $I_n$ 
can be divided into groups such that a $k$-subset is in group $i$,
$k \le i \le n$, if and only if its maximum element is $i$. In order to
determine the number of $k$-subsets with maximum element $i$, we first choose
$i$, then select the remaining $k - 1$ elements from the $i - 1$ elements less
than $i$. Since the number of $k$-subsets of $I_n$ is $\binom{n}{k}$, we must
have that:
\[\binom{n}{k} = \sum_{i=k}^n\binom{i - 1}{k - 1} \quad n \ge k\]

%%%%%%%%%%%%%%%%%%%%%%%%%%%%%%%%%%%Problem 12%%%%%%%%%%%%%%%%%%%%%%%%%%%%%%%%%%%
\item Present combinatorial arguements for the following identities:

\begin{enumerate}
\item \[\sum_{k=1}^nk\binom{n}{k} = n \cdot 2^{n-1}\]
\item \[\sum_{k=1}^n\binom{n}{k}k^2 = 2^{n-2}n(n + 1)\]
\item \[\sum_{k=1}^n\binom{n}{k}k^3 = 2^{n-3}n^2(n + 3)\]
\end{enumerate}

\underline{Solution}

Let $i \le n$, where $i$ and $n$ are positive integers.

\begin{enumerate}
\item Let $c_i$ be the number of ways a committee of $i$ people can be chosen
from a group of $n$ people. One of these $i$ people must also serve as the
chairperson. There are $\binom{n}{i}$ ways of first choosing the committee and
then $i$ ways of choosing the chairperson. Thus $c_i = \binom{n}{i}i$. Also
there are $n$ ways of first choosing the chairperson and $\binom{n-1}{i-1}$ ways
of choosing the remaining committee members, so that
\[c_i = \binom{n}{i}i = n\binom{n-1}{i-1}\]
Thus
\[\sum_{k=1}^nc_k = \sum_{k=1}^n\binom{n}{k}k = \sum_{k=1}^nn\binom{n-1}{k-1} = 
n \cdot \sum_{k=0}^{n-1}\binom{n-1}{k} = n \cdot 2^{n-1}\]

\item Suppose the $i$-person committee now has both a chairperson and a 
secretary (possibly the same person). It follows that $c_i = \binom{n}{i}i^2$. 
Let $c_{i_s}$ be the number of selections where the chairperson and the
secretary are the same person and let $c_{i_d}$ be the selections where they
are not. It is evident that $c_i = c_{i_s} + c_{i_d}$. By choosing a person to
be both the chairperson and secretary, and then selecting the rest of the
members, we see that $c_{i_s} = n\binom{n-1}{k-1}$. Similarly, 
$c_{i_d} = n(n-1)\binom{n-2}{k-2}$. Thus
\begin{align*}
\sum_{k=1}^nc_k &= \sum_{k=1}^n(c_{k_s} + c_{k_d}) \\ 
&= \sum_{k=1}^nn\binom{n-1}{k-1} + \sum_{k=2}^nn(n-1)\binom{n-2}{k-2} \\
&= n\sum_{k=0}^{n-1}\binom{n-1}{k} + n(n-1)\sum_{k=0}^{n-2}\binom{n-2}{k} \\
&= n2^{n-1} + n(n-1)2^{n-2} \\
&= 2^{n-2}(2n + n(n - 1) \\
&= 2^{n-2}n(n+1)
\end{align*}

\item Now suppose the $i$-person committee now has a chairperson, a secretary,
and a treasurer (all possibly the same person). It follows that
$c_i = \binom{n}{i}i^3$. Let $c_{i_s}$ be the number of selections where a 
person holds all postions; let $c_{i_t}$ be the number of selections where a
person holds two positions; and let $c_{i_d}$ be the selections where no person
holds more than one position. We have that
$c_i = c_{i_s} + c_{i_t} + c_{i_d}$. Reasoning as we did in (b), we have:
\begin{align*}
c_{i_s} &= n\binom{n-1}{i-1} \\
c_{i_t} &= 3n(n-1)\binom{n-2}{i-2} \\
c_{i_d} &= n(n-1)(n-2)\binom{n-3}{i-3}
\end{align*}
Thus
\begin{align*}
\sum_{k=1}^nc_k &= \sum_{k=1}^n(c_{k_s} + c_{k_t} + c_{k_d}) \\ 
&= \sum_{k=1}^nn\binom{n-1}{k-1} + \sum_{k=2}^n3n(n-1)\binom{n-2}{k-2} + 
\sum_{k=3}^nn(n-1)(n-2)\binom{n-3}{k-3} \\
&= n\sum_{k=0}^{n-1}\binom{n-1}{k} + 3n(n-1)\sum_{k=0}^{n-2}\binom{n-2}{k} + 
n(n-1)(n-2)\sum_{k=0}^{n-3}\binom{n-3}{k}\\
&= n2^{n-1} + 3n(n-1)2^{n-2} + n(n-1)(n-2)2^{n-3} \\
&= 2^{n-3}n(2^2 + 6(n-1) + (n-1)(n-2)) \\
&= 2^{n-3}n(n^2 + 3n) \\
&= 2^{n-3}n^2(n + 3)
\end{align*}
\end{enumerate}

%%%%%%%%%%%%%%%%%%%%%%%%%%%%%%%%%%%Problem 13%%%%%%%%%%%%%%%%%%%%%%%%%%%%%%%%%%%
\item Show that for $n > 0$,
\[\sum_{i=0}^n(-1)^i\binom{n}{i} = 0\]

\underline{Solution}

Using the binomial theorem:
\[0 = (-1+1)^n = \sum_{i=0}^n(-1)^i1^{n-i}\binom{n}{i} =
\sum_{i=0}^n(-1)^i\binom{n}{i}\]

%%%%%%%%%%%%%%%%%%%%%%%%%%%%%%%%%%%Problem 14%%%%%%%%%%%%%%%%%%%%%%%%%%%%%%%%%%%
\item From a set of $n$ people a committee of size $j$ is to be chosen, and from
this committee a subcommittee of size $i$, $i \le j$, is also to be chosen.
\begin{enumerate}
\item Derive a combinatorial identity by computing, in two ways, the number of 
possible choices of the committee and subcomittee---first by supposing that the
committee is chosen first and then the subcommittee, and second by supposing
that the subcommittee is chosen first and then the remaining members of the
committee are chosen.
\item Use the previous part to prove the following combinatorial identity:
\[\sum_{j=1}^n\binom{n}{j}\binom{j}{i} = \binom{n}{i}2^{n-i} \quad i \le n\]
\item Use the first part and Exercise 13 to show that
\[\sum_{j=1}^n\binom{n}{j}\binom{j}{i}(-1)^{n-j} = 0 \quad i \le n\]
\end{enumerate}

\underline{Solution}

\begin{enumerate}
\item \[\binom{n}{j}\binom{j}{i} = \binom{n}{i}\binom{n-i}{j-i}\]
\item Using the preceding and noting that $\dbinom{n-i}{j-i} = 0$ if $j < i$:
\begin{align*}
\sum_{j=1}^n\binom{n}{j}\binom{j}{i} &=
\binom{n}{i}\sum_{j=1}^n\binom{n-i}{j-i} \\
&= \binom{n}{i}\sum_{j=i}^n\binom{n-i}{j-i} \\
&= \binom{n}{i}\sum_{j=0}^{n-i}\binom{n-i}{j} \\
&= \binom{n}{i}2^{n-i}
\end{align*}
\item Using Exercise $13$ and the preceding:
\begin{align*}
\sum_{j=1}^n\binom{n}{j}\binom{j}{i}(-1)^{n-j} &=
\binom{n}{i}\sum_{j=0}^{n-i}\binom{n-i}{j}(-1)^{j} \\
&= \binom{n}{i} \cdot 0 = 0
\end{align*}
\end{enumerate}

%%%%%%%%%%%%%%%%%%%%%%%%%%%%%%%%%%%Problem 15%%%%%%%%%%%%%%%%%%%%%%%%%%%%%%%%%%%
\item Let $H_k(n)$ be the number of vectors $x_1, \dots, x_k$ for which each 
$x_i$ is a positive integer satisfying $1 \le x_i \le n$ and
$x_1 \le x_2 \le \cdots \le x_k$.

\begin{enumerate}
\item Without any computations argue that
\begin{align*}
H_1(n) &= n \\
H_k(n) &= \sum_{j=1}^nH_{k-1}(j) \quad k > 1
\end{align*} 
\item Use the preceding recursion to compute $H_3(5)$.
\end{enumerate}

\underline{Solution}

\begin{enumerate}
\item Let
\[P_n(k) = \{x_1,\dots, x_k : x_i \in \mathbb{N}, 1 \le x_i \le n \mbox{ and }
x_1 \le x_2 \le \cdots \le x_k \}\]
and for $j \in \{1, 2, \dots, n\}$, let
\[Q_n(k,j) = \{(x_1,\dots, x_k) \in P_n(k) : x_k = j\}\]

By definition
\[\bigcup_{i=1}^nQ_n(k,i) \subseteq P_n(k)\]

Since the last component of every member of $P_n(k)$ must be an element of
$\{1, 2, \ldots, n\}$, it follows that
\[P_n(k) \subseteq \bigcup_{i=1}^nQ_n(k,i)\]

That is:
\[P_n(k) = Q_n(k,1) \cup Q_n(k,2) \cup \cdots \cup Q_n(k,j)\]
and since the sets on the RHS are disjoint we must have that
\[H_k(n) = |P_n(k)| = \sum_{i=1}^n|Q_n(k,i)|\]
If $k = 1$, then $|Q_n(k,j)| = 1$ so that $H_1(n) = n$. So suppose $k > 1$.
For any $y \in Q_n(k,j)$, say $y = (y_1, \ldots, y_k)$, we must have that
$1 \le y_1 \le y_2 \cdots \le y_{k-1} \le y_k = j$; that is
$(y_1, y_2,\ldots,y_{k-1}) \in P_j(k - 1)$. The map
\[Q_n(k,j) \rightarrow P_j(k - 1)\] 
\[(y_1, y_2,\ldots,y_{k-1}, j) \mapsto (y_1, y_2,\ldots,y_{k-1})\]
is bijective; thus $|Q_n(k,j)| = |P_j(k - 1)| = H_{k-1}(j)$

\item 
\begin{align*}
H_3(5) &= H_2(1) + H_2(2) + H_2(3) + H_2(4) + H_2(5) \\
&= 1 + 3 + 6 + 10 + 15 = 35
\end{align*}
\end{enumerate}

%%%%%%%%%%%%%%%%%%%%%%%%%%%%%%%%%%%Problem 16%%%%%%%%%%%%%%%%%%%%%%%%%%%%%%%%%%%
\item Consider a tournament of $n$ contestants in which the outcome is an
ordering of these contestants, with ties allowed. That is, the outcome 
partitions the players into groups, with the first group consisting of the 
players that tied for first place, the next group being those that tied for the
next best position, and so on. Let $N(n)$ denote the number of different 
possible outcomes. For instance, $N(2) = 3$ since in a tournament with $2$ 
contestants, player $1$ could be uniquely first, player $2$ could be uniquely 
first, or they could tie for first.

\begin{enumerate}
\item List all the possible outcomes when $n = 3$.
\item With $N(0)$ defined to equal $1$, argue, without any computations, that
\[N(n) = \sum_{i=1}^n\binom{n}{i}N(n - i)\]
\item Show that the formula of the preceding part is equivalent to the 
following:
\[N(n) = \sum_{i=0}^{n-1}\binom{n}{i}N(i)\]
\item Use the recursion to find $N(3)$ and $N(4)$.
\end{enumerate}

\underline{Solution}

\begin{enumerate}
\item Let $p_k$ be player $k$.

\begin{tabular}{|c|c|c|} 
\hline
$1^{st}$ & $2^{nd}$ & $3^{rd}$ \\
\hline
$p_1$ & $p_2$ & $p_3$ \\
\hline
$p_2$ & $p_1$ & $p_3$ \\
\hline
$p_1$ & $p_3$ & $p_2$ \\
\hline
$p_3$ & $p_1$ & $p_2$ \\
\hline
$p_2$ & $p_3$ & $p_1$ \\
\hline
$p_3$ & $p_2$ & $p_1$ \\
\hline
$p_1$, $p_2$ & $p_3$ \\ \cline{1-2}
$p_1$, $p_3$ & $p_2$ \\ \cline{1-2}
$p_2$, $p_3$ & $p_1$ \\ \cline{1-2}
$p_1$ & $p_2$, $p_3$ \\ \cline{1-2}
$p_2$ & $p_1$, $p_3$ \\ \cline{1-2}
$p_3$ & $p_1$, $p_2$ \\ \cline{1-2}
$p_1$, $p_2$, $p_3$ \\ \cline{1-1}
\end{tabular} 

\item The equation works for $n = 1$. So assume $n > 1$. For
$i \in \{1, \ldots n\} = I_n$, define
\[P_i = \{\mbox{outcome with } i \mbox{ players in last place}\}\]
It follows that every outcome is a member of $P_k$ for some $k \in I_n$. Thus,
\[N(n) = \sum_{m=1}^n|P_m|\]
To compute $|P_k|$, we notice that we can select $\binom{n}{k}$ of the players
to be in the last position. For each of these selections, the remaining
($n-k$) players will be ordered into groups, which is equivalent to them
participating in their own contest. Thus $|P_k| = \binom{n}{k}N(n-k)$ and
\[N(n) = \sum_{i=1}^n\binom{n}{i}N(n - i)\]

\item \[N(n) = \sum_{i=1}^n\binom{n}{i}N(n - i) =
\sum_{i=1}^n\binom{n}{n-i}N(n - i) = \sum_{i=0}^{n-1}\binom{n}{i}N(i)\]
\item $N(3) = \binom{3}{1}N(2) + \binom{3}{2}N(1) + \binom{3}{3}N(0) = 13$. \\
$N(4) = \binom{4}{1}N(3) + \binom{4}{2}N(2) + \binom{4}{3}N(1) +
\binom{4}{4}N(0) = 75$.

\end{enumerate}

%%%%%%%%%%%%%%%%%%%%%%%%%%%%%%%%%%%Problem 17%%%%%%%%%%%%%%%%%%%%%%%%%%%%%%%%%%%
\item Present a combinatorial explanation of why
$\dbinom{n}{r} = \dbinom{n}{r,n-r}$.

\underline{Solution}

Let $n$ and $r$ be nonnegative integers, $r \le n$, $I_n = \{1, \ldots n \}$,
$A_1$ be the set of all unordered $r$ selections from $n$ items, and let
$A_2$ be the set of all possible divisions of $n$ into two distinct groups of 
sizes $n$ and $n - r$ respectively. That is $A_1 = \{P \subseteq I_n: |P| = r\}$
and $A_2 = \{(P, I_n\backslash P) : P \in A_1\}$. We can easily show that the 
map from $A_1$ to $A_2$ given by $P \mapsto (P, I_n\backslash P)$ is bijective;
that is $|A_1| = |A_2|$. Done.

%%%%%%%%%%%%%%%%%%%%%%%%%%%%%%%%%%%Problem 18%%%%%%%%%%%%%%%%%%%%%%%%%%%%%%%%%%%
\item Argue that
\begin{align*}
\binom{n}{n_1,n_2,\ldots,n_r} = &\binom{n-1}{n_1-1,n_2,\ldots,n_r}+
\binom{n-1}{n_1,n_2-1,\ldots,n_r} \\
&+\cdots+\binom{n-1}{n_1,n_2,\ldots,n_r-1}
\end{align*}

\underline{Solution}

Let $p$ be any positive integer $\le n$. Let $Q$ be the set of all divisions of
$n$ items into $r$ distinct groups, where the size of the $i$th group is $n_i$. 
We know that
\[|Q| = \binom{n}{n_1,n_2,\ldots,n_r}\]
For any $q \in Q$, we say that $q$ is in \textit{type} $i$ if $q$'s $i$th group 
contains $p$. So for each $k \in I_r$, define
\[X_k = \{q \in Q : q \mbox{ is in }type\mbox{ }k\}\]
Notice that every member of $Q$ belongs to exactly one $X_k$ for some
$k \in I_r$. Thus
\[|Q| = \binom{n}{n_1,n_2,\ldots,n_r} = \sum_{i=1}^r|X_k|\]
But $|X_1| = \dbinom{n-1}{n_1-1,n_2,\ldots,n_r}$,
$|X_2| = \dbinom{n-1}{n_1,n_2-1,\ldots,n_r}$, and so on. Done.
%%%%%%%%%%%%%%%%%%%%%%%%%%%%%%%%%%%Problem 19%%%%%%%%%%%%%%%%%%%%%%%%%%%%%%%%%%%
\item \textbf{Prove the multinomial theorem}.

\underline{Solution}

%%%%%%%%%%%%%%%%%%%%%%%%%%%%%%%%%%%Problem 20%%%%%%%%%%%%%%%%%%%%%%%%%%%%%%%%%%%
\item In how many ways can $n$ identical balls be distributed into $r$ urns so 
that the $i$th urn contains at least $m_i$ balls, for each $i = 1, \ldots, r$? 
Assume that $n \ge \sum_{i=1}^rm_i$.

\underline{Solution}

This problem is equivalent to finding the number of nonnegative integer 
solutions to
\[x_1 + x_2 + \cdots + x_r = n\]
where $x_i \ge m_i$. Let $y_i = x_i - m_i + 1$. Substituting this in the 
equation above results in
\[y_1 + y_2 + \cdots + y_r = n + r - \sum_{i=1}^rm_i\]
There are
\[\binom{n + r - \sum_{i=1}^rm_i}{r-1}\]
solutions to the above equation such that $y_i \ge 1$ or $x_i - m_i + 1 \ge 1$
or $x_i \ge m_i$. 

%%%%%%%%%%%%%%%%%%%%%%%%%%%%%%%%%%%Problem 21%%%%%%%%%%%%%%%%%%%%%%%%%%%%%%%%%%%
\item Argue that there are exactly $\dbinom{r}{k}\dbinom{n-1}{n-r+k}$ solutions
of
\[x_1+x_2+\cdots+x_r = n\]
for which exactly $k$ of the $x_i$ are equal to $0$.

\underline{Solution}

If $k$ of the $x_i$ are zero, then the rest must be positive integers. We can
choose $\binom{r}{k}$ of the $x_i$ to be zero. The remaining $(r-k)$ $x_i$ must
then add up to $n$. The number of ways this can happen is
$\binom{n-1}{n-(r-k)}$. Done.

%%%%%%%%%%%%%%%%%%%%%%%%%%%%%%%%%%%Problem 22%%%%%%%%%%%%%%%%%%%%%%%%%%%%%%%%%%%
\item Consider a function $f(x_1, \ldots, x_n)$ of $n$ variables. How many 
different partial derivatives of order $r$ does it possess.

\underline{Solution}

$r!$

%%%%%%%%%%%%%%%%%%%%%%%%%%%%%%%%%%%Problem 23%%%%%%%%%%%%%%%%%%%%%%%%%%%%%%%%%%%
\item Determine the number of vectors $(x_1, \ldots, x_n)$, such that each $x_i$
is a nonnegative integer and
\[\sum_{i=1}^nx_i \le k\]

\underline{Solution}

The number of vectors $(x_1, \ldots, x_n)$, such that each $x_i$
is a nonnegative integer and $\sum_{i=1}^nx_i = m$ is $\binom{m+n-1}{n-1}$. 
Hence, the required result is
\[\sum_{i=0}^k\binom{i+n-1}{n-1}\]

\end{enumerate}
