\noindent In all the exercises, $+$ is the usual addition on the set where it is
          specified, and $\cdot$ is the usual multiplication. \\

\noindent \textbf{Computations}
\begin{enumerate}
%%%%%%%%%%%%%%%%%%%%%%%%%%%%%%%%%%Prob3.1%%%%%%%%%%%%%%%%%%%%%%%%%%%%%%%%%%%%%%%
   \item[3.1] What three things must we check to determine whether a function
   $\phi : S \rightarrow S'$ is an isomorphism of a binary structure
   $\cyc{S, *}$ with $\cyc{S', *'}$?

      \textbf{Solution:}

      \begin{itemize}
         \item $\phi$ must be injective, and
         \item $\phi$ must be surjective, and
         \item $\phi(s * t) = \phi(s) *' \phi(t)$ must hold for all
               $s, t \in S$.
      \end{itemize}
\end{enumerate}

\noindent In Exercises 2 through 10, determine whether the given map $\phi$ is
          an isomorphism of the first binary structure with the second. If it is
          not an isomorphism, why not?

\begin{enumerate}
%%%%%%%%%%%%%%%%%%%%%%%%%%%%%%%%%%Prob3.2%%%%%%%%%%%%%%%%%%%%%%%%%%%%%%%%%%%%%%%
   \item[3.2] $\cyc{\Z, +}$ with $\cyc{\Z, +}$ where $\phi(n) = -n$ for
              $n \in \Z$.

      \textbf{Solution:} It is trivial to show that $\phi$ is bijective. We 
      claim that $\phi$ is an isomorphism because
      $\phi(z_1 + z_2) = \phi(z_1) + \phi(z_2)$ for any integers $z_1$ and
      $z_2$.
%%%%%%%%%%%%%%%%%%%%%%%%%%%%%%%%%%Prob3.3%%%%%%%%%%%%%%%%%%%%%%%%%%%%%%%%%%%%%%%
   \item[3.3] $\cyc{\Z, +}$ with $\cyc{\Z, +}$ where $\phi(n) = 2n$ for
              $n \in \Z$.

      \textbf{Solution:} The map $\phi$ is not an isomorphism because it is not
      surjective. This follows because an odd integer has no preimage under
      $\phi$.
%%%%%%%%%%%%%%%%%%%%%%%%%%%%%%%%%%Prob3.4%%%%%%%%%%%%%%%%%%%%%%%%%%%%%%%%%%%%%%%
   \item[3.4] $\cyc{\Z, +}$ with $\cyc{\Z, +}$ where $\phi(n) = n + 1$ for
              $n \in \Z$.

      \textbf{Solution:} The map $\phi$ is not an isomorphism because
      $\phi(0 + 0) = 1 \neq 2 = \phi(0) + \phi(0)$.
%%%%%%%%%%%%%%%%%%%%%%%%%%%%%%%%%%Prob3.5%%%%%%%%%%%%%%%%%%%%%%%%%%%%%%%%%%%%%%%
   \item[3.5] $\cyc{\Q, +}$ with $\cyc{\Q, +}$ where $\phi(x) = x/2$ for
              $x \in \Q$.

      \textbf{Solution:} It is trivial to show that $\phi$ is bijective. We 
      claim that $\phi$ is an isomorphism because
      $\phi(q_1 + q_2) = \phi(q_1) + \phi(q_2)$ for any rationals $q_1$ and
      $q_2$.
%%%%%%%%%%%%%%%%%%%%%%%%%%%%%%%%%%Prob3.6%%%%%%%%%%%%%%%%%%%%%%%%%%%%%%%%%%%%%%%
   \item[3.6] $\cyc{\Q, \cdot}$ with $\cyc{\Q, \cdot}$ where $\phi(x) = x^2$ for
              $x \in \Q$.

      \textbf{Solution:} The map $\phi$ is not an isomorphism because it is not
      injective. This follows because $\phi(1) = \phi(-1) = 1$, but $1 \neq -1$.
%%%%%%%%%%%%%%%%%%%%%%%%%%%%%%%%%%Prob3.7%%%%%%%%%%%%%%%%%%%%%%%%%%%%%%%%%%%%%%%
   \item[3.7] $\cyc{\R, \cdot}$ with $\cyc{\R, \cdot}$ where $\phi(x) = x^3$ for
              $x \in \R$.

      \textbf{Solution:} It is trivial to show that $\phi$ is bijective. We 
      claim that $\phi$ is an isomorphism because
      $\phi(r_1 \cdot r_2) = \phi(r_1) \cdot \phi(r_2)$ for any reals $r_1$ and
      $r_2$.
%%%%%%%%%%%%%%%%%%%%%%%%%%%%%%%%%%Prob3.8%%%%%%%%%%%%%%%%%%%%%%%%%%%%%%%%%%%%%%%
   \item[3.8] $\cyc{M_2(\R), \cdot}$ with $\cyc{\R, \cdot}$ where $\phi(A)$ is
              the determinant of matrix $A$.

      \textbf{Solution:} The map $\phi$ is not an isomorphism because it is not
      injective. This follows because $\phi\left(\left[
      \begin{tabular}{@{}l r@{}} 
         1 & 1 \\ 
         1 & 1
      \end{tabular}\right]\right) = \phi\left(\left[
      \begin{tabular}{@{}l r@{}} 
         0 & 0 \\ 
         0 & 0
      \end{tabular}\right]\right) = 0$, but $\left[
      \begin{tabular}{@{}l r@{}} 
         1 & 1 \\ 
         1 & 1
      \end{tabular}\right] \neq \left[
      \begin{tabular}{@{}l r@{}} 
         0 & 0 \\ 
         0 & 0
      \end{tabular}\right]$.
%%%%%%%%%%%%%%%%%%%%%%%%%%%%%%%%%%Prob3.9%%%%%%%%%%%%%%%%%%%%%%%%%%%%%%%%%%%%%%%
   \item[3.9] $\cyc{M_1(\R), \cdot}$ with $\cyc{\R, \cdot}$ where $\phi(A)$ is
              the determinant of matrix $A$.

      \textbf{Solution:} It is trivial to show that $\phi$ is bijective. We 
      claim that $\phi$ is an isomorphism because
      $\phi([m_1] \cdot [m_2]) = \phi([m_1]) \cdot \phi([m_2])$ for any real
      1 by 1 matrices $[m_1]$ and $[m_2]$.
%%%%%%%%%%%%%%%%%%%%%%%%%%%%%%%%%%Prob3.10%%%%%%%%%%%%%%%%%%%%%%%%%%%%%%%%%%%%%%
   \item[3.10] $\cyc{\R, +}$ with $\cyc{\R^+, \cdot}$ where $\phi(r) = 0.5^r$
               for $r \in \R$.

      \textbf{Solution:} It is trivial to show that $\phi$ is bijective. We 
      claim that $\phi$ is an isomorphism because
      $\phi(r_1 + r_2) = \phi(r_1) \cdot \phi(r_2)$ for any reals $r_1$ and
      $r_2$.
\end{enumerate}

\noindent In Exercises 11 through 15, let $F$ be the set of all functions $f$
          mapping $\R$ into $\R$ that have derivatives of all orders. Follow the
          instructions for Exercises 2 through 10.

\begin{enumerate}
%%%%%%%%%%%%%%%%%%%%%%%%%%%%%%%%%%Prob3.11%%%%%%%%%%%%%%%%%%%%%%%%%%%%%%%%%%%%%%
   \item[3.11] $\cyc{F, +}$ with $\cyc{F, +}$ where $\phi(f) = f'$, the
               derivative of $f$.

      \textbf{Solution:} The map $\phi$ is not an isomorphism because it is not
      injective. This follows because $\phi(x^2) = \phi(x^2 + 10) = 2x$, but
      $x^2 \neq x^2 + 10$.
%%%%%%%%%%%%%%%%%%%%%%%%%%%%%%%%%%Prob3.12%%%%%%%%%%%%%%%%%%%%%%%%%%%%%%%%%%%%%%
   \item[3.12] $\cyc{F, +}$ with $\cyc{\R, +}$ where $\phi(f) = f'(0)$.

      \textbf{Solution:} The map $\phi$ is not an isomorphism because it is not
      injective. This follows because $\phi(x^2) = \phi(x^3) = 0$, but
      $x^2 \neq x^3$.
%%%%%%%%%%%%%%%%%%%%%%%%%%%%%%%%%%Prob3.13%%%%%%%%%%%%%%%%%%%%%%%%%%%%%%%%%%%%%%
   \item[3.13] $\cyc{F, +}$ with $\cyc{F, +}$ where
               $\phi(f)(x) = \int_0^xf(t)dt$.

      \textbf{Solution:} The map $\phi$ is not an isomorphism because it is not
      surjective. This follows because for any $f \in F$, we have
      $\phi(f)(0) = 0$, but $\cos(x) \in F$ and $\cos(0) = 1$. Thus no member
      of $F$ maps onto $\cos(x)$, so that $\phi$ is not onto.
%%%%%%%%%%%%%%%%%%%%%%%%%%%%%%%%%%Prob3.14%%%%%%%%%%%%%%%%%%%%%%%%%%%%%%%%%%%%%%
   \item[3.14] $\cyc{F, +}$ with $\cyc{F, +}$ where
               $\phi(f)(x) = \frac{d}{dx}[\int_0^xf(t)dt]$.

      \textbf{Solution:} Using the Fundamental Theorem of Calculus we easily see
      that this is an isomorphism.
%%%%%%%%%%%%%%%%%%%%%%%%%%%%%%%%%%Prob3.15%%%%%%%%%%%%%%%%%%%%%%%%%%%%%%%%%%%%%%
   \item[3.15] $\cyc{F, \cdot}$ with $\cyc{F, \cdot}$ where
               $\phi(f)(x) = x \cdot f(x)$.

      \textbf{Solution:} The map $\phi$ is not an isomorphism because it is not
      surjective. This follows because for any $f \in F$, we have
      $\phi(f)(0) = 0$, but $\cos(x) \in F$ and $\cos(0) = 1$. Thus no member
      of $F$ maps onto $\cos(x)$, so that $\phi$ is not onto.
%%%%%%%%%%%%%%%%%%%%%%%%%%%%%%%%%%Prob3.16%%%%%%%%%%%%%%%%%%%%%%%%%%%%%%%%%%%%%%
   \item[3.16] The map $\phi : \Z \rightarrow \Z$ defined by $\phi(n) = n + 1$
               for $n \in \Z$ is one to one and onto $\Z$. Give the definition
               of a binary operation $*$ on $\Z$ such that $\phi$ is an
               isomorphism mapping \\
               \textbf{a.} $\cyc{\Z, +}$ onto $\cyc{\Z, *}$,
               \qquad\qquad\qquad\qquad\qquad\qquad\qquad\qquad
               \textbf{b.} $\cyc{\Z, *}$ onto $\cyc{\Z, +}$. \\
               In each case, give the identity element for $*$ on $\Z$.

      \textbf{Solution:}

      \begin{enumerate}
         \item Let $z_1$ and $z_2$ be integers. We want the following to be
               true: $\phi(z_1 + z_2) = \phi(z_1) * \phi(z_2)$. That is, we want
               $z_1 + z_2 + 1 = (z_1 + 1) * (z_2 + 1)$ to hold. Thus if we 
               define $z_1 * z_2 = z_1 + z_2 - 1$, then the homomorphism 
               condition will hold. It can be easily checked that $\cyc{\Z, *}$ 
               is an abelian group. We can see by inspection that the identity 
               element for $*$ on $\Z$ is 1.
         \item Let $z_1$ and $z_2$ be integers. We want the following to be
               true: $\phi(z_1 * z_2) = \phi(z_1) + \phi(z_2)$. That is, we want
               $(z_1 * z_2) + 1 = z_1 + z_2 + 2$ to hold. Thus if we 
               define $z_1 * z_2 = z_1 + z_2 + 1$, then the homomorphism 
               condition will hold. It can be easily checked that $\cyc{\Z, *}$ 
               is an abelian group. We can see by inspection that the identity 
               element for $*$ on $\Z$ is $-1$.
      \end{enumerate}
%%%%%%%%%%%%%%%%%%%%%%%%%%%%%%%%%%Prob3.17%%%%%%%%%%%%%%%%%%%%%%%%%%%%%%%%%%%%%%
   \item[3.17] The map $\phi : \Z \rightarrow \Z$ defined by $\phi(n) = n + 1$
               for $n \in \Z$ is one to one and onto $\Z$. Give the definition 
               of a binary operation $*$ on $\Z$ such that $\phi$ is an
               isomorphism mapping \\
               \textbf{a.} $\cyc{\Z, \cdot}$ onto $\cyc{\Z, *}$,
               \qquad\qquad\qquad\qquad\qquad\qquad\qquad\qquad
               \textbf{b.} $\cyc{\Z, *}$ onto $\cyc{\Z, \cdot}$. \\
               In each case, give the identity element for $*$ on $\Z$.

      \textbf{Solution:}

      \begin{enumerate}
         \item Let $z_1$ and $z_2$ be integers. We want the following to be
               true: $\phi(z_1 \cdot z_2) = \phi(z_1) * \phi(z_2)$. That is, we 
               want $z_1z_2 + 1 = (z_1 + 1) * (z_2 + 1)$ to hold. Thus if we 
               define $z_1 * z_2 = z_1z_2 - z_1 - z_2 + 2$, then the 
               homomorphism condition will hold. It can be easily checked that
               $\cyc{\Z, *}$ is an abelian group. A simple calculation will show
               that the identity element for $*$ on $\Z$ is 2.
         \item Let $z_1$ and $z_2$ be integers. We want the following to be
               true: $\phi(z_1 * z_2) = \phi(z_1) \cdot \phi(z_2)$. That is, we 
               want $(z_1 * z_2) + 1 = z_1z_2 + z_1 + z_2 + 1$ to hold. Thus if 
               we define $z_1 * z_2 = z_1z_2 + z_1 + z_2$, then the homomorphism
               condition will hold. It can be easily checked that $\cyc{\Z, *}$ 
               is an abelian group. A simple calculation will show that the
               identity element for $*$ on $\Z$ is 0.
      \end{enumerate}
%%%%%%%%%%%%%%%%%%%%%%%%%%%%%%%%%%Prob3.18%%%%%%%%%%%%%%%%%%%%%%%%%%%%%%%%%%%%%%
   \item[3.18] The map $\phi : \Q \rightarrow \Q$ defined by $\phi(x) = 3x - 1$
               for $x \in Q$ is one to one and onto $\Q$. Give the definition
               of a binary operation $*$ on $\Q$ such that $\phi$ is an
               isomorphism mapping \\
               \textbf{a.} $\cyc{\Q, +}$ onto $\cyc{\Q, *}$,
               \qquad\qquad\qquad\qquad\qquad\qquad\qquad\qquad
               \textbf{b.} $\cyc{\Q, *}$ onto $\cyc{\Q, +}$. \\
               In each case, give the identity element for $*$ on $\Q$.

      \textbf{Solution:}

      \begin{enumerate}
         \item Let $q_1$ and $q_2$ be rationals. We want the following to be
               true: $\phi(q_1 + q_2) = \phi(q_1) * \phi(q_2)$. That is, we want
               $3(q_1 + q_2) - 1 = (3q_1 - 1) * (3q_2 - 1)$ to hold. Thus if we 
               define $q_1 * q_2 = q_1 + q_2 + 1$, then the homomorphism 
               condition will hold. It can be easily checked that $\cyc{\Q, *}$ 
               is an abelian group. We can see by inspection that the identity 
               element for $*$ on $\Q$ is $-1$.
         \item Let $q_1$ and $q_2$ be rationals. We want the following to be
               true: $\phi(q_1 * q_2) = \phi(q_1) + \phi(q_2)$. That is, we want
               $3(q_1 * q_2) - 1 = 3q_1 + 3q_2 - 2$ to hold. Thus if we 
               define $q_1 * q_2 = q_1 + q_2 - 1/3$, then the homomorphism 
               condition will hold. It can be easily checked that $\cyc{\Q, *}$ 
               is an abelian group. We can see by inspection that the identity 
               element for $*$ on $\Z$ is $1/3$.
      \end{enumerate}
%%%%%%%%%%%%%%%%%%%%%%%%%%%%%%%%%%Prob3.19%%%%%%%%%%%%%%%%%%%%%%%%%%%%%%%%%%%%%%
   \item[3.19] The map $\phi : \Q \rightarrow \Q$ defined by $\phi(x) = 3x - 1$
               for $x \in Q$ is one to one and onto $\Q$. Give the definition
               of a binary operation $*$ on $\Q$ such that $\phi$ is an
               isomorphism mapping \\
               \textbf{a.} $\cyc{\Q, \cdot}$ onto $\cyc{\Q, *}$,
               \qquad\qquad\qquad\qquad\qquad\qquad\qquad\qquad
               \textbf{b.} $\cyc{\Q, *}$ onto $\cyc{\Q, \cdot}$. \\
               In each case, give the identity element for $*$ on $\Q$.

      \textbf{Solution:}

      \begin{enumerate}
         \item Let $q_1$ and $q_2$ be rationals. We want the following to be
               true: $\phi(q_1 \cdot q_2) = \phi(q_1) * \phi(q_2)$. That is, we 
               want $3q_1q_2 - 1 = (3q_1 - 1) * (3q_2 - 1)$ to hold. Thus if we 
               define $q_1 * q_2 = (q_1q_2 + q_1 + q_2 - 2)/3$, then the 
               homomorphism condition will hold. It can be easily checked that
               $\cyc{\Q, *}$ is an abelian group. A simple calculation will show
               that the identity element for $*$ on $\Q$ is 2.
         \item Let $q_1$ and $q_2$ be rationals. We want the following to be
               true: $\phi(q_1 * q_2) = \phi(q_1) \cdot \phi(q_2)$. That is, we 
               want $3(q_1 * q_2) - 1 = (3q_1 - 1) \cdot (3q_2 - 1)$ to hold. 
               Thus if we define $q_1 * q_2 = 3q_1q_2 - q_1 - q_2 + 2/3$, then 
               the homomorphism condition will hold. It can be easily checked 
               that $\cyc{\Q, *}$ is an abelian group. A simple calculation will
               show that the identity element for $*$ on $\Q$ is 2/3.
      \end{enumerate}
\end{enumerate}

\noindent \textbf{Concepts}

\begin{enumerate}
%%%%%%%%%%%%%%%%%%%%%%%%%%%%%%%%%%Prob3.20%%%%%%%%%%%%%%%%%%%%%%%%%%%%%%%%%%%%%%
   \item[3.20] The displayed homomorphism condition for an isomorphism $\phi$ in
               Definition 3.7 is sometimes summarized by saying, ``$\phi$ must
               commute with the binary operation(s)." Explain how that condition
               can be viewed in this manner.

      \textbf{Solution:} [See Manual.]
\end{enumerate}

\noindent In Exercises 21 and 22, correct the definition of the italicized term
          without reference to the text, if correction is needed, so that it is
          in a form acceptable for publication.

\begin{enumerate}
%%%%%%%%%%%%%%%%%%%%%%%%%%%%%%%%%%Prob3.21%%%%%%%%%%%%%%%%%%%%%%%%%%%%%%%%%%%%%%
   \item[3.21] A function $\phi : S \rightarrow S'$ is an \textit{isomorphism}
               if and only if $\phi(a * b) = \phi(a) *' \phi(b)$.

      \textbf{Solution:} A function $\phi : S \rightarrow S'$ is an
      \textit{isomorphism} if and only if $\phi(a * b) = \phi(a) *' \phi(b)$,
      for all $a, b \in S$ and where $*$ is a binary operation on $S$ and $*'$
      is a binary operation on $S'$.
%%%%%%%%%%%%%%%%%%%%%%%%%%%%%%%%%%Prob3.22%%%%%%%%%%%%%%%%%%%%%%%%%%%%%%%%%%%%%%
   \item[3.22] Let $*$ be a binary operation on a set $S$. An element $e$ of $S$
               is an \textit{identity element for} $*$ if and only if
               $s * e = s = e * s$ for all $s \in S$.

      \textbf{Solution:} [Badly worded]. Let $*$ be a binary operation on a set
      $S$. An element $e$ of $S$ is an \textit{identity element for} $*$ if and
      only if $s * e = s = e * s$ for all $s \in S$.

\end{enumerate}

\noindent \textbf{Proof Synopsis}

\noindent A good test of your understanding of a proof is your ability to give
          a one or two sentence synopsis of it, explaining the idea of the 
          proof without all the details and computations. Note that we said 
          ``sentence" and not ``equation." From now on, some of our exercise 
          sets may contain one or two problems asking for a synopsis of a proof 
          in the text. It should rarely exceed three sentences. We should
          illustrate for you what we mean by a synopsis. Here is our
          one-sentence synopsis of Theorem 3.14. Read the statement of the
          theorem now, and then our synopsis.
          \begin{quote}
             Representing as element of $S'$ ad $\phi(s)$ for some $s \in S$, 
             use the homomorphism property of $\phi$ to carry the computation of
             $\phi(e) *' \phi(s)$ back to a computation in $S$.
          \end{quote}

          That is the kind of explanation that one mathematician might give
          another if asked,   ``How does the proof go?" We did not make the
          computation or explain why we could represent an element of $S'$ as
          $\phi(s)$. To supply every detail would result in a completely written
          proof. We just gave the guts of the argument in our synopsis.

\begin{enumerate}
%%%%%%%%%%%%%%%%%%%%%%%%%%%%%%%%%%Prob3.23%%%%%%%%%%%%%%%%%%%%%%%%%%%%%%%%%%%%%%
   \item[3.23] Give a proof synopsis of Theorem 3.13.

      \textbf{Solution:} Assume two elements $e$ and $e'$ are identities for 
      some binary operation $*$ on some set. For each of them, use the property 
      of the element as an identity on the element $e * e'$.
\end{enumerate}
         

\noindent \textbf{Theory}

\begin{enumerate}
%%%%%%%%%%%%%%%%%%%%%%%%%%%%%%%%%%Prob3.24%%%%%%%%%%%%%%%%%%%%%%%%%%%%%%%%%%%%%%
   \item[3.24] An identity element for a binary operation $*$ as described by
               Definition 3.12 is sometimes referred to as ``a two-sided 
               identity element." Using complete sentences, give analogous
               definitions for \\
               \textbf{a.} a \textit{left identity element} $e_L$ for $*$, and
               \qquad\qquad
               \textbf{b.} a \textit{right identity element} $e_R$ for $*$. \\
               Theorem 3.13 shows that if a two-sided identity element for $*$
               exists, it is unique. Is the same true for a one-sided identity
               element you just defined? If so, prove it. If not, give a
               counterexample $\cyc{S, *}$ for a finite set $S$ and find the
               first place where the proof of Theorem 3.13 breaks down.

      \textbf{Solution:} Let $\cyc{S, *}$ be a binary structure.

      \begin{enumerate}
         \item An element $e_L \in S$ is said to be a left identity for $*$ if
               and only if $e_L * s = s$ for all $s \in S$.
         \item An element $e_R \in S$ is said to be a right identity for $*$ if
               and only if $s * e_R = s$ for all $s \in S$.
      \end{enumerate}

      Consider the binary structure $\cyc{\{a, b\}, *}$, where
      $a * a = b * a = a$, and $a * b = b * b = b$. From this definition, we see
      that $a$ and $b$ are both left identities for $*$, but $a \neq b$. If we
      tried to prove that a left identity for $*$ is unique, then we would have
      proceeded like so: Suppose $e_L$ and $e_L'$ are two left identities for
      $*$. Then we have $e_L * e_L' = e_L'$ by definition of $e_L$. But we can't
      do the same for $e_L'$ since it is also a left identity. One can't show
      the aforementioned for a right identity.
%%%%%%%%%%%%%%%%%%%%%%%%%%%%%%%%%%Prob3.25%%%%%%%%%%%%%%%%%%%%%%%%%%%%%%%%%%%%%%
   \item[3.25] Continuing the ideas of Exercise 24 can a binary structure have a
               left identity element $e_L$ and a right identity element $e_R$ 
               where $e_L \neq e_R$? If so, give an example, using an operation
               on a finite set $S$. If not, prove that it is impossible.

      \textbf{Proof:} Let $\cyc{S, *}$ be a binary structure with a left
      identity, $e_L$, and a right identity, $e_R$. We claim that $e_L = e_R$.
      By virtue of $e_L$ as a left identity, we have $e_L * e_R = e_R$. Also, by
      virtue of $e_R$ as a right identity, we have $e_L * e_R = e_L$, so that
      $e_L = e_R$. \qed
%%%%%%%%%%%%%%%%%%%%%%%%%%%%%%%%%%Prob3.26%%%%%%%%%%%%%%%%%%%%%%%%%%%%%%%%%%%%%%
   \item[3.26] Recall that if $f : A \rightarrow B$ is a one-to-one function
               mapping $A$ onto $B$, then $f^{-1}(b)$ is the unique $a \in A$
               such that $f(a) = b$. Prove that if $\phi : S \rightarrow S'$ is
               an isomorphism of $\cyc{S, *}$ with $\cyc{S', *'}$, then
               $\phi^{-1}$ is an isomorphism of $\cyc{S', *'}$ with
               $\cyc{S, *}$.

      \textbf{Proof:} Let $s'$ and $t'$ be members of $S'$. In order to show 
      that $\phi^{-1} : S' \rightarrow S$ is an isomorphism, we must show that
      $\phi^{-1}(s' *' t') = \phi^{-1}(s') * \phi^{-1}(t')$. There exist unique
      $s, t \in S$ such that $\phi(s) = s'$ and $\phi(t) = t'$. Since $\phi$ is
      an isomorphism, we have $\phi(s * t) = \phi(s) *' \phi(t) = s' *' t'$. 
      Thus $\phi^{-1}(s' *' t') = s * t = \phi^{-1}(s') * \phi^{-1}(t')$. \qed
%%%%%%%%%%%%%%%%%%%%%%%%%%%%%%%%%%Prob3.27%%%%%%%%%%%%%%%%%%%%%%%%%%%%%%%%%%%%%%
   \item[3.27] Prove that if $\phi : S \rightarrow S'$ is an isomorphism of
               $\cyc{S, *}$ with $\cyc{S', *'}$ and $\psi : S' \rightarrow S''$
               is an isomorphism of $\cyc{S', *'}$ with $\cyc{S'', *''}$, then
               the composite function $\psi \circ \phi$ is an isomorphism of
               $\cyc{S, *}$ with $\cyc{S'', *''}$.

      \textbf{Proof:} Let $s, t \in S$. Then \begin{align*}
         (\psi \circ \phi)(s * t) &= \psi(\phi(s * t)) \\
         &= \psi(\phi(s) *' \phi(t)) &[\phi\text{ is an isomorphism}] \\
         &= \psi(\phi(s)) *'' \psi(\phi(t)) &[\psi\text{ is an isomorphism}] \\
         &= (\psi\circ\phi)(s) *'' (\psi\circ\phi)(t).
      \end{align*}

      Since $(\psi \circ \phi)(s * t) = (\psi\circ\phi)(s) *''
      (\psi\circ\phi)(t)$, it follows that $\psi \circ \phi$ is an isomorphism 
      of  $\cyc{S, *}$ with $\cyc{S'', *''}$.
%%%%%%%%%%%%%%%%%%%%%%%%%%%%%%%%%%Prob3.28%%%%%%%%%%%%%%%%%%%%%%%%%%%%%%%%%%%%%%
   \item[3.28] Prove that the relation $\simeq$ of being isomorphic, described 
               in Definition 3.7, is an equivalence relation on any set of 
               binary structures. You may simply quote the results you were 
               asked to prove in the preceding two exercises at appropriate 
               places in your proof.

      \textbf{Proof:} Let $\mathcal{S}$ be a set of binary structures. Consider
      $S, T, U \in \mathcal{S}$. We have $S \simeq S$, since the map
      $\phi_1 : S \rightarrow S$, given by $\phi_1(s) = s$ is an isomorphism.
      Thus $\simeq$ is reflexive. Suppose $S \simeq T$. Then there exists an
      isomorphic map $\phi_2 : S \rightarrow T$. By Exercise 3.26, we know that
      the map $\phi_2^{-1} : T \rightarrow S$ is isomorphic, so that
      $T \simeq S$. That is, $\simeq$ is symmetric. Now suppose that
      $S \simeq T$ and $T \simeq U$. That is, there exist isomorphic maps
      $\phi_3 : S \rightarrow T$ and $\phi_4 : T \rightarrow U$. By Exercise 
      3.27 it follows that the map $\phi_4 \circ \phi_3 : S \rightarrow T$ is
      isomorphic, so that $S \simeq T$. That is, $\simeq$ is transitive. Since
      $\simeq$ is reflexive, symmetric, and transitive on $\mathcal{S}$, then it
      is an equivalence relation. \qed
\end{enumerate}

\noindent In Exercises 29 through 32, give a careful proof for a skeptic that
          the indicated property of a binary structure $\cyc{S, *}$ is indeed a
          structural property.

\begin{enumerate}
%%%%%%%%%%%%%%%%%%%%%%%%%%%%%%%%%%Prob3.29%%%%%%%%%%%%%%%%%%%%%%%%%%%%%%%%%%%%%%
   \item[3.29] The operation $*$ is commutative.

      \textbf{Proof:} Suppose $\cyc{S, *} \simeq \cyc{S', *'}$. We want to show
      that $*'$ is also commutative. We know there exists some isomorphic
      map $\phi : S \rightarrow S'$. Let $s', t' \in S'$. Then there exist
      unique $s, t \in S$ such that $\phi(s) = s'$ and $\phi(t) = t'$. So
      we have \begin{align*}
         s' *' t' &= \phi(s) *' \phi(t) \\ 
                  &= \phi(s * t) &[\phi\text{ is an homomorphism}] \\
                  &= \phi(t * s) &[* \text{ is commutative}] \\
                  &= \phi(t) *' \phi(s) &[\phi \text{ is an homomorphism}] \\
                  &= t' *' s'.
      \end{align*}

      Since $s' *' t' = t' *' s'$, it follows that $*'$ is commutative, so that
      commutativity is a structural property. \qed
%%%%%%%%%%%%%%%%%%%%%%%%%%%%%%%%%%Prob3.30%%%%%%%%%%%%%%%%%%%%%%%%%%%%%%%%%%%%%%
   \item[3.30] The operation $*$ is associative.

      \textbf{Proof:} Suppose $\cyc{S, *} \simeq \cyc{S', *'}$. We want to show
      that $*'$ is also associative. We know there exists some isomorphic
      map $\phi : S \rightarrow S'$. Let $s', t', u' \in S'$. Then there exist
      unique $s, t, u \in S$ such that $\phi(s) = s'$, $\phi(t) = t'$, and
      $\phi(u) = u'$. So
      we have \begin{align*}
         s' *' (t' *' u') &= \phi(s) *' (\phi(t) *' \phi(u)) \\ 
                  &= \phi(s) *' \phi(t * u) &[\phi\text{ is an homomorphism}] \\
                  &= \phi(s * (t * u)) &[\phi\text{ is an homomorphism}] \\
                  &= \phi((s * t) * u) &[*\text{ is associative}] \\
                  &= \phi(s * t) *' \phi(u) &[\phi\text{ is an homomorphism}] \\
                  &= (\phi(s) *' \phi(t)) *' \phi(u)
                     &[\phi\text{ is an homomorphism}] \\
                  &= (s' *' t') *' u'.
      \end{align*}

      Since $s' *' (t' *' u') = (s' *' t') *' u'$, it follows that $*'$ is 
      associative. \qed
%%%%%%%%%%%%%%%%%%%%%%%%%%%%%%%%%%Prob3.31%%%%%%%%%%%%%%%%%%%%%%%%%%%%%%%%%%%%%%
   \item[3.31] For each $c \in S$, the equation $x * x = c$ has a solution $x$
               in $S$.

      \textbf{Proof:} Suppose $\cyc{S, *} \simeq \cyc{S', *'}$. Let $c' \in S'$.
      We want to show that the equation $x' *' x' = c'$ has a solution for some 
      $x' \in S'$. By our supposition, it follows that there exists some 
      isomorphic map $\phi : S \rightarrow S'$. Then there exists a unique
      $c \in S$ such that $\phi(c) = c'$. Also, there  exists $y \in S$ such 
      that $y * y = c$, so that $\phi(y * y) = \phi(c)$. Since $\phi$ is a
      homomorphism, we have that $\phi(y) *' \phi(y) = c'$. So we have
      $x' = \phi(y)$. \qed
%%%%%%%%%%%%%%%%%%%%%%%%%%%%%%%%%%Prob3.32%%%%%%%%%%%%%%%%%%%%%%%%%%%%%%%%%%%%%%
   \item[3.32] There exists an element $b$ in $S$ such that $b * b = b$.

      \textbf{Proof:} Suppose $\cyc{S, *} \simeq \cyc{S', *'}$. We want to show
      that there exists some $b' \in S'$ such that $b' *' b' = b'$. By our 
      supposition, it follows that there exists some isomorphic map
      $\phi : S \rightarrow S'$. We also know that there exists $b$ in $S$ such 
      that $b * b = b$. Hence $\phi(b * b) = \phi(b)$, so that
      $\phi(b) *' \phi(b) = \phi(b)$. We let $b' = \phi(b)$. \qed
%%%%%%%%%%%%%%%%%%%%%%%%%%%%%%%%%%Prob3.33%%%%%%%%%%%%%%%%%%%%%%%%%%%%%%%%%%%%%%
   \item[3.33] Let $H$ be the subset of $M_2(\R)$ consisting of all matrices of
               the form $\left[
               \begin{tabular}{@{}l r@{}} 
                  $a$ & $-b$ \\ 
                  $b$ & $a$
               \end{tabular}\right]$
               for $a, b \in \R$. Exercise 23 of Section 2 shows that $H$ is
               closed under both matrix addition and matrix multiplication.

               \begin{enumerate}
                  \item Show that $\cyc{\C, +}$ is isomorphic to $\cyc{H, +}$.
                  \item Show that $\cyc{\C, \cdot}$ is isomorphic to
                        $\cyc{H, \cdot}$.
               \end{enumerate}
               (We say that $H$ is a \textit{matrix representation} of the
               complex numbers $\C$.)

      \textbf{Proof:}
      \begin{enumerate}
         \item The map $\phi : \cyc{\C, +} \rightarrow \cyc{H, +}$, defined by
               $\phi(a + bi) = \left[
               \begin{tabular}{@{}l r@{}} 
                  $a$ & $-b$ \\ 
                  $b$ & $a$
               \end{tabular}\right]$ is a bijection and a homomorphism.
         \item The map $\phi : \cyc{\C, \cdot} \rightarrow \cyc{H, \cdot}$, 
               defined by $\phi(a + bi) = \left[
               \begin{tabular}{@{}l r@{}} 
                  $a$ & $-b$ \\ 
                  $b$ & $a$
               \end{tabular}\right]$ is a bijection and a homomorphism.
      \end{enumerate}
%%%%%%%%%%%%%%%%%%%%%%%%%%%%%%%%%%Prob3.34%%%%%%%%%%%%%%%%%%%%%%%%%%%%%%%%%%%%%%
   \item[3.34] There are 16 possible binary structures on the set $\{a, b\}$ of
               two elements. How many nonisomorphic structures are there amongst
               these 16? Phrased more precisely in terms of the isomorphism
               equivalence relation $\simeq$ on this set of 16 structures, how
               many equivalence classes are there? Write down one structure from
               each equivalence class.

      \textbf{Solution:} TODO
\end{enumerate}
