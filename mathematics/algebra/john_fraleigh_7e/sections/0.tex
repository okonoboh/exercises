In Exercises 1 through 4, describe the set by listing its elements.
\begin{enumerate}
%%%%%%%%%%%%%%%%%%%%%%%%%%%%%%%%%%Prob0.1%%%%%%%%%%%%%%%%%%%%%%%%%%%%%%%%%%%%%%%
   \item[0.1] $\{x \in \R : x^2 = 3\}$.

      \textbf{Solution:} $\{x \in \R : x^2 = 3\} = \{-\sqrt{3}, \sqrt{3}\}$.
%%%%%%%%%%%%%%%%%%%%%%%%%%%%%%%%%%Prob0.2%%%%%%%%%%%%%%%%%%%%%%%%%%%%%%%%%%%%%%%
   \item[0.2] $\{m \in \Z : m^2 = 3\}$.

      \textbf{Solution:} $\{m \in \Z : m^2 = 3\} = \emptyset$.
%%%%%%%%%%%%%%%%%%%%%%%%%%%%%%%%%%Prob0.3%%%%%%%%%%%%%%%%%%%%%%%%%%%%%%%%%%%%%%%
   \item[0.3] $\{m \in \Z : mn = 60 \mbox{ for some } n \in \Z\}$.

      \textbf{Solution:} $\{m \in \Z : mn = 60 \mbox{ for some } n \in \Z\}$ = 
      $$\{0, \pm1, \pm2, \pm3, \pm4, \pm5, \pm6,
          \pm10, \pm12, \pm15, \pm20, \pm30, \pm60\}.$$
%%%%%%%%%%%%%%%%%%%%%%%%%%%%%%%%%%Prob0.4%%%%%%%%%%%%%%%%%%%%%%%%%%%%%%%%%%%%%%%
   \item[0.4] $\{m \in \Z : m^2 - m < 115\}$.

      \textbf{Solution:} $\{m \in \Z : m^2 - m < 115\} =
      \{-10, -9, \ldots, 11\}$.
\end{enumerate}

\noindent In Exercises 5 through 10, decide whether the object described is 
indeed a set (is well defined). Give an alternate description of each set.
 
\begin{enumerate}
%%%%%%%%%%%%%%%%%%%%%%%%%%%%%%%%%%Prob0.5%%%%%%%%%%%%%%%%%%%%%%%%%%%%%%%%%%%%%%%
   \item[0.5] $\{n \in \Z^+ : n \mbox{ is a large number }\}$.

      \textbf{Solution:} A large number is not well defined.
%%%%%%%%%%%%%%%%%%%%%%%%%%%%%%%%%%Prob0.6%%%%%%%%%%%%%%%%%%%%%%%%%%%%%%%%%%%%%%%
   \item[0.6] $\{n \in \Z : n^2 < 0\}$.

      \textbf{Solution:} $\{n \in \Z : n^2 < 0\} = \emptyset$.
%%%%%%%%%%%%%%%%%%%%%%%%%%%%%%%%%%Prob0.7%%%%%%%%%%%%%%%%%%%%%%%%%%%%%%%%%%%%%%%
   \item[0.7] $\{n \in \Z : 39 < n^3 < 57\}$.

      \textbf{Solution:} $\{n \in \Z : 39 < n^3 < 57\} = \emptyset$.
%%%%%%%%%%%%%%%%%%%%%%%%%%%%%%%%%%Prob0.8%%%%%%%%%%%%%%%%%%%%%%%%%%%%%%%%%%%%%%%
   \item[0.8] $\{x \in \Q : x \mbox{ is almost an integer}\}$.

      \textbf{Solution:} Almost an integer is not well defined.
%%%%%%%%%%%%%%%%%%%%%%%%%%%%%%%%%%Prob0.9%%%%%%%%%%%%%%%%%%%%%%%%%%%%%%%%%%%%%%%
   \item[0.9] $\{x \in \Q : x \mbox{ may be written with denominator greater 
                 than } 100\}$.

      \textbf{Solution:} Let $a$ be a rational number. We can write $a$ as
      $p/q$ where $q$ is a positive integer and $p$ is some integer. Since
      $q \ge 1$ and $200 > 100$, it follows that $200q > 100$. Thus, we have
      $p/q = 200p/200q$, so that all rational numbers can be written with a
      denominator greater than 100. It follows that
      $$\{x \in \Q : x \mbox{ may be written with denominator greater than }
        100\} = \Q.$$
%%%%%%%%%%%%%%%%%%%%%%%%%%%%%%%%%%Prob0.10%%%%%%%%%%%%%%%%%%%%%%%%%%%%%%%%%%%%%%
   \item[0.10] $\{x \in \Q : x \mbox{ may be written with positive denominator 
                 less than } 4\}$.

      \textbf{Solution:} 
      $$\Z \cup \{a + 1/2 : a \in \Z\} \cup \{a + 1/3 : a \in \Z\} \cup
        \{a + 2/3 : a \in \Z\}.$$
%%%%%%%%%%%%%%%%%%%%%%%%%%%%%%%%%%Prob0.11%%%%%%%%%%%%%%%%%%%%%%%%%%%%%%%%%%%%%%
   \item[0.11] List the elements in $\{a, b, c\} \times \{1, 2,  c\}$.

      \textbf{Solution:}

      $\{\{a, 1\}, \{a, 2\}, \{a, c\}, \{b, 1\}, \{b, 2\},
       \{b, c\}, \{c, 1\}, \{c, 2\}, \{c, c\}\}$.
%%%%%%%%%%%%%%%%%%%%%%%%%%%%%%%%%%Prob0.12%%%%%%%%%%%%%%%%%%%%%%%%%%%%%%%%%%%%%%
   \item[0.12] Let $A = \{1, 2, 3\}$ and $B = \{2, 4, 6\}$. For each relation
               between $A$ and $B$ given as a subset of $A \times B$, decide
               whether it is a function mapping $A$ into $B$. If it is a 
               function, decide whether it is one to one and whether it is onto
               $B$.

               \textbf{a.} $\{(1,4),(2,4),(3,6)\}$
               \qquad\qquad\qquad\qquad\qquad\qquad\qquad 
               \textbf{b.} $\{(1,4),(2,6),(3,4)\}$ \\
               \textbf{c.} $\{(1,6),(1,2),(1,4)\}$
               \qquad\qquad\qquad\qquad\qquad\qquad\qquad 
               \textbf{d.} $\{(2,2),(1,6),(3,4)\}$ \\
               \textbf{e.} $\{(1,6),(2,6),(3,6)\}$
               \qquad\qquad\qquad\qquad\qquad\qquad\qquad 
               \textbf{f.} $\{(1,2),(2,6),(2,4)\}$

      \textbf{Solution}

      \begin{tabular}{@{}|c|c|c|c|c|@{}}
         \hline & Relation & Is a function? & Is 1-1? & Is onto? \\ \hline
         \textbf{a.} & $\{(1,4),(2,4),(3,6)\}$ & Yes & No & No\\ \hline
         \textbf{b.} & $\{(1,4),(2,6),(3,4)\}$ & Yes & No & No\\ \hline
         \textbf{c.} & $\{(1,6),(1,2),(1,4)\}$ & No & $-$ & $-$\\ \hline
         \textbf{d.} & $\{(2,2),(1,6),(3,4)\}$ & Yes & Yes & Yes\\ \hline
         \textbf{e.} & $\{(1,6),(2,6),(3,6)\}$ & Yes & No & No\\ \hline
         \textbf{f.} & $\{(1,2),(2,6),(2,4)\}$ & No & $-$ & $-$\\ \hline
      \end{tabular}
%%%%%%%%%%%%%%%%%%%%%%%%%%%%%%%%%%Prob0.13%%%%%%%%%%%%%%%%%%%%%%%%%%%%%%%%%%%%%%
   \item[0.13] Illustrate geometrically that two line segments $AB$ and $CD$ of
               different length have the same number of points by indicating in
               Fig. 0.23 what point $y$ of $CD$ might be paired with point $x$
               of $AB$.

      \textbf{Solution} [Geometry]
%%%%%%%%%%%%%%%%%%%%%%%%%%%%%%%%%%Prob0.14%%%%%%%%%%%%%%%%%%%%%%%%%%%%%%%%%%%%%%
   \item[0.14] Recall that for $a, b \in \R$ and $a < b$, the \textbf{closed
               interval} $[a, b]$ in $\R$ is defined by
               $[a, b] = \{x \in \R : a \le x \le b\}$. Show that the given
               intervals have the same cardinality by giving a formula for a
               one-to-one function $f$ mapping the first interval onto the
               second.

               \textbf{a.} $[0, 1]$ and $[0, 2]$
               \qquad\qquad\qquad
               \textbf{b.} $[1, 3]$ and $[5, 25]$
               \qquad\qquad\qquad
               \textbf{c.} $[a, b]$ and $[c, d]$

      \textbf{Solution}

      \begin{enumerate}
         \item $f : [0, 1] \rightarrow [0, 2]$, $f(x) = 2x$.
         \item $g : [1, 3] \rightarrow [5, 25]$, $g(x) = 10x - 5$.
         \item $h : [a, b] \rightarrow [c, d]$, $h(x) =
               \displaystyle\frac{d - c}{b - a}x + \frac{bc - ad}{b - a}$.
      \end{enumerate}
%%%%%%%%%%%%%%%%%%%%%%%%%%%%%%%%%%Prob0.15%%%%%%%%%%%%%%%%%%%%%%%%%%%%%%%%%%%%%%
   \item[0.15] Show that $S = \{x \in \R : 0 < x < 1\}$ has the same cardinality
               as $\R$.

      \textbf{Proof:} To show that $S$ and $\R$ have the same cardinality it 
      suffices to find a   bijective function $f : S \rightarrow \R$. The
      tangent function
      $$\tan : \left(-\frac{\pi}{2}, \frac{\pi}{2}\right) \rightarrow \R$$
      is bijective and so is
      $$h : S \rightarrow \left(-\frac{\pi}{2}, \frac{\pi}{2}\right),$$
      defined by $h(x) = \pi x - \pi/2$. So define $f(x) = \tan(h(x))$, so that
      $f$ is bijective since it is a composition of bijective functions. \qed
%%%%%%%%%%%%%%%%%%%%%%%%%%%%%%%%%%Prob0.16%%%%%%%%%%%%%%%%%%%%%%%%%%%%%%%%%%%%%%
   \item[0.16] List the elements of the power set of the given set and give the
               cardinality of the power set.

               \textbf{a.} $\emptyset$
               \qquad\qquad
               \textbf{b.} $\{a\}$
               \qquad\qquad
               \textbf{c.} $\{a, b\}$
               \qquad\qquad
               \textbf{d.} $\{a, b, c\}$

      \textbf{Solution}

      \begin{enumerate}
         \item $\{\emptyset\}$.
         \item $\{\emptyset, \{a\}\}$.
         \item $\{\emptyset, \{a\}, \{b\}, \{a, b\}\}$.
         \item $\{\emptyset, \{a\}, \{b\}, \{c\},
                \{a, b\}, \{b, c\}, \{a, c\}, \{a, b, c\}\}$.
      \end{enumerate}
%%%%%%%%%%%%%%%%%%%%%%%%%%%%%%%%%%Prob0.17%%%%%%%%%%%%%%%%%%%%%%%%%%%%%%%%%%%%%%
   \item[0.17] Let $A$ be a finite set, and let $|A| = s$. Based on the
               preceding exercise, make a conjecture about the value of
               $|\mathscr{P}(A)|$. Then try to prove your conjecture.
               
      \textbf{Conjecture:} Let $s$ be some natural number. Claim that if $A$ is
      a finite set with $|A| = s$, then $|\mathscr{P}(A)| = 2^s$.
      
      \textbf{Proof:} Let $P(n)$ be the statement that if a finite set has 
      cardinality $n$, then the cardinality of its power set is $2^n$. We shall
      show by Mathematical Induction that $P(n)$ holds for each natural number
      $n$. For the base case, $n = 1$, consider any set with a single element,
      say $\{a_1\}$. It follows that $|\mathscr{P}(\{a_1\})| = 2^1$ since
      $\mathscr{P}(\{a_1\}) = \{\emptyset, \{a_1\}\}$, so that $P(1)$ is true.
      Now suppose that $P(k)$ is true for some natural number $k$. Consider the
      sets
      $$B = \{b_1, b_2, \ldots, b_k\}, C = B \cup \{c_1\},$$
      where $b_i = b_j$ implies $i = j$ and $c_1 \notin B$. Let $C_{c_1}$ be the
      set of subsets of $C$ that contain $c_1$, and let $C_{c_1}'$ denote the
      set of subsets of $C$ that do not contain $c_1$. We note that $C_{c_1}$
      and $C_{c_1}'$ are disjoint and also that
      $$\mathscr{P}(C) = C_{c_1} \cup C_{c_1}',$$
      so that
      $$|\mathscr{P}(C)| = |C_{c_1}| + |C_{c_1}'|.$$
      
      We observe that the subsets of $C$ that do not contain $c_1$ are simply
      the subsets of $B$. Thus $C_{c_1}' = \mathscr{P}(B)$, and since $|B| = k$,
      it follows by our Inductive Hypothesis that
      $|C_{c_1}'| = |\mathscr{P}(B)| = 2^k$.
      
      Now consider the function
      $$f : C_{c_1}' \rightarrow C_{c_1},$$
      defined by $f(M) = M \cup \{c_1\}$.
      It is trivial to show that $f$ is bijective, so we have that
      $|C_{c_1}| = C_{c_1}' = 2^k$. Thus
      $|\mathscr{P}(C)| = 2^k + 2^k = 2^{k + 1}$, so that $P(k + 1)$ holds. It 
      follows by the Principle of Mathematical Induction that $P(n)$ holds for 
      each natural number $n$. \qed
%%%%%%%%%%%%%%%%%%%%%%%%%%%%%%%%%%Prob0.18%%%%%%%%%%%%%%%%%%%%%%%%%%%%%%%%%%%%%%
   \item[0.18] For any set $A$, finite or infinite, let $B^A$ be the set of all
               functions mapping $A$ into the set $B = \{0, 1\}$. Show that the
               cardinality of $B^A$ is the same as the cardinality of the set
               $\mathscr{P}(A)$.
               
      \textbf{Proof:} It suffices to find a bijective map from $B^A$ to
      $\mathscr{P}(A)$. If $A$ is nonempty, then our assertion follows, so
      assume that $A$ is nonempty. Now consider the function
      $$f : B^A \rightarrow \mathscr{P}(A),$$
      defined by $f(g) = \{a \in A : g(a) = 1\}$. Suppose that $f(g_1) = f(g_2)$
      for some functions $g_1 : A \rightarrow \{0, 1\}$ and
      $g_2 : A \rightarrow \{0, 1\}$. Let $a_1 \in A$. If $g_1(a_1) = 1$, then
      $a_1 \in f(g_1)$, and since $f(g_1) = f(g_2)$, we must have that
      $a_1 \in f(g_2)$, so that $g_1(a_1) = g_2(a_1) = 1$. If $g_1(a_1) = 0$,
      then $a_1 \notin f(g_1)$, so that $a_1 \notin f(g_2)$. That is,
      $g_2(a_1) = 0$. Thus $g_1 = g_2$. It follows that $f$ is injective.      
      Let $B \in \mathscr{P}(A)$. Let $g_3$ be a function in $B^A$ such that
      $g_3(a) = 1$ if and only if $a \in B$. It follows that $f(g_3) = B$, so
      that $f$ is surjective. Thus $f$ is a bijection and
      $|B^A| = |\mathscr{P}(A)|$. \qed
%%%%%%%%%%%%%%%%%%%%%%%%%%%%%%%%%%Prob0.19%%%%%%%%%%%%%%%%%%%%%%%%%%%%%%%%%%%%%%
   \item[0.19] Show that the power set of a set $A$, finite or infinite, has
               too many elements to be able to be put in a one-to-one
               correspondence with $A$. Explain why this intuitively means that
               there are an infinite number of infinite cardinal numbers. Is
               \textit{the set of everything} logically acceptable concept? Why
               or why not?
               
      \textbf{Proof:} We already know that $|A| \le |\mathscr{P}(A)|$ since we
      can map $a \in A$ to $\{a\} \in \mathscr{P}(A)$. We shall show by way of
      contradiction that $|A| < |\mathscr{P}(A)|$. So suppose that both of these
      sets are equinumerous; that is, there exists a bijection
      $$f : A \rightarrow \mathscr{P}(A).$$
      
      Now consider a subset of $A$, $S = \{a \in A : a \notin f(a)\}$. Let
      $b \in A$. If $b \in f(b)$, then $b \notin S$, so that $f(b) \neq S$. Also
      if $b \notin f(b)$, then $b \in S$, so that $f(b) \neq S$. Thus no element
      $b$ exists in $A$ such that $f(b) = S$. Thus $|\mathscr{P}(A)| > |A|$.
      
      We intuit that there is an infinite number of cardinals because we can
      define a sequence of strictly increasing cardinals wherein the first
      cardinal is the size of some nonempty set and subsequent cardinals are
      the sizes of power sets of their immediate former set. For example, if
      $A = \{1, 2, 3\}$, then we following:
      $$|A| < |\mathscr{P}(A)| < |\mathscr{P}(\mathscr{P}(A))| <
        |\mathscr{P}(\mathscr{P}(\mathscr{P}(A)))| < \cdots$$
        
      The set of everything would not make sense because it would have to
      contain itself, and we would also arrive at a contradiction since the
      power set of the set of everything would be larger than the set of
      everything. \qed
%%%%%%%%%%%%%%%%%%%%%%%%%%%%%%%%%%Prob0.20%%%%%%%%%%%%%%%%%%%%%%%%%%%%%%%%%%%%%%
   \item[0.20] Let $A = \{1, 2\}$ and let $B = \{3, 4, 5\}$.
      \begin{enumerate}
         \item Illustrate, using $A$ and $B$, why we consider that $2 + 3 = 5$.
               Use similar reasoning with sets of your choice to decide what you
               would consider to be the value of

               i. $3 + \aleph_0$
               \qquad\qquad\qquad\qquad\qquad\qquad\qquad\qquad\qquad\qquad
               ii. $\aleph_0 + \aleph_0$.
         \item Illustrate why we consider that $2 \cdot 3 = 6$ by plotting the
               points of $A \times B$ in the plane $\R \times \R$. Use similar
               reasoning with a figure in the text to decide what you would
               consider to be the value of $\aleph_0 \cdot \aleph_0$.
      \end{enumerate}

      \textbf{Solution:} 

      \begin{enumerate}
         \item Let $a$ and $b$ be two nonnegative integers. We define $a + b$ 
               to be the cardinality of the union of any two disjoint sets $A$ 
               and $B$, where $|A| = a$ and $|B| = b$. Using this definition, we
               see that 2 + 3 = 5. As per our definition, we see that
               $3 + \aleph_0 = |\{\pi, e, e^2\} \cup \N| = \aleph_0$, and
               $\aleph_0 + \aleph_0 = |\N \cup \Z^-| = \aleph_0$.
         \item Let $a$ and $b$ be two nonnegative integers. We define
               $a \cdot b$ to be the cardinality of $A \times B$, where $A$ and
               $B$ are any sets such that $|A| = a$ and $|B| = b$. So if we set
               $A = B = \N$, it follows that
               $\aleph_0 \cdot \aleph_0 = |\N \times \N| = \aleph_0$.
      \end{enumerate}
%%%%%%%%%%%%%%%%%%%%%%%%%%%%%%%%%%Prob0.21%%%%%%%%%%%%%%%%%%%%%%%%%%%%%%%%%%%%%%
   \item[0.21] How many numbers in the interval $0 \le x \le 1$ can be expressed
               in the form \verb|.##|, where each \verb|#| is a digit
               0, 1, 2, 3, \ldots, 9? How many are there of the form
               .\verb|#####|? Following this idea, and Exercise 15, decide what 
               you would consider to be the value of $10^{\aleph_0}$. How about
               $12^{\aleph_0}$ and $2^{\aleph_0}$?

      \textbf{Solution:} 

      Since each available decimal position can be occupied by any of the base
      ten digits, it follows that there are $10 \cdot 10$ (resp. $10^5$) numbers
      in $[0, 1]$ that can be expressed in the form
      \verb|##| (resp. \verb|#####|). Now we observe that we can express every
      number in $[0, 1]$ in the form $.a_1a_2a_3a_4\ldots$, where $a_i$ is a
      nonnegative less than 10, for each natural number $i$; since each decimal
      position can be indexed, it follows that the set of positions is
      countable, so that the number of reals in the interval $[0, 1]$ is
      $10^{\aleph_0}$. A caveat: there are infinite reals with multiple decimal 
      representations, but since they are a subset of the rationals, it follows
      that our ``true answer" will be $10^{\aleph_0} - \aleph_0$. By Exercise
      0.15, it follows that $|\R| = 10^{\aleph_0} - \aleph_0$. Adding $\aleph_0$
      to both sides tells us that $|\R| = 10^{\aleph_0}$.

      If we represent the real numbers in bases 2 and 12 and using the same
      method above, we shall see that $|\R| = 2^{\aleph_0} = 12^{\aleph_0}$.
      
      
%%%%%%%%%%%%%%%%%%%%%%%%%%%%%%%%%%Prob0.22%%%%%%%%%%%%%%%%%%%%%%%%%%%%%%%%%%%%%%
   \item[0.22] Continuing the idea in the preceding exercise and using Exercises
               18 and 19, use exponential notation to fill in the three blanks
               to give a list of five cardinal numbers, each greater than the
               preceding one.
               $$\aleph_0, |\R|, 2^{2^{\aleph_0}}, 2^{2^{2^{\aleph_0}}},
                 2^{2^{2^{2^{\aleph_0}}}}.$$
\end{enumerate}


\noindent In Exercises 23 through 27, find the number of different partitions of
a set having the given number of elements.

\begin{enumerate}
%%%%%%%%%%%%%%%%%%%%%%%%%%%%%%%%%%Prob0.23%%%%%%%%%%%%%%%%%%%%%%%%%%%%%%%%%%%%%%
   \item[0.23] 1 element.

      \textbf{Solution:} $\mathscr{P}(\{1\}) = \{\emptyset, \{1\}\}$.
%%%%%%%%%%%%%%%%%%%%%%%%%%%%%%%%%%Prob0.24%%%%%%%%%%%%%%%%%%%%%%%%%%%%%%%%%%%%%%
   \item[0.24] 2 elements.

      \textbf{Solution:} $\mathscr{P}(\{1, 2\}) =
      \{\emptyset, \{1\}, \{2\}, \{1, 2\}\}$.
%%%%%%%%%%%%%%%%%%%%%%%%%%%%%%%%%%Prob0.25%%%%%%%%%%%%%%%%%%%%%%%%%%%%%%%%%%%%%%
   \item[0.25] 3 elements.

      \textbf{Solution:} $\mathscr{P}(\{1, 2, 3\}) = \{\emptyset, \{1\}, \{2\}, 
      \{3\}, \{1, 2\}, \{2, 3\}, \{1, 3\}, \{1, 2, 3\}\}$.
%%%%%%%%%%%%%%%%%%%%%%%%%%%%%%%%%%Prob0.26%%%%%%%%%%%%%%%%%%%%%%%%%%%%%%%%%%%%%%
   \item[0.26] 4 elements.

      \textbf{Solution:} $\mathscr{P}(\{1, 2, 3, 4\}) = \{\emptyset, \{1\},
      \{2\}, \{3\}, \{1, 2\}, \{2, 3\}, \{1, 3\}, \{1, 2, 3\}, \\
      \{4\}, \{1, 4\}, \{2, 4\}, \{3, 4\}, \{1, 2, 4\}, \{2, 3, 4\},
      \{1, 3, 4\}, \{1, 2, 3, 4\}\}$.
%%%%%%%%%%%%%%%%%%%%%%%%%%%%%%%%%%Prob0.27%%%%%%%%%%%%%%%%%%%%%%%%%%%%%%%%%%%%%%
   \item[0.27] 5 elements.

      \textbf{Solution:} $\mathscr{P}(\{1, 2, 3, 4, 5\}) = \{\emptyset, \{1\},
      \{2\}, \{3\}, \{1, 2\}, \{2, 3\}, \{1, 3\}, \{1, 2, 3\}, \\
      \{4\}, \{1, 4\}, \{2, 4\}, \{3, 4\}, \{1, 2, 4\}, \{2, 3, 4\},
      \{1, 3, 4\}, \{1, 2, 3, 4\}, \{5\}, \{1, 5\}, \\
      \{2, 5\}, \{3, 5\}, \{1, 2, 5\}, \{2, 3, 5\}, \{1, 3, 5\},
      \{1, 2, 3, 5\}, \{4, 5\}, \{1, 4, 5\}, \{2, 4, 5\}, \\
      \{3, 4, 5\}, \{1, 2, 4, 5\}, \{2, 3, 4, 5\},
      \{1, 3, 4, 5\}, \{1, 2, 3, 4, 5\}\}$.
%%%%%%%%%%%%%%%%%%%%%%%%%%%%%%%%%%Prob0.28%%%%%%%%%%%%%%%%%%%%%%%%%%%%%%%%%%%%%%
   \item[0.28] Consider a partition of a set $S$. The paragraph following
               Definition 0.18 explained why the relation
               $$x\mbox{ }\mathscr{R}\mbox{ }y \mbox{ if and only if }x
                  \mbox{ and }y \mbox{ are in the same cell}$$ 
               satisifies the symmetric condition for an equivalence relation.
               Write similar explanations of why the reflexive and transitive
               properties are also satisfied.

      \textbf{Proof:} If $x$ is in a cell, then $x$ and $x$ are in the same cell
      since they are the same element, so that $x$ $\mathscr{R}$ $x$. Suppose
      $x$ and $y$ are in the same cell and $y$ and $z$ are in the same cell; it
      instantly follows that $x$ and $z$ are in the same cell, so that
      $x$ $\mathscr{R}$ $z$. Thus the relation is reflexive and transitive. \qed
\end{enumerate}

\noindent In Exercises 29 through 34, determine whether the given relation is
an equivalence relation on the set. Describe the partition arising from each
equivalence relation.
\begin{enumerate}
%%%%%%%%%%%%%%%%%%%%%%%%%%%%%%%%%%Prob0.29%%%%%%%%%%%%%%%%%%%%%%%%%%%%%%%%%%%%%%
   \item[0.29] $n$ $\mathscr{R}$ $m$ in $\Z$ if $nm > 0$.

      \textbf{Proof:} Not an equivalence relation because $\mathscr{R}$ is not
      reflexive $((0, 0) \notin \mathscr{R})$.
%%%%%%%%%%%%%%%%%%%%%%%%%%%%%%%%%%Prob0.30%%%%%%%%%%%%%%%%%%%%%%%%%%%%%%%%%%%%%%
   \item[0.30] $x$ $\mathscr{R}$ $y$ in $\R$ if $x \ge y$.

      \textbf{Proof:} Not an equivalence relation because $\mathscr{R}$ is not
      symmetric $((1, 4) \notin \mathscr{R}$, but $(4, 1) \in \mathscr{R})$.
%%%%%%%%%%%%%%%%%%%%%%%%%%%%%%%%%%Prob0.31%%%%%%%%%%%%%%%%%%%%%%%%%%%%%%%%%%%%%%
   \item[0.31] $x$ $\mathscr{R}$ $y$ in $\R$ if $|x| = |y|$.

      \textbf{Proof:} One can easily check this is an equivalence relation. Each
      cell of the partition contains exactly a real number and its additive
      inverse.
%%%%%%%%%%%%%%%%%%%%%%%%%%%%%%%%%%Prob0.32%%%%%%%%%%%%%%%%%%%%%%%%%%%%%%%%%%%%%%
   \item[0.32] $x$ $\mathscr{R}$ $y$ in $\R$ if $|x - y| \le 3$.

      \textbf{Proof:} Not an equivalence relation because $\mathscr{R}$ is not
      transitive $((1, 4), (4, 7) \in \mathscr{R}$, but
      $(1, 7) \notin \mathscr{R})$.
%%%%%%%%%%%%%%%%%%%%%%%%%%%%%%%%%%Prob0.33%%%%%%%%%%%%%%%%%%%%%%%%%%%%%%%%%%%%%%
   \item[0.33] $n$ $\mathscr{R}$ $m$ in $\Z^+$ if $n$ and $m$ have the same 
               number of digits in the usual base ten notation.

      \textbf{Proof:} One can easily check this is an equivalence relation. Each
      cell of the partition contains the integers $10^{i-1}, \ldots, 10^i - 1$,
      for some positive integer $i$.
%%%%%%%%%%%%%%%%%%%%%%%%%%%%%%%%%%Prob0.34%%%%%%%%%%%%%%%%%%%%%%%%%%%%%%%%%%%%%%
   \item[0.34] $n$ $\mathscr{R}$ $m$ in $\Z^+$ if $n$ and $m$ have the same 
               final digit in the usual base ten notation.

      \textbf{Proof:} One can easily check this is an equivalence relation. Each
      cell of the partition contains integers with the same unit digit.
%%%%%%%%%%%%%%%%%%%%%%%%%%%%%%%%%%Prob0.35%%%%%%%%%%%%%%%%%%%%%%%%%%%%%%%%%%%%%%
   \item[0.35] Using set notation of the form $\{\#,\#,\#,\cdots\}$ for an
               infinite set, write the residue classes modulo $n$ in $\Z^+$
               discussed in Example 0.17 for the indicated value of $n$.

               \textbf{a.} $n = 2$
               \qquad\qquad\qquad
               \textbf{b.} $n = 3$
               \qquad\qquad\qquad
               \textbf{c.} $n = 5$

      \textbf{Solution:}

      \begin{enumerate}
         \item $\{1, 3, 5, \cdots\}, \{2, 4, 6, \cdots\}$.
         \item $\{1, 4, 7, \cdots\}, \{2, 5, 8, \cdots\}, \{3, 6, 9, \cdots\}$.
         \item $\{1, 6, 11, \cdots\}, \{2, 7, 12, \cdots\},
               \{3, 8, 13, \cdots\}, \{4, 9, 14, \cdots\},
               \{5, 10, 15, \cdots\}$.
      \end{enumerate}
%%%%%%%%%%%%%%%%%%%%%%%%%%%%%%%%%%Prob0.36%%%%%%%%%%%%%%%%%%%%%%%%%%%%%%%%%%%%%%
   \item[0.36] Let $n \in \Z^+$ and let $\sim$ be defined on $\Z$ by $r \sim s$
               if and only if $r - s$ is divisible by $n$, that is, if and only
               if $r - s = nq$ for some $q \in \Z$.

      \begin{enumerate}
         \item Show that $\sim$ is an equivalence relation on $\Z$.
         \item Show that, when restricted to the subset $\Z^+$ of $\Z$,
               this $\sim$ is the equivalence relation,
               \textit{congruence modulo n}, of Example 0.20.
         \item The cells of this partition of $\Z$ are
               \textit{residue classes modulo n} in $\Z$. Repeat Exercise 35 for
               the residue classes modulo in $\Z^+$ using the notation
               $\{\cdots, \#, \#, \#, \cdots\}$ for these infinite sets.
      \end{enumerate}

      \textbf{Solution:}

      \begin{enumerate}
         \item Let $a \in \Z$. Then $a - a = n \cdot 0$, so that $a \sim a$.
               Thus $\sim$ is reflexive. Suppose $x \sim y$ for some integers
               $x$ and $y$. Then we have that $x - y = nq_1$ for some integer
               $q_1$. So we have $y - x = n(-q_1)$, so that $y \sim x$. Thus
               $\sim$ is symmetric. Now further suppose that $y \sim z$. That
               is there exists an integer $q_2$ such that $y - z = nq_2$. 
               Substituting for $y$, we get $x - z = n(q_1 + q_2)$, so that
               $x \sim z$. Thus $\sim$ is transitive. It follows that $\sim$ is
               an equivalence relation.
         \item Let $a, b \in \Z^+$. Suppose $a \equiv_n b$. Then $a$ and $b$ 
               both have remainder $0 \le k < n$ when divided by $n$. So we can 
               write $a = nq_1 + k$ and $b = nq_2 + k$ for some positive 
               integers $q_1$ and $q_2$. Since $a - b = n(q_1 - q_2)$, it 
               follows that $a \sim b$. Now if $a \sim b$, then we have
               $a - b = nq_3$ for some integer $q_3$. Thus $a/n = b/n + q_3$;
               this says that $a$ and $b$ have the same remainder when divided
               by $n$, so that $a \equiv_n b$. Hence $\sim$ is the equivalence
               relation, \textit{congruence modulo n}, of Example 0.20.
         \item \verb|Routine.|
      \end{enumerate}
\end{enumerate}
