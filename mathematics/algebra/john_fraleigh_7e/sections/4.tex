\noindent In Exercises 1 through 6, determine whether the binary operation $*$
          gives a group structure on the given set. If no group results, give
          the first axiom in the order $\mathscr{G}_1$, $\mathscr{G}_2$,
          $\mathscr{G}_3$ from Definition 4.1 that does not hold. \\

\noindent \textbf{Computations}
\begin{enumerate}
%%%%%%%%%%%%%%%%%%%%%%%%%%%%%%%%%%Prob4.1%%%%%%%%%%%%%%%%%%%%%%%%%%%%%%%%%%%%%%%
   \item[4.1] Let $*$ be defined on $\Z$ by letting $a * b = ab$.
	
		\textbf{Proof:} We claim that $\cyc{\Z, *}$ is not a group. We know that 
      1 is the identity for multiplication on $\Z$. But no element apart from 
      $-1$ and 1 have inverses. Thus $\cyc{\Z, *}$ is not a group. \qed
%%%%%%%%%%%%%%%%%%%%%%%%%%%%%%%%%%Prob4.2%%%%%%%%%%%%%%%%%%%%%%%%%%%%%%%%%%%%%%%
   \item[4.2] Let $*$ be defined on $2\Z = \{2n : n \in \Z\}$ by letting
              $a * b = a + b$.
	
		\textbf{Proof:} We claim that $\cyc{2\Z, *}$ is a group. It is trivial to
      show that $2\Z$ is closed under $*$. So we must now show that
      $\cyc{2\Z, *}$ satisfies the axioms of a group. 
		\begin{enumerate}
			\item[$\mathscr{G}_1$:] Since $2\Z \subset \Z$ and since addition is
											associative on $\Z$, it follows that $*$ is
											associative on $2\Z$.
			\item[$\mathscr{G}_2$:] Let $k \in 2\Z$. Since 0 is an even number it
											is a member of $2\Z$. So we have that
											$k * 0 = k + 0 = k = 0 + k = 0 * k$. That is, 0
											is the identity for $*$.
			\item[$\mathscr{G}_3$:] Let $k \in 2\Z$. Then it follows that $-k$ is
											also a member of $2\Z$. But
											$k * (-k) = k + (-k) = 0 = (-k)+ k = (-k) * k$,
											so that $-k$ is the inverse of $k$ under $*$.
		\end{enumerate}
		
		The aforementioned properties showed that $\cyc{2\Z, *}$ is a group. \qed
%%%%%%%%%%%%%%%%%%%%%%%%%%%%%%%%%%Prob4.3%%%%%%%%%%%%%%%%%%%%%%%%%%%%%%%%%%%%%%%
   \item[4.3] Let $*$ be defined on $\R^+$ by letting $a * b = \sqrt{ab}$.

      \textbf{Proof:} We claim that $\cyc{\R^+, *}$ is not a group. It suffices 
      to show that $\cyc{\R^+, *}$ does not have an identity element. Assume by 
      way of contradiction that this group has an identity element $e$. Then 
      consider any $r \in \R^+$. It follows that $r * e = r$, so that
      $\sqrt{re} = r$. That is $re = r^2$, so that $r(e - r) = 0$. Since
      $r > 0$, we must have $e = r$, a contradiction since $r$ was arbitrarily
      chosen. Thus $\cyc{\R^+, *}$ is not a group. \qed
%%%%%%%%%%%%%%%%%%%%%%%%%%%%%%%%%%Prob4.4%%%%%%%%%%%%%%%%%%%%%%%%%%%%%%%%%%%%%%%
   \item[4.4] Let $*$ be defined on $\Q$ by letting $a * b = ab$.
	
		\textbf{Proof:} We claim that $\cyc{\Q, *}$ is not a group. We know that 
      1 is the identity for multiplication on $\Q$. But 0 does not have an
      inverse under $*$ because $0 * q = 0 \neq 1$ for all $q \in \Q$. Thus
      $\cyc{\Q, *}$ is not a group. \qed
%%%%%%%%%%%%%%%%%%%%%%%%%%%%%%%%%%Prob4.5%%%%%%%%%%%%%%%%%%%%%%%%%%%%%%%%%%%%%%%
   \item[4.5] Let $*$ be defined on the set $\R^*$ of nonzero real numbers by 
              letting $a * b = a/b$.
	
		\textbf{Proof:} We claim that $\cyc{\R^*, *}$ is not a group. It suffices
      to show that $*$ is not associative. This instantly follows since
      $1/2 = ((1 * 1) * 2) \neq (1 * (1 * 2)) = 2$. \qed
%%%%%%%%%%%%%%%%%%%%%%%%%%%%%%%%%%Prob4.6%%%%%%%%%%%%%%%%%%%%%%%%%%%%%%%%%%%%%%%
   \item[4.6] Let $*$ be defined on $\C$ by letting $a * b = |ab|$.

      \textbf{Proof:} $\cyc{\C, *}$ is not a group because the binary operation 
      $*$ is not closed on $\C$, since it maps pairs in $\C$ to an element in
      $\R$, instead of mapping to an element in $\C$. \qed
%%%%%%%%%%%%%%%%%%%%%%%%%%%%%%%%%%Prob4.7%%%%%%%%%%%%%%%%%%%%%%%%%%%%%%%%%%%%%%%
   \item[4.7] Give an example of an abelian group $G$ where $G$ has exactly
              1000 elements.

      \textbf{Solution:} $\Z_{1000}$.
%%%%%%%%%%%%%%%%%%%%%%%%%%%%%%%%%%Prob4.8%%%%%%%%%%%%%%%%%%%%%%%%%%%%%%%%%%%%%%%
   \item[4.8] We can also consider multiplication $\cdot_n$ modulo $n$ in
              $\Z_n$. For example, $5 \cdot_7 6 = 2$ in $\Z_7$ because
              $5 \cdot 6 = 30 = 4(7) + 2$. The set $\{1, 3, 5, 7\}$ with
              multiplication $\cdot_8$ modulo 8 is a group. Give the table for
              this group.

      \textbf{Solution:}
      $$
         \begin{tabular}{@{}c | c | c | c | c@{}} 
            $\cdot_8$ & 1 & 3 & 5 & 7 \\ \hline
            1 & 1 & 3 & 5 & 7 \\ \hline
            3 & 3 & 1 & 7 & 5 \\ \hline
            5 & 5 & 7 & 1 & 3 \\ \hline
            7 & 7 & 5 & 3 & 1
         \end{tabular}
      $$
%%%%%%%%%%%%%%%%%%%%%%%%%%%%%%%%%%Prob4.9%%%%%%%%%%%%%%%%%%%%%%%%%%%%%%%%%%%%%%%
   \item[4.9] Show that the group $\cyc{U, \cdot}$ is not isomorphic to either
              $\cyc{\R, +}$ or $\cyc{\R^*, \cdot}$. (All three groups have
              cardinality $|\R|$.)
				  
		\textbf{Proof:} The equation $x \cdot x = 1$ has two solutions in $U$, but
		the equation $x + x = 0$ only has one solution in $\R$. Similarly the
		equation $x \cdot x \cdot x = 1$ has three solutions in $U$, but the 
      equation	$x \cdot x \cdot x = 1$ only has one solution in $\R^*$. Thus
		$\cyc{U, \cdot}$ is not isomorphic to either
		$\cyc{\R, +}$ or $\cyc{\R^*, \cdot}$. \qed
%%%%%%%%%%%%%%%%%%%%%%%%%%%%%%%%%%Prob4.10%%%%%%%%%%%%%%%%%%%%%%%%%%%%%%%%%%%%%%
   \item[4.10] Let $n$ be a positive integer and let $n\Z : \{nm : m \in \Z\}$.
      \begin{enumerate}
         \item Show that $\cyc{n\Z, +}$ is a group.
         \item Show that $\cyc{n\Z, +} \simeq \cyc{\Z, +}$.
      \end{enumerate}
		
		\textbf{Proof:}
		
		\begin{enumerate}
         \item It is clear that $n\Z$ is closed under addition and not empty, so
					it suffices to show that $\cyc{n\Z, +}$ satisfies the three
					group axioms.
					\begin{enumerate}
						\item[$\mathscr{G}_1$:]	Since $n\Z \subset \Z$ and since $\Z$
														is associative under addition, it
														follows that $\cyc{n\Z, +}$ is
														associative.
						\item[$\mathscr{G}_2$:]	Since $n \cdot 0 = 0$, it follows that
														$0 \in n\Z$. It is clear that 0 is the
														identity for this group.
						\item[$\mathscr{G}_3$:]	Let $y \in n\Z$, then $y = nz$ for
														some $z \in \Z$. Since $-z$ is an
														integer, then $n(-z) = -y$ is also a
														member of $n\Z$ and we have
														$y + (-y) = 0$. So the inverse of $y$
														is $-y$.
		         \end{enumerate}

               Thus $\cyc{n\Z, +}$ is a group. \qed
         \item Show that $\cyc{n\Z, +} \simeq \cyc{\Z, +}$.
					Consider the map
					$$\phi : n\Z \rightarrow \Z$$
					defined by $\phi(x) = x/n$ for all $x \in n\Z$. This map can be
					easily shown to be bijective. So we must show that it is an
					homomorphism; that is, $\phi(a + b) = \phi(a) + \phi(b)$ for all
					$a, b \in n\Z$. So let $a, b \in n\Z$. Then $a = nk_1$ and
					$b = nk_2$ for some integers $k_1$ and $k_2$. So we have that
					$\phi(a + b) = \phi(n(k_1 + k_2)) = k_1 + k_2 =
					\phi(a) + \phi(b)$, so that $\phi$ is an isomorphism. That is
					$\cyc{n\Z, +} \simeq \cyc{\Z, +}$. \qed
      \end{enumerate}
\end{enumerate}

\noindent In Exercises 11 through 18, determine whether the given set of
          matrices under the specified operation, matrix addition or 
          multiplication, is a group. Recall that a \textbf{diagonal matrix} is
          a square matrix whose only nonzero entries lie on the
          \textbf{main diagonal}, from the upper left to the lower right corner.
          An \textbf{upper-triangular matrix} is a square matrix with only zero
          entries below the main diagonal. Associated with each $n \times n$
          matrix $A$ is a number called the determinant of $A$, denoted by
          $\det(A)$. If $A$ and $B$ are both $n \times n$ matrices, then
          $\det(AB) = \det(A)\det(B)$. Also, det($I_n$) = 1 and $A$ is
          invertible if and only if $\det(A) \neq 0$. We shall let $D_n(\R)$ be
			 the set of all $n \times n$ diagonal matrices with real entries, let
			 diag[$a_1$ $a_2$ $\cdots$ $a_n$] represent an element in $D_n(\R)$
			 such that $a_i$ is the element in the $i$th row and $i$th column, and
			 let $0_n$ represent the $n \times n$ zero matrix.

\begin{enumerate}
%%%%%%%%%%%%%%%%%%%%%%%%%%%%%%%%%%Prob4.11%%%%%%%%%%%%%%%%%%%%%%%%%%%%%%%%%%%%%%
   \item[4.11] All $n \times n$ diagonal matrices under matrix addition.
	
		\textbf{Proof:} We want to prove that $\cyc{D_n(\R), +}$ is a group. First
      we must show that $D_n(\R)$ is closed under addition. Let $X$ and $Y$ be
      some members of $D_n(\R)$ such that $X$ = diag[$x_1$ $x_2$ $\cdots$ $x_n$]
      and $Y$ = diag[$y_1$ $y_2$ $\cdots$ $y_n$]. Then $X + Y$ = diag[$(x_1 + 
      y_1)$ $(x_2 + y_2)$ $\cdots$ $(x_n + y_n)$] so that $D_n(\R)$ is closed 
      under addition. So now we must now show that the group axioms hold for
      $\cyc{D_n(\R), +}$.
		\begin{enumerate}
			\item[$\mathscr{G}_1$:] Since $M_n(\R)$ is associative under addition
											and since $D_n(\R) \subset M_n(\R)$ it follows
											that $D_n(\R)$ is also associative under
											addition.
											
			\item[$\mathscr{G}_2$:] The matrix $0_n$ is the identity for the binary
											structure $\cyc{D_n(\R), +}$ because
											$Z + 0_n = 0_n + Z$ for all $Z \in D_n(\R)$.
			\item[$\mathscr{G}_3$:] Let diag[$b_1$ $b_2$ $\ldots$ $b_n$]
											$\in D_n(\R)$. Then we have that
											diag[$b_1$ $b_2$ $\cdots$ $b_n$] +
											diag[$-b_1$ $-b_2$ $\cdots$ $-b_n$] =
											diag[$-b_1$ $-b_2$ $\cdots$ $-b_n$] +
											diag[$b_1$ $b_2$ $\cdots$ $b_n$] = $0_n$,
											so that every element of $D_n(\R)$ has an
											inverse under addition.
		\end{enumerate}
		
		We have shown in the above that $\cyc{D_n(\R), +}$ is a group. \qed
%%%%%%%%%%%%%%%%%%%%%%%%%%%%%%%%%%Prob4.12%%%%%%%%%%%%%%%%%%%%%%%%%%%%%%%%%%%%%%
   \item[4.12] All $n \times n$ diagonal matrices under matrix multiplication.
	
		\textbf{Proof:} We want to show that the binary structure
		$\cyc{D_n(\R), \cdot}$ is not a group. Let $A \in  D_n(\R)$. We know that
		$A \cdot I_n = I_n \cdot A = A$. Thus $I_n$ is the identity for
		$\cyc{D_n(\R), \cdot}$. However since $X \cdot 0_n \neq I_n$ for all
		$X \in D_n(\R)$, it follows that $0_n$ has no inverse under	$\cdot$, so
		that $\cyc{D_n(\R), \cdot}$ is not a group. \qed
%%%%%%%%%%%%%%%%%%%%%%%%%%%%%%%%%%Prob4.13%%%%%%%%%%%%%%%%%%%%%%%%%%%%%%%%%%%%%%
   \item[4.13] All $n \times n$ diagonal matrices with no zero diagonal entry 
               under matrix multiplication.
	
		\textbf{Proof:} We want to prove that $\cyc{D_n(\R^*), \cdot}$ is a group.
      First we must show that $D_n(\R^*)$ is closed under $\cdot$. Let $X$ and 
      $Y$ be some members of $D_n(\R^*)$ such that
      $X$ = diag[$x_1$ $x_2$ $\cdots$ $x_n$] and
      $Y$ = diag[$y_1$ $y_2$ $\cdots$ $y_n$]. Then $X \cdot Y$ =
      diag[$(x_1 \cdot y_1)$ $(x_2 \cdot y_2)$ $\cdots$ $(x_n \cdot y_n)$]. 
      Since $x_i$ and $y_i$ are nonzero, we must have that $x_iy_i \neq 0$, so 
      that $X \cdot Y \in D_n(\R^*)$; i.e., $D_n(\R^*)$ is closed under 
      multiplication. So now we must now show that the group axioms hold for
      $\cyc{D_n(\R^*), \cdot}$.
		\begin{enumerate}
			\item[$\mathscr{G}_1$:] See Exercise 4.11.
											
			\item[$\mathscr{G}_2$:] Since $I_n \in \cyc{D_n(\R^*)}$, it is the
											identity for $\cyc{D_n(\R^*), \cdot}$.
			\item[$\mathscr{G}_3$:] Let diag[$b_1$ $b_2$ $\cdots$ $b_n$]
											$\in D_n(\R*)$. Then we have that
											diag[$b_1$ $b_2$ $\cdots$ $b_n$] $\cdot$
											diag[$1/b_1$ $1/b_2$ $\cdots$ $1/b_n$] =
											diag[$1/b_1$ $1/b_2$ $\cdots$ $1/b_n$] $\cdot$
											diag[$b_1$ $b_2$ $\cdots$ $b_n$] = $I_n$,
											so that every element of $D_n(\R)$ has an
											inverse under multiplication. We were able to
											take the reciprocals of the real numbers $b_1$,
											$b_2$, $\ldots$, $b_n$ because they were all
											nonzero numbers.
		\end{enumerate}
		
		We have shown in the above that $\cyc{D_n(\R^*), \cdot}$ is a group. \qed
%%%%%%%%%%%%%%%%%%%%%%%%%%%%%%%%%%Prob4.14%%%%%%%%%%%%%%%%%%%%%%%%%%%%%%%%%%%%%%
   \item[4.14] All $n \times n$ diagonal matrices with all diagonal entries 1 or
               $-$1 under matrix multiplication.
	
		\textbf{Proof:} If we follow the arguments in Exercise 4.13, then we can
      conclude that $\cyc{D_n(\{-1, 1\}), \cdot}$ is a group. \qed
%%%%%%%%%%%%%%%%%%%%%%%%%%%%%%%%%%Prob4.15%%%%%%%%%%%%%%%%%%%%%%%%%%%%%%%%%%%%%%
   \item[4.15] All $n \times n$ upper-triangular matrices under matrix 
               multiplication.
	
		\textbf{Proof:} $0_n$ is an upper-triangular matrix, but has no inverse
      under multiplication, so the binary structure in question is not a group.
      \qed
%%%%%%%%%%%%%%%%%%%%%%%%%%%%%%%%%%Prob4.16%%%%%%%%%%%%%%%%%%%%%%%%%%%%%%%%%%%%%%
   \item[4.16] All $n \times n$ upper-triangular matrices under matrix addition.
	
		\textbf{Proof:} It can easily be shown that this is a group. To show
      closure under addition is trivial. Associativity is also trivial. $0_n$
      is the identity for this binary structure. Negating the numbers in an
      upper-triangular matrix will give us its additive inverse. \qed
%%%%%%%%%%%%%%%%%%%%%%%%%%%%%%%%%%Prob4.17%%%%%%%%%%%%%%%%%%%%%%%%%%%%%%%%%%%%%%
   \item[4.17] All $n \times n$ upper-triangular matrices with determinant 1 
               under matrix multiplication.
	
		\textbf{Proof:} The set---call it $UL_n(\R)$---of all $n \times n$ upper-
      triangular matrices with entries in $\R$ and with determinant 1,  is 
      closed under matrix multiplication because if $A$ and $B$  are in
      $UL_n(\R)$, then $AB$ is upper-triangular and
      $\det(A \cdot B) = \det(A)\det(B) = 1$. We know that
      $\cyc{UL_n(\R), \cdot}$ is associative. Also note that
      $I_n \in \cyc{UL_n(\R), \cdot}$, so it is the identity for
      $\cyc{UL_n(\R), \cdot}$. Let $A \in UL_n(\R)$. Then $A$ has an inverse 
      under $\cdot$ since $\det(A) = 1$. Also, we have that $AA^{-1} = I_n$, so 
      that $\det(A)\det(A^{-1}) = \det(I_n)$. Solving this equation results in
      $\det(A^{-1}) = 1$. From Linear Algebra, it can be shown that the
      multiplicative inverse of an invertible upper-triangular matrix is also
      an upper-triangular matrix. So $A^{-1} \in UL_n(\R)$. Thus $UL_n(\R)$ is
      a group. \qed
%%%%%%%%%%%%%%%%%%%%%%%%%%%%%%%%%%Prob4.18%%%%%%%%%%%%%%%%%%%%%%%%%%%%%%%%%%%%%%
   \item[4.18] All $n \times n$ matrices with determinant either 1 or $-$1 under
               matrix multiplication.
	
		\textbf{Proof:} If we follow the ideas in Exercise 4.17, we can see that
      this binary structure is a group.\qed
%%%%%%%%%%%%%%%%%%%%%%%%%%%%%%%%%%Prob4.19%%%%%%%%%%%%%%%%%%%%%%%%%%%%%%%%%%%%%%
   \item[4.19] Let $S$ be the set of all real numbers except $-$1. Define $*$ on
					$S$ by $$a * b = a + b + ab.$$
      \begin{enumerate}
         \item Show that $*$ gives a binary operation on $S$.
         \item Show that $\cyc{S, *}$ is a group.
         \item Find the solution of the equation $2 * x * 3 = 7$ in $S$.
      \end{enumerate}
	
		\textbf{Proof:}
      \begin{enumerate}
         \item Let $a, b \in S$. We want to show that $a * b \in S$, so it
               suffices to show that $a * b \neq -1$. We shall prove by way of 
               contradiction that $a * b \neq -1$. So assume that $a * b = -1$.
               That is, $a + b + ab = -1$. Then we have that
               $a(1 + b) = -1(1 + b)$. Since $b \in S$, it follows that
               $b \neq -1$ so that $1 + b \neq 0$. So we can divide the equation
               $a(1 + b) = -1(1 + b)$ by $1 + b$ to get $a = -1$, a 
               contradiction since $a \in S$. So $a * b \neq -1$. Hence $S$ is
               closed under $*$.
         \item \begin{enumerate}
			         \item[$\mathscr{G}_1$:] Let $a, b, c \in S$. Then
                                          \begin{align*}
                                             a * (b * c) &= a + (b * c) +
                                                            a(b * c) \\
                                                         &= a + (b + c + bc) +
                                                            a(b + c + bc) \\
                                                         &= a + b + c + ab +
                                                            ac + bc + abc \\
                                                         &= (a + b + ab) + c +
                                                            (a + b + ab)c \\
                                                         &= (a * b) + c +
                                                            (a * b)c \\
                                                         &= (a * b) * c.
                                          \end{align*}

                                          Hence $\cyc{S, *}$ is associative.
											
			         \item[$\mathscr{G}_2$:] Let $a \in S$. Suppose there exists 
                                          $e \in S$ such that $a * e = a$, then
                                          $a + e + ae = a$ so that
                                          $e(1 + a) = 0$. Since $1 + a \neq 0$
                                          it must be the case that $e = 0$.
                                          Similarly we must have that
                                          $0 * a = a$. Thus 0 is the identity
                                          for $\cyc{S, *}$.
                           
			         \item[$\mathscr{G}_3$:] Let $a \in S$. Suppose there exists
                                          $y \in S$ such that $a * y = 0$. 
                                          Solving this equation will give us
                                          $y = -a/(1 + a)$. Thus the identity of
                                          an $a \in S$ is $-a/(1 + a)$.
		\end{enumerate}
         \item Find the solution of the equation $2 * x * 3 = 7$ in $S$. Using 
               the formula in (b) above we have that $-2/3$ and $-3/4$ are the 
               inverses of 2 and 3 in the group $\cyc{S, *}$. So given
               $2 * x * 3 = 7$, it follows that $2 * x * 3 * -3/4 = 7 * -3/4$, 
               so that $2 * x = 7 * -3/4$. Also, $-2/3 *2* x = -2/3 * 7 * -3/4$,
               so that $x = -2/3 * 7 * -3/4 = -1/3$.
      \end{enumerate}
%%%%%%%%%%%%%%%%%%%%%%%%%%%%%%%%%%Prob4.20%%%%%%%%%%%%%%%%%%%%%%%%%%%%%%%%%%%%%%
   \item[4.20] This exercise shows that there are two nonisomorphic group
               structures on a set of 4 elements. Let the set be
               $V = \{e, a, b, c\}$, with $e$ the identity element for the group
               operation. A group table would then have to start in the manner
               shown in Table 4.22. The square indicated by the question mark
               cannot be filled in with $a$. It must be filled in either with 
               the identity element $e$ or with an element different from both
               $e$ and $a$. In this latter case, it is no loss of generality to
               assume that this element is $b$. If this square is filled in with
               $e$, the table can be completed in two ways to give a group. Find
               these two tables. (You need not check the associative law.) If
               this square is filled in with $b$, then the table can only be
               completed in one way to give a group. Find this table. (Again, 
               you need not check the associative law.) Of the three tables you
               now have, two give isomorphic groups. Determine which two tables
               these are, and give the one-to-one onto renaming function which
               is an isomorphism.
      \begin{enumerate}
         \item Are all groups of 4 elements commutative?
         \item Which table gives a group isomorphic to the group $U_4$, so that
               we know the binary operation defined by the table is associative?
         \item Show that the group given by one of the other tables is
               structurally the same as the group in Exercise 14 for one
               particular value of $n$, so that we know that the operation
               defined by that table is associative also.
      \end{enumerate}

      \textbf{Solution:} $$
         \begin{tabular}{@{}c | c | c | c | c@{}}
            \multicolumn{5}{c}{\textbf{Table 1}} \\
            $*$ & $e$ & $a$ & $b$ & $c$ \\ \hline
            $e$ & $e$ & $a$ & $b$ & $c$ \\ \hline
            $a$ & $a$ & $e$ & $c$ & $b$ \\ \hline
            $b$ & $b$ & $c$ & $e$ & $a$ \\ \hline
            $c$ & $c$ & $b$ & $a$ & $e$
         \end{tabular} \qquad
         \begin{tabular}{@{}c | c | c | c | c@{}}
            \multicolumn{5}{c}{\textbf{Table 2}} \\
            $*'$ & $e$ & $a$ & $b$ & $c$ \\ \hline
            $e$ & $e$ & $a$ & $b$ & $c$ \\ \hline
            $a$ & $a$ & $e$ & $c$ & $b$ \\ \hline
            $b$ & $b$ & $c$ & $a$ & $e$ \\ \hline
            $c$ & $c$ & $b$ & $e$ & $a$
         \end{tabular} \qquad
         \begin{tabular}{@{}c | c | c | c | c@{}}
            \multicolumn{5}{c}{\textbf{Table 3}} \\
            $*''$ & $e$ & $a$ & $b$ & $c$ \\ \hline
            $e$ & $e$ & $a$ & $b$ & $c$ \\ \hline
            $a$ & $a$ & $b$ & $c$ & $e$ \\ \hline
            $b$ & $b$ & $c$ & $e$ & $a$ \\ \hline
            $c$ & $c$ & $e$ & $a$ & $b$
         \end{tabular}
      $$

      We can check that the map
      $$\phi : V \rightarrow V$$
      defined by $\phi(a) = b$, $\phi(b) = c$, $\phi(c) = a$, and $\phi(e) = e$
      is an isomorphism from $\cyc{V, *'}$ to $\cyc{V, *''}$. Thus Table 2 is
      isomorphic to Table 3. In Table 1, notice that the equation $x * x = e$ 
      has four solutions, namely every element in $V$, but this equation has
      only two solutions in Table 2, namely $e$ and $a$. Thus, Table 1 and
      Table 2 are not isomorphic.
      \begin{enumerate}
         \item Yes.
         \item We observe that Table 2 is isomorphic to $U_4$. We can see this
               from the isomorpic map
               $$\phi : V \rightarrow U_4$$
               defined by $\phi(e) = 1$, $\phi(a) = \zeta^2$,
               $\phi(b) = \zeta^3$, $\phi(c) = \zeta$, with $*'$ as the binary
               operation on $V$.. Thus $\cyc{V, *'}$ is associative.
               
         \item In Exercise 4.14, we proved that the binary structure below is a
               $$ \left\langle M = \left\{\left[
                  \begin{tabular}{@{}c c@{}}
                     1 & 0 \\
                     0 & 1
                  \end{tabular}\right],
                  \left[
                  \begin{tabular}{@{}r r@{}}
                     $-$1 & 0 \\
                     0    & $-$1
                  \end{tabular}\right],
                  \left[
                  \begin{tabular}{@{}c r@{}}
                     1 & 0 \\
                     0 & $-$1
                  \end{tabular}\right],
                  \left[
                  \begin{tabular}{@{}r c@{}}
                     $-$1 & 0 \\
                     0    & 1
                  \end{tabular}\right]\right\}, \cdot\right\rangle
               $$
               
               group. The map
               $$\phi : V \rightarrow M$$ 
               defined by $\phi(e) = I_2$, $\phi(a) = -I_2$, $\phi(b) = \left[
               \begin{tabular}{@{}c r@{}}
                  1 & 0 \\
                  0 & $-$1
               \end{tabular}\right]$, and $\phi(c) = \left[
               \begin{tabular}{@{}r r@{}}
                  $-$1 & 0 \\
                  0    & 1
               \end{tabular}\right]$ is isomorphic, with $*$ as the binary
               operation on $V$. So Table 1 is isomorphic to $\cyc{M, \cdot}$,
               and since the latter is associative, it follows that
               $\cyc{V, *}$ is also associative.
      \end{enumerate}
%%%%%%%%%%%%%%%%%%%%%%%%%%%%%%%%%%Prob4.21%%%%%%%%%%%%%%%%%%%%%%%%%%%%%%%%%%%%%%
   \item[4.21] According to Exercise 12 of Section 2, there are 16 possible
               binary operations on a set of 2 elements. How many of these give
               a structure of a group? How many of the 19,683 possible binary
               operations on a set of 3 elements give a group structure?

      \textbf{Solution:} From the textbook, it was shown that all groups with  
      two elements have the same structure. Since we can make any of the two
      elements to act as the identity, it follows that of the 16 possible binary
      operations on a set of 2 elements, only 2 give a group. The book also
      shows us that there all groups with two elements have the same structure.
      So there are three choices for the identity, so it follows that of the 
      19,683 possible binary operations on a set of 3 elements, only 3 give a 
      group.
\end{enumerate}

\noindent \textbf{Concepts}

\begin{enumerate}
%%%%%%%%%%%%%%%%%%%%%%%%%%%%%%%%%%Prob4.22%%%%%%%%%%%%%%%%%%%%%%%%%%%%%%%%%%%%%%
   \item[4.22] Consider our axioms $\mathscr{G}_1$, $\mathscr{G}_2$, and
               $\mathscr{G}_3$ for a group. We gave them in the order
               $\mathscr{G}_1\mathscr{G}_2\mathscr{G}_3$. Conceivable other
               orders to state the axioms are
               $\mathscr{G}_1\mathscr{G}_3\mathscr{G}_2$,
               $\mathscr{G}_2\mathscr{G}_1\mathscr{G}_3$,
               $\mathscr{G}_2\mathscr{G}_3\mathscr{G}_1$,
               $\mathscr{G}_3\mathscr{G}_1\mathscr{G}_2$, and
               $\mathscr{G}_3\mathscr{G}_2\mathscr{G}_1$. Of these six possible
               orders, exactly three are acceptable for a definition. Which
               orders are not acceptable, and why?

      \textbf{Solution:} The unacceptable orders are
      $\mathscr{G}_1\mathscr{G}_3\mathscr{G}_2$,
      $\mathscr{G}_3\mathscr{G}_1\mathscr{G}_2$, and
      $\mathscr{G}_3\mathscr{G}_2\mathscr{G}_1$, while the rest are acceptable.
      They are unacceptable because they refer to the existence of an element 
      (identity element) which hasn't yet been defined.
%%%%%%%%%%%%%%%%%%%%%%%%%%%%%%%%%%Prob4.23%%%%%%%%%%%%%%%%%%%%%%%%%%%%%%%%%%%%%%
   \item[4.23] The following ``definitions'' of a group are taken verbatim,
               including spelling and punctuation, from papers of students who
               wrote a bit too quickly and carelessly. Criticize them.
               \begin{enumerate}
                  \item A group $G$ is a set of elements together with a binary
                        operation $*$ such that the following conditions are
                        satisfied \\
                        $*$ is associative \\
                        There exists $e \in G$ such that \\
                        $$e * x = x * e = x \mbox{ identity}.$$
                        For every $a \in G$ there exists an $a'$ (inverse) such
                        that
                        $$a \cdot a' = a' \cdot a = e$$
                  \item A group is a set $G$ such that \\
                        The operation on $G$ is associative. \\
                        there is an identity element ($e$) in $G$. \\
                        for every $a \in G$, there is an $a'$ (inverse for each
                        element)
                  \item A group is a set with a binary operation such \\
                        the binary operation is defined \\
                        an inverse exists \\
                        an identity element exists
                  \item A set $G$ is called a group over the binary operation
                        $*$ such that for all $a, b \in G$. \\
                        Binary operation $*$ is associative under addition \\
                        there exist an element $\{e\}$ such that \\
                        $$a * e = e * a = e$$
                        Fore every element $a$ there exists an element $a'$ such
                        that
                        $$a * a' = a' * a = e$$
               \end{enumerate}

      \textbf{Solution:}

      \begin{enumerate}
         \item To say a ``A group $G$" is wrong because a group is an ordered
               pair $\cyc{G, *}$ where $G$ is a nonempty set and $*$ is a
               binary operation such that
               $\mathscr{G}_1\mathscr{G}_2\mathscr{G}_3$ hold. Although the
               student correctly wrote that $*$ must be associative, they must
               also explicit define what the terminology means. The word
               ``identity" should be removed from the second condition. In the
               third condition, the binary operation should be changed from
               $\cdot$ to $*$. Proper punctuation should also be employed.
         \item A group is not a set. The operation on $G$ was not specified.
               Associativity wasn't defined. What is an identity element?
               What is an inverse element? These terms should be defined. Proper
               grammar must be employed.
         \item The binary operation wasn't explicitly specified, and we don't
               know what set it is acting on. The sentence ``the binary 
               operation" is meaningless. The remaining two lines are also
               meaningless. Proper grammar and punctuation weren't employed.
         \item Bad grammar. Bad punctuation. The first sentence was badly worded
               and it prematurely ended. Although the student wrote down three
               badly worded axioms, they did not explicitly say that the
               $\cyc{G, *}$ must obey them.
               
         
      \end{enumerate}
%%%%%%%%%%%%%%%%%%%%%%%%%%%%%%%%%%Prob4.24%%%%%%%%%%%%%%%%%%%%%%%%%%%%%%%%%%%%%%
   \item[4.24] Give a table for a binary operation on the set $\{e, a, b\}$ of
               three elements satisfying axioms $\mathscr{G}_2$ and
               $\mathscr{G}_3$ for a group but not axiom $\mathscr{G}_1$.

      \textbf{Solution:} $$
         \begin{tabular}{@{}c | c | c | c@{}}
            $*$ & $e$ & $a$ & $b$ \\ \hline
            $e$ & $e$ & $a$ & $b$ \\ \hline
            $a$ & $a$ & $e$ & $b$ \\ \hline
            $b$ & $b$ & $a$ & $e$
         \end{tabular}
      $$

      From observing the table above, we see that $e$ is the identity element.
      Also $b^{-1} = b$ and $a^{-1} = a$. But
      $(a * b) * b = e \neq a = a * (b * b)$, so that $\cyc{\{e, a, b\}, *}$ 
      satisfies $\mathscr{G}_2$ and $\mathscr{G}_3$ for a group but not axiom
      $\mathscr{G}_1$.
      
%%%%%%%%%%%%%%%%%%%%%%%%%%%%%%%%%%Prob4.25%%%%%%%%%%%%%%%%%%%%%%%%%%%%%%%%%%%%%%
   \item[4.25] Mark each of the following true or false.

      \textbf{Solution:}

      \begin{tabularx}{\linewidth}{@{}c c X@{}} 
         F & \textbf{a.} & A group may have more than one identity element. \\
         T & \textbf{b.} & Any two groups of three elements are isomorphic. \\
         T & \textbf{c.} & In a group, each linear equation has a solution. \\
         F & \textbf{d.} & The proper attitude toward a definition is to
                           memorize it so that you can reproduce it word for
                           word as in the text. \\
         F & \textbf{e.} & Any definition a person gives for a group is correct
                           provided that everything that is a group by that
                           person's definition is also a group by the
                           definition in the text. \\
         T & \textbf{f.} & Any definition a person gives for a group is correct
                           provided he or she can show that everything that
                           satisfies the definition satisfies the one in the
                           text and conversely. \\
         T & \textbf{g.} & Every finite group of at most three elements is
                           abelian. \\
         T & \textbf{h.} & An equation of the form $a * x * b = c$ always has a
                           unique solution in a group. \\
         F & \textbf{i.} & The empty set can be considered a group. \\
         T & \textbf{j.} & Every group is a binary algebraic structure.
      \end{tabularx}
\end{enumerate}

\noindent \textbf{Proof synopsis}

\noindent We give an example of a proof synopsis. Here is a one-sentence
          synopsis of the proof that the inverse of an element $a$ in a group
          $\cyc{G, *}$ is unique.
\begin{quote}
Assuming that $a * a' = e$ and $a * a'' = e$, apply the left cancellation law to
the equation $a * a' = a * a''$.
\end{quote}

\noindent Note that we said ``the left cancellation law" and not
          ``Theorem 4.15." We always suppose that our synopsis was given as an
          explanation given during a conversation at lunch, with no reference to
          text numbering and as little notation as is practical.
\begin{enumerate}
%%%%%%%%%%%%%%%%%%%%%%%%%%%%%%%%%%Prob4.26%%%%%%%%%%%%%%%%%%%%%%%%%%%%%%%%%%%%%%
   \item[4.26] Give a one-sentence synopsis of the proof of the left 
               cancellation law in Theorem 4.15.

      \textbf{Solution:} Suppose that $a * b = a * c$ in some group
      $\cyc{G, *}$. Multiply the equation on both sides by $a'$ and then use
      the group axioms to simply the equation.
%%%%%%%%%%%%%%%%%%%%%%%%%%%%%%%%%%Prob4.27%%%%%%%%%%%%%%%%%%%%%%%%%%%%%%%%%%%%%%
   \item[4.27] Give at most a two-sentence synopsis of the proof in Theorem
               4.16 that an equation $ax = b$ has a unique solution in a group.

      \textbf{Solution:} First we must show that a solution exists, and then we 
      next suppose that $x_1$ and $x_2$ are two solutions of the equations, so 
      that $a * x_1 = a * x_2$. Then we apply the left cancellation law to the 
      equation $a * x_1 = a * x_2$.
      


\end{enumerate}

\noindent \textbf{Theory}

\begin{enumerate}
%%%%%%%%%%%%%%%%%%%%%%%%%%%%%%%%%%Prob4.28%%%%%%%%%%%%%%%%%%%%%%%%%%%%%%%%%%%%%%
   \item[4.28] From our intuitive grasp of the notion of isomorphic groups, it
               should be clear that if $\phi : G \rightarrow G'$ is a group
               isomorphism, then $\phi(e)$ is the identity $e'$ of $G'$. Recall
               that Theorem 3.14 gave a proof of this for isomorphic binary
               $\cyc{S, *}$ and $\cyc{S', *'}$. Of course, this covers the case
               of groups.

               \quad It should be also be intuitively clear that if $a$ and $a'$
               are inverse pairs in $G$, then $\phi(a)$ and $\phi(a')$ are
               inverse pairs in $G'$, that is, that $\phi(a)' = \phi(a')$. Give 
               a careful proof of this for a skeptic who can't see the forest 
               for all the trees.

      \textbf{Proof:} It suffices to show that $\phi(a) *' \phi(a') = e'$. Since
      $a * a' = e$, it follows that $\phi(a * a') = \phi(e)$ so that
      $\phi(a) *' \phi(a') = \phi(e) = e'$. Since $e'$ is the identity of the
      group $\cyc{G', *'}$ and since $\phi(a) *' \phi(a') = e'$, it must be the
      case that $\phi(a)' = \phi(a')$. \qed
%%%%%%%%%%%%%%%%%%%%%%%%%%%%%%%%%%Prob4.29%%%%%%%%%%%%%%%%%%%%%%%%%%%%%%%%%%%%%%
   \item[4.29] Show that if $G$ is a finite group with identity $e$ and with an
               even number of elements, then there is $a \neq e$ in $G$ such
               that $a * a = e$.
      
      \textbf{Proof:} Suppose $|G| = 2k$, for some positive integer $k$. We 
      define a binary relation $\sim$ on $G$ such that for all $a, b \in G$,
      $a \sim b$ if and only if $a' = b$. We can show that $\sim$ is an 
      equivalence relation on $G$. Let $\mathcal{C}$ be the set of equivalence 
      classes of $\sim$. We observe that an equivalence class
      $C \in \mathcal{C}$ contains either 1 element or 2 elements. We can write
      $\mathcal{C}$ as a union of two disjoint sets, say
      $$\mathcal{C} = \mathcal{C}_1 \cup \mathcal{C}_2$$
      where $\mathcal{C}_i$ is the set of equivalence classes of $\sim$ that 
      contain $i$ elements. Thus
      $$|G| = |\mathcal{C}_1| + 2|\mathcal{C}_2| = 2k.$$
      Since $\{e\} \in \mathcal{C}_1$, it follows that $|\mathcal{C}_1| \ge 1$.
      To complete the proof it suffices to show that $|\mathcal{C}_1| \ge 2$.
      First we want to show that $|\mathcal{C}_2| \le k - 1$. So assume by
      contradiction that $|\mathcal{C}_2| \ge k$. Then since
      $|\mathcal{C}_1| \ge 1$, it must be the case that
      $2k = |\mathcal{C}_1| + 2|\mathcal{C}_2| \ge 1 + 2k$, a contradiction. So
      we must have that $|\mathcal{C}_2| \le k - 1$, so that
      $2k = |\mathcal{C}_1| + 2|\mathcal{C}_2| \le |\mathcal{C}_1| + 2k - 2$.
      That is, $2k \le |\mathcal{C}_1| + 2k - 2$. Solving this inequality gives
      us $|\mathcal{C}_1| \ge 2$. \qed
%%%%%%%%%%%%%%%%%%%%%%%%%%%%%%%%%%Prob4.30%%%%%%%%%%%%%%%%%%%%%%%%%%%%%%%%%%%%%%
   \item[4.30] Let $\R^*$ be the set of all real numbers except 0. Define $*$ on
               $\R^*$ by letting $a * b = |a|b$.
               \begin{enumerate}
                  \item Show that $*$ gives an associative binary operation on
                        $\R^*$.
                  \item Show that there is a left identity for $*$ and a right
                        inverse for each element in $\R^*$.
                  \item Is $\R^*$ with this binary operation a group?
                  \item Explain the significance of this exercise.
               \end{enumerate}

      \textbf{Solution:}

      \begin{enumerate}
         \item Consider $a, b, c \in \R^*$. Then
               \begin{align*}
                  a * (b * c) &= a * (|b|c)           \\
                              &= |a| \cdot |b|c       \\
                              &= ||a|| \cdot |b|c     \\
                              &= (||a|| \cdot |b|)c   \\
                              &= (||a|b|)c            \\
                              &= (|a * b|)c           \\
                              &= (a * b) * c,         \\
               \end{align*}
               so that $*$ is associative on $\R^*$.
         \item Let $r \in \R^*$. Then $-1 * r = |-1|r = r$, so that $-1$ is a 
               left identity for $*$ on $\R^*$. Also $r * -1/|r| = -1$, so that
               $-1/|r|$ is a right inverse for $r$.
         \item No it is not since the group has more than one left identity,
               namely 1 and $-1$.
         \item A one sided definition of a group cannot mix left and right 
               conditions.
      \end{enumerate}
%%%%%%%%%%%%%%%%%%%%%%%%%%%%%%%%%%Prob4.31%%%%%%%%%%%%%%%%%%%%%%%%%%%%%%%%%%%%%%
   \item[4.31] If $*$ is a binary operation on a set $S$, an element $x$ of $S$
               is an \textbf{idempotent for} $*$ if $x * x = x$. Prove that a
               group has exactly one idempotent element. (You may use any
               theorems proved so far in the text.)

      \textbf{Proof:} It suffices to show that $x * x = x$ if and only if
      $x = e$. The ($\Leftarrow$) direction is trivial. For the
      $(\Rightarrow)$ direction, we multiply both sides of the equation
      $x * x = x$ by $x'$ and use the group axioms to get $x = e$. \qed
%%%%%%%%%%%%%%%%%%%%%%%%%%%%%%%%%%Prob4.32%%%%%%%%%%%%%%%%%%%%%%%%%%%%%%%%%%%%%%
   \item[4.32] Show that every group $G$ with identity $e$ and such that
               $x * x = e$ for all $x \in G$ is abelian.

      \textbf{Proof:} Let $\cyc{G, *}$ be a group with identity $e$ such
      that $x * x = e$ for all $x \in G$, or equivalently, $x' = x$ for all
      $x \in G$. Let $a, b \in G$. Then $(a * b)' = a * b$. But we know from the
      discussion in the textbook that $(a * b)' = b' * a'$. Thus we have that
      $a * b = b' * a' = b * a$, so that $\cyc{G, *}$ is abelian. \qed
%%%%%%%%%%%%%%%%%%%%%%%%%%%%%%%%%%Prob4.33%%%%%%%%%%%%%%%%%%%%%%%%%%%%%%%%%%%%%%
   \item[4.33] Let $G$ be an abelian group and let $c^n = c * c * \cdots * c$
               for $n$ factors $c$, where $c \in G$ and $n \in \Z^+$. Give a 
               mathematical induction proof that $(a * b)^n = (a^n) * (b^n)$ for
               all $a, b \in G$.

      \textbf{Proof:} The base case, $n = 1$, trivially holds. So assume that
      $(a * b)^k = (a^k) * (b^k)$, for some positive integer $k$. Then we have
      that
      \begin{align*}
         (a * b)^{k+1} &= (a * b)^k * (a * b) \\
            &= (a^k * b^k) * (a * b) \qquad &\text{[Inductive Hypothesis]} \\
            &= ((a^k * b^k) * a) * b \qquad &\text{[Associativity]} \\
            &= (a^k * (b^k * a)) * b \qquad &\text{[Associativity]} \\
            &= (a^k * (a * b^k)) * b \qquad &\text{[Commutativity]} \\
            &= ((a^k * a) * b^k) * b \qquad &\text{[Associativity]} \\
            &= (a^{k+1} * b^k) * b   \\
            &= a^{k+1} * (b^k * b) \qquad &\text{[Associativity]} \\
            &= a^{k+1} * b^{k+1}, 
      \end{align*}
      so that, by Mathematical Induction, we have $(a * b)^n = (a^n) * (b^n)$ 
      for all $a, b \in G$. \qed
%%%%%%%%%%%%%%%%%%%%%%%%%%%%%%%%%%Prob4.34%%%%%%%%%%%%%%%%%%%%%%%%%%%%%%%%%%%%%%
   \item[4.34] Let $G$ be a group with a finite number of elements. Show that
               for any $a \in G$, there exists an $n \in \Z^+$ such that
               $a^ n = e$.
   
      \textbf{Proof:} Let $\cyc{G, *}$ be a group such that $|G| = n \in \Z^+$.
      Let $a \in G$. Consider the following elements of $G$:
      $$e, a, a^2, \ldots, a^n.$$
      Since $G$ has $n$ elements and since there are $n + 1$ elements of $G$ 
      listed above, it must be the case that at least two of them are equal. If
      one of these two elements is $e$, then the proof is done. So assume that
      $a^i = a^j$, with $i < j$ and $i, j \in \{1, 2, \ldots, n\}$. Applying the
      Cancellation Law $i$ times to the equation $a^i = a^j$ gives us
      $a^{j - i} = e$. Since $j - i$ is a positive integer, we are done. \qed
      
%%%%%%%%%%%%%%%%%%%%%%%%%%%%%%%%%%Prob4.35%%%%%%%%%%%%%%%%%%%%%%%%%%%%%%%%%%%%%%
   \item[4.35] Show that if $(a * b)^2 = a^2 * b^2$ for $a$ and $b$ in a group
               $G$, then $a * b = b * a$.

      \textbf{Proof:} Let $\cyc{G, *}$ be a group. Suppose that
      $(a * b)^2 = a^2 * b^2$ for $a$ and $b$ in $G$. Then it follows that
      \begin{align}
         (a * b) * (a * b) = (a * a) * (b * b). \label{4_35_1}
      \end{align}
      By associativity of $*$, Equation \eqref{4_35_1} becomes
      \begin{align}
         a * (b * (a * b)) = a * (a * (b * b)). \label{4_35_2}
      \end{align}
      Applying the Cancellation Law to Equation \eqref{4_35_2} gives us
      \begin{align}
         b * (a * b) = a * (b * b). \label{4_35_3}
      \end{align}
      Again, by associativity of $*$, Equation \eqref{4_35_3} gives us
      \begin{align}
         (b * a) * b = (a * b) * b, \label{4_35_4}
      \end{align}
      so that $b * a = a * b$ by applying the Cancellation Law to
      \eqref{4_35_4}. \qed
%%%%%%%%%%%%%%%%%%%%%%%%%%%%%%%%%%Prob4.36%%%%%%%%%%%%%%%%%%%%%%%%%%%%%%%%%%%%%%
   \item[4.36] Let $G$ be a group and let $a, b \in G$. Show that
               $(a * b)' = a' * b'$ if and only if $a * b = b * a$.

      \textbf{Proof:} Let $\cyc{G, *}$ be a group. Let $a, b \in G$.

      $(\Rightarrow)$ Suppose that $(a * b)' = a' * b'$. Since
      $(b * a)' = a' * b'$, it follows that $(a * b)' = (b * a)'$, Taking the
      inverse of each side of the equation $(a * b)' = (b * a)'$ results in
      $a * b = b * a$.

      $(\Leftarrow)$ Now suppose that $a * b = b * a$. Then it follows that
      $(a * b)' = (b * a)' = a' * b'$. \qed
%%%%%%%%%%%%%%%%%%%%%%%%%%%%%%%%%%Prob4.37%%%%%%%%%%%%%%%%%%%%%%%%%%%%%%%%%%%%%%
   \item[4.37] Let $G$ be a group and suppose that $a * b * c = e$ for
               $a, b, c \in G$. Show that $b * c * a = e$ also.

      \textbf{Proof:} Let $\cyc{G, *}$ be a group. Let $a, b, c \in G$, with
      $a * b * c = e$. So we have that $a * (b * c) = e$. This says that the
      inverse of $a$ is $b * c$. Since $\cyc{G, *}$ is a group, it must be the
      case that $a * a' = a' * a = e$, so that $a * (b * c) = (b * c) * a =
      b * c * a = e$. \qed
%%%%%%%%%%%%%%%%%%%%%%%%%%%%%%%%%%Prob4.38%%%%%%%%%%%%%%%%%%%%%%%%%%%%%%%%%%%%%%
   \item[4.38] Prove that a set $G$, together with a binary operation $*$ on $G$
               satisfying the left axioms 1, 2, and 3 given on page 43, is a
               group.

      \textbf{Proof:} Let $e_L$ be the left identity of $\cyc{G, *}$ and let
      $a \in G$. It suffices to show that $e_L * a = a * e_L = a$ and
      $a_L * a = a * a_L = e_L$ where $a_L$ is the left identity of $a$. We have
      \begin{align*}
         (a * a_L) * (a * a_L) &= a * (a_L * a) * a_L \\
                               &= a * e_L * a_L \\
                               &= a * (e_L * a_L) \\
                               &= a * a_L,
      \end{align*}
      so that
      \begin{align}
         (a * a_L) * (a * a_L) = a * a_L. \label{4_38_1}
      \end{align}

      Let $y$ be the left identity of $(a * a_L)$. Multiplying Equation
      \eqref{4_38_1} on the left by $y$ yields 
      $$e_L * (a * a_L) = e_L,$$
      so that $a * a_L = e_L$. It now remains to show that $a * e_L = e_L * a$.
      So we have that
      \begin{align*}
         a * e_L &= a * (a_L * a) \\
                 &= (a * a_L) * a \\
                 &= e_L * a.
      \end{align*}
      That is, $a * e_L = e_L * a$. \qed
%%%%%%%%%%%%%%%%%%%%%%%%%%%%%%%%%%Prob4.39%%%%%%%%%%%%%%%%%%%%%%%%%%%%%%%%%%%%%%
   \item[4.39] Prove that a nonempty set $G$, together with an associative
               binary operation $*$ on $G$ such that  
               $$a * x = b \mbox{ and } y * a = b \mbox{ have solutions in } G
                 \mbox{ for all } a, b \in G,$$
               is a group.

      \textbf{Proof:} Consider $a, b \in G$. Then it follows that the equation
      $x * a = a$ has a solution in $G$. Call this solution $x_1$. We now want
      to show that $x_1$ is a left identity of $\cyc{G, *}$, so it suffices to
      show that $x_1 * b = b$. By our supposition, it must be the case that
      $a * x_2 = b$ for some $x_2 \in G$. So we have that
      $x_1 * b = x_1 * (a * x_2) = (x_1 * a) * x_2 = a * x_2 = b$. So $x_1$ is
      a left identity of $\cyc{G, *}$. By our supposition, the equation
      $y * a = x_1$ has a solution in $G$. This solution is the left invserse of
      $a$. Since the binary structure $\cyc{G, *}$ has a left identity and since
      every element in $G$ has a left inverse, it follows by Exercise 4.38 that
      $\cyc{G, *}$ is a group. \qed
%%%%%%%%%%%%%%%%%%%%%%%%%%%%%%%%%%Prob4.40%%%%%%%%%%%%%%%%%%%%%%%%%%%%%%%%%%%%%%
   \item[4.40] Let $\cyc{G, \cdot}$ be a group. Consider the binary operation
               $*$ on the set $G$ defined by
               $$a * b = b \cdot a$$
               for $a, b \in G$. Show that $\cyc{G, *}$ is a group and that
               $\cyc{G, *}$ is actually isomorphic to $\cyc{G, \cdot}$.

      \textbf{Proof:} Since $\cdot$ is closed on $G$, it follows that $*$ is
      closed on $G$. Let $e$ be the identity of $\cyc{G, \cdot}$, and let
      $a, b, c \in G$. Since $a \cdot e = e \cdot a = a$, it follows that
      $e * a = a * e = a$, so that $e$ is also an identity of $\cyc{G, *}$.
      Similarly $a \cdot a' = a' \cdot a = e$ implies that
      $a' * a = a * a' = e$, so that every element of $G$ has an inverse under
      $*$. Also since $(c \cdot b) \cdot a = c \cdot (b \cdot a)$, it must be
      the case that $a * (b * c) = (a * b) * c$, so that the binary structure
      $\cyc{G, *}$ is associative. Thus $\cyc{G, *}$ is a group. Now we consider
      a map from $\cyc{G, *}$ to $\cyc{G, \cdot}$: $\phi : G \rightarrow G$ 
      defined by $\phi(g) = g'$, for all $g \in G$. If $\phi(x) = \phi(y)$ for
      some $x, y \in G$, then we have that $x' = y'$, and taking the inverse of
      both sides of the equation $x' = y'$ gives us $x = y$, so that the
      function $\phi$ is injective. For any $z \in G$, we have that
      $\phi(z') = (z')' = z$. Thus $\phi$ is onto. We have shown that $\phi$ is
      a bijection. Now it remains to show the homorphism property holds for
      $\phi$. Thus we
      have that
      \begin{align*}
         \phi(a * b) &= (a * b)'       \\
                     &= (b \cdot a)'   \\
                     &= a' \cdot b'    \\
                     &= \phi(a) \cdot \phi(b).
      \end{align*}
      That is $\cyc{G, *} \simeq \cyc{G, \cdot}$. \qed
%%%%%%%%%%%%%%%%%%%%%%%%%%%%%%%%%%Prob4.41%%%%%%%%%%%%%%%%%%%%%%%%%%%%%%%%%%%%%%
   \item[4.41] Let $G$ be a group and let $g$ be one fixed element of $G$. Show
               that the map $i_g$, such that $i_g(x) = gxg'$ for $x \in G$, is
               an isomorphism of $G$ with itself.

      \textbf{Proof:} First we must show injectivity and surjectivity of $i_g$.
      So assume that $i_g(x) = i_g(y)$ for some $x, y \in G$. Then it follows
      that $gxg' = gyg'$. Applying the Left Cancellation Law and then applying
      the Right Cancellation Law to the equation $gxg' = gyg'$ results in
      $x = y$, so that $i_g$ is injective.  Now let $z$ be an element of $G$.
      Then we have that $i_g(g'zg) = gg'zgg' = eze = z$. That is, $i_g$ is onto.
      Finally we must show that the homomorphism property holds for $i_g$. So we
      have that
      \begin{align*}
         i_g(xy) &= gxyg'        \\
                 &= gxeyg'       \\
                 &= gxg'gyg      \\
                 &= i_g(x)i_g(y).
      \end{align*} \qed
\end{enumerate}
