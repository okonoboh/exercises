In Exercises 1 through 9, compute the given arithmetic expression and give the
answer in the form $a + bi$ for $a, b \in \R$.
\begin{enumerate}
%%%%%%%%%%%%%%%%%%%%%%%%%%%%%%%%%%Prob1.1%%%%%%%%%%%%%%%%%%%%%%%%%%%%%%%%%%%%%%%
   \item[1.1] $i^3$.
   
      \textbf{Solution:} $i^3 = i^2 \cdot i = -1 \cdot i = -i$.
%%%%%%%%%%%%%%%%%%%%%%%%%%%%%%%%%%Prob1.2%%%%%%%%%%%%%%%%%%%%%%%%%%%%%%%%%%%%%%%
   \item[1.2] $i^4$.
   
      \textbf{Solution:} $i^4 = i^2 \cdot i^2 = -1 \cdot -1 = 1$.
%%%%%%%%%%%%%%%%%%%%%%%%%%%%%%%%%%Prob1.3%%%%%%%%%%%%%%%%%%%%%%%%%%%%%%%%%%%%%%%
   \item[1.3] $i^{23}$.
   
      \textbf{Solution:} $i^{23} = (i^4)^5 \cdot i^3 = (1)^5 \cdot -i = -i$.
%%%%%%%%%%%%%%%%%%%%%%%%%%%%%%%%%%Prob1.4%%%%%%%%%%%%%%%%%%%%%%%%%%%%%%%%%%%%%%%
   \item[1.4] $(-i)^{35}$.
   
      \textbf{Solution:} $(-i)^{35} = (-1)^{35} \cdot i^{35} =
      -1 \cdot i^{23} \cdot (i^4)^3 = -1 \cdot -i \cdot 1 = i$.
%%%%%%%%%%%%%%%%%%%%%%%%%%%%%%%%%%Prob1.5%%%%%%%%%%%%%%%%%%%%%%%%%%%%%%%%%%%%%%%
   \item[1.5] $(4 - i)(5 + 3i)$.
   
      \textbf{Solution:} $(4 - i)(5 + 3i) = 20 + 12i - 5i + 3 = 23 + 7i$.
%%%%%%%%%%%%%%%%%%%%%%%%%%%%%%%%%%Prob1.6%%%%%%%%%%%%%%%%%%%%%%%%%%%%%%%%%%%%%%%
   \item[1.6] $(8 + 2i)(3 - i)$.
   
      \textbf{Solution:} $(8 + 2i)(3 - i) = 24 - 8i + 6i + 2 = 26 - 2i$.
%%%%%%%%%%%%%%%%%%%%%%%%%%%%%%%%%%Prob1.7%%%%%%%%%%%%%%%%%%%%%%%%%%%%%%%%%%%%%%%
   \item[1.7] $(2 - 3i)(4 + i) + (6 - 5i)$.
   
      \textbf{Solution:} $(2 - 3i)(4 + i) + (6 - 5i) =
      8 + 2i - 12i + 3 + 6 - 5i = 17 - 15i$.
%%%%%%%%%%%%%%%%%%%%%%%%%%%%%%%%%%Prob1.8%%%%%%%%%%%%%%%%%%%%%%%%%%%%%%%%%%%%%%%
   \item[1.8] $(1 + i)^3$.
   
      \textbf{Solution:} $\displaystyle(1 + i)^3 = \binom{3}{0} +
      \binom{3}{1} i + \binom{3}{2} i^2 + \binom{3}{3} i^3 = 1 + 3i - 3 - i =
       -2 + 2i$.
%%%%%%%%%%%%%%%%%%%%%%%%%%%%%%%%%%Prob1.9%%%%%%%%%%%%%%%%%%%%%%%%%%%%%%%%%%%%%%%
   \item[1.9] $(1 - i)^5$.
   
      \textbf{Solution:}
      \begin{align*}
         \displaystyle(1 - i)^5 &= \binom{5}{0} + \binom{5}{1} (-i) +
         \binom{5}{2} (-i)^2 + \binom{5}{3} (-i)^3 + \binom{5}{4} (-i)^4 + 
         \binom{5}{5} (-i)^5 \\
         &= 1 - 5i - 10 + 10i + 5 - i \\
         &= -4 + 4i.
      \end{align*}

%%%%%%%%%%%%%%%%%%%%%%%%%%%%%%%%%%Prob1.10%%%%%%%%%%%%%%%%%%%%%%%%%%%%%%%%%%%%%%
   \item[1.10] Find $|3 - 4i|$.
		
		\textbf{Solution:}
		
		$|3 - 4i| = \sqrt{3^2 + (-4)^2} = 5$.
%%%%%%%%%%%%%%%%%%%%%%%%%%%%%%%%%%Prob1.11%%%%%%%%%%%%%%%%%%%%%%%%%%%%%%%%%%%%%%
   \item[1.11] Find $|6 + 4i|$.
		
		\textbf{Solution:}
		
		$|6 + 4i| = \sqrt{6^2 + 4^2} = 2\sqrt{13}$.
\end{enumerate}

\noindent In Exercises 12 through 15, write the given complex $z$ in the polar
form $|z|(p + qi)$ where $|p + qi| = 1$. 
\begin{enumerate}
%%%%%%%%%%%%%%%%%%%%%%%%%%%%%%%%%%Prob1.12%%%%%%%%%%%%%%%%%%%%%%%%%%%%%%%%%%%%%%
   \item[1.12] $3 - 4i$.
		
		\textbf{Solution:}
		
		$\displaystyle 3 - 4i = 5\left(\frac{3}{5} - \frac{4}{5}i\right)$.
%%%%%%%%%%%%%%%%%%%%%%%%%%%%%%%%%%Prob1.13%%%%%%%%%%%%%%%%%%%%%%%%%%%%%%%%%%%%%%
   \item[1.13] $-1 + i$.
		
		\textbf{Solution:}
		
		$\displaystyle -1 + i = \sqrt{2}\left(-\frac{\sqrt{2}}{2} + 
		 \frac{\sqrt{2}}{2}i\right)$.
%%%%%%%%%%%%%%%%%%%%%%%%%%%%%%%%%%Prob1.14%%%%%%%%%%%%%%%%%%%%%%%%%%%%%%%%%%%%%%
   \item[1.14] $12 + 5i$.
		
		\textbf{Solution:}
		
		$\displaystyle 12 + 5i = 13\left(\frac{12}{13} - \frac{5}{13}i\right)$.
%%%%%%%%%%%%%%%%%%%%%%%%%%%%%%%%%%Prob1.15%%%%%%%%%%%%%%%%%%%%%%%%%%%%%%%%%%%%%%
   \item[1.15] $-3 + 5i$.
		
		\textbf{Solution:}
		
		$\displaystyle -3 + 5i = \sqrt{34}\left(-\frac{3\sqrt{34}}{34} + 
		 \frac{5\sqrt{34}}{34}i\right)$.
\end{enumerate}

\noindent In Exercises 16 through 21, find all solutions in $\C$ of the given
equation.
\begin{enumerate}
%%%%%%%%%%%%%%%%%%%%%%%%%%%%%%%%%%Prob1.16%%%%%%%%%%%%%%%%%%%%%%%%%%%%%%%%%%%%%%
   \item[1.16] $z^4 = 1$.
		
		\textbf{Solution:}
		
		In polar form, we have $z = |z|e^{i\theta_z}$ and $1 = |1|e^{i(2\pi k)}$,
		where $k$ is an integer and $\theta_z \in [0, 2\pi)$. So we then have
		$|z|^4e^{i4\theta_z} = |1|e^{i(2\pi k)}$, so that
		$$z \in \{e^{i\frac{\pi k}{2}} : k = 0, 1, 2, 3\}.$$
%%%%%%%%%%%%%%%%%%%%%%%%%%%%%%%%%%Prob1.17%%%%%%%%%%%%%%%%%%%%%%%%%%%%%%%%%%%%%%
   \item[1.17] $z^4 = -1$.
		
		\textbf{Solution:}
		
		In polar form, we have $|z|^4e^{i4\theta_z} = |1|e^{i(\pi + 2\pi k)}$, so
		that
		$$z \in \{e^{i\frac{\pi + 2\pi k}{4}} : k = 0, 1, 2, 3\}.$$
%%%%%%%%%%%%%%%%%%%%%%%%%%%%%%%%%%Prob1.18%%%%%%%%%%%%%%%%%%%%%%%%%%%%%%%%%%%%%%
   \item[1.18] $z^3 = -8$.
		
		\textbf{Solution:}
		
		In polar form, we have $|z|^3e^{i3\theta_z} = |8|e^{i(\pi + 2\pi k)}$, so
		that
		$$z \in \{2e^{i\frac{\pi + 2\pi k}{3}} : k = 0, 1, 2\}.$$
%%%%%%%%%%%%%%%%%%%%%%%%%%%%%%%%%%Prob1.19%%%%%%%%%%%%%%%%%%%%%%%%%%%%%%%%%%%%%%
   \item[1.19] $z^3 = -27i$.
		
		\textbf{Solution:}
		
		In polar form, we have $|z|^3e^{i3\theta_z} = |27|e^{i(3\pi/2 + 2\pi k)}$,
		so that
		$$z \in \{3e^{i\frac{3\pi + 2\pi k}{6}} : k = 0, 1, 2\}.$$
%%%%%%%%%%%%%%%%%%%%%%%%%%%%%%%%%%Prob1.20%%%%%%%%%%%%%%%%%%%%%%%%%%%%%%%%%%%%%%
   \item[1.20] $z^6 = 1$.
		
		\textbf{Solution:}
		
		In polar form, we have $|z|^6e^{i6\theta_z} = |1|e^{i(2\pi k)}$, so that
		$$z \in \{e^{i\frac{\pi}{3}} : k = 0, 1, 2, 3, 4, 5\}.$$
%%%%%%%%%%%%%%%%%%%%%%%%%%%%%%%%%%Prob1.21%%%%%%%%%%%%%%%%%%%%%%%%%%%%%%%%%%%%%%
   \item[1.21] $z^6 = -64$.
		
		\textbf{Solution:}
		
		In polar form, we have $|z|^6e^{i6\theta_z} = |64|e^{i(\pi + 2\pi k)}$, so
		that
		$$z \in \{2e^{i\frac{\pi + 2\pi k}{6}} : k = 0, 1, 2, 3, 4, 5\}.$$
\end{enumerate}

\noindent In Exercises 22 through 27, compute the given expression using the
indicated modular addition.
\begin{enumerate}
%%%%%%%%%%%%%%%%%%%%%%%%%%%%%%%%%%Prob1.22%%%%%%%%%%%%%%%%%%%%%%%%%%%%%%%%%%%%%%
   \item[1.22] $10 +_{17} 16$.
		
		\textbf{Solution:}
		
		$10 +_{17} 16 = 26 - 17 = 9$.
%%%%%%%%%%%%%%%%%%%%%%%%%%%%%%%%%%Prob1.23%%%%%%%%%%%%%%%%%%%%%%%%%%%%%%%%%%%%%%
   \item[1.23] $8 +_{10} 6$.
		
		\textbf{Solution:}
		
		$8 +_{10} 6 = 14 - 10 = 4$.
%%%%%%%%%%%%%%%%%%%%%%%%%%%%%%%%%%Prob1.24%%%%%%%%%%%%%%%%%%%%%%%%%%%%%%%%%%%%%%
   \item[1.24] $20.5 +_{25} 19.3$.
		
		\textbf{Solution:}
		
		$20.5 +_{25} 19.3 = 39.8 - 25 = 14.8$.
%%%%%%%%%%%%%%%%%%%%%%%%%%%%%%%%%%Prob1.25%%%%%%%%%%%%%%%%%%%%%%%%%%%%%%%%%%%%%%
   \item[1.25] $\frac{1}{2} +_1 \frac{7}{8}$.
		
		\textbf{Solution:}
		
		$\frac{1}{2} +_1 \frac{7}{8} = \frac{11}{8} - 1 = \frac{3}{8}$.
%%%%%%%%%%%%%%%%%%%%%%%%%%%%%%%%%%Prob1.26%%%%%%%%%%%%%%%%%%%%%%%%%%%%%%%%%%%%%%
   \item[1.26] $\frac{3\pi}{4} +_{2\pi} \frac{3\pi}{2}$.
		
		\textbf{Solution:}
		
		$\frac{3\pi}{4} +_{2\pi} \frac{3\pi}{2} =
		 \frac{9\pi}{4} - 2\pi = \frac{\pi}{4}$.
%%%%%%%%%%%%%%%%%%%%%%%%%%%%%%%%%%Prob1.27%%%%%%%%%%%%%%%%%%%%%%%%%%%%%%%%%%%%%%
   \item[1.27] $2\sqrt{2} +_{\sqrt{32}} 3\sqrt{2}$.
		
		\textbf{Solution:}
		
		$2\sqrt{2} +_{\sqrt{32}} 2\sqrt{2} = 5\sqrt{2} - \sqrt{32} = \sqrt{2}$.
%%%%%%%%%%%%%%%%%%%%%%%%%%%%%%%%%%Prob1.28%%%%%%%%%%%%%%%%%%%%%%%%%%%%%%%%%%%%%%
   \item[1.28] Explain why the expression $5 +_6 8$ in $\R_6$ makes no sense.
		
		\textbf{Solution:} It makes no sense because $8 \notin [0, 6)$.
\end{enumerate}

\noindent In Exercises 29 through 34, find $\textit{all}$ solutions $x$ of the
given equation.
\begin{enumerate}
%%%%%%%%%%%%%%%%%%%%%%%%%%%%%%%%%%Prob1.29%%%%%%%%%%%%%%%%%%%%%%%%%%%%%%%%%%%%%%
   \item[1.29] $x +_{15} 7 = 3$ in $\Z_{15}$.
		
		\textbf{Solution:} $x = 3 - 7 = -4 + 15 = 11$.
%%%%%%%%%%%%%%%%%%%%%%%%%%%%%%%%%%Prob1.30%%%%%%%%%%%%%%%%%%%%%%%%%%%%%%%%%%%%%%
   \item[1.30] $x +_{2\pi} \frac{3\pi}{2} = \frac{3\pi}{4}$ in $\R_{2\pi}$.
		
		\textbf{Solution:} $x = \frac{3\pi}{4} - \frac{3\pi}{2} =
		-\frac{3\pi}{4} + 2\pi = \frac{5\pi}{4}$.
%%%%%%%%%%%%%%%%%%%%%%%%%%%%%%%%%%Prob1.31%%%%%%%%%%%%%%%%%%%%%%%%%%%%%%%%%%%%%%
   \item[1.31] $x +_7 x = 3$ in $\Z_7$.
		
		\textbf{Solution:} $x = 5$.
%%%%%%%%%%%%%%%%%%%%%%%%%%%%%%%%%%Prob1.32%%%%%%%%%%%%%%%%%%%%%%%%%%%%%%%%%%%%%%
   \item[1.32] $x +_7 x +_7 x = 5$ in $\Z_7$.
		
		\textbf{Solution:} $x = 4$.
%%%%%%%%%%%%%%%%%%%%%%%%%%%%%%%%%%Prob1.33%%%%%%%%%%%%%%%%%%%%%%%%%%%%%%%%%%%%%%
   \item[1.33] $x +_{12} x = 2$ in $\Z_{12}$.
		
		\textbf{Solution:} $x = 1, 7$.
%%%%%%%%%%%%%%%%%%%%%%%%%%%%%%%%%%Prob1.34%%%%%%%%%%%%%%%%%%%%%%%%%%%%%%%%%%%%%%
   \item[1.34] $x +_4 x +_4 x +_4 x = 0$ in $\Z_4$.
		
		\textbf{Solution:} $x = 0, 1, 2, 3$.
%%%%%%%%%%%%%%%%%%%%%%%%%%%%%%%%%%Prob1.35%%%%%%%%%%%%%%%%%%%%%%%%%%%%%%%%%%%%%%
   \item[1.35] Example 1.15 asserts that there is an isomorphism of $U_8$ with
               $\Z_8$ in which $\zeta = e^{i(\pi/4)} \leftrightarrow 5$ and
               $\zeta^2 \leftrightarrow 2$. Find the element of $\Z_8$ that
               corresponds to each of the remaining six elements $\zeta^m$ in
               $U_8$ for $m = 0, 3, 4, 5, 6,$ and 7.

      \textbf{Solution:}
      \begin{align*}
         \zeta^3 = \zeta\zeta^2 &\leftrightarrow 5 + 2 = 7 \\
         \zeta^4 = \zeta ^2\zeta^2 &\leftrightarrow 2 + 2 = 4 \\
         \zeta^5 = \zeta ^2\zeta^3 &\leftrightarrow 2 + 7 = 1 \\
         \zeta^6 = \zeta ^3\zeta^3 &\leftrightarrow 7 + 7 = 6 \\
         \zeta^7 = \zeta ^3\zeta^4 &\leftrightarrow 7 + 4 = 3 \\
         \zeta^0 &\leftrightarrow 0
      \end{align*}
%%%%%%%%%%%%%%%%%%%%%%%%%%%%%%%%%%Prob1.36%%%%%%%%%%%%%%%%%%%%%%%%%%%%%%%%%%%%%%
   \item[1.36] There is an isomorphism of $U_7$ with $\Z_7$ in which
               $\zeta = e^{i(2\pi/7)} \leftrightarrow 4$. Find the element in
               $\Z_7$ to which $\zeta^m$ must correspond for
               $m = 0, 2, 3, 4, 5$, and 6.

      \textbf{Solution:}
      \begin{align*}
         \zeta^2 = \zeta\zeta &\leftrightarrow 4 + 4 = 1 \\
         \zeta^3 = \zeta\zeta^2 &\leftrightarrow 1 + 4 = 5 \\
         \zeta^4 = \zeta ^2\zeta^2 &\leftrightarrow 1 + 1 = 2 \\
         \zeta^5 = \zeta ^2\zeta^3 &\leftrightarrow 1 + 5 = 6 \\
         \zeta^6 = \zeta ^3\zeta^3 &\leftrightarrow 5 + 5 = 3 \\
         \zeta^0 &\leftrightarrow 0
      \end{align*}
%%%%%%%%%%%%%%%%%%%%%%%%%%%%%%%%%%Prob1.37%%%%%%%%%%%%%%%%%%%%%%%%%%%%%%%%%%%%%%
   \item[1.37] Why can there be no isomorphism of $U_6$ with $\Z_6$ in which
               $\zeta = e^{i(\pi/3)}$ corresponds to 4?

      \textbf{Solution:} It wouldn't be an isomorphism since $\zeta_2$ and
      $\zeta_5$ would be mapped to the same element $2 \in \Z_6$, thus making
      the map not injective.
%%%%%%%%%%%%%%%%%%%%%%%%%%%%%%%%%%Prob1.38%%%%%%%%%%%%%%%%%%%%%%%%%%%%%%%%%%%%%%
   \item[1.38] Derive the formulas
               $$\sin(a + b) = \sin a \cos b + \cos a \sin b$$
               and
               $$\cos(a + b) = \cos a \cos b - \sin a \sin b$$
               by using Euler's formula and computing $e^{ia}e^{ib}$.

      \textbf{Proof:} Using Euler's formula, it follows that
      \begin{align*}
         e^{ia}e^{ib} &= (\cos a + i \sin a)(\cos b + i \sin b) \\
            &= \cos a\cos b - \sin a\sin b + (\sin a\cos b + \sin b\cos a)i.
      \end{align*}

      But $e^{ia}e^{ib} = e^{i(a+b)} = \cos(a + b) + i\sin(a + b)$. Equating
      the real and complex coefficients gives us the desired result.
      
%%%%%%%%%%%%%%%%%%%%%%%%%%%%%%%%%%Prob1.39%%%%%%%%%%%%%%%%%%%%%%%%%%%%%%%%%%%%%%
   \item[1.39] Let $z_1 = |z_1|(\cos\theta_1 + i\sin\theta_1)$ and
               $z_2 = |z_2|(\cos\theta_2 + i\sin\theta_2)$. Use the 
               trigonometric identities in Exercise 38 to derive
               $z_1z_2 = |z_1||z_2|[\cos(\theta_1 + \theta_2) +
                i\sin(\theta_1 + \theta_2)]$.

      \textbf{Solution:} This immediately follows from Exercise 38.
%%%%%%%%%%%%%%%%%%%%%%%%%%%%%%%%%%Prob1.40%%%%%%%%%%%%%%%%%%%%%%%%%%%%%%%%%%%%%%
   \item[1.40]
      \begin{enumerate}
         \item Derive a formula for $\cos 3\theta$ in terms of $\sin\theta$ and
               $\cos\theta$ using Euler's formula.
         \item Derive the formula
               $\cos 3\theta = 4 \cos^3\theta - 3\cos\theta$ from part(a) and
               the identity $\sin^2\theta + \cos^2\theta = 1$.
      \end{enumerate}

      \textbf{Solution:}

      \begin{enumerate}
         \item Since $e^{i3\theta} = (e^{i\theta})^3$, it follows that
               \begin{align*}
                  \cos 3\theta + i\sin 3\theta &= (\cos\theta + i\sin\theta)^3\\
                     &= \cos^3\theta - 3\sin^2\theta\cos\theta +
                        (3\cos^2\theta\sin\theta - \sin^3\theta)i.
               \end{align*}

               By equating real coefficients, we have that 
               $$\cos 3\theta = \cos^3\theta - 3\sin^2\theta\cos\theta.$$

         \item \begin{align*}
                  \cos 3\theta &= \cos^3\theta - 3\sin^2\theta\cos\theta \\
                     &= \cos^3\theta - 3\cos\theta(1 - \cos^2\theta) \\
                     &= 4 \cos^3\theta - 3\cos\theta.
               \end{align*}
      \end{enumerate}
%%%%%%%%%%%%%%%%%%%%%%%%%%%%%%%%%%Prob1.41%%%%%%%%%%%%%%%%%%%%%%%%%%%%%%%%%%%%%%
   \item[1.41] Recall the power series expansions
               \begin{align*}
                  e^x &= 1 + x + \frac{x^2}{2!} + \frac{x^3}{3!} +
                        \frac{x^4}{4!} + \cdots + \frac{x^n}{n!} + \cdots, \\
                  \sin x &= x - \frac{x^3}{3!} + \frac{x^5}{5!} - \frac{x^7}{7!}
                            + \cdots + (-1)^{n - 1}\frac{x^{2n - 1}}{(2n - 1)!} 
                            + \cdots, \mbox{ and } \\
                  \cos x &= 1 - \frac{x^2}{2!} + \frac{x^4}{4!} - \frac{x^6}{6!}
                            + \cdots + (-1)^n\frac{x^{2n}}{(2n)!} + \cdots      
               \end{align*}
               from calculus. Derive Euler's formula
               $e^{i\theta} = \cos\theta + i \sin\theta$ formally from these
               series expansions.

      \textbf{Solution:} Substituting $i\theta$ for $x$ in the power series 
      expansion for $e$ gives us the desired result.
            
\end{enumerate}
