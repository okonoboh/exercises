\noindent      \textbf{Computations}

\noindent      In Exercises 1 through 6, determine whether the given subset of 
               the complex numbers is a subgroup of the group $\C$ of complex 
               numbers under addition. \\

\noindent      \textit{We shall be using the subgroup criterion: that is if $G$ 
               is a group, then $H \le G$ if and only if $H$ is not empty and 
               $ab^{-1} \in H$ for all $a, b \in H$. See Exercise 5.45 below.}

\begin{enumerate}
%%%%%%%%%%%%%%%%%%%%%%%%%%%%%%%%%%Prob5.1%%%%%%%%%%%%%%%%%%%%%%%%%%%%%%%%%%%%%%%
   \item[5.1]  $\R$. Claim that $\cyc{\R, +} \le \cyc{\C, +}$.

      \textbf{Proof.} $\R$ is clearly nonempty and since $a - b \in \R$ for all
      $a$, $b$ in $\R$, it follows by the subgroup criterion that
      $\cyc{\R, +} \le \cyc{\C, +}$. \qed
%%%%%%%%%%%%%%%%%%%%%%%%%%%%%%%%%%Prob5.2%%%%%%%%%%%%%%%%%%%%%%%%%%%%%%%%%%%%%%%
   \item[5.2]  $\Q^+$. Claim that $\cyc{\Q^+, +}$ is not a subgroup of
               $\cyc{\C, +}$.

      \textbf{Proof.} Since $0 \notin \Q^+$ and since 0 is the identity 
      element of $\cyc{\C, +}$, it follows that $\cyc{Q, +}$ is not a subgroup
      of $\cyc{\C, +}$. \qed
%%%%%%%%%%%%%%%%%%%%%%%%%%%%%%%%%%Prob5.3%%%%%%%%%%%%%%%%%%%%%%%%%%%%%%%%%%%%%%%
   \item[5.3]  $7\Z$. Claim that $\cyc{7\Z, +} \le \cyc{\C, +}$.

      \textbf{Proof.} $7\Z$ is clearly nonempty, so consider $a, b \in 7\Z$.
      Then there exist integers $x_1$ and $x_2$ such that $a = 7x_1$ and
      $b = 7x_2$; thus $a - b = 7(x_1 - x_2) \in 7\Z$. Hence, by the subgroup
      criterion, $7\Z$ is a subgroup of $\C$ under addition. \qed
%%%%%%%%%%%%%%%%%%%%%%%%%%%%%%%%%%Prob5.4%%%%%%%%%%%%%%%%%%%%%%%%%%%%%%%%%%%%%%%
   \item[5.4]  The set $i\R$ of pure imaginary numbers including 0. Claim that
               $\cyc{i\R, +} \le \cyc{\C, +}$.

      \textbf{Proof.} $i\R$ is clearly nonempty, so consider $a, b \in i\R$.
      Then there exist real numbers $x_1$ and $x_2$ such that $a = ix_1$ and
      $b = ix_2$; thus $a - b = i(x_1 - x_2) \in i\R$. Hence, by the subgroup
      criterion, $i\R$ is a subgroup  of $\C$ under addition. \qed
%%%%%%%%%%%%%%%%%%%%%%%%%%%%%%%%%%Prob5.5%%%%%%%%%%%%%%%%%%%%%%%%%%%%%%%%%%%%%%%
   \item[5.5]  The set $\pi\Q$ of rational multiples of $\pi$. Claim that
               $\cyc{\pi\Q, +} \le \cyc{\C, +}$.

      \textbf{Proof.} $\pi\Q$ is clearly nonempty, so consider $a, b \in \pi\Q$.
      Then there exist rational numbers $p$ and $q$ such that $a = p\pi$ and
      $b = q\pi$; thus $a - b = \pi(p - q) \in \pi\Q$. Hence, by the subgroup
      criterion, $\pi\Q$ is a subgroup of $\C$ under addition. \qed
%%%%%%%%%%%%%%%%%%%%%%%%%%%%%%%%%%Prob5.6%%%%%%%%%%%%%%%%%%%%%%%%%%%%%%%%%%%%%%%
   \item[5.6]  The set $H = \{\pi^n : n \in \Z\}$. Claim that $\cyc{H, +}$ is 
               not a subgroup of $\cyc{\C, +}$.

      \textbf{Proof.} We know that $\pi^n > 0$ for every integer $n$. It
      follows that the set $\{\pi^n : n \in \Z\}$ is not a subgroup of $\C$ 
      under addition since the former does not contain 0. \qed
%%%%%%%%%%%%%%%%%%%%%%%%%%%%%%%%%%Prob5.7%%%%%%%%%%%%%%%%%%%%%%%%%%%%%%%%%%%%%%%
   \item[5.7]  Which of the sets in Exercises 1 through 6 are subgroups of the
               group $\C^*$ of nonzero complex numbers under multiplication?

      \textbf{Proof.} Claim that the sets $\Q^+$ and
      $H = \{\pi^n : n \in \Z\}$ are both subgroups of $\C$ under 
      multiplication. These two sets are clearly nonempty. It suffices to show 
      that they both satisfy the subgroup criterion. So consider $a, b \in \Q^+$
      and $c, d \in H$. Then there exist positive integers $p, q, r, s$ and
      integers $u, v$ such that $a = p/q$, $b = r/s$, $c = \pi^u$, and
      $d = \pi^v$. Thus $ab^{-1} = ps/qr$ and $cd^{-1} = \pi^{u - v}$, so that
      $ab^{-1} \in \Q^+$ and $cd^{-1} \in H$. We are done. The remaining sets 
      contain 0, and since 0 does not have a multiplicative inverse, these sets 
      are not subgroups of $\cyc{\C, \cdot}$. \qed
\end{enumerate}

\noindent      In Exercises 8 through 13, determine whether the given set of
               invertible $n \times n$ matrices with real number entries is a
               subgroup of $GL(n, \R)$.

\begin{enumerate}
%%%%%%%%%%%%%%%%%%%%%%%%%%%%%%%%%%Prob5.8%%%%%%%%%%%%%%%%%%%%%%%%%%%%%%%%%%%%%%%
   \item[5.8]  The $n \times n$ matrices with determinant 2.

      \textbf{Proof.} Let $F$ denote the set of $n \times n$ matrices with 
      determinant 2. We want to show that $F$ is not a subgroup of $GL(n, \R)$. 
      Since the determinant of the identity matrix---also the identity element
      of $GL(n, \R)$---is 1, it's not in $F$. Thus $F$ is not a subgroup of
      $G$. \qed
%%%%%%%%%%%%%%%%%%%%%%%%%%%%%%%%%%Prob5.9%%%%%%%%%%%%%%%%%%%%%%%%%%%%%%%%%%%%%%%
   \item[5.9]  The diagonal $n \times n$ matrices with no zeroes on the 
               diagonal.

      \textbf{Proof.} We showed in Exercise 4.13 that this was a group. Thus 
      this set is a subgroup of $GL(n, \R)$. \qed
%%%%%%%%%%%%%%%%%%%%%%%%%%%%%%%%%%Prob5.10%%%%%%%%%%%%%%%%%%%%%%%%%%%%%%%%%%%%%%
   \item[5.10] The upper-triangular $n \times n$ matrices with no zeroes on the
               diagonal.
   
      \textbf{Proof.} This is a subgroup of $GL(n, \R)$.[Will give proof later.]
      \qed
%%%%%%%%%%%%%%%%%%%%%%%%%%%%%%%%%%Prob5.11%%%%%%%%%%%%%%%%%%%%%%%%%%%%%%%%%%%%%%
   \item[5.11] The $n \times n$ matrices with determinant $-1$.

      \textbf{Proof.} This is not a subgroup of $GL(n, \R)$ for the same reason
      we gave in Exercise 5.8; that is, the identity element of $GL(n, \R)$ is
      not in the set of $n \times n$ matrices with determinant $-1$. \qed
%%%%%%%%%%%%%%%%%%%%%%%%%%%%%%%%%%Prob5.12%%%%%%%%%%%%%%%%%%%%%%%%%%%%%%%%%%%%%%
   \item[5.12] The $n \times n$ matrices with determinant $-1$ or 1.

      \textbf{Proof.} We showed in Exercise 4.18 that this was a group. Thus 
      this set is a subgroup of $GL(n, \R)$. \qed
%%%%%%%%%%%%%%%%%%%%%%%%%%%%%%%%%%Prob5.13%%%%%%%%%%%%%%%%%%%%%%%%%%%%%%%%%%%%%%
   \item[5.13] The set of all $n \times n$ matrices $A$ such that
               $A^TA = I_n$. [These matrices are called \textbf{orthogonal}.
               Recall that $A^T$, the \textit{transpose} of $A$, is the matrix
               whose $j$th column is the $j$th row of $A$ for $1 \le j \le n$,
               and that the transpose operation has the property
               $(AB)^T = B^TA^T$.]

      \textbf{Proof.} Let $F$ denote the set of all $n \times n$ matrices $A$
      such that $(A^T)A = I_n$. We want to show that $F \le GL(n, \R)$. It is
      clear that $I_n \in F$. So consider $A, B \in F$. It suffices to show that
      $AB^{-1} \in F$; that is, we must show that $(AB^{-1})^T(AB^{-1}) = I_n$. 
      Notice that $(A^T)A = I_n$ if and only if $A^{-1} = A^T$. Thus
      \begin{align*}
         (AB^{-1})^T(AB^{-1}) &= (AB^T)^T(AB^{-1}) \\ 
                              &= ((B^T)^TA^T)(AB^{-1}) \\
                              &= (BA^T)(AB^{-1}) \\
                              &= (BA^{-1})(AB^{-1}) \\
                              &= B((A^{-1})A)B^{-1} \\
                              &= BI_nB^{-1} \\
                              &= BB^{-1} \\
                              &= I_n. \\
      \end{align*}

      It follows that $F \le GL(n, \R)$. \qed
\end{enumerate}

\noindent      Let $F$ be the set of all real-valued functions with domain $\R$
               and let $\tilde{F}$ be the subset of $F$ consisting of those
               functions that have a nonzero value at every point in $\R$. In
               Exercises 14 through 19, determine whether the given subset of
               $F$ with the induced operation is (a) a subgroup of the group $F$
               under addition, (b) a subgroup of the group $\tilde{F}$ under
               multiplication.

\begin{enumerate}
%%%%%%%%%%%%%%%%%%%%%%%%%%%%%%%%%%Prob5.14%%%%%%%%%%%%%%%%%%%%%%%%%%%%%%%%%%%%%%
   \item[5.14] The subset $\tilde{F}$.

      \textbf{Proof.}

      \begin{enumerate}
         \item $\cyc{\tilde{F}, +}$ is not a subgroup of $\cyc{F, +}$ because 
               the zero function(the identity of $\cyc{F, +}$) is not a member
               of $\tilde{F}$.
         \item Yes, it trivially follows that
               $\cyc{\tilde{F}, \cdot} \le \cyc{\tilde{F}, \cdot}$.
      \end{enumerate} \qed
%%%%%%%%%%%%%%%%%%%%%%%%%%%%%%%%%%Prob5.15%%%%%%%%%%%%%%%%%%%%%%%%%%%%%%%%%%%%%%
   \item[5.15] The subset of all $f \in F$ such that $f(1) = 0$.

      \textbf{Proof.} Let $F_1$ denote the subset of all $f \in F$ such that 
      $f(1) = 0$.

      \begin{enumerate}
         \item The set $F_1$ is nonempty because it contains the zero function. 
               So consider $f, g \in F_1$. To conclude that
               $\cyc{F_1, +} \le \cyc{F, +}$, it suffices by the subgroup 
               criterion to show that $f - g \in F_1$. This immediately follows
               because $(f - g)(1) = f(1) - g(1) = 0$, so that $f - g \in F_1$.
         \item No, the statement $\cyc{F_1, \cdot} \le \cyc{\tilde{F}, \cdot}$
               is not true because the zero function is in $F_1$, but it is not
               in $\tilde{F}$.
      \end{enumerate} \qed
%%%%%%%%%%%%%%%%%%%%%%%%%%%%%%%%%%Prob5.16%%%%%%%%%%%%%%%%%%%%%%%%%%%%%%%%%%%%%%
   \item[5.16] The subset of all $f \in \tilde{F}$ such that $f(1) = 1$.

      \textbf{Proof.} Let $F_2$ denote the subset of all $f \in \tilde{F}$ such 
      that $f(1) = 1$.

      \begin{enumerate}
         \item $F_2$ clearly doesn't contain the zero function, so that
               $F_2$ cannot be a subgroup of $F$ under addition.
         \item The set $F_2$ is nonempty because it contains the identity       
               function under multiplication. So consider $f, g \in F_2$. To 
               conclude that $\cyc{F_2, \cdot} \le \cyc{\tilde{F}, \cdot}$, it 
               suffices by the subgroup criterion to show that $fg \in F_2$. 
               This immediately follows because $(fg)(1) = f(1)g(1) = 1$, so 
               that $fg \in F_2$.
      \end{enumerate} \qed
%%%%%%%%%%%%%%%%%%%%%%%%%%%%%%%%%%Prob5.17%%%%%%%%%%%%%%%%%%%%%%%%%%%%%%%%%%%%%%
   \item[5.17] The subset of all $f \in \tilde{F}$ such that $f(0) = 1$.

      \textbf{Solution:} The answers are the same as Exercise 5.16(a) and (b).
%%%%%%%%%%%%%%%%%%%%%%%%%%%%%%%%%%Prob5.18%%%%%%%%%%%%%%%%%%%%%%%%%%%%%%%%%%%%%%
   \item[5.18] The subset of all $f \in \tilde{F}$ such that $f(0) = -1$.

      \textbf{Proof.} Let $F_3$ denote the subset of all $f \in \tilde{F}$ such 
      that $f(0) = -1$.

      \begin{enumerate}
         \item $F_3$ clearly doesn't contain the zero function, so that
               $F_3$ cannot be a subgroup of $F$ under addition.
         \item The constant function $f(x) = -1$ for all real $x$ is a member of
               $F_3$, but $f \cdot f \notin F_3$ because $(f \cdot f)(-1) = 1$.
               Thus $F_3$ is not closed under multiplication so that it is not
               a subgroup of $\tilde{F}$.
      \end{enumerate} \qed
%%%%%%%%%%%%%%%%%%%%%%%%%%%%%%%%%%Prob5.19%%%%%%%%%%%%%%%%%%%%%%%%%%%%%%%%%%%%%%
   \item[5.19] The subset of all constant functions in $F$.

      \textbf{Proof.} Let $F_4$ denote the subset of all constant functions in 
      $F$.

      \begin{enumerate}
         \item It is clear that $F_4$ is a subgroup of $F$.
         \item $F_4$ is not a subgroup of $\tilde{F}$ because the former 
               contains the zero function, while latter doesn't.
      \end{enumerate} \qed
%%%%%%%%%%%%%%%%%%%%%%%%%%%%%%%%%%Prob5.20%%%%%%%%%%%%%%%%%%%%%%%%%%%%%%%%%%%%%%
   \item[5.20] Nine groups are given below. Give a complete list of all subgroup
               relations, of the form $G_i \le G_j$, that exist between these
               groups $G_1$, $G_2$, $\cdots$, $G_9$.

               \begin{tabular}{@{}l l l@{}}
                  $G_1$ & = & $\Z$ under addition \\
                  $G_2$ & = & $12\Z$ under addition \\
                  $G_3$ & = & $\Q^+$ under multiplication \\
                  $G_4$ & = & $\R$ under addition \\
                  $G_5$ & = & $\R^+$ under multiplication \\
                  $G_6$ & = & $\{\pi^n : n \in \Z\}$ under multiplication \\
                  $G_7$ & = & $3\Z$ under addition \\
                  $G_8$ & = & the set of all integral multiples of 6 under 
                              addition \\
                  $G_9$ & = & $\{6^n : n \in \Z\}$ under multiplication
               \end{tabular}

      \textbf{Solution:}

      \begin{itemize}
         \item $G_2 \le G_8 \le G_7 \le G_1 \le G_4$.
         \item $G_9 \le G_3 \le G_5$.
         \item $G_6 \le G_5$.
      \end{itemize}
%%%%%%%%%%%%%%%%%%%%%%%%%%%%%%%%%%Prob5.21%%%%%%%%%%%%%%%%%%%%%%%%%%%%%%%%%%%%%%
   \item[5.21] Write at least 5 elements of each of the following cyclic groups.
               \begin{enumerate}
                  \item $25\Z$ under addition.
                  \item $\{(\frac{1}{2})^n : n \in \Z\}$ under multiplication.
                  \item $\{\pi^n : n \in \Z\}$ under multiplication.
               \end{enumerate}

      \textbf{Solution:}

      \begin{enumerate}
         \item $-$250, $-$125, 0, 275, 300, 2200.
         \item 0, .5, .25, .125, .0625.
         \item 1, $\pi^{10}$, $\pi^{22}$, $\pi^{222}$, $\pi^{7772}$.
      \end{enumerate}
   
\end{enumerate}

\noindent      In Exercises 22 through 25, describe all the elements in the
               cyclic subgroup of $GL(2, \R)$ generated by the given
               $2 \times 2$ matrix.

\begin{enumerate}
%%%%%%%%%%%%%%%%%%%%%%%%%%%%%%%%%%Prob5.22%%%%%%%%%%%%%%%%%%%%%%%%%%%%%%%%%%%%%%
   \item[5.22] $\left[
                  \begin{tabular}{@{}r r@{}}
                     0 & $-$1 \\
                     $-1$ & 0
                  \end{tabular}\right]$.

      \textbf{Solution:}

      $$\left\cyc{\left[
                  \begin{tabular}{@{}r r@{}}
                     0 & $-$1 \\
                     $-1$ & 0
                  \end{tabular}\right]\right} = \left\{\left[
                  \begin{tabular}{@{}r r@{}}
                     0 & $-$1 \\
                     $-1$ & 0
                  \end{tabular}\right], \left[
                  \begin{tabular}{@{}r r@{}}
                     1 & 0 \\
                     0 & 1
                  \end{tabular}\right]\right\}.$$
%%%%%%%%%%%%%%%%%%%%%%%%%%%%%%%%%%Prob5.23%%%%%%%%%%%%%%%%%%%%%%%%%%%%%%%%%%%%%%
   \item[5.23] $\left[
                  \begin{tabular}{@{}r r@{}}
                     1 & 1 \\
                     0 & 1
                  \end{tabular}\right]$.

      \textbf{Solution:}

      $$\left\cyc{\left[
                  \begin{tabular}{@{}r r@{}}
                     1 & 1 \\
                     0 & 1
                  \end{tabular}\right]\right} = \left\{\left[
                  \begin{tabular}{@{}r r@{}}
                     1 & $n$ \\
                     0 & 1
                  \end{tabular}\right] : n \in \Z\right\}.$$
%%%%%%%%%%%%%%%%%%%%%%%%%%%%%%%%%%Prob5.24%%%%%%%%%%%%%%%%%%%%%%%%%%%%%%%%%%%%%%
   \item[5.24] $\left[
                  \begin{tabular}{@{}r r@{}}
                     3 & 0 \\
                     0 & 2
                  \end{tabular}\right]$.

      \textbf{Solution:}

      $$\left\cyc{\left[
                  \begin{tabular}{@{}r r@{}}
                     3 & 0 \\
                     0 & 2
                  \end{tabular}\right]\right} = \left\{\left[
                  \begin{tabular}{@{}r r@{}}
                     $3^n$ & 0 \\
                     0 & $2^n$
                  \end{tabular}\right] : n \in \Z\right\}.$$
%%%%%%%%%%%%%%%%%%%%%%%%%%%%%%%%%%Prob5.25%%%%%%%%%%%%%%%%%%%%%%%%%%%%%%%%%%%%%%
   \item[5.25] $\left[
                  \begin{tabular}{@{}r r@{}}
                     0 & $-$2 \\
                     $-2$ & 0
                  \end{tabular}\right]$.

      \textbf{Solution:}

      $$\left\cyc{\left[
                  \begin{tabular}{@{}r r@{}}
                     0 & $-$2 \\
                     $-2$ & 0
                  \end{tabular}\right]\right} = \left\{\left[
                  \begin{tabular}{@{}r r@{}}
                     $2^{2n}$ & 0 \\
                     0 & $2^{2n}$
                  \end{tabular}\right] \mbox{ or } \left[
                  \begin{tabular}{@{}r r@{}}
                     0 & $-2^{2k + 1}$ \\
                     $-2^{2k + 1}$ & 0
                  \end{tabular}\right]: n, k \in \Z\right\}.$$
%%%%%%%%%%%%%%%%%%%%%%%%%%%%%%%%%%Prob5.26%%%%%%%%%%%%%%%%%%%%%%%%%%%%%%%%%%%%%%
   \item[5.26] Which of the following groups are cyclic? For each cyclic group,
               list all the generators of the group.
               \begin{align*}
                  G_1 &= \cyc{\Z, +} \quad G_2  = \cyc{\Q, +} \quad
                  G_3 = \cyc{\Q^+, \cdot} \quad G_4 = \cyc{6\Z, +} \\
                  G_5 &= \{6^n : n \in \Z\} \text{ under multiplication} \\
                  G_6 &= \{a + b\sqrt{2} : a, b \in \Z\} \text{ under addition}
               \end{align*}

      \textbf{Solution:} The cyclic groups ae $G_1$, $G_4$, and $G_5$. The
      generators for $G_1$ are $\pm1$; the generator for $G_4$ is 6, and the
      generator for $G_5$ is 6.
\end{enumerate}

\noindent      In Exercises 27 through 35, find the order of the cyclic subgroup
               of the given group generated by the indicated element.

\begin{enumerate}
%%%%%%%%%%%%%%%%%%%%%%%%%%%%%%%%%%Prob5.27%%%%%%%%%%%%%%%%%%%%%%%%%%%%%%%%%%%%%%
   \item[5.27] The subgroup of $\Z_4$ generated by 3.

      \textbf{Solution:} We have $\cyc{3} = \{3, 2, 1, 0\}$, so that
      $|\cyc{3}| = 4$.
%%%%%%%%%%%%%%%%%%%%%%%%%%%%%%%%%%Prob5.28%%%%%%%%%%%%%%%%%%%%%%%%%%%%%%%%%%%%%%
   \item[5.28] The subgroup of $V$ generated by $c$ (see Table 5.11).

      \textbf{Solution:} We have $\cyc{c} = \{e, c\}$, so that $|\cyc{c}| = 2$.
%%%%%%%%%%%%%%%%%%%%%%%%%%%%%%%%%%Prob5.29%%%%%%%%%%%%%%%%%%%%%%%%%%%%%%%%%%%%%%
   \item[5.29] The subgroup of $U_6$ generated by
               $\cos\frac{2\pi}{3} + i\sin\frac{2\pi}{3}$.

      \textbf{Solution:} Let $\zeta = \cos\frac{\pi}{3} + i\sin\frac{\pi}{3}$.
      Then $\zeta^2 = \cos\frac{2\pi}{3} + i\sin\frac{2\pi}{3}$, so that
      $\cyc{\zeta^2} = \{\zeta^2, \zeta^4, 1\}$, and $|\cyc{\zeta^2}| = 3$.
%%%%%%%%%%%%%%%%%%%%%%%%%%%%%%%%%%Prob5.30%%%%%%%%%%%%%%%%%%%%%%%%%%%%%%%%%%%%%%
   \item[5.30] The subgroup of $U_5$ generated by
               $\cos\frac{4\pi}{5} + i\sin\frac{4\pi}{5}$.

      \textbf{Solution:} Let $\zeta = \cos\frac{2\pi}{5} + i\sin\frac{2\pi}{5}$.
      Then $\zeta^2 = \cos\frac{4\pi}{5} + i\sin\frac{4\pi}{5}$, so that
      $\cyc{\zeta^2} = \{\zeta^2, \zeta^4, \zeta^1, \zeta^3, 1\}$, and
      $|\cyc{\zeta^2}| = 5$.
%%%%%%%%%%%%%%%%%%%%%%%%%%%%%%%%%%Prob5.31%%%%%%%%%%%%%%%%%%%%%%%%%%%%%%%%%%%%%%
   \item[5.31] The subgroup of $U_8$ generated by
               $\cos\frac{3\pi}{2} + i\sin\frac{3\pi}{2}$.

      \textbf{Solution:} Let $\zeta = \cos\frac{\pi}{4} + i\sin\frac{\pi}{4}$.
      Then $\zeta^{6} = \cos\frac{3\pi}{2} + i\sin\frac{3\pi}{2}$, so that
      $\cyc{\zeta^6} = \{\zeta^6, \zeta^4, \zeta^2, 1\}$, and
      $|\cyc{\zeta^6}| = 4$.
%%%%%%%%%%%%%%%%%%%%%%%%%%%%%%%%%%Prob5.32%%%%%%%%%%%%%%%%%%%%%%%%%%%%%%%%%%%%%%
   \item[5.32] The subgroup of $U_8$ generated by
               $\cos\frac{5\pi}{4} + i\sin\frac{5\pi}{4}$.

      \textbf{Solution:} Let $\zeta = \cos\frac{\pi}{4} + i\sin\frac{\pi}{4}$.
      Then $\zeta^{5} = \cos\frac{5\pi}{4} + i\sin\frac{5\pi}{4}$, so that
      $\cyc{\zeta^5} = \{\zeta^5, \zeta^2, \zeta^7, \zeta^4,
       \zeta, \zeta^6, \zeta^3, 1\}$, and $|\cyc{\zeta^5}| = 8$.
%%%%%%%%%%%%%%%%%%%%%%%%%%%%%%%%%%Prob5.33%%%%%%%%%%%%%%%%%%%%%%%%%%%%%%%%%%%%%%
   \item[5.33] The subgroup of the multiplicative group $G$ of invertible
               $4 \times 4$ matrices generated by
               $$H_1 = \left[
                  \begin{tabular}{@{}c c c c@{}}
                     0 & 0 & 1 & 0 \\
                     0 & 0 & 0 & 1 \\
                     1 & 0 & 0 & 0 \\
                     0 & 1 & 0 & 0
                  \end{tabular}\right].$$

      \textbf{Solution:} Since $\cyc{H_1} = \{H_1, I_4\}$, it follows that
      $|\cyc{H_1}|$ is 2.
%%%%%%%%%%%%%%%%%%%%%%%%%%%%%%%%%%Prob5.34%%%%%%%%%%%%%%%%%%%%%%%%%%%%%%%%%%%%%%
   \item[5.34] The subgroup of the multiplicative group $G$ of invertible
               $4 \times 4$ matrices generated by
               $$H_2 = \left[
                  \begin{tabular}{@{}c c c c@{}}
                     0 & 0 & 0 & 1 \\
                     0 & 0 & 1 & 0 \\
                     1 & 0 & 0 & 0 \\
                     0 & 1 & 0 & 0
                  \end{tabular}\right].$$

      \textbf{Solution:} Since
      $$\cyc{H_2} = \left\{H_2, \left[
           \begin{tabular}{@{}c c c c@{}}
              0 & 1 & 0 & 0 \\
              1 & 0 & 0 & 0 \\
              0 & 0 & 0 & 1 \\
              0 & 0 & 1 & 0
           \end{tabular}\right], \left[
           \begin{tabular}{@{}c c c c@{}}
              0 & 0 & 1 & 0 \\
              0 & 0 & 0 & 1 \\
              0 & 1 & 0 & 0 \\
              1 & 0 & 0 & 0
           \end{tabular}\right], I_4\right\},$$
      it follows that $|\cyc{H_2}|$ is 4.   
%%%%%%%%%%%%%%%%%%%%%%%%%%%%%%%%%%Prob5.35%%%%%%%%%%%%%%%%%%%%%%%%%%%%%%%%%%%%%%
   \item[5.35] The subgroup of the multiplicative group $G$ of invertible
               $4 \times 4$ matrices generated by
               $$H_3 = \left[
                  \begin{tabular}{@{}c c c c@{}}
                     0 & 1 & 0 & 0 \\
                     0 & 0 & 0 & 1 \\
                     0 & 0 & 1 & 0 \\
                     1 & 0 & 0 & 0
                  \end{tabular}\right].$$

      \textbf{Solution:} Since
      $$\cyc{H_3} = \left\{H_3, \left[
           \begin{tabular}{@{}c c c c@{}}
              0 & 0 & 0 & 1 \\
              1 & 0 & 0 & 0 \\
              0 & 0 & 1 & 0 \\
              0 & 1 & 0 & 0
           \end{tabular}\right], I_4\right\},$$
      it follows that $|\cyc{H_3}|$ is 3.
%%%%%%%%%%%%%%%%%%%%%%%%%%%%%%%%%%Prob5.36%%%%%%%%%%%%%%%%%%%%%%%%%%%%%%%%%%%%%%
   \item[5.36] \begin{enumerate}
                  \item Complete Table 5.25 to give the group $\Z_6$ of 6 
                        elements.
                  \item Compute the subgroups $\cyc{0}$, $\cyc{1}$, $\cyc{2}$,
                        $\cyc{3}$, $\cyc{4}$, and $\cyc{5}$ of the group $\Z_6$
                        given in part (a).
                  \item Which elements are generators for the group $\Z_6$ of 
                        part (a)?
                  \item Give the subgroup diagram for the part(b) subgroups of
                        $\Z_6$.
               \end{enumerate}

      \textbf{Solution:}

      \begin{enumerate}
         \item $$
                  \begin{tabular}{@{}c  c | c | c | c | c | c | c@{}} 
                     \multicolumn{8}{c}{\textbf{5.25 Table}} \\
                     $\Z_6:$ & $+$ & 0 & 1 & 2 & 3 & 4 & 5 \\ \cline{2-8}
                             &   0 & 0 & 1 & 2 & 3 & 4 & 5 \\ \cline{2-8}
                             &   1 & 1 & 2 & 3 & 4 & 5 & 0 \\ \cline{2-8}
                             &   2 & 2 & 3 & 4 & 5 & 0 & 1 \\ \cline{2-8}
                             &   3 & 3 & 4 & 5 & 0 & 1 & 2 \\ \cline{2-8}
                             &   4 & 4 & 5 & 0 & 1 & 2 & 3 \\ \cline{2-8}
                             &   5 & 5 & 0 & 1 & 2 & 3 & 4
                  \end{tabular}
               $$
         \item \begin{align*}
                  \cyc{0} &= \{0\}, \\
                  \cyc{1} &= \Z_6, \\
                  \cyc{2} &= \{0, 2, 4\}, \\
                  \cyc{3} &= \{0, 3\}, \\
                  \cyc{4} &= \{0, 4, 2\}, \mbox{ and } \\
                  \cyc{5} &= \Z_6.                  
               \end{align*}
         \item 1 and 5 are generators for $\Z_6$.
         \item TODO [Upgrade your tikz skills].
      \end{enumerate}
\end{enumerate}

\noindent      \textbf{Concepts}

\noindent      In Exercises 37 and 38, correct the definition of the italicized
               term without reference to the text, if correction is needed, so
               that it is in a form acceptable for publication.

\begin{enumerate}
%%%%%%%%%%%%%%%%%%%%%%%%%%%%%%%%%%Prob5.37%%%%%%%%%%%%%%%%%%%%%%%%%%%%%%%%%%%%%%
   \item[5.37] A subgroup of a group $G$ is a subset $H$ of $G$ that contains
               the identity element $e$ of $G$ and also contains the inverse of
               each of its elements.

      \textbf{Solution:} We must also add that $H$ must be closed under the
      operation of $G$.
%%%%%%%%%%%%%%%%%%%%%%%%%%%%%%%%%%Prob5.38%%%%%%%%%%%%%%%%%%%%%%%%%%%%%%%%%%%%%%
   \item[5.38] A group $G$ is cyclic if and only if there exists $a \in G$
               such that $G = \{a^n : n \in \Z\}$.

      \textbf{Solution:} Okay.
%%%%%%%%%%%%%%%%%%%%%%%%%%%%%%%%%%Prob5.39%%%%%%%%%%%%%%%%%%%%%%%%%%%%%%%%%%%%%%
   \item[5.39] Mark each of the following true or false.

      \textbf{Solution:}

      \begin{tabularx}{\linewidth}{@{}c c X@{}} 
         T & \textbf{a.} & The associative law holds in every group. \\
         F & \textbf{b.} & There may be a group in which the cancellation law
                           fails. \\
         T & \textbf{c.} & Every group is a subgroup of itself. \\
         F & \textbf{d.} & Every group has exactly two improper subgroups. \\
         F & \textbf{e.} & In every cyclic group, every element is a
                           generator. \\
         F & \textbf{f.} & A cyclic group has a unique generator. \\
         F & \textbf{g.} & Every set of numbers that is a group under addition
                           is also a group under multiplication. \\
         F & \textbf{h.} & A subroup may be defined as a subset of a group. \\
         T & \textbf{i.} & $\Z_4$ is a cyclic group. \\
         F & \textbf{j.} & Every subset of every group is a subgroup under the
                           induced operation. \\
      \end{tabularx}
%%%%%%%%%%%%%%%%%%%%%%%%%%%%%%%%%%Prob5.40%%%%%%%%%%%%%%%%%%%%%%%%%%%%%%%%%%%%%%
   \item[5.40] Show by means of an example that it is possible for the quadratic
               equation $x^2 = e$ to have more than two solutions in some group
               $G$ with identity $e$.

      \textbf{Solution:} In the Klein 4-group (See Table 5.11), the equation
      $x^2 = e$ has four solutions, namely $a$, $b$, $c$, and $e$.
\end{enumerate}

\noindent      \textbf{Theory}

\noindent      In Exercises 41 and 42, let $\phi : G \rightarrow G'$ be an
               isomorphism of a group $\cyc{G, *}$ with a group $\cyc{G', *'}$.
               Write out a proof to convince a skeptic of the intuitively clear
               statement.

\begin{enumerate}
%%%%%%%%%%%%%%%%%%%%%%%%%%%%%%%%%%Prob5.41%%%%%%%%%%%%%%%%%%%%%%%%%%%%%%%%%%%%%%
   \item[5.41] If $H$ is a subgroup of $G$, then
               $\phi[H] = \{\phi(h) : h \in H\}$ is a subgroup of $G'$. That is,
               an isomorphism carries subgroups into subgroups.

      \textbf{Proof.} $\phi[H]$ is nonempty because it contains $\phi(e)$. So
      consider $g_1, g_2 \in \phi[H]$. Then there exist $h_1, h_2 \in H$ such 
      that $g_1 = \phi(h_1)$ and $g_2 = \phi(h_2)$. It suffices to show that 
      $g_1g_2^{-1}$ is also a member of $\phi[H]$. Then
      \begin{align*}
         g_1g_2^{-1} &= \phi(h_1)\phi(h_2)^{-1} \\
                     &= \phi(h_1)\phi(h_2^{-1}) &[\text{ See Exercise 4.28}] \\
                     &= \phi(h_1h_2^{-1}). &[\text{Homomorphism property}]
      \end{align*}

      Since $H$ is a group, it follows that $\phi(h_1h_2^{-1}) \in H$, so that
      $g_1g_2^{-1} \in \phi[H]$. \qed
%%%%%%%%%%%%%%%%%%%%%%%%%%%%%%%%%%Prob5.42%%%%%%%%%%%%%%%%%%%%%%%%%%%%%%%%%%%%%%
   \item[5.42] If $G$ is cyclic, then $G'$ is cyclic.

      \textbf{Proof.} Suppose $G$ is cyclic. Then $G = \cyc{g}$ for some
      $g \in G$. Claim that $G' = \cyc{\phi(g)}$. So let $h' \in G'$. Then there
      exists an integer $n$ such that  $\phi(g^n) = h'$. Using the homomorphism
      property of $\phi$, we have that $\phi(g^n) = \phi(g)^n = h'$, so that 
      $G'$ is cyclic. \qed
%%%%%%%%%%%%%%%%%%%%%%%%%%%%%%%%%%Prob5.43%%%%%%%%%%%%%%%%%%%%%%%%%%%%%%%%%%%%%%
   \item[5.43] Show that if $H$ and $K$ are subgroups of an abelian group $G$,
               then
               $$\{hk : h \in H  \text{ and } k \in K\}$$
               is a subgroup of $G$.

      \textbf{Proof.} Let $H$ and $K$ be subgroups of an abelian group $G$, and
      let
      $$J = \{hk : h \in H  \text{ and } k \in K\}.$$
      We want to show that $J \le G$. It is clear $J$ is nonempty because it
      contains $e$. So let $j_1$ and $j_2$ be members of $J$. To complete the
      proof we must show that $j_1j_2^{-1} \in J$. By the definition of $J$, 
      there exist $h_1, h_2 \in H$ and $k_1, k_2 \in K$ such that $j_1 = h_1k_1$
      and $j_2 = h_2k_2$. It follows that
      \begin{align*}
         j_1j_2^{-1} &= h_1k_1(h_2k_2)^{-1} \\
                    &= h_1k_1k_2^{-1}h_2^{-1} \\
                    &= (h_1h_2^{-1})(k_1k_2^{-1}). &[G\text{ is abelian}]
      \end{align*}

      The preceding calculations show that $j_1j_2^{-1} \in J$, so that $J$ is
      a subgroup of $G$. \qed
%%%%%%%%%%%%%%%%%%%%%%%%%%%%%%%%%%Prob5.44%%%%%%%%%%%%%%%%%%%%%%%%%%%%%%%%%%%%%%
   \item[5.44] Find the flaw in the following argument: ``Condition 2 of Theorem
               5.14 is reduntant, since it can be derived from 1 and 3, for let
               $a \in H$. Then $a^{-1} \in H$ by 3, and by 1, $aa^{-1} = e$ is 
               an element of $H$, proving 2."

      \textbf{Solution:} We cannot say let $a \in H$ if we haven't shown that
      $H$ is nonempty.
%%%%%%%%%%%%%%%%%%%%%%%%%%%%%%%%%%Prob5.45%%%%%%%%%%%%%%%%%%%%%%%%%%%%%%%%%%%%%%
   \item[5.45] Show that a nonempty subset $H$ of a group $G$ is a subgroup of
               $G$ if and only if $ab^{-1} \in H$ for all $a, b \in H$.

      \textbf{Proof.} $(\Leftarrow)$ Let $H$ be a nonempty subset of $G$ such
      $ab^{-1} \in H$ for all $a, b \in H$. To show that $H \le G$, we must
      show that $H$ satisfies all the three conditions in Theorem 5.14. Let
      $a, b \in H$. Then we must have that  $aa^{-1} = e \in H$, so that 
      condition 2 holds. Since $e, b \in H$, it must be the case that
      $eb^{-1} = b^{-1} \in H$, so that condition 3 three holds. Finally since
      $a, b^{-1} \in H$, we must have that $a(b^{-1})^{-1} = ab \in H$, so that
      condition 1 holds. That is $H \le G$.

      $(\Rightarrow)$ Now suppose that $H \le G$. Let $a, b \in H$. Since $H$ is
      a group it immediately follows that $ab^{-1} \in H$. \qed
%%%%%%%%%%%%%%%%%%%%%%%%%%%%%%%%%%Prob5.46%%%%%%%%%%%%%%%%%%%%%%%%%%%%%%%%%%%%%%
   \item[5.46] Prove that a cyclic group with only one generator can have at
               most 2 elements.

      \textbf{Proof.} Let $G$ be a cyclic group. Then it follows that
      $G = \cyc{a}$ for some $a \in G$. We want to show that if $G$ has 1
      generator, then $|G| \le 2$. We shall instead prove the contrapositive of
      this statement; that is, if $|G| \ge 3$, then $G$ has at least 2
      generators. So suppose that $|G| \ge 3$. It suffices to find two nonequal
      generators for $G$. We observe that $a$ cannot be the identity since that 
      would imply that $G$ is trivial. Also, if $a^2 = e$,  then we would have 
      that $G = \{e, a\}$ so that $|G| = 2$, contradicting our assumption. Thus 
      $a^2 \neq e$ so that $a \neq a^{-1}$. Since $\cyc{a} = \cyc{a^{-1}}$, it
      follows that $a$ and $a^{-1}$ are two generators for $G$. \qed
%%%%%%%%%%%%%%%%%%%%%%%%%%%%%%%%%%Prob5.47%%%%%%%%%%%%%%%%%%%%%%%%%%%%%%%%%%%%%%
   \item[5.47] Prove that if $G$ is an abelian group, written multiplicatively,
               with identity element $e$, then all elements $x$ of $G$ 
               satisfying the equation $x^2 = e$ form a subgroup $H$ of $G$.

      \textbf{Proof.} Take $n = 2$ in Exercise 4.48. \qed
%%%%%%%%%%%%%%%%%%%%%%%%%%%%%%%%%%Prob5.48%%%%%%%%%%%%%%%%%%%%%%%%%%%%%%%%%%%%%%
   \item[5.48] Repeat Exercise 47 for the general situation of the set $H$ of 
               all solutions $x$ of the equation $x^n = e$ for a fixed integer
               $n \ge 1$ in an abelian group $G$ with identity $e$.

      \textbf{Proof.} Let $G$ be an abelian group, let $n$ be a positive 
      integer, and let
      $$H = \{x \in G: x^n = e\}.$$
      We want to show that $H \le G$. $H$ is clearly nonempty since $e \in H$.
      So consider $a, b \in H$. It suffices to show that $ab^{-1} \in H$; that
      is, $(ab^{-1})^n = e$. So
      \begin{align*}
         (ab^{-1})^n &= a^n(b^{-1})^n &[\text{Exercise 4.33}] \\
                     &= a^n(b^n)^{-1}  \\
                     &= e(e)^{-1} &[a, b \text{ is in }H] \\
                     &= ee = e.
      \end{align*}

      Hence $H \le G$. \qed
%%%%%%%%%%%%%%%%%%%%%%%%%%%%%%%%%%Prob5.49%%%%%%%%%%%%%%%%%%%%%%%%%%%%%%%%%%%%%%
   \item[5.49] Show that if $a \in G$, where $G$ is a finite group with identity
               $e$, then there exists $n \in \Z^+$ such that $a^n = e$.

      \textbf{Proof.} See Exercise 4.34. \qed
%%%%%%%%%%%%%%%%%%%%%%%%%%%%%%%%%%Prob5.50%%%%%%%%%%%%%%%%%%%%%%%%%%%%%%%%%%%%%%
   \item[5.50] Let a nonempty finite subset $H$ of a group $G$ be closed under
               the binary operation of $G$. Show that $H$ is a subgroup of $G$.

      \textbf{Proof.} Let $H$ be a nonempty finite subset of a group $G$ closed 
      under the binary operation of $G$. We want to show that $H \le G$. Since
      $H$ is nonempty, let $a, b \in H$. By assumption, we have that $ab \in H$.
      By Theorem 5.14, it only remains to show  that $e$ and $a^{-1}$ are also 
      in $H$. By Exercise 5.49 there exists a positive integer $n$ such that
      $a^n = e$. Since $H$ is closed under the binary operation of $G$, we must
      have that $a^n \in H$, so that $e \in H$. Now if $n = 1$ then $a = e$, so
      that $a^{-1} = e \in H$. So suppose that $n \ge 2$. Thus
      $e = a^n = a^{n-1}a$, so that $a^{n-1} = a^{-1}$. Similarly since $H$ is 
      closed under the binary operation of $G$, we must have that
      $a^{n - 1} \in H$, so that $a^{-1} \in H$. That is, $H \le G$. \qed
%%%%%%%%%%%%%%%%%%%%%%%%%%%%%%%%%%Prob5.51%%%%%%%%%%%%%%%%%%%%%%%%%%%%%%%%%%%%%%
   \item[5.51] Let $G$ be a group and let $a$ be one fixed element of $G$. Show
               that
               $$H_a = \{x \in G : xa = ax\}$$
               is a subgroup of $G$.

      \textbf{Proof.} Let $G$ be a group and let $a \in G$ be fixed. Take
      $S = \{a\}$ in Exercise 5.52(a). \qed
%%%%%%%%%%%%%%%%%%%%%%%%%%%%%%%%%%Prob5.52%%%%%%%%%%%%%%%%%%%%%%%%%%%%%%%%%%%%%%
   \item[5.52] Generalizing Exercise 51, let $S$ be any subset of a group $G$.
               \begin{enumerate}
                  \item Show that
                        $H_S = \{x \in G : xs = sx \text{ for all } s \in S\}$
                        is a subgroup of $G$.
                  \item In reference to part(a), the subgroup $H_G$ is the
                        \textbf{center of} $G$. Show that $H_G$ is an abelian
                        group.
               \end{enumerate}

      \textbf{Proof.}

      \begin{enumerate}
         \item Let $S$ be any subset of a group $G$.
               $$H_S = \{x \in G : xs = sx \text{ for all } s \in S\}.$$
               We want to show that $H_S \le G$. Let $s \in S$. Since
               $es = se = s$, it follows that $e \in H_S$, so that $H_S$ is 
               nonempty. So let $r, t \in H_S$. So we have that $rs = sr$, so 
               that $s = r^{-1}sr$; that is, $sr^{-1} = r^{-1}s$, so
               that $r^{-1} \in H_S$. It remains to show that $rt \in H_S$. I.e.
               we want to show that $(rt)s = s(rt)$; so,
               \begin{align*}
                  (rt)s &= r(ts) \\
                        &= r(st) &[t \in H_S] \\
                        &= (rs)t \\
                        &= (sr)t &[r \in H_S] \\
                        &= s(rt).
               \end{align*}

               Since $e, r^{-1}, rt \in H_S$, it follows by Theorem 5.14 that 
               $H_S$ is a subgroup of $G$. \qed
         \item By (a), it follows that $H_G \le G$. Consider $a, b \in H_G$.
               Since $a \in H_G$ and since $b \in G$, it follows that $ab = ba$
               so that $H_G$ is abelian. \qed
      \end{enumerate}
%%%%%%%%%%%%%%%%%%%%%%%%%%%%%%%%%%Prob5.53%%%%%%%%%%%%%%%%%%%%%%%%%%%%%%%%%%%%%%
   \item[5.53] Let $H$ be a subgroup of a group $G$. For $a, b \in G$, let
               $a \sim b$ if and only if $ab^{-1} \in H$. Show that $\sim$ is an
               equivalence relation on $G$. Then

      \textbf{Proof.} To show that $\sim$ is an equivalence relation on $G$, we
      must show that $\sim$ is reflexive, symmetric, and transitive. So consider
      $a, b, c \in G$. 
      \begin{enumerate}
         \item[Reflexivity:]  We have that $a \sim a$ since $aa^{-1} = e \in H$.
         \item[Symmetry:]     Suppose that $a \sim b$. Then it follows that
                              $ab^{-1} \in H$, and since $H$ is closed under
                              taking inverses, we must have that
                              $(ab^{-1})^{-1} = ba^{-1} \in H$. Thus $b \sim a$.
         \item[Transitivity:] Finally suppose that $a \sim b$ and $b \sim c$.
                              That is, $ab^{-1}, bc^{-1} \in H$. By closure of
                              $H$ under the operation of $G$, we have that
                              $ab^{-1}bc^{-1} = ac^{-1}\in H$. That is
                              $a \sim c$.
      \end{enumerate}

      The above shows that $\sim$ is an equivalence relation on $G$.
%%%%%%%%%%%%%%%%%%%%%%%%%%%%%%%%%%Prob5.54%%%%%%%%%%%%%%%%%%%%%%%%%%%%%%%%%%%%%%
   \item[5.54] For the sets $H$ and $K$, we define the \textbf{intersection}
               $H \cap K$ by
               $$H \cap K = \{x : x \in H \text{ and } x \in K\}.$$
               Show that if $H \le G$ and $K \le G$, then $H \cap K \le G$.

      \textbf{Proof.} Let $G$ be a group. Suppose that $H \le G$ and $K \le G$.
      We want to show that $H \cap K \le G$. We shall proceed by the Subgroup
      Criterion. Since $H$ and $K$ are subgroups of $G$, they both contain $e$
      so that $e \in (H \cap K)$. Let $a$ and $b$ be members of $H \cap K$. Then
      $a$ and $b$ are also members of both $H$ and $K$, so that $ab^{-1}$ is
      in both $H$ and $K$. Thus $ab^{-1} \in (H \cap K)$. Hence $H \cap K$ is a
      subgroup of $G$. \qed
%%%%%%%%%%%%%%%%%%%%%%%%%%%%%%%%%%Prob5.55%%%%%%%%%%%%%%%%%%%%%%%%%%%%%%%%%%%%%%
   \item[5.55] Prove that every cyclic group is abelian.

      \textbf{Proof.} See Theorem 6.1. \qed
%%%%%%%%%%%%%%%%%%%%%%%%%%%%%%%%%%Prob5.56%%%%%%%%%%%%%%%%%%%%%%%%%%%%%%%%%%%%%%
   \item[5.56] Let $G$ be a group and let $G_n = \{g^n : g \in G\}$. Under what
               hypothesis about $G$ can we show that $G_n$ is a subgroup of $G$?

      \textbf{Solution:} We want to first show that $G_n$ contains the identity 
      of $G$ and that $G_n$ is closed under taking inverses. So let $e$ be the 
      identity of $G$. Then since $e^n = e$, it follows that $e \in G_n$. Now 
      consider $h \in G_n$. Then $h = {g_1}^n$ for some $g_1 \in G$. Note that
      $({g_1}^{-1})^n$ is also in $G_n$. Since
      ${g_1}^n({g_1}^{-1})^n = {g_1}^n({g_1}^n)^{-1} = e$, we have that
      $h^{-1} = ({g_1}^n)^{-1} \in G_n$. Thus we have shown that
      $G_n$ satisfies conditions 2 and 3 of Theorem 5.14. So in order for $G_n$ 
      to be a subgroup of $G$, it must closed under the binary operation of $G$.
      That is, if $f = {g_2}^n \in G_n$, where $g_2 \in G$, we want the
      statement ${g_1}^n{g_2}^n \in G_n$ to be true. This will be so if $G$
      is abelian, so that ${g_1}^n{g_2}^n = {g_1g_2}^n$ will be a member of
      $G_n$.
%%%%%%%%%%%%%%%%%%%%%%%%%%%%%%%%%%Prob5.57%%%%%%%%%%%%%%%%%%%%%%%%%%%%%%%%%%%%%%
   \item[5.57] Show that a group with no proper nontrivial subgroups is
               cyclic?

      \textbf{Proof.} Let $G$ be a group with no proper nontrivial subgroup, and
      let $e$ be the identity element of $G$. If $G$ is the trivial group, then 
      we are done. So assume that $|G| > 1$. Then there exists a nonidenity 
      element $a \in G$. Let $H$ be the subgroup of $G$ generated by $a$. 
      Observe that $H$ is nontrivial because it contains at least two unequal 
      members $a$ and $e$. If $H < G$, then we would have a contradiction since 
      we assumed that $G$ contains no proper nontrivial subgroup. Thus we must 
      have that $H = G$; that is, $G$ is cyclic. \qed
\end{enumerate}
