Let $G$ and $H$ be groups.

\begin{enumerate}
%%%%%%%%%%%%%%%%%%%%%%%%%%%%%%%%%%%%%3.1.1%%%%%%%%%%%%%%%%%%%%%%%%%%%%%%%%%%%%%%
   \item[3.1.1]   Let $\varphi : G \rightarrow H$ be a homomorphism and let $E$
                  be a subgroup of $H$. Prove that $\varphi^{-1}(E) \le G$
                  (i.e., the preimage or pullback of a subgroup under a
                  homomorphism is a subgroup). If $E \trianglelefteq H$ prove
                  that $\varphi^{-1}(E) \trianglelefteq G$. Deduce that
                  $\text{ker}\varphi \trianglelefteq G$.
%%%%%%%%%%%%%%%%%%%%%%%%%%%%%%%%%%%%%3.1.2%%%%%%%%%%%%%%%%%%%%%%%%%%%%%%%%%%%%%%
   \item[3.1.2]   Let $\varphi : G \rightarrow H$ be a homomorphism of groups
                  with kernel $K$ and let $a, b \in \varphi(G)$. Let
                  $X \in G/K$ be the fiber above $a$ and let $Y$ be the fiber
                  above $b$, i.e., $X = \varphi^{-1}(a)$, $Y = \varphi^{-1}(b)$.
                  Fix an element $u$ of $X$ (so $\varphi(u) = a$). Prove that if
                  $XY = Z$ in the quotient group $G/K$ and $w$ is any member of
                  $Z$, then there is some $v \in Y$ such that $uv = w$. [Show
                  that $u^{-1}w \in Y$.]
%%%%%%%%%%%%%%%%%%%%%%%%%%%%%%%%%%%%%3.1.3%%%%%%%%%%%%%%%%%%%%%%%%%%%%%%%%%%%%%%
   \item[3.1.3]   Let $A$ be an abelian group and let $B$ be a subgroup of $A$.
                  Prove that $A/B$ is abelian. Give an example of a non-abelian
                  group $G$ containing a proper normal subgroup $N$ such that
                  $G/N$ is abelian.
%%%%%%%%%%%%%%%%%%%%%%%%%%%%%%%%%%%%%3.1.4%%%%%%%%%%%%%%%%%%%%%%%%%%%%%%%%%%%%%%
   \item[3.1.4]   Prove that in the quotient group $G/N$,
                  $(gN)^\alpha = g^\alpha N$ for all $\alpha \in \Z$.
%%%%%%%%%%%%%%%%%%%%%%%%%%%%%%%%%%%%%3.1.5%%%%%%%%%%%%%%%%%%%%%%%%%%%%%%%%%%%%%%
   \item[3.1.5]   Use the preceding exercise to prove that the order of the
                  element $gN$ in $G/N$ is $n$, where $n$ is the smallest
                  positive integer such that $g^n \in N$ (and $gN$ has infinite
                  order if no such positive integer exists). Give an example to
                  show that the order of $gN$ in $G/N$ may be strictly smaller
                  than the order of $g$ in $G$.
%%%%%%%%%%%%%%%%%%%%%%%%%%%%%%%%%%%%%3.1.6%%%%%%%%%%%%%%%%%%%%%%%%%%%%%%%%%%%%%%
   \item[3.1.6]   Define $\varphi : \R^\times \rightarrow \{\pm1\}$ by letting
                  $\varphi(x)$ be $x$ divided by the absolute value of $x$.
                  Describe the fibers of $\varphi$ and prove that $\varphi$ is a
                  homomorphism.
%%%%%%%%%%%%%%%%%%%%%%%%%%%%%%%%%%%%%3.1.7%%%%%%%%%%%%%%%%%%%%%%%%%%%%%%%%%%%%%%
   \item[3.1.7]   Define $\pi : \R^2 \rightarrow \R$ by $\pi((x, y)) = x + y$.
                  Prove that $\pi$ is a surjective homomorphism and descrive the
                  kernel and fibers of $\pi$ geometrically.
%%%%%%%%%%%%%%%%%%%%%%%%%%%%%%%%%%%%%3.1.8%%%%%%%%%%%%%%%%%%%%%%%%%%%%%%%%%%%%%%
   \item[3.1.8]   Let $\varphi : \R^\times \rightarrow \R^\times$ be the map
                  sending $x$ to the absolute value of $x$. Prove that $\varphi$
                  is a homomorphism and find the image of $\varphi$. Describe
                  the kernel and the fibers of $\varphi$.
%%%%%%%%%%%%%%%%%%%%%%%%%%%%%%%%%%%%%3.1.9%%%%%%%%%%%%%%%%%%%%%%%%%%%%%%%%%%%%%%
   \item[3.1.9]   Define $\varphi : \C^\times \rightarrow \R^\times$ by
                  $\varphi(a + bi) = a^2 + b^2$. Prove that $\varphi$ is a
                  homomorphism and find the image of $\varphi$. Describe the
                  kernel and the fibers of $\varphi$ geometrically (as subsets
                  of the plane).
%%%%%%%%%%%%%%%%%%%%%%%%%%%%%%%%%%%%%3.1.10%%%%%%%%%%%%%%%%%%%%%%%%%%%%%%%%%%%%%
   \item[3.1.10]  Let $\varphi : \Z/8\Z \rightarrow \Z/4\Z$ by
                  $\varphi(\overline{a}) = \overline{a}$. Show that this is a
                  well defined, surjective homomorphism and describe its fibers
                  and kernel explicitly (showing that $\varphi$ is well defined
                  involves the fact that $\overline{a}$ has a different meaning
                  in the domain and range of $\varphi$).
%%%%%%%%%%%%%%%%%%%%%%%%%%%%%%%%%%%%%3.1.11%%%%%%%%%%%%%%%%%%%%%%%%%%%%%%%%%%%%%
   \item[3.1.11]  Let $F$ be a field and let
                  $G = \left\{\left(\begin{tabular}{@{}cc@{}}
                     $a$ & $b$ \\
                     0   & $c$
                  \end{tabular}\right) : a, b, c \in F, ac \neq 0\right\} \le
                  GL_2(F)$.
                  \begin{enumerate}
                     \item Prove that the map $\varphi :
                           \left(\begin{tabular}{@{}cc@{}}
                              $a$ & $b$ \\
                              0   & $c$
                           \end{tabular}\right)\mapsto a$ is a surjective
                           homomorphism from $G$ onto $F^\times$. Describe the
                           fibers and kernel of $\varphi$.
                     \item Prove that the map $\psi :
                           \left(\begin{tabular}{@{}cc@{}}
                              $a$ & $b$ \\
                              0   & $c$
                           \end{tabular}\right)\mapsto (a, c)$ is a surjective
                           homomorphism from $G$ onto $F^\times\times F^\times$.
                           Describe the fibers and kernel of $\psi$.
                     \item Let $H : \left\{\left(\begin{tabular}{@{}cc@{}}
                              1 & $b$ \\
                              0 & 1
                           \end{tabular}\right) : b \in F\right\}$. Prove that
                           $H$ is isomorphic to the additive group $F$.
                  \end{enumerate}
%%%%%%%%%%%%%%%%%%%%%%%%%%%%%%%%%%%%%3.1.12%%%%%%%%%%%%%%%%%%%%%%%%%%%%%%%%%%%%%
   \item[3.1.12]  Let $G$ be the additive group of real numbers, let $H$ be the
                  multiplicative group of complex numbers of absolute value 1
                  (the unit circle $S^1$ in the complex plane) and let
                  $\varphi : G \rightarrow H$ be the homomorphism
                  $\varphi : r \mapsto e^{2\pi ir}$. Draw the points on a real
                  line which lie in the kernel of $\varphi$. Describe similarly
                  the elements in the fibers of $\varphi$ above the points $-1$,
                  $i$, and $e^{4\pi i/3}$ of $H$. (Figure 1 of the text for this
                  homomorphism $\varphi$ is usually depicted using the following
                  diagram.)
%%%%%%%%%%%%%%%%%%%%%%%%%%%%%%%%%%%%%3.1.13%%%%%%%%%%%%%%%%%%%%%%%%%%%%%%%%%%%%%
   \item[3.1.13]  Repeat the preceding exercise with the map $\varphi$ replaced
                  by the map $\varphi : r \mapsto e^{4\pi ir}$.
%%%%%%%%%%%%%%%%%%%%%%%%%%%%%%%%%%%%%3.1.14%%%%%%%%%%%%%%%%%%%%%%%%%%%%%%%%%%%%%
   \item[3.1.14]  Consider the additive quotient group $\Q/\Z$.
                  \begin{enumerate}
                     \item Show that every coset of $\Z$ in $\Q$ contains
                           exactly one representative $q \in \Q$ in the range
                           $0 \le q < 1$.
                     \item Show that every element of $\Q/\Z$ has finite order
                           but that there are elements of arbitrarily large
                           order,
                     \item Show that $\Q/\Z$ is the torsion subgroup of $\R/\Z$.
                     \item Prove that $\Q/\Z$ is isomorphic to the
                           multiplicative group of roots of unity in
                           $\C^\times$.
                  \end{enumerate}
%%%%%%%%%%%%%%%%%%%%%%%%%%%%%%%%%%%%%3.1.15%%%%%%%%%%%%%%%%%%%%%%%%%%%%%%%%%%%%%
   \item[3.1.15]  Prove that a quotient of a divisible abelian group by any
                  proper subgroup is also divisible. Deduce that $\Q/\Z$ is
                  divisible.
%%%%%%%%%%%%%%%%%%%%%%%%%%%%%%%%%%%%%3.1.16%%%%%%%%%%%%%%%%%%%%%%%%%%%%%%%%%%%%%
   \item[3.1.16]  Let $G$ be a group, let $N$ be a normal subgroup of $G$, and
                  let $\overline{G} = G/N$. Prove that if $G = \cyc{x, y}$ then
                  $\overline{G} = \cyc{\overline{x}, \overline{y}}$. Prove more 
                  generally that if $G =\cyc{S}$ for any subset $S$ of $G$, then
                  $\overline{G} = \cyc{\overline{S}}$.
%%%%%%%%%%%%%%%%%%%%%%%%%%%%%%%%%%%%%3.1.17%%%%%%%%%%%%%%%%%%%%%%%%%%%%%%%%%%%%%
   \item[3.1.17]  Let $G$ be the dihedral group of order 16 (whose lattice
                  appears in Section 2.5):
                  $$G = \cyc{r, s : r^8 = s^2 = 1, rs = sr^{-1}}$$
                  and let $\overline{G} = G/\cyc{r^4}$ be the quotient of $G$ by
                  the subgroup generated by $r^4$ (this subgroup is the center
                  of $G$, hence is normal).
                  \begin{enumerate}
                     \item Show that the order of $\overline{G}$ is 8.
                     \item Exhibit each element of $\overline{G}$ in the form
                           $\overline{s}^a\overline{r}^b$, for some integers $a$
                           and $b$.
                     \item Find the order of each of the elements of
                           $\overline{G}$ exhibited in (b).
                     \item Write each of the following elements of
                           $\overline{G}$ in the form
                           $\overline{s}^a\overline{r}^b$, for some integers $a$
                           and $b$ as in (b): \quad $\overline{rs}$, \quad
                           $\overline{sr^{-2}s}$, \quad
                           $\overline{s^{-1}r^{-1}sr}$.
                     \item Prove that $\overline{H} = \cyc{\overline{s},
                           \overline{r}^2}$ is a normal subgroup of
                           $\overline{G}$ and $\overline{H}$ is isomorphic to
                           the Klein 4-group. Describe the isomorphism type of
                           the complete preimage of $\overline{H}$ in $G$.
                     \item Find the center of $\overline{G}$ and describe the
                           isomorphism type of $\overline{G}/Z(\overline{G})$.
                  \end{enumerate}
%%%%%%%%%%%%%%%%%%%%%%%%%%%%%%%%%%%%%3.1.18%%%%%%%%%%%%%%%%%%%%%%%%%%%%%%%%%%%%%
   \item[3.1.18]  Let $G$ be the quasidihedral group of order 16 (whose lattice
                  was computed in Exercise 2.5.11):
                  $$G = \cyc{\sigma, \tau : \sigma^8 = \tau^2 = 1,
                             \sigma\tau = \tau\sigma^3}$$
                  and let $\overline{G} = G/\cyc{\sigma^4}$ be the quotient of
                  $G$ by the subgroup generated by $\sigma^4$ (this subgroup is 
                  the center of $G$, hence is normal).
                  \begin{enumerate}
                     \item Show that the order of $\overline{G}$ is 8.
                     \item Exhibit each element of $\overline{G}$ in the form
                           $\overline{\tau}^a\overline{\sigma}^b$, for some 
                           integers $a$ and $b$.
                     \item Find the order of each of the elements of
                           $\overline{G}$ exhibited in (b).
                     \item Write each of the following elements of
                           $\overline{G}$ in the form
                           $\overline{\tau}^a\overline{\sigma}^b$, for some 
                           integers $a$ and $b$ as in (b): \quad
                           $\overline{\sigma\tau}$, \quad
                           $\overline{\tau\sigma^{-2}\tau}$, \quad
                           $\overline{\tau^{-1}\sigma^{-1}\tau\sigma}$.
                     \item Prove that $\overline{G} \cong D_8$.
                  \end{enumerate}
%%%%%%%%%%%%%%%%%%%%%%%%%%%%%%%%%%%%%3.1.19%%%%%%%%%%%%%%%%%%%%%%%%%%%%%%%%%%%%%
   \item[3.1.19]  Let $G$ be the modular group of order 16 (whose lattice was
                  computed in Exercise 2.5.14):
                  $$G = \cyc{u, v : u^2 = v^8 = 1, vu = uv^5}$$
                  and let $\overline{G} = G/\cyc{v^4}$ be the quotient of $G$ by
                  the subgroup generated by $v^4$ (this subgroup is contained in
                  the center of $G$, hence is normal).
                  \begin{enumerate}
                     \item Show that the order of $\overline{G}$ is 8.
                     \item Exhibit each element of $\overline{G}$ in the form
                           $\overline{u}^a\overline{v}^b$, for some integers $a$
                           and $b$.
                     \item Find the order of each of the elements of
                           $\overline{G}$ exhibited in (b).
                     \item Write each of the following elements of
                           $\overline{G}$ in the form
                           $\overline{u}^a\overline{v}^b$, for some integers $a$
                           and $b$ as in (b): \quad $\overline{vu}$, \quad
                           $\overline{uv^{-2}u}$, \quad
                           $\overline{u^{-1}v^{-1}uv}$.
                     \item Prove that $\overline{G}$ is abelian and is
                           isomorphic to $Z_2 \times Z_4$.
                  \end{enumerate}
%%%%%%%%%%%%%%%%%%%%%%%%%%%%%%%%%%%%%3.1.20%%%%%%%%%%%%%%%%%%%%%%%%%%%%%%%%%%%%%
   \item[3.1.20]  Let $G = \Z/24\Z$ and let
                  $\widetilde{G} = G/\cyc{\overline{12}}$,
                  where for each integer $a$ we simplify notation by writing
                  $\widetilde{\overline{a}}$ as $\widetilde{a}$.
                  \begin{enumerate}
                     \item Show that $\widetilde{G} = \{\widetilde{0},
                           \widetilde{1}, \ldots, \widetilde{11}\}$.
                     \item Find the order of each element of $\widetilde{G}$.
                     \item Prove that $\cyc{G} \cong \Z/12\Z$. (Thus
                           $(\Z/24\Z)/(12\Z/24\Z) \cong \Z/12\Z$, just as if we
                           inverted and called the $24\Z\text{'s}$.)
                  \end{enumerate}
%%%%%%%%%%%%%%%%%%%%%%%%%%%%%%%%%%%%%3.1.21%%%%%%%%%%%%%%%%%%%%%%%%%%%%%%%%%%%%%
   \item[3.1.21]  Let $G = Z_4 \times Z_4$ be given in terms of the following
                  generators and relations:
                  $$G = \cyc{x, y : x^4 = y^4 = 1, xy = yx}$$.
                  Let $\overline{G} = G/\cyc{x^2y^2}$ (note that every subgroup
                  of the abelian group $G$ is normal).
                  \begin{enumerate}
                     \item Show that the order of $\overline{G}$ is 8.
                     \item Exhibit each element of $\overline{G}$ in the form
                           $\overline{x}^a\overline{y}^b$, for some integers $a$
                           and $b$.
                     \item Find the order of each of the elements of
                           $\overline{G}$ exhibited in (b).
                     \item Prove that $\overline{G} \cong Z_4 \times Z_2$.
                  \end{enumerate}
%%%%%%%%%%%%%%%%%%%%%%%%%%%%%%%%%%%%%3.1.22%%%%%%%%%%%%%%%%%%%%%%%%%%%%%%%%%%%%%
   \item[3.1.22]  \begin{enumerate}
                     \item Prove that if $H$ and $K$ are normal subgroups of a
                           group $G$ then their intersection $H \cap K$ is also
                           a normal subgroup of $G$.
                     \item Prove that the intersection of an arbitrary nonempty
                           collection of normal subgroups of a group is a normal
                           subgroup (do not assume the collection is countable).
                  \end{enumerate}
%%%%%%%%%%%%%%%%%%%%%%%%%%%%%%%%%%%%%3.1.23%%%%%%%%%%%%%%%%%%%%%%%%%%%%%%%%%%%%%
   \item[3.1.23]  Prove that the join of any nonempty collection of normal
                  subgroups of a group is a normal subgroup.
%%%%%%%%%%%%%%%%%%%%%%%%%%%%%%%%%%%%%3.1.24%%%%%%%%%%%%%%%%%%%%%%%%%%%%%%%%%%%%%
   \item[3.1.24]  Prove that if $N \trianglelefteq G$ and $H$ is any subgroup of
                  $G$ then $N \cap H \trianglelefteq H$.
%%%%%%%%%%%%%%%%%%%%%%%%%%%%%%%%%%%%%3.1.25%%%%%%%%%%%%%%%%%%%%%%%%%%%%%%%%%%%%%
   \item[3.1.25]  \begin{enumerate}
                     \item Prove that a subgroup $N$ of $G$ is normal if and
                           only if $gNg^{-1} \subseteq N$ for \textit{all}
                           $g \in G$.
                     \item Let $G = GL_2(\Q)$, let $N$ be the subgroup of upper
                           triangular matrices with integer entries and 1's on
                           the diagonal, and let $g$ be the diagonal matrix with
                           entries 2,1. Show that $gNg^{-1} \subseteq N$ but $g$
                           does $\textit{not}$ normalize $N$.
                  \end{enumerate}
%%%%%%%%%%%%%%%%%%%%%%%%%%%%%%%%%%%%%3.1.26%%%%%%%%%%%%%%%%%%%%%%%%%%%%%%%%%%%%%
   \item[3.1.26]  Let $a, b \in G$.
                  \begin{enumerate}
                     \item Prove that the conjugate of the product of $a$ and
                           $b$ is the product of the conjugate of $a$ and the
                           conjugate of $b$. Prove that the order of $a$ and the
                           order of any conjugate of $a$ are the same.
                     \item Prove that the conjugate of $a^{-1}$ is the inverse
                           of the conjugate of $a$.
                     \item Let $N = \cyc{S}$ for some subset $S$ of $G$. Prove
                           that $N \trianglelefteq G$ if $gSg^{-1} \subseteq N$
                           for all $g \in G$.
                     \item Deduce that if $N$ is the cyclic group $\cyc{x}$,
                           then $N$ is normal in $G$ if and only if for each
                           $g \in G$, $gxg^{-1} = x^k$ for some $k \in \Z$.
                     \item Let $n$ be a positive integer. Prove that the
                           subgroup $N$ of $G$ generated by all the elements of
                           $G$ of order $n$ is a normal subgroup of $G$.
                  \end{enumerate}
%%%%%%%%%%%%%%%%%%%%%%%%%%%%%%%%%%%%%3.1.27%%%%%%%%%%%%%%%%%%%%%%%%%%%%%%%%%%%%%
   \item[3.1.27]  Let $N$ be a \textit{finite} subgroup of a group $G$. Show
                  that $gNg^{-1} \subseteq N$ if and only if $gNg^{-1} = N$.
                  Deduce $N_G(N) = \{g \in G : gNg^{-1} \subseteq N\}$.
%%%%%%%%%%%%%%%%%%%%%%%%%%%%%%%%%%%%%3.1.28%%%%%%%%%%%%%%%%%%%%%%%%%%%%%%%%%%%%%
   \item[3.1.28]  Let $N$ be a \textit{finite} subgroup of a group $G$ and
                  assume $N = \cyc{S}$ for some subset $S$ of $G$. Prove that an
                  element $g \in G$ normalizes $N$ if and only if
                  $gSg^{-1} \subseteq N$.
%%%%%%%%%%%%%%%%%%%%%%%%%%%%%%%%%%%%%3.1.29%%%%%%%%%%%%%%%%%%%%%%%%%%%%%%%%%%%%%
   \item[3.1.29]  Let $N$ be a \textit{finite} subgroup of a group $G$ and
                  suppose $G = \cyc{T}$ and $N = \cyc{S}$ for some subsets $S$
                  and $T$ of $G$. Prove that $N$ is normal in $G$ if and only if
                  $tSt^{-1} \subseteq N$ for all $t \in T$.
%%%%%%%%%%%%%%%%%%%%%%%%%%%%%%%%%%%%%3.1.30%%%%%%%%%%%%%%%%%%%%%%%%%%%%%%%%%%%%%
   \item[3.1.30]  Let $N \le G$ and let $g \in G$. Prove that $gN = Ng$ if and
                  only if $g \in N_G(N)$.
%%%%%%%%%%%%%%%%%%%%%%%%%%%%%%%%%%%%%3.1.31%%%%%%%%%%%%%%%%%%%%%%%%%%%%%%%%%%%%%
   \item[3.1.31]  Prove that if $H \le G$ and $N$ is a normal subgroup of $H$
                  then $H \le N_G(H)$. Deduce that $N_G(N)$ is the largest
                  subgroup of $G$ in which $N$ is normal (i.e., is the join of
                  all subgroups $H$ for which $N \trianglelefteq H$).
%%%%%%%%%%%%%%%%%%%%%%%%%%%%%%%%%%%%%3.1.32%%%%%%%%%%%%%%%%%%%%%%%%%%%%%%%%%%%%%
   \item[3.1.32]  Prove that every subgroup of $Q_8$ is normal. For each
                  subgroup find the isomorphism type of its corresponding
                  quotient. [You may use the lattice of subgroups for $Q_8$ in
                  Section 2.5]
%%%%%%%%%%%%%%%%%%%%%%%%%%%%%%%%%%%%%3.1.33%%%%%%%%%%%%%%%%%%%%%%%%%%%%%%%%%%%%%
   \item[3.1.33]  Find all normal subgroups of $D_8$ and for each find the
                  isomorphism type of its corresponding quotient. [You may use
                  the lattice of subgroups for $D_8$ in Section 2.5.]
%%%%%%%%%%%%%%%%%%%%%%%%%%%%%%%%%%%%%3.1.34%%%%%%%%%%%%%%%%%%%%%%%%%%%%%%%%%%%%%
   \item[3.1.34]  Let $D_{2n} = \cyc{r, s : r^n = s^2 = 1, rs = sr^{-1}}$ be the
                  usual presentation of the dihedral group of order $2n$ and let
                  $k$ be a positive integer dividing $n$.
                  \begin{enumerate}
                     \item Prove that $\cyc{r^k}$ is a normal subgroup of
                           $D_{2n}$.
                     \item Prove that $D_{2n}/\cyc{r^k} \cong D_{2k}$.
                  \end{enumerate}
%%%%%%%%%%%%%%%%%%%%%%%%%%%%%%%%%%%%%3.1.35%%%%%%%%%%%%%%%%%%%%%%%%%%%%%%%%%%%%%
   \item[3.1.35]  Prove that $SL_n(F) \trianglelefteq GL_n(F)$ and describe the
                  isomorphism type of the quotient group.
%%%%%%%%%%%%%%%%%%%%%%%%%%%%%%%%%%%%%3.1.36%%%%%%%%%%%%%%%%%%%%%%%%%%%%%%%%%%%%%
   \item[3.1.36]  Prove that if $G/Z(G)$ is cyclic then $G$ is abelian. [If
                  $G/Z(G)$ is cyclic with generator $xZ(G)$, show that every
                  element of $G$ can be written in the form $x^az$ for some
                  integer $a \in \Z$ and some element $z \in Z(G)$.]
%%%%%%%%%%%%%%%%%%%%%%%%%%%%%%%%%%%%%3.1.37%%%%%%%%%%%%%%%%%%%%%%%%%%%%%%%%%%%%%
   \item[3.1.37]  Let $A$ and $B$ be groups. Show that $\{(a, 1) : a \in A\}$ is
                  a normal subgroup of $A \times B$ and the quotient of
                  $A \times B$ by this subgroup is isomorphic to $B$.
%%%%%%%%%%%%%%%%%%%%%%%%%%%%%%%%%%%%%3.1.38%%%%%%%%%%%%%%%%%%%%%%%%%%%%%%%%%%%%%
   \item[3.1.38]  Let $A$ be an abelian group and let $D$ be the (diagonal)
                  subgroup $\{(a, a) : a \in A\}$ of $A \times A$. Prove that
                  $D$ is a normal subgroup of $A \times A$ and
                  $(A \times A)/D \cong A$.
%%%%%%%%%%%%%%%%%%%%%%%%%%%%%%%%%%%%%3.1.39%%%%%%%%%%%%%%%%%%%%%%%%%%%%%%%%%%%%%
   \item[3.1.39]  Suppose $A$ is the non-abelian group $S_3$ and $D$ is the
                  diagonal subgroup $\{(a, a) : a \in A\}$ of $A \times A$.
                  Prove that $D$ is not normal in $A \times A$.
%%%%%%%%%%%%%%%%%%%%%%%%%%%%%%%%%%%%%3.1.40%%%%%%%%%%%%%%%%%%%%%%%%%%%%%%%%%%%%%
   \item[3.1.40]  Let $G$ be a group, let $N$ be a normal subgroup of $G$ and
                  let $\overline{G} = G/N$. Prove that $\overline{x}$ and
                  $\overline{y}$ commute in $\overline{G}$ if and only if
                  $x^{-1}y^{-1}xy \in N$. (The element $x^{-1}y^{-1}xy$ is
                  called the \textit{commutator} of $x$ and $y$ and is denoted
                  by $[x, y]$.)
%%%%%%%%%%%%%%%%%%%%%%%%%%%%%%%%%%%%%3.1.41%%%%%%%%%%%%%%%%%%%%%%%%%%%%%%%%%%%%%
   \item[3.1.41]  Let $G$ be a group. Prove that
                  $N = \cyc{x^{-1}y^{-1}xy : x, y \in G}$ is a normal sugroup of
                  $G$ and $G/N$ is abelian ($N$ is called the
                  \textit{commutator subgroup} of $G$).
%%%%%%%%%%%%%%%%%%%%%%%%%%%%%%%%%%%%%3.1.42%%%%%%%%%%%%%%%%%%%%%%%%%%%%%%%%%%%%%
   \item[3.1.42]  Assume both $H$ and $K$ are normal subgroups of $G$ with
                  $H \cap K = 1$. Prove that $xy = yx$ for all $x \in H$ and
                  $y \in K$. [Show that $x^{-1}y^{-1}xy \in H \cap K$.]
%%%%%%%%%%%%%%%%%%%%%%%%%%%%%%%%%%%%%3.1.43%%%%%%%%%%%%%%%%%%%%%%%%%%%%%%%%%%%%%
   \item[3.1.43]  Assume that $\mathcal{P} = \{A_i : i \in I\}$ is any partition
                  of $G$ with the property that $\mathcal{P}$ is a group under
                  the ``quotienet operation" defined as follows: to compute the
                  product of $A_i$ with $A_j$ take any element $a_i$ of $A_i$
                  and any element $a_j$ of $A_j$ and let $A_iA_j$ be the element
                  of $\mathcal{P}$ containing $a_ia_j$ (this operation is
                  assumed to be well defined). Prove that the element of
                  $\mathcal{P}$ that contains the identity of $G$ is a normal
                  subgroup of $G$ and the elements of $\mathcal{P}$ are the
                  cosets of this subgroup (so $\mathcal{P}$ is just a quotient
                  group of $G$ in the usual sense.)
\end{enumerate}
