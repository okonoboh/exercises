Let $G$ and $H$ be groups.

\begin{enumerate}
%%%%%%%%%%%%%%%%%%%%%%%%%%%%%%%%%%%%%3.1.1%%%%%%%%%%%%%%%%%%%%%%%%%%%%%%%%%%%%%%
   \item[3.1.1]   Let $\varphi : G \rightarrow H$ be a homomorphism and let $E$
                  be a subgroup of $H$. Prove that $\varphi^{-1}(E) \le G$
                  (i.e., the preimage or pullback of a subgroup under a
                  homomorphism is a subgroup). If $E \trianglelefteq H$ prove
                  that $\varphi^{-1}(E) \trianglelefteq G$. Deduce that
                  $\text{ker}\varphi \trianglelefteq G$.

      \textbf{Proof.} Since $\varphi(1) = 1 \in E$ (Proposition 1), it follows 
      that $1 \in \varphi^{-1}(E)$, and hence, $\varphi^{-1}(E)$ is not empty. 
      Now let $x, y \in \varphi^{-1}(E)$. That is, $\varphi(x),\varphi(y)\in E$. 
      Since $E$ is a group, it follows that $\varphi(x)\varphi(y)^{-1} \in E$. 
      Thus by the homomorphic property of $\varphi$ and Proposition 1, it
      follows that $\varphi(xy^{-1}) = \varphi(x)\varphi(y^{-1}) = \varphi(x)
      \varphi(y)^{-1} \in E$, so that $xy^{-1} \in \varphi^{-1}(E)$. Hence
      $\varphi^{-1}(E) \le G$ by the Subgroup Criterion. Now suppose that $E$ is 
      a normal subgroup of $H$. Let $g \in G$ and $z \in \varphi^{-1}(E)$. To 
      show that $\varphi^{-1}(E)$ is 
      normal in $G$, it suffices to show that $gzg^{-1} \in \varphi^{-1}(E)$; 
      that is, we must show that the image of $gzg^{-1}$ under $\varphi$ lies in 
      $E$. By Proposition 1, we know that $\varphi(g^{-1}) = \varphi(g)^{-1}$. 
      Thus
      $$\varphi(gzg^{-1}) = \varphi(g)\varphi(z)\varphi(g^{-1}) =
        \varphi(g)\varphi(z)\varphi(g)^{-1}.$$
      The elements $\varphi(g)$ and $\varphi(g)^{-1}$ are clearly in $H$. Also
      $\varphi(z) \in E$ because $z \in \varphi^{-1}(E)$. Now since $E$ is
      normal in $H$, it follows by Theorem 6 that
      $\varphi(gzg^{-1}) = \varphi(g)\varphi(z)\varphi(g)^{-1} \in E$, so that
      $gzg^{-1} \in \varphi^{-1}(E)$; conclude that
      $\varphi^{-1}(E) \trianglelefteq G$. If $E$ is the trivial subgroup, then
      $1 \trianglelefteq H$, so that
      $\varphi^{-1}(1) = \text{ker}\varphi \trianglelefteq G$. \qed
%%%%%%%%%%%%%%%%%%%%%%%%%%%%%%%%%%%%%3.1.2%%%%%%%%%%%%%%%%%%%%%%%%%%%%%%%%%%%%%%
   \item[3.1.2]   Let $\varphi : G \rightarrow H$ be a homomorphism of groups
                  with kernel $K$ and let $a, b \in \varphi(G)$. Let
                  $X \in G/K$ be the fiber above $a$ and let $Y$ be the fiber
                  above $b$, i.e., $X = \varphi^{-1}(a)$, $Y = \varphi^{-1}(b)$.
                  Fix an element $u$ of $X$ (so $\varphi(u) = a$). Prove that if
                  $XY = Z$ in the quotient group $G/K$ and $w$ is any member of
                  $Z$, then there is some $v \in Y$ such that $uv = w$. [Show
                  that $u^{-1}w \in Y$.]

      \textbf{Proof.} Let $Z = XY$ (i.e., $Z$ is the fiber over $ab$) and fix
      $w \in Z$. We have that
      \begin{align*}
         \varphi(u^{-1}w) &= \varphi(u^{-1})\varphi(w) \\
            &= \varphi(u)^{-1}\varphi(w) &[\text{Proposition 1}] \\
            &= a^{-1}(ab) = (a^{-1}a)b = b,
      \end{align*}
      so that $u^{-1}w \in Y$. Since $u(u^{-1}w) = w$, we let $v = u^{-1}w$.\qed
%%%%%%%%%%%%%%%%%%%%%%%%%%%%%%%%%%%%%3.1.3%%%%%%%%%%%%%%%%%%%%%%%%%%%%%%%%%%%%%%
   \item[3.1.3]   Let $A$ be an abelian group and let $B$ be a subgroup of $A$.
                  Prove that $A/B$ is abelian. Give an example of a non-abelian
                  group $G$ containing a proper normal subgroup $N$ such that
                  $G/N$ is abelian.

      \textbf{Proof.} Let $X, Y \in A/B$. Then $X = aB$ and $Y = dB$ for some
      $a, d \in A$. It follows immediately that $A/B$ is abelian because
      \begin{align*}
         XY &= (aB)(dB) \\
            &= (ad)B = (da)B &[A\text{ is abelian}] \\
            &= (dB)(aB) \\
            &= YX,
      \end{align*}
      so that $A/B$ is abelian.

      \textbf{Example.} Let $G = D_8$ and $N = \cyc{r}$. By Example 2.2.3, $N$
      is normal in $D_8$. By computation, we get $G/N = \{sN, N\}$, so that
      $G/N \cong Z_2$. \qed
%%%%%%%%%%%%%%%%%%%%%%%%%%%%%%%%%%%%%3.1.4%%%%%%%%%%%%%%%%%%%%%%%%%%%%%%%%%%%%%%
   \item[3.1.4]   Prove that in the quotient group $G/N$,
                  $(gN)^\alpha = g^\alpha N$ for all $\alpha \in \Z$.

      \textbf{Proof.} Proceed by induction on $\alpha$. The statement is clearly
      true for $\alpha = 1$, so suppose $(gN)^k = g^kN$ for some positive
      integer $k$. So
      \begin{align*}
         (gN)^{k+1} &= (gN)^kgN &[\text{Exercise 1.1.19}] \\
            &= g^kNgN &[\text{Inductive Hypothesis}] \\
            &= (g^kg)N \\
            &= g^{k+1}N. &[\text{Exercise 1.1.19}]
      \end{align*}
      Thus our statement holds for all positive integer $\alpha$. For a postive
      integer $n$, we have that
      \begin{align*}
         (gN)^{-n} &= ((gN)^{-1})^n &[\text{Definition}] \\
            &= (g^{-1}N)^n &[\text{Proposition 5}] \\
            &= (g^{-1})^nN &[\text{From Inductive Proof Above}] \\
            &= g^{-n}N. &[\text{Definition}] 
      \end{align*}
      Finally $(gN)^0 = 1 = N = 1N = g^0N$, so that our statement holds for each
      integer $\alpha$. \qed
%%%%%%%%%%%%%%%%%%%%%%%%%%%%%%%%%%%%%3.1.5%%%%%%%%%%%%%%%%%%%%%%%%%%%%%%%%%%%%%%
   \item[3.1.5]   Use the preceding exercise to prove that the order of the
                  element $gN$ in $G/N$ is $n$, where $n$ is the smallest
                  positive integer such that $g^n \in N$ (and $gN$ has infinite
                  order if no such positive integer exists). Give an example to
                  show that the order of $gN$ in $G/N$ may be strictly smaller
                  than the order of $g$ in $G$.

      \textbf{Proof.} Suppose first that $|gN| = n \in \Z^+$. That is, $n$ is\
      the smallest positive integer such that $(gN)^n = g^nN = N$, where the
      last equality follows from Exercise 4. By Proposition 4, it follows that
      $g^n \in N$.

      \textbf{Example.} Let $G = D_8$ and $N = \cyc{r}$. By Example 2.2.3, $N$
      is normal in $D_8$. Since $G/N = \{N, sN\}$, we have $|rN| = 1 < 4 = |r|$. 
      \qed
%%%%%%%%%%%%%%%%%%%%%%%%%%%%%%%%%%%%%3.1.6%%%%%%%%%%%%%%%%%%%%%%%%%%%%%%%%%%%%%%
   \item[3.1.6]   Define $\varphi : \R^\times \rightarrow \{\pm1\}$ by letting
                  $\varphi(x)$ be $x$ divided by the absolute value of $x$.
                  Describe the fibers of $\varphi$ and prove that $\varphi$ is a
                  homomorphism.
   
      \textbf{Proof.} By inspection, we see that $\varphi$ maps the positive
      reals to 1 and the negative reals to $-1$. Thus the fibers of $\varphi$
      are the set of positive real numbers and the set of negative real numbers.
      The map $\varphi$ is a homomorphism because
      $$\varphi(ab) = \frac{ab}{|ab|} = \frac{ab}{|a||b|} = \frac{a}{|a|}
        \frac{b}{|b|} = \varphi(a)\varphi(b),$$
      for every nonzero real numbers $a$ and $b$. \qed
%%%%%%%%%%%%%%%%%%%%%%%%%%%%%%%%%%%%%3.1.7%%%%%%%%%%%%%%%%%%%%%%%%%%%%%%%%%%%%%%
   \item[3.1.7]   Define $\pi : \R^2 \rightarrow \R$ by $\pi((x, y)) = x + y$.
                  Prove that $\pi$ is a surjective homomorphism and describe the
                  kernel and fibers of $\pi$ geometrically.

      \textbf{Proof.} The map $\pi$ is surjective because if $r \in \R$, then
      $(r, 0) \in \R^2$, so that $\pi(r, 0) = r + 0 = r$; it is also an
      homomorphism because, for $(a, b), (c, d) \in \R^2$, we have that
      $$\pi((a, b) + (c, d)) = \pi(a + c, b + d) = a + c + b + d =
        a + b + c + d =\pi(a, b) + \pi(c, d).$$
      The kernel of $\pi$ is the $y$-axis and the fibers are lines parallel to
      the $y$-axis. \qed
%%%%%%%%%%%%%%%%%%%%%%%%%%%%%%%%%%%%%3.1.8%%%%%%%%%%%%%%%%%%%%%%%%%%%%%%%%%%%%%%
   \item[3.1.8]   Let $\varphi : \R^\times \rightarrow \R^\times$ be the map
                  sending $x$ to the absolute value of $x$. Prove that $\varphi$
                  is a homomorphism and find the image of $\varphi$. Describe
                  the kernel and the fibers of $\varphi$.

      \textbf{Proof.} For $a, b \in \R^\times$, we have that
      $$\varphi(ab) = |ab| = |a| \cdot |b| = \varphi(a)\varphi(b),$$
      so that $\varphi$ is a homomorphism. The image of $\varphi$ is the set of
      all positive real numbers. The kernel of $\varphi$ is $\{\pm 1\}$ and the
      fibers of $\varphi$ are all sets of the form
      $\{\pm a : a \in \R^+\}$. \qed
%%%%%%%%%%%%%%%%%%%%%%%%%%%%%%%%%%%%%3.1.9%%%%%%%%%%%%%%%%%%%%%%%%%%%%%%%%%%%%%%
   \item[3.1.9]   Define $\varphi : \C^\times \rightarrow \R^\times$ by
                  $\varphi(a + bi) = a^2 + b^2$. Prove that $\varphi$ is a
                  homomorphism and find the image of $\varphi$. Describe the
                  kernel and the fibers of $\varphi$ geometrically (as subsets
                  of the plane).

      \textbf{Proof.}  Let $z$ and $w$ be nonzero complex numbers. Then we have
      $z = a + bi$ and $w = c + di$, for some real numbers $a$, $b$, $c$, and
      $d$. So
      \begin{align*}
         \varphi(zw) &= \varphi((a + bi)(c + di)) \\
            &= \varphi(ac - bd + (ad + bc)i) \\
            &= (ac - bd)^2 + (ad + bc)^2 \\
            &= (ac)^2 + (bd)^2 - 2abcd + (ad)^2 + (bc)^2 + 2abcd \\
            &= (ac)^2 + (bd)^2 + (ad)^2 + (bc)^2 \\
            &= a^2(c^2 + d^2) + b^2(c^2 + d^2) \\
            &= (a^2 + b^2)(c^2 + d^2) \\
            &= \varphi(a + bi)\varphi(c + di) \\
            &= \varphi(z)\varphi(w).
      \end{align*}
      Thus $\varphi$ is a homomorphism. The kernel of $\varphi$ is the unit
      circle centered at the origin. The fibers of $\varphi$ are the circles of
      (nonzero radius) centered at the origin. \qed
%%%%%%%%%%%%%%%%%%%%%%%%%%%%%%%%%%%%%3.1.10%%%%%%%%%%%%%%%%%%%%%%%%%%%%%%%%%%%%%
   \item[3.1.10]  Let $\varphi : \Z/8\Z \rightarrow \Z/4\Z$ by
                  $\varphi(\m{a}) = \m{a}$. Show that this is a well defined, 
                  surjective homomorphism and describe its fibers and kernel 
                  explicitly (showing that $\varphi$ is well defined involves
                  the fact that $\m{a}$ has a different meaning in the domain
                  and range of $\varphi$).

      \textbf{Proof.} Suppose $\m{a} = \m{b}$ in $\Z/8\Z$, it follows that
      $a = b + 8k$ for some integer $k$. We thus have that
      \begin{align*}
         \varphi(\m{a}) &= \varphi(a + 8\Z) \\
            &= a + 4\Z \\
            &= b + 4\Z &[a - b = 4(2k) \in 4\Z] \\
            &= \varphi(b + 8\Z) = \varphi(\m{b});
      \end{align*}
      i.e., $\varphi$ is well defined. The map $\varphi$ is an homomorphism
      because
      $$\varphi(\m{a} + \m{b}) = \varphi(\m{a + b}) = \m{a + b} =
        \m{a} + \m{b}.$$
      Surjectivity of $\varphi$ is immediate because if $a + 4\Z \in \Z/4\Z$,
      then $\varphi(a + 8\Z) = a + 4\Z$. Now $\varphi(a + 8\Z) = 4\Z$ if and
      only if $a +4\Z = 4\Z$ if and only if $a$ is a multiple of $4$. Thus
      kernel of $\varphi = \{\m{0}, \m{4}\}$, and the fibers of $\varphi$ are
      $\{\{\m{0}, \m{4}\}, \{\m{1}, \m{5}\}, \{\m{2}, \m{6}\},
       \{\m{3}, \m{7}\}\}$. \qed
%%%%%%%%%%%%%%%%%%%%%%%%%%%%%%%%%%%%%3.1.11%%%%%%%%%%%%%%%%%%%%%%%%%%%%%%%%%%%%%
   \item[3.1.11]  Let $F$ be a field and let
                  $G = \left\{\left(\begin{tabular}{@{}cc@{}}
                     $a$ & $b$ \\
                     0   & $c$
                  \end{tabular}\right) : a, b, c \in F, ac \neq 0\right\} \le
                  GL_2(F)$.
                  \begin{enumerate}
                     \item Prove that the map $\varphi :
                           \left(\begin{tabular}{@{}cc@{}}
                              $a$ & $b$ \\
                              0   & $c$
                           \end{tabular}\right)\mapsto a$ is a surjective
                           homomorphism from $G$ onto $F^\times$. Describe the
                           fibers and kernel of $\varphi$.
                     \item Prove that the map $\psi :
                           \left(\begin{tabular}{@{}cc@{}}
                              $a$ & $b$ \\
                              0   & $c$
                           \end{tabular}\right)\mapsto (a, c)$ is a surjective
                           homomorphism from $G$ onto $F^\times\times F^\times$.
                           Describe the fibers and kernel of $\psi$.
                     \item Let $H : \left\{\left(\begin{tabular}{@{}cc@{}}
                              1 & $b$ \\
                              0 & 1
                           \end{tabular}\right) : b \in F\right\}$. Prove that
                           $H$ is isomorphic to the additive group $F$.
                  \end{enumerate}

      \textbf{Proof.}

      \begin{enumerate}
         \item Let $a \in \F^\times$. It follows immediately that $\varphi$ is
               onto because $\varphi\left(\begin{tabular}{@{}cc@{}}
                  $a$ & 0 \\
                  0   & 1
               \end{tabular}\right) = a$. Now let
               $A = \left(\begin{tabular}{@{}cc@{}}
                  $a$ & $b$ \\
                  0   & $c$
               \end{tabular}\right)$ and $B = \left(\begin{tabular}{@{}cc@{}}
                  $p$ & $q$ \\
                  0   & $r$
               \end{tabular}\right)$ be arbitrary members of $G$; we have that
               \begin{align*}
                  \varphi(AB) &= \varphi\left(\left(\begin{tabular}{@{}cc@{}}
                  $a$ & $b$ \\
                  0   & $c$
               \end{tabular}\right)\left(\begin{tabular}{@{}cc@{}}
                  $p$ & $q$ \\
                  0   & $r$
               \end{tabular}\right)\right) \\
                  &= \varphi\left(\begin{tabular}{@{}cc@{}}
                  $ap$ & $aq + br$ \\
                  0    & $cr$
               \end{tabular}\right) \\
                  &= ap = \varphi(A)\varphi(B),
               \end{align*}
               so that $\varphi$ is a homomorphism. The kernel of $\varphi$ is
               the set of all matrices in $G$ where the first column is a unit
               vector (i.e., $a_{11} = 1$). Hence a fiber $X$ of $\varphi$ is
               the set of all matrices in $G$ such that if
               $(a_{ij}), (b_{ij}) \in X$, then $a_{11} = b_{11}$.
         \item Let $a, c \in \F^\times \times \F^\times$. It follows immediately 
               that $\varphi$ is onto because
               $\varphi\left(\begin{tabular}{@{}cc@{}}
                  $a$ & 0 \\
                  0   & $c$
               \end{tabular}\right) = (a, c)$. Now let
               $A = \left(\begin{tabular}{@{}cc@{}}
                  $a$ & $b$ \\
                  0   & $c$
               \end{tabular}\right)$ and $B = \left(\begin{tabular}{@{}cc@{}}
                  $p$ & $q$ \\
                  0   & $r$
               \end{tabular}\right)$ be arbitrary members of $G$; we have that
               \begin{align*}
                  \varphi(AB) &= \varphi\left(\left(\begin{tabular}{@{}cc@{}}
                  $a$ & $b$ \\
                  0   & $c$
               \end{tabular}\right)\left(\begin{tabular}{@{}cc@{}}
                  $p$ & $q$ \\
                  0   & $r$
               \end{tabular}\right)\right) \\
                  &= \varphi\left(\begin{tabular}{@{}cc@{}}
                  $ap$ & $aq + br$ \\
                  0    & $cr$
               \end{tabular}\right) \\
                  &= (ap, cr) = (a, c)(p, r) = \varphi(A)\varphi(B),
               \end{align*}
               so that $\varphi$ is a homomorphism. The kernel of $\varphi$ is
               the set of all matrices in $G$ with $1$s on the main diagonal. 
               Hence a fiber $X$ of $\varphi$ is the set of all matrices in $G$ 
               such that if $(a_{ij}), (b_{ij}) \in X$, then $a_{11} = b_{11}$
               and $a_{22} = b_{22}$.
         \item The map $\varphi : G \rightarrow F$,
               $\left(\begin{tabular}{@{}cc@{}}
                  $ap$ & $aq + br$ \\
                  0    & $cr$
               \end{tabular}\right) \mapsto b$ is a homomorphism because, for
               $A = \left(\begin{tabular}{@{}cc@{}}
                  1   & $a$ \\
                  0   & 1
               \end{tabular}\right), B = \left(\begin{tabular}{@{}cc@{}}
                  1   & $b$ \\
                  0   & 1
               \end{tabular}\right) \in H$, we have
               \begin{align*}
                  \varphi(AB) &= \varphi\left(\begin{tabular}{@{}cc@{}}
                  1   & $a + b$ \\
                  0   & 1
               \end{tabular}\right) \\
               &= a + b = \varphi(A) + \varphi(B).
               \end{align*}              
               The kernel of $\varphi$ is trivial, so it follows by Exercise
               1.6.14 that $\varphi$ is injective; clearly, $\varphi$ is onto;
               thus, it is an isomorphism. Conclude that $H \cong F$. \qed
      \end{enumerate}
%%%%%%%%%%%%%%%%%%%%%%%%%%%%%%%%%%%%%3.1.12%%%%%%%%%%%%%%%%%%%%%%%%%%%%%%%%%%%%%
   \item[3.1.12]  Let $G$ be the additive group of real numbers, let $H$ be the
                  multiplicative group of complex numbers of absolute value 1
                  (the unit circle $S^1$ in the complex plane) and let
                  $\varphi : G \rightarrow H$ be the homomorphism
                  $\varphi : r \mapsto e^{2\pi ir}$. Draw the points on a real
                  line which lie in the kernel of $\varphi$. Describe similarly
                  the elements in the fibers of $\varphi$ above the points $-1$,
                  $i$, and $e^{4\pi i/3}$ of $H$. (Figure 1 of the text for this
                  homomorphism $\varphi$ is usually depicted using the following
                  diagram.)

      \textbf{Solution.} Kernel($\varphi) = \Z$; fibers above $-1$, $i$, and
      $e^{4\pi i/3}$ are the cosets $\frac{1}{2} + \Z$, $\frac{1}{4} + \Z$, and
      $\frac{2}{3} + \Z$, respectively.

      \textcolor{red}{TODO: Draw line in Tikz}
%%%%%%%%%%%%%%%%%%%%%%%%%%%%%%%%%%%%%3.1.13%%%%%%%%%%%%%%%%%%%%%%%%%%%%%%%%%%%%%
   \item[3.1.13]  Repeat the preceding exercise with the map $\varphi$ replaced
                  by the map $\varphi : r \mapsto e^{4\pi ir}$.

      \textbf{Solution.} Kernel($\varphi) = \frac{1}{2}\Z$; fibers above $-1$,
      $i$, and $e^{4\pi i/3}$ are the cosets
      $\frac{1}{4} + \frac{1}{2}\Z$, $\frac{1}{8} + \frac{1}{2}\Z$, and
      $\frac{1}{3} + \frac{1}{2}\Z$, respectively.
%%%%%%%%%%%%%%%%%%%%%%%%%%%%%%%%%%%%%3.1.14%%%%%%%%%%%%%%%%%%%%%%%%%%%%%%%%%%%%%
   \item[3.1.14]  Consider the additive quotient group $\Q/\Z$.
                  \begin{enumerate}
                     \item Show that every coset of $\Z$ in $\Q$ contains
                           exactly one representative $q \in \Q$ in the range
                           $0 \le q < 1$.
                     \item Show that every element of $\Q/\Z$ has finite order
                           but that there are elements of arbitrarily large
                           order,
                     \item Show that $\Q/\Z$ is the torsion subgroup of $\R/\Z$.
                     \item Prove that $\Q/\Z$ is isomorphic to the
                           multiplicative group of roots of unity in
                           $\C^\times$.
                  \end{enumerate}

      \textbf{Proof.}

      \begin{enumerate}
         \item First we will show that each coset contains such a
               representative. So let $a + \Z$ be a coset in $\Q/\Z$. Assume 
               without loss that $a = c/d$ for some integer $c$ and positive 
               integer $d$. By the Division Algorithm, it follows that
               $c = nd + r$, where $n$ and $r$ are integers, with $0 \le r < d$. 
               The equality $c = nd + r$ implies that $c/d = n + r/d$; i.e.,
               $c/d - r/d = n \in \Z$. Thus $a + \Z = r/d + \Z$ by Proposition
               4. Now since $0 \le r < d$, conclude that $0 \le r/d < 1$.
               Suppose $r/d + \Z$ has another representation $a'$ such that
               $0 \le a' < 1$. Hence $a' = r/d + m$ for some integer $m$. That
               is, if we add $m$ to the inequality $0 \le r/d < 1$, we will get 
               $m \le r/d + m < 1 + m$, so that $a'$ is in the half-closed
               interval $[m,  m + 1)$. But since $0 \le a' < 1$, it follows that
               $m = 0$, so that $a' = r/d$. Thus $r/d$ is unique.
         \item Let $q + \Z$ be a coset in $\Q/\Z$, where $q = a/b$ for 
               appropriate integers $a$ and $b$. Then it follows by Exercise
               4 that
               $$(q + \Z)^{|b|} = (|b|\cdot q) + \Z = \pm a + \Z = \Z
                  \quad(\text{since } \pm a \in \Z).$$
               Thus $|q + \Z| \le |b|$, so that $q + \Z$ has finite order. If
               $n$ is a positive integer, observe that $m \cdot 1/n$ is not an
               integer for each integer $m$ where $1 \le m < n$. Thus the coset
               $1/n + \Z$ has order $n$.
         \item Let $r + \Z$ be a coset in $\R/\Z$, for some real number $r$. We
               already showed that if $r$ is rational, then $r + \Z$ has finite
               order. So assume that $r$ is not rational. Now suppose to the
               contrary that $r + \Z$ also has finite order. Then 
               $(r + \Z)^n = rn + \Z = \Z$, for some positive integer $n$. That
               is, $rn \in \Z$, so that $r$ is a rational number, a
               contradiction. Thus every coset in $(\R - \Q)/\Z$ has infinite
               order. So conclude that the torsion subgroup of $\R/\Z$ is
               $\Q/\Z$.
         \item Consider the map $\alpha : \Q/\Z \rightarrow H$, (where $H$ is
               the multiplicative group of roots of unity in $\C^\times$)
               $q + \Z \mapsto e^{2\pi i q}$. First we will show that this map
               is well defined. So suppose that $p + \Z = q + \Z$ for some
               cosets in $\Q/\Z$, so that $p - q = n \in \Z$ (Proposition 4).
               Now we have that
               $$\alpha(p + \Z) = e^{2\pi i p} = e^{2\pi i (n+q)} = 
                 e^{2\pi i n}e^{2\pi i q} = 1 \cdot e^{2\pi i q} = 
                 \alpha(q + \Z).$$
               Hence $\alpha$ is well defined, and it is a homomorphism because
               $$\alpha((p + \Z) + (q + \Z)) = \alpha((p + q) + \Z) =
                 e^{2\pi i (p+q)} = e^{2\pi i p}e^{2\pi i q} = \alpha(p + \Z)
                 \alpha(q + \Z),$$
               for all $p + \Z, q + \Z \in \Q/\Z$. Now observe that 
               $\alpha(a + \Z) = e^{2\pi i a} = 1$ if and only if $a$ is an 
               integer if and only if $a + \Z = \Z$. Thus the kernel of
               $\alpha$ is trivial so that $\alpha$ is injective by Exercise 
               1.6.14. Let $z \in H$. Then $z^m = 1$ for some positive integer
               $m$. It follows that $z = e^{2\pi i/m}$ for some integer $i$.
               Surjectivity of $\alpha$ follows $\alpha(1/m + \Z) = z$. Conclude
               that $\alpha$ is an isomorphism, so that $\Q/\Z \cong H$.
      \end{enumerate} \qed
%%%%%%%%%%%%%%%%%%%%%%%%%%%%%%%%%%%%%3.1.15%%%%%%%%%%%%%%%%%%%%%%%%%%%%%%%%%%%%%
   \item[3.1.15]  Prove that a quotient of a divisible abelian group by any
                  proper subgroup is also divisible. Deduce that $\Q/\Z$ is
                  divisible.

      \textbf{Proof.} Suppose $A$ is a divisible abelian group with a proper
      subgroup $B$. Let $aB$ be a coset in $A/B$ and let $k$ be a positive
      integer. Since $A$ is divisible, it follows that there exists $x \in A$
      such that $a = x^k$. Thus $aB = x^kB = (xB)^k$. Since $xB \in A/B$ it
      follows that $A/B$ is divisible. Conclude by Exercise 2.4.19(a) and our
      result that $\Q/\Z$ is divisible. \qed
%%%%%%%%%%%%%%%%%%%%%%%%%%%%%%%%%%%%%3.1.16%%%%%%%%%%%%%%%%%%%%%%%%%%%%%%%%%%%%%
   \item[3.1.16]  Let $G$ be a group, let $N$ be a normal subgroup of $G$, and
                  let $\m{G} = G/N$. Prove that if $G = \cyc{x, y}$ then
                  $\m{G} = \cyc{\m{x}, \m{y}}$. Prove more generally that if
                  $G =\cyc{S}$ for any subset $S$ of $G$, then
                  $\m{G} = \cyc{\m{S}}$.

      \textbf{Proof.} We will only prove the general statement. If
      $S = \emptyset$, then the statement trivially holds, so suppose that
      $S = \{g_1, \ldots, g_n\}$, where $g_i \in G$, for $1 \le i \le n$. Now 
      suppose that $G = \cyc{S}$. We want to prove that $\m{G} = \cyc{\m{S}}$, 
      where $\m{S} = \{\m{g_1}, \ldots, \m{g_n}\} = \{g_1N, \ldots, g_nN\}$. Let 
      $X \in \m{G}$. Then $X = \m{g}$ for some $g \in G$. Since $G$ was
      generated by $S$, there exist $m$ elements of $S$ (not necessarily 
      distinct), $h_1$, $\ldots$, $h_m$ and integers $\epsilon_1$, $\ldots$,
      $\epsilon_m$ such that $g = {h_1}^{\epsilon_1} \ldots {h_m}^{\epsilon_m}$. 
      Thus 
      \begin{align*}
         \m{g} &= \m{{h_1}^{\epsilon_1} \ldots {h_m}^{\epsilon_m}} \\
               &= \m{{h_1}^{\epsilon_1}} \ldots \m{{h_m}^{\epsilon_m}}
                  &[\text{Proposition 5 (1)}] \\
               &= \m{h_1}^{\epsilon_1} \ldots \m{h_m}^{\epsilon_m}
                  \in \cyc{\m{S}},
                  &[\text{Exercise 4}]
      \end{align*}
      so that $\m{G} \subseteq \cyc{\m{S}}$. The reverse containment is evident
      because $\m{S} \subseteq \m{G}$. Thus $\m{G} = \cyc{\m{S}}$. \qed
%%%%%%%%%%%%%%%%%%%%%%%%%%%%%%%%%%%%%3.1.17%%%%%%%%%%%%%%%%%%%%%%%%%%%%%%%%%%%%%
   \item[3.1.17]  Let $G$ be the dihedral group of order 16 (whose lattice
                  appears in Section 2.5):
                  $$G = \cyc{r, s : r^8 = s^2 = 1, rs = sr^{-1}}$$
                  and let $\m{G} = G/\cyc{r^4}$ be the quotient of $G$ by
                  the subgroup generated by $r^4$ (this subgroup is the center
                  of $G$, hence is normal).
                  \begin{enumerate}
                     \item Show that the order of $\m{G}$ is 8.
                     \item Exhibit each element of $\m{G}$ in the form
                           $\m{s}^a\m{r}^b$, for some integers $a$ and $b$.
                     \item Find the order of each of the elements of
                           $\m{G}$ exhibited in (b).
                     \item Write each of the following elements of $\m{G}$ in
                           the form $\m{s}^a\m{r}^b$, for some integers $a$ and
                           $b$ as in (b): \quad $\m{rs}$, \quad $\m{sr^{-2}s}$,
                           \quad $\m{s^{-1}r^{-1}sr}$.
                     \item Prove that $\m{H} = \cyc{\m{s}, \m{r}^2}$ is a normal
                           subgroup of $\m{G}$ and $\m{H}$ is isomorphic to the
                           Klein 4-group. Describe the isomorphism type of the
                           complete preimage of $\m{H}$ in $G$.
                     \item Find the center of $\m{G}$ and describe the
                           isomorphism type of $\m{G}/Z(\m{G})$.
                  \end{enumerate}

      \textbf{Solution.}

      \begin{enumerate}
         \item Since $\cyc{r^4} = \{1, r^4\}$, it follows that each left coset
               of $\cyc{r^4}$ in $G$ has order 2. Thus $|\m{G}| = |G|/2 = 8$.
         \item The elements of $\m{G}$ are:
               \begin{align*}
                  \cyc{r^4} = \{1, r^4\} &= \m{r}^0 = \m{1} \\
                  r\cyc{r^4} = \{r, r^5\} &= \m{r}^1 \\
                  r^2\cyc{r^4} = \{r^2, r^6\} &= \m{r}^2 \\
                  r^3\cyc{r^4} = \{r^3, r^7\} &= \m{r}^3 \\
                  s\cyc{r^4} = \{s, sr^4\} &= \m{s}^1 \\
                  sr\cyc{r^4} = \{sr, sr^5\} &= \m{s}^1\m{r}^1 \\
                  sr^2\cyc{r^4} = \{sr^2, sr^6\} &= \m{s}^1\m{r}^2 \\
                  sr^3\cyc{r^4} = \{sr^3, sr^7\} &= \m{s}^1\m{r}^3.
               \end{align*}
               Observe that
               $$\m{G} = \cyc{\m{r}, \m{s} : \m{r}^4 = \m{s}^2 = \m{1},
                 \m{s}\m{r}  = \m{r}^{-1}\m{s}} \cong D_8.$$
         \item \begin{align*}
                  |\m{1}| &= 1 \\
                  |\m{r}| &= |\m{r}^3| = 4 \\
                  |\m{r}^2|&= |\m{s}| = |\m{s}\cdot\m{r}| =
                  |\m{s}\cdot\m{r}^2| = |\m{s}\cdot\m{r}^3| = 2.
               \end{align*}
         \item \begin{align*}
                  \m{rs} &= \m{sr^{-1}} = \m{s}\cdot\m{r}^3 \\
                  \m{sr^{-2}s} &= \m{ssr^2} = \m{r}^2 \\
                  \m{s^{-1}r^{-1}sr} &= \m{s^{-1}srr} = \m{r}^2.
               \end{align*}
         \item Observe first that the generators of $\m{H}$ commute. Since each
               of the generators has order 2, the maximum order of $\m{H}$ is 4;
               indeed, we have $\m{H} = \{\m{1}, \m{s}, \m{r}^2,
               \m{s}\cdot\m{r}^2\}$. By Exercise 2.2.6(a), we have that
               $\m{H} \le N_{\m{G}}(\m{H})$, so that $|N_{\m{G}}(\m{H})| \ge 4$.
               Since $N_{\m{G}}(\m{H}) \le \m{G}$, it follows by Lagrange's
               Theorem that $|N_{\m{G}}(\m{H})| \in \{4, 8\}$. If we can show
               that $N_{\m{G}}(\m{H})$ has at least 5 elements, then we can
               conclude that $|N_{\m{G}}(\m{H})| = 8$. Now
               \begin{align*}
                  \m{r}\m{H}\m{r}^{-1} &= \{\m{r}\m{1}\m{r}^{-1},
                     \m{r}\m{s}\m{r}^{-1}, \m{r}\m{r}^2\m{r}^{-1},
                     \m{r}\m{sr^2}\m{r}^{-1}\} \\
                     &= \{\m{1}, \m{s}\cdot\m{r}^2, \m{r}^2, \m{s}\} = \m{H},
               \end{align*}
               so that $\m{r} \in N_{\m{G}}(\m{H})$. It follows that
               $|N_{\m{G}}(\m{H})| \ge 5$. Conclude that
               $|N_{\m{G}}(\m{H})| = 8$, so that $N_{\m{G}}(\m{H}) = \m{G}$;
               that is, $\m{H} \trianglelefteq \m{G}$. Since $\m{H}$ has no
               element of order 4, it follows by Exercise 2.5.10 that $\m{H}$ is
               isomorphic to the Klein 4-group. By definition, we have that the
               complete preimage of $\m{H}$ in $G$, say X, is
               $$\{g \in G : \m{g} \in \m{H}\} =
                 \{1, r^2, r^4, r^6, s, sr^2, sr^4, sr^6\}.$$
               Now let $x = r^2$ and $y = s$. It follows that $x^4 = r^8 = 1$,
               $y^2 = s^2 = 1$, $yx = sr^2 = r^{-2}s = x^{-1}y$, so that $x$ and
               $y \in X$ satisy the relations that $r$ and $s \in D_8$ satisy, 
               respectively; additionally observe that since $|X| = 8$ and since 
               every element in $X$ can be written in the form $y^ax^b$ for 
               appropriate integers $a$ and $b$, it follows that
               $$X = \cyc{x, y : x^4 = y^2 = 1, yx = x^{-1}y} \cong D_8.$$
         \item By Exercise 2.2.7(b), $Z(D_8) = \cyc{r^2}$. Thus since
               $\m{G} \cong D_8$ (see (b)), it follows that
               $Z(\m{G}) = \cyc{\m{r}^2}$. Thus
               $$\m{G}/Z(\m{G}) = \{Z(\m{G}), \m{r}Z(\m{G}), \m{s}Z(\m{G}),
                 \m{sr}Z(\m{G})\}.$$
               The order of each nonidentity element in the preceding set is 2;
               conclude by Exercise 2.5.10 that $\m{G}/Z(\m{G})$ is isomorphic
               to the Klein 4-group.
               \qed
      \end{enumerate}
%%%%%%%%%%%%%%%%%%%%%%%%%%%%%%%%%%%%%3.1.18%%%%%%%%%%%%%%%%%%%%%%%%%%%%%%%%%%%%%
   \item[3.1.18]  Let $G$ be the quasidihedral group of order 16 (whose lattice
                  was computed in Exercise 2.5.11):
                  $$G = \cyc{\sigma, \tau : \sigma^8 = \tau^2 = 1,
                             \sigma\tau = \tau\sigma^3}$$
                  and let $\m{G} = G/\cyc{\sigma^4}$ be the quotient of $G$ by
                  the subgroup generated by $\sigma^4$ (this subgroup is the
                  center of $G$, hence is normal).
                  \begin{enumerate}
                     \item Show that the order of $\m{G}$ is 8.
                     \item Exhibit each element of $\m{G}$ in the form
                           $\m{\tau}^a\m{\sigma}^b$, for some integers $a$ and
                           $b$.
                     \item Find the order of each of the elements of $\m{G}$
                           exhibited in (b).
                     \item Write each of the following elements of $\m{G}$ in
                           the form $\m{\tau}^a\m{\sigma}^b$, for some integers
                           $a$ and $b$ as in (b): \quad $\m{\sigma\tau}$, \quad
                           $\m{\tau\sigma^{-2}\tau}$, \quad
                           $\m{\tau^{-1}\sigma^{-1}\tau\sigma}$.
                     \item Prove that $\m{G} \cong D_8$.
                  \end{enumerate}

      \textbf{Solution.}

      \begin{enumerate}
         \item Since $\cyc{\sigma^4} = \{1, \sigma^4\}$, it follows that each 
               left coset of $\cyc{\sigma^4}$ in $G$ has order 2. Thus
               $|\m{G}| = |G|/2 = 8$.
         \item The elements of $\m{G}$ are:
               \begin{align*}
                  \cyc{\sigma^4} = \{1, \sigma^4\} &= \m{1} \\
                  \sigma\cyc{\sigma^4} = \{\sigma, \sigma^5\} &= \m{\sigma} \\
                  \sigma^2\cyc{\sigma^4}=\{\sigma^2,\sigma^6\} &=\m{\sigma}^2 \\
                  \sigma^3\cyc{\sigma^4} =\{\sigma^3,\sigma^7\}&=\m{\sigma}^3 \\
                  \tau\cyc{\sigma^4} = \{\tau, \tau\sigma^4\} &= \m{\tau} \\
                  \tau\sigma\cyc{\sigma^4} = \{\tau\sigma, \tau\sigma^5\} &=
                     \m{\tau}\cdot\m{\sigma} \\
                  \tau\sigma^2\cyc{\sigma^4} = \{\tau\sigma^2, \tau\sigma^6\} &= 
                     \m{\tau}\cdot\m{\sigma}^2 \\
                  \tau\sigma^3\cyc{\sigma^4} = \{\tau\sigma^3, \tau\sigma^7\} &= 
                     \m{\tau}\cdot\m{\sigma}^3.
               \end{align*}
         \item \begin{align*}
                  |\m{1}| &= 1 \\
                  |\m{\sigma}| &= |\m{\sigma}^3| = 4 \\
                  |\m{\sigma}^2|&= |\m{\tau}| = |\m{\tau}\cdot\m{\sigma}| =
                  |\m{\tau}\cdot\m{\sigma}^2| = |\m{\tau}\cdot\m{\sigma}^3| = 2.
               \end{align*}
         \item \begin{align*}
                  \m{\sigma\tau} &= \m{\tau}\cdot\m{\sigma}^3 \\
                  \m{\tau\sigma^{-2}\tau} &= \m{\tau\sigma^6\tau} =
                     \m{\sigma^2\tau\tau} = \m{\sigma}^2\\
                  \m{\tau^{-1}\sigma^{-1}\tau\sigma} &=
                     \m{\tau\sigma^3\tau\sigma} =
                     \m{\sigma\tau\tau\sigma} = \m{\sigma}^2.
               \end{align*}
         \item Let $x = \m{\sigma}$ and $y = \m{\tau}$. It follows that
               $x^4 = y^2 = \m{1}$, $yx = x^{-1}y$, so that $x$ and $y \in\m{G}$ 
               satisy the relations that $r$ and $s \in D_8$ satisy,
               respectively; additionally observe that since $|\m{G}| = 8$ and 
               since every element in $\m{G}$ can be written in the form
               $y^ax^b$ for appropriate integers $a$ and $b$, it follows that
               $$\m{G} = \cyc{x, y : x^4 = y^2 = 1, yx = x^{-1}y} \cong D_8.$$
      \end{enumerate} \qed
%%%%%%%%%%%%%%%%%%%%%%%%%%%%%%%%%%%%%3.1.19%%%%%%%%%%%%%%%%%%%%%%%%%%%%%%%%%%%%%
   \item[3.1.19]  Let $G$ be the modular group of order 16 (whose lattice was
                  computed in Exercise 2.5.14):
                  $$G = \cyc{u, v : u^2 = v^8 = 1, vu = uv^5}$$
                  and let $\m{G} = G/\cyc{v^4}$ be the quotient of $G$ by
                  the subgroup generated by $v^4$ (this subgroup is contained in
                  the center of $G$, hence is normal).
                  \begin{enumerate}
                     \item Show that the order of $\m{G}$ is 8.
                     \item Exhibit each element of $\m{G}$ in the form
                           $\m{u}^a\m{v}^b$, for some integers $a$ and $b$.
                     \item Find the order of each of the elements of $\m{G}$
                           exhibited in (b).
                     \item Write each of the following elements of $\m{G}$ in
                           the form $\m{u}^a\m{v}^b$, for some integers $a$ and
                           $b$ as in (b): \quad $\m{vu}$, \quad $\m{uv^{-2}u}$,
                           \quad $\m{u^{-1}v^{-1}uv}$.
                     \item Prove that $\m{G}$ is abelian and is isomorphic to
                           $Z_2 \times Z_4$.
                  \end{enumerate}

      \textbf{Solution.}

      \begin{enumerate}
         \item Since $\cyc{v^4} = \{1, v^4\}$, it follows that each left coset
               of $\cyc{v^4}$ in $G$ has order 2. Thus $|\m{G}| = |G|/2 = 8$.
         \item The elements of $\m{G}$ are:
               \begin{align*}
                  \cyc{v^4} = \{1, v^4\} &= \m{1} \\
                  v\cyc{v^4} = \{v, v^5\} &= \m{v} \\
                  v^2\cyc{v^4} = \{v^2, v^6\} &= \m{v}^2 \\
                  v^3\cyc{v^4} =\{v^3, v^7\} &= \m{v}^3 \\
                  u\cyc{v^4} = \{u, uv^4\} &= \m{u} \\
                  uv\cyc{v^4} = \{uv, uv^5\} &= \m{u} \cdot \m{v} \\
                  uv^2\cyc{v^4} = \{uv^2, uv^6\} &= \m{u} \cdot \m{v}^2 \\
                  uv^3\cyc{v^4} = \{uv^3, uv^7\} &= \m{u} \cdot \m{v}^3.
               \end{align*}
         \item \begin{align*}
                  |\m{1}| &= 1 \\
                  |\m{v}| &= |\m{v}^3| = 4 \\
                  |\m{v}^2|&= |\m{u}| = |\m{u} \cdot \m{v}| =
                  |\m{u} \cdot \m{v}^2| = |\m{u} \cdot \m{v}^3| = 2.
               \end{align*}
         \item \begin{align*}
                  \m{vu} &= \m{u} \cdot \m{v}^5 = \m{u} \cdot \m{v} = \m{uv} \\
                  \m{uv^{-2}u} &= \m{uv^6u} = \m{uuv^{30}} = \m{v}^2\\
                  \m{u^{-1}v^{-1}uv} &= \m{uv^7uv} = \m{uuv^{36}} = \m{1}.
               \end{align*}
         \item By (c), (d), and Exercise 16, it follows that
               \begin{align*}
                  \m{G} = \cyc{\m{u}, \m{v} : \m{u}^2 = \m{v}^8 = \m{v}^4 = 1,
                     \m{vu} = \m{uv}}
               \end{align*}
               Suppose $Z_2 = \cyc{x}$ and $Z_4 = \cyc{y}$. Let $p = (x,1)$ and
               $q = (1, y)$ be elements in $Z_2 \times Z_4$. The elements of
               $Z_2 \times Z_4$ are:
               \begin{align*}
                  (1, 1) &= q^0 & (1, y) &= q &
                     (1, y^2) &= q^2 & (1, y^3) &= q^3 \\
                  (x, 1) &= p & (x, y) &= pq &
                     (x, y^2) &= pq^2 & (x, y^3) &= pq^3
               \end{align*}
               Thus
               $$Z_2 \times Z_4 = \cyc{p , q: p^2 = q^4 = 1, qp = pq}.$$
               We see from above that $p, q \in Z_2 \times Z_4$ satisfy the same
               relations that $\m{u}, \m{v}$ satisfy in $\m{G}$; and since they
               have the same order, we conclude that these two groups are 
               isomorphic.
      \end{enumerate} \qed
%%%%%%%%%%%%%%%%%%%%%%%%%%%%%%%%%%%%%3.1.20%%%%%%%%%%%%%%%%%%%%%%%%%%%%%%%%%%%%%
   \item[3.1.20]  Let $G = \Z/24\Z$ and let
                  $\widetilde{G} = G/\cyc{\m{12}}$, where for each integer $a$
                  we simplify notation by writing $\widetilde{\m{a}}$ as
                  $\widetilde{a}$.
                  \begin{enumerate}
                     \item Show that $\widetilde{G} = \{\widetilde{0},
                           \widetilde{1}, \ldots, \widetilde{11}\}$.
                     \item Find the order of each element of $\widetilde{G}$.
                     \item Prove that $\widetilde{G} \cong \Z/12\Z$. (Thus
                           $(\Z/24\Z)/(12\Z/24\Z) \cong \Z/12\Z$, just as if we
                           inverted and called the $24\Z\text{'s}$.)
                  \end{enumerate}

      \textbf{Solution.}

      \begin{enumerate}
         \item The elements of $\widetilde{G}$ are:
               \begin{align*}
                  \m{0} + \cyc{\m{12}} &= \{\m{0}, \m{12}\} = \widetilde{0}
                  &\m{6} + \cyc{\m{12}} = \{\m{6}, \m{18}\} &= \widetilde{6}\\
                  \m{1} + \cyc{\m{12}} &= \{\m{1}, \m{13}\} = \widetilde{1}
                  &\m{7} + \cyc{\m{12}} = \{\m{7}, \m{19}\} &= \widetilde{7}\\
                  \m{2} + \cyc{\m{12}} &= \{\m{2}, \m{14}\} = \widetilde{2}
                  &\m{8} + \cyc{\m{12}} = \{\m{8}, \m{20}\} &= \widetilde{8}\\
                  \m{3} + \cyc{\m{12}} &= \{\m{3}, \m{15}\} = \widetilde{3}
                  &\m{9} + \cyc{\m{12}} = \{\m{9}, \m{21}\} &= \widetilde{9}\\
                  \m{4} + \cyc{\m{12}} &= \{\m{4}, \m{16}\} = \widetilde{4}
                  &\m{10} + \cyc{\m{12}} = \{\m{10},\m{22}\}&=\widetilde{10}\\
                  \m{5} + \cyc{\m{12}} &= \{\m{5}, \m{17}\} = \widetilde{5}
                  &\m{11}+\cyc{\m{12}} = \{\m{11},\m{23}\} &= \widetilde{11}
               \end{align*}
         \item The orders of $\widetilde{G}$'s elements are:
               \begin{align*}
                  |\widetilde{0}| &= 1 &|\widetilde{3}| &= 4
                     &|\widetilde{6}| &= 2 &|\widetilde{9}| &= 4 \\
                  |\widetilde{1}| &= 12 &|\widetilde{4}| &= 3
                     &|\widetilde{7}| &= 12 &|\widetilde{10}| &= 6 \\
                  |\widetilde{2}| &= 6 &|\widetilde{5}| &= 12
                     &|\widetilde{8}| &= 3 &|\widetilde{11}| &= 12
               \end{align*}
         \item Since $|\widetilde{G}| = |\widetilde{1}| =12$, it follows that
               $\widetilde{G} = \cyc{\widetilde{1}}$, so that
               $\widetilde{G} \cong \Z/12\Z$ by Theorem 2.4.
      \end{enumerate} \qed
%%%%%%%%%%%%%%%%%%%%%%%%%%%%%%%%%%%%%3.1.21%%%%%%%%%%%%%%%%%%%%%%%%%%%%%%%%%%%%%
   \item[3.1.21]  Let $G = Z_4 \times Z_4$ be given in terms of the following
                  generators and relations:
                  $$G = \cyc{x, y : x^4 = y^4 = 1, xy = yx}.$$
                  Let $\m{G} = G/\cyc{x^2y^2}$ (note that every subgroup of the
                  abelian group $G$ is normal).
                  \begin{enumerate}
                     \item Show that the order of $\m{G}$ is 8.
                     \item Exhibit each element of $\m{G}$ in the form
                           $\m{x}^a\m{y}^b$, for some integers $a$ and $b$.
                     \item Find the order of each of the elements of $\m{G}$
                           exhibited in (b).
                     \item Prove that $\m{G} \cong Z_4 \times Z_2$.
                  \end{enumerate}

      \textbf{Solution.}

      \begin{enumerate}
         \item Since $\cyc{x^2y^2} = \{1, x^2y^2\}$, it follows that each left 
               coset of $\cyc{x^2y^2}$ in $G$ has order 2. Thus
               $|\m{G}| = |G|/2 = 8$.
         \item The elements of $\m{G}$ are:
               \begin{align*}
                  1\cyc{x^2y^2} = \{1, x^2y^2\} &= \m{1}
                     &y\cyc{x^2y^2} = \{y, x^2y^3\} &= \m{y} \\
                  x\cyc{x^2y^2} = \{x, x^3y^2\} &= \m{x}
                     &xy\cyc{x^2y^2} = \{xy, x^3y^3\} &= \m{x}\cdot\m{y} \\
                  x^2\cyc{x^2y^2} = \{x^2, y^2\} &= \m{x}^2
                     &x^2y\cyc{x^2y^2} = \{x^2y, y^3\} &= \m{x}^2 \cdot \m{y} \\
                  x^3\cyc{x^2y^2} = \{x^3, xy^2\} &= \m{x}^3
                     &x^3y\cyc{x^2y^2} = \{x^3y, xy^3\} &= \m{x}^3 \cdot \m{y}
               \end{align*}
         \item The orders of $\m{G}$'s elements are:
               \begin{align*}
                  |\m{1}| &= 1 &|\m{x}| &= 4 &|\m{x}^2| &= 2 &|\m{x}^3| &= 4 \\
                  |\m{y}| &= 4 &|\m{xy}| &= 2 &|\m{x^2y}| &= 4 &|\m{x^3y}| &= 2
               \end{align*}
         \item Now $\m{G}$ is abelian by Exercise 3 and observe that every 
               element of $\m{G}$ can be generated by $\m{x}$ and $\m{xy}$.
               Since $\m{x}^4 = \m{xy}^2 = \m{1}$, it follows that
               $$\m{G} = \cyc{\m{x}, \m{xy} : \m{x}^4 = \m{xy}^2 = \m{1},
                  \m{x}\cdot\m{xy} = \m{xy}\cdot\m{x}}.$$
               Thus $\m{x}$ and $\m{xy}$ in $\m{G}$ satisfy the relations that
               the generators of $Z_2 \times Z_4$ satisfy in Exercise 19(e), so
               that $\m{G} \cong Z_2 \times Z_4$. Conclude by Exercise 1.6.11
               that $\m{G} \cong Z_4 \times Z_2$.
      \end{enumerate} \qed
%%%%%%%%%%%%%%%%%%%%%%%%%%%%%%%%%%%%%3.1.22%%%%%%%%%%%%%%%%%%%%%%%%%%%%%%%%%%%%%
   \item[3.1.22]  \begin{enumerate}
                     \item Prove that if $H$ and $K$ are normal subgroups of a
                           group $G$ then their intersection $H \cap K$ is also
                           a normal subgroup of $G$.
                     \item Prove that the intersection of an arbitrary nonempty
                           collection of normal subgroups of a group is a normal
                           subgroup (do not assume the collection is countable).
                  \end{enumerate}

      \textbf{Proof.}

      \begin{enumerate}
         \item Let $I = \{1, 2\}, H_1 = H$, and $H_2 = K$ in part (b).
         \item Let $G$ be a group and let $I$ be an indexing set such that
               $H_i \trianglelefteq G$ for each $i \in I$. Let
               $$H = \bigcap_{i \in I} H_i.$$
               By Exercise 2.1.10(b), $H \le G$, so it remains to show that $H$
               is normal in $G$. Let $g \in G$ and $h \in H$. By definition, it
               follows that $h \in H_i$ for each $i \in I$. Since each $H_i$ is
               normal it follows that $ghg^{-1} \in H_i$ (for each $i \in I$),
               so that $ghg^{-1} \in H$; that is, $H \trianglelefteq G$ by 
               Theorem 6(5).
      \end{enumerate} \qed
%%%%%%%%%%%%%%%%%%%%%%%%%%%%%%%%%%%%%3.1.23%%%%%%%%%%%%%%%%%%%%%%%%%%%%%%%%%%%%%
   \item[3.1.23]  Prove that the join of any nonempty collection of normal
                  subgroups of a group is a normal subgroup.

      \textbf{Proof.} Let $G$ be a group and let $I$ be an indexing set such
      that $H_i \trianglelefteq G$ for each $i \in I$. Let
      $$H = \bigcup_{i \in I} H_i.$$
      We want to show that $\cyc{H} \trianglelefteq G$. Since $\cyc{H} \le G$,
      it remains to show that $\cyc{H}$ is normal in $G$. Let $g \in G$ and
      $h \in \cyc{H}$. It follows that
      $h = {h_1}^{\epsilon_1}{h_2}^{\epsilon_2}\cdots {h_n}^{\epsilon_n}$, where
      $n \in \Z^+$, $\epsilon_i \in \Z$, and $h_i \in H$, for $1 \le i \le n$.
      Now
      \begin{align*}
         ghg^{-1} &= g({h_1}^{\epsilon_1}{h_2}^{\epsilon_2} \cdots
                     {h_n}^{\epsilon_n})g^{-1} \\
           &= (g{h_1}^{\epsilon_1}g^{-1})(g{h_2}^{\epsilon_2}g^{-1})\cdots
                     (g{h_n}^{\epsilon_n}g^{-1}) &[\text{Exercise 26(a)}] \\
            &= (g{h_1}g^{-1})^{\epsilon_1}(g{h_2}g^{-1})^{\epsilon_2}\cdots
                     (g{h_n}g^{-1})^{\epsilon_n} &[\text{Lemma }\ref{1_1_22_1}]
      \end{align*}
      Consider $g{h_r}g^{-1}$, where $1 \le r \le n$. Since $h_r \in H$, it 
      follows that $h_r \in H_k$ for some $k \in I$; and since $H_k$ is normal 
      in $G$, it follows that $g{h_r}g^{-1} \in H_k$; also, since
      $(g{h_r}g^{-1})^{\epsilon_r} = g{h_r}^{\epsilon_r}g^{-1}$
      (Lemma \ref{1_1_22_1}), it follows by closure of $H_k$ that
      $g{h_r}^{\epsilon_r}g^{-1} \in H_k$, so that
      $g{h_r}^{\epsilon_r}g^{-1} \in H$. Hence
      $g{h_i}^{\epsilon_i}g^{-1} \in H \subseteq \cyc{H}$, for each
      $i \in \{1, \ldots, n\}$. So it follows by closure of multiplication that  
      $ghg^{-1} \in \cyc{H}$. That is, $\cyc{H} \trianglelefteq G$ by Theorem
      6(5). \qed
%%%%%%%%%%%%%%%%%%%%%%%%%%%%%%%%%%%%%3.1.24%%%%%%%%%%%%%%%%%%%%%%%%%%%%%%%%%%%%%
   \item[3.1.24]  Prove that if $N \trianglelefteq G$ and $H$ is any subgroup of
                  $G$ then $N \cap H \trianglelefteq H$.

      \textbf{Proof.} Suppose $N \trianglelefteq G$ and $H \le G$. We have
      $N \cap H \le H$ by Exercise 2.1.10(a), so it remains to show that
      $N \cap H$ is normal in $H$. Let $h \in H$ and $m \in N \cap H$. Since 
      $m \in N \cap H$, we must particularly have that $m \in N$. Since $N$ is 
      normal in $G$, we have that $hmh^{-1} \in N$. Also, we must have that
      $m \in H$ and by closure of $H$, it follows that $hmh^{-1} \in H$, so that
      $hmh^{-1} \in N \cap H$. Conclude by Theorem 6(5) that
      $N \cap H \trianglelefteq H$. \qed
%%%%%%%%%%%%%%%%%%%%%%%%%%%%%%%%%%%%%3.1.25%%%%%%%%%%%%%%%%%%%%%%%%%%%%%%%%%%%%%
   \item[3.1.25]  \begin{enumerate}
                     \item Prove that a subgroup $N$ of $G$ is normal if and
                           only if $gNg^{-1} \subseteq N$ for \textit{all}
                           $g \in G$.
                     \item Let $G = GL_2(\Q)$, let $N$ be the subgroup of upper
                           triangular matrices with integer entries and 1's on
                           the diagonal, and let $g$ be the diagonal matrix with
                           entries 2,1. Show that $gNg^{-1} \subseteq N$ but $g$
                           does $\textit{not}$ normalize $N$.
                  \end{enumerate}

      \textbf{Proof.}

      \begin{enumerate}
         \item Let $N \le G$.

               ($\Leftarrow$) Suppose $gNg^{-1} \subseteq N$ for each $g \in G$.
               Let $g \in G$. To show that $N$ is normal in $G$, it suffices to
               show that $gNg^{-1} = N$. By our assumption, we already have that
               $gNg^{-1} \subseteq N$, so we will now show reverse containment.
               Let $n \in N$. Since $g^{-1} \in G$, it follows from our 
               assumption that $g^{-1}Ng = g^{-1}N(g^{-1})^{-1} \subseteq N$.
               Hence $g^{-1}ng \in N$, so that
               $n = g(g^{-1}ng)g^{-1} \in gNg^{-1}$. That is,
               $N \subseteq gNg^{-1}$. Conclude that $gNg^{-1} = N$, so that
               $N \trianglelefteq G$.
      
               ($\Rightarrow$) Suppose $N \trianglelefteq G$. Let $g \in G$. By
               normality, we have that $gNg^{-1} = N$; particularly, it follows
               that $gNg^{-1} \subseteq N$. Since $g$ was arbitrary, conclude
               that $gNg^{-1} \subseteq N$ for each $g \in G$.
         \item Let $X_j$ be the matrix in $N$ whose entry in the 1st row and
               2nd column is $j \in \Z$. Thus each matrix in $N$ can be written
               as $X_j$ for some unique integer $j$. Observe that $g^{-1}$ is
               the diagonal matrix with entries $1/2, 1$. So for some integer
               $i$, we have that
               $$gX_ig^{-1} = X_{2i} \in N.$$
               That is, $gNg^{-1} \subseteq N$. Thus the (1,2)-entry of
               $gng^{-1}$ is even for each $n \in N$; so $X_1 \neq gng^{-1}$ for
               each $n \in N$, but $X_1 \in N$. Conclude that
               $gNg^{-1} \neq N$, so that $g$ does not normalize $N$.
      \end{enumerate} \qed
%%%%%%%%%%%%%%%%%%%%%%%%%%%%%%%%%%%%%3.1.26%%%%%%%%%%%%%%%%%%%%%%%%%%%%%%%%%%%%%
   \item[3.1.26]  Let $a, b \in G$.
                  \begin{enumerate}
                     \item Prove that the conjugate of the product of $a$ and
                           $b$ is the product of the conjugate of $a$ and the
                           conjugate of $b$. Prove that the order of $a$ and the
                           order of any conjugate of $a$ are the same.
                     \item Prove that the conjugate of $a^{-1}$ is the inverse
                           of the conjugate of $a$.
                     \item Let $N = \cyc{S}$ for some subset $S$ of $G$. Prove
                           that $N \trianglelefteq G$ if $gSg^{-1} \subseteq N$
                           for all $g \in G$.
                     \item Deduce that if $N$ is the cyclic group $\cyc{x}$,
                           then $N$ is normal in $G$ if and only if for each
                           $g \in G$, $gxg^{-1} = x^k$ for some $k \in \Z$.
                     \item Let $n$ be a positive integer. Prove that the
                           subgroup $N$ of $G$ generated by all the elements of
                           $G$ of order $n$ is a normal subgroup of $G$.
                  \end{enumerate}

      \textbf{Proof.}

      \begin{enumerate}
         \item Let $g(ab)g^{-1}$ be the conjugate of $ab$ by $g \in G$. Thus
               $$g(ab)g^{-1} = ga1bg^{-1} = gag^{-1} \cdot gbg^{-1},$$
               so that $g(ab)g^{-1}$ is the product of the conjugate of $a$ and
               the conjugate of $b$. The equality $|a| = |gag^{-1}|$ follows
               from Exercise 1.1.22.
         \item Let $ga^{-1}g^{-1}$ be the conjugate of $a^{-1}$ by $g \in G$. 
               Then it follows that
               $$ga^{-1}g^{-1} = (g^{-1})^{-1}a^{-1}g^{-1} = (gag^{-1})^{-1},$$
               as desired.
         \item If $S$ is empty, then $N = 1 \trianglelefteq G$, so suppose that
               $S$ is not empty. Assume $hSh^{-1} \subseteq N$ for each
               $h \in G$. Let $g \in G$ and $n \in N$. Then
               $n = {s_1}^{\epsilon_1}{s_2}^{\epsilon_2} \cdots
                {s_m}^{\epsilon_m}$, where $m \in \Z^+$, and for each
               $1 \le i \le m$, $\epsilon_i \in \Z$ and $s_i \in S$. Now
               \begin{align*}
                  gng^{-1} &= g({s_1}^{\epsilon_1}{s_2}^{\epsilon_2} \cdots
                     {s_m}^{\epsilon_m})g^{-1} \\
                     &= (g{s_1}^{\epsilon_1}g^{-1})\cdot
                        (g{s_2}^{\epsilon_2}g^{-1}) 
                        \cdots (g{s_m}^{\epsilon_m}g^{-1}). &[(a)]
               \end{align*}
               By assumption $g{s_i}g^{-1} \in N$, and by closure of $N$ and 
               Lemma \ref{1_1_22_1}, it follows that
               $(g{s_i}g^{-1})^{\epsilon_i} = g{s_i}^{\epsilon_i}g^{-1} \in N$,
               for each $1 \le i \le m$, so that $gng^{-1} \in N$ by closure of 
               $N$. Thus $gng^{-1} \in N$ by closure of $N$ under 
               multiplication, so that $N \trianglelefteq G$ by Theorem 6(5).
         \item Suppose $N = \cyc{x}$.

               ($\Leftarrow$) Suppose $gxg^{-1} = x^k$ for each $g \in G$ and
               some $k \in \Z$. That is, $g\{x\}g^{-1} \subseteq N$ for each
               $g \in G$. Conclude from (c) that $N$ is normal in $G$.

               ($\Rightarrow$) Suppose $N \trianglelefteq G$. Let $g \in G$.
               Then by normality, it follows that $gxg^{-1} \in \cyc{x}$. That
               is, $gxg^{-1} = x^k$ for some integer $k$.
         \item Let $S = \{g \in G : |g| = n\}$ and $N = \cyc{S}$. If $S$ is
               empty, then $N = 1$ is normal in $G$, so suppose
               $S \neq \emptyset$. Let $g \in G$ and $s \in S$. We have that
               $|gsg^{-1}| = |s| = n$ by (a), so that $gsg^{-1} \in S$. Since
               $S \subseteq N$, it follows that $gsg^{-1} \in N$, so that
               $gSg^{-1} \subseteq N$. Conclude from (c) that $N$ is normal in
               $G$.
      \end{enumerate} \qed
%%%%%%%%%%%%%%%%%%%%%%%%%%%%%%%%%%%%%3.1.27%%%%%%%%%%%%%%%%%%%%%%%%%%%%%%%%%%%%%
   \item[3.1.27]  Let $N$ be a \textit{finite} subgroup of a group $G$. Show
                  that $gNg^{-1} \subseteq N$ if and only if $gNg^{-1} = N$.
                  Deduce $N_G(N) = \{g \in G : gNg^{-1} \subseteq N\}$.

      \textbf{Proof.} Suppose first that $gNg^{-1} \subseteq N$ for some
      $g \in G$. Consider the map
      $$\alpha_g : N \rightarrow gNg^{-1}, \quad n \mapsto gng^{-1}.$$
      The map $\alpha_g$ is injective because it has a left inverse:
      $\alpha_{g^{-1}}$, and since $N$ is finite, it follows that $\alpha_g$ is 
      bijective (Proposition 0.1 (1) and (4)). Thus $|N| = |gNg^{-1}|$. Since 
      $gNg^{-1} \subseteq N$, conclude that $gNg^{-1} = N$. The other direction
      follows from the definition of equality of sets. Thus
      $$N_G(N) = \{g \in G : gNg^{-1} = N\} =
        \{g \in G : gNg^{-1} \subseteq N\}.$$
%%%%%%%%%%%%%%%%%%%%%%%%%%%%%%%%%%%%%3.1.28%%%%%%%%%%%%%%%%%%%%%%%%%%%%%%%%%%%%%
   \item[3.1.28]  Let $N$ be a \textit{finite} subgroup of a group $G$ and
                  assume $N = \cyc{S}$ for some subset $S$ of $G$. Prove that an
                  element $g \in G$ normalizes $N$ if and only if
                  $gSg^{-1} \subseteq N$.

      \textbf{Proof.} Suppose $S$ is nonempty (otherwise, the problem would be
      trivial). Suppose some element $g \in G$ normalizes $N$. Then
      $gNg^{-1} = N$; particularly, $gNg^{-1} \subseteq N$. Since
      $S \subseteq N$, it follows that $gSg^{-1} \subseteq N$. Now suppose that
      $gSg^{-1} \subseteq N$ for some $g \in G$. Let $n \in N$. Then
      $n = {s_1}^{\epsilon_1}{s_2}^{\epsilon_2} \cdots{s_m}^{\epsilon_m}$, where 
      $m \in \Z^+$, and for each $1 \le i \le m$,
      $\epsilon_i \in \Z$ and $s_i \in S$. Now
      \begin{align*}
         gng^{-1} &= g({s_1}^{\epsilon_1}{s_2}^{\epsilon_2} \cdots
            {s_m}^{\epsilon_m})g^{-1} \\
            &= (g{s_1}^{\epsilon_1}g^{-1})\cdot (g{s_2}^{\epsilon_2}g^{-1})
               \cdots(g{s_m}^{\epsilon_m}g^{-1}). &[\text{Exercise 26(a)}]
      \end{align*}
      By assumption $g{s_i}g^{-1} \in N$, and by closure of $N$ and Lemma
      \ref{1_1_22_1}, it follows that
      $(g{s_i}g^{-1})^{\epsilon_i} = g{s_i}^{\epsilon_i}g^{-1} \in N$,
      for each $1 \le i \le m$, so that $gng^{-1} \in N$ by closure of $N$.
      That is, $gNg^{-1} \subseteq N$. Conclude by Exercise 27 that $g$ 
      normalizes $N$. \qed
%%%%%%%%%%%%%%%%%%%%%%%%%%%%%%%%%%%%%3.1.29%%%%%%%%%%%%%%%%%%%%%%%%%%%%%%%%%%%%%
   \item[3.1.29]  Let $N$ be a \textit{finite} subgroup of a group $G$ and
                  suppose $G = \cyc{T}$ and $N = \cyc{S}$ for some subsets $S$
                  and $T$ of $G$. Prove that $N$ is normal in $G$ if and only if
                  $tSt^{-1} \subseteq N$ for all $t \in T$.

      \textbf{Proof.} If $T$ or $S$ is empty, then the problem is trivial, so
      assume that both sets are nonempty. Suppose first that $N$ is normal in
      $G$. Let $t \in T$. Since $T$ generates $G$, it follows particularly that
      $t \in G$, so that $t$ normalizes $N$. Thus $tSt^{-1} \subseteq N$ by
      Exercise 28. Conversely, suppose that $tSt^{-1} \subseteq N$ for each
      $t \in T$. Since $G = \cyc{T}$, it follows that $T \subseteq G$, so that
      each $t\in T$ normalizes $N$ by Exercise 28. Thus $T \subseteq N_G(N)$, 
      and since $N_G(N)$ is a group, it follows by closure that
      $G = \cyc{T} \subseteq N_G(N)$. But $N_G(N) \subseteq G$ by definition;
      thus $G = N_G(N)$; that is, $N \trianglelefteq G$. \qed
%%%%%%%%%%%%%%%%%%%%%%%%%%%%%%%%%%%%%3.1.30%%%%%%%%%%%%%%%%%%%%%%%%%%%%%%%%%%%%%
   \item[3.1.30]  Let $N \le G$ and let $g \in G$. Prove that $gN = Ng$ if and
                  only if $g \in N_G(N)$.
%%%%%%%%%%%%%%%%%%%%%%%%%%%%%%%%%%%%%3.1.31%%%%%%%%%%%%%%%%%%%%%%%%%%%%%%%%%%%%%
   \item[3.1.31]  Prove that if $H \le G$ and $N$ is a normal subgroup of $H$
                  then $H \le N_G(H)$. Deduce that $N_G(N)$ is the largest
                  subgroup of $G$ in which $N$ is normal (i.e., is the join of
                  all subgroups $H$ for which $N \trianglelefteq H$).
%%%%%%%%%%%%%%%%%%%%%%%%%%%%%%%%%%%%%3.1.32%%%%%%%%%%%%%%%%%%%%%%%%%%%%%%%%%%%%%
   \item[3.1.32]  Prove that every subgroup of $Q_8$ is normal. For each
                  subgroup find the isomorphism type of its corresponding
                  quotient. [You may use the lattice of subgroups for $Q_8$ in
                  Section 2.5]
%%%%%%%%%%%%%%%%%%%%%%%%%%%%%%%%%%%%%3.1.33%%%%%%%%%%%%%%%%%%%%%%%%%%%%%%%%%%%%%
   \item[3.1.33]  Find all normal subgroups of $D_8$ and for each find the
                  isomorphism type of its corresponding quotient. [You may use
                  the lattice of subgroups for $D_8$ in Section 2.5.]
%%%%%%%%%%%%%%%%%%%%%%%%%%%%%%%%%%%%%3.1.34%%%%%%%%%%%%%%%%%%%%%%%%%%%%%%%%%%%%%
   \item[3.1.34]  Let $D_{2n} = \cyc{r, s : r^n = s^2 = 1, rs = sr^{-1}}$ be the
                  usual presentation of the dihedral group of order $2n$ and let
                  $k$ be a positive integer dividing $n$.
                  \begin{enumerate}
                     \item Prove that $\cyc{r^k}$ is a normal subgroup of
                           $D_{2n}$.
                     \item Prove that $D_{2n}/\cyc{r^k} \cong D_{2k}$.
                  \end{enumerate}
%%%%%%%%%%%%%%%%%%%%%%%%%%%%%%%%%%%%%3.1.35%%%%%%%%%%%%%%%%%%%%%%%%%%%%%%%%%%%%%
   \item[3.1.35]  Prove that $SL_n(F) \trianglelefteq GL_n(F)$ and describe the
                  isomorphism type of the quotient group.
%%%%%%%%%%%%%%%%%%%%%%%%%%%%%%%%%%%%%3.1.36%%%%%%%%%%%%%%%%%%%%%%%%%%%%%%%%%%%%%
   \item[3.1.36]  Prove that if $G/Z(G)$ is cyclic then $G$ is abelian. [If
                  $G/Z(G)$ is cyclic with generator $xZ(G)$, show that every
                  element of $G$ can be written in the form $x^az$ for some
                  integer $a \in \Z$ and some element $z \in Z(G)$.]
%%%%%%%%%%%%%%%%%%%%%%%%%%%%%%%%%%%%%3.1.37%%%%%%%%%%%%%%%%%%%%%%%%%%%%%%%%%%%%%
   \item[3.1.37]  Let $A$ and $B$ be groups. Show that $\{(a, 1) : a \in A\}$ is
                  a normal subgroup of $A \times B$ and the quotient of
                  $A \times B$ by this subgroup is isomorphic to $B$.
%%%%%%%%%%%%%%%%%%%%%%%%%%%%%%%%%%%%%3.1.38%%%%%%%%%%%%%%%%%%%%%%%%%%%%%%%%%%%%%
   \item[3.1.38]  Let $A$ be an abelian group and let $D$ be the (diagonal)
                  subgroup $\{(a, a) : a \in A\}$ of $A \times A$. Prove that
                  $D$ is a normal subgroup of $A \times A$ and
                  $(A \times A)/D \cong A$.
%%%%%%%%%%%%%%%%%%%%%%%%%%%%%%%%%%%%%3.1.39%%%%%%%%%%%%%%%%%%%%%%%%%%%%%%%%%%%%%
   \item[3.1.39]  Suppose $A$ is the non-abelian group $S_3$ and $D$ is the
                  diagonal subgroup $\{(a, a) : a \in A\}$ of $A \times A$.
                  Prove that $D$ is not normal in $A \times A$.
%%%%%%%%%%%%%%%%%%%%%%%%%%%%%%%%%%%%%3.1.40%%%%%%%%%%%%%%%%%%%%%%%%%%%%%%%%%%%%%
   \item[3.1.40]  Let $G$ be a group, let $N$ be a normal subgroup of $G$ and
                  let $\m{G} = G/N$. Prove that $\m{x}$ and $\m{y}$ commute in
                  $\m{G}$ if and only if $x^{-1}y^{-1}xy \in N$. (The element
                  $x^{-1}y^{-1}xy$ is called the \textit{commutator} of $x$ and
                  $y$ and is denoted by $[x, y]$.)
%%%%%%%%%%%%%%%%%%%%%%%%%%%%%%%%%%%%%3.1.41%%%%%%%%%%%%%%%%%%%%%%%%%%%%%%%%%%%%%
   \item[3.1.41]  Let $G$ be a group. Prove that
                  $N = \cyc{x^{-1}y^{-1}xy : x, y \in G}$ is a normal sugroup of
                  $G$ and $G/N$ is abelian ($N$ is called the
                  \textit{commutator subgroup} of $G$).
%%%%%%%%%%%%%%%%%%%%%%%%%%%%%%%%%%%%%3.1.42%%%%%%%%%%%%%%%%%%%%%%%%%%%%%%%%%%%%%
   \item[3.1.42]  Assume both $H$ and $K$ are normal subgroups of $G$ with
                  $H \cap K = 1$. Prove that $xy = yx$ for all $x \in H$ and
                  $y \in K$. [Show that $x^{-1}y^{-1}xy \in H \cap K$.]
%%%%%%%%%%%%%%%%%%%%%%%%%%%%%%%%%%%%%3.1.43%%%%%%%%%%%%%%%%%%%%%%%%%%%%%%%%%%%%%
   \item[3.1.43]  Assume that $\mathcal{P} = \{A_i : i \in I\}$ is any partition
                  of $G$ with the property that $\mathcal{P}$ is a group under
                  the ``quotienet operation" defined as follows: to compute the
                  product of $A_i$ with $A_j$ take any element $a_i$ of $A_i$
                  and any element $a_j$ of $A_j$ and let $A_iA_j$ be the element
                  of $\mathcal{P}$ containing $a_ia_j$ (this operation is
                  assumed to be well defined). Prove that the element of
                  $\mathcal{P}$ that contains the identity of $G$ is a normal
                  subgroup of $G$ and the elements of $\mathcal{P}$ are the
                  cosets of this subgroup (so $\mathcal{P}$ is just a quotient
                  group of $G$ in the usual sense.)
\end{enumerate}
