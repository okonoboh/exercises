Let $G$ a group.

\begin{enumerate}
%%%%%%%%%%%%%%%%%%%%%%%%%%%%%%%%%%%%%3.2.1%%%%%%%%%%%%%%%%%%%%%%%%%%%%%%%%%%$%%%
   \item[3.2.1]   Which of the following are permissible orders for subgroups of
                  a group of order 120: 1, 2, 5, 7, 9, 15, 60, 240? For each
                  permissible order give the corresponding index.

      \textbf{Solution.} Lagrange Theorem says that the permissible orders for
      this group are: 1, 2, 5, 15, 60, with corresponding index 120, 60, 24, 8,
      2 respectively.
%%%%%%%%%%%%%%%%%%%%%%%%%%%%%%%%%%%%%3.2.2%%%%%%%%%%%%%%%%%%%%%%%%%%%%%%%%%%%%%%
   \item[3.2.2]   Prove that the lattice of subgroups of $S_3$ in Section 2.5 is
                  correct (i.e., prove that it contains all subgroups of $S_3$
                  and that their pairwise joins and intersections are correctly  
                  drawn).

      \textbf{Proof.} Let $H$ be a nontrivial proper subgroup of $S_3$. Since
      $|S_3| = 6$, it follows by Lagrange's Theorem that $|H| \in \{2, 3\}$. If
      $|H| = 2$, then it follows by Lagrange's Theorem that the only nonidentity
      element in $H$ must have order 2. That is, $|H| = 2$ if and only if
      $H \in \{\cyc{(1\;2)}, \cyc{(2\;3)}, \cyc{(1\;3)}\}$. 
      Similarly, if $|H| = 3$, then $H$ contains an element of order 3. That is, 
      at least one of (1 2 3) and (1 3 2) is an element of $H$. Since
      $(1\;2\;3)^2 = (1\;3\;2)$ and $(1\;3\;2)^2 = (1\;2\;3)$, it follows that 
      both elements are in $H$. That is, $H = \cyc{(1\;2\;3)}$. Observe that the 
      intersection of any two nontrivial proper subgroups of $S_3$ is trivial.
      Since $S_3 = \cyc{(1\;2), (1\;2\;3)}$ and since the group generated by any
      two nontrivial subgroups of $S_3$ contains (1 2) and (1 2 3), we conclude 
      that the lattice of $S_3$ is correct. \qed
%%%%%%%%%%%%%%%%%%%%%%%%%%%%%%%%%%%%%3.2.3%%%%%%%%%%%%%%%%%%%%%%%%%%%%%%%%%%%%%%
   \item[3.2.3]   Prove that the lattice of subgroups of $Q_8$ in Section 2.5 is
                  correct.

      \textbf{Proof.} Let $H$ be a nontrivial proper subgroup of $Q_8$. By
      Langrange's Theorem, $|H| = 2$ or $|H| = 4$. Since $-1$ is the only
      element of order 2 in $Q_8$, it follows that if $|H| = 2$, then $H$ must
      be the subgroup $\cyc{-1}$. Now suppose $|H| = 4$. Let
      $x \in S = \{i, j, k\}$. Then $x \in H$ or $-x \in H$ if and only if
      $H = \cyc{x} = \cyc{-x}$. Thus $H \in \{\cyc{i}, \cyc{j}, \cyc{k}\}$. So
      the intersection of any two subgroups of order 4 in $Q_8$ is $\cyc{-1}$.
      Observe that any subgroup that contains two elements of $S$ must generate 
      $Q_8$ because $Q_8 = \cyc{i, j} = \cyc{j, k} = \cyc{i, k}$. Thus the join
      of any two subgroups of order 4 in $Q_8$ is $Q_8$. Conclude that the 
      lattice of $Q_8$ is correct. \qed
%%%%%%%%%%%%%%%%%%%%%%%%%%%%%%%%%%%%%3.2.4%%%%%%%%%%%%%%%%%%%%%%%%%%%%%%%%%%%%%%
   \item[3.2.4]   Show that if $|G| = pq$ for some primes $p$ and $q$ (not
                  necessarily distinct) then either $G$ is abelian or
                  $Z(G) = 1$. [See Exercise 36 in Section 1.]

      \textbf{Proof.} Suppose $|G| = pq$, for some primes $p$ and $q$. If
      $Z(G)$ is trivial, we are done, so assume $|Z(G)| > 1$. It follows by 
      Lagrange's Theorem that $|Z(G)| \in \{p, q, pq\}$. Thus
      $|G/Z(G)| \in \{1, p, q\}$, so that $G/Z(G)$ is cyclic by Corollary 3.10.
      Conclude by Exercise 3.1.36 that $G$ is abelian. \qed
%%%%%%%%%%%%%%%%%%%%%%%%%%%%%%%%%%%%%3.2.5%%%%%%%%%%%%%%%%%%%%%%%%%%%%%%%%%%%%%%
   \item[3.2.5]   Let $H$ be a subgroup of $G$ and fix some element $g \in G$.
                  \begin{enumerate}
                     \item Prove that $gHg^{-1}$ is a subgroup of $G$ of the
                           same order as $H$.
                     \item Deduce that if $n \in \Z^+$ and $H$ is the unique
                           subgroup of $G$ of order $n$ then
                           $H \trianglelefteq G$.
                  \end{enumerate}
                  
      \textbf{Proof.}
      
      \begin{enumerate}
         \item The map $\varphi : G \rightarrow G$, defined by
               $x \mapsto gxg^{-1}$ is an automorphism of $G$ and, according to
               Exercise 1.7.17. This same exercise says that $|H| = |gHg^{-1}|$.
               Let $x, y \in gHg^{-1}$.
      \end{enumerate}
%%%%%%%%%%%%%%%%%%%%%%%%%%%%%%%%%%%%%3.2.6%%%%%%%%%%%%%%%%%%%%%%%%%%%%%%%%%%%%%%
   \item[3.2.6]   Let $H \le G$ and let $g \in G$. Prove that if the right coset
                  $Hg$ equals \textit{some} left coset of $H$ in $G$ then it
                  equals the left coset $gH$ and $g$ must be in $N_G(H)$.
%%%%%%%%%%%%%%%%%%%%%%%%%%%%%%%%%%%%%3.2.7%%%%%%%%%%%%%%%%%%%%%%%%%%%%%%%%%%%%%%
   \item[3.2.7]   Let $H \le G$ and define a relation $\sim$ on $G$ by
                  $a \sim b$ if and only if $b^{-1}a \in H$. Prove that $\sim$
                  is an equivalence relation and describe the equivalence class
                  of each $a \in G$. Use this to prove Proposition 4.
%%%%%%%%%%%%%%%%%%%%%%%%%%%%%%%%%%%%%3.2.8%%%%%%%%%%%%%%%%%%%%%%%%%%%%%%%%%%%%%%
   \item[3.2.8]   Prove that if $H$ and $K$ are finite subgroups of $G$ whose
                  orders are relatively prime then $H \cap K = 1$.
%%%%%%%%%%%%%%%%%%%%%%%%%%%%%%%%%%%%%3.2.9%%%%%%%%%%%%%%%%%%%%%%%%%%%%%%%%%%%%%%
   \item[3.2.9]   This exercise outlines a proof of Cauchy's Theorem due to
                  James McKay (\textit{Another proof of Cauchy's group theorem},
                  Amer. Math. Monthly, 66(1959), p. 119). Let $G$ be a finite
                  group and let $p$ be a prime dividing $|G|$. Let $\mathcal{S}$
                  denote the set of $p$-tuples of elements of $G$ the product of
                  whose coordinates is 1:
                  $$\mathcal{S} = \{(x_1, x_2, \ldots, x_p) : x_i \in G
                    \text{ and } x_1x_2 \cdots x_p = 1\}.$$
                  \begin{enumerate}
                     \item Show that $\mathcal{S}$ has $|G|^{p-1}$ elements,
                           hence has order divisible by $p$.

                           Define the relation on $\sim$ on $\mathcal{S}$ by
                           letting $\alpha \sim \beta$ if $\beta$ is a cyclic
                           permuation of $\alpha$.
                     \item Show that a cyclic permutation of an element of
                           $\mathcal{S}$ is again an element of $\mathcal{S}$.
                     \item Prove that $\sim$ is an equiavalence relation on
                           $\mathcal{S}$.
                     \item Prove that an equivalence class contains a single
                           element if and only if it is of the form
                           $(x, x, \ldots, x)$ with $x^p = 1$.
                     \item Prove that every equivalence class has order 1 or $p$
                           (this uses the fact that $p$ is a \textit{prime}).
                           Deduce that $|G|^{p-1} = k + pd$, where $k$ is the
                           number of classes of size 1 and $d$ is the number of
                           classes of size $p$.
                     \item Since $\{(1, 1, \ldots, 1)\}$ is an equivalence class
                           of size 1, conclude from (e) that there must be a
                           nonidentity element $x$ in $G$ with $x^p = 1$, i.e.,
                           $G$ contains an element of order $p$. [Show
                           $p \mid k$ and so $k > 1$.]
                  \end{enumerate}                  
%%%%%%%%%%%%%%%%%%%%%%%%%%%%%%%%%%%%%3.2.10%%%%%%%%%%%%%%%%%%%%%%%%%%%%%%%%%%%%%
   \item[3.2.10]  Suppose $H$ and $K$ are subgroups of finite index in the
                  (possibly infinite) group $G$ with $|G : H| = m$ and
                  $|G : K| = n$. Prove that lcm($m, n) \le
                  |G : H \cap K| \le mn$. Deduce that if $m$ and $n$ are
                  relatively prime then $|G : H \cap K| =|G : H| \cdot |G : K|$.
%%%%%%%%%%%%%%%%%%%%%%%%%%%%%%%%%%%%%3.2.11%%%%%%%%%%%%%%%%%%%%%%%%%%%%%%%%%%%%%
   \item[3.2.11]  Let $H \le K \le G$. Prove that
                  $|G : H| = |G : K| \cdot |K : H|$ (do not assume $G$ is
                  finite).
%%%%%%%%%%%%%%%%%%%%%%%%%%%%%%%%%%%%%3.2.12%%%%%%%%%%%%%%%%%%%%%%%%%%%%%%%%%%%%%
   \item[3.2.12]  Let $H \le G$. Prove that map $x \mapsto x^{-1}$ sends each
                  left coset of $H$ in $G$ onto a right coset of $H$ and gives a
                  bijection between the set of left cosets and the set of right
                  cosets of $H$ in $G$ (hence the number of left cosets of $H$
                  in $G$ equals the number of right cosets).
%%%%%%%%%%%%%%%%%%%%%%%%%%%%%%%%%%%%%3.2.13%%%%%%%%%%%%%%%%%%%%%%%%%%%%%%%%%%%%%
   \item[3.2.13]  Fix any labelling of the vertices of a square and use this to
                  identity $D_8$ as a subgroup of $S_4$. Prove that the elements
                  of $D_8$ and $\cyc{(1\;2\;3)}$ do not commute in $S_4$.
%%%%%%%%%%%%%%%%%%%%%%%%%%%%%%%%%%%%%3.2.14%%%%%%%%%%%%%%%%%%%%%%%%%%%%%%%%%%%%%
   \item[3.2.14]  Prove that $S_4$ does not have a normal subgroup of order 8 or
                  a normal subgroup of order 3.
%%%%%%%%%%%%%%%%%%%%%%%%%%%%%%%%%%%%%3.2.15%%%%%%%%%%%%%%%%%%%%%%%%%%%%%%%%%%%%%
   \item[3.2.15]  Let $G = S_n$ and for fixed $i \in \{1, 2, \ldots, n\}$ let
                  $G_i$ be the stabilizer of $i$. Prove that
                  $G_i \cong S_{n-1}$.
%%%%%%%%%%%%%%%%%%%%%%%%%%%%%%%%%%%%%3.2.16%%%%%%%%%%%%%%%%%%%%%%%%%%%%%%%%%%%%%
   \item[3.2.16]  Use Lagrange's Theorem in the multiplicative group
                  $(\Z/p\Z)^\times$ to prove \textit{Fermat's Little Theorem}:
                  if $p$ is a prime then $a^p \equiv a$ (mod $p$) for all
                  $a \in \Z$.
%%%%%%%%%%%%%%%%%%%%%%%%%%%%%%%%%%%%%3.2.17%%%%%%%%%%%%%%%%%%%%%%%%%%%%%%%%%%%%%
   \item[3.2.17]  Let $p$ be a prime and let $n$ be a positive integer. Find the
                  order of $\m{p}$ in $(\Z/(p^n-1)\Z)^\times$ and deduce that
                  $n \mid \varphi(p^n - 1)$ (here $\varphi$ is Euler's
                  function).
%%%%%%%%%%%%%%%%%%%%%%%%%%%%%%%%%%%%%3.2.18%%%%%%%%%%%%%%%%%%%%%%%%%%%%%%%%%%%%%
   \item[3.2.18]  Let $G$ be a finite group, let $H$ be a subgroup of $G$ and
                  let $N \trianglelefteq G$. Prove that if $|H|$ and $|G : N|$
                  are relatively prime then $H \le N$.
%%%%%%%%%%%%%%%%%%%%%%%%%%%%%%%%%%%%%3.2.19%%%%%%%%%%%%%%%%%%%%%%%%%%%%%%%%%%%%%
   \item[3.2.19]  Prove that if $N$ is a normal subgroup of the finite group $G$
                  and $(|N|, |G : N|) = 1$, then $N$ is the unique subgroup of
                  $G$ of order $|N|$.
%%%%%%%%%%%%%%%%%%%%%%%%%%%%%%%%%%%%%3.2.20%%%%%%%%%%%%%%%%%%%%%%%%%%%%%%%%%%%%%
   \item[3.2.20]  If $A$ is an abelian group with $A \trianglelefteq G$ and $B$
                  is any subgroup of $G$ prove that
                  $A \cap B \trianglelefteq AB$.
%%%%%%%%%%%%%%%%%%%%%%%%%%%%%%%%%%%%%3.2.21%%%%%%%%%%%%%%%%%%%%%%%%%%%%%%%%%%%%%
   \item[3.2.21]  Prove that $\Q$ has no proper subgroups of finite index.
                  Deduce that $\Q/\Z$ has no proper subgroups of finite index.
                  [Recall Exercises 1.6.21 and 3.1.15]
%%%%%%%%%%%%%%%%%%%%%%%%%%%%%%%%%%%%%3.2.22%%%%%%%%%%%%%%%%%%%%%%%%%%%%%%%%%%%%%
   \item[3.2.22]  Use Lagrange's Theorem in the multiplicative group
                  $(\Z/n\Z)^\times$ to prove \textit{Euler's Theorem}:
                  $a^{\varphi(n) \equiv 1}$ mod $n$ for every integer $a$
                  relatively prime to $n$, where $\varphi$ denotes Euler's
                  $\varphi$-function.
%%%%%%%%%%%%%%%%%%%%%%%%%%%%%%%%%%%%%3.2.23%%%%%%%%%%%%%%%%%%%%%%%%%%%%%%%%%%%%%
   \item[3.2.23]  Determine the last two digits of $3^{3^{100}}$. [Determine
                  $3^{100}$ mod $\varphi(100)$ and use the previous exercise.]
\end{enumerate}
