Let $R$ be a commutative ring with identity $1 \neq 0$.
\begin{enumerate}
%%%%%%%%%%%%%%%%%%%%%%%%%%%%%%%%%%%%%7.5.1%%%%%%%%%%%%%%%%%%%%%%%%%%%%%%%%%%%%%%
   \item[7.5.1]   Fill in all the details in the proof of Theorem 15.
%%%%%%%%%%%%%%%%%%%%%%%%%%%%%%%%%%%%%7.5.2%%%%%%%%%%%%%%%%%%%%%%%%%%%%%%%%%%%%%%
   \item[7.5.2]   Let $R$ be an integral domain and let $D$ be a nonempty subset
                  of $R$ that is closed under multiplication. Prove that the
                  ring of fractions $D^{-1}R$ is isomorphic to a subring of the
                  quotient field of $R$ (hence is also an integral domain).
%%%%%%%%%%%%%%%%%%%%%%%%%%%%%%%%%%%%%7.5.3%%%%%%%%%%%%%%%%%%%%%%%%%%%%%%%%%%%%%%
   \item[7.5.3]   Let $F$ be a field. Prove that $F$ contains a unique smallest
                  subfield $F_0$ and that $F_0$ is isomorphic to either $\Q$ or
                  $\Z/p\Z$ for some prime $p$ ($F_0$ is called the
                  \textit{prime subfield} of $F$. [See Exercuse 7.3.26.]
%%%%%%%%%%%%%%%%%%%%%%%%%%%%%%%%%%%%%7.5.4%%%%%%%%%%%%%%%%%%%%%%%%%%%%%%%%%%%%%%
   \item[7.5.4]  Prove that any subfield of $\R$ must contain $\Q$.
%%%%%%%%%%%%%%%%%%%%%%%%%%%%%%%%%%%%%7.5.5%%%%%%%%%%%%%%%%%%%%%%%%%%%%%%%%%%%%%%
   \item[7.5.5]   If $F$ is a field, prove that the field of fractions of
                  $F[[x]]$ (the ring of formal power series in the indeterminate
                  $x$ with coefficients in $F$) is the ring $F((x))$ of formal
                  Laurent series (cf. Exercises 7.2.3 and 7.2.5). Show that the
                  field of fractions of the power series ring $\Z[[x]]$ is
                  \textit{properly} contained in the field of Laurent series
                  $\Q((x))$. [Consider the series for $e^x$.]
%%%%%%%%%%%%%%%%%%%%%%%%%%%%%%%%%%%%%7.5.6%%%%%%%%%%%%%%%%%%%%%%%%%%%%%%%%%%%%%%
   \item[7.5.6]   Prove that the real numbers, $\R$, contain a subring $A$ with
                  $1 \in A$ and $A$ maximal (under inclusion) with respect to
                  the property that $\frac{1}{2} \notin A$. [Use Zorn's Lemma.]
                  (Exercise 15.3.13 shows $\R$ is the quotient field of $A$, so
                  $\R$ is the quotient field of a proper subring.)
\end{enumerate}
