Let $R$ be a ring with identity $1 \neq 0$.

%---7.4.3
%---Finish 7.4.22(a)
\begin{enumerate}
%%%%%%%%%%%%%%%%%%%%%%%%%%%%%%%%%%%%%7.4.1%%%%%%%%%%%%%%%%%%%%%%%%%%%%%%%%%%%%%%
   \item[7.4.1]   Let $L_j$ be the left ideal of $M_n(R)$ consisting of
                  arbitrary entries in the $j^{\text{th}}$ column and zero in
                  all other entries and let $E_{ij}$ be the element of $M_n(R)$
                  whose $i, j$ entry is 1 and whose other entries are all 0. 
                  Prove that $L_j = M_n(R)E_{ij}$ for any $i$.
                  [See Exercise 7.2.6]

      \textbf{Proof.} Fix $i, j \in \{1, 2, \ldots, n\}$, where $n \in \Z^+$.

      \begin{itemize}
         \item ($\subseteq$) Let $A \in L_j$. By definition of $L_j$, the
               $j^{\text{th}}$ column of $A$ is arbitrary and every other entry 
               is zero. Thus it follows by Exercise 7.2.6(b) that
               $A = (A\cdot E_{ji})E_{ij}$, so that
               $L_j \subseteq M_n(R)E_{ij}$.

         \item ($\supseteq$) It follows immediately from Exercise 7.2.6(b) that
               $L_j \supseteq M_n(R)E_{ij}$.
      \end{itemize}
      Thus conclude that $L_j = M_n(R)E_{ij}$. \qed
%%%%%%%%%%%%%%%%%%%%%%%%%%%%%%%%%%%%%7.4.2%%%%%%%%%%%%%%%%%%%%%%%%%%%%%%%%%%%%%%
   \item[7.4.2]   Assume $R$ is commutative. Prove that the augmentation ideal
                  in the group ring $RG$ is generated by $\{g - 1 : g \in G\}$.
                  Prove that if $G = \cyc{\sigma}$ is cyclic then the 
                  augmentation ideal is generated by $\sigma - 1$.

      \textbf{Proof.} Let $G = \{g_1, g_2, \ldots, g_n\}$,
      $I = (g_1 - 1, g_2 - 1, \ldots, g_n - 1)$, and let $K$ be the augmentation 
      ideal of $RG$. First we want to show that $K = I$.

      \begin{itemize}
         \item ($\subseteq$) Let $r \in K$. Then $r = r_1g_1 + \cdots + r_ng_n$
               for some $r_1, \ldots, r_n \in R$. Since $r \in K$, it follows
               that $r_1 + \cdots + r_n = 0$. Thus
               \begin{align*}
                  r &= r_1g_1 + \cdots + r_ng_n - 0 \\
                    &= r_1g_1 + \cdots + r_ng_n - (r_1 + \cdots + r_n) \\
                    &= r_1g_1 - r_1 + \cdots + r_ng_n - r_n \\
                    &= r_1(g_1 - 1) + \cdots + r_n(g_n - 1) \in I,
               \end{align*}
               so that $K \subseteq I$.

         \item ($\supseteq$) To show that $K \supseteq I$, it suffices to show
               that $K$ contains the generators of $I$. But this follows
               immediately since the sum of the coefficients of the generators
               equal 0, so that $g_i - 1 \in K$ for $1 \le i \le n$. That is,
               $K \supseteq I$.
      \end{itemize}
      We conclude that $K = I$. Now suppose that $G = \cyc{\sigma}$ is cyclic, 
      where $|\sigma| = m \in \Z^+$. As we showed above,
      $K = (\sigma - 1, \sigma^2 - 1, \ldots, \sigma^{m-1} - 1)$. Consider one
      of the generators of $K$, say $\sigma^i - 1$, where $1 \le i \le m - 1$.
      Since
      $$(\sigma^i - 1) = (\sigma^{i-1} + \sigma^{i-2} + \cdots + \sigma + 1)
        (\sigma - 1),$$
      it follows that every generator of $K$ is an element of the ideal
      generated by $\sigma - 1$. Hence that $K = (\sigma - 1)$. \qed
%%%%%%%%%%%%%%%%%%%%%%%%%%%%%%%%%%%%%7.4.3%%%%%%%%%%%%%%%%%%%%%%%%%%%%%%%%%%%%%%
   \item[7.4.3]   \begin{enumerate}
                     \item Let $p$ be a prime and let $G$ be an abelian group of
                           order $p^n$. Prove that the nilradical of the group
                           ring $\F_pG$ is the augmentation ideal
                           (cf. Exercise 7.3.29). [Use the preceding exercise.]
                     \item Let $G = \{g_1, \ldots, g_n\}$ be the finite group
                           and assume $R$ is commutative. Prove that if $r$ is
                           any element of the augmentation ideal of $RG$ then
                           $r(g_1 + \cdots + g_n) = 0$. [Use the preceding
                           exercise.]
                  \end{enumerate}
%%%%%%%%%%%%%%%%%%%%%%%%%%%%%%%%%%%%%7.4.4%%%%%%%%%%%%%%%%%%%%%%%%%%%%%%%%%%%%%%
   \item[7.4.4]   Assume $R$ is commutative. Prove that $R$ is a field if and
                  only if 0 is a maximal ideal.

      \textbf{Proof.}

      ($\Leftarrow$) Suppose that 0 is a maximal ideal of $R$. Let $I$ be a
      nonzero ideal of $R$. Now $I$ contains the zero ideal since every ideal
      contains the zero ideal; but since the zero ideal is maximal in $R$, it
      follows that $I = 0$ or $I = R$; thus the only ideals of $R$ are trivial,
      so that $R$ is a field by Proposition 7.9(2).

      ($\Rightarrow$) Suppose that $R$ is a field, so that, by Proposition
      7.9(2), its only ideals are 0 and R. That is, the only ideal properly
      containing 0 is $R$, so that 0 is a maximal ideal in $R$.

      \qed
%%%%%%%%%%%%%%%%%%%%%%%%%%%%%%%%%%%%%7.4.5%%%%%%%%%%%%%%%%%%%%%%%%%%%%%%%%%%%%%%
   \item[7.4.5]   Prove that if $M$ is an ideal such that $R/M$ is a field then
                  $M$ is a maximal ideal (do not asssume $R$ is commutative).

      \textbf{Proof.} Assume that $M$ is an ideal such that $R/M$ is a field.
      Let $I$ be an ideal of $R$ such that $M$ is a proper subset of $I$. It
      suffices to show that $I = R$. Since $M$ is properly contained in $I$,
      there exists $i \in I$ such that $i \notin M$. Thus $i + M$ is nonzero in
      $R/M$, and since $R/M$ is a field, $ij + M = 1 + M$ for some $j \in R$. 
      That is, $ij - 1 = m$ for some $m \in M$, or equivalently, $ij - m = 1$. 
      But since $I$ is an ideal of $R$ and $i \in I$, it follows that
      $ij \in I$; also since $M \subset I$, it follows that $1 = ij - m \in I$, 
      so that $I = R$ by Proposition 7.9(1). Thus $M$ is a maximal ideal in $R$. 
      \qed
%%%%%%%%%%%%%%%%%%%%%%%%%%%%%%%%%%%%%7.4.6%%%%%%%%%%%%%%%%%%%%%%%%%%%%%%%%%%%%%%
   \item[7.4.6]   Prove that $R$ is a division ring if and only if its only left
                  ideals are (0) and $R$. (The analogous result holds when
                  ``left" is replaced by ``right.")

      \textbf{Proof.}

      $(\Leftarrow)$ Suppose that the only left ideals of $R$ are (0) and $R$.
      Let $x$ be a nonzero element in $R$ and let $Rx$ be the left ideal
      generated by $x$. Since $x \neq 0$, it follows that $Rx \neq \{0\}$, so
      that $Rx = R$; thus $1 \in Rx$, so that $1 = yx$ for some nonzero
      $y \in R$. Similarly $Ry = R$, so that $zy = 1$ for some $z \in R$. Thus
      $z \cdot 1 = z(yx) = (zy)x = 1 \cdot x = x$. Thus $yx = xy = 1$; that is,
      every nonzero element in $R$ is a unit.

      $(\Rightarrow)$ Suppose that $R$ is a division ring. Let $x$ be a unit in
      $R$ and let $r \in R$. Then
      $$r = r \cdot 1 = r(x^{-1}x) = (rx^{-1})x \in Rx,$$
      so that $R \subseteq Rx$; inclusion in the other direction is trivial, so
      $R = Rx$. Thus the left ideal generated by every unit in $R$ is all of
      $R$. Conclude that the only left ideals of $R$ are $(0)$ and $R$. \qed
%%%%%%%%%%%%%%%%%%%%%%%%%%%%%%%%%%%%%7.4.7%%%%%%%%%%%%%%%%%%%%%%%%%%%%%%%%%%%%%%
   \item[7.4.7]   Let $R$ be a commutative ring with 1. Prove that the principal
                  ideal generated by $x$ in the polynomial ring $R[x]$ is a
                  prime ideal if and only if $R$ is an integral domain. Prove
                  that $(x)$ is a maximal ideal if and only if $R$ is a field.

      \textbf{Proof.} The map $\varphi : R[x] \rightarrow R$, defined by
      $r(x) \mapsto r(0)$, is a surjective homomorphism with kernel $(x)$; so
      it follows by the First Isomorphism Theorem for rings that
      $R[x]/(x) \cong R$. Proposition 7.13 says that $(x)$ is a prime ideal in
      $R[x]$ if and only if $R[x]/(x)$ is an integral domain. Since $R$ is 
      isomorphic to $R[x]/(x)$, conclude that $(x)$ is a prime ideal in $R[x]$
      if and only if $R$ is an integral domain. \qed
%%%%%%%%%%%%%%%%%%%%%%%%%%%%%%%%%%%%%7.4.8%%%%%%%%%%%%%%%%%%%%%%%%%%%%%%%%%%%%%%
   \item[7.4.8]   Let $R$ be an integral domain. Prove that $(a) = (b)$ for some
                  elements $a, b \in R$, if and only if $a = ub$ for some unit
                  $u$ of $R$.
                  
      \textbf{Proof.}
      
      $(\Rightarrow)$ Suppose $(a) = (b)$ for some elements $a, b \in R$. If
      $a = 0$, then $b = 0$, so $a = 1 \cdot b$; now assume that $a$ and $b$ are
      nonzero. Particularly, we have that $a \in (b)$ and $b \in (a)$. Thus
      $a = bx$ and $b = ay$ for some $x, y \in R$. So $a = bx = ayx$, so that
      $a(1 - yx) = 0$. Since $R$ is an integral domain and since $a \neq 0$, it
      follows that $1 - yx = 0$; that is, $xy = 1$, or equivalently, $x$ and
      $y$ are units in $R$.
      
      $(\Leftarrow)$ Suppose that $a = bu$ for some $a, b, u \in R$, where $u$
      is a unit. Now we want to show that $(a) = (b)$.
      \begin{itemize}
         \item ($\subseteq$) Let $x \in (a)$. Then $x = at$ for some $t \in R$,
               so that $x = at = (bu)t = b(ut) \in (b)$, and thus,
               $(a) \subseteq (b)$.
         \item ($\supseteq$) Let $y \in (b)$. Then $y = bs$ for some $s \in R$,
               so that $y = bs = (au^{-1})s = a(u^{-1}s) \in (a)$, and thus,
               $(a) \supseteq (b)$.
      \end{itemize}
      So we have that $(a) = (b)$.  \qed
%%%%%%%%%%%%%%%%%%%%%%%%%%%%%%%%%%%%%7.4.9%%%%%%%%%%%%%%%%%%%%%%%%%%%%%%%%%%%%%%
   \item[7.4.9]   Let $R$ be the ring of all continuous functions on $[0, 1]$
                  and let $I$ be the collection of functions $f(x)$ in $R$ with
                  $f(1/3) = f(1/2) = 0$. Prove that $I$ is an ideal of $R$ but
                  is not a prime ideal.
                  
      \textbf{Proof.} $I$ is nonempty because it contains the zero function, so
      let $f, g \in I$. First we will show that $I$ is an ideal of $R$.
      
      \begin{itemize}
         \item \textbf{closure under subtraction:} Since
               $$(f - g)(1/3) = f(1/3) - g(1/3) = 0 =
                 f(1/2) - g(1/2) = (f - g)(1/2),$$
               it follows that $f - g \in I$, so that $I$ is closed under
               subtraction.
         \item \textbf{closure under multiplication $R$:} Let $h \in R$. The
               function $h \cdot f$ is in $I$ because
               $$(h\cdot f)(1/3) = h(1/3)f(1/3) = 0 =
                  h(1/2)f(1/2) = (h\cdot f)(1/2).$$
                  
               Thus  $I$ is closed under multiplication by $R$.
      \end{itemize}
     
      Conclude that $I$ is an ideal of $R$. Now let $p(x) = 3x - 1$ and
      $r(x) = 2x - 1$ be functions in $R$. We have that $p(x)r(x) \in I$, but
      neither $p$ nor $r$ is in $I$; thus $I$ is not a prime ideal of $I$. \qed
%%%%%%%%%%%%%%%%%%%%%%%%%%%%%%%%%%%%%7.4.10%%%%%%%%%%%%%%%%%%%%%%%%%%%%%%%%%%%%%
   \item[7.4.10]  Assume $R$ is commutative. Prove that if $P$ is a prime ideal
                  of $R$ and $P$ contains no zero divisors then $R$ is an
                  integral domain.
                  
      \textbf{Proof.} Assume that $P$ has no zero divisors and is a prime ideal
      of $R$. Suppose $ab = 0$ for some $a, b \in R$. Then since $0 \in P$, it
      follows that $ab \in P$, so that $a = 0$ or $b = 0$ since $P$ has no zero
      divisors. \qed
%%%%%%%%%%%%%%%%%%%%%%%%%%%%%%%%%%%%%7.4.11%%%%%%%%%%%%%%%%%%%%%%%%%%%%%%%%%%%%%
   \item[7.4.11]  Assume $R$ is commutative. Let $I$ and $J$ be ideals of $R$
                  and assume $P$ is a prime ideal of $R$ that contains $IJ$ (for
                  example, if $P$ contains $I \cap J$). Prove either $I$ or $J$
                  is contained in $P$.
                  
      \textbf{Proof.} Suppose that $I$ is not contained in $P$. It suffices to
      show that $J \subseteq P$. Since $I \not\subseteq P$, there exists
      $i \in I$ such that $i \notin P$. Let $j \in J$. Since $P$ contains $IJ$,
      it follows that $ij\in P$. Since $P$ is a prime ideal of $R$, we have that
      $i \in P$ or $j \in P$; but $i \notin P$, so we conclude that $j \in P$.
      Thus, $P \supseteq J$. \qed
%%%%%%%%%%%%%%%%%%%%%%%%%%%%%%%%%%%%%7.4.12%%%%%%%%%%%%%%%%%%%%%%%%%%%%%%%%%%%%%
   \item[7.4.12]  Assume $R$ is commutative and suppose
                  $I = (a_1, a_2, \ldots, a_n)$ and
                  $J = (b_1, b_2, \ldots, b_m)$ are two finitely generated
                  ideals in $R$. Prove that the product ideal $IJ$ is finitely
                  generated by the elements $a_ib_j$ for $i = 1, 2, \ldots, n$,
                  and $j = 1, 2, \ldots, m$.

      \textbf{Proof.} Let $K = \{a_ib_j : i \in \{1, 2, \ldots, n\},
      j \in\{1, 2, \ldots, m\}\}$. We want to show that $IJ = (K)$. It follows
      immediately that $(K) \subseteq IJ$ because by the definition of $IJ$,
      elements of $K$---the generators of $(K)$---are elements of $IJ$, so it 
      suffices to show that $IJ \subseteq (K)$. To that end, let $r \in IJ$. It 
      follows that $r = u_1v_1 + u_2v_2 + \cdots + u_tv_t$, where $t \in \Z^+$, 
      $u_i \in I$, $v_i \in J$, with $1 \le i \le t$. Consider one of terms in 
      the finite sum $u_1v_1 + u_2v_2 + \cdots + u_tv_t$, say $u_1v_1$, without 
      loss of generality. Since $a_1, \ldots, a_n$ generate $I$ and
      $b_1, \ldots, b_m$ generate $J$, it follows that
      $$u_1 = r_1a_1 + r_2a_2 + \cdots + r_na_n$$
      and
      $$v_1 = s_1b_1 + s_2b_2 + \cdots + s_mb_m,$$
      where $r_i, s_j \in R$, with $i = 1, \ldots n$, and $j = 1, \ldots, m$.
      Now let us show by induction on $n$ that $u_1v_1 \in (K)$.

      \begin{itemize}
         \item \textbf{base case:} $n = 1$. We have that
               \begin{align*}
                  (r_1a_1)v_1 &= r_1a_1(s_1b_1 + s_2b_2 + \cdots + s_mb_m) \\
                     &= (r_1s_1)a_1b_1 + (r_1s_2)a_1b_2 + \cdots +
                        (r_1s_m)a_1b_m \in (K).
               \end{align*}
         \item \textbf{inductive hypothesis:} Now suppose that
               $(r_1a_1 + r_2a_2 + \cdots + r_{n-1}a_{n-1})v_1 \in (K)$,
               $n \ge 2$.
         \item \textbf{assertion holds for $n$:} Let
               $r' = r_1a_1 + r_2a_2 + \cdots + r_{n-1}a_{n-1}$, so that
               $r'v_1 \in (K)$ by the inductive hypothesis. The base case tells
               us that $(r_na_n)v_1 \in (K)$. Thus
               $$u_1v_1 = (r' + r_na_n)v_1 = r'v_1 + (r_na_n)v_1 \in (K).$$
      \end{itemize}
      Thus it follows by induction that $u_1v_1 \in (K)$ for all $n \ge 1$. We
      have shown that term in $u_1v_1 + u_2v_2 + \cdots + u_tv_t$ is a member of
      $(K)$; since $r = u_1v_1 + u_2v_2 + \cdots + u_tv_t$, it follows that
      $r \in (K)$; i.e., $IJ \subseteq (K)$, and we conclude that $IJ = (K)$.
      \qed
      
%%%%%%%%%%%%%%%%%%%%%%%%%%%%%%%%%%%%%7.4.13%%%%%%%%%%%%%%%%%%%%%%%%%%%%%%%%%%%%%
   \item[7.4.13]  Let $\varphi : R \rightarrow S$ be a homomorphism of
                  commutative rings.
                  \begin{enumerate}
                     \item Prove that if $P$ is a prime ideal of $S$ then either
                           $\varphi^{-1}(P) = R$ or $\varphi^{-1}(P)$ is a prime
                           ideal of $R$. Apply this to the special case when $R$
                           is a subring of $S$ and $\varphi$ is the inclusion
                           homomorphism to deduce that if $P$ is a prime ideal
                           of $S$ then $P \cap R$ is either $R$ or a prime ideal
                           of $R$.
                     \item Prove that if $M$ is a maximal ideal of $S$ and
                           $\varphi$ is surjective then $\varphi^{-1}(M)$ is a
                           maximal ideal of $R$. Give an example to show that 
                           this need not be the case if $\varphi$ is not
                           surjective.
                  \end{enumerate}
                  
      \textbf{Solution.}
      
      \begin{enumerate}
         \item \textbf{Proof.} Assume that $P$ is a prime ideal of $S$.
               Exercise 7.3.24(a) says that $\varphi^{-1}(P)$ is an ideal of
               $R$. If $\varphi^{-1}(P) = R$, we are done, so suppose otherwise.
               Now suppose that $xy \in \varphi^{-1}(P)$ for some $x, y \in R$.
               Then $\varphi(xy) \in P$;  that is, $\varphi(x)\varphi(y) \in P$.
               Since $P$ is a prime ideal, it follows that $\varphi(x) \in P$ or
               $\varphi(y) \in P$; equivalently, it follows that
               $x \in \varphi^{-1}(P)$ or $y \in \varphi^{-1}(P)$, so that
               $\varphi^{-1}(P)$ is prime in $R$. If $\varphi$ is the inclusion
               homomorphism, then Exercise 7.3.24(b) says that
               $\varphi^{-1}(P) = P \cap R$ and we are done. \qed
         \item \textbf{Proof.} Assume that $M$ is a maximal ideal of $S$ and
               $\varphi$ is onto. That is, $M \neq S$ and
               $\varphi^{-1}(M) \neq R$. Now suppose that $\varphi^{-1}(M)$ is
               contained in an ideal $I$ of $R$. Since
               $\varphi^{-1}(M) \subseteq I$, it follows that $M$ is contained
               in the ideal $\varphi(I)$ of $S$. Hence, by the maximality of
               $M$, we conclude that $\varphi(I) = M$ or $\varphi(I) = S$.
               That is, $I = \varphi^{-1}(M)$ or $I = \varphi^{-1}(S) = R$,
               where the last equality follows from the surjectivity of
               $\varphi$. Thus $\varphi^{-1}(M)$ is maximal in $R$.
               
               \textbf{Example.} Let $R = \Z$, $S = \Q$, and $\varphi$ the
               inclusion homomorphism. By Exercise 7.4.4, $\{0\}$ is a maximal
               ideal in $\Q$. However, $\varphi^{-1}(\{0\}) = \{0\}$ is not
               maximal in $\Z$ because $\Z/\{0\} \cong \Z$ is not a field. \qed
      \end{enumerate}
%%%%%%%%%%%%%%%%%%%%%%%%%%%%%%%%%%%%%7.4.14%%%%%%%%%%%%%%%%%%%%%%%%%%%%%%%%%%%%%
   \item[7.4.14]  Assume $R$ is commutative. Let $x$ be an indeterminate, let
                  $f(x)$ be a monic polynomial in $R[x]$ of degree $n \ge 1$ and
                  use the bar notation to denote passage to the quotient ring
                  $R[x]/(f(x))$.
                  \begin{enumerate}
                     \item Show that every element of $R[x]/(f(x))$ is of the
                           form $\overline{p(x)}$ for some polynomial
                           $p(x) \in R[x]$ of degree less than $n$, i.e.,
                           $$R[x]/(f(x)) = \{\overline{a_0} + \overline{a_1x}
                             + \cdots + \overline{a_{n-1}x^{n-1}} : a_0, a_1, 
                             \ldots, a_{n-1} \in R\}.$$
                           [If $f(x) = x^n + b_{n-1}x^{n-1} + \cdots + b_0$ then
                           $\overline{x^n}$ =
                           $\overline{-(b_{n-1}x^{n-1} + \cdots + b_0)}$. Use
                           this to reduce powers of $\overline{x}$ in the
                           quotient ring.]
                     \item Prove that if $p(x)$ and $q(x)$ are distinct
                           polynomials in $R[x]$ which are both of degree less
                           than $n$, then $\overline{p(x)} \neq\overline{q(x)}$.
                           [Otherwise $p(x) - q(x)$ is an $R[x]$-multiple of the
                           monic polynomial $f(x)$.]
                     \item If $f(x) = a(x)b(x)$ where both $a(x)$ and $b(x)$
                           have degree less than $n$, prove that
                           $\overline{a(x)}$ is a zero divisor in $R[x]/(f(x))$.
                     \item If $f(x) = x^n - a$ for some nilpotent element
                           $a \in R$, prove that $\overline{x}$ is nilpotent in
                           $R[x]/(f(x))$.
                     \item Let $p$ be a prime, assume $R = \F_p$ and
                           $f(x) = x^p - a$ for some $a \in \F_p$. Prove that
                           $\overline{x-a}$ is nilpotent in $R[x]/(f(x))$. [Use
                           7.3.26(c).]
                  \end{enumerate}
%%%%%%%%%%%%%%%%%%%%%%%%%%%%%%%%%%%%%7.4.15%%%%%%%%%%%%%%%%%%%%%%%%%%%%%%%%%%%%%
   \item[7.4.15]  Let $x^2 + x + 1$ be an element of the polynomial ring
                  $E = \F_2[x]$ and use the bar notation to denote passage to 
                  the quotient ring $\F_2[x]/(x^2 + x + 1)$.
                  \begin{enumerate}
                     \item Prove that $\overline{E}$ has 4 elements:
                           $\overline{0}$, $\overline{1}$, $\overline{x}$, and
                           $\overline{x+1}$.
                     \item Write out the $4 \times 4$ addition table for
                           $\overline{E}$ and deduce that the additive group
                           $\overline{E}$ is isomorphic to the Klein 4-group.
                     \item Write out the $4 \times 4$ multiplication table for
                           $\overline{E}$ and prove that $\overline{E}^\times$
                           is isomorphic to the cyclic group of order 3. Deduce
                           that $\overline{E}$ is a field.
                  \end{enumerate}
%%%%%%%%%%%%%%%%%%%%%%%%%%%%%%%%%%%%%7.4.16%%%%%%%%%%%%%%%%%%%%%%%%%%%%%%%%%%%%%
   \item[7.4.16]  Let $x^4 - 16$ be an element of the polynomial ring
                  $E = \Z[x]$ and use the bar notation to denote passage to the
                  quotient ring $\Z[x]/(x^4 - 16)$.
                  \begin{enumerate}
                     \item Find a polynomial of degree $\le 3$ that is congruent
                           to $7x^{13} - 11x^9 + 5x^5 - 2x^3 + 3$ modulo
                           $(x^4 - 16)$.
                     \item Prove that $\overline{x - 2}$ and $\overline{x + 2}$
                           are zero divisors in $\overline{E}$.
                  \end{enumerate}
%%%%%%%%%%%%%%%%%%%%%%%%%%%%%%%%%%%%%7.4.17%%%%%%%%%%%%%%%%%%%%%%%%%%%%%%%%%%%%%
   \item[7.4.17]  Let $x^3 - 2x + 1$ be an element of the polynomial ring
                  $E = \Z[x]$ and use the bar notation to denote passage to the
                  quotient ring $\Z[x]/(x^3 - 2x + 1)$. Let
                  $p(x) = 2x^7 - 7x^5 + 4x^3 - 9x + 1$ and let
                  $q(x) = (x - 1)^4$.
                  \begin{enumerate}
                     \item Express each of the following elements of
                           $\overline{E}$ in the form $\overline{f(x)}$ for some
                           polynomial $f(x)$ of degree $\le 2$:
                           $\overline{p(x)}, \overline{q(x)},
                             \overline{p(x) + q(x)}, \text{ and }
                             \overline{p(x)q(x)}$.
                     \item Prove that $\overline{E}$ is not an integral domain.
                     \item Prove that $\overline{x}$ is a unit in
                           $\overline{E}$.
                  \end{enumerate}
%%%%%%%%%%%%%%%%%%%%%%%%%%%%%%%%%%%%%7.4.18%%%%%%%%%%%%%%%%%%%%%%%%%%%%%%%%%%%%%
   \item[7.4.18]  Prove that if $R$ is an integral domain and $R[[x]]$ is the
                  ring of formal power series in the indeterminate $x$ then the
                  principal ideal generated by $x$ is a prime ideal
                  (cf. Exercise 7.2.3). Prove that the principal ideal generated
                  by $x$ is a maximal ideal if and only if $R$ is a field.
%%%%%%%%%%%%%%%%%%%%%%%%%%%%%%%%%%%%%7.4.19%%%%%%%%%%%%%%%%%%%%%%%%%%%%%%%%%%%%%
   \item[7.4.19]  Let $R$ be a finite commutative ring with identity. Prove that
                  every prime ideal of $R$ is a maximal ideal.
                  
      \textbf{Proof.} Let $P$ be a prime ideal of $R$. It follows by Proposition
      7.13 that $R/P$ is an integral domain. Since $R$ is finite and $R/P$ is a
      partition of $R$, then $R/P$ must be finite. Thus, by Corollary 7.3,
      $R/P$ is a field, so that $P$ is a maximal ideal of $R$ by
      Proposition 7.12. \qed
%%%%%%%%%%%%%%%%%%%%%%%%%%%%%%%%%%%%%7.4.20%%%%%%%%%%%%%%%%%%%%%%%%%%%%%%%%%%%%%
   \item[7.4.20]  Prove that a nonzero finite commutative ring that has no zero
                  divisors is a field (if the ring has an identity, this is
                  Corollary 3, so do not assume the ring has a 1).
                  
      \textbf{Proof.} Suppose that $R$ is a nonzero finite commutative ring that
      has no zero divisors. Let $a$ be a nonzero element in $R$. Consider the
      map $\alpha : R \rightarrow R$, defined by $x \mapsto ax$. If
      $\alpha(x) = \alpha(y)$ for some $x, y \in R$, then $ax = ay$, so that
      $a(x - y) = 0$, and since $R$ has no zero divisors, it follows that
      $x = y$ (since $a \neq 0$). Hence $\alpha$ is injective, and thus,
      bijective since $R$ is finite. So there exists $r \in R$ such that
      $\alpha(r) = a$; i.e., $ar = a$. Let $b \in R$, then there exists
      $r' \in R$ such that $\alpha(r') = b$, so that $ar' = b$. Thus
      $$br = (ar')r = (r'a)r = r'(ar) = r'a = ar' = b.$$
      That is, $r$ is the identity of $R$, so that $R$ is an integral domain. It
      follows by Corollary 7.3 that $R$ is a field. \qed
%%%%%%%%%%%%%%%%%%%%%%%%%%%%%%%%%%%%%7.4.21%%%%%%%%%%%%%%%%%%%%%%%%%%%%%%%%%%%%%
   \item[7.4.21]  Prove that a finite ring with identity $1 \neq 0$ that has no
                  zero divisors is a field (you may quote Wedderburn's Theorem).
                  
      \textbf{Proof.} Assume that $R$ is a finite ring with identity $1 \neq 0$
      that has no zero divisors. Let $a$ be a nonzero element in $R$. Using
      similar arguments as we did for the map in Exercise 7.4.21, it follows
      that the maps $\alpha_1 : R \rightarrow R$, defined by $r \mapsto ar$ and
      $\alpha_2 : R \rightarrow R$, defined by $r \mapsto ra$ are injective, and
      thus, bijective, since $R$ is finite. Thus there exist $x$ and $y$ in $R$
      such that $\alpha_1(x) = \alpha_2(y) = 1$. That is, $ax = ya = 1$. We now
      have that $y = y \cdot 1 = y(ax) = (ya)x = 1 \cdot x = x$. Thus $a$ is a
      unit in $R$. That is, $R$ is a division ring. Conclude by Wedderburn's
      Theorem that $R$ is a field. \qed
%%%%%%%%%%%%%%%%%%%%%%%%%%%%%%%%%%%%%7.4.22%%%%%%%%%%%%%%%%%%%%%%%%%%%%%%%%%%%%%
   \item[7.4.22]  Let $p \in \Z^+$ be a prime and let the $\F_p$ Quaternions be
                  defined by
                  $$a + bi + cj + dk \qquad a, b, c, d \in \Z/p\Z$$
                  where addition is componentwise and multiplication is defined
                  using the same relations $i, j, k$ as for the real
                  Quaternions.
                  \begin{enumerate}
                     \item Prove that the $\F_p$ Quaternions are a homomorphic
                           image of the integral Quaternions (cf. Section 7.1).
                     \item Prove that the $\F_p$ Quaternions contain zero
                           divisors (and so they cannot be a division ring).
                           [Use the preceding exercise.]
                  \end{enumerate}
                  
      \textbf{Solution.}
      
      \begin{enumerate}
         \item \textbf{Proof.} Let $Z$ and $R$ denote the integral and $\F_p$
               Quaternions respectively. Consider the map
               $\varphi : Z \rightarrow R$ defined by
               $a + bi + cj + dk \mapsto \overline{a} + \overline{b}i +
                \overline{c}j +\overline{d}k$. It is clear that $\varphi$ is
                surjective. Now let us show that it is onto. Let
                $z_1, z_2 \in Z$. Then $z_1 = a + bi + cj + dk$ and
                $z_2 = e + fi + gj + hk$. 
      \end{enumerate}
%%%%%%%%%%%%%%%%%%%%%%%%%%%%%%%%%%%%%7.4.23%%%%%%%%%%%%%%%%%%%%%%%%%%%%%%%%%%%%%
   \item[7.4.23]  Prove that in a Boolean ring (cf. Exercise 7.1.15) every prime
                  ideal is a maximal ideal.
                  
      \textbf{Proof.} Let $R$ be a Boolean ring and suppose that $P$ is a prime
      ideal of $R$. By Exercise 7.1.15, $R$ is commutative, so that $R/P$ is
      an integral domain by Proposition 7.13. Observe that $R/P$ is also an
      integral domain, so conclude by Exercise 7.1.16 that $R/P$ is a field, so
      that $P$ is a maximal ideal by Proposition 7.12. \qed
%%%%%%%%%%%%%%%%%%%%%%%%%%%%%%%%%%%%%7.4.24%%%%%%%%%%%%%%%%%%%%%%%%%%%%%%%%%%%%%
   \item[7.4.24]  Prove that in a Boolean ring every finitely generated ideal is
                  principal.

      \textbf{Proof.} Let $R$ be a Booelan ring. For a positive integer $n$, let 
      $P(n)$ be the statement: every ideal generated by $n$ elements of $R$ is 
      principal. We will proceed by induction $n$.

      \begin{itemize}
         \item \textbf{base cases.} $n = 1$ and $n = 2$. The statement $P(1)$ 
               trivially holds. Now consider the ideal $(r \; s)$, where
               $r, s \in R$. Let $a = r + s - rs$. We claim that
               $(r \; s) = (a)$. Since $r, s$, and $rs$ are elements of
               $(r \; s)$, it follows that $a \in (r \; s)$, so that
               $(a) \subseteq (r \; s)$. Using the commutativity of $R$
               (Exercise 7.1.15) and the fact that $x^2 = x$ for all $x \in R$,
               it follows that
               $$ra = r(r + s - rs) = r^2 + rs - r^2s = r + rs - rs = r$$
               and
               $$sa = s(r + s - rs) = rs + s^2 - rs^2 = rs + s - rs = s.$$  
               Because $(a)$ is an ideal of $R$, we have that $ra, sa \in R$;
               i.e., $r, s \in R$ so that $(r \; s) \subseteq (a)$, and we
               conclude that $(r \; s) = (a)$; thus $P(2)$ holds.
         \item \textbf{inductive hypothesis.} Suppose that $P(k)$ holds, where
               $k$ is a positive integer.
         \item \textbf{show $P(k+1)$ holds.} Consider the ideal
               $(a_1 \; a_2 \cdots \; a_k \; a_{k+1})$, where the generators are 
               in $R$. By the inductive hypothesis, there exists $a' \in R$ such
               that $(a_1 \; a_2 \cdots \; a_k) = (a')$, so that
               $(a_1 \; a_2 \cdots \; a_k \; a_{k+1}) = (a' \; a_{k+1})$. Since
               $P(2)$ also holds, we have that $(a' \; a_{k+1})$ is principal;
               i.e., $(a_1 \; a_2 \cdots \; a_k \; a_{k+1})$ is principal, and
               we conclude that $P(k+1)$ is true.
      \end{itemize}
      Thus, by mathematical induction, $P(n)$ holds for all $n \ge 1$. \qed
%%%%%%%%%%%%%%%%%%%%%%%%%%%%%%%%%%%%%7.4.25%%%%%%%%%%%%%%%%%%%%%%%%%%%%%%%%%%%%%
   \item[7.4.25]  Assume $R$ is commutative and for each $a \in R$ there is an
                  integer $n > 1$ (depending on $a$) such that $a^n = a$. Prove
                  that every prime ideal of $R$ is a maximal ideal.

      \textbf{Proof.} Suppose that $P$ is a prime ideal of $R$. By Proposition
      7.13 $R/P$ is an integral domain. Let $x$ be a nonzero element of $R/P$.
      Then $x = \overline{r}$, where $r \in R$ and $r \notin P$. By hypothesis,
      there exists $m \ge 2$ such that $r^m = r$; that is, $r(r^{m-1}-1) = 0$,
      so that $\overline{r} \cdot \overline{r^{m-1}-1} = 0$. Since $R/P$ is an
      integral domain and $\overline{r} \neq 0$, it follows that
      $\overline{r^{m-1}-1} = 0$, or equivalently,
      $\overline{r}\overline{r^{m-2}} = 1$, and we conclude that $x$ is a unit
      in $R/P$; that is, $R/P$ is a field. Conclude by Proposition 7.12 that
      $P$ is a maximal ideal of $R$. \qed
%%%%%%%%%%%%%%%%%%%%%%%%%%%%%%%%%%%%%7.4.26%%%%%%%%%%%%%%%%%%%%%%%%%%%%%%%%%%%%%
   \item[7.4.26]  Prove that a prime ideal in a commutative ring $R$ contains
                  every nilpotent element (cf. Exercise 7.1.13). Deduce that the
                  nilradical of $R$ (cf. Exercise 7.3.29) is contained in the
                  intersection of all the prime ideals of $R$. (It is shown in
                  Section 15.2 that the nilradical of $R$ is equal to the
                  intersection of all prime ideals of $R$.)

      \textbf{Proof.} Assume that $R$ is a commutative ring and suppose that $P$
      is a prime ideal of $R$. We want to show that if $x^n \in P$ for some
      $x \in R$ and $n \in \Z^+$, then $x \in P$. Proceed by induction on $n$.
   
      \begin{itemize}
         \item \textbf{base case.} $n = 1$. The statement trivially holds when
               $n$ equals 1.
         \item \textbf{inductive hypothesis.} Suppose our assertion holds for
               some positive integer $k$.
         \item \textbf{statement holds for $k+1$.} Suppose that $y^{k+1} \in P$
               for some $y \in R$; i.e, $yy^k = y^{k+1} \in P$. Since $P$ is
               prime, it follows that $y \in P$ or $y^k \in P$. If the former
               holds, we are done, so suppose that $y^k \in P$. Then it follows
               by the inductive hypothesis that $y \in P$. In any case,
               $y \in P$. Thus our assertions holds for $k + 1$.
      \end{itemize}
      So conclude by induction that our assertion holds for all $n \ge 1$. Now
      suppose that $r \in R$ is nilpotent. Then $r^m = 0$ for some $m \in \Z^+$.
      Since $0 \in P$, it follows that $0 = r^m \in P$, so that $r \in P$, by
      the proof above. Since $P$ was an arbitrary prime ideal of $R$, it follows
      that $r$ is a member of all prime ideals of $R$. Thus the nilradical of
      $R$ is equal to the intersection of all prime ideals of $R$. \qed
%%%%%%%%%%%%%%%%%%%%%%%%%%%%%%%%%%%%%7.4.27%%%%%%%%%%%%%%%%%%%%%%%%%%%%%%%%%%%%%
   \item[7.4.27]  Let $R$ be a commutative ring with $1 \neq 0$. Prove that if
                  $a$ is a nilpotent element of $R$ then $1 - ab$ is a unit for
                  all $b \in R$.

      \textbf{Proof.} Suppose that $a$ is a nilpotent element of $R$. Let
      $b \in R$. Then Exercise 7.1.14(b) says that $a(-b)$ is nilpotent in $R$.
      Thus $1 + a(-b) = 1 - ab$ is a unit in $R$ by Exercise 7.1.14(c). \qed
%%%%%%%%%%%%%%%%%%%%%%%%%%%%%%%%%%%%%7.4.28%%%%%%%%%%%%%%%%%%%%%%%%%%%%%%%%%%%%%
   \item[7.4.28]  Prove that if $R$ is a commutative ring and
                  $N = (a_1, a_2, \ldots, a_m)$ where each $a_i$ is a nilpotent
                  element, then $N$ is a nilpotent ideal (cf. Exercise 7.3.37).
                  Deduce that if the nilradical of $R$ is finitely generated
                  then it is a nilpotent ideal.
%%%%%%%%%%%%%%%%%%%%%%%%%%%%%%%%%%%%%7.4.29%%%%%%%%%%%%%%%%%%%%%%%%%%%%%%%%%%%%%
   \item[7.4.29]  Let $p$ be a prime and let $G$ be a finite group of order a
                  power of $p$ (i.e., a $p$-group). Prove that the augmentation
                  ideal in the group ring $\Z/p\Z G$ is a nilpotent ideal. (Note
                  that this ring may be noncommutative.) [Use Exercise 7.4.2]
%%%%%%%%%%%%%%%%%%%%%%%%%%%%%%%%%%%%%7.4.30%%%%%%%%%%%%%%%%%%%%%%%%%%%%%%%%%%%%%
   \item[7.4.30]  Let $I$ be an ideal of the commutative ring $R$ and define
                  $$\text{rad }I = \{r \in R : r^n \in I
                    \text{ for some }n \in \Z^+\}$$
                  called the \textit{radical} of $I$. Prove that rad $I$ is an
                  ideal containing $I$ and that $(\text{rad }I)/I$ is the
                  nilradical of the quotient ring $R/I$, i.e.,
                  $(\text{rad } I)/I = \mathfrak{N}(R/I)$ (cf. Exercise 7.3.29).
                  
      \textbf{Proof.} If $r \in I$ then $r^1 \in I$, so that
      $r \in \text{rad }I$; that is, $I \subseteq \text{rad }I$. We will now
      show that rad $I$ is an ideal of $R$. Let $x, y \in \text{rad }I$. That
      is, $x^m, y^n \in I$ for some $m, n \in \Z^+$. Observe that if $j \ge n$, 
      then $y^j = (y^{j-n})y^n \in I$ (since $I$ is an ideal of $R$).
      \begin{itemize}
         \item \textbf{closure under addition.} The Binomial Theorem tells us
               that
               $$(x + y)^{m + n} = \sum_{i=0}^{m+n}\binom{m+n}{i}x^iy^{m+n-i}.$$
               If $m \le i \le m + n$, then $x^i \in I$; however, if
               $0 \le i < m$, so that $m + n - i > n$, then $y^{m+n-i} \in I$.
               Thus, if $0  \le i \le m + n$, then
               $\binom{m+n}{i}x^iy^{m+n-i} \in I$ because $I$ is an ideal of
               $R$. Conclude by the closure of $I$ under addition that
               $$(x + y)^{m + n} =
                 \sum_{i=0}^{m+n}\binom{m+n}{i}x^iy^{m+n-i} \in I.$$
               That is, $x + y \in \text{rad }I$, and we conclude that rad $I$
               is closed under addition.
         \item \textbf{closure under additive inverse.} We have that
               $(-y)^n = y^n$ (if $n$ is even) or $(-y)^n = -y^n$ (if $n$ is 
               odd). In either case, it follows that $(-y)^n \in I$, so that
               $-y \in \text{rad } I$; i.e., rad $I$ is closed under additive 
               inverses.
         \item \textbf{closure under multiplication by $R$.} Let $r \in R$. It
               follows immediately that $(ry)^n = (r^n)y^n \in I$ because $I$ is 
               closed under multiplication by $R$. Thus $ry \in \text{rad } I$, 
               so that rad $I$ is closed under multiplication by $R$.
      \end{itemize}
      Conclude that rad $I$ is an ideal of $R$ containing $I$. Finally, let us
      now show that $(\text{rad } I)/I = \mathfrak{N}(R/I)$.
      \begin{itemize}
         \item $(\subseteq)$ Let $\overline{s} \in (\text{rad } I)/I$, where
               $s \in \text{rad } I$. That is, $s^p \in I$ for some
               $p \in \Z^+$. Thus $\overline{s}^p = \overline{s^p} = I =
               \overline{0}$, so that $\overline{s} \in \mathfrak{N}(R/I)$. It
               follows that $(\text{rad } I)/I \subseteq \mathfrak{N}(R/I)$.
         \item $(\supseteq)$ Now let $\overline{a} \in \mathfrak{N}(R/I)$. That 
               is, there exists a positive integer $r$ such that
               $\overline{a}^r = \overline{a^r} = \overline{0}$; it follows that
               $a^r \in I$, so that $a \in \text{rad }I$. Thus
                $(\text{rad } I)/I \supseteq \mathfrak{N}(R/I)$.
      \end{itemize}
      Conclude that  $(\text{rad } I)/I = \mathfrak{N}(R/I)$. \qed
%%%%%%%%%%%%%%%%%%%%%%%%%%%%%%%%%%%%%7.4.31%%%%%%%%%%%%%%%%%%%%%%%%%%%%%%%%%%%%%
   \item[7.4.31]  An ideal $I$ of the commutative ring $R$ is called a
                  \textit{radical ideal} if rad $I = I$.
                  \begin{enumerate}
                     \item Prove that every prime ideal of $R$ is a radical
                           ideal.
                     \item Let $n > 1$ be an integer. Prove that 0 is a radical
                           ideal in $\Z/n\Z$ if and only if $n$ is a product of
                           distinct primes to the first power (i.e., $n$ is
                           square free). Deduce that $(n)$ is a radical ideal of
                           $\Z$ if and only if $n$ is a product of distinct
                           primes in $\Z$.
                  \end{enumerate}

      \textbf{Solution.}

      \begin{enumerate}
         \item \textbf{Proof.} Suppose that $P$ is a prime ideal of $R$. By
               Proposition 7.13, $R/P$ is an integral domain; if $x$ is a
               nilpotent element of $R/P$, then Exercise 7.1.14(a) says that
               $x$ is either $\overline{0}$ or $x$ is a zero divisor. Since
               $R/P$ is an integral domain, it follows that $x$ must zero; that 
               is, $\mathfrak{N}(R/P) = \{\overline{0}\}$. Exercise 7.4.30 says 
               that $(\text{rad } P)/P = \mathfrak{N}(R/P)$. Thus
               $(\text{rad } P)/P =  \{\overline{0}\}$, so that rad $P = P$.
               \qed
      \end{enumerate}
%%%%%%%%%%%%%%%%%%%%%%%%%%%%%%%%%%%%%7.4.32%%%%%%%%%%%%%%%%%%%%%%%%%%%%%%%%%%%%%
   \item[7.4.32]  Let $I$ be an ideal of the commutative ring $R$ and define
                  $$\text{Jac $I$ to be the intersection of all maximal ideals
                     of $R$ that contain $I$}$$
                  where the convention is that Jac $R = R$. (If $I$ is the zero
                  ideal, Jac 0 is called the \textit{Jacobson radical} of the
                  ring $R$, so Jac $I$ is the preimage in $R$ of the Jacobson
                  radical of $R/I$.)
                  \begin{enumerate}
                     \item Prove that Jac $I$ is an ideal of $R$ containing $I$.
                     \item Prove that $\text{rad } I \subseteq \text{Jac }I$,
                           where rad $I$ is the radical of $I$ defined in
                           Exercise 7.4.30.
                     \item Let $n > 1$ be an integer. Describe Jac $n\Z$ in
                           terms of the prime factorization of $n$.
                  \end{enumerate}
%%%%%%%%%%%%%%%%%%%%%%%%%%%%%%%%%%%%%7.4.33%%%%%%%%%%%%%%%%%%%%%%%%%%%%%%%%%%%%%
   \item[7.4.33]  Let $R$ be the ring of all continuous functions from the
                  closed interval $[0, 1]$ to $\R$ and for each $c \in [0, 1]$
                  let $M_c = \{f \in R : f(c) = 0\}$ (recall that $M_c$ was
                  shown to be a maximal ideal of $R$).
                  \begin{enumerate}
                     \item Prove that if $M$ is \textit{any} maximal ideal of
                           $R$ then there is a real number $c \in [0, 1]$ such
                           that $M = M_c$.
                     \item Prove that if $b$ and $c$ are distinct points in
                           $[0, 1]$ then $M_b \neq M_c$.
                     \item Prove that $M_c$ is not equal to the principal ideal
                           generated by $x - c$.
                     \item Prove that $M_c$ is not a finitely generated ideal.
                  \end{enumerate}
\end{enumerate}

\noindent The preceding exercise shows that there is a bijection between the
\textit{points} of the closed interval $[0, 1]$ and the set of
\textit{maximal ideals} in the ring $R$ of all continuous functions on $[0, 1]$
given by $c \leftrightarrow M_c$. For any subset $X$ of $\R$ or, more generally,
for any completely regular topological space $X$, the map $c \mapsto M_c$ is an
\textit{injection} from $X$ to the set of maximal ideals of $R$, where $R$ is
the ring of all bounded continuous real valued functions on $X$ and $M_c$ is the 
maximal ideal of functions that vanish at $c$. Let $\beta(X)$ be the set of
maximal ideals of $R$. One can put a topology on $\beta(X)$ in such a way that
if we identity $X$ with its image in $\beta(X)$ then $X$ (in its given topology)
becomes a subspace of $\beta(X)$. Moreover, $\beta(X)$ is a compact space under
this topology and is called the \textit{Stone-$\check{C}$ech compactification}
of $X$.

\begin{enumerate}
%%%%%%%%%%%%%%%%%%%%%%%%%%%%%%%%%%%%%7.4.34%%%%%%%%%%%%%%%%%%%%%%%%%%%%%%%%%%%%%
   \item[7.4.34]  Let $R$ be the ring of all continuous functions from $\R$ to
                  $\R$ and for each $c \in \R$ let $M_c$ be the maximal ideal
                  $\{f \in R : f(c) = 0\}$.
                  \begin{enumerate}
                     \item Let $I$ be the collection of functions $f(x)$ in $R$
                           with \textit{compact support} (i.e., $f(x) = 0$ for
                           $|x|$ sufficiently large). Prove that $I$ is an ideal
                           of $R$ that is not a prime ideal.
                     \item Let $M$ be a maximal ideal of $R$ containing $I$
                           (properly, by (a)). Prove that $M \neq M_c$ for any
                           $c \in \R$ (cf. the preceding exercise).
                  \end{enumerate}
%%%%%%%%%%%%%%%%%%%%%%%%%%%%%%%%%%%%%7.4.35%%%%%%%%%%%%%%%%%%%%%%%%%%%%%%%%%%%%%
   \item[7.4.35]  Let $A = (a_1, a_2, \ldots, a_n)$ be a nonzero finitely
                  generated ideal of $R$. Prove that there is an ideal $B$ which
                  is maximal with respect to the property that it does not
                  contained $A$. [Use Zorn's Lemma.]
%%%%%%%%%%%%%%%%%%%%%%%%%%%%%%%%%%%%%7.4.35%%%%%%%%%%%%%%%%%%%%%%%%%%%%%%%%%%%%%
   \item[7.4.36]  Assume $R$ is commutative. Prove that the set of prime ideals
                  in $R$ has a minimal element with respect to inclusion
                  (possibly the zero ideal). [Use Zorn's Lemma.]
%%%%%%%%%%%%%%%%%%%%%%%%%%%%%%%%%%%%%7.4.36%%%%%%%%%%%%%%%%%%%%%%%%%%%%%%%%%%%%%
   \item[7.4.37]  A commutative ring $R$ is called a \textit{local ring} if it
                  has a unique maximal ideal. Prove that if $R$ is a local ring
                  with maximal ideal $M$ then every element of $R - M$ is a
                  unit. Prove conversely that if $R$ is a commutative ring with
                  1 in which the set of nonunits forms an ideal of $M$, then $R$
                  is a local ring with unique maximal ideal $M$.

      \textbf{Proof.} Assume that $R$ is a local ring with unique maximal ideal
      $M$. Let $x \in R - M$. Suppose to the contrary that $(x)$ is a proper
      ideal of $R$. By Proposition 7.11, $(x)$ is contained in a maximal ideal
      $I$ of $R$; but since $R$ is a local ring, it follows that $I = M$, so
      that $(x) \subseteq M$, and thus $x \in M$, a contradiction, since
      $x \in R - M$; thus $(x)$ is not proper and we have that $(x) = R$. So
      $1 = xy$ for some $y \in R$; i.e., $x$ is a unit in $R$ and we conclude
      that every element in $R - M$ is a unit of $R$. \qed
%%%%%%%%%%%%%%%%%%%%%%%%%%%%%%%%%%%%%7.4.37%%%%%%%%%%%%%%%%%%%%%%%%%%%%%%%%%%%%%
   \item[7.4.38]  Prove that the ring of all rational numbers whose denominators
                  is odd is a local ring whose unique maximal ideal is the
                  principal ideal generated by 2.
%%%%%%%%%%%%%%%%%%%%%%%%%%%%%%%%%%%%%7.4.39%%%%%%%%%%%%%%%%%%%%%%%%%%%%%%%%%%%%%
   \item[7.4.39]  Following the notation of Exercise 7.1.26, ket $K$ be a field,
                  let $\nu$ be a discrete valuation on $K$ and let $R$ be the
                  valuation ring of $\nu$. For each integer $k \ge 0$ define
                  $A_k = \{r \in R : \nu(r) \ge k\} \cup \{0\}$.
                  \begin{enumerate}
                     \item Prove that $A_k$ is a principal ideal and that
                           $A_0 \supseteq A_1 \supseteq A_2 \supseteq \cdots$.
                     \item Prove that if $I$ is any nonzero ideal of $R$, then
                           $I = A_k$ for some $k \ge 0$. Deduce that $R$ is a
                           local ring with unique maximal ideal $A_1$.
                  \end{enumerate}
%%%%%%%%%%%%%%%%%%%%%%%%%%%%%%%%%%%%%7.4.40%%%%%%%%%%%%%%%%%%%%%%%%%%%%%%%%%%%%%
   \item[7.4.40]  Assume $R$ is commutative. Prove that the following are
                  equivalent: (see also Exercises 7.1.13 and 7.1.14)
                  \begin{enumerate}\renewcommand{\labelenumii}{(\roman{enumii})}
                     \item $R$ has exactly one prime ideal
                     \item every element of $R$ is either nilpotent or a unit
                     \item $R/\mathfrak{N}(R)$ is a field (cf. Exercise 7.3.29).
                  \end{enumerate}
%%%%%%%%%%%%%%%%%%%%%%%%%%%%%%%%%%%%%7.4.41%%%%%%%%%%%%%%%%%%%%%%%%%%%%%%%%%%%%%
   \item[7.4.41]  A proper ideal $Q$ of the commutative ring $R$ is called
                  \textit{primary} if whenever $ab \in Q$ and $a \notin Q$ then
                  $b^n \in Q$ for some positive integer $n$. (Note the symmetry
                  between $a$ and $b$ in this definition implies that if $Q$ is
                  a primary ideal and $ab \in Q$ with \textit{neither} $a$ nor
                  $b$ in $Q$, then a positive power of $a$ and a positive power
                  of $b$ both lie in $Q$.) Establish the following facts about
                  primary ideals.
                  \begin{enumerate}
                     \item The primary ideals of $\Z$ are 0 and $(p^n)$, where
                           $p$ is a prime and $n$ is a positive integer.
                     \item Every prime ideal of $R$ is a primary ideal.
                     \item An ideal $Q$ of $R$ is primary if and only if every
                           zero divisor in $R/Q$ is a nilpotent element of
                           $R/Q$.
                     \item If $Q$ is a primary ideal then rad($Q$) is a prime
                           ideal (cf. Exercise 7.4.30).
                  \end{enumerate}
\end{enumerate}
