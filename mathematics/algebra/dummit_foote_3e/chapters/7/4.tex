Let $R$ be a ring with identity $1 \neq 0$.
\begin{enumerate}
%%%%%%%%%%%%%%%%%%%%%%%%%%%%%%%%%%%%%7.4.1%%%%%%%%%%%%%%%%%%%%%%%%%%%%%%%%%%%%%%
   \item[7.4.1]   Let $L_j$ be the left ideal of $M_n(R)$ consisting of
                  arbitrary entries in the $j^{\text{th}}$ column and zero in
                  all other entries and let $E_{ij}$ be the element of $M_n(R)$
                  whose $i, j$ entry is 1 and whose other entries are all 0. 
                  Prove that $L_j = M_n(R)E_{ij}$ for any $i$.
                  [See Exercise 7.2.6]

      \textbf{Proof.} Fix $i, j \in \{1, 2, \ldots, n\}$, where $n \in \Z^+$.

      \begin{itemize}
         \item ($\subseteq$) Let $A \in L_j$. By definition of $L_j$, the
               $j^{\text{th}}$ column of $A$ is arbitrary and every other entry 
               is zero. Thus it follows by Exercise 7.2.6(b) that
               $A = (A\cdot E_{ji})E_{ij}$, so that
               $L_j \subseteq M_n(R)E_{ij}$.

         \item ($\supseteq$) It follows immediately from Exercise 7.2.6(b) that
               $L_j \supseteq M_n(R)E_{ij}$.
      \end{itemize}
      Thus conclude that $L_j = M_n(R)E_{ij}$. \qed
%%%%%%%%%%%%%%%%%%%%%%%%%%%%%%%%%%%%%7.4.2%%%%%%%%%%%%%%%%%%%%%%%%%%%%%%%%%%%%%%
   \item[7.4.2]   Assume $R$ is commutative. Prove that the augmentation ideal
                  in the group ring $RG$ is generated by $\{g - 1 : g \in G\}$.
                  Prove that if $G = \cyc{\sigma}$ is cyclic then the 
                  augmentation ideal is generated by $\sigma - 1$.

      \textbf{Proof.} Let $G = \{g_1, g_2, \ldots, g_n\}$,
      $I = (g_1 - 1, g_2 - 1, \ldots, g_n - 1)$, and let $K$ be the augmentation 
      ideal of $RG$. First we want to show that $K = I$.

      \begin{itemize}
         \item ($\subseteq$) Let $r \in K$. Then $r = r_1g_1 + \cdots + r_ng_n$
               for some $r_1, \ldots, r_n \in R$. Since $r \in K$, it follows
               that $r_1 + \cdots + r_n = 0$. Thus
               \begin{align*}
                  r &= r_1g_1 + \cdots + r_ng_n - 0 \\
                    &= r_1g_1 + \cdots + r_ng_n - (r_1 + \cdots + r_n) \\
                    &= r_1g_1 - r_1 + \cdots + r_ng_n - r_n \\
                    &= r_1(g_1 - 1) + \cdots + r_n(g_n - 1) \in I,
               \end{align*}
               so that $K \subseteq I$.

         \item ($\supseteq$) To show that $K \supseteq I$, it suffices to show
               that $K$ contains the generators of $I$. But this follows
               immediately since the sum of the coefficients of the generators
               equal 0, so that $g_i - 1 \in K$ for $1 \le i \le n$. That is,
               $K \supseteq I$.
      \end{itemize}
      We conclude that $K = I$. Now suppose that $G = \cyc{\sigma}$ is cyclic, 
      where $|\sigma| = m \in \Z^+$. As we showed above,
      $K = (\sigma - 1, \sigma^2 - 1, \ldots, \sigma^{m-1} - 1)$. Consider one
      of the generators of $K$, say $\sigma^i - 1$, where $1 \le i \le m - 1$.
      Since
      $$(\sigma^i - 1) = (\sigma^{i-1} + \sigma^{i-2} + \cdots + \sigma + 1)
        (\sigma - 1),$$
      it follows that every generator of $K$ is an element of the ideal
      generated by $\sigma - 1$. Hence that $K = (\sigma - 1)$. \qed
%%%%%%%%%%%%%%%%%%%%%%%%%%%%%%%%%%%%%7.4.3%%%%%%%%%%%%%%%%%%%%%%%%%%%%%%%%%%%%%%
   \item[7.4.3]   \begin{enumerate}
                     \item Let $p$ be a prime and let $G$ be an abelian group of
                           order $p^n$. Prove that the nilradical of the group
                           ring $\F_pG$ is the augmentation ideal
                           (cf. Exercise 7.3.29). [Use the preceding exercise.]
                     \item Let $G = \{g_1, \ldots, g_n\}$ be the finite group
                           and assume $R$ is commutative. Prove that if $r$ is
                           any element of the augmentation ideal of $RG$ then
                           $r(g_1 + \cdots + g_n) = 0$. [Use the preceding
                           exercise.]
                  \end{enumerate}
%%%%%%%%%%%%%%%%%%%%%%%%%%%%%%%%%%%%%7.4.4%%%%%%%%%%%%%%%%%%%%%%%%%%%%%%%%%%%%%%
   \item[7.4.4]   Assume $R$ is commutative. Prove that $R$ is a field if and
                  only if 0 is a maximal ideal.
%%%%%%%%%%%%%%%%%%%%%%%%%%%%%%%%%%%%%7.4.5%%%%%%%%%%%%%%%%%%%%%%%%%%%%%%%%%%%%%%
   \item[7.4.5]   Prove that if $M$ is an ideal such that $R/M$ is a field then
                  $M$ is a maximal ideal (do not asssume $R$ is commutative).
%%%%%%%%%%%%%%%%%%%%%%%%%%%%%%%%%%%%%7.4.6%%%%%%%%%%%%%%%%%%%%%%%%%%%%%%%%%%%%%%
   \item[7.4.6]   Prove that $R$ is a division ring if and only if its only left
                  ideals are (0) and $R$. (The analogous result holds when
                  ``left" is replaced by ``right.")
%%%%%%%%%%%%%%%%%%%%%%%%%%%%%%%%%%%%%7.4.7%%%%%%%%%%%%%%%%%%%%%%%%%%%%%%%%%%%%%%
   \item[7.4.7]   Let $R$ be a commutative ring with 1. Prove that the principal
                  ideal generated by $x$ in the polynomial ring $R[x]$ is a
                  prime ideal if and only if $R$ is an integral domain. Prove
                  that $(x)$ is a maximal ideal if and only if $R$ is a field.
%%%%%%%%%%%%%%%%%%%%%%%%%%%%%%%%%%%%%7.4.8%%%%%%%%%%%%%%%%%%%%%%%%%%%%%%%%%%%%%%
   \item[7.4.8]   Let $R$ be an integral domain. Prove that $(a) = (b)$ for some
                  elements $a, b \in R$, if and only if $a = ub$ for some unit
                  $u$ of $R$.
%%%%%%%%%%%%%%%%%%%%%%%%%%%%%%%%%%%%%7.4.9%%%%%%%%%%%%%%%%%%%%%%%%%%%%%%%%%%%%%%
   \item[7.4.9]   Let $R$ be the ring of all continuous functions on $[0, 1]$
                  and let $I$ be the collection of functions $f(x)$ in $R$ with
                  $f(1/3) = f(1/2) = 0$. Prove that $I$ is an ideal of $R$ but
                  is not a prime ideal.
%%%%%%%%%%%%%%%%%%%%%%%%%%%%%%%%%%%%%7.4.10%%%%%%%%%%%%%%%%%%%%%%%%%%%%%%%%%%%%%
   \item[7.4.10]  Assume $R$ is commutative. Prove that if $P$ is a prime ideal
                  of $R$ and $P$ contains no zero divisors then $R$ is an
                  integral domain.
%%%%%%%%%%%%%%%%%%%%%%%%%%%%%%%%%%%%%7.4.11%%%%%%%%%%%%%%%%%%%%%%%%%%%%%%%%%%%%%
   \item[7.4.11]  Assume $R$ is commutative. Let $I$ and $J$ be ideals of $R$
                  and assume $P$ is a prime ideal of $R$ that contains $IJ$ (for
                  example, if $P$ contains $I \cap J$). Prove either $I$ or $J$
                  is contained in $P$.
%%%%%%%%%%%%%%%%%%%%%%%%%%%%%%%%%%%%%7.4.12%%%%%%%%%%%%%%%%%%%%%%%%%%%%%%%%%%%%%
   \item[7.4.12]  Assume $R$ is commutative and suppose
                  $I = (a_1, a_2, \ldots, a_n)$ and
                  $J = (b_1, b_2, \ldots, b_m)$ are two finitely generated
                  ideals in $R$. Prove that the product ideal $IJ$ is finitely
                  generated by the elements $a_ib_j$ for $i = 1, 2, \ldots, n$,
                  and $j = 1, 2, \ldots, m$.
%%%%%%%%%%%%%%%%%%%%%%%%%%%%%%%%%%%%%7.4.13%%%%%%%%%%%%%%%%%%%%%%%%%%%%%%%%%%%%%
   \item[7.4.13]  Let $\varphi : R \rightarrow S$ be a homomorphism of
                  commutative rings.
                  \begin{enumerate}
                     \item Prove that if $P$ is a prime ideal of $S$ then either
                           $\varphi^{-1}(P) = R$ or $\varphi^{-1}(P)$ is a prime
                           ideal of $R$. Apply this to the special case when $R$
                           is a subring of $S$ and $\varphi$ is the inclusion
                           homomorphism to deduce that if $P$ is a prime ideal
                           of $S$ then $P \cap R$ is either $R$ or a prime ideal
                           of $R$.
                     \item Prove that if $M$ is a maximal ideal of $S$ and
                           $\varphi$ is surjective then $\varphi^{-1}(M)$ is a
                           maximal ideal of $R$. Give an example to show that 
                           this need not be the case if $\varphi$ is not
                           surjective.
                  \end{enumerate}
%%%%%%%%%%%%%%%%%%%%%%%%%%%%%%%%%%%%%7.4.14%%%%%%%%%%%%%%%%%%%%%%%%%%%%%%%%%%%%%
   \item[7.4.14]  Assume $R$ is commutative. Let $x$ be an indeterminate, let
                  $f(x)$ be a monic polynomial in $R[x]$ of degree $n \ge 1$ and
                  use the bar notation to denote passage to the quotient ring
                  $R[x]/(f(x))$.
                  \begin{enumerate}
                     \item Show that every element of $R[x]/(f(x))$ is of the
                           form $\overline{p(x)}$ for some polynomial
                           $p(x) \in R[x]$ of degree less than $n$, i.e.,
                           $$R[x]/(f(x)) = \{\overline{a_0} + \overline{a_1x}
                             + \cdots + \overline{a_{n-1}x^{n-1}} : a_0, a_1, 
                             \ldots, a_{n-1} \in R\}.$$
                           [If $f(x) = x^n + b_{n-1}x^{n-1} + \cdots + b_0$ then
                           $\overline{x^n}$ =
                           $\overline{-(b_{n-1}x^{n-1} + \cdots + b_0)}$. Use
                           this to reduce powers of $\overline{x}$ in the
                           quotient ring.]
                     \item Prove that if $p(x)$ and $q(x)$ are distinct
                           polynomials in $R[x]$ which are both of degree less
                           than $n$, then $\overline{p(x)} \neq\overline{q(x)}$.
                           [Otherwise $p(x) - q(x)$ is an $R[x]$-multiple of the
                           monic polynomial $f(x)$.]
                     \item If $f(x) = a(x)b(x)$ where both $a(x)$ and $b(x)$
                           have degree less than $n$, prove that
                           $\overline{a(x)}$ is a zero divisor in $R[x]/(f(x))$.
                     \item If $f(x) = x^n - a$ for some nilpotent element
                           $a \in R$, prove that $\overline{x}$ is nilpotent in
                           $R[x]/(f(x))$.
                     \item Let $p$ be a prime, assume $R = \F_p$ and
                           $f(x) = x^p - a$ for some $a \in \F_p$. Prove that
                           $\overline{x-a}$ is nilpotent in $R[x]/(f(x))$. [Use
                           7.3.26(c).]
                  \end{enumerate}
%%%%%%%%%%%%%%%%%%%%%%%%%%%%%%%%%%%%%7.4.15%%%%%%%%%%%%%%%%%%%%%%%%%%%%%%%%%%%%%
   \item[7.4.15]  Let $x^2 + x + 1$ be an element of the polynomial ring
                  $E = \F_2[x]$ and use the bar notation to denote passage to 
                  the quotient ring $\F_2[x]/(x^2 + x + 1)$.
                  \begin{enumerate}
                     \item Prove that $\overline{E}$ has 4 elements:
                           $\overline{0}$, $\overline{1}$, $\overline{x}$, and
                           $\overline{x+1}$.
                     \item Write out the $4 \times 4$ addition table for
                           $\overline{E}$ and deduce that the additive group
                           $\overline{E}$ is isomorphic to the Klein 4-group.
                     \item Write out the $4 \times 4$ multiplication table for
                           $\overline{E}$ and prove that $\overline{E}^\times$
                           is isomorphic to the cyclic group of order 3. Deduce
                           that $\overline{E}$ is a field.
                  \end{enumerate}
%%%%%%%%%%%%%%%%%%%%%%%%%%%%%%%%%%%%%7.4.16%%%%%%%%%%%%%%%%%%%%%%%%%%%%%%%%%%%%%
   \item[7.4.16]  Let $x^4 - 16$ be an element of the polynomial ring
                  $E = \Z[x]$ and use the bar notation to denote passage to the
                  quotient ring $\Z[x]/(x^4 - 16)$.
                  \begin{enumerate}
                     \item Find a polynomial of degree $\le 3$ that is congruent
                           to $7x^{13} - 11x^9 + 5x^5 - 2x^3 + 3$ modulo
                           $(x^4 - 16)$.
                     \item Prove that $\overline{x - 2}$ and $\overline{x + 2}$
                           are zero divisors in $\overline{E}$.
                  \end{enumerate}
%%%%%%%%%%%%%%%%%%%%%%%%%%%%%%%%%%%%%7.4.17%%%%%%%%%%%%%%%%%%%%%%%%%%%%%%%%%%%%%
   \item[7.4.17]  Let $x^3 - 2x + 1$ be an element of the polynomial ring
                  $E = \Z[x]$ and use the bar notation to denote passage to the
                  quotient ring $\Z[x]/(x^3 - 2x + 1)$. Let
                  $p(x) = 2x^7 - 7x^5 + 4x^3 - 9x + 1$ and let
                  $q(x) = (x - 1)^4$.
                  \begin{enumerate}
                     \item Express each of the following elements of
                           $\overline{E}$ in the form $\overline{f(x)}$ for some
                           polynomial $f(x)$ of degree $\le 2$:
                           $\overline{p(x)}, \overline{q(x)},
                             \overline{p(x) + q(x)}, \text{ and }
                             \overline{p(x)q(x)}$.
                     \item Prove that $\overline{E}$ is not an integral domain.
                     \item Prove that $\overline{x}$ is a unit in
                           $\overline{E}$.
                  \end{enumerate}
%%%%%%%%%%%%%%%%%%%%%%%%%%%%%%%%%%%%%7.4.18%%%%%%%%%%%%%%%%%%%%%%%%%%%%%%%%%%%%%
   \item[7.4.18]  Prove that if $R$ is an integral domain and $R[[x]]$ is the
                  ring of formal power series in the indeterminate $x$ then the
                  principal ideal generated by $x$ is a prime ideal
                  (cf. Exercise 7.2.3). Prove that the principal ideal generated
                  by $x$ is a maximal ideal if and only if $R$ is a field.
%%%%%%%%%%%%%%%%%%%%%%%%%%%%%%%%%%%%%7.4.19%%%%%%%%%%%%%%%%%%%%%%%%%%%%%%%%%%%%%
   \item[7.4.19]  Let $R$ be a finite commutative ring with identity. Prove that
                  every prime ideal of $R$ is a maximal ideal.
%%%%%%%%%%%%%%%%%%%%%%%%%%%%%%%%%%%%%7.4.20%%%%%%%%%%%%%%%%%%%%%%%%%%%%%%%%%%%%%
   \item[7.4.20]  Prove that a nonzero finite commutative ring that has no zero
                  divisors is a field (if the ring has an identity, this is
                  Corollary 3, so do not assume the ring has a 1).
%%%%%%%%%%%%%%%%%%%%%%%%%%%%%%%%%%%%%7.4.21%%%%%%%%%%%%%%%%%%%%%%%%%%%%%%%%%%%%%
   \item[7.4.21]  Prove that a finite ring with identity $1 \neq 0$ that has no
                  zero divisors is a field (you may quote Wedderburn's Theorem).
%%%%%%%%%%%%%%%%%%%%%%%%%%%%%%%%%%%%%7.4.22%%%%%%%%%%%%%%%%%%%%%%%%%%%%%%%%%%%%%
   \item[7.4.22]  Let $p \in \Z^+$ be a prime and let the $\F_p$ Quaternions be
                  defined by
                  $$a + bi + cj + dk \qquad a, b, c, d \in \Z/p\Z$$
                  where addition is componentwise and multiplication is defined
                  using the same relations $i, j, k$ as for the real
                  Quaternions.
                  \begin{enumerate}
                     \item Prove that the $\F_p$ Quaternions are a homomorphic
                           image of the integral Quaternions (cf. Section 7.1).
                     \item Prove that the $\F_p$ Quaternions contain zero
                           divisors (and so they cannot be a division ring).
                           [Use the preceding exercise.]
                  \end{enumerate}
%%%%%%%%%%%%%%%%%%%%%%%%%%%%%%%%%%%%%7.4.23%%%%%%%%%%%%%%%%%%%%%%%%%%%%%%%%%%%%%
   \item[7.4.23]  Prove that in a Boolean ring (cf. Exercise 7.1.15) every prime
                  ideal is a maximal ideal.
%%%%%%%%%%%%%%%%%%%%%%%%%%%%%%%%%%%%%7.4.24%%%%%%%%%%%%%%%%%%%%%%%%%%%%%%%%%%%%%
   \item[7.4.24]  Prove that in a Boolean ring every finitely generated ideal is
                  principal.
%%%%%%%%%%%%%%%%%%%%%%%%%%%%%%%%%%%%%7.4.25%%%%%%%%%%%%%%%%%%%%%%%%%%%%%%%%%%%%%
   \item[7.4.25]  Assume $R$ is commutative and for each $a \in R$ there is an
                  integer $n > 1$ (depending on $a$) such that $a^n = a$. Prove
                  that every prime ideal of $R$ is a maximal ideal.
%%%%%%%%%%%%%%%%%%%%%%%%%%%%%%%%%%%%%7.4.26%%%%%%%%%%%%%%%%%%%%%%%%%%%%%%%%%%%%%
   \item[7.4.26]  Prove that a prime ideal in a commutative ring $R$ contains
                  every nilpotent element (cf. Exercise 7.1.13). Deduce that the
                  nilradical of $R$ (cf. Exercise 7.3.29) is contained in the
                  intersection of all the prime ideals of $R$. (It is shown in
                  Section 15.2 that the nilradical of $R$ is equal to the
                  intersection of all prime ideals of $R$.)
%%%%%%%%%%%%%%%%%%%%%%%%%%%%%%%%%%%%%7.4.27%%%%%%%%%%%%%%%%%%%%%%%%%%%%%%%%%%%%%
   \item[7.4.27]  Let $R$ be a commutative ring with $1 \neq 0$. Prove that if
                  $a$ is a nilpotent element of $R$ then $1 - ab$ is a unit for
                  all $b \in R$.
%%%%%%%%%%%%%%%%%%%%%%%%%%%%%%%%%%%%%7.4.28%%%%%%%%%%%%%%%%%%%%%%%%%%%%%%%%%%%%%
   \item[7.4.28]  Prove that if $R$ is a commutative ring and
                  $N = (a_1, a_2, \ldots, a_m)$ where each $a_i$ is a nilpotent
                  element, then $N$ is a nilpotent ideal (cf. Exercise 7.3.37).
                  Deduce that if the nilradical of $R$ is finitely generated
                  then it is a nilpotent ideal.
%%%%%%%%%%%%%%%%%%%%%%%%%%%%%%%%%%%%%7.4.29%%%%%%%%%%%%%%%%%%%%%%%%%%%%%%%%%%%%%
   \item[7.4.29]  Let $p$ be a prime and let $G$ be a finite group of order a
                  power of $p$ (i.e., a $p$-group). Prove that the augmentation
                  ideal in the group ring $\Z/p\Z G$ is a nilpotent ideal. (Note
                  that this ring may be noncommutative.) [Use Exercise 7.4.2]
%%%%%%%%%%%%%%%%%%%%%%%%%%%%%%%%%%%%%7.4.30%%%%%%%%%%%%%%%%%%%%%%%%%%%%%%%%%%%%%
   \item[7.4.30]  Let $I$ be an ideal of the commutative ring $R$ and define
                  $$\text{rad }I = \{r \in R : r^n \in I
                    \text{ for some }n \in \Z^+\}$$
                  called the \textit{radical} of $I$. Prove that rad $I$ is an
                  ideal containing $I$ and that $(\text{rad }I)/I$ is the
                  nilradical of the quotient ring $R/I$, i.e.,
                  $(\text{rad} I)/I = \mathfrak{N}(R/I)$ (cf. Exercise 7.3.29).
%%%%%%%%%%%%%%%%%%%%%%%%%%%%%%%%%%%%%7.4.31%%%%%%%%%%%%%%%%%%%%%%%%%%%%%%%%%%%%%
   \item[7.4.31]  An ideal $I$ of the commutative ring $R$ is called a
                  \textit{radical ideal} if rad $I = I$.
                  \begin{enumerate}
                     \item Prove that every prime ideal of $R$ is a radical
                           ideal.
                     \item Let $n > 1$ be an integer. Prove that 0 is a radical
                           ideal in $\Z/n\Z$ if and only if $n$ is a product of
                           distinct primes to the first power (i.e., $n$ is
                           square free). Deduce that $(n)$ is a radical ideal of
                           $\Z$ if and only if $n$ is a product of distinct
                           primes in $\Z$.
                  \end{enumerate}
%%%%%%%%%%%%%%%%%%%%%%%%%%%%%%%%%%%%%7.4.32%%%%%%%%%%%%%%%%%%%%%%%%%%%%%%%%%%%%%
   \item[7.4.32]  Let $I$ be an ideal of the commutative ring $R$ and define
                  $$\text{Jac $I$ to be the intersection of all maximal ideals
                     of $R$ that contain $I$}$$
                  where the convention is that Jac $R = R$. (If $I$ is the zero
                  ideal, Jac 0 is called the \textit{Jacobson radical} of the
                  ring $R$, so Jac $I$ is the preimage in $R$ of the Jacobson
                  radical of $R/I$.)
                  \begin{enumerate}
                     \item Prove that Jac $I$ is an ideal of $R$ containing $I$.
                     \item Prove that $\text{rad } I \subseteq \text{Jac }I$,
                           where rad $I$ is the radical of $I$ defined in
                           Exercise 7.4.30.
                     \item Let $n > 1$ be an integer. Describe Jac $n\Z$ in
                           terms of the prime factorization of $n$.
                  \end{enumerate}
%%%%%%%%%%%%%%%%%%%%%%%%%%%%%%%%%%%%%7.4.33%%%%%%%%%%%%%%%%%%%%%%%%%%%%%%%%%%%%%
   \item[7.4.33]  Let $R$ be the ring of all continuous functions from the
                  closed interval $[0, 1]$ to $\R$ and for each $c \in [0, 1]$
                  let $M_c = \{f \in R : f(c) = 0\}$ (recall that $M_c$ was
                  shown to be a maximal ideal of $R$).
                  \begin{enumerate}
                     \item Prove that if $M$ is \textit{any} maximal ideal of
                           $R$ then there is a real number $c \in [0, 1]$ such
                           that $M = M_c$.
                     \item Prove that if $b$ and $c$ are distinct points in
                           $[0, 1]$ then $M_b \neq M_c$.
                     \item Prove that $M_c$ is not equal to the principal ideal
                           generated by $x - c$.
                     \item Prove that $M_c$ is not a finitely generated ideal.
                  \end{enumerate}
\end{enumerate}

\noindent The preceding exercise shows that there is a bijection between the
\textit{points} of the closed interval $[0, 1]$ and the set of
\textit{maximal ideals} in the ring $R$ of all continuous functions on $[0, 1]$
given by $c \leftrightarrow M_c$. For any subset $X$ of $\R$ or, more generally,
for any completely regular topological space $X$, the map $c \mapsto M_c$ is an
\textit{injection} from $X$ to the set of maximal ideals of $R$, where $R$ is
the ring of all bounded continuous real valued functions on $X$ and $M_c$ is the 
maximal ideal of functions that vanish at $c$. Let $\beta(X)$ be the set of
maximal ideals of $R$. One can put a topology on $\beta(X)$ in such a way that
if we identity $X$ with its image in $\beta(X)$ then $X$ (in its given topology)
becomes a subspace of $\beta(X)$. Moreover, $\beta(X)$ is a compact space under
this topology and is called the \textit{Stone-$\check{C}$ech compactification}
of $X$.

\begin{enumerate}
%%%%%%%%%%%%%%%%%%%%%%%%%%%%%%%%%%%%%7.4.34%%%%%%%%%%%%%%%%%%%%%%%%%%%%%%%%%%%%%
   \item[7.4.34]  Let $R$ be the ring of all continuous functions from $\R$ to
                  $\R$ and for each $c \in \R$ let $M_c$ be the maximal ideal
                  $\{f \in R : f(c) = 0\}$.
                  \begin{enumerate}
                     \item Let $I$ be the collection of functions $f(x)$ in $R$
                           with \textit{compact support} (i.e., $f(x) = 0$ for
                           $|x|$ sufficiently large). Prove that $I$ is an ideal
                           of $R$ that is not a prime ideal.
                     \item Let $M$ be a maximal ideal of $R$ containing $I$
                           (properly, by (a)). Prove that $M \neq M_c$ for any
                           $c \in \R$ (cf. the preceding exercise).
                  \end{enumerate}
%%%%%%%%%%%%%%%%%%%%%%%%%%%%%%%%%%%%%7.4.35%%%%%%%%%%%%%%%%%%%%%%%%%%%%%%%%%%%%%
   \item[7.4.35]  Let $A = (a_1, a_2, \ldots, a_n)$ be a nonzero finitely
                  generated ideal of $R$. Prove that there is an ideal $B$ which
                  is maximal with respect to the property that it does not
                  contained $A$. [Use Zorn's Lemma.]
%%%%%%%%%%%%%%%%%%%%%%%%%%%%%%%%%%%%%7.4.35%%%%%%%%%%%%%%%%%%%%%%%%%%%%%%%%%%%%%
   \item[7.4.36]  Assume $R$ is commutative. Prove that the set of prime ideals
                  in $R$ has a minimal element with respect to inclusion
                  (possibly the zero ideal). [Use Zorn's Lemma.]
%%%%%%%%%%%%%%%%%%%%%%%%%%%%%%%%%%%%%7.4.36%%%%%%%%%%%%%%%%%%%%%%%%%%%%%%%%%%%%%
   \item[7.4.37]  A commutative ring $R$ is called a \textit{local ring} if it
                  has a unique maximal ideal. Prove that if $R$ is a local ring
                  with maximal ideal $M$ then every element of $R - M$ is a
                  unit. Prove conversely that if $R$ is a commutative ring with
                  1 in which the set of nonunits forms an ideal of $M$, then $R$
                  is a local ring with unique maximal ideal $M$.
%%%%%%%%%%%%%%%%%%%%%%%%%%%%%%%%%%%%%7.4.37%%%%%%%%%%%%%%%%%%%%%%%%%%%%%%%%%%%%%
   \item[7.4.38]  Prove that the ring of all rational numbers whose denominators
                  is odd is a local ring whose unique maximal ideal is the
                  principal ideal generated by 2.
%%%%%%%%%%%%%%%%%%%%%%%%%%%%%%%%%%%%%7.4.39%%%%%%%%%%%%%%%%%%%%%%%%%%%%%%%%%%%%%
   \item[7.4.39]  Following the notation of Exercise 7.1.26, ket $K$ be a field,
                  let $\nu$ be a discrete valuation on $K$ and let $R$ be the
                  valuation ring of $\nu$. For each integer $k \ge 0$ define
                  $A_k = \{r \in R : \nu(r) \ge k\} \cup \{0\}$.
                  \begin{enumerate}
                     \item Prove that $A_k$ is a principal ideal and that
                           $A_0 \supseteq A_1 \supseteq A_2 \supseteq \cdots$.
                     \item Prove that if $I$ is any nonzero ideal of $R$, then
                           $I = A_k$ for some $k \ge 0$. Deduce that $R$ is a
                           local ring with unique maximal ideal $A_1$.
                  \end{enumerate}
%%%%%%%%%%%%%%%%%%%%%%%%%%%%%%%%%%%%%7.4.40%%%%%%%%%%%%%%%%%%%%%%%%%%%%%%%%%%%%%
   \item[7.4.40]  Assume $R$ is commutative. Prove that the following are
                  equivalent: (see also Exercises 7.1.13 and 7.1.14)
                  \begin{enumerate}\renewcommand{\labelenumii}{(\roman{enumii})}
                     \item $R$ has exactly one prime ideal
                     \item every element of $R$ is either nilpotent or a unit
                     \item $R/\mathfrak{N}(R)$ is a field (cf. Exercise 7.3.29).
                  \end{enumerate}
%%%%%%%%%%%%%%%%%%%%%%%%%%%%%%%%%%%%%7.4.41%%%%%%%%%%%%%%%%%%%%%%%%%%%%%%%%%%%%%
   \item[7.4.41]  A proper ideal $Q$ of the commutative ring $R$ is called
                  \textit{primary} if whenever $ab \in Q$ and $a \notin Q$ then
                  $b^n \in Q$ for some positive integer $n$. (Note the symmetry
                  between $a$ and $b$ in this definition implies that if $Q$ is
                  a primary ideal and $ab \in Q$ with \textit{neither} $a$ nor
                  $b$ in $Q$, then a positive power of $a$ and a positive power
                  of $b$ both lie in $Q$.) Establish the following facts about
                  primary ideals.
                  \begin{enumerate}
                     \item The primary ideals of $\Z$ are 0 and $(p^n)$, where
                           $p$ is a prime and $n$ is a positive integer.
                     \item Every prime ideal of $R$ is a primary ideal.
                     \item An ideal $Q$ of $R$ is primary if and only if every
                           zero divisor in $R/Q$ is a nilpotent element of
                           $R/Q$.
                     \item If $Q$ is a primary ideal then rad($Q$) is a prime
                           ideal (cf. Exercise 7.4.30).
                  \end{enumerate}
\end{enumerate}
