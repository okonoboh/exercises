Let $R$ be a ring with identity $1 \neq 0$.
\begin{enumerate}
%%%%%%%%%%%%%%%%%%%%%%%%%%%%%%%%%%%%%7.4.1%%%%%%%%%%%%%%%%%%%%%%%%%%%%%%%%%%%%%%
   \item[7.4.1]   Let $L_{ij}$ be the left ideal of $M_n(R)$ consisting of
                  arbitrary entries in the $j^{\text{th}}$ column and zero in
                  all other entries and let $E_{ij}$ be the element of $M_n(R)$
                  whose $i, j$ entry is 1 and whose other entries are all 0. 
                  Prove that $L_j = M_n(R)E_{ij}$ for any $i$.
                  [See Exercise 7.2.6]
%%%%%%%%%%%%%%%%%%%%%%%%%%%%%%%%%%%%%7.4.2%%%%%%%%%%%%%%%%%%%%%%%%%%%%%%%%%%%%%%
   \item[7.4.2]   Assume $R$ is commutative. Prove that the augmentation ideal
                  in the group ring $RG$ is generated by $\{g - 1 : g \in G\}$.
                  Prove that if $G = \cyc{\sigma}$ is cyclic then the 
                  augmentation ideal is generated by $\sigma - 1$.
%%%%%%%%%%%%%%%%%%%%%%%%%%%%%%%%%%%%%7.4.3%%%%%%%%%%%%%%%%%%%%%%%%%%%%%%%%%%%%%%
   \item[7.4.3]   \begin{enumerate}
                     \item Let $p$ be a prime and let $G$ be an abelian group of
                           order $p^n$. Prove that the nilradical of the group
                           $\F_pG$ is the augmentation ideal
                           (cf. Exercise 7.3..29). [Use the preceding exercise.]
                     \item Let $G = \{g_1, \ldots, g_n\}$ be the finite group
                           and assume $R$ is commutative. Prove that if $r$ is
                           any element of the augmentation ideal of $RG$ then
                           $r(g_1 + \cdots + g_n) = 0$. [Use the preceding
                           exercise.]
                  \end{enumerate}
%%%%%%%%%%%%%%%%%%%%%%%%%%%%%%%%%%%%%7.4.4%%%%%%%%%%%%%%%%%%%%%%%%%%%%%%%%%%%%%%
   \item[7.4.4]   Assume $R$ is commutative. Prove that $R$ is a field if and
                  only if 0 is a maximal ideal.
%%%%%%%%%%%%%%%%%%%%%%%%%%%%%%%%%%%%%7.4.5%%%%%%%%%%%%%%%%%%%%%%%%%%%%%%%%%%%%%%
   \item[7.4.5]   Prove that if $M$ is an ideal such that $R/M$ is a field then
                  $M$ is a maximal ideal (do not asssume $R$ is commutative)
%%%%%%%%%%%%%%%%%%%%%%%%%%%%%%%%%%%%%7.4.6%%%%%%%%%%%%%%%%%%%%%%%%%%%%%%%%%%%%%%
   \item[7.4.6]   Prove that $R$ is a division ring if and only if its only left
                  ideals are (0) and $R$. (The analogous result holds when
                  ``left" is replaced by ``right.")
%%%%%%%%%%%%%%%%%%%%%%%%%%%%%%%%%%%%%7.4.7%%%%%%%%%%%%%%%%%%%%%%%%%%%%%%%%%%%%%%
   \item[7.4.7]   Let $R$ be a commutative ring with 1. Prove that the principal
                  ideal generated by $x$ in the polynomial ring $R[x]$ is a
                  prime ideal if and only if $R$ is an integral domain. Prove
                  that $(x)$ is a maximal ideal if and only if $R$ is a field.
%%%%%%%%%%%%%%%%%%%%%%%%%%%%%%%%%%%%%7.4.8%%%%%%%%%%%%%%%%%%%%%%%%%%%%%%%%%%%%%%
   \item[7.4.8]   Let $R$ be an integral domain. Prove that $(a) = (b)$ for some
                  elements $a, b \in R$, if and only if $a = ub$ for some unit
                  $u$ of $R$.
%%%%%%%%%%%%%%%%%%%%%%%%%%%%%%%%%%%%%7.4.9%%%%%%%%%%%%%%%%%%%%%%%%%%%%%%%%%%%%%%
   \item[7.4.9]   Let $R$ be the ring of all continuous functions on $[0, 1]$
                  and let $I$ be the collection of functions $f(x)$ in $R$ with
                  $f(1/3) = f(1/2) = 0$. Prove that $I$ is an ideal of $R$ but
                  is not a prime ideal.
%%%%%%%%%%%%%%%%%%%%%%%%%%%%%%%%%%%%%7.4.10%%%%%%%%%%%%%%%%%%%%%%%%%%%%%%%%%%%%%
   \item[7.4.10]  Assume $R$ is commutative. Prove that if $P$ is a prime ideal
                  of $R$ and $P$ contains no zero divisors then $R$ is an
                  integral domain.
%%%%%%%%%%%%%%%%%%%%%%%%%%%%%%%%%%%%%7.4.11%%%%%%%%%%%%%%%%%%%%%%%%%%%%%%%%%%%%%
   \item[7.4.11]  Assume $R$ is commutative. Let $I$ and $J$ be ideals of $R$
                  and assume $P$ is a prime ideal of $R$ that contains $IJ$ (for
                  example, if $P$ contains $I \cap J$). Prove either $I$ or $J$
                  is contained in $P$.
%%%%%%%%%%%%%%%%%%%%%%%%%%%%%%%%%%%%%7.4.12%%%%%%%%%%%%%%%%%%%%%%%%%%%%%%%%%%%%%
   \item[7.4.12]  Assume $R$ is commutative and suppose
                  $I = (a_1, a_2, \ldots, a_n)$ and
                  $J = (b_1, b_2, \ldots, b_m)$ are two finitely generated
                  ideals in $R$. Prove that the product ideal $IJ$ is finitely
                  generated by the elements $a_ib_j$ for $i = 1, 2, \ldots, n$,
                  and $j = 1, 2, \ldots, m$.
%%%%%%%%%%%%%%%%%%%%%%%%%%%%%%%%%%%%%7.4.13%%%%%%%%%%%%%%%%%%%%%%%%%%%%%%%%%%%%%
   \item[7.4.13]  Let $\varphi : R \rightarrow S$ be a homomorphism of
                  commutative rings.
                  \begin{enumerate}
                     \item Prove that if $P$ is a prime ideal of $S$ then either
                           $\varphi^{-1}(P) = R$ or $\varphi^{-1}(P)$ is a prime
                           ideal of $R$. Apply this to the special case when $R$
                           is a subring of $S$ and $\varphi$ is the includsion
                           homomorphism to deduce that if $P$ is a prime ideal
                           of $S$ then $P \cap R$ is either $R$ or a prime ideal
                           of $R$.
                     \item Prove that if $M$ is a maximal ideal of $S$ and
                           $\varphi$ is surjective then $\varphi{-1}(M)$ is a
                           maximal ideal of $R$. Give an example to show that 
                           this need not be the case if $\varphi$ is not
                           surjective.
                  \end{enumerate}
%%%%%%%%%%%%%%%%%%%%%%%%%%%%%%%%%%%%%7.4.14%%%%%%%%%%%%%%%%%%%%%%%%%%%%%%%%%%%%%
   \item[7.4.14]  Assume $R$ is commutative. Let $x$ be an indeterminate, let
                  $f(x)$ be a monic polynomial in $R[x]$ of degree $n \ge 1$ and
                  use the bar notation to denote passage to the quotient ring
                  $R[x]/(f(x))$.
                  \begin{enumerate}
                     \item Show that every element of $R[x]/(f(x))$ is of the
                           form $\overline{p(x)}$ for some polynomial
                           $p(x) \in R[x]$ of degree less than $n$, i.e.,
                           $$R[x]/(f(x)) = \{\overline{a_0} + \overline{a_1x}
                             + \cdots + \overline{a_{n-1}x^{n-1}} : a_0, a_1, 
                             \ldots, a_{n-1} \in R\}.$$
                     \item Prove that if $p(x)$ and $q(x)$ are distinct
                           polynomials in $R[x]$ which are both of degree less
                           than $n$, then $\overline{p(x)} \neq\overline{q(x)}$.
                     \item If $f(x) = a(x)b(x)$ where both $a(x)$ and $b(x)$
                           have degree less than $n$, prove that
                           $\overline{a(x)}$ is a zero divisor in $R[x]/(f(x))$.
                     \item If $f(x) = x^n - a$ for some nilpotent element
                           $a \in R$, prove that $\overline{x}$ is nilpotent in
                           $R[x]/(f(x))$.
                     \item Let $p$ be a prime, assume $R = \F_p$ and
                           $f(x) = x^p - a$ for some $a \in \F_p$. Prove that
                           $\overline{x-a}$ is nilpotent in $R[x]/(f(x))$.
                  \end{enumerate}
%%%%%%%%%%%%%%%%%%%%%%%%%%%%%%%%%%%%%7.4.15%%%%%%%%%%%%%%%%%%%%%%%%%%%%%%%%%%%%%
   \item[7.4.15]  Let $x^2 + x + 1$ be an element of the polynomial ring
                  $E = \F_2[x]$ and use the bar notation to denote passage to 
                  the quotient ring $F_2[x]/(x^2 + x + 1)$.
                  \begin{enumerate}
                     \item Prove that $\overline{E}$ has 4 elements:
                           $\overline{0}$, $\overline{1}$, $\overline{x}$, and
                           $\overline{x+1}$.
                     \item Write out the $4 \times 4$ addition table for
                           $\overline{E}$ and deduce that the additive group
                           $\overline{E}$ is isomorphic to the Klein 4-group.
                     \item Write out the $4 \times 4$ multiplication table for
                           $\overline{E}$ and prove that $\overline{E}^\times$
                           is isomorphic to the cyclic group of order 3. Deduce
                           that $\overline{E}$ is a field.
                  \end{enumerate}
%%%%%%%%%%%%%%%%%%%%%%%%%%%%%%%%%%%%%7.4.16%%%%%%%%%%%%%%%%%%%%%%%%%%%%%%%%%%%%%
   \item[7.4.16]  Let $x^4 - 16$ be an element of the polynomial ring
                  $E = \Z[x]$ and use the bar notation to denote passage to the
                  quotient ring $\Z[x]/(x^4 - 16)$.
                  \begin{enumerate}
                     \item Find a polynomial of degree $\le 3$ that is congruent
                           to $7x^{13} - 11x^9 + 5x^5 - 2x^3 + 3$ modulo
                           $(x^4 - 16)$.
                     \item Prove that $\overline{x - 2}$ and $\overline{x + 2}$
                           are zero divisors in $\overline{E}$/
                  \end{enumerate}
%%%%%%%%%%%%%%%%%%%%%%%%%%%%%%%%%%%%%7.4.17%%%%%%%%%%%%%%%%%%%%%%%%%%%%%%%%%%%%%
   \item[7.4.17]  Let $x^3 - 2x + 1$ be an element of the polynomial ring
                  $E = \Z[x}$ and use the bar notation to denote passage to the
                  quotient ring $\Z[x]/(x^3 - 2x + 1)$. Let
                  $p(x) = 2x^7 - 7x^5 + 4x^3 - 9x + 1$ and let
                  $q(x) = (x - 1)^4$.
                  \begin{enumerate}
                     \item Express each of the following elements of
                           $\overline{E}$ in the form $\overline{f(x)}$ for some
                           polynomial $f(x)$ of degree $\le 2$:
                           $$\overline{p(x)}, \overline{q(x)},
                             \overline{p(x) + q(x)}, \text{ and }
                             \overline{p(x)q(x)}.$$
                     \item Prove that $\overline{E}$ is not an integral domain.
                     \item Prove that $\overline{x}$ is a unit in
                           $\overline{E}$.
                  \end{enumerate}
%%%%%%%%%%%%%%%%%%%%%%%%%%%%%%%%%%%%%7.4.18%%%%%%%%%%%%%%%%%%%%%%%%%%%%%%%%%%%%%
   \item[7.4.18]  Prove that if $R$ is an integral domain and $R[[x]]$ is the
                  ring of formal power series in the indeterminate $x$ then the
                  principal ideal generated by $x$ is a prime ideal
                  (cf Exercise 7.2.3). Prove that the principal ideal generated
                  by $x$ is a maximal ideal if and only if $R$ is a field.
%%%%%%%%%%%%%%%%%%%%%%%%%%%%%%%%%%%%%7.4.19%%%%%%%%%%%%%%%%%%%%%%%%%%%%%%%%%%%%%
   \item[7.4.19]  Let $R$ be a finite commutative ring with identity. Prove that
                  every prime ideal of $R$ is a maximal ideal.
%%%%%%%%%%%%%%%%%%%%%%%%%%%%%%%%%%%%%7.4.20%%%%%%%%%%%%%%%%%%%%%%%%%%%%%%%%%%%%%
   \item[7.4.20]  Prove that a nonzero finite commutative ring that has no zero
                  divisors is a field (if the ring has an identity, this is
                  Corollary 3, so do not assume the ring has a 1).
%%%%%%%%%%%%%%%%%%%%%%%%%%%%%%%%%%%%%7.4.21%%%%%%%%%%%%%%%%%%%%%%%%%%%%%%%%%%%%%
   \item[7.4.21]  Prove that a finite ring with identity $1 \neq 0$ that has no
                  zero divisors is a field (you may qupte Wedderburn's Theorem).
%%%%%%%%%%%%%%%%%%%%%%%%%%%%%%%%%%%%%7.4.22%%%%%%%%%%%%%%%%%%%%%%%%%%%%%%%%%%%%%
   \item[7.4.22]  Let $p \in \Z^+$ be a prime and let the $\F_p$ Quaternions be
                  defined by
                  $$a + bi + cj + dk \qquad a, b, c, d \in \Z/p\Z$$
                  where addition is componentwise and multiplication is defined
                  using the same relations $i, j, k$ as for the real
                  Quaternions.
                  \begin{enumerate}
                     \item Prove that the $\F_p$ Quaternions are a homomorphic
                           image of the integral Quaternions (cf. Section 7.1).
                     \item Prove that the $\F_p$ Quaternions contain zero
                           divisors (and so they cannot be a division ring).
                           [Use the preceding exercise.]
                  \end{enumerate}
%%%%%%%%%%%%%%%%%%%%%%%%%%%%%%%%%%%%%7.4.23%%%%%%%%%%%%%%%%%%%%%%%%%%%%%%%%%%%%%
   \item[7.4.23]  Prove that in a Boolean ring (cf. Exercise 7.1.15) every prime
                  ideal is a maximal ideal.
%%%%%%%%%%%%%%%%%%%%%%%%%%%%%%%%%%%%%7.4.24%%%%%%%%%%%%%%%%%%%%%%%%%%%%%%%%%%%%%
   \item[7.4.24]  Let $\varphi : R \rightarrow S$ be a ring homomorphism.
                  \begin{enumerate}
                     \item Prove that if $J$ is an ideal of $S$ then
                           $\varphi^{-1}(J)$ is an ideal of $R$. Apply this to
                           the special case when $R$ is a subring of $S$ and
                           $\varphi$ is the inclusion homomorphism to deduce
                           that if $J$ is an ideal of $S$ then $J \cap R$ is an
                           ideal of $R$.
                     \item Prove that if $\varphi$ is surjective and $I$ is an
                           ideal of $R$ then $\varphi(I)$ is an ideal of $S$.
                           Give an example where this fails if $\varphi$ is not
                           surjective.
                  \end{enumerate}
%%%%%%%%%%%%%%%%%%%%%%%%%%%%%%%%%%%%%7.4.25%%%%%%%%%%%%%%%%%%%%%%%%%%%%%%%%%%%%%
   \item[7.4.25]  Assume $R$ is a commutative ring with 1. Prove that the
                  Binomial Theorem
                  $$(a + b)^n = \sum_{k=0}^n\binom{n}{k}a^kb^{n-k}$$
                  holds in $R$, where the binomial coefficient $\binom{n}{k}$ is
                  interpreted in $R$ as the sum $1 + 1 + \cdots + 1$ of the
                  identity 1 in $R$ taken $\binom{n}{k}$ times.
%%%%%%%%%%%%%%%%%%%%%%%%%%%%%%%%%%%%%7.4.26%%%%%%%%%%%%%%%%%%%%%%%%%%%%%%%%%%%%%
   \item[7.4.26]  The \textit{characteristic} of a ring $R$ is the smallest
                  positive integer $n$ such that $1 + 1 + \cdots + 1 = 0$
                  ($n$ times) in $R$; if no such integer exists the 
                  characteristic of $R$ is said to be 0. For example,
                  $\Z/n\Z$ is a ring of characteristic $n$ for each positive 
                  integer $n$ and $\Z$ is a ring of characteristic 0.
                  \begin{enumerate}
                     \item Prove that the map $\Z \rightarrow R$ defined by
                           \begin{equation*}
                              k \mapsto \left\{
                                 \begin{array}{ll}
                                    1 + 1 + \cdots + 1 (k \text{ times}) &
                                       \text{if } k > 0 \\
                                    0 & \text{if } k = 0 \\
                                    -1 - 1 - \cdots - 1 (-k \text{ times}) &
                                       \text{if } k < 0
                                 \end{array} \right.
                           \end{equation*}
                           is a ring homomorphism whose kernel is $n\Z$, where
                           $n$ is the characteristic of $R$ (this explains the
                           use of the terminology ``characteristic 0" instead of
                           the archaic phrase ``characteristic $\infty$" for
                           rings in which no sum of 1's is zero).
                     \item Determine the characteristics of the rings $\Q$,
                           $\Z[x]$, $\Z/n\Z[x]$.
                     \item Prove that if $p$ is a prime and if $R$ is a
                           commutative ring of characteristic $p$, then
                           $(a + b)^p = a^p + b^p$ for all $a, b \in R$.
                  \end{enumerate}
%%%%%%%%%%%%%%%%%%%%%%%%%%%%%%%%%%%%%7.4.27%%%%%%%%%%%%%%%%%%%%%%%%%%%%%%%%%%%%%
   \item[7.4.27]  Prove that a nonzero Boolean ring has characteristic 2
                  (cf. Exericse 7.1.15).
%%%%%%%%%%%%%%%%%%%%%%%%%%%%%%%%%%%%%7.4.28%%%%%%%%%%%%%%%%%%%%%%%%%%%%%%%%%%%%%
   \item[7.4.28]  Prove that an integral domain has characteristic $p$, where
                  $p$ is either a prine or 0 (cf Exercise 7.4.26).
%%%%%%%%%%%%%%%%%%%%%%%%%%%%%%%%%%%%%7.4.29%%%%%%%%%%%%%%%%%%%%%%%%%%%%%%%%%%%%%
   \item[7.4.29]  Let $R$ be a commutative ring. Recall (cf. Exercise 7.1.13)
                  that an element $x \in R$ is nilpotent if $x^n = 0$ for some
                  $n \in \Z^+$. Prove that the set of nilpotent elements form an 
                  ideal---called the \textit{nilradical} of $R$ and denoted by
                  $\mathfrak{N}(R)$. [Use the Binomial Theorem to show
                  $\mathfrak{N}(R)$ is closed under addition.]
%%%%%%%%%%%%%%%%%%%%%%%%%%%%%%%%%%%%%7.4.30%%%%%%%%%%%%%%%%%%%%%%%%%%%%%%%%%%%%%
   \item[7.4.30]  Prove that if $R$ is a commutative ring and $\mathfrak{N}(R)$
                  is its nilradical (cf. the preceding exercise) then zero is
                  the only nilpotent element of $R/\mathfrak{N}(R)$ i.e., prove
                  that $\mathfrak{N}(R/\mathfrak{N}(R)) = 0$.
%%%%%%%%%%%%%%%%%%%%%%%%%%%%%%%%%%%%%7.4.31%%%%%%%%%%%%%%%%%%%%%%%%%%%%%%%%%%%%%
   \item[7.4.31]  Prove that the elements $\left(\begin{tabular}{@{}cc@{}}
                     0 & 1 \\
                     0 & 0
                  \end{tabular}\right)$ and $\left(\begin{tabular}{@{}cc@{}}
                     0 & 0 \\
                     1 & 0
                  \end{tabular}\right)$ are nilpotent elements of $M_2(\Z)$
                  whose sum is not nilpotent (note that these two matrices do
                  not commute). Deduce that the set of nilpotent elements in the
                  noncommutative ring $M_2(\Z)$ is not an ideal.
%%%%%%%%%%%%%%%%%%%%%%%%%%%%%%%%%%%%%7.4.32%%%%%%%%%%%%%%%%%%%%%%%%%%%%%%%%%%%%%
   \item[7.4.32]  Let $\varphi : R \rightarrow S$ be a homomorphism of rings.
                  Prove that if $x$ is a nilpotent element of $R$ then
                  $\varphi(x)$ is nilpotent in $S$.
%%%%%%%%%%%%%%%%%%%%%%%%%%%%%%%%%%%%%7.4.33%%%%%%%%%%%%%%%%%%%%%%%%%%%%%%%%%%%%%
   \item[7.4.33]  Assume $R$ is commutative. Let
                  $p(x) = a_nx^n + a_{n-1}x^{n-1} + \cdots + a_1x + a_0$ be an
                  element of the polynomial ring $R[x]$.
                  \begin{enumerate}
                     \item Prove that $p(x)$ is unit in $R[x]$ if and only if
                           $a_0$ is a unit and $a_1, a_2, \ldots, a_n$ are
                           nilpotent in $R$. [See Exercise 7.1.14]
                     \item Prove that $p(x)$ is nilpotent in $R[x]$ if and only
                           if $a_0, a_1, \ldots, a_n$ are nilpotent elements of
                           $R$.
                  \end{enumerate}
%%%%%%%%%%%%%%%%%%%%%%%%%%%%%%%%%%%%%7.4.34%%%%%%%%%%%%%%%%%%%%%%%%%%%%%%%%%%%%%
   \item[7.4.34]  Let $I$ and $J$ be ideals of $R$.
                  \begin{enumerate}
                     \item Prove that $I + J$ is the smallest ideal of $R$
                           containing both $I$ and $J$.
                     \item Prove that $IJ$ is an ideal contained in $I \cap J$.
                     \item Give an example where $IJ \neq I \cap J$.
                     \item Prove that if $R$ is commutative and if $I + J = R$
                           then $IJ = I \cap J$.
                  \end{enumerate}
%%%%%%%%%%%%%%%%%%%%%%%%%%%%%%%%%%%%%7.4.35%%%%%%%%%%%%%%%%%%%%%%%%%%%%%%%%%%%%%
   \item[7.4.35]  Let $I, J, K$ be ideals of $R$.
                  \begin{enumerate}
                     \item Prove that $I(J + K) = IJ + IK$ and
                           $(I + J)K = IK + JK$.
                     \item Prove that if $J \subseteq I$ then
                           $I \cap (J + K) = J + (I \cap K)$.
                  \end{enumerate}
%%%%%%%%%%%%%%%%%%%%%%%%%%%%%%%%%%%%%7.4.36%%%%%%%%%%%%%%%%%%%%%%%%%%%%%%%%%%%%%
   \item[7.4.36]  Show that if $I$ is the ideal of all polynomials in $\Z[x]$
                  with zero constant term then $I^n = \{a_nx^n + a_{n+1}x^{n+1} 
                  + \cdots + a_{n+m}x^{n+m} : a_i \in \Z, m \ge 0\}$ is the set
                  of polynomials whose first nonzero term has degree at least
                  $n$.
%%%%%%%%%%%%%%%%%%%%%%%%%%%%%%%%%%%%%7.4.37%%%%%%%%%%%%%%%%%%%%%%%%%%%%%%%%%%%%%
   \item[7.4.37]  An ideal $N$ is called $\textit{nilpotent}$ if $N^n$ is the
                  zero ideal for some $n \ge 1$. Prove that the ideal
                  $p\Z/p^m\Z$ is a nilpotent ideal in the ring $\Z/p^m\Z$.
\end{enumerate}
