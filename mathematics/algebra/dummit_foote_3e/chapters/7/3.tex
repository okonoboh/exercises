Let $R$ be a ring with identity $1 \neq 0$.

\begin{enumerate}
%%%%%%%%%%%%%%%%%%%%%%%%%%%%%%%%%%%%%7.3.1%%%%%%%%%%%%%%%%%%%%%%%%%%%%%%%%%%%%%%
   \item[7.3.1]   Prove that the rings $2\Z$ and $3\Z$ are not isomorphic.
%%%%%%%%%%%%%%%%%%%%%%%%%%%%%%%%%%%%%7.3.2%%%%%%%%%%%%%%%%%%%%%%%%%%%%%%%%%%%%%%
   \item[7.3.2]   Prove that the rings $\Z[x]$ and $\Q[x]$ are not isomorphic.
%%%%%%%%%%%%%%%%%%%%%%%%%%%%%%%%%%%%%7.3.3%%%%%%%%%%%%%%%%%%%%%%%%%%%%%%%%%%%%%%
   \item[7.3.3]   Find all homomorphic images of $\Z$.
%%%%%%%%%%%%%%%%%%%%%%%%%%%%%%%%%%%%%7.3.4%%%%%%%%%%%%%%%%%%%%%%%%%%%%%%%%%%%%%%
   \item[7.3.4]   Find all ring homomorphisms from $\Z$ to $\Z/30\Z$. In each
                  case describe the kernel and the image.
%%%%%%%%%%%%%%%%%%%%%%%%%%%%%%%%%%%%%7.3.5%%%%%%%%%%%%%%%%%%%%%%%%%%%%%%%%%%%%%%
   \item[7.3.5]   Describe all ring homomorphisms from the ring $\Z \times \Z$
                  to $\Z$. In each case describe the kernel and the image.
%%%%%%%%%%%%%%%%%%%%%%%%%%%%%%%%%%%%%7.3.6%%%%%%%%%%%%%%%%%%%%%%%%%%%%%%%%%%%%%%
   \item[7.3.6]   Decide which of the following are ring homomorphisms from
                  $M_2(\Z)$ to $\Z$:
                  \begin{enumerate}
                     \item $\left(\begin{tabular}{@{}cc@{}}
                              $a$ & $b$ \\
                              $c$ & $d$
                           \end{tabular}\right) \mapsto a$ (projection onto the
                           1,1 entry)
                     \item $\left(\begin{tabular}{@{}cc@{}}
                              $a$ & $b$ \\
                              $c$ & $d$
                           \end{tabular}\right) \mapsto a + d$ (the
                           \textit{trace} of the matrix)
                     \item $\left(\begin{tabular}{@{}cc@{}}
                              $a$ & $b$ \\
                              $c$ & $d$
                           \end{tabular}\right) \mapsto ad - bc$ (the
                           \textit{determinant} of the matrix).
                  \end{enumerate}
%%%%%%%%%%%%%%%%%%%%%%%%%%%%%%%%%%%%%7.3.7%%%%%%%%%%%%%%%%%%%%%%%%%%%%%%%%%%%%%%
   \item[7.3.7]   Let $R = \left\{\left(\begin{tabular}{@{}cc@{}}
                     $a$ & $b$ \\
                     0 & $d$
                  \end{tabular}\right) : a, b, d \in \Z\right\}$ be the subring
                  of $M_2(\Z)$ of upper triangular matrices. Prove that the map
                  $$\varphi : R \rightarrow \Z \times \Z \text{ defined by }
                    \varphi : \left(\begin{tabular}{@{}cc@{}}
                     $a$ & $b$ \\
                      0  & $d$
                  \end{tabular}\right) \rightarrow (a, d)$$
                  is a surjective homomorphism and describe its kernel.
%%%%%%%%%%%%%%%%%%%%%%%%%%%%%%%%%%%%%7.3.8%%%%%%%%%%%%%%%%%%%%%%%%%%%%%%%%%%%%%%
   \item[7.3.8]   Decide which of the following are ideals of the ring
                  $\Z \times \Z$:
                  \begin{enumerate}
                     \item $\{(a, a) : a \in \Z\}$
                     \item $\{(2a, 2b) : a, b \in \Z\}$
                     \item $\{(2a, 0) : a \in \Z\}$
                     \item $\{(a, -a) : a \in \Z\}$.
                  \end{enumerate}
%%%%%%%%%%%%%%%%%%%%%%%%%%%%%%%%%%%%%7.3.9%%%%%%%%%%%%%%%%%%%%%%%%%%%%%%%%%%%%%%
   \item[7.3.9]   Decide which of the sets in Exercise 6 of Section 1 are ideals
                  of the ring of all functions from $[0, 1]$ to $\R$.
%%%%%%%%%%%%%%%%%%%%%%%%%%%%%%%%%%%%%7.3.10%%%%%%%%%%%%%%%%%%%%%%%%%%%%%%%%%%%%%
   \item[7.3.10]  Decide which of the following are ideals of the ring $\Z[x]$:
                  \begin{enumerate}
                     \item the set of all polynomials whose constant term is a
                           multiple of 3
                     \item the set of all polynomials whose coefficient of $x^2$
                           is a multiple of 3
                     \item the set of all polynomials whose constant term,
                           coefficient of $x$ and coefficient of $x^2$ are zero
                     \item $\Z[x^2]$ (i.e., the polynomials in which only even
                           powers of $x$ appear)
                     \item the set of polynomials whose coefficients sum to
                           zero
                     \item the set of polynomials $p(x)$ such that $p'(0) = 0$,
                           where $p'(x)$ is the usual first derivative of $p(x)$
                           with respect to $x$.
                  \end{enumerate}
%%%%%%%%%%%%%%%%%%%%%%%%%%%%%%%%%%%%%7.3.11%%%%%%%%%%%%%%%%%%%%%%%%%%%%%%%%%%%%%
   \item[7.3.11]  Let $R$ be the ring of all continuous real valued functions on
                  the closed interval $[0, 1]$. Prove that the map
                  $\varphi : R \rightarrow \R$ defined by
                  $\varphi(f) = \int_0^1f(t)dt$ is a homomorphism of additive
                  groups but not a ring homomorphism.
%%%%%%%%%%%%%%%%%%%%%%%%%%%%%%%%%%%%%7.3.12%%%%%%%%%%%%%%%%%%%%%%%%%%%%%%%%%%%%%
   \item[7.3.12]  Let $D$ be an integer that is not a perfect square in $\Z$ and
                  let $S = \left\{\left(\begin{tabular}{@{}cc@{}}
                     $a$ & $b$ \\
                     $Db$  & $d$
                  \end{tabular}\right) : a, b \in \Z\right\}$.
                  \begin{enumerate}
                     \item Prove that $S$ is a subring of $M_2(\Z)$.
                     \item If $D$ is not a perfect square in $\Z$ prove that the
                           map $\varphi : \Z[\sqrt{D}] \rightarrow S$ defined by
                           $\varphi(a + b\sqrt{D}) =
                            \left(\begin{tabular}{@{}cc@{}}
                               $a$ & $b$ \\
                               $Db$  & $d$
                           \end{tabular}\right)$ is a ring isomorphism.
                     \item If $D \equiv 1$ mod 4 is squarefree, prove that the
                           set $\left\{\left(\begin{tabular}{@{}cc@{}}
                              $a$ & $b$ \\
                              $(D-1)b/4$  & $a+b$
                           \end{tabular}\right) : a, b \in \Z\right\}$ is a
                           subring of $M_2(\Z)$ and is isomorphic to the 
                           quadratic integer ring $\mathcal{O}$.
                  \end{enumerate}
%%%%%%%%%%%%%%%%%%%%%%%%%%%%%%%%%%%%%7.3.13%%%%%%%%%%%%%%%%%%%%%%%%%%%%%%%%%%%%%
   \item[7.3.13]  Prove that the ring $M_2(\R)$ contains a subring that is
                  isomorphic to $\C$.
%%%%%%%%%%%%%%%%%%%%%%%%%%%%%%%%%%%%%7.3.14%%%%%%%%%%%%%%%%%%%%%%%%%%%%%%%%%%%%%
   \item[7.3.14]  Prove that the ring $M_4(\R)$ contains a subring that is
                  isomorphic to the real Hamilton Quaternions, $\mathbb{H}$.
%%%%%%%%%%%%%%%%%%%%%%%%%%%%%%%%%%%%%7.3.15%%%%%%%%%%%%%%%%%%%%%%%%%%%%%%%%%%%%%
   \item[7.3.15]  Let $X$ be a nonempty set and let $\mathcal{P}(X)$ be the
                  Boolean ring of all subsets of $X$ defined in Exercise 7.1.21.
                  Let $R$ be the ring of all functions from $X$ into $\Z/2\Z$.
                  For each $A \in \mathcal{P}(X)$ define the function
                  $$\chi_A : X \rightarrow \Z/2\Z \quad \text{by} \quad
                    \chi_A(x) = \left\{
                       \begin{array}{ll}
                          1 & \text{if } x \in A \\
                          0 & \text{if } x \notin A
                       \end{array} \right.
                  $$
                  ($\chi_A$ is called the \textit{characteristic function of }
                  $A$ with values in $\Z/2\Z$). Prove that the map
                  $\mathcal{P}(X) \rightarrow R$ defined by $A \mapsto \chi_A$
                  is a ring isomorphism.
%%%%%%%%%%%%%%%%%%%%%%%%%%%%%%%%%%%%%7.3.16%%%%%%%%%%%%%%%%%%%%%%%%%%%%%%%%%%%%%
   \item[7.3.16]  Let $\varphi : R \rightarrow S$ be a surjective homomorphism
                  of rings. Prove that the image of the center of $R$ is
                  contained in the center of $S$ (cf. Exercise 7.1.7).
%%%%%%%%%%%%%%%%%%%%%%%%%%%%%%%%%%%%%7.3.17%%%%%%%%%%%%%%%%%%%%%%%%%%%%%%%%%%%%%
   \item[7.3.17]  Let $R$ and $S$ be nonzero rings with identity and denote
                  their respective identities by $1_R$ and $1_S$. Let
                  $\varphi : R \rightarrow S$ be a nonzero homomorphism of
                  rings.
                  \begin{enumerate}
                     \item Prove that if $\varphi(1_R) \neq 1_S$ then
                           $\varphi(1_R)$ is a zero divisor in $S$. Deduce that
                           if $S$ is an integral domain then every nonzero ring
                           homomorphism from $R$ to $S$ sends the identity of
                           $R$ to the identity of $S$.
                     \item Prove that if $\varphi(1_R) = 1_S$ then $\varphi(u)$
                           is a unit in $S$ and that
                           $\varphi(u^{-1}) = \varphi(u)^{-1}$ for each unit $u$ 
                           of $R$.
                  \end{enumerate}
%%%%%%%%%%%%%%%%%%%%%%%%%%%%%%%%%%%%%7.3.18%%%%%%%%%%%%%%%%%%%%%%%%%%%%%%%%%%%%%
   \item[7.3.18]  \begin{enumerate}
                     \item If $I$ and $J$ are ideals of $R$ prove that their
                           intersection $I \cap J$ is also an ideal of $R$.
                     \item Prove that the intersection of an arbitrary nonempty
                           collection of ideals is again an ideal (do not assume
                           that the collection is countable).
                  \end{enumerate}
%%%%%%%%%%%%%%%%%%%%%%%%%%%%%%%%%%%%%7.3.19%%%%%%%%%%%%%%%%%%%%%%%%%%%%%%%%%%%%%
   \item[7.3.19]  Prove that if $I_1 \subseteq I_2 \subseteq \cdots$ are ideals
                  of $R$ then $\cup_{n=1}^\infty I_n$ is an ideal of $R$.
%%%%%%%%%%%%%%%%%%%%%%%%%%%%%%%%%%%%%7.3.20%%%%%%%%%%%%%%%%%%%%%%%%%%%%%%%%%%%%%
   \item[7.3.20]  Let $I$ be an ideal of $R$ and let $S$ be a subring of $R$.
                  Prove that $I \cap S$ is an ideal of $S$. Show by example that
                  not every ideal of a subring $S$ of a ring $R$ need be of the
                  form $I \cap S$ for some ideal $I$ of $R$.
%%%%%%%%%%%%%%%%%%%%%%%%%%%%%%%%%%%%%7.3.21%%%%%%%%%%%%%%%%%%%%%%%%%%%%%%%%%%%%%
   \item[7.3.21]  Prove that every (two-sided) ideal of $M_n(R)$ is equal to
                  $M_n(J)$ for some (two-sided) ideal $J$ of $R$. [Use Exercise
                  6(c) of Section 2 to show first that the set of entries of
                  matrices in an ideal of $M_n(R)$ form an ideal in $R$.]
%%%%%%%%%%%%%%%%%%%%%%%%%%%%%%%%%%%%%7.3.22%%%%%%%%%%%%%%%%%%%%%%%%%%%%%%%%%%%%%
   \item[7.3.22]  Let $a$ be an element of the ring $R$.
                  \begin{enumerate}
                     \item Prove that $\{x \in R : ax = 0\}$ is right ideal and
                           $\{y \in R : ya = 0\}$ is a left ideal (called
                           respectively the right and left \textit{annihilators}
                           of $a$ in $R$).
                     \item Prove that if $L$ is a left ideal of $R$ then
                           $\{x \in R : xa = 0 \text{ for all} a \in L\}$ is a
                           two-sided ideal (called the left \textit{annihilator}
                           of $L$ in $R$).
                  \end{enumerate}
%%%%%%%%%%%%%%%%%%%%%%%%%%%%%%%%%%%%%7.3.23%%%%%%%%%%%%%%%%%%%%%%%%%%%%%%%%%%%%%
   \item[7.3.23]  Let $S$ be a subring of $R$ and let $I$ be an ideal of $R$.
                  Prove that if $S \cap I = 0$ then $\overline{S} \cong S$,
                  where the bar denotes passage to $R/I$.
%%%%%%%%%%%%%%%%%%%%%%%%%%%%%%%%%%%%%7.3.24%%%%%%%%%%%%%%%%%%%%%%%%%%%%%%%%%%%%%
   \item[7.3.24]  Let $\varphi : R \rightarrow S$ be a ring homomorphism.
                  \begin{enumerate}
                     \item Prove that if $J$ is an ideal of $S$ then
                           $\varphi^{-1}(J)$ is an ideal of $R$. Apply this to
                           the special case when $R$ is a subring of $S$ and
                           $\varphi$ is the inclusion homomorphism to deduce
                           that if $J$ is an ideal of $S$ then $J \cap R$ is an
                           ideal of $R$.
                     \item Prove that if $\varphi$ is surjective and $I$ is an
                           ideal of $R$ then $\varphi(I)$ is an ideal of $S$.
                           Give an example where this fails if $\varphi$ is not
                           surjective.
                  \end{enumerate}
%%%%%%%%%%%%%%%%%%%%%%%%%%%%%%%%%%%%%7.3.25%%%%%%%%%%%%%%%%%%%%%%%%%%%%%%%%%%%%%
   \item[7.3.25]  Assume $R$ is a commutative ring with 1. Prove that the
                  Binomial Theorem
                  $$(a + b)^n = \sum_{k=0}^n\binom{n}{k}a^kb^{n-k}$$
                  holds in $R$, where the binomial coefficient $\binom{n}{k}$ is
                  interpreted in $R$ as the sum $1 + 1 + \cdots + 1$ of the
                  identity 1 in $R$ taken $\binom{n}{k}$ times.
%%%%%%%%%%%%%%%%%%%%%%%%%%%%%%%%%%%%%7.3.26%%%%%%%%%%%%%%%%%%%%%%%%%%%%%%%%%%%%%
   \item[7.3.26]  The \textit{characteristic} of a ring $R$ is the smallest
                  positive integer $n$ such that $1 + 1 + \cdots + 1 = 0$
                  ($n$ times) in $R$; if no such integer exists the 
                  characteristic of $R$ is said to be 0. For example $\Z/n\Z$ is
                  a ring of characteristic $n$ for each positive integer $n$ and
                  $\Z$ is a ring of characteristic 0.
                  \begin{enumerate}
                     \item Prove that the map $\Z \rightarrow R$ defined by
                           \begin{equation*}
                              k \mapsto \left\{
                                 \begin{array}{ll}
                                    1 + 1 + \cdots + 1 (k \text{ times}) &
                                       \text{if } k > 0 \\
                                    0 & \text{if } k = 0 \\
                                    -1 - 1 - \cdots - 1 (-k \text{ times}) &
                                       \text{if } k < 0
                                 \end{array} \right.
                           \end{equation*}
                           is a ring homomorphism whose kernel is $n\Z$, where
                           $n$ is the characteristic of $R$ (this explains the
                           use of the terminology ``characteristic 0" instead of
                           the archaic phrase ``characteristic $\infty$" for
                           rings in which no sum of 1's is zero).
                     \item Determine the characteristics of the rings $\Q$,
                           $\Z[x]$, $\Z/n\Z[x]$.
                     \item Prove that if $p$ is a prime and if $R$ is a
                           commutative ring of characteristic $p$, then
                           $(a + b)^p = a^p + b^p$ for all $a, b \in R$.
                  \end{enumerate}
%%%%%%%%%%%%%%%%%%%%%%%%%%%%%%%%%%%%%7.3.27%%%%%%%%%%%%%%%%%%%%%%%%%%%%%%%%%%%%%
   \item[7.3.27]  Prove that a nonzero Boolean ring has characteristic 2
                  (cf. Exericse 7.1.15).
%%%%%%%%%%%%%%%%%%%%%%%%%%%%%%%%%%%%%7.3.28%%%%%%%%%%%%%%%%%%%%%%%%%%%%%%%%%%%%%
   \item[7.3.28]  Prove that an integral domain has characteristic $p$, where
                  $p$ is either a prine or 0 (cf Exercise 7.3.26).
%%%%%%%%%%%%%%%%%%%%%%%%%%%%%%%%%%%%%7.3.29%%%%%%%%%%%%%%%%%%%%%%%%%%%%%%%%%%%%%
   \item[7.3.29]  Let $R$ be a commutative ring. Recall (cf. Exercise 13,
                  Section 1) that an element $x \in R$ is nilpotent if $x^n = 0$
                  for some $n \in \Z^+$. Prove that the set of nilpotent
                  elements form an ideal---called the \textit{nilradical} of $R$
                  and denoted by $\mathfrak{N}(R)$. [Use the Binomial Theorem to
                  show $\mathfrak{N}(R)$ is closed under addition.]
%%%%%%%%%%%%%%%%%%%%%%%%%%%%%%%%%%%%%7.3.30%%%%%%%%%%%%%%%%%%%%%%%%%%%%%%%%%%%%%
   \item[7.3.30]  Prove that if $R$ is a commutative ring and $\mathfrak{N}(R)$
                  is its nilradical (cf. the preceding exercise) then zero is
                  the only nilpotent element of $R/\mathfrak{N}(R)$ i.e., prove
                  that $\mathfrak{N}(R/\mathfrak{N}(R)) = 0$.
%%%%%%%%%%%%%%%%%%%%%%%%%%%%%%%%%%%%%7.3.31%%%%%%%%%%%%%%%%%%%%%%%%%%%%%%%%%%%%%
   \item[7.3.31]  Prove that the elements $\left(\begin{tabular}{@{}cc@{}}
                     0 & 1 \\
                     0 & 0
                  \end{tabular}\right)$ and $\left(\begin{tabular}{@{}cc@{}}
                     0 & 0 \\
                     1 & 0
                  \end{tabular}\right)$ are nilpotent elements of $M_2(\Z)$
                  whose sum is not nilpotent (note that these two matrices do
                  not commute). Deduce that the set of nilpotent elements in the
                  noncommutative ring $M_2(\Z)$ is not an ideal.
%%%%%%%%%%%%%%%%%%%%%%%%%%%%%%%%%%%%%7.3.32%%%%%%%%%%%%%%%%%%%%%%%%%%%%%%%%%%%%%
   \item[7.3.32]  Let $\varphi : R \rightarrow S$ be a homomorphism of rings.
                  Prove that if $x$ is a nilpotent element of $R$ then
                  $\varphi(x)$ id nilpotent in $S$.
%%%%%%%%%%%%%%%%%%%%%%%%%%%%%%%%%%%%%7.3.33%%%%%%%%%%%%%%%%%%%%%%%%%%%%%%%%%%%%%
   \item[7.3.33]  Assume $R$ is commutative. Let
                  $p(x) = a_nx^n + a_{n-1}x^{n-1} + \cdots + a_1x + a_0$ be an
                  element of the polynomial ring $R[x]$.
                  \begin{enumerate}
                     \item Prove that $p(x)$ is unit in $R[x]$ if and only if
                           $a_0$ is a unit and $a_1, a_2, \ldots, a_n$ are
                           nilpotent in $R$. [See Exercise 7.1.14]
                     \item Prove that $p(x)$ is nilpotent in $R[x]$ if and only
                           if $a_0, a_1, \ldots, a_n$ are nilpotent elements of
                           $R$.
                  \end{enumerate}
%%%%%%%%%%%%%%%%%%%%%%%%%%%%%%%%%%%%%7.3.34%%%%%%%%%%%%%%%%%%%%%%%%%%%%%%%%%%%%%
   \item[7.3.34]  Let $I$ and $J$ be ideals of $R$.
                  \begin{enumerate}
                     \item Prove that $I + J$ is the smallest ideal of $R$
                           containing both $I$ and $J$.
                     \item Prove that $IJ$ is an ideal contained in $I \cap J$.
                     \item Give an example where $IJ \neq I \cap J$.
                     \item Prove that if $R$ is commutative and if $I + J = R$
                           then $IJ = I \cap J$.
                  \end{enumerate}
%%%%%%%%%%%%%%%%%%%%%%%%%%%%%%%%%%%%%7.3.35%%%%%%%%%%%%%%%%%%%%%%%%%%%%%%%%%%%%%
   \item[7.3.35]  Let $I, J, K$ be ideals of $R$.
                  \begin{enumerate}
                     \item Prove that $I(J + K) = IJ + IK$ and
                           $(I + J)K = IK + JK$.
                     \item Prove that if $J \subseteq I$ then
                           $I \cap (J + K) = J + (I \cap K)$.
                  \end{enumerate}
%%%%%%%%%%%%%%%%%%%%%%%%%%%%%%%%%%%%%7.3.36%%%%%%%%%%%%%%%%%%%%%%%%%%%%%%%%%%%%%
   \item[7.3.36]  Show that if $I$ is the ideal of all polynomials in $\Z[x]$
                  with zero constant term then $I^n = \{a_nx^n + a_{n+1}x^{n+1} 
                  + \cdots + a_{n+m}x^{n+m} : a_i \in \Z, m \ge 0\}$ is the set
                  of polynomials whose first nonzero term has degree at least
                  $n$.
%%%%%%%%%%%%%%%%%%%%%%%%%%%%%%%%%%%%%7.3.37%%%%%%%%%%%%%%%%%%%%%%%%%%%%%%%%%%%%%
   \item[7.3.37]  An ideal $N$ is called $\textit{nilpotent}$ if $N^n$ is the
                  zero ideal for some $n \ge 1$. Prove that the ideal
                  $p\Z/p^m\Z$ is a nilpotent ideal in the ring $\Z/p^m\Z$.
\end{enumerate}
