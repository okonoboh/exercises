Let $R$ be a ring with identity $1 \neq 0$.

\begin{enumerate}
%%%%%%%%%%%%%%%%%%%%%%%%%%%%%%%%%%%%%7.3.1%%%%%%%%%%%%%%%%%%%%%%%%%%%%%%%%%%%%%%
   \item[7.3.1]   Prove that the rings $2\Z$ and $3\Z$ are not isomorphic.

      \textbf{Proof.} Suppose to the contrary that $2\Z$ and $3\Z$ are
      isomorphic. Thus there exists a ring isomorphism
      $\alpha : 2\Z \rightarrow 3\Z$. Let $a \in 2\Z$, so that $a = 2a'$ for
      some integer $a'$. Since $\alpha$ is also an additive group homomorphism,
      it follows by Exercise 1.6.1(b) that
      $\alpha(a) = \alpha(a' \cdot 2) = a' \cdot \alpha(2)$. If $\alpha(2) = 0$,
      then $\alpha$ is the zero homomorphism so that $\alpha$ is not surjective,
      a contradiction; thus $\alpha(2) \neq 0$. Now we have that
      $$\alpha(4) = \alpha(2 \cdot 2) = \alpha(2) \cdot
        \alpha(2) = \alpha(2)^2$$
      and
      $$\alpha(4) = \alpha(2 \cdot 2) = 2 \cdot \alpha(2)$$
      so that $\alpha(2)^2 = 2\cdot\alpha(2)$, and thus, $\alpha(2) = 2$, a
      contradiction because $2 \notin 3\Z$. So no such isomorphism exists, and
      we conclude that $2\Z \not\cong 3\Z$. \qed
%%%%%%%%%%%%%%%%%%%%%%%%%%%%%%%%%%%%%7.3.2%%%%%%%%%%%%%%%%%%%%%%%%%%%%%%%%%%%%%%
   \item[7.3.2]   Prove that the rings $\Z[x]$ and $\Q[x]$ are not isomorphic.

      \textbf{Proof.} Suppose to the contrary that $\Z[x]$ and $\Q[x]$ are
      isomorphic. Thus there exists a ring isomorphism
      $\alpha : \Z[x] \rightarrow \Q[x]$. Since $\Q$ is an integral domain, it
      follows by Proposition 7.4(3) that $\Q[x]$ is an integral domain, so that
      $\alpha(1) = 1$ by Exercise 7.3.17(a). Let $z$ be an integer. Since
      $\alpha$ is also an additive group homomorphism, it follows by Exercise
      1.6.1(b) that $\alpha(z) = \alpha(z \cdot 1) = z \cdot \alpha(1) = z$. Let
      $q$ be an arbitrary nonintegral rational constant polynomial in $\Q[x]$.
      Since $\alpha$ is surjective, there exists a polynomial of degree $n$, say
      $y = y_0 + y_1x + \cdots + y_nx^n \in \Z[x]$ such that $\alpha(y) = q$.
      Observe that $n \ge 1$, for if $n = 0$, then $\alpha(y_0) = y_0 \neq q$
      because $y_0$ is an integer. Since $\alpha$ is a ring homomorphism, it 
      follows that
      \begin{align*}
         \alpha(y) &= \alpha(y_0 + y_1x + \cdots + y_nx^n) \\
            &= \alpha(y_0) + \alpha(y_1x) + \cdots + \alpha(y_nx^n) \\
            &= \alpha(y_0) + \alpha(y_1)\alpha(x) + \cdots +
               \alpha(y_n)\alpha(x^n) \\
            &= y_0 + y_1\alpha(x) + \cdots + y_n\alpha(x)^n.
      \end{align*}
      If $\alpha(x) = 0$, then $\alpha(y) = y_0$, a contradiction since $y_0$ is
      an integer. So $\alpha(x) \neq 0$; now let $m$ be the degree of
      $\alpha(x)$. Thus, the degree of $\alpha(y)$ is $mn$. But since the degree
      of $q$ is 0, it follows that $mn = 0$. Hence $m = 0$ because $n \neq 0$.
      That is, the degree of $\alpha(x)$ is 0, so that $\alpha(x)$ is a nonzero
      rational number. Thus $\alpha(\Z[x]) \subseteq \Q$, so that $\alpha$ is
      not surjective, a contradition. We thus conclude that
      $\Z[x]\not\cong \Q[x]$. \qed
%%%%%%%%%%%%%%%%%%%%%%%%%%%%%%%%%%%%%7.3.3%%%%%%%%%%%%%%%%%%%%%%%%%%%%%%%%%%%%%%
   \item[7.3.3]   Find all homomorphic images of $\Z$.

      \textbf{Solution.} Let $S$ be a ring and let $\alpha : \Z \rightarrow S$
      be an homomorphism of rings. Let $K$ be the kernel of $\Z$. Then it
      follows by Example 3 (Page 243) that $K = n\Z$ for some nonnegative
      integer $n$. By The First Isomorphism Theorem for Rings, we have that
      $\Z/n\Z \cong \alpha(\Z)$. Thus the image of every ring homomorphism from
      $\Z$ to another ring is isomorphic to $\Z/n\Z$ for some nonnegative
      integer $n$. For a nonnegative integer $m$, let $S = \Z/m\Z$, so that
      $\alpha(\Z) \cong \Z/m\Z$ by Theorem 7.7.2. That is, the set of the
      homomorphic images of $\Z$ is:
      $\{\Z/x\Z : x \text{ is a nonnegative integer}\}$.
%%%%%%%%%%%%%%%%%%%%%%%%%%%%%%%%%%%%%7.3.4%%%%%%%%%%%%%%%%%%%%%%%%%%%%%%%%%%%%%%
   \item[7.3.4]   Find all ring homomorphisms from $\Z$ to $\Z/30\Z$. In each
                  case describe the kernel and the image.

      \textbf{Solution.} There exist at least two ring homomorphisms from $\Z$
      to $\Z/30\Z$, namely the zero ring homomorphism and natural projection, so
      let $\alpha : \Z \rightarrow \Z/30\Z$ be a ring homomorphism. Let
      $a, b \in \Z$. Exercise 1.6.1(b) tells us that
      $\alpha(n) = \alpha(n \cdot 1) = \overline{n} \cdot \alpha(1)$ for every 
      integer $n$. So $\alpha$ is completely determined by $\alpha(1)$. Observe 
      that $\alpha$ is a group homomorphism for any value of $\alpha(1)$ because
      \begin{align*}
         \alpha(a + b) &= (\overline{a + b}) \cdot \alpha(1) \\
            &= (\overline{a} + \overline{b}) \cdot \alpha(1) \\
            &= \overline{a} \cdot \alpha(1) + \overline{b} \cdot \alpha(1) \\
            &= \alpha(a) + \alpha(b).            
      \end{align*}
      Now we must also have that
      \begin{align*}
         \overline{ab} \cdot \alpha(1) &= \alpha(ab) \\
            &= \alpha(a)\alpha(b) \\
            &= \overline{a}\cdot\alpha(1)\cdot\overline{b}\cdot\alpha(1) \\ 
            &= \overline{ab}\cdot\alpha(1)^2,           
      \end{align*}
      so that $\overline{ab}(\alpha(1)^2 - \alpha(1)) = \overline{0}$ for all
      $\overline{a}, \overline{b} \in \Z/30\Z$. That is,
      $\alpha(1)^2 - \alpha(1) = \overline{0}$, and we conclude that $\alpha$ is
      a ring homomorphism if and only if $\alpha(1)$ is idempotent under
      multiplication. The idempotents in $\Z/30\Z$ are:
      $$0, 1, 6, 10, 15, 16, 21, \text{ and } 25,$$
      so that there are 8 homomorphisms from $\Z$ to $\Z/30$. We summarize our
      findings below:
      $$
      \begin{tabular}{@{}|c|c|@{}} \hline
         $\alpha(1)$ & Kernel \\ \hline
         0 & $\Z$ \\ \hline
         1 & $30\Z$ \\ \hline
         6 & $5\Z$ \\ \hline
         10 & $3\Z$ \\ \hline
         15 & $2\Z$ \\ \hline
         16 & $15\Z$ \\ \hline
         21 & $10\Z$ \\ \hline
         25 & $6\Z$ \\ \hline
      \end{tabular}
      $$
%%%%%%%%%%%%%%%%%%%%%%%%%%%%%%%%%%%%%7.3.5%%%%%%%%%%%%%%%%%%%%%%%%%%%%%%%%%%%%%%
   \item[7.3.5]   Describe all ring homomorphisms from the ring $\Z \times \Z$
                  to $\Z$. In each case describe the kernel and the image.

      \textbf{Solution.} We want to find a nonzero ring homomorphism from
      $\Z\times\Z$ to $\Z$. Suppose one exists and call it $\alpha$. Let
      $a, b \in \Z\times\Z$, so that $a = (a_1, a_2)$ and $b = (b_1, b_2)$ for
      some integers $a_1, a_2, b_1$, and $b_2$. So we have that
      \begin{align*}
         \alpha(a) &= \alpha(a_1, a_2) \\
            &= \alpha((a_1, 0) + (0, a_2)) \\
            &= \alpha(a_1, 0) + \alpha(0, a_2) &[\alpha
                  \text{ is a group homomorphism}] \\
            &= \alpha(a_1 \cdot( 1, 0)) + \alpha(a_2 \cdot (0, 1))  \\
            &= a_1 \cdot \alpha(1, 0) + a_2 \cdot \alpha(0, 1),
                  &[\text{Exercise 1.6.1(b)}]
      \end{align*}
      so that $\alpha$ is completely determined by $\alpha(1, 0)$ and
      $\alpha(0, 1)$. Now observe that $\alpha$ is a group homomorphism
      independent of the values of $\alpha(1, 0)$ and $\alpha(0, 1)$ because
      \begin{align*}
         \alpha(a + b) &= \alpha((a_1, a_2) + (b_1, b_2)) \\
            &= \alpha(a_1 + b_1, a_2 + b_2) \\
            &= (a_1 + b_1) \cdot \alpha(1, 0) + (a_2 + b_2) \cdot\alpha(0, 1) \\
            &= a_1 \cdot \alpha(1, 0) + b_1 \cdot \alpha(1, 0) +
               a_2 \cdot \alpha(0, 1) + b_2 \cdot \alpha(0, 1) \\
            &= \alpha(a_1, 0) +\alpha(b_1, 0) +\alpha(0, a_2) +\alpha(0, b_2) \\
            &= \alpha(a_1, 0) +\alpha(0, a_2) +\alpha(b_1, 0) +\alpha(0, b_2) \\
            &= \alpha(a_1, a_2) + \alpha(b_1, b_2) \\
            &= \alpha(a) + \alpha(b).
      \end{align*}
   
      Since $\Z$ is an integral domain an $\alpha$ is nonzero, it follows by
      Exercise 7.3.17(a) that $\alpha(1, 1) = 1$. So
      $1 = \alpha(1, 1) = \alpha(1, 0) + \alpha(0, 1)$. Also
      \begin{align*}
         \alpha(1, 0) &= \alpha((1, 0) \cdot (1, 0)) \\
            &= \alpha(1, 0) \cdot \alpha(1, 0)
                  &[\alpha \text{ is a ring homomorphism}] \\
            &= \alpha(1, 0)^2,
      \end{align*}
      so that $\alpha(1, 0)^2 - \alpha(1, 0) = 0$; i.e.,
      $\alpha(1, 0) \cdot (\alpha(1, 0) - 1) = 0$. Since $\Z$ is an integral
      domain, it follows that $\alpha(1, 0) = 0$ or $\alpha(1, 0) = 1$.
      Replace $(1, 0)$ with $(0, 1)$ in the procedure above to conclude
      similarly that  $\alpha(0, 1) = 0$ or $\alpha(0, 1) = 1$. Recall that
      $\alpha(1, 0) + \alpha(0, 1) = 1$, so it follows that
      $$\alpha(1, 0) = 1 \text{ and } \alpha(0, 1) = 0$$
      or
      $$\alpha(1, 0) = 0 \text{ and } \alpha(0, 1) = 1.$$

      Suppose first that the former holds. So we have that
      \begin{align*}
         \alpha(ab) &= \alpha((a_1, a_2)(b_1, b_2)) \\
            &= \alpha(a_1b_1, a_2b_2) \\
            &= a_1b_1 \cdot \alpha(1, 0) + a_2b_2 \cdot \alpha(0, 1) \\
            &= a_1b_1 \cdot 1 + a_2b_2 \cdot 0 \\
            &= a_1b_1 \\
            &= (a_1 \cdot 1 + a_2 \cdot 0)(b_1 \cdot 1 + b_2 \cdot 0) \\
            &= (a_1 \cdot \alpha(1, 0) + a_2 \cdot \alpha(0, 1))
               (b_1 \cdot \alpha(1, 0) + b_2 \cdot \alpha(0, 1)) \\
            &= (\alpha(a_1, 0) + \alpha(0, a_2))
               (\alpha(b_1, 0) + \alpha(0, b_2)) \\
            &= (\alpha(a_1, a_2))(\alpha(b_1, b_2)) \\
            &= \alpha(a)\alpha(b).
      \end{align*}
      However, if the latter holds, use the same procedure above to conclude
      that $\alpha(ab) = \alpha(a)\alpha(b)$. Thus we have shown that $\alpha$
      is a nonzero ring homomorphism if and only if exactly one of
      $\alpha(1, 0)$ and $\alpha(0, 1)$ is 0 while the other is 1. So there are
      two such nonzero ring homomorphisms for a total of three ring
      homomorphisms (if we include the zero homomorphism). We tabulate our
      results:
      $$
      \begin{tabular}{@{}|c|c|c|@{}} \hline
         $\alpha(1, 0)$ & $\alpha(0, 1)$ & Kernel \\ \hline
         0 & 0 & $\Z\times\Z$ \\ \hline
         1 & 0 & $\{0\}\times\Z$ \\ \hline
         0 & 1 & $\Z\times\{0\}$ \\ \hline
      \end{tabular}
      $$
%%%%%%%%%%%%%%%%%%%%%%%%%%%%%%%%%%%%%7.3.6%%%%%%%%%%%%%%%%%%%%%%%%%%%%%%%%%%%%%%
   \item[7.3.6]   Decide which of the following are ring homomorphisms from
                  $M_2(\Z)$ to $\Z$:
                  \begin{enumerate}
                     \item $\left(\begin{tabular}{@{}cc@{}}
                              $a$ & $b$ \\
                              $c$ & $d$
                           \end{tabular}\right) \mapsto a$ (projection onto the
                           1,1 entry)
                     \item $\left(\begin{tabular}{@{}cc@{}}
                              $a$ & $b$ \\
                              $c$ & $d$
                           \end{tabular}\right) \mapsto a + d$ (the
                           \textit{trace} of the matrix)
                     \item $\left(\begin{tabular}{@{}cc@{}}
                              $a$ & $b$ \\
                              $c$ & $d$
                           \end{tabular}\right) \mapsto ad - bc$ (the
                           \textit{determinant} of the matrix).
                  \end{enumerate}

      \textbf{Solution.}

      \begin{enumerate}
         \item Consider the map $\alpha : M_2(\Z) \rightarrow \Z$, defined by
               $(a_{ij}) \mapsto a_{11}$. Let $A \in M_2(\Z)$, such that every
               entry of $A$ is 1. It follows immediately that $\alpha$ is not a
               ring homomorphism because
               $$\alpha(A^2) = 2 \neq 1 = \alpha(A) \cdot \alpha(A).$$
         \item Consider the map $\alpha : M_2(\Z) \rightarrow \Z$, defined by
               $(a_{ij}) \mapsto a_{11} + a_{22}$. Let $A, B \in M_2(\Z)$, such 
               that the only nonzero entry of $A$ is the 2, 1 entry which is 1
               and the only nonzero entry of $B$ is the 1, 2 entry which is also
               1. It follows immediately that $\alpha$ is not a ring
               homomorphism because
               $$\alpha(AB) = 1 \neq 0 = \alpha(A) \cdot \alpha(B).$$
         \item Consider the map $\alpha : M_2(\Z) \rightarrow \Z$, defined by
               $(a_{ij}) \mapsto a_{11}a_{22} - a_{12}a_{21}$. Let
               $A \in M_2(\Z)$, such that every entry of $A$ is 1, except the
               2, 1 entry which is 0. It follows immediately that $\alpha$ is
               not a ring homomorphism because
               $$\alpha(A + A) = 4 \neq 2 = \alpha(A) + \alpha(A).$$
      \end{enumerate}
%%%%%%%%%%%%%%%%%%%%%%%%%%%%%%%%%%%%%7.3.7%%%%%%%%%%%%%%%%%%%%%%%%%%%%%%%%%%%%%%
   \item[7.3.7]   Let $R = \left\{\left(\begin{tabular}{@{}cc@{}}
                     $a$ & $b$ \\
                     0 & $d$
                  \end{tabular}\right) : a, b, d \in \Z\right\}$ be the subring
                  of $M_2(\Z)$ of upper triangular matrices. Prove that the map
                  $$\varphi : R \rightarrow \Z \times \Z \text{ defined by }
                    \varphi : \left(\begin{tabular}{@{}cc@{}}
                     $a$ & $b$ \\
                      0  & $d$
                  \end{tabular}\right) \rightarrow (a, d)$$
                  is a surjective homomorphism and describe its kernel.
%%%%%%%%%%%%%%%%%%%%%%%%%%%%%%%%%%%%%7.3.8%%%%%%%%%%%%%%%%%%%%%%%%%%%%%%%%%%%%%%
   \item[7.3.8]   Decide which of the following are ideals of the ring
                  $\Z \times \Z$:
                  \begin{enumerate}
                     \item $\{(a, a) : a \in \Z\}$
                     \item $\{(2a, 2b) : a, b \in \Z\}$
                     \item $\{(2a, 0) : a \in \Z\}$
                     \item $\{(a, -a) : a \in \Z\}$.
                  \end{enumerate}
%%%%%%%%%%%%%%%%%%%%%%%%%%%%%%%%%%%%%7.3.9%%%%%%%%%%%%%%%%%%%%%%%%%%%%%%%%%%%%%%
   \item[7.3.9]   Decide which of the sets in Exercise 6 of Section 1 are ideals
                  of the ring of all functions from $[0, 1]$ to $\R$.
%%%%%%%%%%%%%%%%%%%%%%%%%%%%%%%%%%%%%7.3.10%%%%%%%%%%%%%%%%%%%%%%%%%%%%%%%%%%%%%
   \item[7.3.10]  Decide which of the following are ideals of the ring $\Z[x]$:
                  \begin{enumerate}
                     \item the set of all polynomials whose constant term is a
                           multiple of 3
                     \item the set of all polynomials whose coefficient of $x^2$
                           is a multiple of 3
                     \item the set of all polynomials whose constant term,
                           coefficient of $x$ and coefficient of $x^2$ are zero
                     \item $\Z[x^2]$ (i.e., the polynomials in which only even
                           powers of $x$ appear)
                     \item the set of polynomials whose coefficients sum to
                           zero
                     \item the set of polynomials $p(x)$ such that $p'(0) = 0$,
                           where $p'(x)$ is the usual first derivative of $p(x)$
                           with respect to $x$.
                  \end{enumerate}
%%%%%%%%%%%%%%%%%%%%%%%%%%%%%%%%%%%%%7.3.11%%%%%%%%%%%%%%%%%%%%%%%%%%%%%%%%%%%%%
   \item[7.3.11]  Let $R$ be the ring of all continuous real valued functions on
                  the closed interval $[0, 1]$. Prove that the map
                  $\varphi : R \rightarrow \R$ defined by
                  $\varphi(f) = \int_0^1f(t)dt$ is a homomorphism of additive
                  groups but not a ring homomorphism.
%%%%%%%%%%%%%%%%%%%%%%%%%%%%%%%%%%%%%7.3.12%%%%%%%%%%%%%%%%%%%%%%%%%%%%%%%%%%%%%
   \item[7.3.12]  Let $D$ be an integer that is not a perfect square in $\Z$ and
                  let $S = \left\{\left(\begin{tabular}{@{}cc@{}}
                     $a$ & $b$ \\
                     $Db$  & $d$
                  \end{tabular}\right) : a, b \in \Z\right\}$.
                  \begin{enumerate}
                     \item Prove that $S$ is a subring of $M_2(\Z)$.
                     \item If $D$ is not a perfect square in $\Z$ prove that the
                           map $\varphi : \Z[\sqrt{D}] \rightarrow S$ defined by
                           $\varphi(a + b\sqrt{D}) =
                            \left(\begin{tabular}{@{}cc@{}}
                               $a$ & $b$ \\
                               $Db$  & $d$
                           \end{tabular}\right)$ is a ring isomorphism.
                     \item If $D \equiv 1$ mod 4 is squarefree, prove that the
                           set $\left\{\left(\begin{tabular}{@{}cc@{}}
                              $a$ & $b$ \\
                              $(D-1)b/4$  & $a+b$
                           \end{tabular}\right) : a, b \in \Z\right\}$ is a
                           subring of $M_2(\Z)$ and is isomorphic to the 
                           quadratic integer ring $\mathcal{O}$.
                  \end{enumerate}
%%%%%%%%%%%%%%%%%%%%%%%%%%%%%%%%%%%%%7.3.13%%%%%%%%%%%%%%%%%%%%%%%%%%%%%%%%%%%%%
   \item[7.3.13]  Prove that the ring $M_2(\R)$ contains a subring that is
                  isomorphic to $\C$.
%%%%%%%%%%%%%%%%%%%%%%%%%%%%%%%%%%%%%7.3.14%%%%%%%%%%%%%%%%%%%%%%%%%%%%%%%%%%%%%
   \item[7.3.14]  Prove that the ring $M_4(\R)$ contains a subring that is
                  isomorphic to the real Hamilton Quaternions, $\mathbb{H}$.
%%%%%%%%%%%%%%%%%%%%%%%%%%%%%%%%%%%%%7.3.15%%%%%%%%%%%%%%%%%%%%%%%%%%%%%%%%%%%%%
   \item[7.3.15]  Let $X$ be a nonempty set and let $\mathcal{P}(X)$ be the
                  Boolean ring of all subsets of $X$ defined in Exercise 7.1.21.
                  Let $R$ be the ring of all functions from $X$ into $\Z/2\Z$.
                  For each $A \in \mathcal{P}(X)$ define the function
                  $$\chi_A : X \rightarrow \Z/2\Z \quad \text{by} \quad
                    \chi_A(x) = \left\{
                       \begin{array}{ll}
                          1 & \text{if } x \in A \\
                          0 & \text{if } x \notin A
                       \end{array} \right.
                  $$
                  ($\chi_A$ is called the \textit{characteristic function of }
                  $A$ with values in $\Z/2\Z$). Prove that the map
                  $\mathcal{P}(X) \rightarrow R$ defined by $A \mapsto \chi_A$
                  is a ring isomorphism.
%%%%%%%%%%%%%%%%%%%%%%%%%%%%%%%%%%%%%7.3.16%%%%%%%%%%%%%%%%%%%%%%%%%%%%%%%%%%%%%
   \item[7.3.16]  Let $\varphi : R \rightarrow S$ be a surjective homomorphism
                  of rings. Prove that the image of the center of $R$ is
                  contained in the center of $S$ (cf. Exercise 7.1.7).
%%%%%%%%%%%%%%%%%%%%%%%%%%%%%%%%%%%%%7.3.17%%%%%%%%%%%%%%%%%%%%%%%%%%%%%%%%%%%%%
   \item[7.3.17]  Let $R$ and $S$ be nonzero rings with identity and denote
                  their respective identities by $1_R$ and $1_S$. Let
                  $\varphi : R \rightarrow S$ be a nonzero homomorphism of
                  rings.
                  \begin{enumerate}
                     \item Prove that if $\varphi(1_R) \neq 1_S$ then
                           $\varphi(1_R)$ is a zero divisor in $S$. Deduce that
                           if $S$ is an integral domain then every nonzero ring
                           homomorphism from $R$ to $S$ sends the identity of
                           $R$ to the identity of $S$.
                     \item Prove that if $\varphi(1_R) = 1_S$ then $\varphi(u)$
                           is a unit in $S$ and that
                           $\varphi(u^{-1}) = \varphi(u)^{-1}$ for each unit $u$ 
                           of $R$.
                  \end{enumerate}
%%%%%%%%%%%%%%%%%%%%%%%%%%%%%%%%%%%%%7.3.18%%%%%%%%%%%%%%%%%%%%%%%%%%%%%%%%%%%%%
   \item[7.3.18]  \begin{enumerate}
                     \item If $I$ and $J$ are ideals of $R$ prove that their
                           intersection $I \cap J$ is also an ideal of $R$.
                     \item Prove that the intersection of an arbitrary nonempty
                           collection of ideals is again an ideal (do not assume
                           that the collection is countable).
                  \end{enumerate}
%%%%%%%%%%%%%%%%%%%%%%%%%%%%%%%%%%%%%7.3.19%%%%%%%%%%%%%%%%%%%%%%%%%%%%%%%%%%%%%
   \item[7.3.19]  Prove that if $I_1 \subseteq I_2 \subseteq \cdots$ are ideals
                  of $R$ then $\cup_{n=1}^\infty I_n$ is an ideal of $R$.
%%%%%%%%%%%%%%%%%%%%%%%%%%%%%%%%%%%%%7.3.20%%%%%%%%%%%%%%%%%%%%%%%%%%%%%%%%%%%%%
   \item[7.3.20]  Let $I$ be an ideal of $R$ and let $S$ be a subring of $R$.
                  Prove that $I \cap S$ is an ideal of $S$. Show by example that
                  not every ideal of a subring $S$ of a ring $R$ need be of the
                  form $I \cap S$ for some ideal $I$ of $R$.
%%%%%%%%%%%%%%%%%%%%%%%%%%%%%%%%%%%%%7.3.21%%%%%%%%%%%%%%%%%%%%%%%%%%%%%%%%%%%%%
   \item[7.3.21]  Prove that every (two-sided) ideal of $M_n(R)$ is equal to
                  $M_n(J)$ for some (two-sided) ideal $J$ of $R$. [Use Exercise
                  6(c) of Section 2 to show first that the set of entries of
                  matrices in an ideal of $M_n(R)$ form an ideal in $R$.]
%%%%%%%%%%%%%%%%%%%%%%%%%%%%%%%%%%%%%7.3.22%%%%%%%%%%%%%%%%%%%%%%%%%%%%%%%%%%%%%
   \item[7.3.22]  Let $a$ be an element of the ring $R$.
                  \begin{enumerate}
                     \item Prove that $\{x \in R : ax = 0\}$ is right ideal and
                           $\{y \in R : ya = 0\}$ is a left ideal (called
                           respectively the right and left \textit{annihilators}
                           of $a$ in $R$).
                     \item Prove that if $L$ is a left ideal of $R$ then
                           $\{x \in R : xa = 0 \text{ for all } a \in L\}$ is a
                           two-sided ideal (called the left \textit{annihilator}
                           of $L$ in $R$).
                  \end{enumerate}
%%%%%%%%%%%%%%%%%%%%%%%%%%%%%%%%%%%%%7.3.23%%%%%%%%%%%%%%%%%%%%%%%%%%%%%%%%%%%%%
   \item[7.3.23]  Let $S$ be a subring of $R$ and let $I$ be an ideal of $R$.
                  Prove that if $S \cap I = 0$ then $\overline{S} \cong S$,
                  where the bar denotes passage to $R/I$.
%%%%%%%%%%%%%%%%%%%%%%%%%%%%%%%%%%%%%7.3.24%%%%%%%%%%%%%%%%%%%%%%%%%%%%%%%%%%%%%
   \item[7.3.24]  Let $\varphi : R \rightarrow S$ be a ring homomorphism.
                  \begin{enumerate}
                     \item Prove that if $J$ is an ideal of $S$ then
                           $\varphi^{-1}(J)$ is an ideal of $R$. Apply this to
                           the special case when $R$ is a subring of $S$ and
                           $\varphi$ is the inclusion homomorphism to deduce
                           that if $J$ is an ideal of $S$ then $J \cap R$ is an
                           ideal of $R$.
                     \item Prove that if $\varphi$ is surjective and $I$ is an
                           ideal of $R$ then $\varphi(I)$ is an ideal of $S$.
                           Give an example where this fails if $\varphi$ is not
                           surjective.
                  \end{enumerate}
%%%%%%%%%%%%%%%%%%%%%%%%%%%%%%%%%%%%%7.3.25%%%%%%%%%%%%%%%%%%%%%%%%%%%%%%%%%%%%%
   \item[7.3.25]  Assume $R$ is a commutative ring with 1. Prove that the
                  Binomial Theorem
                  $$(a + b)^n = \sum_{k=0}^n\binom{n}{k}a^kb^{n-k}$$
                  holds in $R$, where the binomial coefficient $\binom{n}{k}$ is
                  interpreted in $R$ as the sum $1 + 1 + \cdots + 1$ of the
                  identity 1 in $R$ taken $\binom{n}{k}$ times.
%%%%%%%%%%%%%%%%%%%%%%%%%%%%%%%%%%%%%7.3.26%%%%%%%%%%%%%%%%%%%%%%%%%%%%%%%%%%%%%
   \item[7.3.26]  The \textit{characteristic} of a ring $R$ is the smallest
                  positive integer $n$ such that $1 + 1 + \cdots + 1 = 0$
                  ($n$ times) in $R$; if no such integer exists the 
                  characteristic of $R$ is said to be 0. For example,
                  $\Z/n\Z$ is a ring of characteristic $n$ for each positive 
                  integer $n$ and $\Z$ is a ring of characteristic 0.
                  \begin{enumerate}
                     \item Prove that the map $\Z \rightarrow R$ defined by
                           \begin{equation*}
                              k \mapsto \left\{
                                 \begin{array}{ll}
                                    1 + 1 + \cdots + 1 (k \text{ times}) &
                                       \text{if } k > 0 \\
                                    0 & \text{if } k = 0 \\
                                    -1 - 1 - \cdots - 1 (-k \text{ times}) &
                                       \text{if } k < 0
                                 \end{array} \right.
                           \end{equation*}
                           is a ring homomorphism whose kernel is $n\Z$, where
                           $n$ is the characteristic of $R$ (this explains the
                           use of the terminology ``characteristic 0" instead of
                           the archaic phrase ``characteristic $\infty$" for
                           rings in which no sum of 1's is zero).
                     \item Determine the characteristics of the rings $\Q$,
                           $\Z[x]$, $\Z/n\Z[x]$.
                     \item Prove that if $p$ is a prime and if $R$ is a
                           commutative ring of characteristic $p$, then
                           $(a + b)^p = a^p + b^p$ for all $a, b \in R$.
                  \end{enumerate}
%%%%%%%%%%%%%%%%%%%%%%%%%%%%%%%%%%%%%7.3.27%%%%%%%%%%%%%%%%%%%%%%%%%%%%%%%%%%%%%
   \item[7.3.27]  Prove that a nonzero Boolean ring has characteristic 2
                  (cf. Exericse 7.1.15).
%%%%%%%%%%%%%%%%%%%%%%%%%%%%%%%%%%%%%7.3.28%%%%%%%%%%%%%%%%%%%%%%%%%%%%%%%%%%%%%
   \item[7.3.28]  Prove that an integral domain has characteristic $p$, where
                  $p$ is either a prine or 0 (cf Exercise 7.3.26).
%%%%%%%%%%%%%%%%%%%%%%%%%%%%%%%%%%%%%7.3.29%%%%%%%%%%%%%%%%%%%%%%%%%%%%%%%%%%%%%
   \item[7.3.29]  Let $R$ be a commutative ring. Recall (cf. Exercise 7.1.13)
                  that an element $x \in R$ is nilpotent if $x^n = 0$ for some
                  $n \in \Z^+$. Prove that the set of nilpotent elements form an 
                  ideal---called the \textit{nilradical} of $R$ and denoted by
                  $\mathfrak{N}(R)$. [Use the Binomial Theorem to show
                  $\mathfrak{N}(R)$ is closed under addition.]
%%%%%%%%%%%%%%%%%%%%%%%%%%%%%%%%%%%%%7.3.30%%%%%%%%%%%%%%%%%%%%%%%%%%%%%%%%%%%%%
   \item[7.3.30]  Prove that if $R$ is a commutative ring and $\mathfrak{N}(R)$
                  is its nilradical (cf. the preceding exercise) then zero is
                  the only nilpotent element of $R/\mathfrak{N}(R)$ i.e., prove
                  that $\mathfrak{N}(R/\mathfrak{N}(R)) = 0$.
%%%%%%%%%%%%%%%%%%%%%%%%%%%%%%%%%%%%%7.3.31%%%%%%%%%%%%%%%%%%%%%%%%%%%%%%%%%%%%%
   \item[7.3.31]  Prove that the elements $\left(\begin{tabular}{@{}cc@{}}
                     0 & 1 \\
                     0 & 0
                  \end{tabular}\right)$ and $\left(\begin{tabular}{@{}cc@{}}
                     0 & 0 \\
                     1 & 0
                  \end{tabular}\right)$ are nilpotent elements of $M_2(\Z)$
                  whose sum is not nilpotent (note that these two matrices do
                  not commute). Deduce that the set of nilpotent elements in the
                  noncommutative ring $M_2(\Z)$ is not an ideal.
%%%%%%%%%%%%%%%%%%%%%%%%%%%%%%%%%%%%%7.3.32%%%%%%%%%%%%%%%%%%%%%%%%%%%%%%%%%%%%%
   \item[7.3.32]  Let $\varphi : R \rightarrow S$ be a homomorphism of rings.
                  Prove that if $x$ is a nilpotent element of $R$ then
                  $\varphi(x)$ is nilpotent in $S$.
%%%%%%%%%%%%%%%%%%%%%%%%%%%%%%%%%%%%%7.3.33%%%%%%%%%%%%%%%%%%%%%%%%%%%%%%%%%%%%%
   \item[7.3.33]  Assume $R$ is commutative. Let
                  $p(x) = a_nx^n + a_{n-1}x^{n-1} + \cdots + a_1x + a_0$ be an
                  element of the polynomial ring $R[x]$.
                  \begin{enumerate}
                     \item Prove that $p(x)$ is unit in $R[x]$ if and only if
                           $a_0$ is a unit and $a_1, a_2, \ldots, a_n$ are
                           nilpotent in $R$. [See Exercise 7.1.14]
                     \item Prove that $p(x)$ is nilpotent in $R[x]$ if and only
                           if $a_0, a_1, \ldots, a_n$ are nilpotent elements of
                           $R$.
                  \end{enumerate}
%%%%%%%%%%%%%%%%%%%%%%%%%%%%%%%%%%%%%7.3.34%%%%%%%%%%%%%%%%%%%%%%%%%%%%%%%%%%%%%
   \item[7.3.34]  Let $I$ and $J$ be ideals of $R$.
                  \begin{enumerate}
                     \item Prove that $I + J$ is the smallest ideal of $R$
                           containing both $I$ and $J$.
                     \item Prove that $IJ$ is an ideal contained in $I \cap J$.
                     \item Give an example where $IJ \neq I \cap J$.
                     \item Prove that if $R$ is commutative and if $I + J = R$
                           then $IJ = I \cap J$.
                  \end{enumerate}
%%%%%%%%%%%%%%%%%%%%%%%%%%%%%%%%%%%%%7.3.35%%%%%%%%%%%%%%%%%%%%%%%%%%%%%%%%%%%%%
   \item[7.3.35]  Let $I, J, K$ be ideals of $R$.
                  \begin{enumerate}
                     \item Prove that $I(J + K) = IJ + IK$ and
                           $(I + J)K = IK + JK$.
                     \item Prove that if $J \subseteq I$ then
                           $I \cap (J + K) = J + (I \cap K)$.
                  \end{enumerate}
%%%%%%%%%%%%%%%%%%%%%%%%%%%%%%%%%%%%%7.3.36%%%%%%%%%%%%%%%%%%%%%%%%%%%%%%%%%%%%%
   \item[7.3.36]  Show that if $I$ is the ideal of all polynomials in $\Z[x]$
                  with zero constant term then $I^n = \{a_nx^n + a_{n+1}x^{n+1} 
                  + \cdots + a_{n+m}x^{n+m} : a_i \in \Z, m \ge 0\}$ is the set
                  of polynomials whose first nonzero term has degree at least
                  $n$.
%%%%%%%%%%%%%%%%%%%%%%%%%%%%%%%%%%%%%7.3.37%%%%%%%%%%%%%%%%%%%%%%%%%%%%%%%%%%%%%
   \item[7.3.37]  An ideal $N$ is called $\textit{nilpotent}$ if $N^n$ is the
                  zero ideal for some $n \ge 1$. Prove that the ideal
                  $p\Z/p^m\Z$ is a nilpotent ideal in the ring $\Z/p^m\Z$.
\end{enumerate}
