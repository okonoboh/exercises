Let
   $$\mathcal{A} \mbox{ be the set of real } 2 \times 2
     \mbox{ matrices}, M = \left(
      \begin{tabular}{@{}l r@{}} 
         1 & 1\\ 
         0 & 1
      \end{tabular}\right)$$
and let
   $$\mathcal{B} = \{X \in \mathcal{A} : MX = XM\}.$$
\begin{enumerate}
%%%%%%%%%%%%%%%%%%%%%%%%%%%%%%%%%%%%%0.1.1%%%%%%%%%%%%%%%%%%%%%%%%%%%%%%%%%%%%%%
   \item[0.1.1] Determine which of the following elements of $\mathcal{A}$ lie
                in $\mathcal{B}$:

                $$M = \left(
                   \begin{tabular}{@{}l r@{}} 
                      1 & 1 \\ 
                      0 & 1
                   \end{tabular}\right), \left(
                   \begin{tabular}{@{}l r@{}} 
                      1 & 1 \\ 
                      1 & 1
                   \end{tabular}\right), \left(
                   \begin{tabular}{@{}l r@{}} 
                      0 & 0 \\ 
                      0 & 0
                   \end{tabular}\right), \left(
                   \begin{tabular}{@{}l r@{}} 
                      1 & 1 \\ 
                      1 & 0
                   \end{tabular}\right), \left(
                   \begin{tabular}{@{}l r@{}} 
                      1 & 0 \\ 
                      0 & 1
                   \end{tabular}\right), \left(
                   \begin{tabular}{@{}l r@{}} 
                      0 & 1 \\ 
                      1 & 1
                   \end{tabular}\right).$$
      \textbf{Solution.} The matrices $\left(\begin{tabular}{@{}l r@{}} 
         1 & 1 \\ 
         0 & 1
      \end{tabular}\right)$, $\left(\begin{tabular}{@{}l r@{}} 
         0 & 0 \\ 
         0 & 0
      \end{tabular}\right)$, and $\left(\begin{tabular}{@{}l r@{}} 
         1 & 0 \\ 
         0 & 1
      \end{tabular}\right)$ lie in $\mathcal{B}$ because they commute with $M$.
      However, the remaining matrices do not lie in $\mathcal{B}$ because they
      do not commute with $M$ (see below).
         
      \begin{align*}
         \left(\begin{tabular}{@{}l r@{}} 1 & 1 \\ 1 & 1\end{tabular}\right)M 
            &= \left(
            \begin{tabular}{@{}l r@{}} 
               1 & 2 \\ 
               1 & 2
            \end{tabular}\right)\neq \left(
            \begin{tabular}{@{}l r@{}} 
                2 & 2 \\ 
                1 & 1
            \end{tabular}\right) = M
         \left(\begin{tabular}{@{}l r@{}} 1 & 1 \\ 1 & 1\end{tabular}\right), \\
         \left(\begin{tabular}{@{}l r@{}} 1 & 1 \\ 1 & 0\end{tabular}\right)M 
            &= \left(
            \begin{tabular}{@{}l r@{}} 
               1 & 2 \\ 
               1 & 1
            \end{tabular}\right)\neq \left(
            \begin{tabular}{@{}l r@{}} 
                2 & 1 \\ 
                1 & 0
            \end{tabular}\right) = M
         \left(\begin{tabular}{@{}l r@{}} 1 & 1 \\ 1 & 0\end{tabular}\right), \\
         \left(\begin{tabular}{@{}l r@{}} 0 & 1 \\ 1 & 1\end{tabular}\right)M 
            &= \left(
            \begin{tabular}{@{}l r@{}} 
               0 & 1 \\ 
               1 & 2
            \end{tabular}\right)\neq \left(
            \begin{tabular}{@{}l r@{}} 
               1 & 2 \\ 
               1 & 1
            \end{tabular}\right) = M
         \left(\begin{tabular}{@{}l r@{}} 0 & 1 \\ 1 & 1\end{tabular}\right).
         \end{align*} \qed
%%%%%%%%%%%%%%%%%%%%%%%%%%%%%%%%%%%Prob0.1.2%%%%%%%%%%%%%%%%%%%%%%%%%%%%%%%%%%%%
   \item[0.1.2] Prove that if $P, Q \in \mathcal{B}$, then
                $P + Q \in \mathcal{B}$.

      \textbf{Proof.} Let $P, Q \in \mathcal{B}$. To show that
      $P + Q \in \mathcal{B}$, we must show that $P + Q$ commutes with $M$. This
      follows since
      \begin{align*}
         M(P + Q) &= MP + MQ   &[\text{Distributive Law}]\\
                  &= PM + QM   &[P, Q \in \mathcal{B}] \\
                  &= (P + Q)M. &[\text{Distributive Law}]
      \end{align*} \qed
%%%%%%%%%%%%%%%%%%%%%%%%%%%%%%%%%%%Prob0.1.3%%%%%%%%%%%%%%%%%%%%%%%%%%%%%%%%%%%%
   \item[0.1.3] Prove that if $P, Q \in \mathcal{B}$, then
                $P \cdot Q \in \mathcal{B}$.

      \textbf{Proof.} Let $P, Q \in \mathcal{B}$. The matrix $PQ$ commutes with
      $M$ because
      \begin{align*}
         M(PQ) &= (MP)Q  &[\text{Associative Law}]\\
               &= (PM)Q  &[P \in \mathcal{B}] \\
               &= P(MQ)  &[\text{Associative Law}] \\
               &= P(QM)  &[Q \in \mathcal{B}] \\
               &= (PQ)M.  &[\text{Associative Law}]
      \end{align*}
      That is, $PQ \in \mathcal{B}$. \qed
%%%%%%%%%%%%%%%%%%%%%%%%%%%%%%%%%%%Prob0.1.4%%%%%%%%%%%%%%%%%%%%%%%%%%%%%%%%%%%%
   \item[0.1.4] Find conditions on $p, q, r, s$ which determine precisely when
                $$\left(
                     \begin{tabular}{@{}l r@{}} 
                        $p$ & $q$ \\ 
                        $r$ & $s$
                     \end{tabular}\right) \in \mathcal{B}.$$

      \textbf{Proof.} Since $\mathcal{B}$ is nonempty (it contains $M$), we let
      $A = \left(\begin{tabular}{@{}l r@{}} 
         $p$ & $q$ \\ 
         $r$ & $s$
      \end{tabular}\right)$ be an arbitrary element of $\mathcal{B}$, wherein
      the entries are real numbers. It follows by definition that $AM = MA$, so 
      that
      $$ \left(
         \begin{tabular}{@{}l r@{}} 
            $p$ & $q$ \\ 
            $r$ & $s$
         \end{tabular}\right)\left(
         \begin{tabular}{@{}l r@{}} 
            1 & 1 \\ 
            0 & 1
         \end{tabular}\right) = \left(
         \begin{tabular}{@{}l r@{}} 
            1 & 1 \\ 
            0 & 1
         \end{tabular}\right)\left(
         \begin{tabular}{@{}l r@{}} 
            $p$ & $q$ \\ 
            $r$ & $s$
         \end{tabular}\right).
      $$

      That is,
      $$ \left(
         \begin{tabular}{@{}l r@{}} 
            $p$ & $p + q$ \\ 
            $r$ & $r + s$
         \end{tabular}\right) = \left(
         \begin{tabular}{@{}c c@{}} 
            $p + r$ & $q + s$ \\ 
            $r$ & $s$
         \end{tabular}\right).
      $$

      Equating corresponding entries tells us that
      $r = 0$ and $p = s$. Thus
      $$\mathcal{B} = \left\{\left(
         \begin{tabular}{@{}l r@{}} 
            $p$ & $q$ \\ 
            $0$ & $p$
         \end{tabular}\right) : p, q \in \R\right\}.$$
%%%%%%%%%%%%%%%%%%%%%%%%%%%%%%%%%%%Prob0.1.5%%%%%%%%%%%%%%%%%%%%%%%%%%%%%%%%%%%%
   \item[0.1.5] Determine whether the following functions $f$ are well defined:
                \begin{enumerate}
                   \item $f : \Q \rightarrow \Z$ defined by $f(a/b) = a$.
                   \item $f : \Q \rightarrow \Q$ defined by $f(a/b) = a^2/b^2$.
                \end{enumerate}

      \textbf{Solution:}
         \begin{enumerate}
            \item $f$ is not well defined because $4/1 = 8/2$, but
                  $f(4/1) = 4 \neq 8 = f(8/2)$.
            \item We claim that $f$ is well defined.

                  \textbf{Proof.} We want to show that all representatives for 
                  an element in $\Q$ have the same output under $f$. So suppose
                  that $q$ and $r$ are equal rational numbers. That is,
                  $$q = \frac{a}{b} \text{ and } r = \frac{c}{d},$$
                  where $a$, $b$, $c$, $d \in \Z$ ($b$ and $d$ nonzero). Well
                  definedness of $f$ follows because
                  $$f(q) = \frac{a^2}{b^2} = \left(\frac{a}{b}\right)^2 =
                    \left(\frac{c}{d}\right)^2 = \frac{c^2}{d^2} = f(r).$$ \qed
         \end{enumerate}
%%%%%%%%%%%%%%%%%%%%%%%%%%%%%%%%%%%Prob0.1.6%%%%%%%%%%%%%%%%%%%%%%%%%%%%%%%%%%%%
   \item[0.1.6] Determine whether the function $f: \R^+ \rightarrow \Z$ defined
                by mapping a real number $r$ to the first digit to the right of
                the decimal point in a decimal expansion of $r$ is well defined.

      \textbf{Proof.} The map $f$ is not well defined because
      $0.\overline{9} = 1.\overline{0}$, but
      $f(0.\overline{9}) = 9 \neq 0 = f(1.\overline{0})$. \\ \mbox{ } \qed
%%%%%%%%%%%%%%%%%%%%%%%%%%%%%%%%%%%Prob0.1.7%%%%%%%%%%%%%%%%%%%%%%%%%%%%%%%%%%%%
   \item[0.1.7] Let $f : A \rightarrow B$ be a surjective map of sets. Prove
                that the relation
                $$a \sim b \mbox{ if and only if } f(a) = f(b)$$
                is an equivalence relation whose equivalence classes are the
                fibers of $f$.

      \textbf{Proof.} We want to show that the relation $\sim$ is an equivalence
      relation on $A$. That is, we want to show that $\sim$ is reflexive,
      symmetric, and transitive on $A$. Thus

      \textbf{Reflexivity.} Let $a \in A$. Since $f(a) = f(a)$, it follows that 
      $a \sim a$, so that $\sim$ is reflexive on $A$.

      \textbf{Symmetry.} Let $a, b \in A$ with $a \sim b$. Since $a \sim b$, we 
      must have that $f(a) = f(b)$; it immediately follows that $f(b) = f(a)$, 
      so that $b \sim a$; that is, $\sim$ is symmetric on $A$.

      \textbf{Transitivity.} Let $a, b, c \in A$ with $a \sim b$ and $b \sim c$.
      Thus by definition, we have that $f(a) = f(b)$ and $f(b) = f(a)$. By
      transitivity of the equality relation it follows that $f(a) = f(c)$, so 
      that $a \sim c$. Thus $\sim$ is transitive on $A$.


      An equivalence class of $\sim$, say $\overline{a}$, where $a \in A$, is,
      by definition, the fiber of $f$ over $f(a)$. Also, if $X$ is a fiber of
      $f$, then every member of $X$ maps to the same element of $B$, so that $X$
      is an equivalence class of $\sim$. Thus the equivalence classes are the 
      fibers of $f$. \qed
%%%%%%%%%%%%%%%%%%%%%%%%%%%%%%%%%%%Prob0.1.8%%%%%%%%%%%%%%%%%%%%%%%%%%%%%%%%%%%%
   \item[0.1.8] \textbf{Proposition 1:}
                \begin{enumerate}
                   \item The map $f$ is injective if and only if $f$ has a left
                         inverse.
                   \item The map $f$ is surjective if and only if $f$ has a
                         right inverse.
                   \item The map $f$ is a bijection if and only if there exists
                         $g : B \rightarrow A$ such that $f \circ g$ is the
                         identity map on $B$ and $g \circ f$ is the identity map
                         on $A$.
                   \item If $A$ and $B$ are finite sets with the same number of
                         elements, then $f : A \rightarrow B$ is bijective if
                         and only if $f$ is injective if and only if $f$ is
                         surjective.
                \end{enumerate}

      \textbf{Proof.}
      \begin{enumerate}
         \item ($\Leftarrow$) Suppose first that the map $f : A \rightarrow B$ 
               has a left inverse, say $g : B \rightarrow A$. We want to show 
               that $f$ has is injective. So consider $a, b \in A$ such that
               $f(a) = f(b)$. Then we have that $g(f(a)) = g(f(b))$ so that
               $a = b$ since $g$ is a left inverse of $f$. Thus $f$ is
               injective. \\
               ($\Rightarrow$) Now suppose that $f$ is injective. Then we must 
               show that $f$ has	a left inverse. Notice that since $f$ is 
               injective, it must be the case that for every $b'' \in f(A)$ 
               there exists a unique $a'' \in A$ such that $f(a'') = b''$. Fix 
               $a' \in A$ and consider the map
	            $$g : B \rightarrow A,$$
               defined by
	            \begin{equation*}
		            b \mapsto \left\{
			            \begin{array}{ll}
				            a & \text{if } f(a) = b,\\
                        a' & \text{if } b \in B\setminus f(A).
                     \end{array} \right.
               \end{equation*}
   
               So $(g \circ f)(a_1) = g(f(a_1)) = a_1$ for all $a_1 \in A$.
               Hence $g \circ f$ is the identity function on $A$, so that $g$ is
               a left inverse of $f$.

         \item $(\Leftarrow)$ Suppose that $f$ has a right inverse, say
               $g : B \rightarrow A$. We must now show that $f$ is also 
               surjective; so let $b \in B$. Then, by supposition, we have that 
               $f(g(b)) = b$. Thus $f$ maps $g(b)$---a member of $A$---to $b$, 
               so that $f$ is onto.
   
               $(\Rightarrow)$ Now suppose that $f$ is surjective. We now want 
               to show that $f$ has a right inverse. Consider
               $h : B \rightarrow A$, such that $h(b) \in f^{-1}(\{b\})$. 
               Observe that $f^{-1}(\{b\})$ is nonempty for all $b \in B$, 
               since $f$ is surjective. Now if $c \in B$, then $f(h(c)) = c$, 
               so that $h$ is a right inverse of $f$.
         \item $(\Leftarrow)$ Suppose that $g$ is a left and right inverse of
               $f$. Then by 1 and 2, $f$ is an injection and a surjection, so 
               that $f$ is a bijection.
   
               $(\Rightarrow)$ Now suppose that $f$ is a bijection. Let
               $b \in B$. We notice that the fiber of $f$ over $b$ is not 
               empty since $f$ is surjective, and this fiber contains exactly 
               one element of $A$. The latter is so since if $a_1, a_2$ are in 
               the fiber of $f$ over $b$, then $f(a_1) = f(a_2)$ so that 
               $a_1 = a_2$ by the injectivity of $f$. So let $g$ be the map
               $g : B \rightarrow A$ that maps $c \in B$ to the only element in 
               the fiber of $f$ over $c$. It follows that $f \circ g$ is the
               identity on $B$ and $g \circ f$ is the identity on $A$.

         \item Let $|A| = |B| = n \in \Z^+$. First we shall assume that $f$ is
               bijective. It immediately follows that $f$ is injective. Now 
               assume that $f$ is injective. Since $f$ is one to one, no two 
               elements of $A$ map to the same element in $B$. This implies that
               $f(A)$ must contain exactly $n$ elements since $A$ contains $n$ 
               elements. But $|B| = n$, so that $f(A) = B$. That is, $f$ is 
               surjective. Now suppose that $f$ is surjective. Since $f$ is
               surjective, none of its fibers is empty; thus, the number of
               fibers of $f$ must equal $|B| = n$. We shall argue by    
               contradiction that $f$ is injective. So suppose that $f$ is not
               injective. Let $f_1$, $f_2$, $\ldots$, $f_n$ be the $n$ fibers of
               $f$. Since $f$ is not injective, one of this fibers must contain
               more than 1 element. so assume without loss of generality that
               $|f_1| \ge 2$. Since these fibers are a partition of $A$ and
               since each fiber contains at least one element it follows that
               \begin{align*}
                  |A| &= |f_1| + |f_2| + \cdots + |f_n| \\
                      &\ge 2 + (n - 1) \\
                      &\ge n + 1,
               \end{align*}
               a contradiction since $|A| = n$. Thus $f$ is injective.	
      \end{enumerate}
\end{enumerate}
