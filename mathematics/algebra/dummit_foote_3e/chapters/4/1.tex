Let $G$ be a gorup and let $A$ be a nonempty set.

\begin{enumerate}
%%%%%%%%%%%%%%%%%%%%%%%%%%%%%%%%%%%%%4.1.1%%%%%%%%%%%%%%%%%%%%%%%%%%%%%%%%%%%%%%
   \item[4.1.1]   Let $G$ act on the set $A$. Prove that if $a, b \in A$ and
                  $b = g \cdot a$ for some $g \in G$, then $G_b = gG_ag^{-1}$
                  ($G_a$ is the stabilizer of $a$). Deduce that if $G$ acts
                  transitively on $A$ then the kernel of the action is
                  $\cap_{g \in G}gG_ag^{-1}$.
%%%%%%%%%%%%%%%%%%%%%%%%%%%%%%%%%%%%%4.1.2%%%%%%%%%%%%%%%%%%%%%%%%%%%%%%%%%%%%%%
   \item[4.1.2]   Let $G$ be a \textit{permuation group} on the set $A$
                  (i.e., $G \le S_A$), let $\sigma \in G$ and let $a \in A$.
                  Prove that $\sigma G_a \sigma^{-1} = G_{\sigma(a)}$. Deduce
                  that if $G$ acts transitively on $A$ then
                  $$\cap_{\sigma \in G}\sigma G_a\sigma^{-1} = 1.$$
%%%%%%%%%%%%%%%%%%%%%%%%%%%%%%%%%%%%%4.1.3%%%%%%%%%%%%%%%%%%%%%%%%%%%%%%%%%%%%%%
   \item[4.1.3]   Assume that $G$ is an abelian, transitive subgroup of $S_A$.
                  Show that $\sigma(a) \neq a$ for all $\sigma \in G - \{1\}$
                  and all $a \in A$. Deduce that $|G| = |A|$. [Use the preceding
                  exercise.]
%%%%%%%%%%%%%%%%%%%%%%%%%%%%%%%%%%%%%4.1.4%%%%%%%%%%%%%%%%%%%%%%%%%%%%%%%%%%%%%%
   \item[4.1.4]   Let $S_3$ act on the set $\Omega$ of ordered pairs:
                  $\{(i, j) : 1 \le i, j \le 3\}$ by
                  $\sigma((i, j) = (\sigma(i), \sigma(j))$. Find the orbits of
                  $S_3$ on $\Omega$. For each $\sigma \in S_3$ find the cycle
                  decomposition of $\sigma$ under this action (i.e., find its
                  cycle decomposition when $\sigma$ is considered as an element
                  of $S_9$---first fix a labelling of these nine ordered pairs).
                  For each orbit $\mathcal{O}$ of $S_3$ acting on these nine
                  points pick some $a \in \mathcal{O}$ and find the stabilizer
                  of $a$ in $S_3$.
%%%%%%%%%%%%%%%%%%%%%%%%%%%%%%%%%%%%%4.1.5%%%%%%%%%%%%%%%%%%%%%%%%%%%%%%%%%%%%%%
   \item[4.1.5]   For each of parts (a) and (b) repeat the preceding exercise
                  but with $S_3$ acting on the specified set:
                  \begin{enumerate}
                     \item the set of 27 triples $\{(i, j , k) :
                           1 \le i, j, k \le 3\}$
                     \item the set $\mathcal{P}(\{1, 2, 3\}) - \{\emptyset\}$ of
                           all 7 nonempty subsets of $\{1, 2, 3\}$.
                  \end{enumerate}
%%%%%%%%%%%%%%%%%%%%%%%%%%%%%%%%%%%%%4.1.6%%%%%%%%%%%%%%%%%%%%%%%%%%%%%%%%%%%%%%
   \item[4.1.6]   As in Exercise 12 of Section 2.2 let $R$ be the set of all
                  polynomials with integer coefficients in the independent
                  variables $x_1, x_2, x_3, x_4$ and let $S_4$ act on $R$ by
                  permuting the indices of the four variables:
                  $$\sigma \cdot p(x_1, x_2, x_3, x_4) =
                    p(x_{\sigma(1)}, x_{\sigma(2)},
                      x_{\sigma(3)}, x_{\sigma(4)})$$
                  for all $\sigma \in S_4$.
                  \begin{enumerate}
                     \item Find the polynomials in the orbit of $S_4$ on $R$
                           containing $x_1 + x_2$ (recall from Exercise 2.2.12
                           that the stabilizer of this polynomial has order 4).
                     \item Find the polynomials in the orbit of $S_4$ on $R$
                           containing $x_1x_2 + x_3x_4$ (recall from Exercise 12
                           in Section 2.2 that the stabilizer of this polynomial
                           has order 8).
                     \item Find the polynomials in the orbit of $S_4$ on $R$
                           containing $(x_1 + x_2)(x_3 + x_4)$.
                  \end{enumerate}
%%%%%%%%%%%%%%%%%%%%%%%%%%%%%%%%%%%%%4.1.7%%%%%%%%%%%%%%%%%%%%%%%%%%%%%%%%%%%%%%
   \item[4.1.7]   Let $G$ be a transitive permutation group on the finite set
                  $A$. A \textit{block} is a nonempty subset $B$ of $A$ such
                  that for all $\sigma \in G$ either $\sigma(B) = B$ or
                  $\sigma(B) \cap B = \emptyset$ (here $\sigma(B)$ is the set
                  $\{\sigma(b) : b \in B\}$.
                  \begin{enumerate}
                     \item Prove that if $B$ is a block containing the element
                           $a$ of $A$, then the set $G_B$ defined by
                           $G_B = \{\sigma \in G : \sigma(B) = B\}$ is a
                           subgroup of $G$ containing $G_a$.
                     \item Show that if $B$ is a block and $\sigma_1(B)$,
                           $\sigma_2(B)$, $\ldots$, $\sigma_n(B)$ are all the
                           distinct images of $B$ under the elements of $G$,
                           then these form a partition of $A$.
                     \item A (transitive) group $G$ on a set $A$ is said to be
                           \textit{primitive} if the only blocks in $A$ are the
                           trivial ones: the sets of size 1 and $A$ itself. Show
                           that $S_4$ is primitive on $A = \{1, 2, 3, 4\}$. Show
                           that $D_8$ is not primitive as a permutation group on
                           the four vertices of a square.
                     \item Prove that the transitive group $G$ is primitive on
                           $A$ if and only if for each $a \in A$, the only
                           subgroups of $G$ containing $G_a$ are $G_a$ and $G$
                           (i.e., $G_a$ is a \textit{maximal} subgroup of $G$,
                           cf. Exercise 2.4.16). [Use part(a).]
                  \end{enumerate}
%%%%%%%%%%%%%%%%%%%%%%%%%%%%%%%%%%%%%4.1.8%%%%%%%%%%%%%%%%%%%%%%%%%%%%%%%%%%%%%%
   \item[4.1.8]   A transitive permutation group $G$ on a set $A$ is called
                  \textit{doubly transitive} if for any (hence all) $a \in A$
                  the subgroup $G_a$ is transitive on the set $A - \{a\}$.
                  \begin{enumerate}
                     \item Prove that $S_n$ is doubly transitive on
                           $\{1, 2, \ldots, n\}$ for all $n \ge 2$.
                     \item Prove that a doubly transitive group is primitive.
                           Deduce that $D_8$ is not doubly transitive in its
                           action on the 4 vertices of a square.
                  \end{enumerate}
%%%%%%%%%%%%%%%%%%%%%%%%%%%%%%%%%%%%%4.1.9%%%%%%%%%%%%%%%%%%%%%%%%%%%%%%%%%%%%%%
   \item[4.1.9]   Assume $G$ acts transitively on the finite set $A$ and let $H$
                  be a normal subgroup of $G$. Let $\mathcal{O}_1$,
                  $\mathcal{O}_2$, $\ldots$, $\mathcal{O}_r$ be the distinct
                  orbits of $H$ on $A$.
                  \begin{enumerate}
                     \item Prove that $G$ permutes the sets $\mathcal{O}_1$,
                           $\mathcal{O}_2$, $\ldots$, $\mathcal{O}_r$ in the
                           sense that for each $g \in G$ and each
                           $i \in \{1, \ldots, r\}$ there is a $j$ such that
                           $g\mathcal{O}_i = \mathcal{O}_j$, where
                           $g\mathcal{O} = \{g \cdot a : a \in \mathcal{O}\}$
                           (i.e., in the notation of Exercise 7 the sets
                           $\mathcal{O}_1$, $\ldots$, $\mathcal{O}_r$ are
                           blocks). Prove that $G$ is transitive on
                           $\{\mathcal{O}_1, \ldots, \mathcal{O}_r\}$. Deduce
                           that all orbits of $H$ on $A$ have the same
                           cardinality.
                     \item Prove that if $a \in \mathcal{O}_1$ then
                           $|\mathcal{O}_1| = |H : H \cap G_a|$ and prove that
                           $r = |G : HGa|$. [Draw the sublattice describing the
                           Second Isomorphism Theorem for the subgroups $H$ and
                           $G_a$ of $G$. Note that $H \cap G_a = Ha$.]
                  \end{enumerate}
%%%%%%%%%%%%%%%%%%%%%%%%%%%%%%%%%%%%%4.1.10%%%%%%%%%%%%%%%%%%%%%%%%%%%%%%%%%%%%%
   \item[4.1.10]  Let $H$ and $K$ be subgroups of the group $G$. For each
                  $x \in G$ define the $HK$ \textit{double coset} of $x$ in $G$
                  to be the set
                  $$HxK = \{hxk : h \in H, k \in K\}.$$
                  \begin{enumerate}
                     \item Prove that $HxK$ is the union of the left cosets 
                           $x_1K, \ldots, x_nK$ where $\{x_1K, \ldots, x_nK\}$
                           is the orbit containing $xK$ of $H$ acting by left
                           multiplication on the set of left cosets of $K$.
                     \item Prove that $HxK$ is a union of right cosets of $H$.
                     \item Show that $HxK$ and $HyK$ are either the same set or
                           are disjoint for all $x, y \in G$. Show that the set
                           of $HK$ double cosets partitions $G$.
                     \item Prove that $|HxK| = |K| \cdot |H : H \cap xKx^{-1}|$.
                     \item Prove that $|HxK| = |H| \cdot |K : K \cap x^{-1}Hx|$.
                  \end{enumerate}
\end{enumerate}
