\begin{enumerate}
%%%%%%%%%%%%%%%%%%%%%%%%%%%%%%%%%%%%%2.4.1%%%%%%%%%%%%%%%%%%%%%%%%%%%%%%%%%%%%%%
   \item[2.4.1]   Prove that if $H$ is a subgroup of $G$ then $\cyc{H} = H$.
%%%%%%%%%%%%%%%%%%%%%%%%%%%%%%%%%%%%%2.4.2%%%%%%%%%%%%%%%%%%%%%%%%%%%%%%%%%%%%%%
   \item[2.4.2]   Prove that if $A$ is a subset of $B$ then
                  $\cyc{A} \le \cyc{B}$. Give an example where $A \subseteq B$
                  with $A \neq B$ but $\cyc{A} = \cyc{B}$.
%%%%%%%%%%%%%%%%%%%%%%%%%%%%%%%%%%%%%2.4.3%%%%%%%%%%%%%%%%%%%%%%%%%%%%%%%%%%%%%%
   \item[2.4.3]   Prove that if $H$ is an abelian subgroup of a group $G$ then
                  $\cyc{H, Z(G)}$ is abelian. Give an explicit example of an
                  abelian subgroup $H$ of a group $G$ such that
                  $\cyc{H, C_G(H)}$ is not abelian.
%%%%%%%%%%%%%%%%%%%%%%%%%%%%%%%%%%%%%2.4.4%%%%%%%%%%%%%%%%%%%%%%%%%%%%%%%%%%%%%%
   \item[2.4.4]   Prove that if $H$ is a subgroup of $G$ then $H$ is generated
                  by the set $H - \{1\}$.
%%%%%%%%%%%%%%%%%%%%%%%%%%%%%%%%%%%%%2.4.5%%%%%%%%%%%%%%%%%%%%%%%%%%%%%%%%%%%%%%
   \item[2.4.5]   Prove that the subgroup generated by any two distinct elements
                  of order 2 in $S_3$ is all of $S_3$.
%%%%%%%%%%%%%%%%%%%%%%%%%%%%%%%%%%%%%2.4.6%%%%%%%%%%%%%%%%%%%%%%%%%%%%%%%%%%%%%%
   \item[2.4.6]   Prove that the subgroup of $S_4$ generated by (1 2) and
                  (1 2)(3 4) is a noncyclic group of order 4.
%%%%%%%%%%%%%%%%%%%%%%%%%%%%%%%%%%%%%2.4.7%%%%%%%%%%%%%%%%%%%%%%%%%%%%%%%%%%%%%%
   \item[2.4.7]   Prove that the subgroup of $S_4$ generated by (1 2) and
                  (1 3)(2 4) is isomorphic to the dihegral group of order 8.
%%%%%%%%%%%%%%%%%%%%%%%%%%%%%%%%%%%%%2.4.8%%%%%%%%%%%%%%%%%%%%%%%%%%%%%%%%%%%%%%
   \item[2.4.8]   Prove that $S_4 = \cyc{(1\;2\;3\;4), (1\;2\;4\;3)}$.
%%%%%%%%%%%%%%%%%%%%%%%%%%%%%%%%%%%%%2.4.9%%%%%%%%%%%%%%%%%%%%%%%%%%%%%%%%%%%%%%
   \item[2.4.9]   Prove that $SL_2(\F_3)$ is the subgroup of $GL_2(\F_3)$
                  generated by $\left(\begin{tabular}{@{}cc@{}}
                     1 & 1 \\
                     0 & 1
                  \end{tabular}\right)$ and $\left(\begin{tabular}{@{}cc@{}}
                     1 & 0 \\
                     1 & 1
                  \end{tabular}\right)$. [Recall from Exercise 2.1.9 that
                  $SL_2(\F_3)$ is the subgroup of matrices of determinant 1. You
                  may assume this subgroup has order 24---this will be an
                  exercise in Section 3.2.]
%%%%%%%%%%%%%%%%%%%%%%%%%%%%%%%%%%%%%2.4.10%%%%%%%%%%%%%%%%%%%%%%%%%%%%%%%%%%%%%
   \item[2.4.10]  Prove that the subgroup of $SL_2(\F_3)$ generated by
                  $\left(\begin{tabular}{@{}cr@{}}
                     0 & $-1$ \\
                     1 & 0
                  \end{tabular}\right)$ and $\left(\begin{tabular}{@{}cr@{}}
                     1 & 1 \\
                     1 & $-1$
                  \end{tabular}\right)$ is isomorphic to the quaternion group of
                  order 8. [Use a presentation for $Q_8$].
%%%%%%%%%%%%%%%%%%%%%%%%%%%%%%%%%%%%%2.4.11%%%%%%%%%%%%%%%%%%%%%%%%%%%%%%%%%%%%%
   \item[2.4.11]  Show that $SL_2(\F_3)$ and $S_4$ are two nonisomorphic groups
                  of order 24.
%%%%%%%%%%%%%%%%%%%%%%%%%%%%%%%%%%%%%2.4.12%%%%%%%%%%%%%%%%%%%%%%%%%%%%%%%%%%%%%
   \item[2.4.12]  Prove that the subgroup of upper triangular matrices in
                  $GL_3(\F_2)$ is isomorphic to the dihedral group of order 8.
%%%%%%%%%%%%%%%%%%%%%%%%%%%%%%%%%%%%%2.4.13%%%%%%%%%%%%%%%%%%%%%%%%%%%%%%%%%%%%%
   \item[2.4.13]  Prove that the multiplicative group of positive rational
                  numbers is generated by the set
                  $\left\{\D\frac{1}{p} : p \text{ is a prime}\right\}$.
%%%%%%%%%%%%%%%%%%%%%%%%%%%%%%%%%%%%%2.4.14%%%%%%%%%%%%%%%%%%%%%%%%%%%%%%%%%%%%%
   \item[2.4.14]  A group $H$ is called \textit{finitely generated} if there is
                  a finite set $A$ such that $H = \cyc{A}$.
                  \begin{enumerate}
                     \item Prove that every finite group is finitely generated.
                     \item Prove that $\Z$ is finitely generated.
                     \item Prove that every finitely generated subgroup of the
                           additive group $\Q$ is cyclic. [If $H$ is a finitely
                           generated subgroup of $\Q$, show that
                           $H \le \cyc{\frac{1}{k}}$, where $k$ is the product
                           of all the denominators which appear in a set of
                           generators for $H$.]
                     \item Prove that $\Q$ is not finitely generated.
                  \end{enumerate}
%%%%%%%%%%%%%%%%%%%%%%%%%%%%%%%%%%%%%2.4.15%%%%%%%%%%%%%%%%%%%%%%%%%%%%%%%%%%%%%
   \item[2.4.15]  Exhibit a proper subgroup of $\Q$ which is not cyclic.
%%%%%%%%%%%%%%%%%%%%%%%%%%%%%%%%%%%%%2.4.16%%%%%%%%%%%%%%%%%%%%%%%%%%%%%%%%%%%%%
   \item[2.4.16]  A subgroup $M$ of a group $G$ is called a
                  \textit{maximal subgroup} if $M \neq G$ and the only subgroups
                  of $G$ which contain $M$ are $M$ and $G$.
                  \begin{enumerate}
                     \item Prove that if $H$ is a proper subgroup of the finite
                           group $G$ then there is a maximal subgroup of $G$
                           containing $H$.
                     \item Show that the subgroup of all rotations in a dihedral
                           group is a maximal subgroup.
                     \item Show that if $G = \cyc{x}$ is a cyclic group of order
                           $n \ge 1$ then a subgroup $H$ is maximal if and only
                           if $H = \cyc{x^p}$ for some prime $p$ dividing $n$.
                  \end{enumerate}
%%%%%%%%%%%%%%%%%%%%%%%%%%%%%%%%%%%%%2.4.17%%%%%%%%%%%%%%%%%%%%%%%%%%%%%%%%%%%%%
   \item[2.4.17]  This is an exercise involving Zorn's Lemma (see Appendix I) to
                  prove that every nontrivial finitely generated group possesses
                  maximal subgroups. Let $G$ be a finitely generated group, say
                  $G = \cyc{g_1, g_2, \ldots, g_n}$, and let $\mathcal{S}$ be
                  the set of all proper subgroups of $G$. Then $\mathcal{S}$ is
                  partially ordered by inclusion. Let $\mathcal{C}$ be a chain
                  in $\mathcal{S}$.
                  \begin{enumerate}
                     \item Prove that the union, $H$, of all the subgroups in
                           $\mathcal{C}$ is a subgroup of $G$.
                     \item Prove that $H$ is a \textit{proper} subgroup. [If
                           not, each $g_i$ must lie in $H$ and so must lie in
                           some element of the chain $\mathcal{C}$. Use the
                           definition of a chain to arrive at a contradiction.]
                     \item Use Zorn's Lemma to show that $\mathcal{S}$ has a
                           maximal element (which is, by definition, a maximal
                           subgroup).
                  \end{enumerate}
%%%%%%%%%%%%%%%%%%%%%%%%%%%%%%%%%%%%%2.4.18%%%%%%%%%%%%%%%%%%%%%%%%%%%%%%%%%%%%%
   \item[2.4.18]  Let $p$ be a prime and let
                  $Z = \{z \in \C : z^{p^n} = 1 \text{ for some } n \in \Z^+\}$
                  (so $Z$ is the multiplicative group of all $p$-power roots of
                  unity in $\C$). For each $k \in \Z^+$ let
                  $H_k = \{z \in Z : z^{p^k} = 1\}$(the group of $p^k$th roots
                  of unity). Prove the following:
                  \begin{enumerate}
                     \item $H_k \le H_m$ if and only if $k \le m$
                     \item $H_k$ is cyclic for all $k$ (assume that for any
                           $n \in \Z^+$, $\{e^{2\pi it/n}:t =0,1,\ldots, n-1\}$
                           is the set of all $n^{\text{th}}$ roots of 1 in
                           $\C$)
                     \item every proper subgroup of $Z$ equals $H_k$ for some
                           $k \in \Z^+$ (in particular, every proper subgroup of
                           $Z$ is finite and cyclic)
                     \item $Z$ is not finitely generated.
                  \end{enumerate}
%%%%%%%%%%%%%%%%%%%%%%%%%%%%%%%%%%%%%2.4.19%%%%%%%%%%%%%%%%%%%%%%%%%%%%%%%%%%%%%
   \item[2.4.19]  A nontrivial abelian group $A$ (written multiplicatively) is
                  called \textit{divisible} if for each element $a \in A$ and
                  each nonzero integer $k$ there is an element $x \in A$ such
                  that $x^k = a$, i.e., each element has a $k^{\text{th}}$ root
                  in $A$ (in additive notation, each element is the
                  $k^{\text{th}}$ multiple of some element of $A$).
                  \begin{enumerate}
                     \item Prove that the additive group of rational numbers,
                           $\Q$, is divisible.
                     \item Prove that no finite abelian group is divisible.
                  \end{enumerate}
%%%%%%%%%%%%%%%%%%%%%%%%%%%%%%%%%%%%%2.4.20%%%%%%%%%%%%%%%%%%%%%%%%%%%%%%%%%%%%%
   \item[2.4.20]  Prove that if $A$ and $B$ are nontrivial abelian groups, then
                  $A \times B$ is divisible if and only if both $A$ and $B$ are
                  divisible groups.
\end{enumerate}

































