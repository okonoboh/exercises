\begin{enumerate}
%%%%%%%%%%%%%%%%%%%%%%%%%%%%%%%%%%%%%2.4.1%%%%%%%%%%%%%%%%%%%%%%%%%%%%%%%%%%%%%%
   \item[2.4.1]   Prove that if $H$ is a subgroup of $G$ then $\cyc{H} = H$.

      \textbf{Proof.} Suppose $H$ is a subgroup of some group $G$. Let
      $h \in H$. By definition of $\cyc{H}$, we have that $h^1 \in \cyc{H}$, so
      that $H \subseteq \cyc{H}$. Now let $h' \in \cyc{H}$. Then it follows that
      $h' = {h_1}^{a_1}{h_2}^{a_2}\cdots {h_n}^{a_n}$, where $h^i \in H$ and
      $a_i$ is an integer for all $1 \le i \le n$. Since $H$ is closed under
      multiplication, it follows that
      $h' = h_1^{a_1}h_2^{a_2}\cdots h_n^{a_n} \in H$. That is, 
      $\cyc{H} \subseteq H$, and we can conclude that $\cyc{H} = H$. \qed
%%%%%%%%%%%%%%%%%%%%%%%%%%%%%%%%%%%%%2.4.2%%%%%%%%%%%%%%%%%%%%%%%%%%%%%%%%%%%%%%
   \item[2.4.2]   Prove that if $A$ is a subset of $B$ then
                  $\cyc{A} \le \cyc{B}$. Give an example where $A \subseteq B$
                  with $A \neq B$ but $\cyc{A} = \cyc{B}$.

      \textbf{Proof.} Suppose $A$ is a subset of some set $B$. If
      $A = \emptyset$, then $\cyc{A} = \{1\} \le \cyc{B}$. So suppose that
      $A \neq \emptyset$. Let $a \in \cyc{A}$. Then it follows by definition
      that $a = {a_1}^{\varepsilon_1}{a_2}^{\varepsilon_2}\cdots
      {a_n}^{\varepsilon_n}$ for some $a_i \in A$ and integers $\varepsilon_i$
      for all $1 \le i \le n$. But since $a_i \in A$ and $A \subseteq B$, it
      follows that $a_i \in B$; thus $a \in B$, and we conclude that
      $\cyc{A} \subseteq \cyc{B}$, so that $\cyc{A} \le \cyc{B}$. \qed
%%%%%%%%%%%%%%%%%%%%%%%%%%%%%%%%%%%%%2.4.3%%%%%%%%%%%%%%%%%%%%%%%%%%%%%%%%%%%%%%
   \item[2.4.3]   Prove that if $H$ is an abelian subgroup of a group $G$ then
                  $\cyc{H, Z(G)}$ is abelian. Give an explicit example of an
                  abelian subgroup $H$ of a group $G$ such that
                  $\cyc{H, C_G(H)}$ is not abelian.

      \textbf{Proof.} Suppose $H$ is an abelian subgroup of some group $G$. Let
      $h \in H$. It follows that $h$ commutes with all elements of $H$ since
      $H$ is abelian; also $h$ commutes with all elements of $Z(G)$. Similarly,
      every element in $Z(G)$ commutes with one another and with the elements of
      $H$. Thus the set $H \cup Z(G)$ is abelian, so that $\cyc{H, Z(G)}$ is
      abelian.

      \textbf{Example.} We have $\cyc{\{1\}, C_{D_6}(\{1\})} =
      \cyc{\{1\}, D_6}= D_6$, a nonabelian group.
%%%%%%%%%%%%%%%%%%%%%%%%%%%%%%%%%%%%%2.4.4%%%%%%%%%%%%%%%%%%%%%%%%%%%%%%%%%%%%%%
   \item[2.4.4]   Prove that if $H$ is a subgroup of $G$ then $H$ is generated
                  by the set $H - \{1\}$.

      \textbf{Proof.} Let $H$ be a subgroup of some group $G$. If $H = \{1\}$,
      then $\cyc{H - \{1\}} = \cyc{\emptyset} = \{1\} = H$. So suppose that
      $H$ has more than 1 element. Let $h$ be a nonidentity element in $H$.
      Since $h \neq 1$, it follows that $h^{-1} \neq 1$, so that
      $h^{-1} \in H - \{1\}$. By definition, we must
      have $1 = h^1(h^{-1})^1 \in  \cyc{H - \{1\}}$. Since
      $H - \{1\} \subseteq \cyc{H - \{1\}}$ and $1 \in \cyc{H - \{1\}}$, it 
      follows that $H = [(H - \{1\}) \cup \{1\}] \subseteq \cyc{H - \{1\}}$. We
      know that $\cyc{H - \{1\}} \subseteq H$; thus $H = \cyc{H - \{1\}}$. \qed
%%%%%%%%%%%%%%%%%%%%%%%%%%%%%%%%%%%%%2.4.5%%%%%%%%%%%%%%%%%%%%%%%%%%%%%%%%%%%%%%
   \item[2.4.5]   Prove that the subgroup generated by any two distinct elements
                  of order 2 in $S_3$ is all of $S_3$.

      \textbf{Proof.} We know from the discussion on Page 43 of the textbook
      that $D_6 \cong S_3$, so it suffices to show that any two distinct
      elements of order 2 in $D_6$ is all of $D_6$. The elements of order 2 in
      $D_6$ are: $s$, $sr$, and $sr^2$. Since $\cyc{r, s} = D_6$, it suffices to
      show that $r$ and $s$ are both elements of $\cyc{s, sr}$, $\cyc{s, sr^2}$,
      and $\cyc{sr, sr^2}$. Now we have $s(sr) = r$, so that
      $r, s \in \cyc{s, sr}$. Also $(s)(sr^2)(s)(sr^2) = r$, so that
      $r, s \in \cyc{s, sr^2}$. Finally we have $(sr)(sr^2) = r$ and
      $(sr)(sr^2)(sr) = s$, so that $r, s \in \cyc{sr, sr^2}$. We can thus
      conclude that
      $$\cyc{s, sr} = \cyc{s, sr^2} = \cyc{sr, sr^2} = D_6,$$
      as desired. \qed
%%%%%%%%%%%%%%%%%%%%%%%%%%%%%%%%%%%%%2.4.6%%%%%%%%%%%%%%%%%%%%%%%%%%%%%%%%%%%%%%
   \item[2.4.6]   Prove that the subgroup of $S_4$ generated by (1 2) and
                  (1 2)(3 4) is a noncyclic group of order 4.

      \textbf{Proof.} Let $x = (1\;2)$ and $y = (1\;2)(3\;4)$. It is easy to see
      that $xy = yx^{-1}$. Thus every element of $\cyc{x, y}$ can be written in
      the form $x^iy^j$ for some integers $i$ and $j$. Since $|x| = |y| = 2$, it
      follows that $|\cyc{x, y}| \le 4$. But $x^2 = 1$ and $xy = (3\;4)$, so
      that $|\cyc{x, y}| = 4$. Notice that no element of $|\cyc{x, y}|$ has
      order 4. Thus $\cyc{x, y}$ is not cyclic. \qed
%%%%%%%%%%%%%%%%%%%%%%%%%%%%%%%%%%%%%2.4.7%%%%%%%%%%%%%%%%%%%%%%%%%%%%%%%%%%%%%%
   \item[2.4.7]   Prove that the subgroup of $S_4$ generated by (1 2) and
                  (1 3)(2 4) is isomorphic to the dihedral group of order 8.

      \textbf{Proof.} Let $y = (1\;2)$ and $z = (1\;3)(2\;4)$ be elements in
      $S_4$. Also let $x = yz = (1\;3\;2\;4)$. Since $yx = z$, it follows that
      $\cyc{z, y} = \cyc{x, y}$. Observe that $x$ and $y$ satisfy the same
      relations as $r$ and $s$ in $D_8$; that is, $x^4 = y^2 = 1$ and
      $xy = yx^{-1}$. The last relation tells us that every element in
      $\cyc{x, y}$ can be written in the form $y^ix^j$ for some integers
      $0 \le i < 4$ and $0 \le j < 2$. Computing these elements we shall get 
      that $|\cyc{x, y}| = 8$. By the discussion on Page 38, it follows that
      there exists a unique isomorphism that maps $r$ to $x$ and $s$ to $y$;
      thus $D_8 \cong \cyc{x, y}$. \qed
%%%%%%%%%%%%%%%%%%%%%%%%%%%%%%%%%%%%%2.4.8%%%%%%%%%%%%%%%%%%%%%%%%%%%%%%%%%%%%%%
   \item[2.4.8]   Prove that $S_4 = \cyc{(1\;2\;3\;4), (1\;2\;4\;3)}$.

      \textbf{Proof.} Let $x = (1\;2\;3\;4)$ and $y = (1\;2\;4\;3)$ be members
      of $S_4$. By computation, we see that the following 13 elements 
      $$1, x, x^2, x^3, y, y^2, y^3, xy, xyxy, yx, yxyx, xyxy^2,
        \text{ and } xyxy^2xyxy^2$$
      of $\cyc{x, y}$ are distinct. Thus $|\cyc{x, y}| \ge 13$. By Lagrange's
      Theorem, $|\cyc{x, y}|$ must divide $|S_4| = 24$. The only possibility is
      $|\cyc{x, y}| = 24$, so that $\cyc{x, y} = S_4$. \qed
%%%%%%%%%%%%%%%%%%%%%%%%%%%%%%%%%%%%%2.4.9%%%%%%%%%%%%%%%%%%%%%%%%%%%%%%%%%%%%%%
   \item[2.4.9]   Prove that $SL_2(\F_3)$ is the subgroup of $GL_2(\F_3)$
                  generated by
                  $$X = \left(\begin{tabular}{@{}cc@{}}
                     1 & 1 \\
                     0 & 1
                  \end{tabular}\right) \text{ and }
                    Y = \left(\begin{tabular}{@{}cc@{}}
                     1 & 0 \\
                     1 & 1
                  \end{tabular}\right).$$
                  [Recall from Exercise 2.1.9 that $SL_2(\F_3)$ is the subgroup
                  of matrices of determinant 1. You may assume this subgroup has
                  order 24---this will be an exercise in Section 3.2.]
                  
      \textbf{Proof.} Assume that $|SL_2(\F_3)| = 24$. It suffices to show that
      $\cyc{X, Y}$ has at least 13 elements(see preceding exercise). Since the
      following 13 distinct elements are in $\cyc{X, Y}$
      $$1, X, X^2, Y, Y^2, XY, XYXY, XYXYXY, YX, YXYXYX, X^2Y, X^2YX^2Y,
        \text{ and } YX^2$$
      it follows that $|\cyc{X, Y}| = 24$; thus we can conclude that
      $\cyc{X, Y} = SL_2(\F_3)$. \qed
%%%%%%%%%%%%%%%%%%%%%%%%%%%%%%%%%%%%%2.4.10%%%%%%%%%%%%%%%%%%%%%%%%%%%%%%%%%%%%%
   \item[2.4.10]  Prove that the subgroup of $SL_2(\F_3)$ generated by
                  $\left(\begin{tabular}{@{}cr@{}}
                     0 & $-1$ \\
                     1 & 0
                  \end{tabular}\right)$ and $\left(\begin{tabular}{@{}cr@{}}
                     1 & 1 \\
                     1 & $-1$
                  \end{tabular}\right)$ is isomorphic to the quaternion group of
                  order 8. [Use a presentation for $Q_8$].
                  
      \textbf{Proof.} Recall our presentation for $Q_8$ from Exercise 1.5.3
      $$Q_8 = \cyc{i, j : i^2 = j^2 = (ij)^2 = -1 \text{ and } (-1)^2 = 1}.$$
      Now let
      $$
         X = \left(\begin{tabular}{@{}cr@{}}
            0 & $-1$ \\
            1 & 0
         \end{tabular}\right) \text{ and }
         Y = \left(\begin{tabular}{@{}cr@{}}
            1 & 1 \\
            1 & $-1$
         \end{tabular}\right)
      $$
      be elements in $SL_2(\F_3)$. We want to show that $\cyc{X, Y} \cong Q_8$.
      Since $$X^2 = Y^2 = (XY)^2 = -I \text{ and }(-I)^2 = I,$$
      it follows that $X$ and $Y$ satisfy the relations that $i$ and $j$ satisfy
      in $Q_8$. Thus, according to the discussion on Page 38 of the text, there
      exists a unique surjective homomorphism, say $\varphi$, from $Q_8$ to
      $\cyc{X, Y}$ that maps $X$ to $i$ and $Y$ to $j$. Since $\varphi$ is
      surjective and since $Q_8$ is finite, it follows that $\varphi$ is
      bijective if and only if $8 = |Q_8| = |\cyc{X, Y}|$. Thus to complete the
      proof it suffices to show that $|\cyc{X, Y}| = 8$. To that end, we observe
      that $YX = -XY = X^2(XY) = X^3Y$; that is, every element of $\cyc{X, Y}$
      can be written in the form $X^rY^s$ for some integers $r$ and $s$. Since
      $|X| = |Y| = 4$, we have the following upper bounds for $r$ and $s$:
      $0 \le r < 4$ and $0 \le s < 4$; thus $|\cyc{X, Y}| \le 16$.  By computing
      these elements (for all the possible $r$ and $s$), we shall see that there
      are only 8 distinct elements, namely:
      $$1, Y, Y^2, Y^3, X, XY, XY^2, \text{ and } XY^3.$$
      Hence $|\cyc{X, Y}| = 8$, so that $\varphi$ is an isomorphism and we can
      conclude that $Q_8 \cong \cyc{X, Y}$. \mbox{ } \qed
%%%%%%%%%%%%%%%%%%%%%%%%%%%%%%%%%%%%%2.4.11%%%%%%%%%%%%%%%%%%%%%%%%%%%%%%%%%%%%%
   \item[2.4.11]  Show that $SL_2(\F_3)$ and $S_4$ are two nonisomorphic groups
                  of order 24.

      \textbf{Proof.} Let
      $$
         X = \left(\begin{tabular}{@{}cc@{}}
            1 & 1 \\
            0 & 1
         \end{tabular}\right) \text{ and }
         Y = \left(\begin{tabular}{@{}cc@{}}
            1 & 0 \\
            1 & 1
         \end{tabular}\right)
      $$
      be members of $SL_2(\F_3)$. From Exercise 2.4.9, we know that
      $\cyc{X, Y} = SL_2(\F_3)$. By computation, we have that $|X| = |Y| = 3$
      and $|XY| = |YX| = 4$. So if $SL_2(\F_3)$ were isomorphic to $S_4$, there
      would exist permutations $\sigma$ and $\alpha$ in $S_4$ such that
      $\cyc{\sigma, \alpha} = S_4$, $|\sigma| = |\alpha| = 3$, and
      $|\sigma\alpha| = |\alpha\sigma| = 4$. Let $S$ be all the permutations of 
      order 3 in $S_4$, so that
      $$S = \{(1\;2\;3), (1\;3\;2), (1\;2\;4), (1\;4\;2),
              (1\;3\;4), (1\;4\;3), (2\;3\;4), (2\;4\;3)\}.$$
      But no two elements $\sigma$ and $\alpha$ exist in $S$ such that
      $|\sigma\alpha| = |\alpha\sigma| = 4$. Thus $SL_2(\F_3)$ and $S_4$ are not
      congruent. \qed
%%%%%%%%%%%%%%%%%%%%%%%%%%%%%%%%%%%%%2.4.12%%%%%%%%%%%%%%%%%%%%%%%%%%%%%%%%%%%%%
   \item[2.4.12]  Prove that the subgroup of upper triangular matrices in
                  $GL_3(\F_2)$ is isomorphic to the dihedral group of order 8.

      \textbf{Proof.} Let $U$ be the subgroup of upper triangular matrices in
      $GL_3(\F_2)$; also let
      $$
         R = \left(\begin{tabular}{@{}ccc@{}}
            1 & 1 & 1 \\
            0 & 1 & 1 \\
            0 & 0 & 1
         \end{tabular}\right) \text{ and }
         S = \left(\begin{tabular}{@{}ccc@{}}
            1 & 1 & 0 \\
            0 & 1 & 0 \\
            0 & 0 & 1
         \end{tabular}\right)
      $$
      be members of $U$. By counting we have that $|U| = 8$. We observe that
      $$R^4 = S^2 = 1 \text{ and } RS = SR^{-1},$$
      so that $R$ and $S$ satisfy the same relations as $r$ and $s$ in $D_8$.
      Since
      $$U = \{1, R, R^2, R^3, S, SR, SR^2, SR^3\},$$
      it follows that
      $$U = \cyc{R, S: R^4 = S^2 = 1 \text{ and } RS = SR^{-1}}$$
      so that $U \cong D_8$. \qed
%%%%%%%%%%%%%%%%%%%%%%%%%%%%%%%%%%%%%2.4.13%%%%%%%%%%%%%%%%%%%%%%%%%%%%%%%%%%%%%
   \item[2.4.13]  Prove that the multiplicative group of positive rational
                  numbers is generated by the set
                  $\left\{\D\frac{1}{p} : p \text{ is a prime}\right\}$.

      \textbf{Proof.} Let $\Q^+ = \left\{\D\frac{a}{b} : \text{$a$ and $b$ are
      positive integers}\right\}$ be the multiplicative group of positive
      rational numbers and $P = \left\{\D\frac{1}{p} : p
      \text{ is a prime}\right\}$. Clearly $\cyc{P} \subseteq \Q^+$, so it 
      suffices to show reverse containment. It is clear that $1 \in \cyc{P}$.
      Now consider $x/y \in \Q^+$. We shall now investigate the following cases:

      \textbf{Case 1.} $x = 1$ and $y > 1$. By the Fundamental Theorem of
      Arithmetic, there exist primes $p_1, p_2, \ldots, p_n$ such that
      $y = p_1p_2\cdots p_n$. Since $\cyc{P}$ is closed under multiplication, it
      follows that
      $$\frac{x}{y} = \frac{1}{y} = \frac{1}{p_1}\frac{1}{p_2}\cdots
        \frac{1}{p_n} \in \cyc{P}.$$

      \textbf{Case 2.} $x > 1$ and $y = 1$. By Case 1, we have that
      $y/x \in \cyc{P}$. By closure we have that $(y/x)^{-1} = x/y \in \cyc{P}$.

      \textbf{Case 3.} $x > 1$ and $y > 1$. By Case 1, we have that
      $1/y \in \cyc{P}$; by Case 2, we have that $x/1 \in \cyc{P}$; thus it
      follows by closure of $\cyc{P}$ that $(x/1)(1/y) = x/y \in \cyc{P}$.

      Thus we conclude that $\Q^+ \subseteq \cyc{P}$ so that $\Q^+ = \cyc{P}$.
      \qed
%%%%%%%%%%%%%%%%%%%%%%%%%%%%%%%%%%%%%2.4.14%%%%%%%%%%%%%%%%%%%%%%%%%%%%%%%%%%%%%
   \item[2.4.14]  A group $H$ is called \textit{finitely generated} if there is
                  a finite set $A$ such that $H = \cyc{A}$.
                  \begin{enumerate}
                     \item Prove that every finite group is finitely generated.
                     \item Prove that $\Z$ is finitely generated.
                     \item Prove that every finitely generated subgroup of the
                           additive group $\Q$ is cyclic. [If $H$ is a finitely
                           generated subgroup of $\Q$, show that
                           $H \le \CYC{\D\frac{1}{k}}$, where $k$ is the product
                           of all the denominators which appear in a set of
                           generators for $H$.]
                     \item Prove that $\Q$ is not finitely generated.
                  \end{enumerate}

      \textbf{Proof.}

      \begin{enumerate}
         \item Let $H$ be a finite group. Then by Exercise 2.4.1, we have
               $\cyc{H} = H$, so that $H$ is finitely generated.
         \item The group of integers $\Z$ is finitely generated because
               $\Z = \cyc{1}$.
         \item Let $H$ be a finitely generated subgroup of $\Q$. We can assume
               that $H$ is nontrivial (since $H = \{1\}$ is clearly cyclic), so 
               that its set of generators is nonempty. Then it follows that
               $H = \CYC{\D\frac{a_1}{b_1}, \D\frac{a_2}{b_2}, \cdots,
               \D\frac{a_n}{b_n}}$, where $a_i/b_i \in \Q$. Let us now
               consider the cyclic group $\CYC{\D\frac{1}{k}}$, where
               $k = b_1b_2\cdots b_n$. Since $\CYC{\D\frac{1}{k}}$ is an
               additive group, it follows that $n/k \in \CYC{\D\frac{1}{k}}$
               for each integer $n$. To show that $H \le \CYC{\D\frac{1}{k}}$,
               it suffices to show that $a_i/b_i \in \CYC{\D\frac{1}{k}}$, for
               $1 \le i \le n$. So let $a_j/b_j$ be one of the generators for
               some $1 \le j \le n$. Notice that $k/b_j$ is an integer. Thus
               $(k/b_j)(1/k) = 1/b_j \in \CYC{\D\frac{1}{k}}$, so that
               $a_j/b_j = a_j (1/b_j) \in \CYC{\D\frac{1}{k}}$; that is,
               $H \le \CYC{\D\frac{1}{k}}$, as desired. By Theorem 7 (1), Page
               58, it follows that $H$ is cyclic.
         \item If $\Q$ were finitely generated, then, according to
               Exercise 2.4.14 (c) $\Q$ would be cyclic, a contradiction, since
               $\Q$ is not cyclic by our arguments in Exercise 2.3.15.
      \end{enumerate} \qed
%%%%%%%%%%%%%%%%%%%%%%%%%%%%%%%%%%%%%2.4.15%%%%%%%%%%%%%%%%%%%%%%%%%%%%%%%%%%%%%
   \item[2.4.15]  Exhibit a proper subgroup of $\Q$ which is not cyclic.

      \textbf{Solution.} Let $H = \left\{\D\frac{1}{2^i} : i \text{ is a 
      nonnegative integer}\right\}$ be a subset of the additive group $\Q$.
      We thus have that
      \begin{align*}
         \cyc{H} &= \left\{\frac{a_1}{2^{\varepsilon_1}} +
            \frac{a_2}{2^{\varepsilon_2}} + \cdots +
            \frac{a_n}{2^{\varepsilon_n}} : a_i, \varepsilon_i, n \in \Z, 
            \varepsilon_i \ge 0, n \ge 1 \right\} \\
                 &= \left\{\frac{a}{2^i} : a, i \in \Z, i \ge 0\right\}.
      \end{align*}
      Notice that the only prime factor that the denominator of each rational
      number(in its lowest term) in $\cyc{H}$ can have is 2; thus
      $1/7 \notin \cyc{H}$, so that $\cyc{H} \lneq Q$. Now suppose to the
      contrary that $\cyc{H}$ is cyclic; thus it follows by definition that
      $\cyc{H}$ is generated by some rational number $c/d$, where $c, d \in \Z$
      and $d \neq 0$. Let $k$ be the maximum nonnegative integer such that
      $2^k \mid d$. Since $c/2^{k+1} \in \cyc{H}$, it follows that there exists
      some integer $m$ such that $mc/d = c/2^{k+1}$. That is $m = d / 2^{k+1}$,
      so that $2^{k+1} \mid d$, contradicting the maximality of $k$. Thus
      $\cyc{H}$ is a proper non cyclic subgroup of $Q$.
%%%%%%%%%%%%%%%%%%%%%%%%%%%%%%%%%%%%%2.4.16%%%%%%%%%%%%%%%%%%%%%%%%%%%%%%%%%%%%%
   \item[2.4.16]  A subgroup $M$ of a group $G$ is called a
                  \textit{maximal subgroup} if $M \neq G$ and the only subgroups
                  of $G$ which contain $M$ are $M$ and $G$.
                  \begin{enumerate}
                     \item Prove that if $H$ is a proper subgroup of the finite
                           group $G$ then there is a maximal subgroup of $G$
                           containing $H$.
                     \item Show that the subgroup of all rotations in a dihedral
                           group is a maximal subgroup.
                     \item Show that if $G = \cyc{x}$ is a cyclic group of order
                           $n \ge 1$ then a subgroup $H$ is maximal if and only
                           if $H = \cyc{x^p}$ for some prime $p$ dividing $n$.
                  \end{enumerate}

      \textbf{Proof.}

      \begin{enumerate}
         \item Let $G$ be a finite group with more than 1 element and $H$ a 
               proper subgroup of $G$. If $H$ is a maximal subgroup then we are
               done; so assume that $H$ is not maximal in $G$. Suppose to the
               contrary that $H$ is not contained in some proper subgroup (other
               than itself) of $G$. Then it follows by definition that $H$ is a
               maximal subgroup, a contradiction. Thus there exists a proper
               subgroup $H_1$ such that $H \lneq H_1$. If $H_1$ is maximal, then
               we are done; otherwise, there exists a proper subgroup $H_2$
               such that $H \lneq H_1 \lneq H_2$; this process shall terminate
               for some $H_n$; that is,
               $H \lneq H_1 \lneq H_2 \lneq \cdots \lneq H_n$. We know that the
               process terminates since, otherwise, the sequence:
               $|H|$, $|H_1|$, $|H_2|$, $\ldots$ of strictly increasing positive 
               integers will increase without an upper bound, so that $G$ has a 
               proper subgroup of every order, contradicting our assumption that 
               $|G|$ is finite.
               Hence $H_n$ is maximal in $G$ and the proof is done. \qed
         \item Let $R = \{1, r, r^2, \ldots, r^{n-1}\}$ be the group of
               rotations in $D_{2n}$. Suppose $R \le H$ for some $H \le D_{2n}$.
               Since $H$ contains $R$, it follows that $|H| \ge n$. But
               according to Lagrange's Theorem we must have that $|H| \mid 2n$.
               Thus $|H| = n$, so that $H = R$; that is, the group of rotations
               is maximal in $D_{2n}$. \qed
         \item \textbf{Lemma 2.4.1} \textit{If $G = \cyc{x}$ is a finite cyclic 
               group and $\cyc{x^a} \le \cyc{x^b} \le G$, where $a$ and $b$ both 
               divide $|G|$, then $b \mid a$}.

               \textbf{Proof.} Let $G = \cyc{x}$ be a finite cyclic group.
               Suppose $\cyc{x^a}$ and $\cyc{x^b}$ are two subgroups of $G$,
               such that $a \mid |G|$ and $b \mid |G|$ and
               $\cyc{x^a} \le \cyc{x^b}$. Since $\cyc{x^a} \le \cyc{x^b}$, there
               exists an integer $t$ such that $x^a = x^{bt}$, so that
               $x^{a-bt} = 1$; that is, $a = bt + |G|k$ (Proposition 3, Pg. 55) 
               for some integer $k$. Since $b \mid bt$ and $b \mid |G|$, it
               follows that $b \mid a$, as desired. \qed

               Now we are ready to prove the main problem. So let $G = \cyc{x}$
               be a noncyclic group of order $n$. Suppose first that
               $H$ is a maximal subgroup of $G$. It follows by Theorem 7 (3),
               Pg. 58, that $H = \cyc{x^r}$, where $r = n / |H|$ (so that
               $r \mid n$). Now assume to the contrary that $r$ is composite; 
               thus $r = ab$ for some integers $1 < a \le b < r$. Thus
               $x^r = (x^a)^b$ so that $\cyc{x^r} \le \cyc{x^a}$. But
               $|x^a| = n/a < n$; that is, $\cyc{x^a}$ is a proper subgroup of
               $G$, contradicting the maximality of $\cyc{x^r}$; we can thus 
               conclude that $r$ is  prime. Now suppose conversely that
               $H = \cyc{x^p}$ for some prime $p$ dividing $n$. By Exercise 
               2.4.16 (a), $\cyc{x^p} \le \cyc{x^s}$, for some maximal subgroup
               $\cyc{x^s}$ in $G$. We can assume without loss of generality that
               $s \mid n$ (Theorem 7 (3), Pg. 58). Thus by Lemma 2.4.1, $s$ must
               divide $p$, so that $s = 1$ or $s = p$; since $\cyc{x^s}$ is a
               proper subgroup, it follows that $s = p$. That is $\cyc{x^p}$ is
               maximal. \qed
     \end{enumerate}
%%%%%%%%%%%%%%%%%%%%%%%%%%%%%%%%%%%%%2.4.17%%%%%%%%%%%%%%%%%%%%%%%%%%%%%%%%%%%%%
   \item[2.4.17]  This is an exercise involving Zorn's Lemma (see Appendix I) to
                  prove that every nontrivial finitely generated group possesses
                  maximal subgroups. Let $G$ be a finitely generated group, say
                  $G = \cyc{g_1, g_2, \ldots, g_n}$, and let $\mathcal{S}$ be
                  the set of all proper subgroups of $G$. Then $\mathcal{S}$ is
                  partially ordered by inclusion. Let $\mathcal{C}$ be a
                  nonempty chain in $\mathcal{S}$.
                  \begin{enumerate}
                     \item Prove that the union, $H$, of all the subgroups in
                           $\mathcal{C}$ is a subgroup of $G$.
                     \item Prove that $H$ is a \textit{proper} subgroup. [If
                           not, each $g_i$ must lie in $H$ and so must lie in
                           some element of the chain $\mathcal{C}$. Use the
                           definition of a chain to arrive at a contradiction.]
                     \item Use Zorn's Lemma to show that $\mathcal{S}$ has a
                           maximal element (which is, by definition, a maximal
                           subgroup).
                  \end{enumerate}

      \textbf{Proof.}

      \begin{enumerate}
         \item Let $H_1 \in \mathcal{C}$. Since $H_1 \le G$ and since
               $H_1 \subseteq H$, it follows that $1 \in H$. Now let
               $a, b \in H$. Then $a \in H_2$ and $b \in H_3$ for some $H_2$,
               $H_3 \in \mathcal{C}$. Since $\mathcal{C}$ is a chain, we have
               $H_2 \le H_3$ or $H_3 \le H_2$; so assume without loss of 
               generality that $H_2 \le H_3$, so that $a \in H_3$; since
               $H_3 \le G$, it follows that $ab^{-1} \in H_3$, so that
               $ab^{-1} \in H$, and thus, $H \le G$.
         \item Suppose to the contrary that $H = G$. We now want to show by
               induction on $n$ that some subgroup in $\mathcal{C}$ contains
               $g_1, g_2, \ldots, g_n$. Since $H = G$, we have that $g_1 \in H$,
               so that $g_1 \in H_1$ for some $H_1 \in \mathcal{C}$. So suppose
               $g_1, g_2, \ldots, g_{n-1} \in H_2$ for some
               $H_2 \in \mathcal{C}$. Similarly we must have that $g_n \in H_3$ 
               for some $H_3 \in \mathcal{C}$ because $g_n \in H$. Since
               $\mathcal{C}$ is a chain, either $H_2 \le H_3$ or $H_3 \le H_2$.
               If the former is true then $g_1, \ldots, gn \le H_3$; otherwise,
               $g_1, \ldots, gn \le H_2$; in either case, a member of
               $\mathcal{C}$ contains all the generators of $G$, so that
               $G \in \mathcal{C}$, a contradiction because we assumed that
               every element in $\mathcal{C}$ is a proper subgroup; thus
               $H \lneq G$.
         \item Since $H$ a proper subgroup of $G$, it follows that
               $H \in \mathcal{S}$. Also since $H$ is the union of all the 
               subgroups in $\mathcal{C}$, it follows that $H' \le H$ for all 
               $H' \in \mathcal{C}$, so that $H$ is an upper bound for
               $\mathcal{C}$. Thus it follows by Zorn's Lemma that
               $\mathcal{S}$.               
      \end{enumerate} \qed
%%%%%%%%%%%%%%%%%%%%%%%%%%%%%%%%%%%%%2.4.18%%%%%%%%%%%%%%%%%%%%%%%%%%%%%%%%%%%%%
   \item[2.4.18]  Let $p$ be a prime and let
                  $Z = \{z \in \C : z^{p^n} = 1 \text{ for some } n \in \Z^+\}$
                  (so $Z$ is the multiplicative group of all $p$-power roots of
                  unity in $\C$). For each $k \in \Z^+$ let
                  $H_k = \{z \in Z : z^{p^k} = 1\}$(the group of $p^k$th roots
                  of unity). Prove the following:
                  \begin{enumerate}
                     \item $H_k \le H_m$ if and only if $k \le m$
                     \item $H_k$ is cyclic for all $k$ (assume that for any
                           $n \in \Z^+$, $\{e^{2\pi it/n}:t =0,1,\ldots, n-1\}$
                           is the set of all $n^{\text{th}}$ roots of 1 in
                           $\C$)
                     \item every proper subgroup of $Z$ equals $H_k$ for some
                           $k \in \Z^+$ (in particular, every proper subgroup of
                           $Z$ is finite and cyclic)
                     \item $Z$ is not finitely generated.
                  \end{enumerate}
%%%%%%%%%%%%%%%%%%%%%%%%%%%%%%%%%%%%%2.4.19%%%%%%%%%%%%%%%%%%%%%%%%%%%%%%%%%%%%%
   \item[2.4.19]  A nontrivial abelian group $A$ (written multiplicatively) is
                  called \textit{divisible} if for each element $a \in A$ and
                  each nonzero integer $k$ there is an element $x \in A$ such
                  that $x^k = a$, i.e., each element has a $k^{\text{th}}$ root
                  in $A$ (in additive notation, each element is the
                  $k^{\text{th}}$ multiple of some element of $A$).
                  \begin{enumerate}
                     \item Prove that the additive group of rational numbers,
                           $\Q$, is divisible.
                     \item Prove that no finite abelian group is divisible.
                  \end{enumerate}
%%%%%%%%%%%%%%%%%%%%%%%%%%%%%%%%%%%%%2.4.20%%%%%%%%%%%%%%%%%%%%%%%%%%%%%%%%%%%%%
   \item[2.4.20]  Prove that if $A$ and $B$ are nontrivial abelian groups, then
                  $A \times B$ is divisible if and only if both $A$ and $B$ are
                  divisible groups.
\end{enumerate}

































