\begin{enumerate}
%%%%%%%%%%%%%%%%%%%%%%%%%%%%%%%%%%%%%2.5.1%%%%%%%%%%%%%%%%%%%%%%%%%%%%%%%%%%%%%%
   \item[2.5.1]   Let $H$ and $K$ be subgroups of $G$. Exhibit all possible
                  sublattices which show only $G$, 1, $H$, $K$, and their joins
                  and intersections. What distinguishes the different drawings?
%%%%%%%%%%%%%%%%%%%%%%%%%%%%%%%%%%%%%2.5.2%%%%%%%%%%%%%%%%%%%%%%%%%%%%%%%%%%%%%%
   \item[2.5.2]   In each of (a) to (d) list all subgroups of $D_{16}$ that
                  satisfy the given condition.
                  \begin{enumerate}
                     \item Subgroups that are contained in $\cyc{sr^2, r^4}$
                     \item Subgroups that are contained in $\cyc{sr^7, r^4}$
                     \item Subgroups that contain $\cyc{r^4}$
                     \item Subgroups that contain $\cyc{s}$.
                  \end{enumerate}
%%%%%%%%%%%%%%%%%%%%%%%%%%%%%%%%%%%%%2.5.3%%%%%%%%%%%%%%%%%%%%%%%%%%%%%%%%%%%%%%
   \item[2.5.3]   Show that the subgroup $\cyc{s, r^2}$ of $D_8$ is isomorphic
                  to $V_4$.
%%%%%%%%%%%%%%%%%%%%%%%%%%%%%%%%%%%%%2.5.4%%%%%%%%%%%%%%%%%%%%%%%%%%%%%%%%%%%%%%
   \item[2.5.4]   Use the given lattice to find all pairs of elements that
                  generate $D_8$ (there are 12 pairs).
%%%%%%%%%%%%%%%%%%%%%%%%%%%%%%%%%%%%%2.5.5%%%%%%%%%%%%%%%%%%%%%%%%%%%%%%%%%%%%%%
   \item[2.5.5]   Use the given lattice to find all elements $x \in D_{16}$
                  such that $D_{16} = \cyc{x, s}$ (there are 8 such elements
                  $x$).
%%%%%%%%%%%%%%%%%%%%%%%%%%%%%%%%%%%%%2.5.6%%%%%%%%%%%%%%%%%%%%%%%%%%%%%%%%%%%%%%
   \item[2.5.6]   Use the given lattices to help find the centralizers of every
                  element in the following groups:

                  (a) $D_8$ \qquad (b) $Q_8$ \qquad
                  (c) $S_3$ \qquad (d) $D_{16}$.
%%%%%%%%%%%%%%%%%%%%%%%%%%%%%%%%%%%%%2.5.7%%%%%%%%%%%%%%%%%%%%%%%%%%%%%%%%%%%%%%
   \item[2.5.7]   Find the center of $D_{16}$.
%%%%%%%%%%%%%%%%%%%%%%%%%%%%%%%%%%%%%2.5.8%%%%%%%%%%%%%%%%%%%%%%%%%%%%%%%%%%%%%%
   \item[2.5.8]   In each of the following groups find the normalizer of each
                  subgroup:

                  (a) $S_3$ \qquad (b) $Q_8$.
%%%%%%%%%%%%%%%%%%%%%%%%%%%%%%%%%%%%%2.5.9%%%%%%%%%%%%%%%%%%%%%%%%%%%%%%%%%%%%%%
   \item[2.5.9]   Draw the lattices of subgroups of the following groups:

                  (a) $\Z/16\Z$ \qquad (b) $\Z/24\Z$ \qquad
                  (c) $\Z/48\Z$. [See Exercise 6 in Section 3.]
%%%%%%%%%%%%%%%%%%%%%%%%%%%%%%%%%%%%%2.5.10%%%%%%%%%%%%%%%%%%%%%%%%%%%%%%%%%%%%%
   \item[2.5.10]  Classify groups of order 4 by proving that if $|G| = 4$ then
                  $G \cong Z_4$ or $G\cong V_4$. [See Exercise 36, Section 1.1.]
%%%%%%%%%%%%%%%%%%%%%%%%%%%%%%%%%%%%%2.5.11%%%%%%%%%%%%%%%%%%%%%%%%%%%%%%%%%%%%%
   \item[2.5.11]  Consider the group of order 16 with the following
                  presentation:

                  $$QD_{16} = \cyc{\sigma, \tau : \sigma^8 = \tau^2 = 1,
                    \sigma\tau = \tau\sigma^3}$$
                  (called the \textit{quasidihedral} or \textit{semidihedral}
                  group of order 16). This group has three subgroups of order 8:
                  $\cyc{\tau, \sigma^2} \cong D_8, \cyc{\sigma} \cong Z_8$ and
                  $\cyc{\sigma^2, \sigma\tau} \cong Q_8$ and every proper
                  subgroup is contained in one of these three subgroups. Fill in
                  the missing subgroups in the lattice of all subgroups of the 
                  quasidiheral group on the following page, exhibiting each
                  subgroup with at most two generators. (This is another example
                  of a nonplanar lattice.)
\end{enumerate}

\noindent The next three examples lead to two nonisomorphic groups that have the 
          same lattice of subgroups.

\begin{enumerate}
%%%%%%%%%%%%%%%%%%%%%%%%%%%%%%%%%%%%%2.5.12%%%%%%%%%%%%%%%%%%%%%%%%%%%%%%%%%%%%%
   \item[2.5.12]  The group
                  $A = Z_2 \times Z_4 = \cyc{a, b : a^2 = b^4 = 1, ab = ba}$ has
                  order 8 and has three subgroups of order 4:
                  $\cyc{a, b^2} \cong V_4$, $\cyc{b} \cong Z_4$ and
                  \begin{verbatim}
                     *
                     *
                     *
                     *
                     *
                     *
                     *
                     *
                     *
                  \end{verbatim}
                  $\cyc{ab} \cong Z_4$ and every proper subgroup is contained in
                  one of these three. Draw the lattice of all subgroups of $A$,
                  giving each subgroup in terms of at most two generators.
%%%%%%%%%%%%%%%%%%%%%%%%%%%%%%%%%%%%%2.5.13%%%%%%%%%%%%%%%%%%%%%%%%%%%%%%%%%%%%%
   \item[2.5.13]  The group
                  $G = Z_2 \times Z_8 = \cyc{x, y : x^2 = y^8 = 1, xy = yx}$ has
                  order 16 and has three subgroups of order 8:
                  $\cyc{x, y^2} \cong Z_2 \times Z_4$, $\cyc{y} \cong Z_8$ and
                  $\cyc{xy} \cong Z_8$ and every proper subgroup is contained in
                  one of these three. Draw the lattice of all subgroups of $G$,
                  giving each subgroup in terms of at most two generators.
%%%%%%%%%%%%%%%%%%%%%%%%%%%%%%%%%%%%%2.5.14%%%%%%%%%%%%%%%%%%%%%%%%%%%%%%%%%%%%%
   \item[2.5.14]  Let $M$ be the group of order 16 with the following 
                  presentation:
                  $$\cyc{u, v : u^2 v^8 = 1, vu = uv^5}$$
                  (sometimes called the \textit{modular} group of order 16). It
                  has three subgroups of order 8: $\cyc{u, v^2}$, $\cyc{v}$, and
                  $\cyc{uv}$ and every proper subgroup is contained in one of
                  these three. Prove that $\cyc{u, v^2} \cong Z_2 \times Z_4$,
                  $\cyc{v} \cong Z_8$ and $\cyc{uv} \cong Z_8$. Show that the
                  lattice of subgroups of $M$ is the same as the lattice of
                  subgroups of $Z_2 \times Z_8$ (cf. Exercise 13) but that these
                  two groups are not isomorphic.
%%%%%%%%%%%%%%%%%%%%%%%%%%%%%%%%%%%%%2.5.15%%%%%%%%%%%%%%%%%%%%%%%%%%%%%%%%%%%%%
   \item[2.5.15]  Describe the isomorphism type of each of the three subgroups
                  of $D_{16}$ of order 8.
%%%%%%%%%%%%%%%%%%%%%%%%%%%%%%%%%%%%%2.5.16%%%%%%%%%%%%%%%%%%%%%%%%%%%%%%%%%%%%%
   \item[2.5.16]  Use the lattice of subgroups of the quasidihedral group of
                  order 16 to show that every element of order 2 is contained in
                  the proper subgroup $\cyc{\tau, \sigma^2}$.
%%%%%%%%%%%%%%%%%%%%%%%%%%%%%%%%%%%%%2.5.17%%%%%%%%%%%%%%%%%%%%%%%%%%%%%%%%%%%%%
   \item[2.5.17]  Use the lattice of subgroups of the modular group $M$ of order
                  16 to show that the set $\{x \in M : x^2 = 1\}$ is a subgroup
                  of $M$ isomorphic to the Klein 4-group.
%%%%%%%%%%%%%%%%%%%%%%%%%%%%%%%%%%%%%2.5.18%%%%%%%%%%%%%%%%%%%%%%%%%%%%%%%%%%%%%
   \item[2.5.18]  Use the lattice to help find the centralizer of every element
                  of $QD_{16}$.
%%%%%%%%%%%%%%%%%%%%%%%%%%%%%%%%%%%%%2.5.19%%%%%%%%%%%%%%%%%%%%%%%%%%%%%%%%%%%%%
   \item[2.5.19]  Use the lattice to help find $N_{D_{16}}(\cyc{s, r^4})$.
%%%%%%%%%%%%%%%%%%%%%%%%%%%%%%%%%%%%%2.5.20%%%%%%%%%%%%%%%%%%%%%%%%%%%%%%%%%%%%%
   \item[2.5.20]  Use the lattice of subgroups of $QD_{16}$ to help find the
                  normalizers.

                  (a) $N_{QD_{16}}(\cyc{\tau\sigma})$ \qquad
                  (b) $N_{QD_{16}}(\cyc{\tau, \sigma^4})$.
\end{enumerate}
