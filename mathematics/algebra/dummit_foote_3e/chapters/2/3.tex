\begin{enumerate}
   \item[2.3.0]  \textcolor{red}{MISSING CONTAINMENT.}
%%%%%%%%%%%%%%%%%%%%%%%%%%%%%%%%%%%%%2.3.1%%%%%%%%%%%%%%%%%%%%%%%%%%%%%%%%%%%%%%
   \item[2.3.1]   Find all subgroups of $Z_{45} = \cyc{x}$, giving a generator
                  for each. Describe the containments between these subgroups.
                  
      \textbf{Solution.} Since the positive divisors of 45 are: 1, 3, 5, 9, 15,
      and 45, it follows that the subgroups of $Z_{45}$ are
      $$\cyc{x}, \cyc{x^3}, \cyc{x^5}, \cyc{x^9}, \cyc{x^{15}}, \text{ and }
        \cyc{x^{45}}.$$
        
      We have the following containments:
      $$
         \begin{tabular}{>{$}c<{$}>{$}c<{$}>{$}c<{$}>{$}c<{$}>{$}c<{$}>{$}c<{$}>{$}c<{$}}
            \cyc{x^{45}} & \le & \cyc{x^{15}} & \le & \cyc{x^5} & \le & \cyc{x} \\
            \cyc{x^{15}} & \le &  \cyc{x^3} & \le & \cyc{x} \\
            \cyc{x^9} & \le &  \cyc{x^3} & \le & \cyc{x}
         \end{tabular}
      $$
%%%%%%%%%%%%%%%%%%%%%%%%%%%%%%%%%%%%%2.3.2%%%%%%%%%%%%%%%%%%%%%%%%%%%%%%%%%%%%%%
   \item[2.3.2]   If $x$ is an element of the finite group $G$ and $|x| = |G|$,
                  prove that $G = \cyc{x}$. Give an explicit example to show 
                  that this result need not be true if $G$ is an infinite group.
                  
      \textbf{Proof.} Let $G$ be a finite group, so that $|G| = n \in \Z^+$.
      Suppose that there exists $x \in G$ such that $|x| = n$. Clearly
      $\cyc{x} \subseteq G$. But $|\cyc{x}| = n$ since $|x| = n$; thus
      $G \subseteq \cyc{x}$ so that $G = \cyc{x}$. Now let $G = \Z$. We have
      that $|\cyc{2}| = |G|$ but $G \neq \cyc{2}$. \qed
%%%%%%%%%%%%%%%%%%%%%%%%%%%%%%%%%%%%%2.3.3%%%%%%%%%%%%%%%%%%%%%%%%%%%%%%%%%%%%%%
   \item[2.3.3]   Find all generators for $\Z/48\Z$.
   
      \textbf{Solution.} The generators for $\Z/48\Z$ are: $\cyc{\overline{1}}$,
      $\cyc{\overline{5}}$, $\cyc{\overline{7}}$, $\cyc{\overline{11}}$,
      $\cyc{\overline{13}}$, $\cyc{\overline{17}}$, $\cyc{\overline{19}}$,
      $\cyc{\overline{23}}$, $\cyc{\overline{25}}$, $\cyc{\overline{29}}$,
      $\cyc{\overline{31}}$, $\cyc{\overline{35}}$, $\cyc{\overline{37}}$,
      $\cyc{\overline{41}}$, $\cyc{\overline{43}}$, and $\cyc{\overline{47}}$.
%%%%%%%%%%%%%%%%%%%%%%%%%%%%%%%%%%%%%2.3.4%%%%%%%%%%%%%%%%%%%%%%%%%%%%%%%%%%%%%%
   \item[2.3.4]   Find all generators for $\Z/202\Z$.
   
      \textbf{Solution.} Let $S$ be the set of generators for $\Z/202\Z$. Then
      $|S| = 100$ since
      $$S = \{\cyc{x} : x \text{ is odd and positive}, x \neq 101, \text{ and } x < 202\}.$$
%%%%%%%%%%%%%%%%%%%%%%%%%%%%%%%%%%%%%2.3.5%%%%%%%%%%%%%%%%%%%%%%%%%%%%%%%%%%%%%%
   \item[2.3.5]   Find the number of generators for $\Z/49000\Z$.
   
      \textbf{Solution.} For a positive integer $n$ let $\varphi(n)$ be the
      number of positive integers---less than or equal to $n$---that are
      relatively prime to $n$. Then the number of generators for $\Z/49000\Z$ is
      $\varphi(49000) = \varphi(2^35^37^2) =
      \varphi(2^3)\varphi(5^3)\varphi(7^2) = 16800$. 
%%%%%%%%%%%%%%%%%%%%%%%%%%%%%%%%%%%%%2.3.6%%%%%%%%%%%%%%%%%%%%%%%%%%%%%%%%%%%%%%
   \item[2.3.6]   In $\Z/48\Z$ write out all elements of $\cyc{\overline{a}}$
                  for every $\overline{a}$. Find all inclusions between
                  subgroups in $\Z/48\Z$.
      
      \textbf{Solution.}
      $$
         \begin{tabular}{|c|c|} \hline
            \textbf{Generators} & \textbf{Subgroups in} $\Z/48\Z$ \\ \hline
            0 & $\{0\}$ \\ \hline
            24 & $\{0, 24\}$ \\ \hline
            16, 32 & $\{0, 16, 32\}$ \\ \hline
            12, 36 & $\{0, 12, 24, 36\}$ \\ \hline
            8, 40 & $\{0, 8, 16, 24, 32, 40\}$ \\ \hline
            6, 18, 30, 42 & $\{0, 6, 12, 18, 24, 30, 36, 42\}$ \\ \hline
            4,20,28,44 & $\{0,4,8,12,16, 20, 24, 28, 32, 36, 40, 44\}$ \\ \hline
            3, 9, 15, 21, 27, 33, 39, 45 & $\{0, 3, 6, 9, 12, 15, 18, 21, 24,
            27, 30, 33, 36, 39, 42, 45\}$ \\ \hline            
            2, 10, 14, 22, 26, 34, 38, 46 & $\{x : 0 \le x \le 46,
            x \text{ is even}\}$ \\ \hline
            \text{See Exercise } 2.3.3 & $\Z/48\Z$ \\ \hline
         \end{tabular}
      $$
%%%%%%%%%%%%%%%%%%%%%%%%%%%%%%%%%%%%%2.3.7%%%%%%%%%%%%%%%%%%%%%%%%%%%%%%%%%%%%%%
   \item[2.3.7]   Let $Z_{48} = \cyc{x}$ and use the isomorphism
                  $\Z/48\Z \cong Z_{48}$ given by $\overline{1} \mapsto x$ to
                  list all subgroups of $Z_{48}$ as computed in the preceding
                  exercise.
                  
      \textbf{Solution.}
      $$
         \begin{tabular}{|c|} \hline
            \textbf{Subgroups in} $Z_{48}$ \\ \hline
            $\{1\}$ \\ \hline
            $\{1, x^{24}\}$ \\ \hline
            $\{1, x^{16}, x^{32}\}$ \\ \hline
            $\{1, x^{12}, x^{24}, x^{36}\}$ \\ \hline
            $\{1, x^8, x^{16}, x^{24}, x^{32}, x^{40}\}$ \\ \hline
            $\{1, x^6, x^{12}, x^{18}, x^{24}, x^{30},x^{36},x^{42}\}$ \\ \hline
            $\{1,x^4,x^8,x^{12},x^{16}, x^{20}, x^{24}, x^{28}, x^{32}, x^{36},
               x^{40}, x^{44}\}$ \\ \hline
            $\{1, x^3, x^6, x^9, x^{12}, x^{15}, x^{18}, x^{21}, x^{24},
            x^{27}, x^{30}, x^{33}, x^{36}, x^{39}, x^{42}, x^{45}\}$ \\ \hline
            $\{x^y : 0 \le y \le 46, y \text{ is even}\}$ \\ \hline
            $Z_{48}$ \\ \hline
         \end{tabular}
      $$
%%%%%%%%%%%%%%%%%%%%%%%%%%%%%%%%%%%%%2.3.8%%%%%%%%%%%%%%%%%%%%%%%%%%%%%%%%%%%%%%
   \item[2.3.8]   Let $Z_{48} = \cyc{x}$. For which integers $a$ does the map
                  $\varphi_a$ defined by $\varphi_a : \overline{1} \mapsto x^a$
                  extend to an \textit{isomorphism} from $\Z/48\Z$ onto
                  $Z_{48}$.
                  
      \textbf{Solution.} Suppose that $(a, 48) = 1$. Then it follows that $x^a$
      generates $Z_{48}$. Thus $\varphi_a$ is an isomorphism by Theorem 4 (Page
      56). Now suppose that $a$ is not relatively prime to 48. Then $x^a$ does
      not generate $Z_{48}$, so that the image of $\varphi_a$ is not $Z_{48}$.
      Hence $\varphi_a$ is an isomorphism if and only if $(a, 48) = 1$.
%%%%%%%%%%%%%%%%%%%%%%%%%%%%%%%%%%%%%2.3.9%%%%%%%%%%%%%%%%%%%%%%%%%%%%%%%%%%%%%%
   \item[2.3.9]   Let $Z_{36} = \cyc{x}$. For which integers $a$ does the map
                  $\psi_a$ defined by $\psi_a : \overline{1} \mapsto x^a$ extend
                  to a \textit{well defined homomorphism} from $\Z/48\Z$ into
                  $Z_{36}$. Can $\psi_a$ ever be a surjective homomorphism?
                  
      \textbf{Solution.} First we shall find the restriction(s) on $a$ such that
      $\psi_a$ is well defined. Suppose $b = c$ for some $b, c \in \Z/48\Z$. It
      suffices to show that $\psi_a(b) = \psi_a(c)$. Since $b = c$, there exists
      an integer $k$ such that $b = c + 48k$. Thus $\psi_a(b) = \psi_a(c+48k)$,
      so that
      $\psi_a(b)=(x^a)^{c+48k}=x^{ac + 48ak}= x^{ac}x^{48ak}=\psi_a(c)x^{12ak}$.
      So we must require $x^{12ak} = 1$ for all $k \in \Z$. Now $x^{12ak} = 1$
      for all $k \in \Z$ if and only if $3 \mid a$ if and only if $\psi_a$ is
      well defined. It follows immediately that
      $\psi_a$ is an homomorphism since
      \begin{align*}
         \psi_a(p + q) &= (x^a)^{p+q} \\
            &= x^{ap+aq} \\
            &= x^{ap}x^{aq} \\
            &= (x^a)^p(x^a)^q \\
            &= \psi_a(p)\psi_a(q)
      \end{align*}      
      for all $p, q \in \Z/48\Z$.
      
      \textit{Can $\psi_a$ ever be a surjective homomorphism?} No!
      
      \textbf{Proof.} Suppose to the contrary that $\psi_a$ is surjective. Then
      there exists $y \in \Z/48\Z$ such that $\psi_a(y) = x$. That is
      $x^{ay} = x$, so that $x^{ay-1} = 1$; thus $ay - 1 = 36m$ for some integer
      $m$. Rearrange the equality $ay - 1 = 36m$ to get $1 = ay - 36m$. Recall
      that $3 \mid a$; since $3$ also divides 36, it follows that 3 must divide
      1, a contradiction. Thus $\psi_a$ can never be surjective. \qed
%%%%%%%%%%%%%%%%%%%%%%%%%%%%%%%%%%%%%2.3.10%%%%%%%%%%%%%%%%%%%%%%%%%%%%%%%%%%%%%
   \item[2.3.10]  What is the order of $\overline{30}$ in $\Z/54\Z$? Write out
                  all the elements and their orders in $\cyc{\overline{30}}$.
                  
      \textbf{Solution.} The order of $30$ in $\Z/54\Z$ is
      $$\frac{54}{(30, 54)} = 9.$$
      The elements of $\cyc{30}$ and their respective orders are:
      $$
         \begin{tabular}{|c|c|} \hline
            Element of $\cyc{30}$ & Order \\ \hline
            30 & 9 \\ \hline
             6 & 9 \\ \hline
            36 & 3 \\ \hline
            12 & 9 \\ \hline
            42 & 9 \\ \hline
            18 & 3 \\ \hline
            48 & 9 \\ \hline
            24 & 9 \\ \hline
             0 & 1 \\ \hline
         \end{tabular}
      $$
%%%%%%%%%%%%%%%%%%%%%%%%%%%%%%%%%%%%%2.3.11%%%%%%%%%%%%%%%%%%%%%%%%%%%%%%%%%%%%%
   \item[2.3.11]  Find all cyclic subgroups of $D_8$. Find a proper subgroup of
                  $D_8$ which is not cyclic.
                  
      \textbf{Solution.} In $D_8$, only $r$ and $r^4$ have order 4. Thus
      $\{1, r, r^2, r^3\}$ is the only cyclic subgroup of order 4. The trivial
      subgroup is the only cyclic subgroup of order 1. Finally there are 5
      cyclic subgroups of order 2 and they are of the form $\{1, x\}$ where
      $x \in \{r^2, s, sr, sr^2, sr^3\}$. The set $\{1, s, r^2, sr^2\}$ is a
      non-cyclic proper subgroup of $D_8$.
%%%%%%%%%%%%%%%%%%%%%%%%%%%%%%%%%%%%%2.3.12%%%%%%%%%%%%%%%%%%%%%%%%%%%%%%%%%%%%%
   \item[2.3.12]  Prove that the following groups are \textit{not} cyclic:
                  \begin{enumerate}
                     \item $Z_2 \times Z_2$
                     \item $Z_2 \times \Z$
                     \item $\Z \times \Z$.
                  \end{enumerate}
      
      \textbf{Proof.}
      \begin{enumerate}
         \item The order of $Z_2 \times Z_2$ is 4, but no element in this group
               has order 4; thus $Z_2 \times Z_2$ is not cyclic.
         \item Let $Z_2 = \cyc{x}$. Observe that $Z_2 \times \Z$ is not finite,
               so in order for it to be cyclic it must be isomorphic to $\Z$.
               But this is not the case since $Z_2 \times \Z$ has two elements
               of finite order(namely $(1, 0)$ and $(x, 0)$) while $\Z$ has
               exactly 1 element of finite order.
         \item Suppose to the contrary that $\Z \times \Z$ is cyclic. Then there
               exist nonzero integers $a$ and $b$ such that
               $$\Z \times \Z = \cyc{(a,b)} = \{(na, nb) : n \in \Z\}.$$
               Thus there exists an integer $m$ such that
               $(ma, mb) = (0, 1)$. That is, $ma = 0$ and $mb = 1$. Since
               $ma = 0$, we must have $m = 0$ or $a = 0$. If $m$ is 0, then
               $(ma, mb) = (0, 0) \neq (0, 1)$, a contradiction; thus we must
               have $a = 0$, contradicting our assumption that $a$ is nonzero.
               Thus $\Z \times \Z$ is not cyclic.
      \end{enumerate} \qed
%%%%%%%%%%%%%%%%%%%%%%%%%%%%%%%%%%%%%2.3.13%%%%%%%%%%%%%%%%%%%%%%%%%%%%%%%%%%%%%
   \item[2.3.13]  Prove that the following pairs of groups are \textit{not}
                  isomorphic:
                  \begin{enumerate}
                     \item $\Z \times Z_2$ and $\Z$
                     \item $\Q \times Z_2$ and $\Q$.
                  \end{enumerate}
      
      \textbf{Proof.}
      \begin{enumerate}
         \item By Exercise 1.6.11, we know that $\Z \times Z_2$ is isomorphic to
               $Z_2 \times \Z$. By Exercise 2.3.12, $Z_2 \times \Z$ is not
               cyclic; thus $\Z \times Z_2$ is not cyclic. That is,
               $\Z \times Z_2$ is not isomorphic to $\Z$.
         \item Let $Z_2 = \cyc{x}$. It immediately follows that
               $\Q \times Z_2$ and $\Q$ are not isomorphic since $\Q \times Z_2$
               has two elements of finite order(namely $(0, 1)$ and $(0, x)$)
               while $\Q$ has exactly 1 element of finite order.
      \end{enumerate} \qed
%%%%%%%%%%%%%%%%%%%%%%%%%%%%%%%%%%%%%2.3.14%%%%%%%%%%%%%%%%%%%%%%%%%%%%%%%%%%%%%
   \item[2.3.14]  Let $\sigma =$ (1 2 3 4 5 6 7 8 9 10 11 12). For each of the
                  following integers $a$ compute $\sigma^a$:
                  $$a = 13, 65, 626, 1195, -6, -81, -570,\text{ and } {-1211}.$$
                  
      \textbf{Solution.}
      
      \begin{alignat*}{4}
         &\sigma^{13}   &&= \sigma &&\text{ } \\
         &\sigma^{65}   &&= \sigma^5 &&=
            (1\;6\;11\;4\;9\;2\;7\;12\;5\;10\;3\;8) \\
         &\sigma^{626}  &&= \sigma^2 &&= (1\;3\;5\;7\;9\;11) \\
         &\sigma^{1195} &&= \sigma^7 &&=
            (1\;8\;3\;10\;5\;12\;7\;2\;9\;4\;11\;6\;13) \\
         &\sigma^{-6} &&= \sigma^6 &&= (1\;7)
            (1\;8\;3\;10\;5\;12\;7\;2\;9\;4\;11\;6\;13) \\
         &\sigma^{-81} &&= \sigma^3 &&= (1\;4\;7\;10) \\
         &\sigma^{-570} &&= \sigma^6 &&= (1\;7) \\
         &\sigma^{-1211} &&= \sigma
      \end{alignat*}
%%%%%%%%%%%%%%%%%%%%%%%%%%%%%%%%%%%%%2.3.15%%%%%%%%%%%%%%%%%%%%%%%%%%%%%%%%%%%%%
   \item[2.3.15]  Prove that $\Q \times \Q$ is not cyclic.
   
      \textbf{Proof.} Since $\Q$ is infinite and, by Exercise 1.6.6, $\Q$ is not
      isomorphic to $\Z$, it follows that $\Q$ is not cyclic. We know that the
      subgroup of every cyclic group is cyclic; since $\Q \times\{1\} \cong \Q$,
      it follows that $\Q \times \{1\}$ is not cyclic; thus $\Q \times \Q$ is
      not cyclic because it has a noncyclic subgroup, namely $\Q \times \{1\}$.
      \qed
%%%%%%%%%%%%%%%%%%%%%%%%%%%%%%%%%%%%%2.3.16%%%%%%%%%%%%%%%%%%%%%%%%%%%%%%%%%%%%%
   \item[2.3.16]  Assume $|x| = n$ and $|y| = m$. Suppose that $x$ and $y$
                  \textit{commute}: $xy = yx$. Prove that $|xy|$ divides the
                  least common multiple of $m$ and $n$. Need this be true if $x$
                  and $y$ do \textit{not} commute? Give an example of commuting
                  elements $x$, $y$ such that the order of $xy$ is not equal to
                  the least common multiple of $|x|$ and $|y|$.
                  
      \textbf{Proof.} Let $l = \text{lcm}(m, n)$. So there exist integers
      $m'$ and $n'$ such that $mm' = nn' = l$. So we have that
      $$(xy)^l = x^ly^l = x^{nn'}y^{mm'} = (x^n)^{n'}(y^m)^{m'} = 1.$$
      That is $|xy|$ divides $l$ (by Proposition 3, Page 55).
      
      \textit{Need this be true if $x$ and $y$ do not commute?} No! Let
      $$
         A = \left(\begin{tabular}{@{}cc@{}}
            0 & 1/2 \\
            2 & 0
         \end{tabular}\right) \text{ and }
         B = \left(\begin{tabular}{@{}cc@{}}
            0 & 1 \\
            1 & 0
         \end{tabular}\right).
      $$
      A simple computation will show us that although $|A| = |B| = 2$, we have
      that $|AB| = \infty$.
      
      \textbf{Example.} Consider $\Z/2\Z = \{0, 1\}$. Let $x = y = 1$. Then we
      have $|x| = |y| = 2$, so that lcm($|x|, |y|) = 2 \neq |x + y| = |0| = 1$.
      \qed
%%%%%%%%%%%%%%%%%%%%%%%%%%%%%%%%%%%%%2.3.17%%%%%%%%%%%%%%%%%%%%%%%%%%%%%%%%%%%%%
   \item[2.3.17]  Find a presentation for $Z_n$ with one generator.
   
      \textbf{Solution.} $Z_n = \cyc{x : x^n = 1}$.
%%%%%%%%%%%%%%%%%%%%%%%%%%%%%%%%%%%%%2.3.18%%%%%%%%%%%%%%%%%%%%%%%%%%%%%%%%%%%%%
   \item[2.3.18]  Show that if $H$ is any group and $h$ is an element of $H$
                  with $h^n = 1$, then there is a unique homomorphism from
                  $Z_n = \cyc{x}$ to $H$ such that $x \mapsto h$.
                  
      \textbf{Proof.} Let $n \in \Z^+$, $Z_n = \cyc{x}$, $H$ a group, and
      $h^n  = 1$ for some $h \in H$. First we shall show the existence of a
      homomorphism from $Z_n$ to $H$ such that $x \mapsto h$. So consider the
      map $\alpha : \cyc{x} \rightarrow H$ defined by $\alpha(x^a) = h^a$.
      Clearly $\alpha(x) = h$. Now we will show that $\alpha$ is well defined.
      Suppose $x^w = x^y$ for some $x^w, x^y \in Z_n$. Thus $w = y + nk$ for
      some integer $k$. Thus
      $$\alpha(x^w) = \alpha(x^{y+nk})=h^{y+nk}=h^{y}{h^n}^k =h^y=\alpha(x^y),$$
      so that $\alpha$ is well defined. Now we have that
      $$\alpha(x^px^q)=\alpha(x^{p+q})=h^{p+q}=h^ph^q=\alpha(x^p)\alpha(x^q),$$
      so that $\alpha$ is an homomorphism. Now to show uniqueness, we suppose
      that $\phi : \cyc{x} \rightarrow H$ is an homommorphism such that
      $\phi(x) = h$. Since $\phi$ is a homomorphism, it follows that
      $\phi(x^a) = h^a$. Thus $\phi = \alpha$, as desired. \qed
%%%%%%%%%%%%%%%%%%%%%%%%%%%%%%%%%%%%%2.3.19%%%%%%%%%%%%%%%%%%%%%%%%%%%%%%%%%%%%%
   \item[2.3.19]  Show that if $H$ is any group and $h$ is an element of $H$,
                  then there is a unique homomorphism from $\Z$ to $H$ such that
                  $1 \mapsto h$.
                  
      \textbf{Proof.} Let $H$ be a group and let $h \in H$. First we shall show
      that there exists a homomorphism from $\Z$ to $H$ such that $1 \mapsto h$.
      So consider the map $\alpha : \Z \rightarrow H$ defined by
      $n \mapsto h^n$. Clearly $\alpha(1) = h$ and
      $$\alpha(x+y) = h^{x+y} = h^xh^y = \alpha(x)\alpha(y) \text{ for all }
        x, y \in \Z^+,$$
      so that $\alpha$ is a homomorphism. To show uniqueness, suppose that
      $\alpha' : \Z \rightarrow H$ is an homomorphism such that
      $\alpha'(1) = h$. Then according to Exercise 1.6.1, we have that
      $\alpha'(n) = \alpha'(n\cdot1) = \alpha'(1)^n = h^n$ for all $n \in \Z$;
      that is, $\alpha' = \alpha$, as desired. \qed
%%%%%%%%%%%%%%%%%%%%%%%%%%%%%%%%%%%%%2.3.20%%%%%%%%%%%%%%%%%%%%%%%%%%%%%%%%%%%%%
   \item[2.3.20]  Let $p$ be a prime and let $n$ be a positive integer. Show
                  that if $x$ is an element of the group $G$ such that
                  $x^{p^n} = 1$ then $|x| = p^m$ for some $m \le n$.
                  
      \textbf{Proof.} Suppose that $x \in G$ such that $x^{p^n} = 1$. Then it
      follows by Proposition 3 (Page 55) that $|x|$ divides $p^n$. Since $p$ is
      a prime, its factors are $p^i$, $0 \le i \le n$. Thus $|x| = p^m$ for
      some nonnegative $m$ not greater than $n$. \qed
%%%%%%%%%%%%%%%%%%%%%%%%%%%%%%%%%%%%%2.3.21%%%%%%%%%%%%%%%%%%%%%%%%%%%%%%%%%%%%%
   \item[2.3.21]  Let $p$ be an odd prime and let $n$ be a positive integer
                  $\ge 2$. Use the Binomial Theorem to show that
                  $(1+p)^{p^{n-1}} \equiv 1$ (mod $p^n$) but
                  $(1+p)^{p^{n-2}} \not\equiv 1$ (mod $p^n$). Deduce that $1+p$
                  is an element of order $p^{n-1}$ in the multiplicative group
                  $(\Z/p^n\Z)^\times$.

      \textbf{Lemma 2.3.1.} \textit{For a positive integer $t$ and a prime $q$, 
      let $f_q(t)$ be the number of $q$ factors of $t!$ (i.e., $f_q(t)$ is
      the greatest nonnegative integer $j$ such that $q^j \mid t!$), then it 
      follows that $f_q(t) < \D\frac{t}{2}$ if $q > 2$ or $f_q(t) \le t - 1$ if
      $q = 2$}.

      \textbf{Proof.} Let $t$ be a positive integer and $q$ a prime. For a
      positive integer $r$, let $g_q(t, r)$ be the number of positive integers,
      less than or equal to $t$, that have at least $r$ number of $q$ factors.
      It follows that $g_q(t, r) = \D\gint{\frac{t}{q^r}}$, where $\gint{x}$ is 
      the greatest integer less than or equal to $x$. Finally let $k_t$ be the 
      maximum nonnegative integer such that $q^{k_t}$ is a multiple of some 
      positive integer not greater than $t$. That is
      $$k_t := \max\{i \in \Z^+ : q^i \mid m, \text{where } m \le t\}.$$
      Thus we have that
      \begin{align*}
         f_q(t) &= g_q(t, 1) + g_q(t, 2) + \cdots + g_q(t, k_t) \\
            &= \sum_{i=1}^{k_t} g_q(t, i)
            = \sum_{i=1}^{k_t} \gint{\frac{t}{q^i}} \\
            &\le \sum_{i=1}^{k_t} \frac{t}{q^i}
            < \sum_{i=1}^\infty \frac{t}{q^i} \\
            &= \frac{t}{q-1}. &[\text{Sum of Geometric Series}]
      \end{align*}

      That is $f_q(t) < \frac{t}{q-1} < \frac{t}{2}$ if $p > 2$ and
      $f_q(t) < \frac{t}{q-1} = t \le t - 1$ if $p = 2$. Observe that this
      allows us to write $t! = q^{f_q(t)} h_t$ for some unique $h_t \in \Z^+$, 
      so that $(h_t, q) = 1$. \qed \\

      \textit{We now want to show that $(1+p)^{p^{n-1}} \equiv 1$ (mod $p^n$).} 
      Note that since
      \begin{align*}
         \binom{p^{n-1}}{i} &= \frac{p^{n-1}(p^{n-1}-1)(p^{n-1}-2)
               \cdots(p^{n-1}-i+1)}{i!} \\
               &= \frac{p^{n-1}(p^{n-1}-1)(p^{n-1}-2)
               \cdots(p^{n-1}-i+1)}{p^{f_p(i)} h_i}
      \end{align*}
      is an integer and since $(h_i,p)=1$, for $0 \le i \le p^{n-1}$, it follows 
      that $h_i$ must divide $(p^{n-1}-1)(p^{n-1}-2)\cdots(p^{n-1}-i+1)$. By the 
      Binomial Theorem, we have that
      \begin{align*}
         (1+p)^{p^{n-1}} &= \sum_{i=0}^{p^{n-1}}\binom{p^{n-1}}{i}p^i \\
            &= \sum_{i=0}^{p^{n-1}}p^i\frac{p^{n-1}(p^{n-1}-1)(p^{n-1}-2)
               \cdots(p^{n-1}-i+1)}{i!} \\
            &= \sum_{i=0}^{p^{n-1}}p^i\frac{p^{n-1}(p^{n-1}-1)(p^{n-1}-2)
               \cdots(p^{n-1}-i+1)}{p^{f_p(i)} h_i} \\
            &= 1 + p^n + p^n\sum_{i=2}^{p^{n-1}}\frac{p^{i-1}(p^{n-1}-1)
               (p^{n-1}-2) \cdots(p^{n-1}-i+1)}{p^{f_p(i)} h_i}.
      \end{align*}
      By Lemma 2.3.1, $f_p(i) < i / 2$, and $i / 2 \le i - 1$ for $i \ge 2$. 
      Thus $i - 1 - f_p(i) \ge 0$ (so that $p^{i - 1 - f_p(i)}$ is an integer) 
      if $i \ge 2$. We then have
      \begin{equation} \label{2_3_21_1}
         (1+p)^{p^{n-1}} = 1 + p^n + p^n\sum_{i=2}^{p^{n-1}}\frac{p^{i-1-f_p(i)}
        (p^{n-1}-1)(p^{n-1}-2) \cdots(p^{n-1}-i+1)}{h_i}
      \end{equation}
      Since $h_i$ divides $(p^{n-1}-1)(p^{n-1}-2) \cdots(p^{n-1}-i+1)$, it
      follows that
      $$\sum_{i=2}^{p^{n-1}}\frac{p^{i-1-f_p(i)}
        (p^{n-1}-1)(p^{n-1}-2) \cdots(p^{n-1}-i+1)}{h_i}$$
      is an integer and we can conclude from \eqref{2_3_21_1} that
      $(1+p)^{p^{n-1}} \equiv 1$ (mod $p^n$).

      \textit{We now want to show that $(1+p)^{p^{n-2}}\not\equiv1$
      (mod $p^n$)}. So
      \begin{IEEEeqnarray}{rCl}
         (1+p)^{p^{n-2}} &=& \sum_{i=0}^{p^{n-2}}\binom{p^{n-2}}{i}p^i
         \nonumber \\
            &=& \sum_{i=0}^{p^{n-2}}p^i\frac{p^{n-2}(p^{n-2}-1)(p^{n-2}-2)
               \cdots(p^{n-2}-i+1)}{i!} \nonumber \\
            &=& 1 + p^{n-1} + \sum_{i=2}^{p^{n-2}}p^i\frac{p^{n-2}(p^{n-2}-1)
            (p^{n-2}-2)\cdots(p^{n-2}-i+1)}{p^{f_p(i)} h_i} \nonumber \\
            &=& 1 + p^{n-1} \nonumber \\
            && +\: p^n\sum_{i=2}^{p^{n-2}}\frac{p^{i-2-f_p(i)}
            (p^{n-2}-1)(p^{n-2}-2)\cdots(p^{n-2}-i+1)}{h_i} \label{2_3_21_2}
      \end{IEEEeqnarray}

      By Lemma 2.3.1, $f_p(i) < i / 2$, and $i / 2 \le i - 2$ for $i \ge 4$. 
      Thus $i - 1 - f_p(i) \ge 0$ (so that $p^{i - 1 - f_p(i)}$ is an integer) 
      if $i \ge 4$. Notice that $f_p(2) = 0$ and $f_p(3) \le 1$ (equality holds 
      if and only if $p = 3$) for all odd primes $p$. Thus
      $i - 1 - f_p(i) \ge 0$ if $i \ge 2$. Since $h_i$ divides
      $(p^{n-2}-1)(p^{n-2}-2) \cdots(p^{n-2}-i+1)$, it follows that
      $$\sum_{i=2}^{p^{n-2}}\frac{p^{i-2-f_p(i)}
        (p^{n-2}-1)(p^{n-2}-2) \cdots(p^{n-2}-i+1)}{h_i}$$
      is an integer and we can conclude from \eqref{2_3_21_2} that
      $(1+p)^{p^{n-2}} \equiv 1 + p^{n-1}$ (mod $p^n$). But since
      $p^n$ does not divide $p^{n-1}$, it follows that
      $(1+p)^{p^{n-2}} \not\equiv 1$. By Exercise 2.3.20, we know that
      $|1+p| = p^m$ for some $m \le n - 1$. Suppose $m < n - 1$, so that
      $n - 2 - m \ge 0$. Then it follows that
      $$(1+p)^{n-2} = (1+p)^{p^m \cdot p^{n-2-m}} \equiv 1^{p^{n-2-m}} \equiv
         1 (\text{ mod } p^n),$$
      a contradiction. We can thus conclude that $|1 + p| = p^{n-1}$. \qed
%%%%%%%%%%%%%%%%%%%%%%%%%%%%%%%%%%%%%2.3.22%%%%%%%%%%%%%%%%%%%%%%%%%%%%%%%%%%%%%
   \item[2.3.22]  Let $n$ be an integer $\ge 3$. Use the Binomial Theorem to
                  show that $(1+2^2)^{2^{n-2}} \equiv 1$ (mod $2^n$) but
                  $(1+2^2)^{2^{n-3}} \not\equiv 1$ (mod $2^n$). Deduce that 5 is
                  an element of order $2^{n-2}$ in the multiplicative group
                  $(\Z/2^n\Z)^\times$.

      \textbf{Proof.} Let $n \ge 3$ be an integer and, for a positive integer
      $i$, let $f(i)$ be the greatest nonnegative integer $j$ such that
      $2^j \mid i!$. Then we can write $i! = 2^{f(i)} \cdot h_i$ for some unique
      $h_i \in \Z^+$, so that $(h_i, 2) = 1$. By the Binomial Theorem, we
      have that
      \begin{align}
         (1+2^2)^{2^{n-2}} &= \sum_{i=0}^{2^{n-2}}\binom{2^{n-2}}{i}2^{2i}
            \nonumber \\
            &= \sum_{i=0}^{2^{n-2}}2^{2i}\frac{2^{n-2}(2^{n-2}-1)(2^{n-2}-2)
               \cdots(2^{n-2}-i+1)}{i!}  \nonumber \\
            &= \sum_{i=0}^{2^{n-2}}2^{2i}\frac{2^{n-2}(2^{n-2}-1)(2^{n-2}-2)
               \cdots(2^{n-2}-i+1)}{2^{f(i)}h_i}  \nonumber \\
            &= 1 + 2^n\sum_{i=1}^{2^{n-2}}\frac{2^{2i-2-f(i)}(2^{n-2}-1)
               (2^{n-2}-2) \cdots(2^{n-2}-i+1)}{h_i}. \label{2_3_22_1}
      \end{align}
      For $i \ge 1$, we have by Lemma 2.3.1 that  $f(i) \le i - 1$, so that 
      $2i-2-f(i) \ge 0$; that is, $2^{2i-2-f(i)}$ is an integer. Also observe 
      that since $(h_i, 2) = 1$, $h_i$ must divide
      $(2^{n-2}-1)\cdots(2^{n-2}-i+1)$. Thus we can conclude from
      \eqref{2_3_22_1} that $(1+2^2)^{2^{n-2}} \equiv 1$ (mod $2^n$). Similarly
      we have that
      \begin{align}
         (1+2^2)^{2^{n-3}} &= 1 + 2^{n-1} + 2^n\sum_{i=2}^{2^{n-3}}
            \frac{2^{2i-3-f(i)}(2^{n-3}-1)(2^{n-3}-2) \cdots(2^{n-3}-i+1)}{h_i}. 
            \label{2_3_22_2}
      \end{align}
      For $i \ge 2$, we have that $2i-3-f(i) \ge 0$ (by Lemma 2.3.1); thus we
      can conclude from \eqref{2_3_22_2} that
      $(1+2^2)^{2^{n-3}} \equiv 1 + 2^{n-1}$ (mod $2^n$). But since
      $2^n$ does not divide $2^{n-1}$, it follows that
      $(1+2^2)^{2^{n-3}} \not\equiv 1$. By Exercise 2.3.20, we know that
      $|1+2^2| = 2^m$ for some $m \le n - 2$. Suppose $m < n - 2$, so that
      $n - 3 - m \ge 0$. Then it follows that
      $$(1+2^2)^{n-3} = (1+2^2)^{p^m \cdot 2^{n-3-m}} \equiv1^{2^{n-3-m}} \equiv
         1 (\text{ mod } 2^n),$$
      a contradiction. We can thus conclude that $|1 + 2^2| = 2^{n-2}$. \qed
%%%%%%%%%%%%%%%%%%%%%%%%%%%%%%%%%%%%%2.3.23%%%%%%%%%%%%%%%%%%%%%%%%%%%%%%%%%%%%%
   \item[2.3.23]  Show that $(\Z/2^n\Z)^\times$ is not cyclic for any $n \ge 3$.
                  [Find two distinct subgroups of order 2.]

      \textbf{Proof.} Let $n \ge 3$ be an integer. Consider
      $2^{n-1}-1, 2^n-1 \in \Z/2^n\Z$. Note that $2^{n-1}-1 \neq 1$ and
      $2^n-1 \neq 1$ because $1 < 2^{n-1}-1 < 2^n-1 < 2^n$. Since $2^{n-1}-1$
      and $2^n-1$ are both odd, it follows that both of them are relatively
      prime to $2^n$, so that they are both in $(\Z/2^n\Z)^\times$. Now we have
      that
      $$(2^{n-1}-1)(2^{n-1}-1) = 2^n2^{n-2}-2^n + 1 \equiv 1\text{ (mod }2^n)$$
      and
      $$(2^n-1)(2^n-1) = 2^n2^n-2^n2 + 1 \equiv 1\text{ (mod }2^n).$$
      Thus both $2^{n-1}-1$ and $2^n-1$ have order 2 in $(\Z/2^n\Z)^\times$.
      However, since these two elements are not equal, it follows that
      $(\Z/2^n\Z)^\times$ has two distinct subgroups of order 2, so that
      $(\Z/2^n\Z)^\times$ is not cyclic. \qed
%%%%%%%%%%%%%%%%%%%%%%%%%%%%%%%%%%%%%2.3.24%%%%%%%%%%%%%%%%%%%%%%%%%%%%%%%%%%%%%
   \item[2.3.24]  Let $G$ be a finite group and let $x \in G$.
                  \begin{enumerate}
                     \item Prove that if $g \in N_G(\cyc{x})$ then
                           $gxg^{-1} = x^a$ for some $a \in \Z$. 
                     \item Prove conversely that if $gxg^{-1} = x^a$ for some
                           $a \in \Z$ then $g \in N_G(\cyc{x})$. [Show first
                           that $gx^kg^{-1} = (gxg^{-1})^k = x^{ak}$ for any
                           integer $k$, so that $g\cyc{x}g^{-1} \le \cyc{x}$.
                           If $x$ has order $n$, show the elements $gx^ig^{-1}$,
                           $i = 0, 1, \ldots, n-1$ are distinct, so that
                           $|g\cyc{x}g^{-1}| = |\cyc{x}| = n$ and conclude that
                           $g\cyc{x}g^{-1} = \cyc{x}$.]
                  \end{enumerate}
                  Note that this cuts down some of the work in computing
                  normalizers of cyclic subgroups since one does not have to
                  check $ghg^{-1} \in \cyc{x}$ for every $h \in \cyc{x}$.

      \textbf{Proof.}

      \begin{enumerate}
         \item Let $g \in N_G(\cyc{x})$. It follows by definition that
               $g\cyc{x}g^{-1} = \cyc{x}$. Since $gxg^{-1} \in g\cyc{x}g^{-1}$,
               it follows that $gxg^{-1} \in \cyc{x}$, so that $gxg^{-1} = x^a$
               for some $a \in \Z$.
         \item Suppose $gxg^{-1} = x^a$ for some integer $a$. We can show by
               induction that $gx^kg^{-1} = (gxg^{-1})^k$ for any integer $k$.
               Let $gx^ig^{-1} \in g\cyc{x}g^{-1}$. Then it follows that
               $(gx^ig^{-1}) = (gxg^{-1})^i = x^{ai} \in \cyc{x}$, so that
               $g\cyc{x}g^{-1} \subseteq \cyc{x}$. Since $G$ is finite, we let
               $|x| = n$. Now suppose that $gx^rg^{-1} = gx^sg^{-1}$, where
               $0 \le r < s \le n-1$. By cancellation we get $x^s = x^r$, so
               that $n \mid s - r$, a contradiction since $s - r < n$. Thus the
               elements $1, gxg^{-1},, \ldots, gx^{n-1}g^{-1}$ are distinct, so
               that $|g\cyc{x}g^{-1}| = n = |\cyc{x}|$. That is,
               $g\cyc{x}g^{-1} = \cyc{x}$.
      \end{enumerate} \qed
%%%%%%%%%%%%%%%%%%%%%%%%%%%%%%%%%%%%%2.3.25%%%%%%%%%%%%%%%%%%%%%%%%%%%%%%%%%%%%%
   \item[2.3.25]  Let $G$ be a cyclic group of order $n$ and let $k$ be an
                  integer relatively prime to $n$. Prove that the map
                  $x \mapsto x^k$ is surjective. Use Lagrange's Theorem
                  (Exercise 1.7.19) to prove the same is true for any finite
                  group of order $n$. (For such $k$ each element has a
                  $k^{\text{th}}$ root in $G$. It follows from Cauchy's Theorem
                  in Section 3.2 that if $k$ is not relatively prime to the
                  order of $G$ then the map $x \mapsto x^k$ is not surjective.)

      \textbf{Proof.} We will only prove the second part. Let $G$ be a group of
      order $n$, $(k, n) = 1$, and $x \mapsto x^k$ be a map from $G$ to itself.
      Let $y \in G$. Since $k$ is relatively prime to $n$, there exist integers 
      $a$ and $b$ such that $ak+bn=1$. Thus
      $y = y^{ak+bn} = y^{ak}{y^n}^b = (y^a)^k$. Thus $y^a \mapsto (y^a)^k = y$;
      that is, the map is surjective. \qed
%%%%%%%%%%%%%%%%%%%%%%%%%%%%%%%%%%%%%2.3.26%%%%%%%%%%%%%%%%%%%%%%%%%%%%%%%%%%%%%
   \item[2.3.26]  Let $Z_n$ be a cyclic group of order $n$ and for each integer
                  $a$ let
                  $$\sigma_a : Z_n \mapsto Z_n \qquad by \qquad \sigma_a(x) =
                  x^a \quad \text{for all } x \in Z_n.$$
                  \begin{enumerate}
                     \item Prove that $\sigma_a$ is an automorphism of $Z_n$ if
                           and only if $a$ and $n$ are relatively prime(
                           automorphisms were introduced in Exercise 1.6.20).
                     \item Prove that $\sigma_a = \sigma_b$ if and only if
                           $a \equiv b$ (mod $n$).
                     \item Prove that \textit{every} automorphism of $Z_n$ is
                           equal to $\sigma_a$ for some integer $a$.
                     \item Prove that $\sigma_a\circ\sigma_b=\sigma_{ab}$.
                           Deduce that the map $\overline{a} \mapsto \sigma_a$
                           is an isomorphism of $(\Z/n\Z)^\times$ onto the
                           automorphism group of $Z_n$ (so Aut($Z_n$) is an
                           abelian group of order $\varphi(n)$).
                  \end{enumerate}

      \textbf{Proof.} Let $Z_n = \cyc{y}$. 

      \begin{enumerate}
         \item Suppose first that $(a, n) = 1$. By Proposition 5, Pg 57, it 
               follows that $|y^a| = n$, so that $y^a$ generates $Z_n$. That is, 
               $\cyc{y} =\cyc{y^a}$. Now it follows by definition that
               $\sigma_a(y^k) = (y^{a})^k$ for each integer $k$; thus
               $\sigma_a$ is an isomorphism by Theorem 4 (1), Pg 56. Now suppose
               conversely that $\sigma_a$ is an automorphism. Let $z \in Z_n$. 
               Since $\sigma_a$ is surjective, it follows that
               $\sigma_a(y^i) = z$ for some integer $i$. Thus
               $z = \sigma_a(y^i) = y^{ia} = ({y^a})^i \in \cyc{y^a}$. That is,
               every element of $Z_n$ is also in $\cyc{y^a}$, so that
               $Z_n = \cyc{y^a}$ and, thus, $|y^a| = n$. It follows that
               $(a, n) = 1$ by Proposition 5 (2), Pg 57.
         \item Suppose that $\sigma_a = \sigma_b$. Then it follows that
               $\sigma_a(y) = \sigma_b(y)$. Thus $y^a = y^b$, so that
               $y^{a-b} = 1$. Conclude by Proposition 3, Pg 55, that
               $n \mid a - b$, so that $a \equiv b$ (mod $n$). Conversely
               suppose that $a = b + tn$ for some integer $t$. Let $x \in Z_n$.
               Then we have that
               $\sigma_a(x) = x^a = x^{b+tn} = x^b = \sigma_b(x)$; i.e,
               $\sigma_a = \sigma_b$.
         \item Let $\alpha$ be an automorphism of $Z_n$. Let $z \in Z_n$. Then
               $z = y^i$ for some integer $i$. By Exercise 1.6.1, it follows
               that $\alpha(y^i) = \alpha(y)^i$. Since $\alpha(y) \in Z_n$, it
               follows that $\alpha(y) = y^j$ for some integer $j$. Thus
               $$\alpha(z) = \alpha(y^i) = \alpha(y)^i = (y^{j})^i = (y^{i})^j
                           = z^j.$$
               Thus $\alpha = \sigma_j$, as desired. \qed
         \item Let $x \in Z_n$. Then we have that
               $$(\sigma_a\circ\sigma_b)(x) = \sigma_a(\sigma_b(x)) =
                  \sigma_a(x^b) = (x^b)^a = x^{ab} = \sigma_{ab}(x).$$
               Thus $\sigma_a \circ \sigma_b = \sigma_{ab}$. Now consider the
               map $\phi : (\Z/n\Z)^\times \rightarrow \text{Aut}(Z_n)$ defined 
               by $c \mapsto \sigma_c$. According to Exercise 2.3.26, we notice
               that $\phi$ is well defined. It follows immediately that $\phi$
               is a homomorphism because 
               $$\phi(cd) = \sigma_{cd} = \sigma_c\circ\sigma_d =
                 \phi(c)\circ\phi(d).$$
               The map $c \mapsto \sigma_{c^{-1}}$ is a two-sided inverse of
               $\phi$. Thus $\phi$ is bijective and we conclude that it is an
               isomorphism.
      \end{enumerate} \qed
\end{enumerate}

































