\begin{enumerate}
%%%%%%%%%%%%%%%%%%%%%%%%%%%%%%%%%%%%%2.3.1%%%%%%%%%%%%%%%%%%%%%%%%%%%%%%%%%%%%%%
   \item[2.3.1]   Find all subgroups of $Z_{45} = \cyc{x}$, giving a generator
                  for each. Describe the containments between these subgroups.
%%%%%%%%%%%%%%%%%%%%%%%%%%%%%%%%%%%%%2.3.2%%%%%%%%%%%%%%%%%%%%%%%%%%%%%%%%%%%%%%
   \item[2.3.2]   If $x$ is an element of the finite group $G$ and $|x| = |G|$,
                  prove that $G = \cyc{x}$. Give an explicit example to show 
                  that this result need not be true if $G$ is an infinite group.
%%%%%%%%%%%%%%%%%%%%%%%%%%%%%%%%%%%%%2.3.3%%%%%%%%%%%%%%%%%%%%%%%%%%%%%%%%%%%%%%
   \item[2.3.3]   Find all generators for $\Z/48\Z$.
%%%%%%%%%%%%%%%%%%%%%%%%%%%%%%%%%%%%%2.3.4%%%%%%%%%%%%%%%%%%%%%%%%%%%%%%%%%%%%%%
   \item[2.3.4]   Find all generators for $\Z/202\Z$.
%%%%%%%%%%%%%%%%%%%%%%%%%%%%%%%%%%%%%2.3.5%%%%%%%%%%%%%%%%%%%%%%%%%%%%%%%%%%%%%%
   \item[2.3.5]   Find the number of generators for $\Z/49000\Z$.
%%%%%%%%%%%%%%%%%%%%%%%%%%%%%%%%%%%%%2.3.6%%%%%%%%%%%%%%%%%%%%%%%%%%%%%%%%%%%%%%
   \item[2.3.6]   In $\Z/48\Z$ write out all elements of $\cyc{\overline{a}}$
                  for every $\overline{a}$. Find all inclusions between
                  subgroups in $\Z/48\Z$.
%%%%%%%%%%%%%%%%%%%%%%%%%%%%%%%%%%%%%2.3.7%%%%%%%%%%%%%%%%%%%%%%%%%%%%%%%%%%%%%%
   \item[2.3.7]   Let $Z_{48} = \cyc{x}$ and use the isomorphism
                  $\Z/48\Z \cong Z_{48}$ given by $\overline{1} \mapsto x$ to
                  list all subgroups of $Z_{48}$ as computed in the preceding
                  exercise.
%%%%%%%%%%%%%%%%%%%%%%%%%%%%%%%%%%%%%2.3.8%%%%%%%%%%%%%%%%%%%%%%%%%%%%%%%%%%%%%%
   \item[2.3.8]   Let $Z_{48} = \cyc{x}$. For which integers $a$ does the map
                  $\varphi_a$ defined by $\varphi_a : \overline{1} \mapsto x^a$
                  extend to an \textit{isomorphism} from $\Z/48\Z$ onto
                  $Z_{48}$.
%%%%%%%%%%%%%%%%%%%%%%%%%%%%%%%%%%%%%2.3.9%%%%%%%%%%%%%%%%%%%%%%%%%%%%%%%%%%%%%%
   \item[2.3.9]   Let $Z_{36} = \cyc{x}$. For which integers $a$ does the map
                  $\psi_a$ defined by $\psi_a : \overline{1} \mapsto x^a$ extend
                  to a \textit{well defined homomorphism} from $\Z/48\Z$ into
                  $Z_{36}$. Can $\psi_a$ ever be a surjective homomorphism?
%%%%%%%%%%%%%%%%%%%%%%%%%%%%%%%%%%%%%2.3.10%%%%%%%%%%%%%%%%%%%%%%%%%%%%%%%%%%%%%
   \item[2.3.10]  What is the order of $\overline{30}$ in $\Z/54\Z$? Write out
                  all the elements and their orders in $\cyc{\overline{30}}$.
%%%%%%%%%%%%%%%%%%%%%%%%%%%%%%%%%%%%%2.3.11%%%%%%%%%%%%%%%%%%%%%%%%%%%%%%%%%%%%%
   \item[2.3.11]  Find all cyclic subgroups of $D_8$. Find a proper subgroup of
                  $D_8$ which is not cyclic.
%%%%%%%%%%%%%%%%%%%%%%%%%%%%%%%%%%%%%2.3.12%%%%%%%%%%%%%%%%%%%%%%%%%%%%%%%%%%%%%
   \item[2.3.12]  Prove that the following groups are \textit{not} cyclic:
                  \begin{enumerate}
                     \item $Z_2 \times Z_2$
                     \item $Z_2 \times \Z$
                     \item $\Z \times \Z$.
                  \end{enumerate}
%%%%%%%%%%%%%%%%%%%%%%%%%%%%%%%%%%%%%2.3.13%%%%%%%%%%%%%%%%%%%%%%%%%%%%%%%%%%%%%
   \item[2.3.13]  Prove that the following pairs of groups are \textit{not}
                  isomorphic:
                  \begin{enumerate}
                     \item $\Z \times Z_2$ and $\Z$
                     \item $\Q \times Z_2$ and $\Q$.
                  \end{enumerate}
%%%%%%%%%%%%%%%%%%%%%%%%%%%%%%%%%%%%%2.3.14%%%%%%%%%%%%%%%%%%%%%%%%%%%%%%%%%%%%%
   \item[2.3.14]  Let $\sigma =$ (1 2 3 4 5 6 7 8 9 10 11 12). For each of the
                  following integers $a$ compute $\sigma^a$:
                  $$a = 13, 65, 626, 1195, -6, -81, -570,\text{ and } {-1211}.$$
%%%%%%%%%%%%%%%%%%%%%%%%%%%%%%%%%%%%%2.3.15%%%%%%%%%%%%%%%%%%%%%%%%%%%%%%%%%%%%%
   \item[2.3.15]  Prove that $\Q \times \Q$ is not cyclic.
%%%%%%%%%%%%%%%%%%%%%%%%%%%%%%%%%%%%%2.3.16%%%%%%%%%%%%%%%%%%%%%%%%%%%%%%%%%%%%%
   \item[2.3.16]  Assume $|x| = n$ and $|y| = m$. Suppose that $x$ and $y$
                  \textit{commute}: $xy = yx$. Prove that $|xy|$ divides the
                  least common multiple of $m$ and $n$. Need this be true if $x$
                  and $y$ do \textit{not} commute? Give an example of commuting
                  elements $x$, $y$ such that the order of $xy$ is not equal to
                  the least common multiple of $|x|$ and $|y|$.
%%%%%%%%%%%%%%%%%%%%%%%%%%%%%%%%%%%%%2.3.17%%%%%%%%%%%%%%%%%%%%%%%%%%%%%%%%%%%%%
   \item[2.3.17]  Find a presentation for $Z_n$ with one generator.
%%%%%%%%%%%%%%%%%%%%%%%%%%%%%%%%%%%%%2.3.18%%%%%%%%%%%%%%%%%%%%%%%%%%%%%%%%%%%%%
   \item[2.3.18]  Show that if $H$ is any group and $h$ is an element of $H$
                  with $h^n = 1$, then there is a unique homomorphism from
                  $Z_n = \cyc{x}$ to $H$ such that $x \mapsto h$.
%%%%%%%%%%%%%%%%%%%%%%%%%%%%%%%%%%%%%2.3.19%%%%%%%%%%%%%%%%%%%%%%%%%%%%%%%%%%%%%
   \item[2.3.19]  Show that if $H$ is any group and $h$ is an element of $H$,
                  then there is a unique homomorphism from $\Z$ to $H$ such that
                  $1 \mapsto h$.
%%%%%%%%%%%%%%%%%%%%%%%%%%%%%%%%%%%%%2.3.20%%%%%%%%%%%%%%%%%%%%%%%%%%%%%%%%%%%%%
   \item[2.3.20]  Let $p$ be a prime and let $n$ be a positive integer. Show
                  that if $x$ is an element of the group $G$ such that
                  $x^{p^n} = 1$ then $|x| = p^m$ for some $m \le n$.
%%%%%%%%%%%%%%%%%%%%%%%%%%%%%%%%%%%%%2.3.21%%%%%%%%%%%%%%%%%%%%%%%%%%%%%%%%%%%%%
   \item[2.3.21]  Let $p$ be an odd prime and let $n$ be a positive integer. Use
                  the Binomial Theorem to show that
                  $(1+p)^{p^{n-1}} \equiv 1$ (mod $p^n$) but
                  $(1+p)^{p^{n-2}} \not\equiv 1$ (mod $p^n$). Deduce that $1+p$
                  is an element of order $p^{n-1}$ in the multiplicative group
                  $(\Z/p^n\Z)^\times$.
%%%%%%%%%%%%%%%%%%%%%%%%%%%%%%%%%%%%%2.3.22%%%%%%%%%%%%%%%%%%%%%%%%%%%%%%%%%%%%%
   \item[2.3.22]  Let $n$ be an integer $\ge 3$. Use the Binomial Theorem to
                  show that $(1+2^2)^{2^{n-2}} \equiv 1$ (mod $2^n$) but
                  $(1+2^2)^{2^{n-3}} \not\equiv 1$ (mod $2^n$). Deduce that 5 is
                  an element of order $2^{n-2}$ in the multiplicative group
                  $(\Z/2^n\Z)^\times$.
%%%%%%%%%%%%%%%%%%%%%%%%%%%%%%%%%%%%%2.3.23%%%%%%%%%%%%%%%%%%%%%%%%%%%%%%%%%%%%%
   \item[2.3.23]  Show that $(\Z/2^n\Z)^\times$ is not cyclic for any $n \ge 3$.
                  [Find two distinct subgroups of order 2.]
%%%%%%%%%%%%%%%%%%%%%%%%%%%%%%%%%%%%%2.3.24%%%%%%%%%%%%%%%%%%%%%%%%%%%%%%%%%%%%%
   \item[2.3.24]  Let $G$ be a finite group and let $x \in G$.
                  \begin{enumerate}
                     \item Prove that if $g \in N_G(\cyc{x})$ then
                           $gxg^{-1} = x^a$ for some $a \in \Z$. 
                     \item Prove conversely that if $gxg^{-1} = x^a$ for some
                           $a \in \Z$ then $g \in N_G(\cyc{x})$. [Show first
                           that $gx^kg^{-1} = (gxg^{-1})^k = x^{ak}$ for any
                           integer $k$, so that $g\cyc{x}g^{-1} \le \cyc{x}$.
                           If $x$ has order $n$, show the elements $gx^ig^{-1}$,
                           $i = 0, 1, \ldots, n-1$ are distinct, so that
                           $|g\cyc{x}g^{-1}| = |\cyc{x}| = n$ and conclude that
                           $g\cyc{x}g^{-1} = \cyc{x}$.]
                  \end{enumerate}
                  Note that this cuts down some of the work in computing
                  normalizers of cyclic subgroups since one does not have to
                  check $ghg^{-1} \in \cyc{x}$ for every $h \in \cyc{x}$.
%%%%%%%%%%%%%%%%%%%%%%%%%%%%%%%%%%%%%2.3.25%%%%%%%%%%%%%%%%%%%%%%%%%%%%%%%%%%%%%
   \item[2.3.25]  Let $G$ be a cyclic group of order $n$ and let $k$ be an
                  integer relatively prime to $n$. Prove that the map
                  $x \mapsto x^k$ is surjective. Use Lagrange's Theorem
                  (Exercise 1.7.19) to prove the same is true for any finite
                  group of order $n$. (For such $k$ each element has a
                  $k^{\text{th}}$ root in $G$. It follows from Cauchy's Theorem
                  in Section 3.2 that if $k$ is not relatively prime to the
                  order of $G$ then the map $x \mapsto x^k$ is not surjective.)
%%%%%%%%%%%%%%%%%%%%%%%%%%%%%%%%%%%%%2.3.26%%%%%%%%%%%%%%%%%%%%%%%%%%%%%%%%%%%%%
   \item[2.3.26]  Let $Z_n$ be a cyclic group of order $n$ and for each integer
                  $a$ let
                  $$\sigma_a : Z_n \mapsto Z_n \qquad by \qquad \sigma_a(x) =
                  x^a \quad \text{for all } x \in Z_n.$$
                  \begin{enumerate}
                     \item Prove that $\sigma_a$ is an automorphism of $Z_n$ if
                           and only if $a$ and $n$ are relatively prime(
                           automorphisms were introduced in Exercise 1.6.20).
                     \item Prove that $\sigma_a = \sigma_b$ if and only if
                           $a \equiv b$ (mod $n$).
                     \item Prove that \textit{every} automorphism of $Z_n$ is
                           equal to $\sigma_a$ for some integer $a$.
                     \item Prove that $\sigma_a\circ\sigma_b=\sigma_{ab}$.
                           Deduce that the map $\overline{a} \mapsto \sigma_a$
                           is an isomorphism of $(\Z/n\Z)^\times$ onto the
                           automorphism group of $Z_n$ (so Aut($Z_n$) is an
                           abelian group of order $\varphi(n)$).
                  \end{enumerate}
\end{enumerate}

































