\begin{enumerate}
%%%%%%%%%%%%%%%%%%%%%%%%%%%%%%%%%%%%%2.2.1%%%%%%%%%%%%%%%%%%%%%%%%%%%%%%%%%%%%%%
   \item[2.2.1]   Prove that
                  $C_G(A) = \{g \in G : g^{-1}ag = a \text{ for all } a\in A\}$.

      \textbf{Proof.} We have that
      \begin{align*}
         g \in C_G(A) &\Longleftrightarrow gag^{-1}=a \text{ for all } a\in A \\
             &\Longleftrightarrow ga = ag \text{ for all } a\in A \\
             &\Longleftrightarrow ag = ga \text{ for all } a\in A \\
             &\Longleftrightarrow g^{-1}ag = a \text{ for all } a\in A \\
             &\Longleftrightarrow g \in \{h \in G : h^{-1}ah = a
              \text{ for all } a\in A\},
      \end{align*}
      so that $C_G(A) = \{g \in G : g^{-1}ag = a\text{ for all } a\in A\}$. \qed
%%%%%%%%%%%%%%%%%%%%%%%%%%%%%%%%%%%%%2.2.2%%%%%%%%%%%%%%%%%%%%%%%%%%%%%%%%%%%%%%
   \item[2.2.2]   Prove that $C_G(Z(G)) = G$ and deduce that $N_G(Z(G)) = G$.

      \textbf{Proof.} Clearly $C_G(Z(G)) \subseteq G$, so it suffices to show
      that containment in the other direction. Now let $g \in G$ and
      $z \in Z(G)$. Since $z$ is in the center of $G$, it follows that
      $zg = gz$, so that $g$ commutes with every element of $Z(G)$; that is,
      $g \in C_G(Z(G))$ and we can conclude that $G \subseteq C_G(Z(G))$. Since
      $$G = C_G(Z(G)) \subseteq N_G(Z(G)) \subseteq G,$$
      it follows that $N_G(Z(G)) = G$. \qed
%%%%%%%%%%%%%%%%%%%%%%%%%%%%%%%%%%%%%2.2.3%%%%%%%%%%%%%%%%%%%%%%%%%%%%%%%%%%%%%%
   \item[2.2.3]   Prove that if $A$ and $B$ are subsets of $G$ with
                  $A \subseteq B$ then $C_G(B)$ is a subgroup of $C_G(A)$.

      \textbf{Proof.} Suppose that $A \subseteq B \subseteq G$, with
      $A \neq \emptyset$. By the discussion on Page 49 of the textbook, we know
      that $C_G(B) \le G$ and $C_G(A) \le G$. Since $A \subseteq B$, it follows
      that every element $g \in G$ that commutes with all elements in $B$ must
      necessarily commute with all elements in $A$; thus
      $C_G(B) \subseteq C_G(A)$, so that $C_G(B) \le C_G(A)$. \qed
%%%%%%%%%%%%%%%%%%%%%%%%%%%%%%%%%%%%%2.2.4%%%%%%%%%%%%%%%%%%%%%%%%%%%%%%%%%%%%%%
   \item[2.2.4]   For each of $S_3$, $D_8$, and $Q_8$ compute the centralizers
                  of each element and find the center of each group. Does
                  Lagrange's Theorem simplify your work?

      \textbf{Solution.}
      \begin{center}
         $S_3$

         \begin{tabular}{@{}|c|c|@{}} \hline
            Element & Centralizer \\ \hline
            (1) & $S_3$ \\ \hline
            (1 2) & \{(1), (1 2)\} \\ \hline
            (1 3) & \{(1), (1 3)\} \\ \hline
            (2 3) & \{(1), (2 3)\} \\ \hline
            (1 2 3) & \{(1), (1 2 3), (1 3 2)\} \\ \hline
            (1 3 2) & \{(1), (1 2 3), (1 3 2)\} \\ \hline
         \end{tabular}
      \end{center}

      \begin{center}
         $D_8$

         \begin{tabular}{@{}|c|c|@{}} \hline
            Element & Centralizer \\ \hline
            1 & $D_8$ \\ \hline
            $r$ & $\{1, r, r^2, r^3\}$ \\ \hline
            $r^2$ & $D_8$ \\ \hline
            $r^3$ & $\{1, r, r^2, r^3\}$ \\ \hline
            $s$ & $\{1, s, r^2, sr^2\}$ \\ \hline
            $sr$ & $\{1, sr, r^2, sr^3\}$ \\ \hline
            $sr^2$ & $\{1, s, r^2, sr^2\}$ \\ \hline
            $sr^3$ & $\{1, sr, r^2, sr^3\}$ \\ \hline
         \end{tabular}
      \end{center}

      \begin{center}
         $Q_8$

         \begin{tabular}{@{}|c|c|@{}} \hline
            Element & Centralizer \\ \hline
            $\pm1$ & $Q_8$ \\ \hline
            $\pm i$ & $\{\pm1, \pm i\}$ \\ \hline
            $\pm j$ & $\{\pm1, \pm j\}$ \\ \hline
            $\pm k$ & $\{\pm1, \pm k\}$ \\ \hline
         \end{tabular}
      \end{center}

      Now $Z(S_3) = \{(1)\}$, $Z(D_8) = \{1, r^2\}$, and $Z(Q_8) = \{\pm1\}$.
%%%%%%%%%%%%%%%%%%%%%%%%%%%%%%%%%%%%%2.2.5%%%%%%%%%%%%%%%%%%%%%%%%%%%%%%%%%%%%%%
   \item[2.2.5]   In each parts (a) to (c) show that for the specified group $G$
                  and subgroup $A$ of $G$, $C_G(A) = A$ and $N_G(A) = G$.
                  \begin{enumerate}
                     \item $G = S_3$ and $A = $ \{1, (1 2 3), (1 3 2)\}.
                     \item $G = D_8$ and $A = \{1, s, r^2, sr^2\}$.
                     \item $G = D_{10}$ and $A = \{1, r, r^2, r^3, r^4\}$.
                  \end{enumerate}
      
      \textbf{Solution.}
      
      \begin{enumerate}
         \item The set $A$ is a cyclic subgroup of $G$ since $A$ is generated by
               (1 2 3). Thus $A \le C_G(A)$, so that $|C_G(A)| \ge 3$. By
               Lagrange's Theorem, it follows that $|C_G(A)| = 3$ or 6. Since
               (1 3) does not commute with (1 2 3), it follows that
               $|C_G(A)| = 3$, so that $A = C_G(A)$. We know that
               $A \le N_G(A)$, so we have to check if the other elements of $G$
               normalize $A$. Thus
               \begin{align*}
                  (1\;2)A(1\;2)^{-1} &= (1\;2)A(1\;2) \\
                     &= \{(1\;2)(1)(1\;2), (1\;2)(1\;2\;3)(1\;2),
                          (1\;2)(1\;3\;2)(1\;2)\} \\
                     &= \{(1), (1\;3\;2), (1\;2\;3)\} \\
                     &= A.
               \end{align*}
               
               This says that (1 2) $\in N_G(A)$ so that $|N_G(A)| \ge 4$. But
               $|N_G(A)| = 3$ or 6 by Lagrange's Theorem; thus $N_G(A) = G$.
         \item Using Exercise 2.2.4, we have that
               $C_G(A) = C_G(1) \cap C_G(s) \cap C_G(r^2) \cap C_G(sr^2) = A$.
               Since $C_G(A) \le N_G(A)$, we have that $|N_G(A)| \ge 4$. Now
               $|N_G(A)| = 4$ or 8 according to Lagrange's Theorem. Since $r$
               does not commute with $s$, it follows that $r \notin N_G(A)$.
               That is $|N_G(A)| = 4$, so that $N_G(A) = G$.
         \item The set $A$ is the cyclic subgroup of rotations of $G$, so
               $|C_G(A)| \ge 5$. Since $r$ does not commute with $s$, it follows
               that $r \notin C_G(A)$; thus $|C_G(A)| = 5$ by Lagrange's
               Theorem. Note that $N_G(A) \ge 5$ since $C_G(A) \le N_G(A)$; so
               since
               \begin{align*}
                  sA^s{-1} &= sAs \\
                     &= \{s1s, srs, sr^2s, sr^3s, sr^4s\} \\
                     &= \{s^2, s^2r^{-1}, s^2r^{-2}, s^2r^{-3}, s^2r^{-4}\} \\
                     &= \{1, r^4, r^3, r^2, r^1\} \\
                     &= A,
               \end{align*}
               it follows that $s \in N_G(A)$, and we have that
               $|N_G(A)| \ge 6$. Thus $|N_G(A)| = 10$ by Lagrange's Theorem and
               we can conclude that $N_G(A) = G$.
      \end{enumerate}
%%%%%%%%%%%%%%%%%%%%%%%%%%%%%%%%%%%%%2.2.6%%%%%%%%%%%%%%%%%%%%%%%%%%%%%%%%%%%%%%
   \item[2.2.6]   Let $H$ be a subgroup of the group $G$.
                  \begin{enumerate}
                     \item Show that $H \le N_G(H)$. Give an example to show
                           that this is not necessarily true if $H$ is not a
                           subgroup.
                     \item Show that $H \le C_G(H)$ if and only if $H$ is
                           abelian.
                  \end{enumerate}

      \textbf{Proof.}

      \begin{enumerate}
         \item To show that $H$ is a subgroup of $N_G(H)$, it suffices to show
               that $H$ is a subset of $N_G(H)$. So let $h_1 \in H$. Clearly
               $h_1H{h_1}^{-1} \subseteq H$ because $H$ is closed under
               multiplication. Now let $h_2 \in H$, so that
               ${h_1}^{-1}h_2h_1 \in H$. Since
               $$h_1({h_1}^{-1}h_2h_1){h_1}^{-1} = h_2,$$
               it follows that $h_2 \in h_1H{h_1}^{-1}$; that is,
               $H \subseteq h_1H{h_1}^{-1}$, and thus, $H = h_1H{h_1}^{-1}$. We
               have thus shown that $h_1 \in N_G(H)$. Thus $H \le N_G(H)$. \\

               \textbf{Example.} Let $G = D_6$ and $H = \{r, s\}$. Note that
               $H$ is not a subgroup of $G$ and $H$ is not even a subset of
               $N_G(H) = \{1\}$.
         \item ($\Rightarrow$) Suppose that $H \le C_G(H)$. Let $h_1$,
               $h_2 \in H$. By our supposition, it follows that
               $h_1 \in C_G(H)$, so that $h_1h_2 = h_2h_1$; since $h_1$ and
               $h_2$ were arbitrarily chosen, it follows that $H$ is abelian.

               ($\Leftarrow$) Suppose conversely that $H$ is abelian. That is,
               every element of $H$ commutes with all elements of $H$, so it
               follows immediately that $H \le C_G(H)$.
      \end{enumerate} \qed
%%%%%%%%%%%%%%%%%%%%%%%%%%%%%%%%%%%%%2.2.7%%%%%%%%%%%%%%%%%%%%%%%%%%%%%%%%%%%%%%
   \item[2.2.7]   Let $n \in \Z$ with $n \ge 3$. Prove the following:
                  \begin{enumerate}
                     \item $Z(D_{2n}) = 1$ if $n$ is odd
                     \item $Z(D_{2n}) = \{1, r^k\}$ if $n = 2k$.
                  \end{enumerate}

      \textbf{Proof.}

      \begin{enumerate}
         \item Suppose that $n$ is odd. We claim that $r^{-i} = r^{i}$ if and
               only if $r^i = 1$. So suppose first that $r^{-i} = r^{i}$; i.e.,
               $r^{2i} = 1$. Thus it follows by Proposition 2.3 that
               $n \mid 2i$. Since $(n, 2) = 1$, we must have that $n \mid i$, so
               that $r^i = 1$. The other direction is trivial. Now let $r^j$ be
               a nonidentity element in $D_{2n}$ (so that $1 \le j < n$). Since 
               $r^j \neq r^{-j}$, it follows that
               $$sr^j \neq sr^{-j} = r^js$$
               and
               $$(sr^j)s = r^{-j} \neq r^j = s(sr^j).$$
               That is $r^j$ and $sr^j$ are not elements of $Z(D_{2n})$. Thus
               $D_{2n} = \{1\}$.
         \item Suppose that $n$ is even, so that $n = 2k$ for some positive
               integer $k$. We claim that $r^{-i} = r^{i}$ if and only if
               $k \mid i$. So suppose first that $r^{-i} = r^{i}$; i.e.,
               $r^{2i} = 1$. Thus it follows by Proposition 2.3 that
               $2k \mid 2i$ so that $k \mid i$. Conversely suppose that
               $k \mid i$. Then we have that $i = yk$. Now we have
               $r^{-i} = r^{-yk} = (r^{-k})^y = (r^k)^{y} = r^i$. Let $r^j$ be 
               an element in $D_{2n}$ such that $k \nmid j$ . That is, 
               $r^j \neq r^{-j}$, so that
               $$sr^j \neq sr^{-j} = r^js$$
               and
               $$(sr^j)s = r^{-j} \neq r^j = s(sr^j).$$
               That is $r^j$ and $sr^j$ are not elements of $Z(D_{2n})$.
               Finally we want to show that $r^k \in Z(D_{2n})$. So let
               $s^pr^q \in D_{2n}$. It follows that
               $$r^k(s^pr^q) = s^pr^{-k}r^q = s^pr^kr^q = (s^pr^q)r^k.$$
               Thus $D_{2n} = \{1, r^k\}$.
      \end{enumerate} \qed
%%%%%%%%%%%%%%%%%%%%%%%%%%%%%%%%%%%%%2.2.8%%%%%%%%%%%%%%%%%%%%%%%%%%%%%%%%%%%%%%
   \item[2.2.8]   Let $G = S_n$, fix an $i \in \{1, 2, \ldots, n\}$ and let
                  $G_i = \{\sigma \in G : \sigma(i) = i\}$ (the stabilizer of
                  $i$ in $G$). Use group actions to prove that $G_i$ is a
                  subgroup of $G$. Find $|G_i|$.

      \textbf{Proof.} Let $I_n = \{1, 2, \ldots, n\}$. Now consider the map 
      $f: G \times I_n \rightarrow I_n$, $(g, a) \mapsto g(a)$. We want to first
      show that $f$ is a group action of $G$ on $I_n$. Let $\sigma_1$,
      $\sigma_2 \in G$ and $a \in I_n$. So it follows that $f(1, a) = 1(a) = a$
      and
      \begin{align*}
         f(\sigma_1, f(\sigma_2, a)) &= f(\sigma_1, \sigma_2(a)) \\
            &= \sigma_1(\sigma_2(a)) \\
            &= (\sigma_1 \circ \sigma_2)(a) \\
            &= f(\sigma_1 \circ \sigma_2, a),
      \end{align*}
      so that $f$ is a group action. Now $G_i$ is the stabilizer of $i$ in $G$;
      thus $|G_i| \le G$ by Exercise 1.7.4(b). By counting we have that
      $|G_i| = (n - 1)!$ \qed
%%%%%%%%%%%%%%%%%%%%%%%%%%%%%%%%%%%%%2.2.9%%%%%%%%%%%%%%%%%%%%%%%%%%%%%%%%%%%%%%
   \item[2.2.9]   For any subgroup $H$ of $G$ and any nonempty subset $A$ of $G$
                  define $N_H(A)$ to be the set $\{h \in H : hAh^{-1} = A\}$.
                  Show that $N_H(A) = N_G(A) \cap H$ and deduce that $N_H(A)$ is
                  a subgroup of $H$ (note that $A$ need not be a subset of $H$).

      \textbf{Proof.} By definition we have that
      \begin{align*}
          h \in N_H(A) &\Longleftrightarrow hAh^{-1} = A \text{ and } h \in H \\
            &\Longleftrightarrow h \in N_G(A) \text{ and } h \in H\\
            &\Longleftrightarrow h \in N_G(A) \cap H.
      \end{align*}
      Hence $N_H(A) = N_G(A) \cap H$. Since $N_G(A)$ and $H$ are both subgroups
      of $G$, it follows by Exercise 2.1.10(a) that
      $N_G(A) \cap H = N_H(A) \le G$. \qed
%%%%%%%%%%%%%%%%%%%%%%%%%%%%%%%%%%%%%2.2.10%%%%%%%%%%%%%%%%%%%%%%%%%%%%%%%%%%%%%
   \item[2.2.10]  Let $H$ be a subgroup of order 2 in $G$. Show that
                  $N_G(H) = C_G(H)$. Deduce that if $N_G(H) = G$ then
                  $H \le Z(G)$.

      \textbf{Proof.} Since $|H| = 2$, there exists a nonidentity $h \in G$ such
      that $H = \{1, h\}$. From the discussion in the textbook, we know that
      $C_G(H) \subseteq N_G(H)$, so it suffices to show reverse containment. Now
      let $g \in N_G(H)$. Then it follows that
      $$H = gHg^{-1} = \{g1g^{-1}, ghg^{-1}\} = \{1, ghg^{-1}\}.$$
      That is, $ghg^{-1} = h$ so that $gh = hg$. Thus $g$ commutes with both 1
      and $h$ so that $g \in C_G(H)$(so that $N_G(H) \subseteq C_G(H)$ and we 
      have that $N_G(H) = C_G(H)$. Now suppose that $N_G(H) = G$. Then, as we
      have just shown, we must have that $C_G(H) = G$. That is, every element of
      $G$ commutes with every element of $H$; in order words, every element of
      $H$ commutes with every element of $G$, so that $H \le Z(G)$. \qed
%%%%%%%%%%%%%%%%%%%%%%%%%%%%%%%%%%%%%2.2.11%%%%%%%%%%%%%%%%%%%%%%%%%%%%%%%%%%%%%
   \item[2.2.11]  Prove that $Z(G) \le N_G(A)$ for any subset $A$ of $G$.

      \textbf{Proof.} Let $A$ be a nonempty subset of a group $G$. We know that
      $C_G(A) \le N_G(A)$. Now if $g \in Z(G)$, then $g$ commutes with all the
      elements of $G$; particularly, $g$ must commute with all the elements of
      $A$, so that $g \in C_G(A)$; thus $Z(G) \subseteq C_G(A)$. Since $Z(G)$ is
      a subgroup of $G$, it follows that $Z(G) \le C_G(A) \le N_G(A)$. \qed
%%%%%%%%%%%%%%%%%%%%%%%%%%%%%%%%%%%%%2.2.12%%%%%%%%%%%%%%%%%%%%%%%%%%%%%%%%%%%%%
   \item[2.2.12]  Let $R$ be the set of all polynomials with integer
                  coefficients in the independent variables $x_1$, $x_2$, $x_3$,
                  $x_4$, i.e., the members of $R$ are finite sums of elements of
                  the form ${ax_1}^{r_1}{x_2}^{r_2}{x_3}^{r_3}{x_4}^{r_4}$,
                  where $a$ is any integer $r_1, \ldots, r_4$ are nonnegative
                  integers. For example,

                  \begin{equation} \label{2_2_12_1}
                     12{x_1}^5{x_2}^7x_4 - 18{x_2}^3x_3 +
                        11{x_1}^6x_2{x_3}^3{x_4}^{23}
                  \end{equation}

                  is a typical element of $R$. Each $\sigma \in S_4$ gives a
                  permutation of $\{x_1, \ldots, x_4\}$ by defining
                  $\sigma \cdot x_i = x_{\sigma(i)}$. This may be extended to a
                  map from $R$ to $R$ by defining
                  $$\sigma \cdot p(x_1, x_2, x_3, x_4) = p(x_{\sigma(1)},
                    x_{\sigma(2)}, x_{\sigma(3)}, x_{\sigma(4)})$$
                  for all $p(x_1, x_2, x_3, x_4) \in R$ (i.e., $\sigma$ simply
                  permutes the indices of the variables). For example, if
                  $\sigma =$(1 2)(3 4) and $p(x_1, \ldots, x_4)$ is the
                  polynomial in \eqref{2_2_12_1} above, then

                  \begin{align*}
                     \sigma \cdot p(x_1, x_2, x_3, x_4) &= 12{x_2}^5{x_1}^7x_3 -
                        18{x_1}^3x_4 + 11{x_2}^6x_1{x_4}^3{x_3}^{23} \\
                        &= 12{x_1}^7{x_2}^5x_3 -
                        18{x_1}^3x_4 + 11x_1{x_2}^6{x_3}^{23}{x_4}^3.
                  \end{align*}

                  \begin{enumerate}
                     \item Let $p = p(x_1, \ldots, x_4)$ be the polynomial in
                           \eqref{2_2_12_1} above, let $\sigma =$(1 2 3 4) and
                           let $\tau = $(1 2 3). Compute $\sigma \cdot p$,
                           $\tau \cdot (\sigma \cdot p)$,
                           $(\tau \circ \sigma) \cdot p$, and
                           $(\sigma \circ \tau) \cdot p$.
                     \item Prove that these definitions give a (left) group
                           action of $S_4$ on $R$.
                     \item Exhibit all permutations in $S_4$ that stabilize
                           $x_4$ and prove that they form a subgroup isomorphic
                           to $S_3$.
                     \item Exhibit all permutations in $S_4$ that stabilize the
                           element $x_1 + x_2$ and prove that they form an
                           abelian subgroup of order 4.
                     \item Exhibit all permutations in $S_4$ that stabilize the
                           element $x_1x_2 + x_3x_4$ and prove that they form a
                           subgroup isomorphic to the dihedral group of order 8.
                     \item Show that the permutations in $S_4$ that stabilize
                           the element $(x_1 + x_2)(x_3 + x_4)$ are exactly the
                           same as those found in part (e). (The two polynomials
                           appearing in parts (e) and (f) and the subgroup that
                           stabilizes them will play an important role in the
                           study of roots of quartic equations in Section 14.6)
                  \end{enumerate}

      \textbf{Solution.}

      \begin{enumerate}
         \item We have
               \begin{align*}
                  \sigma \cdot p &= 12{x_2}^5{x_3}^7x_1-18{x_3}^3x_4+11{x_2}^6
                        x_3{x_4}^3{x_1}^{23} \\
                     &= 12x_1{x_2}^5{x_3}^7-18{x_3}^3x_4+11{x_1}^{23}{x_2}^6
                        x_3{x_4}^3 \\
                  \tau \cdot (\sigma \cdot p) &= \tau(12x_1{x_2}^5{x_3}^7-
                        18{x_3}^3x_4+11{x_1}^{23}{x_2}^6x_3{x_4}^3) \\
                     &= 12x_2{x_3}^5{x_1}^7-18{x_1}^3x_4+11{x_2}^{23}{x_3}^6
                        x_1{x_4}^3 \\
                     &= 12{x_1}^7x_2{x_3}^5-18{x_1}^3x_4+11x_1{x_2}^{23}{x_3}^6
                        {x_4}^3 \\
                  (\tau\circ\sigma) \cdot p &= (1\;3\;4\;2) \cdot p \\
                     &= 12{x_3}^5{x_1}^7x_2 - 18{x_1}^3x_4 +
                        11{x_3}^6x_1{x_4}^3{x_2}^{23} \\
                     &= 12{x_1}^7x_2{x_3}^5 - 18{x_1}^3x_4 +
                        11x_1{x_2}^{23}{x_3}^6{x_4}^3 \\
                  (\sigma\circ\tau) \cdot p &= (1\;3\;2\;4) \cdot p \\
                     &= 12{x_3}^5{x_4}^7x_1 - 18{x_4}^3x_2 +
                        11{x_3}^6x_4{x_2}^3{x_1}^{23} \\
                     &= 12x_1{x_3}^5{x_4}^7 - 18x_2{x_4}^3 +
                        11{x_1}^{23}{x_2}^3{x_3}^6x_4.
               \end{align*}

      \end{enumerate}
%%%%%%%%%%%%%%%%%%%%%%%%%%%%%%%%%%%%%2.2.13%%%%%%%%%%%%%%%%%%%%%%%%%%%%%%%%%%%%%
   \item[2.2.13]  Let $n$ be a positive integer and let $R$ be the set of all
                  polynomials with integer coefficients in the independent
                  variables $x_1, x_2, \ldots, x_n$, i.e., the members of $R$
                  are finite sums of elements of the form
                  $a{x_1}^{r_1}{x_2}^{r_2}\ldots{x_n}^{r_n}$, where $a$ is any
                  integer and $r_1, \ldots, r_n$ are nonnegative integers. For
                  each $\sigma \in S_n$ define a map
                  $$\sigma : R \rightarrow R \quad by \quad
                    \sigma \cdot p(x_1, x_2, \ldots, x_n) = p(x_{\sigma(1)},
                    x_{\sigma(2)}, \ldots, x_{\sigma(n)}).$$
                  Prove that this defines a (left) group action of $S_n$ on $R$.
%%%%%%%%%%%%%%%%%%%%%%%%%%%%%%%%%%%%%2.2.14%%%%%%%%%%%%%%%%%%%%%%%%%%%%%%%%%%%%%
   \item[2.2.14]  Let $H(F)$ be the Heisenberg group over the field $F$
                  introduced in Exercise 1.4.11. Determine which matrices lie in
                  the center of $H(F)$ and prove that $Z(H(F))$ is isomorphic to
                  the additive group $F$.
\end{enumerate}
