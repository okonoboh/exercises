\begin{enumerate}
%%%%%%%%%%%%%%%%%%%%%%%%%%%%%%%%%%%%%2.1.1%%%%%%%%%%%%%%%%%%%%%%%%%%%%%%%%%%%%%%
   \item[2.1.1]   In each of (a) - (e) prove that the specified subset is a
                  subgroup of the given group:
                  \begin{enumerate}
                     \item the set of complex numbers of the form $a + ai$,
                           $a \in \R$ (under addition)
                     \item the set of complex numbers of absolute value 1 (under
                           multiplication)
                     \item for fixed $n \in \Z^+$ the set of rational numbers(in
                           lowest terms) whose denominators divide $n$
                           (under addition)
                     \item for fixed $n \in \Z^+$ the set of rational numbers
                           whose denominators are relatively prime to $n$ (under
                           addition)
                     \item the set of nonzero real numbers whose square is a
                           rational number (under multiplication).
                  \end{enumerate}
                  
      \textbf{Proof.}
      
      \begin{enumerate}
         \item Let $S = \{a + ai: a \in \R, i^2 = -1\}$. The set $S$ is clearly
               nonempty, so let $x+xi, y+yi \in S$. We then have that
               $(x+xi)-(y+yi) = (x-y) + (x-y)i \in S$, so that $S \le \C$.
         \item Let $S = \{a \in \C: |a| = 1\}$. The set $S$ is clearly nonempty,
               so let $x, y \in S$. We then have that
               $|xy| = |x||y| = 1 \cdot 1 = 1$, so that $xy \in S$. Write
               $x = r + si$, so that $x^{-1} = r - si$. Since
               $|x| = r^2 + s^2 = |x^{-1}| = 1$, we have that $x^{-1} \in S$.
               Hence $S$ is closed under inverses and multiplication and we have
               that $S \le \C$.
         \item Let $n$ be a positive integer and \
               $S = \left\{\D\frac{a}{b} \in \Q: b \mid n\right\}$. The set $S$
               is nonempty since it contains 0. Let $\D\frac{a}{b}$,
               $\D\frac{c}{d} \in S$. Let $l = \mbox{lcm}(b, d)$. By membership
               in $S$, we have that $b \mid n$ and $d \mid n$. Thus $l \mid n$.
               So there exists an integer $m$ such that 
               $\D\frac{a}{b} - \frac{c}{d} = \frac{m}{l}$. Let
               $\D\frac{m'}{l'} = \frac{m}{l}$ such that $\gcd(m', l') = 1$.
               Then $l' \mid l$; since $l \mid n$, it follows that $l' \mid n$,
               so that
               $\D\frac{a}{b} - \frac{c}{d} = \frac{m}{l} =\frac{m'}{l'} \in S$.
               Thus $S$ is an additive group. \qed
      \end{enumerate}
%%%%%%%%%%%%%%%%%%%%%%%%%%%%%%%%%%%%%2.1.2%%%%%%%%%%%%%%%%%%%%%%%%%%%%%%%%%%%%%%
   \item[2.1.2]   In each of (a) - (e) prove that the specified subset is
                  \textit{not} a subgroup of the given group:
                  \begin{enumerate}
                     \item the set of 2-cycles in $S_n$ for $n \ge 3$.
                     \item the set of reflections in $D_{2n}$ for $n \ge 3$.
                     \item for $n$ a composite integer greater than 1 and $G$ a
                           group containing an element of order $n$, the set
                           $\{x \in G : |x| = n\} \cup \{1\}$
                     \item the set of (positive and negative) odd integers in
                           $\Z$ together with 0.
                     \item the set of real numbers whose square is a rational
                           number (under addition).
                  \end{enumerate}
%%%%%%%%%%%%%%%%%%%%%%%%%%%%%%%%%%%%%2.1.3%%%%%%%%%%%%%%%%%%%%%%%%%%%%%%%%%%%%%%
   \item[2.1.3]   Show that the following subsets of the dihedral group $D_8$
                  are actually subgroups:
                  \begin{enumerate}
                     \item $\{1, r^2, s, sr^2\}$,
                     \item $\{1, r^2, sr, sr^3\}$.
                  \end{enumerate}
%%%%%%%%%%%%%%%%%%%%%%%%%%%%%%%%%%%%%2.1.4%%%%%%%%%%%%%%%%%%%%%%%%%%%%%%%%%%%%%%
   \item[2.1.4]   Give an explicit example of a group $G$ and an infinite subset
                  $H$ of $G$ that is closed under the group operation but is not
                  a subgroup of $G$.
%%%%%%%%%%%%%%%%%%%%%%%%%%%%%%%%%%%%%2.1.5%%%%%%%%%%%%%%%%%%%%%%%%%%%%%%%%%%%%%%
   \item[2.1.5]   Prove that $G$ cannot have a subgroup $H$ with $|H| = n - 1$,
                  where $n = |G| > 2$.
%%%%%%%%%%%%%%%%%%%%%%%%%%%%%%%%%%%%%2.1.6%%%%%%%%%%%%%%%%%%%%%%%%%%%%%%%%%%%%%%
   \item[2.1.6]   Let $G$ be an abelian group. Prove that
                  $\{g \in G : |g| < \infty\}$ is a subgroup of $G$ (called the
                  \textit{torsion subgroup} of $G$). Give an explicit example
                  where this set is not a subgroup when $G$ is non-abelian.
%%%%%%%%%%%%%%%%%%%%%%%%%%%%%%%%%%%%%2.1.7%%%%%%%%%%%%%%%%%%%%%%%%%%%%%%%%%%%%%%
   \item[2.1.7]   Fix some $n \in \Z$ with $n > 1$. Find the torsion subgroup
                  of $\Z \times (\Z/n\Z)$. Show that the set of elements of
                  infinite order together with the identity is \textit{not} a
                  subgroup of this direct product.
%%%%%%%%%%%%%%%%%%%%%%%%%%%%%%%%%%%%%2.1.8%%%%%%%%%%%%%%%%%%%%%%%%%%%%%%%%%%%%%%
   \item[2.1.8]   Let $H$ and $K$ be subgroups of $G$. Prove that $H \cup K$ is
                  a subgroup if and only if either $H \subseteq K$ or
                  $K \subseteq H$.
%%%%%%%%%%%%%%%%%%%%%%%%%%%%%%%%%%%%%2.1.9%%%%%%%%%%%%%%%%%%%%%%%%%%%%%%%%%%%%%%
   \item[2.1.9]   Let $G = GL_n(F)$, where $F$ is any field. Define
                  $$SL_n(F) = \{A \in GL_n(F) : \det(A) = 1\}$$
                  (called the \textit{special linear group}). Prove that
                  $SL_n(F) \le GL_n(F)$,
%%%%%%%%%%%%%%%%%%%%%%%%%%%%%%%%%%%%%2.1.10%%%%%%%%%%%%%%%%%%%%%%%%%%%%%%%%%%%%%
   \item[2.1.10]  \begin{enumerate}
                     \item Prove that if $H$ and $K$ are subgroups of $G$ then
                           so is their intersection $H \cap K$.
                     \item Prove that the intersection of an arbitrary nonempty
                           collection of subgroups of $G$ is again a subgroup of
                           $G$ (do not assume the collection is countable).
                  \end{enumerate}
%%%%%%%%%%%%%%%%%%%%%%%%%%%%%%%%%%%%%2.1.11%%%%%%%%%%%%%%%%%%%%%%%%%%%%%%%%%%%%%
   \item[2.1.11]  Let $A$ and $B$ be groups. Prove that the following sets are
                  subgroups of the direct product $A \times B$.
                  \begin{enumerate}
                     \item $\{(a, 1) : a \in A\}$
                     \item $\{(1, b) : b \in B\}$
                     \item $\{(a, a) : a \in A\}$, where here we assume $B = A$
                           (called the \textit{diagonal subgroup}).
                  \end{enumerate}
%%%%%%%%%%%%%%%%%%%%%%%%%%%%%%%%%%%%%2.1.12%%%%%%%%%%%%%%%%%%%%%%%%%%%%%%%%%%%%%
   \item[2.1.12]  Let $A$ be an abelian group and fix some $n \in \Z$. Prove
                  that the following sets are subgroups of $A$:
                  \begin{enumerate}
                     \item $\{a^n : a \in A\}$
                     \item $\{a \in A : a^n = 1\}$.
                  \end{enumerate}
%%%%%%%%%%%%%%%%%%%%%%%%%%%%%%%%%%%%%2.1.13%%%%%%%%%%%%%%%%%%%%%%%%%%%%%%%%%%%%%
   \item[2.1.13]  Let $H$ be a subgroup of the additive group of rational
                  numbers with the property that $1/x \in H$ for every nonzero
                  element $x \in H$. Prove that $H = 0$ or $\Q$.
%%%%%%%%%%%%%%%%%%%%%%%%%%%%%%%%%%%%%2.1.14%%%%%%%%%%%%%%%%%%%%%%%%%%%%%%%%%%%%%
   \item[2.1.14]  Show that $\{x \in D_{2n} : x^2 = 1\}$ is not a subgroup of
                  $D_{2n}$ (here $n \ge 3$).
%%%%%%%%%%%%%%%%%%%%%%%%%%%%%%%%%%%%%2.1.15%%%%%%%%%%%%%%%%%%%%%%%%%%%%%%%%%%%%%
   \item[2.1.15]  Let $H_1 \le H_2 \le \cdots$ be an ascending chain of
                  subgroups of $G$. Prove that $\cup_{i=1}^\infty H_i$ is a 
                  subgroup of $G$.
%%%%%%%%%%%%%%%%%%%%%%%%%%%%%%%%%%%%%2.1.16%%%%%%%%%%%%%%%%%%%%%%%%%%%%%%%%%%%%%
   \item[2.1.16]  Let $n \in \Z^+$ and let $F$ be a field. Prove that the set
                  $$\{(a_{ij}) \in GL_n(F) : (a_{ij}) = 0 \text{ for all }
                     i > j\}$$
                  is a subgroup of $GL_n(F)$ (called the group of
                  \textit{upper triangular} matrices).
%%%%%%%%%%%%%%%%%%%%%%%%%%%%%%%%%%%%%2.1.17%%%%%%%%%%%%%%%%%%%%%%%%%%%%%%%%%%%%%
   \item[2.1.17]  Let $n \in \Z^+$ and let $F$ be a field. Prove that the set
                  $$\{(a_{ij}) \in GL_n(F) : (a_{ij}) = 0 \text{ for all }
                      i > j, \text{ and } a_{ii} = 1 \text{ for all }i\}$$
                  is a subgroup of $GL_n(F)$.
\end{enumerate}
