\begin{enumerate}
%%%%%%%%%%%%%%%%%%%%%%%%%%%%%%%%%%%%%2.1.1%%%%%%%%%%%%%%%%%%%%%%%%%%%%%%%%%%%%%%
   \item[2.1.1]   In each of (a) - (e) prove that the specified subset is a
                  subgroup of the given group:
                  \begin{enumerate}
                     \item the set of complex numbers of the form $a + ai$,
                           $a \in \R$ (under addition)
                     \item the set of complex numbers of absolute value 1 (under
                           multiplication)
                     \item for fixed $n \in \Z^+$ the set of rational numbers(in
                           lowest terms) whose denominators divide $n$
                           (under addition)
                     \item for fixed $n \in \Z^+$ the set of rational numbers
                           whose denominators are relatively prime to $n$ (under
                           addition)
                     \item the set of nonzero real numbers whose square is a
                           rational number (under multiplication).
                  \end{enumerate}
                  
      \textbf{Proof.}
      
      \begin{enumerate}
         \item Let $S = \{a + ai: a \in \R, i^2 = -1\}$. The set $S$ is clearly
               nonempty, so let $x+xi, y+yi \in S$. We then have that
               $(x+xi)-(y+yi) = (x-y) + (x-y)i \in S$, so that $S < \C$.
         \item Let $S = \{a \in \C: |a| = 1\}$. The set $S$ is clearly nonempty,
               so let $x, y \in S$. We then have that
               $|xy| = |x||y| = 1 \cdot 1 = 1$, so that $xy \in S$. Write
               $x = r + si$, so that $x^{-1} = r - si$. Since
               $|x| = r^2 + s^2 = |x^{-1}| = 1$, we have that $x^{-1} \in S$.
               Hence $S$ is closed under inverses and multiplication and we have
               that $S < \C$.
         \item Let $n$ be a positive integer and
               $$S = \left\{\frac{a}{b} \in
               \Q : \text{ if } \frac{a'}{b'} = \frac{a}{b},
               \text{ with }\gcd(a', b') = 1, \text{ then } b' \mid n\right\}.$$ 
               The set $S$ is nonempty since it contains 0. Let $\D\frac{a}{b}$,
               $\D\frac{c}{d} \in S$. Writing these rationals in lowest terms,
               we have $\D\frac{a}{b} = \frac{a'}{b'}$ and
               $\D\frac{c}{d} = \frac{c'}{d'}$. Let $l = \mbox{lcm}(b', d')$.
               By membership in $S$, we have that $b' \mid n$ and $d' \mid n$. 
               Thus $l \mid n$. Now there exists an integer $m$ such that 
               $\D\frac{a'}{b'} - \frac{c'}{d'} = \frac{m}{l}$. Let
               $\D\frac{m'}{l'} = \frac{m}{l}$ such that $\gcd(m', l') = 1$.
               Then $l' \mid l$; since $l \mid n$, it follows that $l' \mid n$,
               so that $\D\frac{a}{b} - \frac{c}{d} = \D\frac{a'}{b'} -
               \frac{c'}{d'} = \frac{m}{l} =\frac{m'}{l'} \in S$. Thus $S < \Q$.
         \item Let $n$ be a positive integer and
               $$S = \left\{\frac{a}{b} \in
               \Q : \text{ if } \frac{a'}{b'} = \frac{a}{b},
               \text{ with }\gcd(a',b')=1, \text{ then } \gcd(b',n)=1\right\}.$$ 
               The set $S$ is nonempty since it contains 0. Let $\D\frac{a}{b}$,
               $\D\frac{c}{d} \in S$. Writing these rationals in lowest terms,
               we have $\D\frac{a}{b} = \frac{a'}{b'}$ and
               $\D\frac{c}{d} = \frac{c'}{d'}$. Let $l = \mbox{lcm}(b', d')$.
               By membership in $S$, we have that $\gcd(b',n) = \gcd(d',n) = 1$. 
               Thus $\gcd(l,n)=1$. Now there exists an integer $m$ such that 
               $\D\frac{a'}{b'} - \frac{c'}{d'} = \frac{m}{l}$. Let
               $\D\frac{m'}{l'} = \frac{m}{l}$ such that $\gcd(m', l') = 1$.
               Then $l' \mid l$; since $\gcd(l,n)=1$, it follows that
               $\gcd(l',n)=1$, so that
               $$\frac{a}{b}-\frac{c}{d}=\D\frac{a'}{b'}-\frac{c'}{d'} =
                 \frac{m}{l} =\frac{m'}{l'} \in S.$$ Thus $S < \Q$.
         \item Let $S = \{x \in \R : x \neq 0, x^2 \in \Q\}$. The set $S$
               has 1 as its identity. Now let $a, b \in S$. Then we have that
               $(ab^{-1})^2 = \D\frac{a^2}{b^2} \in \Q$, so that $S < \R$.
      \end{enumerate} \qed
%%%%%%%%%%%%%%%%%%%%%%%%%%%%%%%%%%%%%2.1.2%%%%%%%%%%%%%%%%%%%%%%%%%%%%%%%%%%%%%%
   \item[2.1.2]   In each of (a) - (e) prove that the specified subset is
                  \textit{not} a subgroup of the given group:
                  \begin{enumerate}
                     \item the set of 2-cycles in $S_n$ for $n \ge 3$.
                     \item the set of reflections in $D_{2n}$ for $n \ge 3$.
                     \item for $n$ a composite integer greater than 1 and $G$ a
                           group containing an element of order $n$, the set
                           $\{x \in G : |x| = n\} \cup \{1\}$
                     \item the set of (positive and negative) odd integers in
                           $\Z$ together with 0.
                     \item the set of real numbers whose square is a rational
                           number (under addition).
                  \end{enumerate}

      \textbf{Proof.}

      \begin{enumerate}
         \item Let $n \ge 3$ and
               $S = \{\sigma \in S_n : \sigma \text{ is a 2-cycle}\}$. Now we
               have (1 2), (2, 3) $\in S$, but (1 2)(2 3) = (1 2 3) $\notin S$,
               so that $S$ is not closed under composition. Thus $S$ is not a
               subgroup of $S_n$.
         \item Let $n \ge 3$ and $S=\{\text{The set of reflections in }D_{2n}\}
               = \{s, sr, \ldots, sr^n\}$. Since $S$ does not contain the
               identity, it follows immediately that $S$ is not a subgroup of
               $D_{2n}$.
         \item Let $n = ab$ where $a$ and $b$ are greater than 1, $G$ a group 
               that has an element of order $n$ and
               $S = \{x \in G: |x| = n\} \cup \{1_G\}$. Let $y$ be a nonidentity
               element in $S$. Then $1 = y^n = y^{ab} = (y^a)^b$, so that
               $|y^a| \le b < n$. Now $y^a \neq 1$ since $1 < a < n$ and
               $|y| = n$. Since $|y^a| \le b$ and $y^a \neq 1$, it follows that
               $y^a \notin S$, so that $S$ is not closed under the operation of
               $G$; thus $S$ is not a subgroup of $G$.
         \item Let $S = \{2k + 1 : k \in \Z\} \cup \{0\}$. The set $S$ is not
               closed under addition because $7 \in S$, but
               $7 + 7 = 14 \notin S$, so that $S$ is not a subgroup of $\Z$.
         \item Let $S = \{x \in \R : x \neq 0, x^2 \in \Q\}$. The set $S$ is not
               closed under addition because $1, \sqrt{2} \in S$, but
               $(1 + \sqrt{2})^2 = 3 + 2\sqrt{2} \notin S$, so that $S$ is not a 
               subgroup of $\R$.
      \end{enumerate} \qed
%%%%%%%%%%%%%%%%%%%%%%%%%%%%%%%%%%%%%2.1.3%%%%%%%%%%%%%%%%%%%%%%%%%%%%%%%%%%%%%%
   \item[2.1.3]   Show that the following subsets of the dihedral group $D_8$
                  are actually subgroups:
                  \begin{enumerate}
                     \item $\{1, r^2, s, sr^2\}$,
                     \item $\{1, r^2, sr, sr^3\}$.
                  \end{enumerate}

      \textbf{Proof.} Since these two subsets are finite, we need only show that
      they are closed under multiplication. The group tables below do just that.

      \begin{enumerate}
         \item $$
                  \begin{tabular}{@{}|c|c|c|c|c|@{}} \hline
                       & 1 & $r^2$ & $s$ & $sr^2$ \\ \hline
                     1 & 1 & $r^2$ & $s$ & $sr^2$ \\ \hline
                     $r^2$ & $r^2$ & 1 & $sr^2$ & $s$ \\ \hline
                     $s$ & $s$ & $sr^2$ & $1$ & $r^2$ \\ \hline
                     $sr^2$ & $sr^2$ & $s$ & $r^2$ & 1 \\ \hline
                  \end{tabular}
               $$
         \item $$
                  \begin{tabular}{@{}|c|c|c|c|c|@{}} \hline
                       & 1 & $r^2$ & $sr$ & $sr^3$ \\ \hline
                     1 & 1 & $r^2$ & $sr$ & $sr^3$ \\ \hline
                     $r^2$ & $r^2$ & 1 & $sr^3$ & $sr$ \\ \hline
                     $sr$ & $sr$ & $sr^3$ & $1$ & $r^2$ \\ \hline
                     $sr^3$ & $sr^3$ & $sr$ & $r^2$ & 1 \\ \hline
                  \end{tabular}
               $$
      \end{enumerate} \qed
%%%%%%%%%%%%%%%%%%%%%%%%%%%%%%%%%%%%%2.1.4%%%%%%%%%%%%%%%%%%%%%%%%%%%%%%%%%%%%%%
   \item[2.1.4]   Give an explicit example of a group $G$ and an infinite subset
                  $H$ of $G$ that is closed under the group operation but is not
                  a subgroup of $G$.

      \textbf{Solution.} Let $G = \Z$ (under addition) and $H = \Z^+$. Clearly
      $H$ is a closed infinite subset of $\Z$, but it has no identity, so it is 
      not a subgroup of $G$.
%%%%%%%%%%%%%%%%%%%%%%%%%%%%%%%%%%%%%2.1.5%%%%%%%%%%%%%%%%%%%%%%%%%%%%%%%%%%%%%%
   \item[2.1.5]   Prove that $G$ cannot have a subgroup $H$ with $|H| = n - 1$,
                  where $n = |G| > 2$.

      \textbf{Proof.} Suppose that $G$ is a finite group of order $n > 2$. Now
      suppose to the contrary that $H < G$, where $|H| = n - 1$. Let $y \in G$
      be the element not in $H$. Since $|G| \ge 3$, there exists an
      $x \in H$ such that $x \neq 1$ (and $x \neq y$). Now consider the element
      $x^{-1}y$. If $x^{-1}y = y$, then $x = 1$, contrary to our assumption. If
      $x^{-1}y = 1$, then $x = y$, another contradiction. Thus $x^{-1}y$ is
      neither equal to $y$ nor 1. Particularly we have $x^{-1}y \in H$; thus
      $y = x(x^{-1}y) \in H$, a contradiction. Thus $H$ cannot have size
      $n - 1$. \qed

      \textbf{N.B.} Or use Lagrange's Theorem. See Exercise 1.7.19.
%%%%%%%%%%%%%%%%%%%%%%%%%%%%%%%%%%%%%2.1.6%%%%%%%%%%%%%%%%%%%%%%%%%%%%%%%%%%%%%%
   \item[2.1.6]   Let $G$ be an abelian group. Prove that
                  $\{g \in G : |g| < \infty\}$ is a subgroup of $G$ (called the
                  \textit{torsion subgroup} of $G$). Give an explicit example
                  where this set is not a subgroup when $G$ is non-abelian.

      \textbf{Proof.} Let $H = \{g \in G : |g| < \infty\}$. Clearly $1 \in H$.
      Now suppose that $x, y \in H$. Let $|x| = m$ and $|y| = n$. Then we have
      that $(xy^{-1})^{mn} = x^{mn}(y^{-1})^{mn} = (x^m)^n(y^n)^{-m} = 1$, so 
      that $|xy^{-1}| \le mn < \infty$; that is $xy^{-1} \in H$, so that
      $H \le G$.

      \textcolor{red}{\textbf{Example.} Todo}
%%%%%%%%%%%%%%%%%%%%%%%%%%%%%%%%%%%%%2.1.7%%%%%%%%%%%%%%%%%%%%%%%%%%%%%%%%%%%%%%
   \item[2.1.7]   Fix some $n \in \Z$ with $n > 1$. Find the torsion subgroup
                  of $\Z \times (\Z/n\Z)$. Show that the set of elements of
                  infinite order together with the identity is \textit{not} a
                  subgroup of this direct product.

      \textbf{Proof.} The torsion subgroup of $\Z \times (\Z/n\Z)$ is
      $\{(0, a) : a \in \Z/n\Z\}$. Let $H$ be the set of elements of infinite 
      order together with the identity. We have (1, 1), (1, 2) $\in H$, but
      $(1,\;2) - (1,\;1) = (0,\;1) \notin H$, so that $H$ is not closed under
      addition; that is, $H$ is not a subgroup of $\Z \times (\Z/n\Z)$. \qed
%%%%%%%%%%%%%%%%%%%%%%%%%%%%%%%%%%%%%2.1.8%%%%%%%%%%%%%%%%%%%%%%%%%%%%%%%%%%%%%%
   \item[2.1.8]   Let $H$ and $K$ be subgroups of $G$. Prove that $H \cup K$ is
                  a subgroup if and only if either $H \subseteq K$ or
                  $K \subseteq H$.

      \textbf{Proof.} Suppose first that $H \cup K \le G$. Now assume to the
      contrary that neither $H$ nor $K$ is a subset of the other. Thus there
      exist $h \in H$ and $k \in K$ such that $h \notin K$ and $k \notin H$.
      Since $h, k \in H \cup K$ and $H \cup K$ is a group, it follows that
      $hk \in H \cup K$. Assume without loss that $hk \in H$. Then it follows
      that $k = (h^{-1})hk \in H$, a contradiction. Thus we must have that
      either $H \subseteq K$ or $K \subseteq H$. The converse of the proof is 
      trivial. \qed
%%%%%%%%%%%%%%%%%%%%%%%%%%%%%%%%%%%%%2.1.9%%%%%%%%%%%%%%%%%%%%%%%%%%%%%%%%%%%%%%
   \item[2.1.9]   Let $G = GL_n(F)$, where $F$ is any field. Define
                  $$SL_n(F) = \{A \in GL_n(F) : \det(A) = 1\}$$
                  (called the \textit{special linear group}). Prove that
                  $SL_n(F) \le GL_n(F)$.

      \textbf{Proof.} The set $SL_n(F)$ is nonempty since it contains the
      $n \times n$ identity matrix. Let $A, B \in SL_n(F)$. By membership in
      $SL_n(F)$ we have $\det(A) = 1$, so that $A^{-1}$ exists. Since the
      determinant of a product is the product of the determinants, it follows
      that $\det(A)\det(A^{-1}) = 1$, so that $\det(A^{-1}) = 1$; now
      $\det(AB^{-1}) = \det(A)\det(B^{-1}) = 1$, so that $AB^{-1} \in SL_n(F)$; 
      that is $SL_n(F) \le GL_n(F)$. \qed
%%%%%%%%%%%%%%%%%%%%%%%%%%%%%%%%%%%%%2.1.10%%%%%%%%%%%%%%%%%%%%%%%%%%%%%%%%%%%%%
   \item[2.1.10]  \begin{enumerate}
                     \item Prove that if $H$ and $K$ are subgroups of $G$ then
                           so is their intersection $H \cap K$.
                     \item Prove that the intersection of an arbitrary nonempty
                           collection of subgroups of $G$ is again a subgroup of
                           $G$ (do not assume the collection is countable).
                  \end{enumerate}

      \textbf{Proof.}

      \begin{enumerate}
         \item See (b).
         \item Let $G$ be a group and $S = \D\bigcap_{i \in I}G_i$ be the
               intersection of an arbitrary collection of subgroups of $G$,
               where $G_i \le G$, for each $i \in I$, and $I$ is some indexing   
               set. Since 1 belongs to all $G_i$, it follows that $1 \in S$. Now
               let $x, y \in S$. Then each $G_i$ must have $x$ and $y$ as
               elements. Also since each $G_i$ is a group, it must also contain
               $xy^{-1}$, so that $xy^{-1} \in S$. That is, $S \le G$.
      \end{enumerate} \qed
%%%%%%%%%%%%%%%%%%%%%%%%%%%%%%%%%%%%%2.1.11%%%%%%%%%%%%%%%%%%%%%%%%%%%%%%%%%%%%%
   \item[2.1.11]  Let $A$ and $B$ be groups. Prove that the following sets are
                  subgroups of the direct product $A \times B$.
                  \begin{enumerate}
                     \item $\{(a, 1) : a \in A\}$
                     \item $\{(1, b) : b \in B\}$
                     \item $\{(a, a) : a \in A\}$, where here we assume $B = A$
                           (called the \textit{diagonal subgroup}).
                  \end{enumerate}

      \textbf{Proof.}

      \begin{enumerate}
         \item Let $H = \{(a, 1) : a \in A\}$. We have $(1, 1) \in H$, so that
               $H$ is nonempty. Now let $(x, 1), (y, 1) \in H$. It follows that
               $(x, 1)(y, 1)^{-1} = (x, 1)(y^{-1}, 1) = (xy^{-1}, 1) \in H$, so 
               that $H \le A \times B$.
         \item Let $H = \{(1, b) : b \in B\}$. We have $(1, 1) \in H$, so that
               $H$ is nonempty. Now let $(1, x), (1, y) \in H$. It follows that 
               $(1, x)(1, y)^{-1} = (1, x)(1, y^{-1}) = (1, xy^{-1}) \in H$, so 
               that $H \le A \times B$.
         \item Let $H = \{(a, a) : a \in A\}$. We have $(1, 1) \in H$, so that
               $H$ is nonempty. Now let $(x, x), (y, y) \in H$. It follows that 
               $(x, x)(y, y)^{-1} = (x, x)(y^{-1}, y^{-1}) = (xy^{-1}, xy^{-1}) 
               \in H$, so that $H \le A \times A$.
      \end{enumerate} \qed
%%%%%%%%%%%%%%%%%%%%%%%%%%%%%%%%%%%%%2.1.12%%%%%%%%%%%%%%%%%%%%%%%%%%%%%%%%%%%%%
   \item[2.1.12]  Let $A$ be an abelian group and fix some $n \in \Z$. Prove
                  that the following sets are subgroups of $A$:
                  \begin{enumerate}
                     \item $\{a^n : a \in A\}$
                     \item $\{a \in A : a^n = 1\}$.
                  \end{enumerate}

      \textbf{Proof.}

      \begin{enumerate}
         \item Let $H = \{a^n : a \in A\}$. Since $1^n = 1$, it follows that
               $1 \in H$. So let $x^n, y^n \in H$ for some $x, y \in A$. Then we 
               have that $x^n(y^n)^{-1} = x^n(y^{-1})^n = (xy^{-1})^n$; i.e.,
               $x^n(y^n)^{-1} \in H$ and thus $H \le A$.
         \item Let $H = \{a \in A : a^n=1\}$. Since $1^n = 1$, it follows that
               $1 \in H$. So let $x, y \in H$. Then we have that
               $(xy^{-1})^n = x^n(y^{-1})^n = x^n(y^n)^{-1} = 1$; i.e.,
               $xy^{-1} \in H$ and thus $H \le A$.
      \end{enumerate} \qed
%%%%%%%%%%%%%%%%%%%%%%%%%%%%%%%%%%%%%2.1.13%%%%%%%%%%%%%%%%%%%%%%%%%%%%%%%%%%%%%
   \item[2.1.13]  Let $H$ be a subgroup of the additive group of rational
                  numbers with the property that $1/x \in H$ for every nonzero
                  element $x \in H$. Prove that $H = 0$ or $\Q$.

      \textbf{Proof.} If $H = 0$, then we are done, so assume that $H \neq 0$.
      First notice that if $h \in H$, then $nh \in H$ for all $n \in \Z$ because
      $H$ is closed under addition. Now let $a/b$ be a rational number. To 
      complete the proof, it suffices to show that $a/b \in H$. Since
      $H \neq 0$, it contains a nonzero rational number, say $p/q$.
      Thus $q \cdot p/q = p \in H$. Since $p \neq 0$, it follows by our 
      hypothesis that $1/p \in H$ so that $p \cdot 1/p = 1 \in H$. That is
      $b \cdot 1 = b \in H$, and since $b \neq 0$, we have must have that
      $1/b \in H$, so that $a \cdot 1/b = a/b \in H$, as desired. \qed
%%%%%%%%%%%%%%%%%%%%%%%%%%%%%%%%%%%%%2.1.14%%%%%%%%%%%%%%%%%%%%%%%%%%%%%%%%%%%%%
   \item[2.1.14]  Show that $\{x \in D_{2n} : x^2 = 1\}$ is not a subgroup of
                  $D_{2n}$ (here $n \ge 3$).

      \textbf{Proof.} Let $n \ge 3$. Let $H = \{x \in D_{2n} : x^2 = 1\}$. Since
      $n \ge 3$, we have $sr^2, sr \in D_{2n}$. Also since
      $(sr^2)^2 = (sr)^2 = 1$, we have that $sr^2, sr \in H$. But
      $(sr^2)(sr) = r^{-2}ssr = r^{-2}r = r^{n-1} \notin H$ because
      $(r^{n-1})^2 = r^{2n-2} = r^{-2} = r^{n-2} \neq 1$, so that $H$ is not 
      closed under multiplication. Thus $H$ is not a subgroup of $D_{2n}$. \qed
%%%%%%%%%%%%%%%%%%%%%%%%%%%%%%%%%%%%%2.1.15%%%%%%%%%%%%%%%%%%%%%%%%%%%%%%%%%%%%%
   \item[2.1.15]  Let $H_1 \le H_2 \le \cdots$ be an ascending chain of
                  subgroups of $G$. Prove that $\cup_{i=1}^\infty H_i$ is a 
                  subgroup of $G$.

      \textbf{Proof.} Let $S = \cup_{i=1}^\infty H_i$. First observe that we
      can show by induction that if $x \in H_i$, then $x \in H_{i+n}$ for all
      $n \ge 0$. Since $H_1 \le G$, it must contain 1, so that $1 \in S$. So
      let $x, y \in S$. Then $x \in H_k$ and $y \in H_j$ for some
      $k, j \in \Z^+$. Let us now investigate the following cases:

      \textbf{Case 1.} $k = j$. Thus $H_k = H_j$, so that $x, y \in H_k$. Since
      $H_k$ is a group, we must have that $xy^{-1} \in H_k$ so that
      $xy^{-1} \in S$.

      \textbf{Case 2.} $k < j$. Then since $x \in H_k$, we must have
      $x \in H_j$. Thus $xy^{-1} \in H_j$ so that $xy^{-1} \in S$.

      \textbf{Case 3.} $j < k$. Interchange the roles of $j$ and $k$ and $x$
      and $y$ in Case 2 to conclude that $yx^{-1} \in S$. \\

      In any case, we have thus shown that $S$ is closed under multiplication
      and inverses, so that $S \le G$. \qed
%%%%%%%%%%%%%%%%%%%%%%%%%%%%%%%%%%%%%2.1.16%%%%%%%%%%%%%%%%%%%%%%%%%%%%%%%%%%%%%
   \textcolor{red}{\item[2.1.16]  Let $n \in \Z^+$ and let $F$ be a field. Prove 
               that the set
                  $$H = \{(a_{ij}) \in GL_n(F) : (a_{ij}) = 0 \text{ for all }
                     i > j\}$$
                  is a subgroup of $GL_n(F)$ (called the group of
                  \textit{upper triangular} matrices).}

      \textcolor{red}{\textbf{Proof.} Let $A = (a_{ij})$ and $B = (b_{ij})$ be matrices in $H$. 
      Let $C = (c_{ij}) = AB$. Recall that
      $$c_{ij} = \sum_{k=1}^na_{ik}b_{kj}.$$
      So assume that $i > j$. It follows that
      \begin{align*}
         c_{ij} &= \sum_{k=1}^na_{ik}b_{kj} \\
            &= \sum_{k=i}^na_{ik}b_{kj}
               &[\text{Since }a_{ik} = 0 \text{ if } i > k] \\
            &= 0, &[\text{Since }k \ge i > j \text{ so } b_{kj} = 0]
      \end{align*}
      so that $C \in H$.}
%%%%%%%%%%%%%%%%%%%%%%%%%%%%%%%%%%%%%2.1.17%%%%%%%%%%%%%%%%%%%%%%%%%%%%%%%%%%%%%
   \textcolor{red}{\item[2.1.17]  Let $n \in \Z^+$ and let $F$ be a field. Prove that the set
                  $$\{(a_{ij}) \in GL_n(F) : (a_{ij}) = 0 \text{ for all }
                      i > j, \text{ and } a_{ii} = 1 \text{ for all }i\}$$
                  is a subgroup of $GL_n(F)$.}
\end{enumerate}
