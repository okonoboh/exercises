\begin{enumerate}
   \item[]        In these exercises $D_{2n}$ has the usual presentation
                  $D_{2n} = \cyc{r, s : r^n = s^2 = 1, rs = sr^{-1}}$.
   \item[]        \textbf{Lemma 1.2.1} \textit{For the generators
                  $r, s \in D_{2n}$ and an integer $i$ we have that
                  $sr^i = r^{-i}s$}.
                  
      \textbf{Proof.} First we shall show that our assertion holds for each
      nonnegative integer $i$. So let us induct on $i$. It is clear that our
      assertion holds for $i = 0$ and, using the presentation for $D_{2n}$, for
      $i = 1$. So assume that it also holds for some nonnegative integer $k$.
      Thus we have that
      \begin{align*}
         sr^{k + 1} &= s(r^kr) &[\text{Exercise 1.1.19}] \\
                    &= (sr^k)r &[\text{Associativity}] \\
                    &= (r^{-k}s)r &[\text{Inductive hypothesis}] \\
                    &= r^{-k}(sr) &[\text{Associativity}] \\
                    &= r^{-k}(r^{-1}s) &[D_{2n} \text{ presentation}] \\
                    &= (r^{-k}r^{-1})s &[\text{Associativity}] \\
                    &= r^{-(k+1)}s, &[\text{Exercise 1.1.19}]
      \end{align*}
      so that our assertion holds for $k + 1$; thus, by the Principle of 
      Mathematical Induction, our assertion holds for each nonnegative integer 
      $i$. Now suppose that $i$ is negative. By the Division Algorithm, we can 
      write $i = qn + t$, for unique integers $q$ and $t$, with $0 \le t < n$, 
      so that $r^i = r^{qn+t} = r^{qn}r^t = (r^n)qr^t = 1^qr^t = r^t$, and 
      hence, $r^{-i} = r^{-t}$. Thus we have that
      \begin{align*}
         sr^i &= sr^t \\
              &= r^{-t}s &[\text{Induction above}] \\
              &= r^{-i}s,
      \end{align*}
      as desired, and our assertion holds for each integer $i$. \qed
                  $D_{2n} = \cyc{r, s : r^n = s^2 = 1, rs = sr^{-1}}$.
   \item[]        \textbf{Lemma 1.2.2} \textit{If $P$ is a platonic solid with
                  $f$ faces and $n$ vertices on each face, then the number of
                  rigid motions of $P$ is $f \cdot n$}.
                  
      \textbf{Proof.} Let $P$ be a platonic solid with $f$ faces and $n$
      vertices per face. Denote the faces of $P$ (arbitrarily) by $F_1$, $F_2$,
      $\ldots$, $F_f$. Observe that given any face $F_i$, there is a symmetry 
      that sends face $F_1$ to $F_i$. Now fix face $F_1$ in this new position. 
      Now notice that any of the $n$ vertices on face $F_1$ can be sent to any
      of the other $n-1$ vertices by rotation through an axis that passes
      through the center of $F_1$. Thus there are $f \cdot n$ positions that
      face $F_1$ may be sent to. Since symmetries are rigid motions, it follows 
      that once the position of face $F_1$ has been specified, then the
      positions of the other faces are completely determined. Thus there are
      $f \cdot n$ rigid motions of $P$. \qed
%%%%%%%%%%%%%%%%%%%%%%%%%%%%%%%%%%%%%1.2.1%%%%%%%%%%%%%%%%%%%%%%%%%%%%%%%%%%%%%%
   \item[1.2.1]   Compute the order of each of the elements in the following
                  groups:
                  \begin{enumerate}
                     \item $D_6$.
                     \item $D_8$.
                     \item $D_{10}$.
                  \end{enumerate}
                 
      \textbf{Solution.}
      
      \begin{enumerate}
         \item The elements of $D_6$ are 1, $r$, $r^2$, $s$, $sr$, and $sr^2$
               and their respective orders are 1, 3, 3, 2, 2, and 2.
         \item The elements of $D_8$ are 1, $r$, $r^2$, $r^3$, $s$, $sr$,
               $sr^2$, and $sr^3$ and their respective orders are 1, 4, 2, 4, 2,
               2, 2, and 2.
         \item The elements of $D_{10}$ are 1, $r$, $r^2$, $r^3$, $r^4$, $s$,
               $sr$, $sr^2$, $sr^3$, and $sr^4$ and their respective orders are
               1, 5, 5, 5, 5, 2, 2, 2, 2, and 2.
      \end{enumerate}
%%%%%%%%%%%%%%%%%%%%%%%%%%%%%%%%%%%%%1.2.2%%%%%%%%%%%%%%%%%%%%%%%%%%%%%%%%%%%%%%
   \item[1.2.2]   Use the generators and relations above to show that if $x$ is
                  any element of $D_{2n}$ which is not a power of $r$, then
                  $rx = xr^{-1}$. 
                  
      \textbf{Proof.} Suppose that $x \in D_{2n}$ such that $x$ is not an
      integral power of $r$. We know from the argument in the text that we can
      then write $x = sr^i$, with $0 \le i < n$. Our assertion follows
      because
      \begin{align*}
         rx &= rsr^i \\
            &= (rs)r^i &[\text{Associativity}] \\
            &= (sr^{-1})r^i &[D_{2n}\text{ Presentation}] \\
            &= s(r^{-1}r^i) &[\text{Associativity}] \\
            &= s(r^ir^{-1}) &[\text{Integral powers of an element commute}] \\
            &= (sr^i)(r^{-1}) &[\text{Associativity}] \\
            &= xr^{-1}.
      \end{align*}
      \qed
%%%%%%%%%%%%%%%%%%%%%%%%%%%%%%%%%%%%%1.2.3%%%%%%%%%%%%%%%%%%%%%%%%%%%%%%%%%%%%%%
   \item[1.2.3]   Use the generators and relations above to show that every
                  element of $D_{2n}$ which is not a power of $r$ has order 2.
                  Deduce that $D_{2n}$ is generated by the two elements $s$ and
                  $sr$, both of which have order 2.
                  
      \textbf{Proof.} Suppose that $x \in D_{2n}$ such that $x$ is not an
      integral power of $r$. Thus $x = sr^i$, with $0 \le i < n$ so that
      \begin{align*}
         x^2 &= (sr^isr^i) \\
             &= (sr^ir^{-i}s) &[\text{Lemma 1.2.1}] \\
             &= s1s \\
             &= s^2 = 1.
      \end{align*}
      That is, $|x| \le 2$. Now $sr^i \neq 1$ because $s \neq r^j$ for any
      integer $j$, so that we can conclude that $|x| = 2$. Recall: for
      $w \in D_{2n}$, there exist integers $x$ and $y$, such that $w = s^xr^y$.
      That is, $w = s^x(ssr)^y$. This says that $D_{2n}$ is generated by $s$ and
      $sr$. \qed
%%%%%%%%%%%%%%%%%%%%%%%%%%%%%%%%%%%%%1.2.4%%%%%%%%%%%%%%%%%%%%%%%%%%%%%%%%%%%%%%
   \item[1.2.4]   If $n = 2k$ is even and $n \ge 4$, show that $z = r^k$ is an
                  element of order 2 which commutes with all elements of
                  $D_{2n}$. Show also that $z$ is the only nonidentity element
                  of $D_{2n}$ which commutes with all elements of $D_{2n}$. [cf.
                  Exercise 33 of Section 1.]
                  
      \textbf{Proof.} Consider $D_{2n}$ where $n = 2k$ for some positive integer
      $k \ge 2$. Let $z = r^k$. Since $k < n$ and since $|r| = n$, it must be
      the case that $|z| \neq 1$, so that $|z| \ge 2$. Now we have that
      $z^2 = r^kr^k = r^{2k} = r^n = 1$; i.e, $|z| = 2$. In particular, since
      $r^kr^k = 1$, it follows that $r^{-k} = r^k$. It is clear that $z$
      commutes with all elements that are integral powers of $r$. So consider
      $sr^i$, with $0 \le i < n$. Then we have that
      \begin{align*}
         zsr^i &= r^ksr^i \\
               &= sr^{-k}r^i  &[\text{Lemma 1.2.1}] \\
               &= sr^kr^i     &[r^k = r^{-k}] \\
               &= sr^ir^k = sr^iz.
      \end{align*}
      That is $z$ commutes with all elements of $D_{2n}$. Now we want to show
      that $z$ is the only nonidentity element of $D_{2n}$ that commutes with
      every element of $D_{2n}$. So let $y \in D_{2n}$ be such an element (we
      already showed that at least one, $r^k$, exists).
      
      \textbf{Case 1:} \textit{$y = r^i$, with $1 \le i < n$.} Particularly, we
      must have that $y$ commutes with $s$. Using Lemma 1.2.1, we have that
      $r^{-i}s = sr^i = r^is$, so that $r^{-i} = r^i$. Then we must have that
      $r^{2i} = 1$, so that by Exercise 1.1.33, we have $i = k$.
      
      \textbf{Case 2:} \textit{$y = sr^i$, with $0 \le i < n$.} Particularly, we
      must have that $y$ commutes with $r$. Then we have that
      \begin{align*}
         (sr^i)r &= r(sr^i) &[yr = ry] \\
                 &= (rs)r^i &[\text{Associativity}] \\
                 &= (sr^{-1})r^i &[D_{2n}\text{ presentation}] \\
                 &= s(r^{-1}r^i) &[\text{Associativity}] \\
                 &= s(r^ir^{-1}) &[\text{Powers of $r$ commute}] \\
                 &= (sr^i)r^{-1}, &[\text{Associativity}]
      \end{align*}
      so that $r = r^{-1}$ by left cancellation; thus $r^2 = 1$, a
      contradiction. We can now conclude that $r^k$ is the only nonidentity
      element of $D_{2n}$ that commutes with every element of $D_{2n}$. \qed
%%%%%%%%%%%%%%%%%%%%%%%%%%%%%%%%%%%%%1.2.5%%%%%%%%%%%%%%%%%%%%%%%%%%%%%%%%%%%%%%
   \item[1.2.5]   If $n$ is odd and $n \ge 3$, show that the identity is the
                  only element of $D_{2n}$ which commutes with all elements of
                  $D_{2n}$. [cf. Exercise 33 of Section 1.]

      \textbf{Proof.} Let $n$ be an odd integer such that $n \ge 3$. Suppose to
      the contrary that there exists a nonidentity $y \in D_{2n}$ that commutes
      with every other element of $D_{2n}$.
      
      \textbf{Case 1:} \textit{$y = r^i$, with $1 \le i < n$.} By Case 1 in
      Exercise 1.2.4, it follows that $i = k$, and this contradicts
      Exercise 1.1.33(a).
      
      
      \textbf{Case 2:} \textit{$y = sr^i$, with $0 \le i < n$.} By Case 2 in
      Exercise 1.24, we will arrive at a contradiction.

      We can now conclude that the identity is the only element of $D_{2n}$ 
      which commutes with all elements of $D_{2n}$. \qed

%%%%%%%%%%%%%%%%%%%%%%%%%%%%%%%%%%%%%1.2.6%%%%%%%%%%%%%%%%%%%%%%%%%%%%%%%%%%%%%%
   \item[1.2.6]   Let $x$ and $y$ be elements of order 2 in any group $G$. Prove
                  that if $t = xy$ then $tx = xt^{-1}$ (so that if
                  $n = |xy| < \infty$ then $x, t$ satisfy the same relations in
                  $G$ as $s, r$ do in $D_{2n}$).

      \textbf{Proof.} Suppose that $t = xy$. Since $x^2 = y^2 = 1$, it follows 
      that $x = x^{-1}$ and $y = y^{-1}$, so that
      \begin{align*}
         tx &= (xy)x \\
            &= x(yx) &[\text{Associativity}] \\
            &= x(y^{-1}x^{-1}) \\
            &= x(xy)^{-1} \\
            &= xt^{-1},
      \end{align*}
      as desired. \qed
%%%%%%%%%%%%%%%%%%%%%%%%%%%%%%%%%%%%%1.2.7%%%%%%%%%%%%%%%%%%%%%%%%%%%%%%%%%%%%%%
   \item[1.2.7]   Show that $F = \cyc{a, b : a^2 = b^2 = (ab)^n = 1}$ gives a
                  presentation for $D_{2n}$ in terms of the two generators
                  $a = s$ and $b = sr$ of order 2 computed in Exercise 3 above.
                  [Show that the relations for $r$ and $s$ follow from the
                  relations for $a$ and $b$ and, conversely, the relations for
                  $a$ and $b$ follow from those for $r$ and $s$.]
                  
      \textbf{Proof.} We already showed in Exercise 1.2.3 that $a$ and $b$
      generate $D_{2n}$, so first we want to show that the relations for $a$ and
      $b$ follow from that of $r$ and $s$. Since $s^2 = 1$ and since $a = s$, it
      follows that $a^2 = 1$. Also we have that $b^2 = (sr)^2 = 1$ by Exercise
      1.2.3. Now we have that $(ab)^n = (s(sr))^n = ((ss)r)^n = r^n = 1$.
      Conversely, we have that $s^2 = 1$ since $a^2 = 1$ and $s = a$. Also we
      have that $1 = (ab)^n = r^n$. Finally, since $srsr = 1$, it follows that
      $rsr = s^{-1} = s$, so that $rs = sr^{-1}$, as desired. \qed
%%%%%%%%%%%%%%%%%%%%%%%%%%%%%%%%%%%%%1.2.8%%%%%%%%%%%%%%%%%%%%%%%%%%%%%%%%%%%%%%
   \item[1.2.8]   Find the order of the cyclic subgroup of $D_{2n}$ generated by
                  $r$ (cf. Exercise 27 of Section 1)
                  
      \textbf{Solution.} By inspection, we can conclude that the order of the
      subgroup of $D_{2n}$ generated by $r$ is $n$.
%%%%%%%%%%%%%%%%%%%%%%%%%%%%%%%%%%%%%1.2.8%%%%%%%%%%%%%%%%%%%%%%%%%%%%%%%%%%%%%%
   \item[]        \textbf{For Exercises 1.2.9 - 1.2.13, use Lemma 1.2.2.}
%%%%%%%%%%%%%%%%%%%%%%%%%%%%%%%%%%%%%1.2.9%%%%%%%%%%%%%%%%%%%%%%%%%%%%%%%%%%%%%%
   \item[1.2.9]   Let $G$ be the group of rigid motions in $\R^3$ of a
                  tetrahedron. Show that $|G| = 12$.
%%%%%%%%%%%%%%%%%%%%%%%%%%%%%%%%%%%%%1.2.10%%%%%%%%%%%%%%%%%%%%%%%%%%%%%%%%%%%%%
   \item[1.2.10]  Let $G$ be the group of rigid motions in $\R^3$ of a cube.
                  Show that $|G| = 24$.
%%%%%%%%%%%%%%%%%%%%%%%%%%%%%%%%%%%%%1.2.11%%%%%%%%%%%%%%%%%%%%%%%%%%%%%%%%%%%%%
   \item[1.2.11]  Let $G$ be the group of rigid motions in $\R^3$ of an
                  octahedron. Show that $|G| = 24$.
%%%%%%%%%%%%%%%%%%%%%%%%%%%%%%%%%%%%%1.2.12%%%%%%%%%%%%%%%%%%%%%%%%%%%%%%%%%%%%%
   \item[1.2.12]  Let $G$ be the group of rigid motions in $\R^3$ of a
                  dodecahedron. Show that $|G| = 60$.
%%%%%%%%%%%%%%%%%%%%%%%%%%%%%%%%%%%%%1.2.13%%%%%%%%%%%%%%%%%%%%%%%%%%%%%%%%%%%%%
   \item[1.2.13]  Let $G$ be the group of rigid motions in $\R^3$ of a
                  icosahedron. Show that $|G| = 60$.
%%%%%%%%%%%%%%%%%%%%%%%%%%%%%%%%%%%%%1.2.14%%%%%%%%%%%%%%%%%%%%%%%%%%%%%%%%%%%%%
   \item[1.2.14]  Find a set of generators for $\Z$.
   
      \textbf{Solution.} Let $z \in \Z$. Since $z = (2z) \cdot 3 - z \cdot 5$,
      it follows that $\Z$ is generated by 3 and 5, so that $\Z = \cyc{3, 5}$.
%%%%%%%%%%%%%%%%%%%%%%%%%%%%%%%%%%%%%1.2.15%%%%%%%%%%%%%%%%%%%%%%%%%%%%%%%%%%%%%
   \item[1.2.15]  Find a set of generators and relations for $\Z/n\Z$.
   
      \textbf{Solution.} We have that
      $$\Z/n\Z = \cyc{1 : 1^n = 0}.$$
      In this case 0, not 1, is the identity element.
%%%%%%%%%%%%%%%%%%%%%%%%%%%%%%%%%%%%%1.2.16%%%%%%%%%%%%%%%%%%%%%%%%%%%%%%%%%%%%%
   \item[1.2.16]  Show that the group $S =
                  \cyc{x_1, y_1 : {x_1}^2 = {y_1}^2 = (x_1y_1)^2 = 1}$ is the
                  dihedral group $D_4$ (where $x_1$ may be replaced by the
                  letter $r$ and $y_1$ by $s$). [Show that the last relation is
                  the same as: $x_1y_1 = y_1{x_1}^{-1}$.]

      \textbf{Proof.} We can think of $x_1$ as $r$ and $y_1$ as $s$. In $D_4$,
      we know that $r^2 = s^2 = 1$ and $rs = sr^{-1}$, and since
      ${x_1}^2 = {y_1}^2 = 1$, it only remains to show that
      $x_1x_2 = x_2{x_1}^{-1}$. Note that ${y_1}^{-1} = y_1$ because
      ${y_1}^2 = 1$. By hypothesis, we have that $x_1y_1x_1y_1 = 1$,
      so that $x_1y_1 = {y_1}^{-1}{x_1}^{-1} = y_1{x_1}^{-1}$, and, in 
      conclusion, $S = D_4$. \qed
%%%%%%%%%%%%%%%%%%%%%%%%%%%%%%%%%%%%%1.2.17%%%%%%%%%%%%%%%%%%%%%%%%%%%%%%%%%%%%%
   \item[1.2.17]  Let $X_{2n}$ be the group whose presentation
                                  is displayed in (1.2).
                  \begin{enumerate}
                     \item Show that if $n = 3k$, then $X_{2n}$ has order 6, and
                           it has the same generators and relations as $D_6$
                           when $x$ is replaced by $r$ and $y$ by $s$.
                     \item Show that if $(3, n) = 1$, then $x$ satisfies the
                           additional relation: $x = 1$. In this case deduce
                           that $X_{2n}$ has order 2. [Use the facts that
                           $x^n = 1$ and $x^3 = 1$.]
                  \end{enumerate}

      \textbf{Proof.} Given $X_{2n} = \cyc{x, y :  x^n = y^2 = 1, \;xy = yx^2}$.
      Also from the discussion on Page 27 of the textbook, we have $x^3 = 1$.

      \begin{enumerate}
         \item Let $n = 3k$. Notice that we can deduce the relation
               $x^n = x^{3k} = 1$ from the relation $x^3 = 1$. Thus
               $$X_{2n} = \cyc{x, y :  x^3 = y^2 = 1, \;xy = yx^2}.$$
               Since no other collapsing is possible it follows that
               $|X_{2n}| = |x| \cdot |y| = 3 \cdot 2 = 6$. Since $x^3 = 1$, it
               follows that $x^2 = x^{-1}$. Thus the relation $xy = yx^2$ is
               equivalent to $xy = yx^{-1}$. If we replace $x$ by $r$ and $y$ by
               $s$, then we see that $X_{2n}$ has the same generators and
               relations as $D_6$.
         \item Suppose $(3, n) = 1$. Thus there exist integers $p$ and $q$ such
               that $1 = 3p + nq$. So
               $x^1 = x^{3p+nq} = (x^3)^p(x^n)^q = 1^p1^q = 1$. Thus
               $X_{2n} = \cyc{y : y^2 = 1}$, so that $|X_{2n}| = 2$.
      \end{enumerate} \qed
%%%%%%%%%%%%%%%%%%%%%%%%%%%%%%%%%%%%%1.2.18%%%%%%%%%%%%%%%%%%%%%%%%%%%%%%%%%%%%%
   \item[1.2.18]  Let $Y$ be the group whose presentation is displayed in (1.3).
                  \begin{enumerate}
                     \item Show that $v^2 = v^{-1}$.
                           [Use the relation: $v^3 = 1$.]
                     \item Show that $v$ commutes with $u^3$. [Show that
                           $v^2u^3v = u^3$ by writing the left hand side as
                           $(v^2u^2)(uv)$ and using the relations to reduce this
                           to the right hand side. Then use part (a).]
                     \item Show that $v$ commutes with $u$. [Show that $u^9 = u$
                           and then use part (b).]
                     \item Show that $uv = 1$. [Use part (c) and the last
                           relation.]
                     \item Show that $u = 1$, deduce that $v = 1$, and conclude
                           that $Y = 1$. [Use part (d) and the equation
                           $u^4v^3 = 1$.]
                  \end{enumerate}

      \textbf{Proof.} Given $Y = \cyc{u, v :  u^4 = v^3 = 1, \; uv = v^2u^2}$.

      \begin{enumerate}
         \item Multiply the relation $v^3 = 1$ by $v^{-1}$ to conclude that
               $v^2 = v^{-1}$.
         \item We have that
               \begin{align*}
                  v^2u^3v &= (v^2u^2)(uv) \\
                          &= (uv)(v^2u^2) &[uv = v^2u^2] \\
                          &= uv^3u^2 \\
                          &= u^3. &[v^3 = 1]
               \end{align*}
               Now multiply the equality $v^2u^3v = u^3$ on the left by $v$ to
               get $u^3v = vu^3$; that is $v$ commutes with $u^3$.
         \item First notice that since $u^4 = 1$, we must have that
               $$u^9 = u(u^8) = u(u^4u^4) = u(1)(1) = u.$$
               Now from (b) we know that $v$ commutes with $u^3$; thus $v$ must
               commute with all powers of $u^3$, so that $v$ commutes with
               $u^9$. But $u^9 = u$ and we can thus conclude that $v$ commutes
               with $u$.
         \item We have
               \begin{align*}
                  vu &= uv &[\text{(c)}] \\
                     &= v^2u^2 &[\text{Given}] \\
                     &= v(vu)u \\
                     &= v(uv)u &[\text{(c)}],
               \end{align*}
               so that $uv = 1$ by cancellation.
         \item We have
               \begin{align*}
                  1 &= u^4v^3 &[\text{Given }u^4 = v^3 = 1] \\
                    &= (uv)^3u &[\text{Since }u\text{ and }v\text{ commute}] \\
                    &= u, &[uv = 1]
               \end{align*}
               so that $u = 1$ and since $uv = 1$, it follows that $v = 1$. We
               can thus conclude that $Y = \{1\}$. \qed
      \end{enumerate}
\end{enumerate}
