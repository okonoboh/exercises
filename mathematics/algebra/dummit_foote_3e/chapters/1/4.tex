Let $F$ be a field and let $n \in \Z^+$.
\begin{enumerate}
%%%%%%%%%%%%%%%%%%%%%%%%%%%%%%%%%%%%%1.4.1%%%%%%%%%%%%%%%%%%%%%%%%%%%%%%%%%%%%%%
   \item[1.4.1]   Prove that $|GL_2(\F_2)| = 6$.

      \textbf{Solution.} Set $p = 2$ in Exercise 1.4.7.
%%%%%%%%%%%%%%%%%%%%%%%%%%%%%%%%%%%%%1.4.2%%%%%%%%%%%%%%%%%%%%%%%%%%%%%%%%%%%%%%
   \item[1.4.2]   Write out all the elements of $GL_2(\F_2)$ and compute the
                  order of each element.

      \textbf{Solution.}
      \begin{center}
         \begin{tabular}{@{}|c|c|@{}} \hline
            Matrix in $GL_2(\F_2)$           & Order  \\ \hline            
            $\left(\begin{tabular}{@{}cc@{}}
               1 & 0 \\
               0 & 1
            \end{tabular}\right)$            & 1      \\ \hline           
            $\left(\begin{tabular}{@{}cc@{}}
               1 & 0 \\
               1 & 1
            \end{tabular}\right)$            & 2      \\ \hline           
            $\left(\begin{tabular}{@{}cc@{}}
               1 & 1 \\
               0 & 1
            \end{tabular}\right)$            & 2      \\ \hline           
            $\left(\begin{tabular}{@{}cc@{}}
               1 & 1 \\
               1 & 0
            \end{tabular}\right)$            & 3      \\ \hline           
            $\left(\begin{tabular}{@{}cc@{}}
               0 & 1 \\
               1 & 0
            \end{tabular}\right)$            & 2      \\ \hline           
            $\left(\begin{tabular}{@{}cc@{}}
               0 & 1 \\
               1 & 1
            \end{tabular}\right)$            & 3      \\ \hline
         \end{tabular}
      \end{center}
%%%%%%%%%%%%%%%%%%%%%%%%%%%%%%%%%%%%%1.4.3%%%%%%%%%%%%%%%%%%%%%%%%%%%%%%%%%%%%%%
   \item[1.4.3]   Show that $GL_2(\F_2)$ is non-abelian.

      \textbf{Proof.} This follows immediately because
      $$\left(\begin{tabular}{@{}cc@{}}
            1 & 0 \\
            1 & 1
         \end{tabular}\right)\left(\begin{tabular}{@{}cc@{}}
            0 & 1 \\
            1 & 1
         \end{tabular}\right) = \left(\begin{tabular}{@{}cc@{}}
            0 & 1 \\
            1 & 0
         \end{tabular}\right) \neq \left(\begin{tabular}{@{}cc@{}}
            1 & 1 \\
            0 & 1
         \end{tabular}\right) = \left(\begin{tabular}{@{}cc@{}}
            0 & 1 \\
            1 & 1
         \end{tabular}\right)\left(\begin{tabular}{@{}cc@{}}
            1 & 0 \\
            1 & 1
         \end{tabular}\right).$$
         \qed 
%%%%%%%%%%%%%%%%%%%%%%%%%%%%%%%%%%%%%1.4.4%%%%%%%%%%%%%%%%%%%%%%%%%%%%%%%%%%%%%%
   \item[1.4.4]   Show that if $n$ is not prime then $\Z/n\Z$ is not a field.

      \textbf{Proof.} Suppose that $n$ is not prime. Then there exist positive
      integers $a$ and $b$, with $1 < a \le b < n$, such that $n = ab$. Now
      suppose to the contrary that $\Z/n\Z$ is a field. Since $a$ is not 0 in
      $\Z/n\Z$ it must have a multiplicative inverse, say $a^{-1}$, so that
      $b \equiv a^{-1}(ab) \equiv a^{-1}n \equiv 0$ mod $n$, a contradiction
      since $1 < b < n$. It follows that $\Z/n\Z$ is not a field. \qed
%%%%%%%%%%%%%%%%%%%%%%%%%%%%%%%%%%%%%1.4.5%%%%%%%%%%%%%%%%%%%%%%%%%%%%%%%%%%%%%%
   \item[1.4.5]   Show that $GL_n(F)$ is a finite group if and only if $F$ has a
                  finite number of elements.

      \textbf{Proof.} ($\Leftarrow$) Suppose $|F| = m$. Let $M_n(F)$ be the
      set of $n \times n$ matrices with entries in $F$. Clearly
      $GL_n(F) \subset M_n(F)$; since $|M_n(F)| = m^{n^2}$, it follows that
      $GL_n(F)$ is finite.

      ($\Rightarrow$) We shall instead prove the contrapositive in this
      direction. Let $M(a) = (a_{ij})$ denote the $n \times n$ diagonal matrix 
      where $a_{11} = a$ and $a_{ii} = 1$, for $1 < i \le n$. Hence
      det$(M(a)) = a$. So suppose $F$ is infinite (so that $F^\times$ is also
      infinite), then the set $\left\{M(a) : a \in F^\times\right\}$ is not only 
      infinite but also a subset of $GL_n(F)$. Thus $GL_n(F)$ is infinite, as 
      desired. \qed
%%%%%%%%%%%%%%%%%%%%%%%%%%%%%%%%%%%%%1.4.6%%%%%%%%%%%%%%%%%%%%%%%%%%%%%%%%%%%%%%
   \item[1.4.6]   If $|F| = q$ is finite prove that $|GL_n(F)| < q^{n^2}$.
   
      \textbf{Proof.} Let $M_n(F)$ be the set of all $n \times n$ matrices with
      entries in $F$. Now we have that $|M_n(F)| = q^{n^2}$. Since $GL_n(F)$ is
      a subset of $M_n(F)$ and since $GL_n(F)$ does not contain the zero matrix
      it follows that $GL_n(F)$ is a proper subset of $M_n(F)$ so that
      $|GL_n(F)| < |M_n(F)| = q^{n^2}$. \qed
%%%%%%%%%%%%%%%%%%%%%%%%%%%%%%%%%%%%%1.4.7%%%%%%%%%%%%%%%%%%%%%%%%%%%%%%%%%%%%%%
   \item[1.4.7]   Let $p$ be a prime. Prove that the order of $GL_2(\F_p)$ is
                  $p^4 - p^3 - p^2 + p$.
                  
      \textbf{Proof.} We shall the use following fact from Linear Algebra: a
      matrix with entries in a field is invertible if and only if its rows are
      linearly independent. So it suffices to count the number of matrices with
      independent rows. Since $|\F_p| = p$, it follows that there are $p^2$
      possible row vectors in $GL_2(\F_p)$. To form an invertible matrix, we
      cannot choose the zero vector, so there are $p^2 - 1$ choices for the
      first row.  Now two vectors are linearly independent if and only they are
      not a multiple of each other. Thus there are $p^2 - p$ for the second row.
      Thus $|GL_2(\F_p)| = (p^2 - 1)(p^2 - p) = p^4 - p^3 - p^2 + p$. \qed
%%%%%%%%%%%%%%%%%%%%%%%%%%%%%%%%%%%%%1.4.8%%%%%%%%%%%%%%%%%%%%%%%%%%%%%%%%%%%%%%
   \item[1.4.8]   Show that $GL_n(F)$ is non-abelian for any $n \ge 2$ and any
                  $F$.
                  
      \textbf{Proof.} Let $U_n$ be the $n \times n$ upper triangular matrix with
      1s on and above the main diagonal; similarly, let $L_n$ be the
      $n \times n$ lower triangular matrix with 1s on and below the main
      diagonal. Since $U_n$ and $L_n$ are triangular matrices with 1s on their
      main diagonals, it follows that $\det(U_n) = \det(L_n) = 1$, so that
      $U_n, L_n \in G_n(F)$. Let $(a_{ij}) = U_n \cdot L_n$ and
      $(b_{ij}) = L_n \cdot U_n$. We have that $a_{11} = n \neq 1 = b_{11}$ so
      that $U_n$ and $L_n$ do not commute. Thus $G_n(F)$ is non-abelian. \qed
%%%%%%%%%%%%%%%%%%%%%%%%%%%%%%%%%%%%%1.4.9%%%%%%%%%%%%%%%%%%%%%%%%%%%%%%%%%%%%%%
   \item[1.4.9]   Prove that the binary operation of matrix multiplication of
                  $2 \times 2$ matrices with real number entries is associative.
                  
      \textbf{Proof.} Let
      $$X_1 = \left(\begin{tabular}{@{}cc@{}}
         $a_1$ & $b_1$ \\
         $c_1$ & $d_1$
      \end{tabular}\right),
      X_2 = \left(\begin{tabular}{@{}cc@{}}
         $a_2$ & $b_2$ \\
         $c_2$ & $d_2$
      \end{tabular}\right), \text{ and }
      X_3 = \left(\begin{tabular}{@{}cc@{}}
         $a_3$ & $b_3$ \\
         $c_3$ & $d_3$
      \end{tabular}\right)$$
      be real matrices. Then we have that
      \begin{align*}
         (X_1X_2)X_3 &= \left(\begin{tabular}{@{}cc@{}}
            $a_1a_2 + b_1c_2$ & $a_1b_2+b_1d_2$ \\
            $a_2c_1 + c_2d_1$ & $b_2c_1+d_1d_2$
         \end{tabular}\right)\left(\begin{tabular}{@{}cc@{}}
            $a_3$ & $b_3$ \\
            $c_3$ & $d_3$
         \end{tabular}\right) \\
         &= \left(\begin{tabular}{@{}cc@{}}
            $a_1a_2a_3 + a_3b_1c_2 + a_1b_2c_3+b_1c_3d_2$ &
            $a_1a_2b_3 + b_1b_3c_2 +a_1b_2d_3+b_1d_2d_3$ \\
            $a_2a_3c_1 + a_3c_2d_1+b_2c_1c_3+c_3d_1d_2$ &
            $a_2b_3c_1 + b_3c_2d_1+b_2c_1d_3+d_1d_2d_3$
         \end{tabular}\right) \\
         &= \left(\begin{tabular}{@{}cc@{}}
            $a_1(a_2a_3+b_2c_3)+b_1(a_3c_2 + c_3d_2)$ &
            $a_1(a_2b_3+b_2d_3)+b_1(b_3c_2+d_2d_3)$ \\
            $c_1(a_2a_3+b_2c_3)+d_1(a_3c_2+c_3d_2)$ &
            $c_1(a_2b_3+b_2d_3)+d_1(b_3c_2+d_2d_3)$
         \end{tabular}\right) \\
         &= \left(\begin{tabular}{@{}cc@{}}
            $a_1$ & $b_1$ \\
            $c_1$ & $d_1$
         \end{tabular}\right)\left(\begin{tabular}{@{}cc@{}}
            $a_2a_3 + b_2c_3$ & $a_2b_3+b_2d_3$ \\
            $a_3c_2 + c_3d_2$ & $b_3c_2+d_2d_3$
         \end{tabular}\right) \\
         &= X_1(X_2X_3),
      \end{align*}
      so that the binary operation of matrix multiplication of $2 \times 2$ 
      matrices with real number entries is associative. \qed
%%%%%%%%%%%%%%%%%%%%%%%%%%%%%%%%%%%%%1.4.10%%%%%%%%%%%%%%%%%%%%%%%%%%%%%%%%%%%%%
   \item[1.4.10]  Let $G = \left\{\left(\begin{tabular}{@{}cc@{}}
                     $a$ & $b$ \\
                      0  & $c$
                  \end{tabular}\right) : a, b, c \in \R, a \neq 0, c \neq 0
                  \right\}$.

                  \begin{enumerate}
                     \item Compute the product of
                           $\left(\begin{tabular}{@{}cc@{}}
                              $a_1$ & $b_1$ \\
                              0  & $c_1$
                           \end{tabular}\right)$ and
                           $\left(\begin{tabular}{@{}cc@{}}
                              $a_2$ & $b_2$ \\
                              0  & $c_2$
                           \end{tabular}\right)$ to show that $G$ is closed under
                           matrix multiplication.
                     \item Find the matrix inverse of
                           $\left(\begin{tabular}{@{}cc@{}}
                              $a$ & $b$ \\
                              0  & $c$
                           \end{tabular}\right)$ and deduce that $G$ is closed 
                           under inverses.
                     \item Deduce that $G$ is a subgroup of $GL_2(\R)$.
                     \item Prove that the set of elements of $G$ whose two
                           diagonal entries are equal is also a subgroup of
                           $GL_2(\R)$.
                  \end{enumerate}

      \textbf{Solution.}

      \begin{enumerate}
         \item It follows immediately that $G$ is closed under matrix
               multiplication because
               $$\left(\begin{tabular}{@{}cc@{}}
                     $a_1$ & $b_1$ \\
                     0  & $c_1$
                  \end{tabular}\right)\left(\begin{tabular}{@{}cc@{}}
                     $a_2$ & $b_2$ \\
                     0  & $c_2$
                  \end{tabular}\right) = \left(\begin{tabular}{@{}cc@{}}
                     $a_1a_2$ & $a_1b_2+b_1c_2$ \\
                     0  & $c_1c_2$
                  \end{tabular}\right) \in G.$$
         \item Since $a \neq 0$ and $c \neq 0$, it follows that $ac \neq 0$
               so that $\left(\begin{tabular}{@{}cc@{}}
                  $a$ & $b$ \\
                   0  & $c$
               \end{tabular}\right)^{-1} = \D\frac{1}{ac}
               \left(\begin{tabular}{@{}cr@{}}
                  $c$ & $-b$ \\
                   0  & $a$
               \end{tabular}\right) \in G$; i.e. $G$ is closed under inverses.
         \item Since $G$ is nonempty, it follows by (a) and (b) that $G$ is a
               subgroup of $GL_2(\R)$.
         \item Proof is identical to what we did in (a)$-$(c).
      \end{enumerate}
\end{enumerate}

The next exercise introduces the \textit{Heisenberg group} over the field $F$
and develops some of its basic properties. When $F = \R$ this groups plays an
important role in quantum mechanics and signal theory by giving a group
theoretic interpretation (due to H. Weyl) of Heisenberg's Uncertainty Principle.
Note also that the Heisenberg group may be defined more generally---for example,
with entries in $\Z$.

\begin{enumerate}
%%%%%%%%%%%%%%%%%%%%%%%%%%%%%%%%%%%%%1.4.11%%%%%%%%%%%%%%%%%%%%%%%%%%%%%%%%%%%%%
   \item[1.4.11]  Let $H(F) = \left\{\left(\begin{tabular}{@{}ccc@{}}
                     1 & $a$ & $b$ \\
                     0 & 1 & $c$ \\
                     0 & 0 & 1
                  \end{tabular}\right) : a, b, c \in F\right\}$---called the
                  \textit{Heisenberg group} over $F$. Let
                  $X = \left(\begin{tabular}{@{}ccc@{}}
                     1 & $a$ & $b$ \\
                     0 & 1 & $c$ \\
                     0 & 0 & 1
                  \end{tabular}\right)$ and $Y =\left(\begin{tabular}{@{}ccc@{}}
                     1 & $d$ & $e$ \\
                     0 & 1 & $f$ \\
                     0 & 0 & 1
                  \end{tabular}\right)$ be elements of $H(F)$.

                  \begin{enumerate}
                     \item Compute the matrix product $XY$ and deduce that
                           $H(F)$ is closed under matrix multiplication. Exhibit
                           explicit matrices such that $XY \neq YX$ (so that
                           $H(F)$ is always non-abelian).
                     \item Find an explicit formula for the matrix inverse
                           $X^{-1}$ and deduce that $H(F)$ is closed under
                           inverses.
                     \item Prove the associative law for $H(F)$ and deduce that
                           $H(F)$ is a group of order $|F|^3$. 

                           (Do not assume that matrix multiplication is 
                           associative).
                     \item Find the order of each element of the finite group
                           $H(\Z/2\Z)$.
                     \item Prove that every nonidentity element of the group
                           $H(\R)$ has infinite order.
                  \end{enumerate}

   \textbf{Solution.}

   \begin{enumerate}
      \item We have that
            $$XY = \left(\begin{tabular}{@{}ccc@{}}
               1 & $a+d$ & $af + b + e$ \\
               0 & 1 & $c+f$ \\
               0 & 0 & 1
            \end{tabular}\right) \in H(F),$$
            so that $H(F)$ is closed under matrix multiplication. Now $H(F)$ is
            non-abelian since
            \begin{align*}
               \left(\begin{tabular}{@{}ccc@{}}
                  1 & 0 & 0 \\
                  0 & 1 & 1 \\
                  0 & 0 & 1
               \end{tabular}\right)
               \left(\begin{tabular}{@{}ccc@{}}
                  1 & 1 & 0 \\
                  0 & 1 & 0 \\
                  0 & 0 & 1
               \end{tabular}\right) &=
               \left(\begin{tabular}{@{}ccc@{}}
                  1 & 1 & 0 \\
                  0 & 1 & 1 \\
                  0 & 0 & 1
               \end{tabular}\right) \\ &\neq
               \left(\begin{tabular}{@{}ccc@{}}
                  1 & 1 & 1 \\
                  0 & 1 & 1 \\
                  0 & 0 & 1
               \end{tabular}\right) =
               \left(\begin{tabular}{@{}ccc@{}}
                  1 & 1 & 0 \\
                  0 & 1 & 0 \\
                  0 & 0 & 1
               \end{tabular}\right)
               \left(\begin{tabular}{@{}ccc@{}}
                  1 & 0 & 0 \\
                  0 & 1 & 1 \\
                  0 & 0 & 1
               \end{tabular}\right).
            \end{align*}
      \item Using row reduction we find that
            $X^{-1} = \left(\begin{tabular}{@{}crr@{}}
               1 & $-a$ & $ac-b$ \\
               0 & 1 & $-c$ \\
               0 & 0 & 1
            \end{tabular}\right) \in H(F)$, so that $H(F)$ is closed under
            inverses.
      \item Let $Z = \left(\begin{tabular}{@{}ccc@{}}
               1 & $g$ & $h$ \\
               0 & 1 & $i$ \\
               0 & 0 & 1
            \end{tabular}\right)$. Thus we have that
            \begin{align*}
               (XY)Z &= \left(\begin{tabular}{@{}ccc@{}}
                  1 & $a+d$ & $af + b + e$ \\
                  0 & 1 & $c+f$ \\
                  0 & 0 & 1
               \end{tabular}\right)\left(\begin{tabular}{@{}ccc@{}}
                  1 & $g$ & $h$ \\
                  0 & 1 & $i$ \\
                  0 & 0 & 1
               \end{tabular}\right) \\ &=
               \left(\begin{tabular}{@{}ccc@{}}
                  1 & $a+d+g$ & $af+ai+b+di+e+h$ \\
                  0 & 1 & $c+f+i$ \\
                  0 & 0 & 1
            \end{tabular}\right) \\ &=
            \left(\begin{tabular}{@{}ccc@{}}
               1 & $a$ & $b$ \\
               0 & 1 & $c$ \\
               0 & 0 & 1
            \end{tabular}\right)\left(\begin{tabular}{@{}ccc@{}}
               1 & $d+g$ & $di+e+h$ \\
               0 & 1 & $f+i$ \\
               0 & 0 & 1
            \end{tabular}\right) \\ &=
            X(YZ).
            \end{align*}
            Hence $H(F)$ is associative under matrix multiplication. Since
            $H(F)$ contains the 3 by 3 identity matrix, it follows by (a) and
            (b) that $H(F)$ is a group. Since we have $|F|$ choices for each of 
            $a$, $b$, and $c$, it follows that $|H(F)| = |F|^3$.
      \item \textbf{Solution.}
            \begin{center}
               \begin{tabular}{@{}|c|c|@{}} \hline
                  Matrix in $H(\Z/2\Z)$           & Order  \\ \hline            
                  $\left(\begin{tabular}{@{}ccc@{}}
                     1 & 0 & 0 \\
                     0 & 1 & 0 \\
                     0 & 0 & 1
                  \end{tabular}\right)$            & 1      \\ \hline           
                  $\left(\begin{tabular}{@{}ccc@{}}
                     1 & 0 & 0 \\
                     0 & 1 & 1 \\
                     0 & 0 & 1
                  \end{tabular}\right)$            & 2      \\ \hline           
                  $\left(\begin{tabular}{@{}ccc@{}}
                     1 & 0 & 1 \\
                     0 & 1 & 0 \\
                     0 & 0 & 1
                  \end{tabular}\right)$            & 2      \\ \hline           
                  $\left(\begin{tabular}{@{}ccc@{}}
                     1 & 0 & 1 \\
                     0 & 1 & 1 \\
                     0 & 0 & 1
                  \end{tabular}\right)$            & 2      \\ \hline           
                  $\left(\begin{tabular}{@{}ccc@{}}
                     1 & 1 & 0 \\
                     0 & 1 & 0 \\
                     0 & 0 & 1
                  \end{tabular}\right)$            & 2      \\ \hline           
                  $\left(\begin{tabular}{@{}ccc@{}}
                     1 & 1 & 0 \\
                     0 & 1 & 1 \\
                     0 & 0 & 1
                  \end{tabular}\right)$            & 4      \\ \hline         
                  $\left(\begin{tabular}{@{}ccc@{}}
                     1 & 1 & 1 \\
                     0 & 1 & 0 \\
                     0 & 0 & 1
                  \end{tabular}\right)$            & 2      \\ \hline         
                  $\left(\begin{tabular}{@{}ccc@{}}
                     1 & 1 & 1 \\
                     0 & 1 & 1 \\
                     0 & 0 & 1
                  \end{tabular}\right)$            & 4      \\ \hline
               \end{tabular}
            \end{center}
      \item \textbf{Proof.} Let $X = \left(\begin{tabular}{@{}ccc@{}}
               1 & $a$ & $b$ \\
               0 & 1 & $c$ \\
               0 & 0 & 1
            \end{tabular}\right)$ be a nonidentity matrix in $H(\R)$. Also let
            $P_n(i,j)$ denote the element in the $i$-th row and $j$-th column of
            $X^n$. So we have the following three cases:

            \textbf{Case 1.} $a \neq 0$. Notice that for every positive integer
            $n$, $P_n(1,2) = na \neq 0$, so that $X^n$ is not the identity
            element. Thus $|X| = \infty$.

            \textbf{Case 2.} $c \neq 0$. Similarly $P_n(2,3) = nc \neq 0$, so 
            that $X^n$ is not the identity element. Thus $|X| = \infty$.

            \textbf{Case 3.} $b \neq 0$ and $a = c = 0$. Also
            $P_n(1,3) = nb \neq 0$, so that $X^n$ is not the identity element. 
            Thus $|X| = \infty$.

            As we have shown above, every nonidentity element in $H(\R)$ is of
            infinite order. \qed
   \end{enumerate}
\end{enumerate}
