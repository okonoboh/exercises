\begin{enumerate}
   \item[]        Let $G$ be a group.
%%%%%%%%%%%%%%%%%%%%%%%%%%%%%%%%%%%%%1.1.1%%%%%%%%%%%%%%%%%%%%%%%%%%%%%%%%%%%%%%
   \item[1.1.1]   Determine which of the following binary operations are
                  associative:
                  \begin{enumerate}
                     \item the operation $*$ on $\Z$ defined by $a * b = a - b$.
                     \item the operation $*$ on $\R$ defined by
                           $a * b = a + b + ab$.
                     \item the operation $*$ on $\Q$ defined by
                           $\displaystyle a * b = \frac{a + b}{5}$.
                     \item the operation $*$ on $\Z \times \Z$ defined by
                           $(a, b) * (c, d) = (ad + bc, bd)$.
                     \item the operation $*$ on $\Q - \{0\}$ defined by
                           $\displaystyle a * b = \frac{a}{b}$.
                  \end{enumerate}
                  
      \textbf{Solution.}
   
      \begin{enumerate}
         \item The binary operation $*$ on $\Z$ is not associative because
               $$(0 * 0) * 1 = -1 \neq 1 = 0 * (0 * 1).$$
         \item We claim that $*$ is associative on $\R$.
      
               \textbf{Proof.} Let $a, b, c \in \R$. Then it follows that
               \begin{align*}
                  (a * b) * c &= (a + b + ab) * c \\
                     &= (a + b + ab + c) + (ac + bc + abc) \\
                     &= a + b + c + ab + bc + ac + abc \\
                     &= (a + b + c + bc) + a(b + c + bc) \\
                     &= a + (b * c) + a(b * c) \\
                     &= a * (b * c),
               \end{align*}
               so that our claim holds. \qed
         \item The binary operation $*$ on $\Q$ is not associative because
               $$(0 * 0) * 25 = 5 \neq 1 = 0 * (0 * 25).$$
         \item We claim that $*$ is associative on $\Z \times \Z$.
      
               \textbf{Proof.} Let $(a, b)$, $(c, d)$,
               $(r, s) \in \Z \times \Z$. Then it follows that
               \begin{align*}
                  (a, b) * ((c, d) * (r, s)) 
                     &= (a, b) * (cs + dr, ds) \\
                     &= (ads + bcs + bdr, bds) \\
                     &= ((ad + bc) \cdot s + bd \cdot r, bd \cdot s) \\
                     &= (ad + bc, bd) * (r, s) \\
                     &= ((a, b) * (c, d)) * (r, s),
               \end{align*}
               so that our claim holds. \qed
         \item The binary operation $*$ on $\Q - \{0\}$ is not associative
               because
               $$(4 * 1) * 2 = 2 \neq 8 = 4 * (1 * 2).$$
      \end{enumerate}
%%%%%%%%%%%%%%%%%%%%%%%%%%%%%%%%%%%%%1.1.2%%%%%%%%%%%%%%%%%%%%%%%%%%%%%%%%%%%%%%
   \item[1.1.2]   Decide which of the binary operations in the preceding
                  exercise are commutative.
                  
      \begin{enumerate}      
         \item The binary operation $*$ on $\Z$ is not commutative because
               $$1 * 0 = 1 \neq -1 = 0 * 1.$$
         \item The binary operation $*$ on $\R$ is commutative because addition
               and multiplication are commutative on $\R$.
         \item The binary operation $*$ on $\Q$ is commutative because addition
               is commutative on $\Q$.
         \item A quick check will show us that $*$ is commutative on
               $\Z \times \Z$. That is, for all $(a, b)$, $(c, d)$
               $\in \Z \times \Z$, we must have that
               \begin{align*}
                  (a, b) * (c, d) &= (ad + bc, bd) \\
                                          &= (cb + da, db) \\
                                          &= (c, d) * (a, b).
               \end{align*}
         \item The binary operation $*$ on $\Q - \{0\}$ is not commutative
               because
               $$1 * 2 = \frac{1}{2} \neq \frac{2}{1} = 2 * 1.$$
      \end{enumerate}
%%%%%%%%%%%%%%%%%%%%%%%%%%%%%%%%%%%%%1.1.3%%%%%%%%%%%%%%%%%%%%%%%%%%%%%%%%%%%%%%
   \item[1.1.3]   Prove that addition of residue classes in $\Z/n\Z$ is
                  associative (you may assume it is well defined).
                  
      \textbf{Proof.} Fix $n \in \Z^+$. Consider $\overline{a}$, $\overline{b}$,
      and $\overline{c}$ in $\Z/n\Z$. By Theorem 3, Pg. 9, we have that
      \begin{align*}
         \overline{a} + (\overline{b} + \overline{c})
            &= \overline{a} + \overline{b + c} \\
            &= \overline{a + b + c} \\
            &= \overline{a + b} + \overline{c} \\
            &= (\overline{a} + \overline{b}) + \overline{c},
      \end{align*}
      so that addition of residue classes in $\Z/n\Z$ is associative. \qed
%%%%%%%%%%%%%%%%%%%%%%%%%%%%%%%%%%%%%1.1.4%%%%%%%%%%%%%%%%%%%%%%%%%%%%%%%%%%%%%%
   \item[1.1.4]   Prove that multiplication of residue classes in $\Z/n\Z$ is
                  associative (you may assume it is well defined).
                  
      \textbf{Proof.} Fix $n \in \Z^+$. Consider $\overline{a}$, $\overline{b}$,
      and $\overline{c}$ in $\Z/n\Z$. By Theorem 3, Pg. 9, we have that
      \begin{align*}
         \overline{a} \cdot (\overline{b} \cdot \overline{c})
            &= \overline{a} \cdot \overline{bc} \\
            &= \overline{abc} \\
            &= \overline{ab} \cdot \overline{c} \\
            &= (\overline{a} \cdot \overline{b}) \cdot \overline{c},
      \end{align*}
      so that multiplication of residue classes in $\Z/n\Z$ is associative. \qed
%%%%%%%%%%%%%%%%%%%%%%%%%%%%%%%%%%%%%1.1.5%%%%%%%%%%%%%%%%%%%%%%%%%%%%%%%%%%%%%%
   \item[1.1.5]   Prove that for all $n > 1$ that $\Z/n\Z$ is not a group under
                  multiplication of residue classes.
                  
      \textbf{Proof.} Let $n$ be positive integer greater than 1. It follows
      that $\Z/n\Z$ is not a group under multiplication because $\overline{0}$
      has no multiplicative inverse. \qed
%%%%%%%%%%%%%%%%%%%%%%%%%%%%%%%%%%%%%1.1.6%%%%%%%%%%%%%%%%%%%%%%%%%%%%%%%%%%%%%%
   \item[1.1.6]   Determine which of the following sets are groups under
                  addition:
                  \begin{enumerate}
                     \item the set of rational numbers (including $0 = 0/1$) in
                           lowest terms whose denominators are odd.
                     \item the set of rational numbers (including $0 = 0/1$) in
                           lowest terms whose denominators are even.
                     \item the set of rational numbers of absolute value $< 1$.
                     \item the set of rational numbers of absolute value $\ge 1$
                           together with 0.
                     \item the set of rational numbers with denominators equal
                           to 1 or 2.
                     \item the set of rational numbers with denominators equal
                           to 1, 2, or 3.
                  \end{enumerate}

      \textbf{Solution.}

      \begin{enumerate}
         \item We claim that the set
               $$S = \left\{\frac{a}{b} \in \Q : b \text{ is odd} \text{ and }
                 \gcd(a, b) = 1\right\},$$
               is a group under addition.

               \textbf{Proof.} First we must show that $S$ is closed under 
               addition. Notice that $S$ is nonempty since it contains 7/5, so 
               let $r, s \in S$. By definition of $S$, we have that
               $r = a/b$ and $s = c/d$ for some integers $a$ and $c$, and odd 
               integers $b$ and $d$, such that
               $\gcd(a, b) = \gcd(c, d) = 1$. It follows that
               \begin{align*}
                  r + s &= \frac{a}{b} + \frac{c}{d} \\
                        &= \frac{ad + bc}{bd}.
               \end{align*}

               Since $b$ and $d$ are both odd, it must necessarily be the case 
               that $bd$ is also odd. In order words, $bd$ contains no factor of 
               2, so that if we reduce $r + s$ to its lowest term, the 
               denominator of this lowest term will still be odd. Hence
               $r + s \in S$, so that $S$ is closed under addition. To complete 
               the proof we must now show that $S$ satisfies the group axioms. 
               We observe that $0/1$ is the identity element in $S$. Also, it is 
               clear that for all $s \in S$, we have $-s \in S$, so that every 
               element of $S$ has an inverse under addition. Since
               $S \subseteq \Q$, and since $\Q$ is associative under addition, 
               it follows that $S$ is also associative under addition. Thus $S$ 
               satisfies the group axioms, so that $(S, +)$ is a group. \qed
         \item The set
               $$S = \left\{\frac{a}{b} \in \Q : b \text{ is even} \text{ and }
                 \gcd(a, b) = 1\right\},$$
               is not a group under addition because it is not closed. Indeed,
               for $3/14 \in S$, we have $3/14 + 3/14 = 3/7 \notin S$.
         \item The set
               $$S = \left\{\frac{a}{b} \in \Q :
                     \left|\frac{a}{b}\right| < 1\right\},$$
               is not a group under addition because it is not closed. Indeed,
               for $9/10 \in S$, we have $9/10 + 9/10 = 18/10 \notin S$.
         \item The set
               $$S = \left\{\frac{a}{b} \in \Q : a = 0 \text{ or }
                     \left|\frac{a}{b}\right| \ge 1\right\},$$
               is not a group under addition because it is not closed. Indeed,
               for $-11/10, 10/10 \in S$, we have
               $-11/10 + 10/10 = -1/10 \notin S$.
         \item We claim that the set
               $$S = \left\{\frac{a}{b} \in \Q : b = 1 \text{ or }
                 b = 2\right\},$$
               is a group under addition.

               \textbf{Proof.} It is clear that 0 is the identity for $S$ under
               addition, that $S$ is associative under addition (because
               $S \subset \Q$ and $\Q$ is associative under addition, and that
               the inverse of an element in $S$ is its additive inverse in $\Q$.
               So to complete the proof, we need only show that $S$ is closed
               under addition. Let $a/b, c/d \in \Q$. By observation, we note
               that $a/b + c/d$ must have a denominator of 1 or 2, so that it is
               in $S$. Thus $S$ is closed under addition. \qed
         \item The set
               $$S = \left\{\frac{a}{b} \in \Q : b \in {1, 2, 3} \right\},$$
               is not a group under addition because it is not closed. Indeed,
               for $1/2, 1/3 \in S$, we have $1/2 + 1/3 = 5/6 \notin S$.
      \end{enumerate}
%%%%%%%%%%%%%%%%%%%%%%%%%%%%%%%%%%%%%1.1.7%%%%%%%%%%%%%%%%%%%%%%%%%%%%%%%%%%%%%%
   \item[1.1.7]   Let $G = \{x \in \R : 0 \le x < 1\}$ and for $x, y \in G$ let
                  $x * y$ be the fractional part of $x + y$ (i.e.,
                  $x * y = x + y - [x + y]$ where $[a]$ is the greatest integer
                  less than or equal to $a$). Prove that $*$ is a well defined
                  binary operation on $G$ and that $G$ is an abelian group under
                  $*$ (called the \textit{real numbers mod }1).
                  
      \textbf{Proof.} The set $G$ is clearly non-empty, so consider
      $x, y, z \in G$. To show that $G$ is a group, we shall now prove that it 
      is well defined, associative, has an identity, and is closed under taking
      inverses.

      \textbf{Well Defined.} To show that $*$ is well defined is tantamount to
      showing that $G$ is closed under $*$.  First we want to show that
      $0 \le r - [r] < 1$ for each nonnegative real number $r$. If $r$ is an
      integer, then $r - [r] = 0$, so our claim follows. Now suppose $r$ is not
      an integer and write it as a decimal expansion: $r = r_0.r_1r_2r_3\ldots$,
      where $r_i$ is a nonnegative integer for all $i \ge 0$
      ($r \neq r_0.\overline{9}$ because $r_0.\overline{9} = r_0 + 1$ is an 
      integer). Then
      $$r - [r] = r_0.r_1r_2r_3\ldots - r_0 = 0.r_1r_2r_3\ldots,$$
      so that $0 \le r - [r] < 1$. By definition, we have that
      $0 \le x < 1$ and $0 \le y < 1$, so that $0 \le x + y < 2$. Thus if we
      substitute $x + y$ for $r$ above, we get
      $0 \le x + y  - [x + y] = x * y < 1$; i.e. $x * y \in G$, so that $G$ 
      is closed under $*$. Also we have that
      $$x * y = x + y - [x + y] = y + x - [y + x],$$
      so that $G$ is abelian.

      \begin{lemma} \label{1_1_7_1} Let $r \in \R$ and $a \in \Z$ such that
      $0 \le a \le r$. Then
         $$[r - a] = [r] - a.$$
      \end{lemma}

      \textbf{Proof.} The result is clear if $r$ is an integer. So suppose $r$ 
      is not an integer. Now write $r$ as a decimal expansion:
      $r = r_0.r_1r_2r_3\ldots$, (note that $r \neq r_0.\overline{9}$ because
      $r_0.\overline{9}$ is an integer). Since $r \ge a$, it follows that
      $r_0 \ge a$, so that
      $$[r - a] = [(r_0 - a).r_1r_2r_3\ldots] = r_0 - a =
      [r_0.r_1r_2r_3\ldots] - a = [r] - a.$$ \qed

      \textbf{Associativity.} We have that 
      \begin{align*}
         x * (y * z) &= x * (y + z - [y + z]) \\
              &= x + y + z - [y + z] - [x + y + z - [y + z]] \\
              &= x + y + z - [y + z] - ([x + y + z] - [y + z])
                 & \text{[Lemma }\ref{1_1_7_1}] \\
              &= x + y + z - [x + y + z] \\
              &= x + y + z - [x + y] - ([x + y + z] - [x + y]) \\
              &= x + y + z - [x + y] - [x + y + z - [x + y]]
                 & \text{[Lemma }\ref{1_1_7_1}] \\
          &= (x + y - [x + y]) * z \\
                     &=  (x * y) * z,
      \end{align*}
      so that $G$ is associative under $*$.


      \textbf{Identity.} We observe that $0 \in G$ is the identity element since
      $$x * 0 = x + 0 - [x + 0] = x - [x] = x - 0 = x.$$

      \textbf{Inverse.} Suppose $x \neq 0$, so that $0 < x < 1$, and thus
      $0 < 1 - x < 1$; that is $1 - x \in G$. It follows that
      $$x * (1 - x) = x + (1 - x) + [(x + (1 - x))] = 1 - 1 = 0,$$
      so that $1 - x$ is the inverse of $x \in G$, with $x \neq 0$. Clearly, the 
      inverse of 0 is 0. \\

      We can now conclude that $(G, *)$ is a group. \qed 
%%%%%%%%%%%%%%%%%%%%%%%%%%%%%%%%%%%%%1.1.8%%%%%%%%%%%%%%%%%%%%%%%%%%%%%%%%%%%%%%
   \item[1.1.8]   Let $G = \{z \in \C : z^n = 1 \text{ for some } n \in \Z^+\}$.
                  \begin{enumerate}
                     \item Prove that $G$ is a group under multiplication
                           (called the group of \textit{roots of unity} in
                           $\C$).
                     \item Prove that $G$ is not a group under addition.
                  \end{enumerate}
                  
      \textbf{Proof.}
      
      \begin{enumerate}
         \item Since $1^1 = 1$, it follows that $1 \in G$, so that $G$ is
               nonempty. So let $x, y\in G$. 
               
               \textbf{Closure.} By definition, there exist positive integers
               $m$ and $n$ such that $x^m = y^n = 1$. Thus
               $(xy)^{mn} = (x^m)^n(y^n)^m = 1^n1^m = 1$. This says that $G$ is
               closed under multiplication.
               
               \textbf{Associativity.} Since $\C$ is associative under
               multiplication and since $G \subseteq \C$, it follows that $G$ is
               associative under multiplication.
               
               \textbf{Identity.} Since
               $$1 \cdot x = x \cdot 1 = x,$$
               it follows that $1$ is the identity of $G$.
               
               \textbf{Inverse.} Notice that since
               $(x^{m - 1})^m = (x^m)^{m - 1} = 1$, we must have that
               $x^{m - 1} \in G$. Thus we have $x^{m - 1}x = x^m = 1$; i.e., the
               inverse of $x$ is $x^{m - 1}$.
               
               We have thus shown that $G$ is a group under multiplication. \qed
         \item $G$ is not a group under addition because it is not closed under
               addition. In particular, we have $1 \in G$, but
               $1 + 1 = 2 \notin G$ because $2^n \neq 1$ for any positive
               integer.
      \end{enumerate}
%%%%%%%%%%%%%%%%%%%%%%%%%%%%%%%%%%%%%1.1.9%%%%%%%%%%%%%%%%%%%%%%%%%%%%%%%%%%%%%%
   \item[1.1.9]   Let $G = \{a + b\sqrt{2} \in \R : a, b \in \Q\}$.
                  \begin{enumerate}
                     \item Prove that $G$ is a group under addition.
                     \item Prove that the nonzero elements of $G$ are a group 
                           under multiplication. [``Rationalize the
                           denominators" to find multiplicative inverse.]
                  \end{enumerate}
                  
      \textbf{Proof.}
      
      \begin{enumerate}
         \item \textbf{Closure.} $G$ is clearly nonempty, so let $x, y \in G$.
               By definition of $G$, it follows that $x = a_1 + b_1\sqrt{2}$ and
               $y = a_2 + b_2\sqrt{2}$ for some rational numbers $a_1$, $b_1$,
               $a_2$, and $b_2$. Thus
               $$x + y = (a_1 + a_2) + (b_1 + b_2)\sqrt{2} \in G,$$
               because $\Q$ is closed under addition; thus $G$ is also closed 
               under addition.
               
               \textbf{Associativity.} Since $\R$ is associative under addition
               and since $G \subseteq \R$, it follows that $G$ is associative
               under addition.
               
               \textbf{Identity.} The identity of $G$ is 0.
               
               \textbf{Inverse.} For an element $x = a_1 + b_1\sqrt{2} \in G$,
               the additive inverse of $x$ is $-a_1 - b_1\sqrt{2} \in G$.
               
               We have thus shown that $G$ is a group under addition. \qed
         \item Let $G^{\times}$ denote the set of nonzero elements of $G$.
         
               \textbf{Closure.} Let $x, y \in G^{\times}$. By definition of
               $G$, it follows that $x = a_1 + b_1\sqrt{2}$ and
               $y = a_2 + b_2\sqrt{2}$ for some rational numbers $a_1$, $b_1$,
               $a_2$, and $b_2$, with $a_1$ and $b_1$ not both zero and $a_2$
               and $b_2$ not both zero. Thus
               $$xy = (a_1a_2 + 2b_1b_2) + (a_1b_2 + a_2b_1)\sqrt{2} \in G,$$
               by the closure of $\Q$ under addition and multiplication. Now
               since $xy$ is a real number and neither $x$ nor $y$ is zero, it 
               must be the case that $xy \neq 0$; that is,
               $xy \in  G^{\times}$. Conclude that $G^{\times}$ is closed under 
               multiplication.
               
               \textbf{Associativity.} Since $\R$ is associative under
               multiplication and since $G^{\times} \subseteq \R$, it follows
               that $G^{\times}$ is associative under multiplication.
               
               \textbf{Identity.} The element $1 = 1 + 0\sqrt{2} \in G^{\times}$
               is the identity of $G^{\times}$.
               
               \textbf{Inverse.} Let $x = a_1 + b_1\sqrt{2} \in G^{\times}$.
               Since $x \neq 0$, the real number $1/x$ exists, and we have that
               $$\frac{1}{x} = \frac{1}{a_1 + b_1\sqrt{2}}
                 \frac{a_1 - b_1\sqrt{2}}{a_1 - b_1\sqrt{2}} =
                 \left(\frac{a_1}{{a_1}^2 - 2{b_1}^2} -
                 \frac{b_1}{{a_1}^2 - 2{b_1}^2}\sqrt{2}\right) \in G^{\times}.
               $$
               
               Since $1/x \in G^{\times}$ and since $x \cdot 1/x = 1$, we have
               that $1/x$ is the multiplicative inverse of $x$.
               
               We have thus shown that $G^{\times}$ is a group under
               multiplication. \qed
      \end{enumerate}
%%%%%%%%%%%%%%%%%%%%%%%%%%%%%%%%%%%%%1.1.10%%%%%%%%%%%%%%%%%%%%%%%%%%%%%%%%%%%%%
   \item[1.1.10]  Prove that a finite group is abelian if and only if its group
                  table is a symmetric matrix.
                  
      \textbf{Proof.} Let $G$ be a group such that $|G| = n \in \Z^+$, and let
      $(a_{ij})$ denote the matrix of the group table of $G$. Since $G$ is
      finite, we can enumerate the elements of $G$ like so:
      $$G = \{g_1, g_2, \ldots, g_n\}.$$      
      $(\Leftarrow)$ Suppose that $(a_{ij})$ is a symmetric matrix. Let
      $a, b \in G$. Then we have that $a = g_r$ and $b = g_s$ for some
      $r, s \in \{1, 2, \ldots, n\}$. Since $(a_{ij})$ is symmetric, we must
      have that
      $$ab = g_rg_s = a_{rs} = a_{sr} = g_sg_r = ba,$$
      so that $G$ is abelian.
      
      $(\Rightarrow)$ Now suppose that $G$ is abelian. Consider
      $a_{rs} \in (a_{ij})$. It follows that
      $$a_{rs} = g_rg_s = g_sg_r = a_{sr},$$
      so that $(a_{ij})$ is symmetric. \qed
%%%%%%%%%%%%%%%%%%%%%%%%%%%%%%%%%%%%%1.1.11%%%%%%%%%%%%%%%%%%%%%%%%%%%%%%%%%%%%%
   \item[1.1.11]  Find the orders of each element of the additive group
                  $\Z/12\Z$.
                  
      \textbf{Solution.} The orders of the elements $\overline{0}$,
      $\overline{1}$, $\overline{2}$, $\overline{3}$, $\overline{4}$,
      $\overline{5}$, $\overline{6}$, $\overline{7}$, $\overline{8}$,
      $\overline{9}$, $\overline{10}$, and $\overline{11}$ in $\Z/12\Z$ are
      1, 12, 6, 4, 3, 12, 2, 12, 3, 4, 6, and 12.
%%%%%%%%%%%%%%%%%%%%%%%%%%%%%%%%%%%%%1.1.12%%%%%%%%%%%%%%%%%%%%%%%%%%%%%%%%%%%%%
   \item[1.1.12]  Find the orders of the following elements of the
                  multiplicative group $(\Z/12\Z)^\times: \overline{1},
                  \overline{-1}, \overline{5}, \overline{7}, \overline{-7}, 
                  \overline{13}$.
                  
      \textbf{Solution.} The orders of the elements $\overline{1}$,
      $\overline{-1}$, $\overline{5}$, $\overline{7}$, $\overline{-7}$,
      $\overline{13}$ in $(\Z/12\Z)^\times$ are 1, 2, 2, 2, 2, and 1.
%%%%%%%%%%%%%%%%%%%%%%%%%%%%%%%%%%%%%1.1.13%%%%%%%%%%%%%%%%%%%%%%%%%%%%%%%%%%%%%
   \item[1.1.13]  Find the orders of the following elements of the additive
                  group $\Z/36\Z: \overline{1}, \overline{2}, \overline{6}, 
                  \overline{9}, \overline{10}, \overline{12}, \overline{-1}, 
                  \overline{-10}, \overline{-18}$.
                  
      \textbf{Solution.} The orders of the elements $\overline{1}$,
      $\overline{2}$, $\overline{6}$, $\overline{9}$, $\overline{10}$,
      $\overline{12}$, $\overline{-1}$, $\overline{-10}$, and $\overline{-18}$
      in $\Z/36\Z$ are 1, 18, 6, 4, 18, 3, 36, 18, and 2.
%%%%%%%%%%%%%%%%%%%%%%%%%%%%%%%%%%%%%1.1.14%%%%%%%%%%%%%%%%%%%%%%%%%%%%%%%%%%%%%
   \item[1.1.14]  Find the orders of the following elements of the
                  multiplicative group $(\Z/36\Z)^\times: \overline{1},
                  \overline{-1}, \overline{5}, \overline{13}, \overline{-13},
                  \overline{17}$.
                  
      \textbf{Solution.} The orders of the elements $\overline{1}$,
      $\overline{-1}$, $\overline{5}$, $\overline{13}$, $\overline{-13}$,
      $\overline{17}$ in $(\Z/36\Z)^\times$ are 1, 2, 6, 3, 6, and 2.
%%%%%%%%%%%%%%%%%%%%%%%%%%%%%%%%%%%%%1.1.15%%%%%%%%%%%%%%%%%%%%%%%%%%%%%%%%%%%%%
   \item[1.1.15]  Prove that $(a_1a_2\cdots a_n)^{-1} =
                  {a_n}^{-1}{a_{n-1}}^{-1}\cdots {a_1}^{-1}$ for all
                  $a_1, a_2, \ldots, a_n \in G$.
                  
      \textbf{Proof.} We shall proceed by induction on $n$. The statement is
      trivial for $n = 1$. So assume that it also holds for some positive
      integer $k$. Let $b = a_1a_2\cdots a_k$. It then follows that
      \begin{align*}
         (a_1a_2\cdots a_ka_{k+1})^{-1} &= (b \cdot a_{k+1})^{-1} \\
            &= {a_{k+1}}^{-1}b^{-1} &[\text{By Proposition 1 (4)}] \\
            &= {a_{k+1}}^{-1}{a_k}^{-1}\cdots {a_1}^{-1}.
                  &[\text{Inductive hypothesis}]
      \end{align*}
      That is, our statement holds for $k + 1$, so that, by the Principle of
      Mathematical Induction, it holds for each positive integer $n$. \qed
%%%%%%%%%%%%%%%%%%%%%%%%%%%%%%%%%%%%%1.1.16%%%%%%%%%%%%%%%%%%%%%%%%%%%%%%%%%%%%%
   \item[1.1.16]  Let $x$ be an element of $G$. Prove that $x^2 = 1$ if and only
                  if $|x|$ is either 1 or 2.
                  
      \textbf{Proof.}
      
      $(\Leftarrow)$ Suppose that $x^2 = 1$. Then by definition,
      $|x| \le 2$. That is, $|x| = 1$ or $|x| = 2$.
      
      $(\Rightarrow)$ If $|x| = 1$, then $x = 1$, so that $x^2 = 1^2 = 1$; now
      if $|x| = 2$, then $x^2 = 1$. \qed
%%%%%%%%%%%%%%%%%%%%%%%%%%%%%%%%%%%%%1.1.17%%%%%%%%%%%%%%%%%%%%%%%%%%%%%%%%%%%%%
   \item[1.1.17]  Let $x$ be an element of $G$. Prove that if $|x| = n$ for some
                  positive integer $n$ then $x^{-1} = x^{n-1}$.
                  
      \textbf{Proof.} Suppose that $|x| = n \in \Z^+$. By Exercise 1.1.19(d), it
      follows that $x^{n-1}x^1 = x^{n-1+1} = x^n = 1$, so that
      $x^{-1} = x^{n-1}$. \qed
%%%%%%%%%%%%%%%%%%%%%%%%%%%%%%%%%%%%%1.1.18%%%%%%%%%%%%%%%%%%%%%%%%%%%%%%%%%%%%%
   \item[1.1.18]  Let $x$ and $y$ be elements of $G$. Prove that $xy = yx$ if
                  and only if $y^{-1}xy =x$ if and only if $x^{-1}y^{-1}xy = 1$.
                  
      \textbf{Proof.} First assume that $xy = yx$. We then have that
      $yx = xy = 1xy = yy^{-1}xy$, so that $x = y^{-1}xy$ by left cancellation.
      Now assume that $y^{-1}xy = x$. Thus
      $x1 = x = y^{-1}xy = 1y^{-1}xy = xx^{-1}y^{-1}xy$, so that
      $1 = x^{-1}y^{-1}xy$ by left cancellation. Finally assume that
      $x^{-1}y^{-1}xy = 1$. Multiplying on the left by $yx$ will yield the
      equation $xy = yx$. \qed
%%%%%%%%%%%%%%%%%%%%%%%%%%%%%%%%%%%%%1.1.19%%%%%%%%%%%%%%%%%%%%%%%%%%%%%%%%%%%%%
   \item[1.1.19]  Let $x \in G$ and let $a, b \in \Z^+$.
                  \begin{enumerate}
                     \item Prove that $x^{a+b} = x^ax^b$.
                     \item Prove that $(x^a)^b = x^{ab}$.
                     \item Prove that $(x^a)^{-1} = x^{-a}$.
                     \item Establish part (a) for arbitrary integers $a$ and $b$
                           (positive, negative or zero).
                     \item Establish part (b) for arbitrary integers $a$ and $b$
                           (positive, negative or zero).
                  \end{enumerate}
               
      \textbf{Proof.} We recursively define $x^0 = 1$, $x^1 = x$,
      $x^{n+1} = x^nx$, and $x^{-n} = (x^{-1})^n$, for $n \in \Z^+$.
      
      \begin{enumerate}
         \item It suffices to show that $x^{a+n} = x^ax^n$ for each
               $n \in \Z^+$. Proceed by induction on $n$. The base case,
               $n = 1$, follows by definition. So asssume $x^{a+k} = x^ax^k$,for 
               some positive integer $k$. Thus
               \begin{align*}
                  x^{a+(k+1)} &= x^{(a+k)+1} \\
                          &= x^{a+k}x &\text{[Definition]} \\
                          &= (x^ax^k)x &\text{[Inductive Hypothesis]} \\
                          &= x^a(x^kx) &\text{[Associativity]}\\
                          &= x^ax^{k+1}. &\text{[Definition]}
               \end{align*}
               Our statement follows by induction. Particularly,
               $x^{a+b} = x^ax^b$. \qed
         \item Similarly, we will show that $(x^a)^n = x^{an}$ for each
               $n \in \Z^+$ by induction on $n$. The base case ($n=1$) holds
               because $(x^a)^1 = x^a = x^{a1}$. So asssume
               $(x^a)^k = x^{ak}$ for some positive integer $k$. Thus
               \begin{align*}
                  (x^a)^{k+1} &= (x^a)^k(x^a) &\text{[Definition]} \\
                          &= x^{ak}x^a &\text{[Inductive Hypothesis]} \\
                          &= x^{ak+a} &\text{[(a)]} \\
                          &= x^{a(k+1)}.
               \end{align*} 
               Our statement follows by induction. Particularly,
               $(x^a)^b = x^{ab}$. \qed
         \item We will use induction on $n$ to show that $x^{-n} = (x^n)^{-1}$.
               The base case($n = 1$) follows because $x^{-1} = (x^1)^{-1}$, so
               asssume $x^{-k} = (x^k)^{-1}$ for some positive integer $k$. Now
               \begin{align*}
                  x^{-(k+1)} &= (x^{-1})^{k+1} &\text{[Definition]} \\
                          &= (x^{-1})^kx^{-1} &\text{[Definition]} \\
                          &= (x^k)^{-1}x^{-1} &\text{[Inductive Hypothesis]} \\
                          &= (xx^k)^{-1} &\text{[Proposition 1(4)]}\\
                          &= (x^{k+1})^{-1}. &\text{[(a)]}
               \end{align*} \qed
         \item We now want to prove that
               \begin{equation} \label{1_1_19_1}
                  x^{a+b} = x^ax^b,
               \end{equation}
               for all $a, b \in \Z$. So let $a$ and $b$ be arbitrary integers.
               We showed in (a) that \eqref{1_1_19_1} holds for $a, b \in \Z^+$.
               Now suppose $a = 0$ or $b = 0$. Further suppose without loss that
               $a = 0$, so that
               $$x^{a+b} = x^{0+b} = x^b = 1x^b = x^0x^b = x^ax^b.$$
               The other possibilities can be proven if we assume that $b$ is
               negative and place no restriction on $a$. (for the case where
               $b > 0$ and $a < 0$, we swap the roles of $a$ and $b$ in the
               proof below.) So assume without loss
               that $b$ is negative. First, we want to show that
               \begin{equation} \label{1_1_19_2}
                  x^{m-1} = x^mx^{-1},
               \end{equation}
               for each integer $m$. From our recursive definition, we have that
               $$x^m = x^{(m-1) + 1} = x^{m-1}x,$$
               so that $x^{m-1} = x^mx^{-1}$ if $m \ge 2$. The equality in
               \eqref{1_1_19_2} trivially holds when $m = 0$ and $m = 1$, so
               that it holds for $m \ge 0$. Now if $m$ is negative, so that
               $m = -r$ for some positive integer $r$, it follows that
               \begin{align*}
                  x^{m-1} = x^{-r-1} &= x^{-(r+1)} \\
                          &= (x^{-1})^{r+1} &\text{[Definition]} \\
                          &= (x^{-1})^rx^{-1} &\text{[Definition]} \\
                          &= x^{-r}x^{-1} &\text{[Definition]} \\
                          &= x^mx^{-1}.
               \end{align*}

               Thus \eqref{1_1_19_2} holds for each integer $m$. Finally, we
               will show by induction on $n$ that
               $x^{a - n} = x^ax^{-n}$. Equality \eqref{1_1_19_2} says that our 
               statement holds for $n = 1$. So assume that
               $x^{a - k} = x^ax^{-k}$ for some positive integer $k$. Thus
               \begin{align*}
                  x^{a - (k+1)} &= x^{(a - k) - 1}  \\
                             &= x^{a-k}x^{-1} &\text{[\eqref{1_1_19_2}]} \\
                             &= x^ax^{-k}x^{-1} &\text{[Inductive Hypothesis]}\\
                             &= x^a(x^{-1})^kx^{-1}&\text{[Definition]} \\
                             &= x^a(x^{-1})^{k+1} &\text{[Definition]} \\
                             &= x^ax^{-(k+1)}, &\text{[Definition]} \\
               \end{align*}
               so that $x^{a - n} = x^ax^{-n}$ holds for each positive integer
               $n$ by induction; particularly,
               $$x^{a + b} = x^{a - s} = x^ax^{-s} = x^ax^b,$$
               where $s = -b$ (i.e., $s$ is positive). We have thus shown that
               \eqref{1_1_19_1} holds for all integers $a$ and $b$. \qed
         \item It is clear that part (b) holds if $a$ is 0 or $b$ is 0, so let
               us complete the proof for arbritrary integers $a$ and $b$.

               \textbf{Case 1.} \textit{$a$ is positive and $b$ is negative}. 
               Let $m = -b$ be a positive integer. Hence
               \begin{align*}
                  (x^a)^b &= (x^a)^{-m} \\
                          &= ((x^a)^{-1})^m &[\text{Definition}] \\
                          &= (x^{-a})^m &[\text{Part (c)}] \\
                          &= ((x^{-1})^a)^m &[\text{Definition}] \\
                          &= (x^{-1})^{am} &[\text{Part (b)}] \\
                          &= x^{-am} &[\text{Definition}] \\
                          &= x^{ab}.
               \end{align*}

               \textbf{Case 2.} \textit{$a$ and $b$ are negative}. Let $s = -a$ 
               be a positive integer. Thus
               \begin{align*}
                  (x^a)^b &= (x^{-s})^b \\
                          &= ((x^{-1})^s)^b &[\text{Definition}] \\
                          &= (x^{-1})^{sb} &[\text{Case 1}] \\
                          &= (x^{-1})^{-(ab)}  \\
                          &= ((x^{-1})^{-1})^{ab}. &[\text{Definition}] \\
                          &= x^{ab}. &[\text{Proposition 1 (3)}]
               \end{align*}

               \textbf{Case 3.} \textit{$a$ is negative and $b$ is positive}. 
               Let $s = -a$ be a positive integer. Thus
               Thus
               \begin{align*}
                  (x^a)^b &= (x^{-s})^b \\
                          &= ((x^{-1})^s)^b &[\text{Definition}] \\
                          &= (x^{-1})^{sb} &[\text{Case 1}] \\
                          &= x^{-sb} &[\text{Definition}] \\
                          &= x^{ab}.
               \end{align*}               

               \textbf{Case 4.} $b = 0$. It follows immediately that
               $$(x^a)^b = (x^a)^0 = 1 = x^0 = x^{a\cdot 0} = x^{ab}.$$

               \textbf{Case 5.} $a = 0$. It follows immediately that
               $$(x^a)^b = (x^0)^b = 1^b = 1 = x^0 = x^{0\cdot b} = x^{ab}.$$

               Combining these results with part (a), we can conclude that
               $(x^a)^b = x^{ab}$ holds for all integers $a$ and $b$ and
               $x \in G$. \qed
      \end{enumerate}
%%%%%%%%%%%%%%%%%%%%%%%%%%%%%%%%%%%%%1.1.20%%%%%%%%%%%%%%%%%%%%%%%%%%%%%%%%%%%%%
   \item[1.1.20]  For $x$ an element in $G$ show that $x$ and $x^{-1}$ have the
                  same order.

      \textbf{Proof.}

      \textbf{Case 1.} \textit{$|x| = n \in \Z^+$}. Since
      $(x^{-1})^n = (x^n)^{-1} = 1^{-1} = 1$, it follows that $|x^{-1}| \le n$,
      so suppose to the contrary that $|x^{-1}| = m < n$. Then we have that
      $$x^m = ((x^{-1})^{-1})^m = ((x^{-1})^m)^{-1} = 1^{-1} = 1,$$
      a contradiction, so that $|x^{-1}| = n = |x|$.

      \textbf{Case 2.} \textit{$|x| = \infty$}. Suppose to the contrary that
      $|x^{-1}| = n \in \Z^+$. As we argued in Case 1, it must be the case that
      $x^n = 1$, a contradiction. Thus $|x| = \infty = |x^{-1}|$. \qed
%%%%%%%%%%%%%%%%%%%%%%%%%%%%%%%%%%%%%1.1.21%%%%%%%%%%%%%%%%%%%%%%%%%%%%%%%%%%%%%
   \item[1.1.21]  Let $G$ be a finite group and let $x$ be an element of $G$ of
                  order $n$. Prove that if $n$ is odd, then $x = (x^2)^k$ for
                  some $k$.

      \textbf{Proof.} Suppose that $n$ is odd. We can then write $n = 2k + 1$
      for some nonnegative integer $k$. Since $|x| = n$, we have that
      $xx^{2k} = x^{2k+1} = 1 = x^{-2k}x^{2k}$, so that by right cancellation,
      we can conclude that $x = x^{-2k} = (x^2)^{-k}$. \qed
%%%%%%%%%%%%%%%%%%%%%%%%%%%%%%%%%%%%%1.1.22%%%%%%%%%%%%%%%%%%%%%%%%%%%%%%%%%%%%%
   \item[1.1.22]  If $x$ and $g$ are elements of the group $G$, prove that
                  $|x| = |g^{-1}xg|$. Deduce that $|ab| = |ba|$ for all
                  $a, b \in G$.

      \begin{lemma} \label{1_1_22_1} \textit{Let $x$ and $g$ be members of a 
                  group $G$, and let $n$ be a positive integer, then it follows 
                  that $(g^{-1}xg)^n = g^{-1}x^ng$.}
      \end{lemma}

      \textbf{Proof.} Let $x, g \in G$. We shall show by induction that the 
      equation
      \begin{equation}
         (g^{-1}xg)^n = g^{-1}x^ng \label{1_1_22_2}
      \end{equation}
      holds for every positive integer $n$. It is clear that \eqref{1_1_22_2} 
      holds for $n = 1$. So assume that it also holds for some positive integer 
      $k$. So we must now show that the equation also holds for $k + 1$. Thus
      \begin{align*}
         (g^{-1}xg)^{k+1} &= (g^{-1}xg)^kg^{-1}xg &[\text{Execise 1.1.19}] \\
                     &= g^{-1}x^kgg^{-1}xg &[\text{Inductive hypothesis}] \\
                     &= g^{-1}x^kxg \\
                     &= g^{-1}x^{k+1}g,
      \end{align*}
      so that \eqref{1_1_22_2} holds for $k+1$. Hence by the Principle 
      of Mathematical Induction, equation \eqref{1_1_22_1} holds for every 
      positive integer $n$. Now we show $|x| = |g^{-1}xg|$.

      \textbf{Case 1.} \textit{$|x| = n \in \Z^+$}. By Lemma \ref{1_1_22_1}, it 
      follows that $(g^{-1}xg)^n = g^{-1}x^ng = g^{-1}g =1$, so that
      $|g^{-1}xg| \le n$, so suppose to the contrary that $|g^{-1}xg| = m < n$. 
      Then we have that
      $$g^{-1}1g = 1 = (g^{-1}xg)^m = g^{-1}x^mg,$$
      so that $x^m = 1$ by left and right cancellations, a contradiction; thus,  
      $|g^{-1}xg| = n = |x|$.

      \textbf{Case 2.} \textit{$|x| = \infty$}. Suppose to the contrary that
      $|g^{-1}xg| = n \in \Z^+$. As we argued in Case 1, it must then be the 
      case that $x^n = 1$, a contradiction. Thus $|x| = \infty = |g^{-1}xg|$.

      Now consider $a, b \in G$. Set $x = ab$ and $g = a$. Since 
      $|x| = |g^{-1}xg|$, it follows that $|ab| = |a^{-1}aba| = |ba|$. \qed
%%%%%%%%%%%%%%%%%%%%%%%%%%%%%%%%%%%%%1.1.23%%%%%%%%%%%%%%%%%%%%%%%%%%%%%%%%%%%%%
   \item[1.1.23]  Suppose $x \in G$ and $|x| = n < \infty$. If $n = st$ for some
                  positive integers $s$ and $t$, prove that $|x^s| = t$.

      \textbf{Proof.} Suppose $n = st$ for some positive integers $s$ and $t$.
      By supposition, we have that $1 = x^n = x^{st} = (x^s)^t$; i.e.,
      $|x^s| \le t$. Suppose to the contrary that $|x^s| = m < t$. Then we have
      that $1 = (x^s)^m = x^{sm}$. Since $0 < m < t$, it follows that
      $0 < sm < st = n$. However $|x| = n$ and we just showed that $x^{sm} = 1$, 
      so that we have a contradiction. Hence we can conclude that $|x^s| = |t|$.
      \qed
%%%%%%%%%%%%%%%%%%%%%%%%%%%%%%%%%%%%%1.1.24%%%%%%%%%%%%%%%%%%%%%%%%%%%%%%%%%%%%%
   \item[1.1.24]  If $a$ and $b$ are \textit{commuting} elements of $G$, prove 
                  that $(ab)^n = a^nb^n$ for all $n \in \Z$. [Do this by 
                  induction for positive $n$ first.]

      \textbf{Proof.} Let $R(n)$ be the statement that $(ab)^n = a^nb^n$, for
      commuting elements $a$ and $b$.
               
      We now want to show using induction that $R(n)$ holds for every positive 
      integer $n$. It is clear that $R(1)$ is true. So suppose that $R(k)$ is 
      true for some positive integer $k$. We must now show that $R(k + 1)$ is 
      also true. Now we have that
      \begin{align*}
         (ab)^{k+1} &= (ab)^k(ab)^1 &[\text{Exercise 1.1.19}] \\
                    &= a^kb^k(ab)^1 &[\text{Since }R(k) \text{ is true}] \\
                    &= a^kb^k(ba)^1 &[ab = ba] \\
                    &= a^kb^kba \\
                    &= a^kb^{k+1}a \\
                    &= a^kab^{k+1} &[\text{$a$ commutes with $b$}] \\
                    &= a^{k+1}b^{k+1}, \\
      \end{align*}
      so that $R(k + 1)$ holds. It follows by the Principle of Mathematical 
      Induction that $R(n)$ holds for every positive integer $n$. By inpsection 
      we can see that $R(0)$ also holds. To complete the proof, we must now show 
      that $(ab)^{m} = a^mb^m$, where $m$ is a negative integer. First we notice 
      that
      \begin{equation}
         a^{-1}b^{-1} = (ba)^{-1} = (ab)^{-1} = b^{-1}a^{-1},
         \label{1_1_24_1}
      \end{equation}
      so that $a^{-1}$ and $b^{-1}$ are commuting elements. Now let $r = -m$ be
      a positive integer; thus
      \begin{align*}
         (ab)^m &= (ab)^{-r} \\
                &= ((ab)^{-1})^r &[\text{Definition}] \\
                &= (a^{-1}b^{-1})^r &[\eqref{1_1_24_1}] \\
                &= (a^{-1})^r(b^{-1})^r &[\text{$R(r)$ holds}] \\
                &= a^{-r}b^{-r} \\
                &= a^mb^m,
      \end{align*}
      as desired. \qed
%%%%%%%%%%%%%%%%%%%%%%%%%%%%%%%%%%%%%1.1.25%%%%%%%%%%%%%%%%%%%%%%%%%%%%%%%%%%%%%
   \item[1.1.25]  Prove that if $x^2 = 1$ for all $x \in G$ then $G$ is abelian.

      \textbf{Proof.} Let $G$ be a group. Suppose that $x^2 = 1$ for all
      $x \in G$. We want to show that $G$ is abelian; that is, we want to show 
      that $xy = yx$ for all $x, y \in G$. So let $x, y \in G$. By hypothesis, 
      we have that $x^2 = e$, $y^2 = e$, and $(xy)^2 = e$, so that according to 
      Proposition 2, we must have that $x = x^{-1}$, $y = y^{-1}$, and
      $xy = (xy)^{-1}$. Thus
      \begin{align*}
         xy &= (xy)^{-1}      &[\text{By Hypothesis}] \\
            &= y^{-1}x^{-1}   &[\text{Proposition 1}] \\
            &= yx.
      \end{align*}
      Thus $G$ is abelian. \qed
%%%%%%%%%%%%%%%%%%%%%%%%%%%%%%%%%%%%%1.1.26%%%%%%%%%%%%%%%%%%%%%%%%%%%%%%%%%%%%%
   \item[1.1.26]  Assume $H$ is a nonempty subset of $(G, *)$ which is closed 
                  under the binary operation on $G$ and is closed under
                  inverses, i.e., for all $h$ and
                  $k \in H$, $hk$ and $h^{-1} \in H$. Prove that $H$ is a group 
                  under the operation $*$ restricted to $H$ (such a subset $H$
                  is called a subgroup of $G$).

      \textbf{Proof.} We know that $H$ is closed under $*$ and under inverses, 
      so it suffices to show that $*$ is associative on $H$ and that $H$ has an 
      identity under $*$. The associativity of $H$ under $*$ follows because $H$ 
      is a subset of $G$ and $G$ is associative under $*$. Since $H$ is nonempty
      we pick an $h \in H$. Then by hypothesis, we have that
      $1 = hh^{-1} \in H$, so that $H$ contains the identity. (Note that
      $hh^{-1} = h^{-1}h = 1$ and $h1 = 1h = h$ because these equalities hold in
      $G$.) Thus $H$ is a subgroup of $G$. \qed
%%%%%%%%%%%%%%%%%%%%%%%%%%%%%%%%%%%%%1.1.27%%%%%%%%%%%%%%%%%%%%%%%%%%%%%%%%%%%%%
   \item[1.1.27]  Prove that if $x$ is an element of the group $G$ then
                  $\{x^n : n \in \Z\}$ is a subgroup of $G$ (called the
                  \textit{cyclic subgroup} of $G$ generated by $x$).

      \textbf{Proof.} Consider the set
      $$H = \{x^n : n \in \Z\}.$$
      $H$ is nonempty because it contains $1 = x^0$. So let $h_1, h_2 \in H$.
      Thus we have $h_1 = x^a$ and $h_2 = x^b$ for some integers $a$ and $b$, so
      that $h_1h_2 = x^ax^b = x^{a+b} \in H$; in other words, $H$ is closed
      under the operation of $G$. Since $h_1^{-1} = (x^a)^{-1} = x^{-a} \in H$, 
      it follows that $H$ is also closed under inverses, so that $H$ is a
      subgroup of $G$ by Exercise 1.1.26.
%%%%%%%%%%%%%%%%%%%%%%%%%%%%%%%%%%%%%1.1.28%%%%%%%%%%%%%%%%%%%%%%%%%%%%%%%%%%%%%
   \item[1.1.28]  Let $(A, *)$ and $(B, \diamond)$ be groups and let
                  $A \times B$ be their direct product (as defined in Example
                  6). Verify all the group axioms for $A \times B$.
                  \begin{enumerate}
                     \item prove that the associative law holds: for all
                           $(a_i, b_i) \in A \times B, i = 1, 2, 3$
                           $$(a_1, b_1)[(a_2, b_2)(a_3, b_3)] =
                            [(a_1, b_1)(a_2, b_2)](a_3, b_3),$$
                     \item prove that (1, 1) is the identity of $A \times B$,
                           and
                     \item prove that the inverse of $(a, b)$ is
                           $(a^{-1}, b^{-1})$.
                  \end{enumerate}

      \textbf{Proof.} Let $(a_1, b_1)$, $(a_2, b_2)$, and
      $(a_3, b_3) \in A \times B$.

      \begin{enumerate}
         \item The set $A \times B$ is associative under the component wise
               operations of $A$ and $B$ because
               \begin{align*}
                  (a_1, b_1)[(a_2, b_2)(a_3, b_3)]
                     &= (a_1, b_1)(a_2a_3, b_2b_3) \\
                     &= (a_1a_2a_3, b_1b_2b_3) \\
                     &= [(a_1a_2)a_3, (b_1b_2)b_3] &[\text{Associativity}] \\
                     &= (a_1a_2, b_1b_2)(a_3, b_3) \\
                     &= [(a_1, b_1)(a_2, b_2)](a_3, b_3).
               \end{align*}
         \item Consider $(1, 1) \in A \times B$. It follows that
               \begin{align*}
                  (1, 1)(a_1, b_1) &= (1a_1, 1b_1) \\
                                   &= (a_1, b_1) \\
                                   &= (a_11, b_11) \\
                                   &= (a_1, b_1)(1, 1),
               \end{align*}
               so that $(1, 1)$ is the identity of $A \times B$.
         \item Consider $(a, b) \in A \times B$. It 
               follows that
               \begin{align*}
                  (a, b)(a^{-1}, b^{-1}) &= (aa^{-1}, bb^{-1}) \\
                                   &= (1, 1) \\
                                   &= (a^{-1}a, b^{-1}b) \\
                                   &= (a^{-1}, b^{-1})(a, b),
               \end{align*}
               so that $(a^{-1}, b^{-1})$ is the inverse of $(a, b)$.
      \end{enumerate}
%%%%%%%%%%%%%%%%%%%%%%%%%%%%%%%%%%%%%1.1.29%%%%%%%%%%%%%%%%%%%%%%%%%%%%%%%%%%%%%
   \item[1.1.29]  Prove that $A \times B$ is an abelian group if and only if
                  both $A$ and $B$ are abelian.

      \textbf{Proof.} 

      $(\Leftarrow)$ Suppose that $A$ and $B$ are abelian. Let $(a_1, b_1)$ and
      $(a_2, b_2) \in A \times B$. It follows that $A \times B$ is abelian
      because
      \begin{align*}
         (a_1, b_1)(a_2, b_2) &= (a_1a_2, b_1b_2) \\
            &= (a_2a_1, b_2b_1) &[\text{$A$ and $B$ are abelian}] \\
            &= (a_2, b_2)(a_1, b_1).
      \end{align*}

      $(\Rightarrow)$ Now suppose that $A \times B$ is abelian. Let $a_1$ and
      $a_2$ be members of $A$ and let $b_1$ and $b_2$ be members of $B$. Then
      we have that
      \begin{align*}
         (a_1a_2, b_1b_2) &= (a_1, b_1)(a_2, b_2) \\
            &= (a_2, b_2)(a_1, b_1) &[\text{$A \times B$ is abelian}] \\
            &= (a_2a_1, b_2b_1),
      \end{align*}
      so that $(a_1a_2, b_1b_2) = (a_2a_1, b_2b_1)$; i.e., $a_1a_2 = a_2a_1$ and
      $b_1b_2 = b_2b_1$. We can now conclude that $A$ and $B$ are both abelian.
      \qed
%%%%%%%%%%%%%%%%%%%%%%%%%%%%%%%%%%%%%1.1.30%%%%%%%%%%%%%%%%%%%%%%%%%%%%%%%%%%%%%
   \item[1.1.30]  Prove that the elements $(a, 1)$ and $(1, b)$ of $A \times B$
                  commute and deduce that the order of $(a, b)$ is the least 
                  common multiple of $|a|$ and $|b|$.

      \textbf{Proof.} Let $A$ and $B$ be groups, and let $a \in A$, $b \in B$.
      We shall be assuming that there exist positive integers $m$ and $n$ such 
      that $|a| = m$ and $|b| = n$, for the problem does not make sense if the
      order of $a$ or $b$ is not finite. Consider $(a, 1)$,
      $(1, b) \in A \times B$. We have that
      \begin{align*}
         (a, 1)(1, b) &= (a1, 1b) \\
                      &= (a, b) \\
                      &= (1a, b1) \\
                      &= (1, b)(a, 1),
      \end{align*}
      so that $(a, 1)$ and $(b, 1)$ commute. To complete the proof, we let
      $s = \text{lcm}(m, n)$. Thus we can write $s = mx = ny$ for positive 
      integers $x$ and $y$. Thus we have that
      \begin{align*}
         (a, b)^s &= ((a, 1)(1, b))^s \\
                  &= (a, 1)^s(1, b)^s &[\text{Exercise 1.1.24}] \\
                  &= (a^s, 1^s)(1^s, b^s) \\
                  &= (a^s, b^s) \\
                  &= (a^{mx}, a^{ny}) \\
                  &= ((a^m)^x, (a^n)^y) \\
                  &= (1^x, 1^y) \\
                  &= (1, 1).
      \end{align*}
      This say that $|(a, b)| \le s$, so there exists a positive integer $q$ 
      such that $|(a, b)| = q$. That is, $q \le s$ and
      $1 = (1, 1) = (a, b)^q = (a^q, b^q)$. If $q < s$, then both $m$
      and $n$ cannot divide $q$, so suppose without loss that $m$ does not
      divide $q$. So use the Division Algorithm to write
      $q = q'm + r$, with $0 < r < m$. Thus
      $$1 = a^q = a^{q'm + r} = (a^m)^{q'}a^r = 1a^r = a^r,$$
      a contradiction because $|a| = m$. Thus $q = s$, as desired. \qed
%%%%%%%%%%%%%%%%%%%%%%%%%%%%%%%%%%%%%1.1.31%%%%%%%%%%%%%%%%%%%%%%%%%%%%%%%%%%%%%
   \item[1.1.31]  Prove that any finite group $G$ of even order contains an
                  element of order 2. [Let $t(G)$ be the set
                  $\{g \in G : g \neq g^{-1}\}$. Show that $t(G)$ has an even 
                  number of elements and every nonidentity element of $G - t(G)$ 
                  has order 2.]

      \textbf{Proof.} Let $G$ be a finite group of even order. We wish to show
      that there exists some $g \in G$ such that $|g| = 2$. Consider the sets:
      $$S = \{g \in G: g \neq g^{-1}\} \text{ and }
        S' = \{\{g, g^{-1}\} : g \in S\}.$$

      Since every element of $S$ is not equal to its inverse, it follows that
      each set in $S'$ contains exactly two elements. Also observe that $S'$ is
      a partition of $S$; so $|S|$ is even because each set in $S'$ has 2 
      elements. Now let $S'' = G - S$. It follows that $|G| = |S| + |S''|$. 
      Notice that $S''$ is not empty because $1 \in S''$. Since $|G|$ and $|S|$ 
      are both even, it follows that $|S''|$ must also be even. That is,
      $|S''| \ge 2$, so that $S''$ contains a non-identity $a$, where
      $a = a^{-1}$. That is, $|a| = 2$. \qed
%%%%%%%%%%%%%%%%%%%%%%%%%%%%%%%%%%%%%1.1.32%%%%%%%%%%%%%%%%%%%%%%%%%%%%%%%%%%%%%
   \item[1.1.32]  If $x$ is an element of finite order $n$ in $G$, prove that
                  the elements 1, $x$, $x^2$, $\ldots$, $x^{n-1}$ are all 
                  distinct. Deduce that $|x| \le |G|$.

      \textbf{Proof.} Suppose that $|x| = n \in \Z^+$ for some $x \in G$. 
      Suppose to the contrary that the elements $x^0$, $x^1$, $x^2$, $\ldots$, 
      $x^{n-1}$ are not distinct. Then we must have that $x^i = x^j$ for some
      integer $i$ and $j$ where $0 \le i < j \le n - 1$. That is, $x^{j-i} = 1$,
      a contradiction because $j - i$ is a positive integer less thatn $n$. It
      follows that the elements $x^0$, $x$, $x^2$, $\ldots$, $x^{n-1}$ are all 
      distinct. Since there are clearly $n$ of these elements and since they are
      all members of $G$, we can conclude that $|x| = n \le |G|$. \qed
%%%%%%%%%%%%%%%%%%%%%%%%%%%%%%%%%%%%%1.1.33%%%%%%%%%%%%%%%%%%%%%%%%%%%%%%%%%%%%%
   \item[1.1.33]  Let $x$ be an element of finite order $n$ in $G$.
                  \begin{enumerate}
                     \item Prove that if $n$ is odd then $x^i \neq x^{-i}$ for
                           all $i = 1, 2, \ldots, n - 1$,
                     \item Prove that if $n = 2k$ and $1 \le i < n$ then
                           $x^i = x^{-i}$ if and only if $i = k$.
                  \end{enumerate}

      \textbf{Proof.}

      \begin{enumerate}
         \item Suppose that $n$ is odd, $n \ge 3$ (Result holds vacuously if
               $n = 1$). Now we shall suppose to the contrary
               that $x^i = x^{-i}$ for some integer $1 \le i \le n - 1$. Since
               $x^i = x^{-i}$, it follows that $x^{2i} = 1$. Since
               $i \in \{1, 2, \ldots, n - 1\}$, it follows that
               $2i \in \{2, 4, \ldots, n - 1, n + 1, \ldots, 2n - 2\}$. If
               $2i \le n - 1$, then $x^{2i} \neq 1$ because $|x| = n$. Thus
               $n + 1 \le 2i \le 2n - 2$; particularly, since $2i \ge n + 1$,
               write $2i = n + t$, for some positive integer $t$. Now
               $n + 1 \le n + t \le 2n - 2$, so that $1 \le t \le n - 2$. Thus
               $$x^{2i} = x^{n+t} = x^nx^t = 1x^t = x^t \neq 1,$$
               because $t < n$. Thus $x^{2i} \neq 1$ and we conclude that
               $x^i \neq x^{-i}$ for all $1 \le i \le n - 1$. \qed
         \item Suppose that $n$ is even and $1 \le i < n$. Write $n = 2k$ for
               some positive integer $k$.

               $(\Leftarrow)$ Suppose that $i = k$. Then we have that
               $1 = x^{2k} = x^{2i} = x^ix^i$, so that $x^i = x^{-i}$.

               $(\Rightarrow)$ Conversely suppose that $x^i = x^{-i}$, so that
               $x^{2i} =1$; that is, $2k = |x| \le 2i$. Now since $i < 2k$, we
               have that $2i \le 4k - 2$, and hence, $2k \le 2i \le 4k - 2$.
               Particularly, since $2i \ge 2k$, write $2i = 2k + t$, for some
               nonnegative integer $t$. Now $2k \le 2k + t \le 4k - 2$ implies 
               that $0 \le t \le 2k - 2$. We then have that
               $$1 = x^{2i} = x^{2k+t} = x^{2k}x^t = 1x^t = x^t.$$
               Since $0 \le t \le 2k - 2 < 2k$, $x^t = 1$, and $|x| = 2k$, it 
               follows that $t = 0$. Thus $2i = 2k$, so that $i = k$. \qed
      \end{enumerate}
%%%%%%%%%%%%%%%%%%%%%%%%%%%%%%%%%%%%%1.1.34%%%%%%%%%%%%%%%%%%%%%%%%%%%%%%%%%%%%%
   \item[1.1.34]  If $x$ is an element of infinite order in $G$, prove that the
                  elements $x^n$, $n \in \Z$ are all distinct.

      \textbf{Proof.} Assume that $x$ is an element of infinite order in $G$.
      Now suppose to the contrary that $x^i = x^j$ for some unequal integers
      $i$ and $j$. We can further assume without loss of generality that
      $i < j$. Thus $x^{j-i} = 1$, a contradiction because this says that
      $|x| \le j - i$. It follows that distinct integral powers of $x$ yield 
      distinct elements of $G$. \qed
%%%%%%%%%%%%%%%%%%%%%%%%%%%%%%%%%%%%%1.1.35%%%%%%%%%%%%%%%%%%%%%%%%%%%%%%%%%%%%%
   \item[1.1.35]  If $x$ is an element of finite order $n$ in $G$, use the 
                  Division Algorithm to show that any integral power of $x$ 
                  equals one of the elements in the set
                  $\{1, x, x^2, \ldots, x^{n-1}\}$ (so these are all the
                  distinct elements of the cyclic subgroup of $G$ generated by
                  $x$).

      \textbf{Proof.} Assume that $x$ is an element of finite order $n$ in $G$.
      Let $z \in \Z$. By the Division Algorithm, there exist unique integers
      $q$ and $r$ such that $z = qn + r$ and $0 \le r < n$. That is
      $$x^z = x^{qn+r} = x^{qn}x^r = (x^n)^qx^r = 1^qx^r = x^r.$$
      Since $r \in \{0, 1, \ldots, n - 1\}$ and since $x^z = x^r$, it follows
      that $x^z \in \{x^0, x^1, \ldots, x^{n-1}\}$. \qed
%%%%%%%%%%%%%%%%%%%%%%%%%%%%%%%%%%%%%1.1.36%%%%%%%%%%%%%%%%%%%%%%%%%%%%%%%%%%%%%
   \item[1.1.36]  Assume $G = \{1, a, b, c\}$ is a group of order 4 with
                  identity 1. Assume also that $G$ has no elements of order 4
                  (so by Exercise 32, every element has order $\le$ 3). Use the
                  cancellation laws to show that there is a unique group table
                  for $G$. Deduce that $G$ is abelian.

      \textbf{Proof.} Assume $G = \{1, a, b, c\}$. We can immediately fill out 
      the group table for $G$ like so:      
      $$
         \begin{tabular}{@{}c | c | c | c | c@{}} 
                & $1$ & $a$ & $b$ & $c$ \\ \hline
            $1$ & $1$ & $a$ & $b$ & $c$ \\ \hline
            $a$ & $a$ & $ $ & $ $ & $ $ \\ \hline
            $b$ & $b$ & $ $ & $ $ & $ $ \\ \hline
            $c$ & $c$ & $ $ & $ $ & $ $
         \end{tabular}
      $$
      We observe that the cancellation laws imply that no two elements in a 
      column(or row) are the same; that is, each column and each row is a 
      permutation of $G$. The equality $ab = a$ implies $b = 1$ and the equality 
      $ab = b$ implies $a = 1$, both of which are contradictions. The only 
      remaining possiblities are $ab = c$ or $ab = 1$.

      \textbf{Case 1.} $ab = c$. Now $ac \neq a$ and $ac \neq c$; otherwise,
      we would have $c = 1$ or $a = 1$, both contradictions. Thus $ac = 1$ or
      $ac = b$. So suppose first that $ac = b$. Then our table will now look
      like so:
      $$
         \begin{tabular}{@{}c | c | c | c | c@{}} 
                & $1$ & $a$ & $b$ & $c$ \\ \hline
            $1$ & $1$ & $a$ & $b$ & $c$ \\ \hline
            $a$ & $a$ & $ $ & $c$ & $b$ \\ \hline
            $b$ & $b$ & $ $ & $ $ & $ $ \\ \hline
            $c$ & $c$ & $ $ & $ $ & $ $
         \end{tabular}
      $$
      From the table above, we see that $aa$ must be equal to 1, since that is
      the only remaining possibility. Also $ba$ is forced to equal $c$ and $ca$
      is forced to equal $b$.
      $$
         \begin{tabular}{@{}c | c | c | c | c@{}} 
                & $1$ & $a$ & $b$ & $c$ \\ \hline
            $1$ & $1$ & $a$ & $b$ & $c$ \\ \hline
            $a$ & $a$ & $1$ & $c$ & $b$ \\ \hline
            $b$ & $b$ & $c$ & $ $ & $ $ \\ \hline
            $c$ & $c$ & $b$ & $ $ & $ $
         \end{tabular}
      $$
      Note that we cannot have $bb = a$ because that would imply that $bbb = c$,
      so that $|b| > 3$, contradicting our hypothesis. Thus we must have that
      $bb = 1$. The remaining positions are thus completely determined, so that
      we have
      $$
         \begin{tabular}{@{}c | c | c | c | c@{}} 
                & $1$ & $a$ & $b$ & $c$ \\ \hline
            $1$ & $1$ & $a$ & $b$ & $c$ \\ \hline
            $a$ & $a$ & $1$ & $c$ & $b$ \\ \hline
            $b$ & $b$ & $c$ & $1$ & $a$ \\ \hline
            $c$ & $c$ & $b$ & $a$ & $1$
         \end{tabular}
      $$
      Now suppose that $ac = 1$, then we would be forced to fill in the table
      like so:
      $$
         \begin{tabular}{@{}c | c | c | c | c@{}} 
                & $1$ & $a$ & $b$ & $c$ \\ \hline
            $1$ & $1$ & $a$ & $b$ & $c$ \\ \hline
            $a$ & $a$ & $b$ & $c$ & $1$ \\ \hline
            $b$ & $b$ & $c$ & $ $ & $ $ \\ \hline
            $c$ & $c$ & $1$ & $ $ & $ $
         \end{tabular}
      $$
      Since $a^2 = b$ and $a^3 = c$, we have that $|a| > 3$, contradicting our
      hypothesis, so this is a dead end.

      \textbf{Case 2.} $ab = 1$. We already know that $ac \neq a$ and
      $ac \neq c$. Now $ac = 1$ implies that $ac = ab$, so that $c = b$, a
      contradiction. Thus $ac = b$; and all the other positions are completely
      determined, so that our table looks like so:
      $$
         \begin{tabular}{@{}c | c | c | c | c@{}} 
                & $1$ & $a$ & $b$ & $c$ \\ \hline
            $1$ & $1$ & $a$ & $b$ & $c$ \\ \hline
            $a$ & $a$ & $c$ & $1$ & $b$ \\ \hline
            $b$ & $b$ & $1$ & $c$ & $a$ \\ \hline
            $c$ & $c$ & $b$ & $a$ & $1$
         \end{tabular}
      $$
      Since $a^2 = c$ and $a^3 = b$, we have that $|a| > 3$, contradicting our
      hypothesis, so this is another dead end. From our arguments above, we see
      that the only viable and legal table is thus:
      $$
         \begin{tabular}{@{}c | c | c | c | c@{}} 
                & $1$ & $a$ & $b$ & $c$ \\ \hline
            $1$ & $1$ & $a$ & $b$ & $c$ \\ \hline
            $a$ & $a$ & $1$ & $c$ & $b$ \\ \hline
            $b$ & $b$ & $c$ & $1$ & $a$ \\ \hline
            $c$ & $c$ & $b$ & $a$ & $1$
         \end{tabular}
      $$
      This table is unique, and since it is symmeteric it follows that $G$ is
      abelian. \qed
\end{enumerate}
