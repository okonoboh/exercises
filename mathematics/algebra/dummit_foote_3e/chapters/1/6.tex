Let $G$ and $H$ be groups.
\begin{enumerate}
%%%%%%%%%%%%%%%%%%%%%%%%%%%%%%%%%%%%%1.6.1%%%%%%%%%%%%%%%%%%%%%%%%%%%%%%%%%%%%%%
   \item[1.6.1]   Let $\varphi : G \rightarrow H$ be a homomorphism.
                  \begin{enumerate}
                     \item Prove that $\varphi(x^n) = \varphi(x)^n$ for all
                           $n \in \Z^+$.
                     \item Do part (a) for $n = -1$ and deduce that
                           $\varphi(x^n) = \varphi(x)^n$ for all $n \in \Z$.
                  \end{enumerate}

      \textbf{Solution.}

      \begin{enumerate}
         \item \textbf{Proof.} We shall proceed by induction on $n$. It is clear
               that $\varphi(x^1) = \varphi(x)^1$. Now suppose that
               $\varphi(x^k) = \varphi(x)^k$ for some integer $k$. Thus
               \begin{align*}
                  \varphi(x^{k+1}) &= \varphi(x^kx)
                        &[\text{Exercise 1.1.19(a)}] \\
                     &= \varphi(x^k)\varphi(x)
                        &[\varphi\text{ is a homomorphism}] \\
                     &= \varphi(x)^k\varphi(x) &[\text{Inductive hypothesis}] \\
                     &= \varphi(x)^{k+1}, &[\text{Exercise 1.1.19(a)}]
               \end{align*}
               so that, by Mathematical Induction, $\varphi(x^n) = \varphi(x)^n$ 
               for all $n \in \Z^+$. \qed
         \item Since
               $$1 \cdot \varphi(1) = \varphi(1) = \varphi(1 \cdot 1) =
                 \varphi(1)\cdot\varphi(1),$$
               it follows by cancellation that $\varphi(1) = 1$. Thus
               $$\varphi(x)\varphi(x^{-1}) = \varphi(xx^{-1}) =\varphi(1) = 1,$$
               so that $\varphi(x^{-1}) = \varphi(x)^{-1}$. Now let $n$ be a
               positive integer. Then it follows that
               \begin{align*}
                  \varphi(x^{-n}) &= \varphi((x^{-1})^n) \\
                     &= (\varphi(x^{-1}))^n &[\text{1.6.1(a)}] \\
                     &= (\varphi(x)^{-1})^n \\
                     &= \varphi(x)^{-n}.
               \end{align*}
               Moreover $\varphi(x^0) = 1 = \varphi(x)^0$; thus we can conclude 
               that $\varphi(x^n) = \varphi(x)^n$ for all $n \in \Z$.
      \end{enumerate}
%%%%%%%%%%%%%%%%%%%%%%%%%%%%%%%%%%%%%1.6.2%%%%%%%%%%%%%%%%%%%%%%%%%%%%%%%%%%%%%%
   \item[1.6.2]   If $\varphi : G \rightarrow H$ is an isomorphism, prove that
                  $|\varphi(x)| = |x|$ for all $x \in G$. Deduce that any two
                  isomorphic groups have the same number of elements of order
                  $n$ for each $n \in \Z^+$. Is the result true if $\varphi$ is
                  only assumed to be a homomorphism?

      \textbf{Proof.} Assume that $\varphi : G \rightarrow H$ is a group
      isomorphism. Let $x \in G$. Suppose $|x| = n \in \Z^+$. By the preceding 
      exercise, we have that $\varphi(x)^n = \varphi(x^n) = \varphi(1) = 1$, so 
      that $|\varphi(x)| \le n$. Now suppose that $|\varphi(x)| = m < n$. Then
      we must have that $\varphi(1) = 1 = \varphi(x)^m = \varphi(x^m)$. That is,
      $x^m = 1$ (since $\varphi$ is injective), a contradiction since $|x| = n$.
      Thus $|\varphi(x)| = |x| = n$. Finally suppose that $|x| = \infty$ and
      $|\varphi(x)| = r < \infty$. Then, as previously argued, we must have
      that $x^r = 1$, a contradiction. Thus if $y \in G$, it must follow that
      $|y| = |\varphi(y)|$. \qed

      For a positive integer $n$, we can now exhibit a bijection (using
      $\varphi$) between the elements of $G$ of order $n$ and the elements of
      $H$ of order $n$. Thus any two isomorphic groups must have the same number 
      of elements of order $n$ for each $n \in \Z^+$. If $\varphi$ is only 
      assumed to be a homomorphism then the result is not generally true. 
      Consider the homomorphism
      $$\alpha : S_3 \rightarrow \{1\}.$$
      Although $S_3$ has an element of order 2, the trivial group $\{1\}$ has no 
      element of order 2.
%%%%%%%%%%%%%%%%%%%%%%%%%%%%%%%%%%%%%1.6.3%%%%%%%%%%%%%%%%%%%%%%%%%%%%%%%%%%%%%%
   \item[1.6.3]   If $\varphi : G \rightarrow H$ is an isomorphism, prove that
                  $G$ is abelian if and only if $H$ is abelian. If
                  $\varphi : G \rightarrow H$ is a homomorphism, what additional
                  conditions on $\varphi$ (if any) are sufficient to ensure that
                  if $G$ is abelian, then so is $H$?

      \textbf{Proof.} Suppose $\varphi : G \rightarrow H$ is an isomorphism.

      ($\Rightarrow$) Assume $G$ is abelian. Consider $x$ and $y$ in $H$. Since
      $\varphi$ is surjective, there exist $a$, $b \in G$ such that
      $\varphi(a) = x$ and $\varphi(b) = y$. Since $G$ is abelian, we have that
      $$xy = \varphi(a)\varphi(b) = \varphi(ab) = \varphi(ba) = 
        \varphi(b)\varphi(a) = yx,$$
      so that $H$ is also abelian.

      ($\Rightarrow$) Now assume that $H$ is abelian. Consider $a$ and $b$ 
      in $G$. Since $H$ is abelian, we have that
      $$\varphi(ab) = \varphi(a)\varphi(b) = \varphi(b)\varphi(a) =
        \varphi(ba).$$
      We thus conclude that $ab = ba$ since $\varphi$ is injective. That is, $G$ 
      is abelian. \qed

      Finally suppose $\varphi$ is an homomorphism and $G$ is abelian. From 
      observing the first direction of our proof above, we see that restricting
      $\varphi$ to be surjective is sufficient to make $H$ abelian.
%%%%%%%%%%%%%%%%%%%%%%%%%%%%%%%%%%%%%1.6.4%%%%%%%%%%%%%%%%%%%%%%%%%%%%%%%%%%%%%%
   \item[1.6.4]   Prove that the multiplicative groups $\R - \{0\}$ and
                  $\C - \{0\}$ are not isomorphic.

      \textbf{Proof.} Since $\R - \{0\}$ has no element of order 4, and since
      $\C - \{0\}$ has 2 elements ($i$ and $-i$) of order 4, it follows that
      these two multiplicative groups are not isomorphic. \qed
%%%%%%%%%%%%%%%%%%%%%%%%%%%%%%%%%%%%%1.6.5%%%%%%%%%%%%%%%%%%%%%%%%%%%%%%%%%%%%%%
   \item[1.6.5]   Prove that the additive groups of $\R$ and $\Q$ are not
                  isomorphic.

      \textbf{Proof.} Since $|\R| \neq |\Q|$, it follows that $(\R, +)$ and
      $(\Q, +)$ are not isomorphic. \qed
%%%%%%%%%%%%%%%%%%%%%%%%%%%%%%%%%%%%%1.6.6%%%%%%%%%%%%%%%%%%%%%%%%%%%%%%%%%%%%%%
   \item[1.6.6]   Prove that the additive groups of $\Z$ and $\Q$ are not
                  isomorphic.

      \textbf{Proof.} First we shall show that $(\Q, +)$ is not cyclic. So
      suppose that $(\Q, +) = \cyc{c}$. Thus there exists an integer $n$ such 
      that $nc = \D\frac{c}{2}$, so that $\D n = \frac{1}{2}$, a contradiction. 
      That is, $(\Q, +)$ is not cyclic. Now suppose to the contrary that
      $\varphi : \Z \rightarrow \Q$ is a homomorphism from $(\Z, +)$ to
      $(\Q, +)$. Let $q \in \Q$ and let $z$ be the unique preimage of $q$ under
      $\varphi$. Thus
      \begin{align*}
         q &= \varphi(z) \\
           &= \varphi(z \cdot 1) \\
           &= z \cdot \varphi(1),  &[1.6.1(a)]
      \end{align*}
      so that $\Q = \cyc{\varphi(1)}$, a contradiction. Thus $(\Z, +)$ and
      $(\Q, +)$ are not isomorphic. \qed
%%%%%%%%%%%%%%%%%%%%%%%%%%%%%%%%%%%%%1.6.7%%%%%%%%%%%%%%%%%%%%%%%%%%%%%%%%%%%%%%
   \item[1.6.7]   Prove that $D_8$ and $Q_8$ are not isomorphic.

      \textbf{Proof.} Since $D_8$ has exactly 5 elements of order 2 and $Q_8$
      has exactly 1 element of order 2, it follows that these two groups are not
      isomorphic. \qed
%%%%%%%%%%%%%%%%%%%%%%%%%%%%%%%%%%%%%1.6.8%%%%%%%%%%%%%%%%%%%%%%%%%%%%%%%%%%%%%%
   \item[1.6.8]   Prove that if $n \neq m$, $S_n$ and $S_m$ are not isomorphic.

      \textbf{Proof.} Let $n$ and $m$ be unequal positive integers so that
      $n! \neq m!$; i.e., $n! = |S_n| \neq |S_m| = m!$. Thus no bijective map
      can exist between $S_n$ and $S_m$, so that these two groups are not
      isomorphic. \qed
%%%%%%%%%%%%%%%%%%%%%%%%%%%%%%%%%%%%%1.6.9%%%%%%%%%%%%%%%%%%%%%%%%%%%%%%%%%%%%%%
   \item[1.6.9]   Prove that $D_{24}$ and $S_4$ are not isomorphic.

      \textbf{Proof.} Since $D_{24}$ has exactly 2 elements of order 4 and $S_4$
      has exactly 6 elements of order 4, it follows that these two groups are
      not isomorphic. \qed
%%%%%%%%%%%%%%%%%%%%%%%%%%%%%%%%%%%%%1.6.10%%%%%%%%%%%%%%%%%%%%%%%%%%%%%%%%%%%%%
   \item[1.6.10]  Fill in the details of the proof that the symmetric groups
                  $S_\triangle$ and $S_\Omega$ are isomorphic if
                  $|\triangle| = |\Omega|$ as follows: let
                  $\theta : \triangle \rightarrow \Omega$ be a bijection. Define
                  $$\varphi : S_\triangle \rightarrow S_\Omega \qquad
                    \text{by} \qquad \varphi(\sigma) = \theta \circ \sigma \circ 
                    \theta^{-1} \text{ for all } \sigma \in S_\triangle$$
                  and prove the following:
                  \begin{enumerate}
                     \item $\varphi$ is well defined, that is, if $\sigma$ is a
                           permutation of $\triangle$ then
                           $\theta\circ\sigma\circ\theta^{-1}$ is a permutation
                           of $\Omega$.
                     \item $\varphi$ is a bijection from $S_\triangle$ onto
                           $S_\Omega$. [Find a 2-sided inverse for $\varphi$.]
                     \item $\varphi$ is a homomorphism, that is,
                           $\varphi(\sigma\circ\tau) =
                            \varphi(\sigma)\circ\varphi(\tau)$.
                  \end{enumerate}
                  Note the similarity to the \textit{change of basis} or
                  \textit{similarity} transformations for matrices.

      \textbf{Proof.}

      \begin{enumerate}
         \item Let $\sigma \in S_\triangle$. Notice that
               $\varphi(\sigma) = \theta\circ\sigma\circ\theta^{-1}$ maps
               $\Omega$ into $\Omega$, and is also a bijecton since it is a
               composition of bijective maps. Thus
               $\varphi(\sigma) \in S_\Omega$, so that $\varphi$ is well
               defined.
         \item Consider the map
               $$\alpha : S_\Omega \rightarrow S_\triangle \qquad
                 \text{by} \qquad \alpha(\sigma) = \theta \circ \sigma^{-1} 
                 \circ \theta^{-1} \text{ for all } \sigma \in S_\Omega.$$
               A trivial computation will show us that $\varphi\circ\alpha$ is
               the identity map on $S_\Omega$ and $\alpha\circ\varphi$ is the
               identity map on $S_\triangle$. Thus $\varphi$ is a bijection.
         \item Let $\sigma, \tau \in S_\triangle$. So
               \begin{align*}
                  \varphi(\sigma\circ\tau) &=
                     \theta\circ\sigma\circ\tau\circ\theta^{-1} \\
                     &= \theta\circ\sigma\circ\theta^{-1}\circ\theta\circ
                        \tau\circ\theta^{-1} \\
                     &= \varphi(\sigma) \circ \varphi(\tau);
               \end{align*}
               i.e., $\varphi$ is a homomoirphism.
      \end{enumerate} \qed
%%%%%%%%%%%%%%%%%%%%%%%%%%%%%%%%%%%%%1.6.11%%%%%%%%%%%%%%%%%%%%%%%%%%%%%%%%%%%%%
   \item[1.6.11]  Let $A$ and $B$ be groups. Prove that
                  $A \times B \cong B\times A$.

      \textbf{Proof.} Consider the map $f : A \times B \rightarrow B \times A$,
      $(a, b) \mapsto (b, a)$. Now define
      $$g : B \times A\rightarrow A\times B, \quad (b, a) \mapsto (a, b).$$
      So $f$ is bijective since $g$ is its two-sided inverse. Now $f$ is a
      homomorphism because for each $(a, b), (c, d) \in A \times B$,
      \begin{align*}
         f((a, b)(c, d)) &= f(ac, bd) \\
            &= (bd, ac) \\
            &= (b, a)(d, c) \\
            &= f(a, b)f(c, d).
      \end{align*}
      Conclude that $A \times B \cong B\times A$. \qed
%%%%%%%%%%%%%%%%%%%%%%%%%%%%%%%%%%%%%1.6.12%%%%%%%%%%%%%%%%%%%%%%%%%%%%%%%%%%%%%
   \item[1.6.12]  Let $A$, $B$, and $C$ be groups and let $G = A \times B$ and
                  $H = B \times C$. Prove that $G \times C \cong A \times H$.

      \textbf{Proof.} Proceed as we did in Exercise 1.6.11 with the following
      modification:
      $$f : G \times C \rightarrow A \times H,
        \quad((a, b), c) \mapsto (a, (b, c))$$ and 
      $$g : A \times H\rightarrow G\times C,
        \quad (a, (b, c)) \mapsto ((a, b), c).$$ \qed
%%%%%%%%%%%%%%%%%%%%%%%%%%%%%%%%%%%%%1.6.13%%%%%%%%%%%%%%%%%%%%%%%%%%%%%%%%%%%%%
   \item[1.6.13]  Let $G$ and $H$ be groups and let $\varphi : G \rightarrow H$
                  be a homomorphism. Prove that the image of $\varphi$,
                  $\varphi(G)$, is a subgroup of $H$. Prove that if $\varphi$ is
                  injective then $G \cong \varphi(G)$.

      \textbf{Proof.} The set $\varphi(G)$ is nonempty since it contains
      $\varphi(1) = 1$. So let $x, y \in \varphi(G)$. Thus there exist
      $a, b \in G$ such that $\varphi(a) = x$ and $\varphi(b) = y$. Since
      $ab \in G$ (by closure), it follows that
      $$\varphi(ab) = \varphi(a)\varphi(b) = xy \in \varphi(G).$$
      That is, $\varphi(G)$ is closed under the operation of $H$. Since $G$ is a
      group, it follows that $a^{-1} \in G$, and by Exercise 1.6.1 (b), we have
      that $x^{-1} = (\varphi(a))^{-1} = \varphi(a^{-1}) \in \varphi(G)$. That
      is, $\varphi(G)$ is closed under inverses. Conclude by Exercise 1.1.26
      that $\varphi(G)$ is a subgroup of $H$. Now suppose that $\varphi$ is
      injective and consider the map $\alpha : G \rightarrow \varphi(G)$,
      $g \mapsto \varphi(g)$. Since $\varphi$ is an injective homomorphism, it
      follows that $\alpha$ is also an injective homomorphism. Also it is clear
      that $\alpha$ is onto. Thus $\alpha$ is an isomorphism and we have that
      $G \cong \varphi(G)$. \qed
%%%%%%%%%%%%%%%%%%%%%%%%%%%%%%%%%%%%%1.6.14%%%%%%%%%%%%%%%%%%%%%%%%%%%%%%%%%%%%%
   \item[1.6.14]  Let $G$ and $H$ be groups and let $\varphi : G \rightarrow H$
                  be a homomorphism. Define the \text{kernel} of $\varphi$ to be
                  $\{g \in G : \varphi(g) = 1_H\}$. Prove that the kernel of
                  $\varphi$ is a subgroup of $G$. Prove that $\varphi$ is
                  injective if and only if the kernel of $\varphi$ is the
                  identity subgroup of $G$.

      \textbf{Solution.} First we will show that ker$(\varphi)$ is a subgroup of
      $G$.

      \textbf{Proof.} Let ker($\varphi$) be the kernel of $\varphi$. The set
      ker($\varphi$) is not empty because
      $$1_H\varphi(1_G) = \varphi(1_G) = \varphi(1_G1_G) = 
        \varphi(1_G)\varphi(1_G),$$
      so that $\varphi(1_G) = 1_H$, by cancellation; hence,
      $1_G \in \text{ker}(\varphi)$. Now let $x, y \in \text{ker}(\varphi)$.
      Then, by definition, $\varphi(x) = \varphi(y) = 1_H$, and
      $$\varphi(xy) = \varphi(x)\varphi(y) = 1_H1_H = 1_H,$$
      so that $xy \in \text{ker}(\varphi)$. That is, ker($\varphi$) is closed
      under the operation of $G$. By Exercise 1.6.1(b), it follows that
      $$\varphi(x^{-1}) = (\varphi(x))^{-1} = {1_H}^{-1} = 1_H,$$
      so that $x^{-1} \in \text{ker}(\varphi)$. Conclude by Exercise 1.1.26 that
      ker($\varphi$) is a subgroup of $G$. \qed

      Now we will show that $\varphi$ is injective if and only if the kernel of
      $\varphi$ is the identity subgroup of $G$.

      \textbf{Proof.} ($\Rightarrow$) Assume that $\varphi$ is injective. Let
      $x \in \text{ker}(\varphi)$. Thus we have that
      $\varphi(1_G) = \varphi(x) = 1_H$, so that $x = 1_G$ since $\varphi$ is
      one to one. That is, ker($\varphi$) = $\{1_G\}$, the identity subgroup of
      $G$.

      ($\Leftarrow$) Assume that ker($\varphi$) = $\{1_G\}$. Suppose that
      $\varphi(x) = \varphi(y)$ for some $x, y \in G$. Then it follows
      that $\varphi(x)\varphi(y)^{-1} = 1_H$, and  by Exercise 1.6.1, that
      $\varphi(x)\varphi(y)^{-1} = \varphi(x)\varphi(y^{-1}) =
      \varphi(xy^{-1}) = 1_H$, so that $xy^{-1} \in \text{ker}(\varphi)$. Since 
      ker($\varphi$) = $\{1_G\}$, we can conclude that $xy^{-1} = 1_G$, so that
      $x = y$; i.e., $\varphi$ is injective. \qed      
%%%%%%%%%%%%%%%%%%%%%%%%%%%%%%%%%%%%%1.6.15%%%%%%%%%%%%%%%%%%%%%%%%%%%%%%%%%%%%%
   \item[1.6.15]  Define a map $\pi : \R^2 \rightarrow \R$ by $\pi((x, y)) = x$.
                  Prove that $\pi$ is a homomorphism and find the kernel of
                  $\pi$.

      \textbf{Proof.} Let $(a, b), (c, d) \in \R^2$. Then it follows immediately
      that $\pi$ is a homomorphism since
      $$\pi((a, b) + (c, d)) = \pi(a + c, b + d) =
         a + c = \pi(a, b) + \pi(c, d).$$
      The kernel of $\pi$ is the $y$-axis = $\{(0, y) : y \in \R\}$. \qed
%%%%%%%%%%%%%%%%%%%%%%%%%%%%%%%%%%%%%1.6.16%%%%%%%%%%%%%%%%%%%%%%%%%%%%%%%%%%%%%
   \item[1.6.16]  Let $A$ and $B$ be groups and let $G$ be their direct product,
                  $A \times B$. Prove that the maps $\pi_1 : G \rightarrow A$
                  and $\pi_2 : G \rightarrow B$ defined by $\pi_1((a, b)) = a$
                  and $\pi_2((a, b)) = b$ are homomorphisms and find their
                  kernels.

      \textbf{Proof.} Let $(a, b), (c, d) \in G$. Then it follows immediately 
      that $\pi_1$ and $\pi_2$ are homomorphisms since
      $$\pi_1((a, b)(c, d)) = \pi_1(ac, bd) = ac = \pi_1(a, b)\pi_1(c, d)$$
      and
      $$\pi_2((a, b)(c, d)) = \pi_2(ac, bd) = bd = \pi_2(a, b)\pi_2(c, d).$$
      The kernel of $\pi_1 = \{(1, b) : b \in B\}$ and
      the kernel of $\pi_2 = \{(a, 1) : a \in A\}$. \qed
%%%%%%%%%%%%%%%%%%%%%%%%%%%%%%%%%%%%%1.6.17%%%%%%%%%%%%%%%%%%%%%%%%%%%%%%%%%%%%%
   \item[1.6.17]  Let $G$ be any group. Prove that the map from $G$ to itself
                  defined by $g \mapsto g^{-1}$ is a homomorphism if and only if
                  $G$ is abelian.

      \textbf{Proof.} Let $x, y \in G$. Consider the map
      $\alpha : G \rightarrow G$, $g \mapsto g^{-1}$. 

      ($\Leftarrow$) Assume that $G$ is abelian. Then it
      follows that
      \begin{align*}
         \alpha(xy) &= (xy)^{-1} &[\text{By Definition}] \\
            &= y^{-1}x^{-1} &[\text{Proposition 1.4}] \\
            &= x^{-1}y^{-1} &[G \text{ is abelian}] \\
            &= \alpha(x)\alpha(y),
      \end{align*}
      so that $\alpha$ is a homomorphism.

      ($\Rightarrow$) Assume that $\alpha$ is a homomorphism. Then it
      follows that
      \begin{align*}
         xy &= \alpha(x^{-1})\alpha(y^{-1}) &[\text{Definition}] \\
            &= \alpha(x^{-1}y^{-1}) \\
            &= \alpha((yx)^{-1}) &[\text{Proposition 1.4}] \\
            &= ((yx)^{-1})^{-1} = yx, &[\text{Proposition 1.3}]
      \end{align*}
      so that $G$ is abelian. \qed
%%%%%%%%%%%%%%%%%%%%%%%%%%%%%%%%%%%%%1.6.18%%%%%%%%%%%%%%%%%%%%%%%%%%%%%%%%%%%%%
   \item[1.6.18]  Let $G$ be any group. Prove that the map from $G$ to itself
                  defined by $g \mapsto g^2$ is a homomorphism if and only if
                  $G$ is abelian.

      \textbf{Proof.} Let $x, y \in G$. Consider the map
      $\alpha : G \rightarrow G$, $g \mapsto g^2$. 

      ($\Leftarrow$) Assume that $G$ is abelian. Then it
      follows that
      \begin{align*}
         \alpha(xy) &= (xy)^2 &[\text{Definition}] \\
            &= x^2y^2 &[\text{Exercise 1.1.24}] \\
            &= \alpha(x)\alpha(y),
      \end{align*}
      so that $\alpha$ is a homomorphism.

      ($\Rightarrow$) Assume that $\alpha$ is a homomorphism. Then it
      follows that
      \begin{align*}
         x^2y^2 &= \alpha(x)\alpha(y) \\
            &= \alpha(xy) \\
            &= (xy)^2 \\
            &= xyxy,
      \end{align*}
      so that $xxyy = xyxy$. By cancellation we thus have $xy = yx$; i.e, $G$ is
      abelian. \qed
%%%%%%%%%%%%%%%%%%%%%%%%%%%%%%%%%%%%%1.6.19%%%%%%%%%%%%%%%%%%%%%%%%%%%%%%%%%%%%%
   \item[1.6.19]  Let $G = \{z \in \C : z^n = 1 \text{ for some }n \in \Z^+\}$.
                  Prove that for any fixed integer $k > 1$ the map from $G$ to
                  itself defined by $z \mapsto z^k$ is a surjective homomorphism
                  but is not an isomorphism.

      \textbf{Proof.} Consider an integer $k > 1$ and the map
      $\alpha : G \rightarrow G$, $g \mapsto g^k$. Let $x, y \in G$. The map
      $\alpha$ is a homomorphism since
      $\alpha(xy) = (xy)^k = x^ky^k = \alpha(x)\alpha(y)$. Since $x \in G$, it
      follows that $x^m = 1$ for some positive integer $m$. Notice that
      $x^{1/k} \in G$ since $(x^{1/k})^{km} = 1$. Thus $\alpha$ is onto because
      $\alpha(x^{1/k}) = x$. Now consider the complex number
      $e^{2\pi/k} = \cos2\pi/k + i \sin2\pi/k$. Notice that $e^{2\pi/k} \in G$
      since $(e^{2\pi/k})^k = 1$. Also notice that $e^{2\pi/k} \neq 1$ (since
      $2\pi/k$ is not a multiple of $2\pi$), but we have that
      $\alpha(e^{2\pi/k}) = \alpha(1) = 1$, so that $\alpha$ is not injective; 
      i.e, $\alpha$ is not an isomorphism. \qed
%%%%%%%%%%%%%%%%%%%%%%%%%%%%%%%%%%%%%1.6.20%%%%%%%%%%%%%%%%%%%%%%%%%%%%%%%%%%%%%
   \item[1.6.20]  Let $G$ be a group and let Aut($G$) be the set of all
                  isomorphisms from $G$ onto $G$. Prove that Aut($G$) is a
                  group under function composition (called the
                  \text{automorphism group} of $G$ and the elements of Aut($G$)
                  are called \text{automorphisms} of $G$).

      \textbf{Proof.}

      \textbf{Closure.} The set Aut($G$) is not empty because it contains the
      identity automorphism. So let $\alpha, \gamma \in \text{Aut}(G)$. Since
      the composition of two bijective functions is also bijective, it follows 
      that $\alpha \circ \gamma$ is bijective. Now let $x, y \in G$. It follows 
      that
      $$(\alpha\circ\gamma)(xy) = \alpha(\gamma(xy)) =
         \alpha(\gamma(x)\gamma(y)) = \alpha(\gamma(x))\alpha(\gamma(y)) =
         ((\alpha\circ\gamma)(x))((\alpha\circ\gamma)(y)),$$
      so that $(\alpha\circ\gamma)$ is also an isomorphism on $G$, and thus,
      Aut($G$) is closed.

      \textbf{Associativity.} This follows from the associativity of functions.

      \textbf{Identity.} The identity map is the identity of Aut($G$).

      \textbf{Inverse.} Let $\alpha \in \text{Aut}(G)$. Since $\alpha$ is 
      bijective, $\alpha^{-1}$ exists. It remains to show that $\alpha^{-1}$ is
      an isomorphism. Let $x, y \in G$. Since $\alpha$ is a bijection, there 
      exist unique $a, b \in G$ such that $\alpha(a) = x$ and $\alpha(b) = y$. 
      Since $\alpha$ is a homomorphism, we have that
      $xy = \alpha(a)\alpha(b) = \alpha(ab)$, so that $\alpha^{-1}(xy) = ab$.
      Hence,
      $$\alpha^{-1}(xy) = ab = \alpha^{-1}(x)\alpha^{-1}(y),$$
      so that $\alpha^{-1}$ is also an isomorphism, and thus it is in Aut($G$).
      Conclude that Aut($G$) is a group under composition. \qed      
%%%%%%%%%%%%%%%%%%%%%%%%%%%%%%%%%%%%%1.6.21%%%%%%%%%%%%%%%%%%%%%%%%%%%%%%%%%%%%%
   \item[1.6.21]  Prove that for each fixed nonzero $k \in \Q$ the map from $\Q$
                  to itself defined by $q \mapsto kq$ is an automorphism of
                  $\Q$.

      \textbf{Proof.} Let $k$ be a nonzero rational number. Consider the map
      $f : \Q \rightarrow \Q$, $q \mapsto kq$. Let $x, y \in Q$. We have that
      $f(x + y) = k(x + y) = kx + ky = f(x) + f(y)$, so that $f$ is a 
      homomorphism. Now suppose $f(x) = f(y)$, so that $kx = ky$. Since
      $k \neq 0$, we shall multiply the equality $kx = ky$ by $1/k$ to get
      $x = y$; i.e, $f$ is injective. Since $f(x/k) = x$, it follows that $f$ is
      onto, so that $f$ is an automorphism of $\Q$. \qed
%%%%%%%%%%%%%%%%%%%%%%%%%%%%%%%%%%%%%1.6.22%%%%%%%%%%%%%%%%%%%%%%%%%%%%%%%%%%%%%
   \item[1.6.22]  Let $A$ be an abelian group and fix some $k \in \Z$. Prove
                  that the map $a \mapsto a^k$ is a homomorphism from $A$ to 
                  itself. If $k = -1$ prove that this homomorphism is an
                  isomorphism.

      \textbf{Proof.} Let $k$ be an integer, $x, y \in A$. Consider the map
      $f : A \rightarrow A$, $a \mapsto a^k$. Since $A$ is abelian, it follows
      by definition and Exercise 1.1.24 that
      $f(xy) = (xy)^k = x^ky^k = f(x)f(y)$, so that $f$ is a homomorphism. Now 
      assume that $k = -1$. In this case, notice that the map $f$ is its own
      2-sided inverse; thus $f$ is bijective, so that $f$ is an isomorphism.
      \qed
%%%%%%%%%%%%%%%%%%%%%%%%%%%%%%%%%%%%%1.6.23%%%%%%%%%%%%%%%%%%%%%%%%%%%%%%%%%%%%%
   \item[1.6.23]  Let $G$ be a finite group which possesses an automorphism
                  $\sigma$ such that $\sigma(g) = g$ if and only if $g = 1$. If
                  $\sigma^2$ is the identity map from $G$ to $G$, prove that $G$
                  is abelian (such an automorphism $\sigma$ is called
                  \text{fixed point free} of order 2). [Hint. Show that every
                  element of $G$ can be written in the form $x^{-1}\sigma(x)$
                  and apply $\sigma$ to such an expression.]

      \textbf{Proof.} Consider the map $\alpha : G \rightarrow G$,
      $g \mapsto g^{-1}\sigma(g)$. Suppose that for some $x, y \in G$, we have
      that $\alpha(x) = \alpha(y)$. Then it follows that
      $x^{-1}\sigma(x) = y^{-1}\sigma(y)$, so that
      $yx^{-1} =\nobreak \sigma(y)\sigma(x)^{-1} = \sigma(yx^{-1})$. Since
      $\sigma(g) = g$ if and only if $g = 1$, we must then have that
      $yx^{-1} = 1$; thus $y = x$, so that $\alpha$ is injective; $\alpha$ is
      also surjective since $G$ is finite. So let $z \in G$. Then we must have 
      that $z = h^{-1}\sigma(h)$ for some $h \in G$. So
      $\sigma(z) = \sigma(h^{-1}\sigma(h)) = \sigma(h)^{-1}\sigma^2(h) =
       \sigma(h)^{-1}h = z^{-1}$. We can then conclude by Exercise 1.6.17 that
      $G$ is abelian. \qed
%%%%%%%%%%%%%%%%%%%%%%%%%%%%%%%%%%%%%1.6.24%%%%%%%%%%%%%%%%%%%%%%%%%%%%%%%%%%%%%
   \item[1.6.24]  Let $G$ be a finite group and let $x$ and $y$ be distinct
                  elements of order 2 in $G$ that generate $G$. Prove that
                  $G \cong D_{2n}$, where $n = |xy|$. [See Exercise 1.2.6]
                  
      \textbf{Proof.} Let $a = xy$. Since $G$ is finite, $|a|$ must also be
      finite. So write $|a| = n$. By Exercise 1.2.6, we have that
      $ax = xa^{-1}$. Also note since $y = x^2y = x(xy) = xa$, the elements
      $a$ and $x$ generate $G$. Thus we have that
      $$G = \cyc{a, x : a^n = x^2 = 1, ax = xa^{-1}}.$$
      By the discussion on Page 38-39 of the Textbook, the map
      $\varphi : D_{2n} \rightarrow G$, given by $\varphi(r) = a$ and
      $\varphi(s) = x$ is an isomorphism. Hence $G \cong D_{2n}$. \qed      
%%%%%%%%%%%%%%%%%%%%%%%%%%%%%%%%%%%%%1.6.25%%%%%%%%%%%%%%%%%%%%%%%%%%%%%%%%%%%%%
   \item[1.6.25]  Let $n \in \Z^+$, let $r$ and $s$ be the usual generators of
                  $D_{2n}$ and let $\theta = 2\pi/n$.
                  \begin{enumerate}
                     \item Prove that the matrix
                           $\left(\begin{tabular}{@{}cr@{}}
                              $\cos\theta$ & $-\sin\theta$ \\
                              $\sin\theta$ & $\cos\theta$
                           \end{tabular}\right)$ is the matrix of the linear
                           transformation which rotates the $x$, $y$ plane about
                           the origin in a counterclockwise direction by
                           $\theta$ radians.
                     \item Prove that the map
                           $\varphi : D_{2n} \rightarrow GL_2(\R)$ defined on
                           generators by
                           $$\varphi(r) = \left(\begin{tabular}{@{}cr@{}}
                              $\cos\theta$ & $-\sin\theta$ \\
                              $\sin\theta$ & $\cos\theta$
                           \end{tabular}\right) \quad\text{and}\quad\varphi(s) =
                           \left(\begin{tabular}{@{}cc@{}}
                              0 & 1 \\
                              1 & 0
                           \end{tabular}\right)$$
                           extends to a homomorphism of $D_{2n}$ into
                           $GL_2(\R)$.
                     \item Prove that the homomorphism $\varphi$ in part (b) is
                           injective.
                  \end{enumerate}
                  
      \textbf{Proof.}
      
      \begin{enumerate}
         \item Consider $T_\theta : \R^2 \rightarrow \R^2$,
               $(x, y) \mapsto (x\cos\theta - y\sin\theta,
               x\sin\theta + y\cos\theta)$, the linear transformation which
               rotates the cartesian plane about the origin in a
               counterclockwise direction by $\theta$ radians. Consider the
               standard basis $\{(1, 0), (0, 1)\}$ for $\R^2$. Since
               $T_\theta(1, 0) = (\cos\theta, \sin\theta)$ and
               $T_\theta(0, 1) = (-\sin\theta, \cos\theta)$, it follows that the
               matrix of $T_\theta$ with respect to the standard basis is
               $\left(\begin{tabular}{@{}cr@{}}
                  $\cos\theta$ & $-\sin\theta$ \\
                  $\sin\theta$ & $\cos\theta$
               \end{tabular}\right)$. 
         \item It suffices to show that $\varphi(r)$ and $\varphi(s)$ satisfy
               (in $GL_2(\R)$) the relations satisfied by $r$ and $s$
               (in $D_{2n}$). It is clear that $\varphi(s)^2$ is the identity
               matrix. Also $\varphi(r)^n$ is the identity matrix because it
               represents a rotation by $(2\pi/n) \cdot n = 2\pi$. Finally we 
               have that
               $$\varphi(r)\varphi(s) = \left(\begin{tabular}{@{}rc@{}}
                  $-\sin\theta$ & $\cos\theta$ \\
                  $\cos\theta$ & $\sin\theta$
               \end{tabular}\right) = \varphi(s)\varphi(r)^{-1},$$
               as desired.
         \item Let $x \in \text{ker}(\varphi)$, so that $\varphi(x) = 1$.

               \textbf{Case 1.} $x = sr^t$, $0 \le t < n$. It follows by
               Exercise 1.6.1 that
               $$\varphi(s)\varphi(s) = 1 = \varphi(x) = \varphi(sr^t) =
                 \varphi(s)\varphi(r)^t,$$
               so that $\varphi(s) = \varphi(r)^t$, by cancellation. Observe
               that the matrix $\varphi(r)^t$ represents rotation by
               $t \cdot \theta$. So the equality $\varphi(s) = \varphi(r)^t$
               implies that
               $$\left(\begin{tabular}{@{}cc@{}}
                  0 & 1 \\
                  1 & 0
               \end{tabular}\right) = \left(\begin{tabular}{@{}cr@{}}
                  $\cos t\theta$ & $-\sin t\theta$ \\
                  $\sin t\theta$ & $\cos t\theta$
               \end{tabular}\right).$$
               That is, $\sin(t\theta) = 1$ and $-\sin(t\theta) = 1$, so that
               $\sin(t\theta) = -1$, a contradiction. Thus $sr^t$ is not in the
               kernel of $\varphi$.

               \textbf{Case 2.} $x = r^t$, $0 \le t < n$. By Exercise 1.6.1, we
               have that $1 = \varphi(x) = \varphi(r^t) = \varphi(r)^t$, so that
               $$\left(\begin{tabular}{@{}cc@{}}
                  1 & 0 \\
                  0 & 1
               \end{tabular}\right) = \left(\begin{tabular}{@{}cr@{}}
                  $\cos t\theta$ & $-\sin t\theta$ \\
                  $\sin t\theta$ & $\cos t\theta$
               \end{tabular}\right).$$
               It follows that $\sin(t\theta) = 0$ and $\cos(t\theta) = 1$. But
               this only occurs when $t\theta = 2\pi k$, for some integer $k$.
               Since $\theta = 2\pi/n$, it follows that $t = nk$. Since
               $0 \le t < n$, conclude that $k = 0$, so that $t = 0$. Thus
               $x = r^0 = 1$. That is, the kernel of $\varphi$ is trivial. So
               conclude by Exercise 1.6.14 that $\varphi$ is injective. \qed
      \end{enumerate}
%%%%%%%%%%%%%%%%%%%%%%%%%%%%%%%%%%%%%1.6.26%%%%%%%%%%%%%%%%%%%%%%%%%%%%%%%%%%%%%
   \item[1.6.26]  Let $i$ and $j$ be the generators of $Q_8$ described in
                  Section 5. Prove that the map $\varphi$ from $Q_8$ to
                  $GL_2(\C)$ defined on generators by
                  $$\varphi(i) = \left(\begin{tabular}{@{}cc@{}}
                        $\sqrt{-1}$ & 0 \\
                        0         &  $-\sqrt{-1}$
                    \end{tabular}\right) \qquad \text{and} \qquad
                    \varphi(j) = \left(\begin{tabular}{@{}cr@{}}
                        0 & $-1$ \\
                        1 & 0
                    \end{tabular}\right)$$
                  extends to a homomorphism. Prove that $\varphi$ is injective.

      \textbf{Proof.} To show that $\varphi$ extends to an homomorphism, it
      suffices to show that $\varphi(i)$ and $\varphi(j)$ satisfy
      (in $GL_2(\C)$) the relations satisfied by $i$ and $j$ (in $Q_8$). Using
      Exercise 1.5.3, we must then show that $\varphi(i)^2 = \varphi(j)^2$ and
      $\varphi(i)^4 = 1$. These equalities immediately follow using matrix
      multiplication. Since
      \begin{align*}
         \varphi(1) &= \varphi(i^4) = \varphi(i)^4 =
            \left(\begin{tabular}{@{}cc@{}}
               1 & 0 \\
               0 & 1
            \end{tabular}\right), \\
         \varphi(-1) &= \varphi(i^2) = \varphi(i)^2 =
            \left(\begin{tabular}{@{}rr@{}}
               $-1$ & 0 \\
               0 & $-1$
            \end{tabular}\right), \\
         \varphi(i) &= \left(\begin{tabular}{@{}cc@{}}
               $\sqrt{-1}$ & 0 \\
               0         &  $-\sqrt{-1}$
            \end{tabular}\right), \\
         \varphi(-i) &= \varphi(-1\cdot i) = \varphi(-1)\varphi(i) =
            \left(\begin{tabular}{@{}cc@{}}
               $-\sqrt{-1}$ & 0 \\
               0         &  $\sqrt{-1}$
            \end{tabular}\right), \\
         \varphi(j) &= \left(\begin{tabular}{@{}cr@{}}
               0 & $-1$ \\
               1 & 0
            \end{tabular}\right), \\
         \varphi(-j) &= \varphi(-1\cdot j) = \varphi(-1)\varphi(j) =
            \left(\begin{tabular}{@{}rc@{}}
               0 & 1 \\
               $-1$ & 0
            \end{tabular}\right), \\
         \varphi(k) &= \varphi(i\cdot j) = \varphi(i)\varphi(j) =
            \left(\begin{tabular}{@{}cc@{}}
               0 & $-\sqrt{-1}$ \\
               $-\sqrt{-1}$ & 0
            \end{tabular}\right), \\
         \varphi(-k) &= \varphi(-1\cdot k) = \varphi(-1)\varphi(k) =
            \left(\begin{tabular}{@{}cc@{}}
               0 & $\sqrt{-1}$ \\
               $\sqrt{-1}$ & 0
            \end{tabular}\right),
      \end{align*}
      conclude that $\varphi$ is injective.
      \qed
\end{enumerate}
