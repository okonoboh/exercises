Let $G$ and $H$ be groups.
\begin{enumerate}
%%%%%%%%%%%%%%%%%%%%%%%%%%%%%%%%%%%%%1.6.1%%%%%%%%%%%%%%%%%%%%%%%%%%%%%%%%%%%%%%
   \item[1.6.1]   Let $\varphi : G \rightarrow H$ be a homomorphism.
                  \begin{enumerate}
                     \item Prove that $\varphi(x^n) = \varphi(x)^n$ for all
                           $n \in \Z^+$.
                     \item Do part (a) for $n = -1$ and deduce that
                           $\varphi(x^n) = \varphi(x)^n$ for all $n \in \Z$.
                  \end{enumerate}
%%%%%%%%%%%%%%%%%%%%%%%%%%%%%%%%%%%%%1.6.2%%%%%%%%%%%%%%%%%%%%%%%%%%%%%%%%%%%%%%
   \item[1.6.2]   If $\varphi : G \rightarrow H$ is an isomorphism, prove that
                  $|\varphi(x)| = |x|$ for all $x \in G$. Deduce that any two
                  isomorphic groups have the same number of elements of order
                  $n$ for each $n \in \Z^+$. Is the result true if $\varphi$ is
                  only assumed to be a homomorphism?
%%%%%%%%%%%%%%%%%%%%%%%%%%%%%%%%%%%%%1.6.3%%%%%%%%%%%%%%%%%%%%%%%%%%%%%%%%%%%%%%
   \item[1.6.3]   If $\varphi : G \rightarrow H$ is an isomorphism, prove that
                  $G$ is abelian if and only if $H$ is abelian. If
                  $\varphi : G \rightarrow H$ is a homomorphism, what additional
                  conditions on $\varphi$ (if any) are sufficient to ensure that
                  if $G$ is abelian, then so is $H$?
%%%%%%%%%%%%%%%%%%%%%%%%%%%%%%%%%%%%%1.6.4%%%%%%%%%%%%%%%%%%%%%%%%%%%%%%%%%%%%%%
   \item[1.6.4]   Prove that the multiplicative groups $\R - \{0\}$ and
                  $\C - \{0\}$ are not isomorphic.
%%%%%%%%%%%%%%%%%%%%%%%%%%%%%%%%%%%%%1.6.5%%%%%%%%%%%%%%%%%%%%%%%%%%%%%%%%%%%%%%
   \item[1.6.5]   Prove that the additive groups of $\R$ and $\Q$ are not
                  isomorphic.
%%%%%%%%%%%%%%%%%%%%%%%%%%%%%%%%%%%%%1.6.6%%%%%%%%%%%%%%%%%%%%%%%%%%%%%%%%%%%%%%
   \item[1.6.6]   Prove that the additive groups of $\Z$ and $\Q$ are not
                  isomorphic.
%%%%%%%%%%%%%%%%%%%%%%%%%%%%%%%%%%%%%1.6.7%%%%%%%%%%%%%%%%%%%%%%%%%%%%%%%%%%%%%%
   \item[1.6.7]   Prove that $D_8$ and $Q_8$ are not isomorphic.
%%%%%%%%%%%%%%%%%%%%%%%%%%%%%%%%%%%%%1.6.8%%%%%%%%%%%%%%%%%%%%%%%%%%%%%%%%%%%%%%
   \item[1.6.8]   Prove that if $n \neq m$, $S_n$ and $S_m$ are not isomorphic.
%%%%%%%%%%%%%%%%%%%%%%%%%%%%%%%%%%%%%1.6.9%%%%%%%%%%%%%%%%%%%%%%%%%%%%%%%%%%%%%%
   \item[1.6.9]   Prove that $D_{24}$ and $S_4$ are not isomorphic.
%%%%%%%%%%%%%%%%%%%%%%%%%%%%%%%%%%%%%1.6.10%%%%%%%%%%%%%%%%%%%%%%%%%%%%%%%%%%%%%
   \item[1.6.10]  Fill in the details of the proof that the symmetric groups
                  $S_{\triangle}$ and $S_{\Omega}$ are isomorphic if
                  $|\triangle| = |\Omega|$ as follows: let
                  $\theta : \triangle \rightarrow \Omega$ be a bijection. Define
                  $$\varphi : S_{\triangle} \rightarrow S_{\Omega} \qquad
                    \text{by} \qquad \varphi(\sigma) = \theta \circ \sigma \circ 
                    \theta^{-1} \text{ for all } \sigma \in S_{\triangle}$$
                  and prove the following:
                  \begin{enumerate}
                     \item $\varphi$ is well defined, that is, if $\sigma$ is a
                           permutation of $\triangle$ then
                           $\theta\circ\sigma\circ\theta^{-1}$ is a permutation
                           of $\Omega$.
                     \item $\varphi$ is a bijection from $S_{\triangle}$ onto
                           $S_{\Omega}$. [Find a 2-sided inverse for $\varphi$.]
                     \item $\varphi$ is a homomorphism, that is,
                           $\varphi(\sigma\circ\tau) =
                            \varphi(\sigma)\circ\varphi(\tau)$.
                  \end{enumerate}
                  Note the similarity to the \textit{change of basis} or
                  \textit{similarity} transformations for matrices. 
%%%%%%%%%%%%%%%%%%%%%%%%%%%%%%%%%%%%%1.6.11%%%%%%%%%%%%%%%%%%%%%%%%%%%%%%%%%%%%%
   \item[1.6.11]  Let $A$ and $B$ be groups. Prove that
                  $A \times B \cong B\times A$.
%%%%%%%%%%%%%%%%%%%%%%%%%%%%%%%%%%%%%1.6.12%%%%%%%%%%%%%%%%%%%%%%%%%%%%%%%%%%%%%
   \item[1.6.12]  Let $A$, $B$, and $C$ be groups and let $G = A \times B$ and
                  $H = B \times C$. Prove that $G \times C \cong A \times H$.
%%%%%%%%%%%%%%%%%%%%%%%%%%%%%%%%%%%%%1.6.13%%%%%%%%%%%%%%%%%%%%%%%%%%%%%%%%%%%%%
   \item[1.6.13]  Let $G$ and $H$ be groups and let $\varphi : G \rightarrow H$
                  be a homomorphism. Prove that the image of $\varphi$,
                  $\varphi(G)$, is a subgroup of $H$. Prove that if $\varphi$ is
                  injective then $G \cong \varphi(G)$.
%%%%%%%%%%%%%%%%%%%%%%%%%%%%%%%%%%%%%1.6.14%%%%%%%%%%%%%%%%%%%%%%%%%%%%%%%%%%%%%
   \item[1.6.14]  Let $G$ and $H$ be groups and let $\varphi : G \rightarrow H$
                  be a homomorphism. Define the \text{kernel} of $\varphi$ to be
                  $\{g \in G : \varphi(g) = 1_H\}$. Prove that the kernel of
                  $\varphi$ is a subgroup of $G$. Prove that $\varphi$ is
                  injective if and only if the kernel of $\varphi$ is the
                  identity subgroup of $G$.
%%%%%%%%%%%%%%%%%%%%%%%%%%%%%%%%%%%%%1.6.15%%%%%%%%%%%%%%%%%%%%%%%%%%%%%%%%%%%%%
   \item[1.6.15]  Define a map $\pi : \R^2 \rightarrow \R$ by $\pi((x, y)) = x$.
                  Prove that $\pi$ is a homomorphism and find the kernel of
                  $\pi$.
%%%%%%%%%%%%%%%%%%%%%%%%%%%%%%%%%%%%%1.6.16%%%%%%%%%%%%%%%%%%%%%%%%%%%%%%%%%%%%%
   \item[1.6.16]  Let $A$ and $B$ be groups and let $G$ be their direct product,
                  $A \times B$. Prove that the maps $\pi_1 : G \rightarrow A$
                  and $\pi_2 : G \rightarrow B$ defined by $\pi_1((a, b)) = a$
                  and $\pi_2((a, b)) = b$ are homomorphisms and find their
                  kernels.
%%%%%%%%%%%%%%%%%%%%%%%%%%%%%%%%%%%%%1.6.17%%%%%%%%%%%%%%%%%%%%%%%%%%%%%%%%%%%%%
   \item[1.6.17]  Let $G$ be any group. Prove that the map from $G$ to itself
                  defined by $g \mapsto g^{-1}$ is a homomorphism if and only if
                  $G$ is abelian.
%%%%%%%%%%%%%%%%%%%%%%%%%%%%%%%%%%%%%1.6.18%%%%%%%%%%%%%%%%%%%%%%%%%%%%%%%%%%%%%
   \item[1.6.18]  Let $G$ be any group. Prove that the map from $G$ to itself
                  defined by $g \mapsto g^2$ is a homomorphism if and only if
                  $G$ is abelian.
%%%%%%%%%%%%%%%%%%%%%%%%%%%%%%%%%%%%%1.6.19%%%%%%%%%%%%%%%%%%%%%%%%%%%%%%%%%%%%%
   \item[1.6.19]  Let $G = \{z \in \C : z^n = 1 \text{ for some }n \in \Z^+\}$.
                  Prove that for any fixed integer $k > 1$ the map from $G$ to
                  itself defined by $z \mapsto z^k$ is a surjective homomorphism
                  but is not an isomorphism.
%%%%%%%%%%%%%%%%%%%%%%%%%%%%%%%%%%%%%1.6.20%%%%%%%%%%%%%%%%%%%%%%%%%%%%%%%%%%%%%
   \item[1.6.20]  Let $G$ be a group and let Aut($G$) be the set of all
                  isomorphisms from $G$ onto $G$. Prove that Aut($G$) is a
                  group under function composition (called the
                  \text{automorphism group} of $G$ and the elements of Aut($G$)
                  are called \text{automorphisms} of $G$).
%%%%%%%%%%%%%%%%%%%%%%%%%%%%%%%%%%%%%1.6.21%%%%%%%%%%%%%%%%%%%%%%%%%%%%%%%%%%%%%
   \item[1.6.21]  Prove that for each fixed nonzero $k \in \Q$ the map from $\Q$
                  to itself defined by $q \mapsto kq$ is an automorphism of
                  $\Q$.
%%%%%%%%%%%%%%%%%%%%%%%%%%%%%%%%%%%%%1.6.22%%%%%%%%%%%%%%%%%%%%%%%%%%%%%%%%%%%%%
   \item[1.6.22]  Let $A$ be an abelian group and fix some $k \in \Z$. Prove
                  that the map $a \mapsto a^k$ is a homomorphism from $A$ to 
                  itself. If $k = -1$ prove that this homomorphism is an
                  isomorphism.
%%%%%%%%%%%%%%%%%%%%%%%%%%%%%%%%%%%%%1.6.23%%%%%%%%%%%%%%%%%%%%%%%%%%%%%%%%%%%%%
   \item[1.6.23]  Let $G$ be a finite group which possesses an automorphism
                  $\sigma$ such that $\sigma(g) = g$ if and only if $g = 1$. If
                  $\sigma^2$ is the identity map from $G$ to $G$, prove that $G$
                  is abelian (such an automorphism $\sigma$ is called
                  \text{fixed point free} of order 2). [Hint. Show that every
                  element of $G$ can be written in the form $x^{-1}\sigma(x)$
                  and apply $\sigma$ to such an expression.]
%%%%%%%%%%%%%%%%%%%%%%%%%%%%%%%%%%%%%1.6.24%%%%%%%%%%%%%%%%%%%%%%%%%%%%%%%%%%%%%
   \item[1.6.24]  Let $G$ be a finite group and let $x$ and $y$ be distinct
                  elements of order 2 in $G$ that generate $G$. Prove that
                  $G \cong D_{2n}$, where $n = |xy|$. [See Exercise 1.2.6]
%%%%%%%%%%%%%%%%%%%%%%%%%%%%%%%%%%%%%1.6.25%%%%%%%%%%%%%%%%%%%%%%%%%%%%%%%%%%%%%
   \item[1.6.25]  Let $n \in \Z^+$, let $r$ and $s$ be the usual generators of
                  $D_{2n}$ and let $\theta = 2\pi/n$.
                  \begin{enumerate}
                     \item Prove that the matrix
                           $\left(\begin{tabular}{@{}cr@{}}
                              $\cos\theta$ & $-\sin\theta$ \\
                              $\sin\theta$ & $\cos\theta$
                           \end{tabular}\right)$ is the matrix of the linear
                           transformation which rotates the $x$, $y$ plane about
                           the origin in a counterclockwise direction by
                           $\theta$ radians.
                     \item Prove that the map
                           $\varphi : D_{2n} \rightarrow GL_2(\R)$ defined on
                           generators by
                           $$\varphi(r) = \left(\begin{tabular}{@{}cr@{}}
                              $\cos\theta$ & $-\sin\theta$ \\
                              $\sin\theta$ & $\cos\theta$
                           \end{tabular}\right) \quad\text{and}\quad\varphi(s) =
                           \left(\begin{tabular}{@{}cc@{}}
                              0 & 1 \\
                              1 & 0
                           \end{tabular}\right)$$
                           extends to a homomorphism of $D_{2n}$ into
                           $GL_2(\R)$.
                     \item Prove that the homomorphism $\varphi$ in part (b) is
                           injective.
                  \end{enumerate}
%%%%%%%%%%%%%%%%%%%%%%%%%%%%%%%%%%%%%1.6.26%%%%%%%%%%%%%%%%%%%%%%%%%%%%%%%%%%%%%
   \item[1.6.26]  Let $i$ and $j$ be the generators of $Q_8$ described in
                  Section 5. Prove that the map $\varphi$ from $Q_8$ to
                  $GL_2(\C)$ defined on generators by
                  $$\varphi(i) = \left(\begin{tabular}{@{}cc@{}}
                        $\sqrt{-1}$ & 0 \\
                        0         &  $-\sqrt{-1}$
                    \end{tabular}\right) \qquad \text{and} \qquad
                    \varphi(j) = \left(\begin{tabular}{@{}cc@{}}
                        0 & $-1$ \\
                        1 & 0
                    \end{tabular}\right)$$
                  extends to a homomorphism. Prove that $\varphi$ is injective.
\end{enumerate}
