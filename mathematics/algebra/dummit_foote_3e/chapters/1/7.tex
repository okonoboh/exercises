\begin{enumerate}
%%%%%%%%%%%%%%%%%%%%%%%%%%%%%%%%%%%%%1.7.1%%%%%%%%%%%%%%%%%%%%%%%%%%%%%%%%%%%%%%
   \item[1.7.1]   Let $F$ be a field. Show that the multiplicative group of
                  nonzero elements of $F$ (denoted by $F^\times$) acts on the
                  set $F$ by $g \cdot a = ga$, where $g \in F^\times$,
                  $a \in F$ and $ga$ is the usual product in $F$ of the two
                  field elements (state clearly which axioms in the definition
                  of a field are used).

      \textbf{Proof.} Let $a, b \in F^\times$ and $c \in F$. It is clear
      that $ab \in F$ because $F$ is closed under multiplication. Since 1 is
      is the identity for $(F^\times, \cdot)$, we have that $1 \cdot c = f_3$.
      By the associativity of multiplication in $F$, it follows that
      $a(bc) = (ab)c$. Thus $F^\times$ acts on $F$ by multiplication. \qed
%%%%%%%%%%%%%%%%%%%%%%%%%%%%%%%%%%%%%1.7.2%%%%%%%%%%%%%%%%%%%%%%%%%%%%%%%%%%%%%%
   \item[1.7.2]   Show that the additive group $\Z$ acts on itself by
                  $z \cdot a = z + a$ for all $z, a \in \Z$.

      \textbf{Proof.} Let $a$, $b$, and $c$ be integers. Since $\Z$ is closed
      under addition, we have that $a \cdot b = a + b \in \Z$. Notice that the 
      identity of $(\Z, +)$ is 0; thus $a \cdot 0 = a + 0 = a$. Using the 
      associativity of $\Z$ under +, we have that
      $$a \cdot (b \cdot c) = a + (b + c) = (a + b) + c = (a \cdot b) \cdot c.$$
      Thus $\Z$ acts on itself by the given operation. \qed
%%%%%%%%%%%%%%%%%%%%%%%%%%%%%%%%%%%%%1.7.3%%%%%%%%%%%%%%%%%%%%%%%%%%%%%%%%%%%%%%
   \item[1.7.3]   Show that the additive group $\R$ acts on the $x, y$ plane
                  $\R \times \R$ by $r \cdot (x, y) = (x + ry, y)$.

      \textbf{Proof.} Let $a$, $b$, $c$, and $d$ be real numbers. Since $\R$ is
      closed under addition, we have that
      $a \cdot (b, c) = (b + ac, c) \in \R \times \R$. We also have that
      $$0 \cdot (a, b) = (a + 0 \cdot b, b) = (a, b).$$
      Finally we have that
      \begin{align*}
         a \cdot (b \cdot (c, d)) &= a \cdot (c + bd, d) \\
            &= (c + bd + ad, d) \\
            &= (c + (a + b)d, d) \\
            &= (a + b) \cdot (c, d), \\
      \end{align*}
      so that $\R$ acts on the cartesian plane by the given operation. \qed
%%%%%%%%%%%%%%%%%%%%%%%%%%%%%%%%%%%%%1.7.4%%%%%%%%%%%%%%%%%%%%%%%%%%%%%%%%%%%%%%
   \item[1.7.4]   Let $G$ be a group acting on a set $A$ and fix some $a \in A$.
                  Show that the following sets are subgroups of $G$.
                  \begin{enumerate}
                     \item the kernel of the action,
                     \item $\{g \in G : ga = a\}$---this subgroup is called the
                           \textit{stablizer} of $a$ in $G$.
                  \end{enumerate}

      \textbf{Proof.}

      \begin{enumerate}
         \item Let $K$ be the kernel of this action. The set $K$ is not empty
               because $1 \in K$. Now let $g, h \in K$ and $x \in A$. So we
               have that
               \begin{align*}
                  (gh)x &= g(hx) \\
                     &= gx &[hx = x \text{ since } h \in K] \\
                     &= x; &[gx = x \text{ since } g \in K]
               \end{align*}
               thus $gh \in K$. Finally
               $$h^{-1}x = h^{-1}(hx) = (h^{-1}h)x = 1x = x,$$
               so that $h^{-1} \in K$. Thus, $K$ is a subgroup of $G$ by
               Exercise 1.1.26.

         \item Let $G_a$ be the stabilizer of $a$. Argue similarly as in (a),
               replacing $x$ with $a$ and $K$ with $G_a$ to conclude that $G_a$ 
               is a subgroup of $G$.
      \end{enumerate}  \qed
%%%%%%%%%%%%%%%%%%%%%%%%%%%%%%%%%%%%%1.7.5%%%%%%%%%%%%%%%%%%%%%%%%%%%%%%%%%%%%%%
   \item[1.7.5]   Prove that the kernel of an action of the group $G$ on the set
                  $A$ is the same as the kernel of the corresponding permutation
                  representation $G \rightarrow S_A$.

      \textbf{Proof.} Let $K$ be the kernel of the action of $G$ on $A$ and let
      $K'$ be the kernel of the permutation representation
      $\varphi : G \rightarrow S_A$. The proof follows since
      \begin{align*}
         g \in K &\Longleftrightarrow ga = a &\quad\text{ for each } a \in A \\
            &\Longleftrightarrow (\varphi(g))(a)= a
               &\quad\text{ for each } a\in A \\
            &\Longleftrightarrow\varphi(g)\text{ is the identity permutation} \\
            &\Longleftrightarrow g \in K'.
      \end{align*} \qed
%%%%%%%%%%%%%%%%%%%%%%%%%%%%%%%%%%%%%1.7.6%%%%%%%%%%%%%%%%%%%%%%%%%%%%%%%%%%%%%%
   \item[1.7.6]   Prove that a group $G$ acts faithfully on a set $A$ if and
                  only if the kernel of the action is the set consisting only of
                  the identity.

      \textbf{Proof.} Let $\varphi : G \rightarrow S_A$ be the corresponding 
      permutation representation.

      ($\Rightarrow$) Assume that $G$ acts faithfully on $A$. That is, if
      $g, h \in G$, with $g \neq h$, then $\varphi(g) \neq \varphi(h)$, or
      equivalently $\varphi$ is injective. So let $g$ be an element of the
      kernel of this action. Then it follows that
      $\varphi(g) = 1 = \varphi(1)$, so that $g = 1$, and it follows that the 
      kernel of the action is trivial.

      ($\Leftarrow$) Conversely, suppose the kernel of this action is trivial.
      Suppose $xa = ya$ for some $x, y \in G$ and for each $a \in A$. Then
      $$(y^{-1}x)a = y^{-1}(xa) = y^{-1}(ya) = (y^{-1}y)a = 1a = a,$$
      for each $a \in A$. That is, $y^{-1}x$ is in the kernel of the action;
      but this kernel is trivial, so that $y^{-1}x = 1$, and thus $x = y$. It
      follows that distinct elements in $G$ induce distinct permutations of $A$.
      Conclude that $G$ acts faithfully on $A$. \qed
%%%%%%%%%%%%%%%%%%%%%%%%%%%%%%%%%%%%%1.7.7%%%%%%%%%%%%%%%%%%%%%%%%%%%%%%%%%%%%%%
   \item[1.7.7]   Prove that in Example 2 in this section the action is
                  faithful when the vector space is nonzero.

      \textbf{Proof.} Consider a nonzero vector space $V$ over $F$. Let
      $F^\times$ act on $V$ by left multiplication. Suppose to the contrary that
      two unequal members of $F^\times$---say $\alpha$ and $\beta$---induce the 
      same permutation of $V$. Let $v$ be a nonzero element in $V$. Then we must 
      particularly have that $\alpha v = \beta v$, so that
      $(\alpha - \beta)v = 0$. Since $\alpha \neq \beta$, it follows that
      $\alpha - \beta \neq 0$, so that $\alpha - \beta$ has a multiplicative 
      inverse in $F^\times$. Multiply the equality $(\alpha - \beta)v = 0$ (on
      the left) by this inverse to get $v = 0$, a contradiction. Thus no two 
      members of $F^\times$ induce the same permutation, so that the action of
      $F^\times$ on $V$ is faithful. \qed
%%%%%%%%%%%%%%%%%%%%%%%%%%%%%%%%%%%%%1.7.8%%%%%%%%%%%%%%%%%%%%%%%%%%%%%%%%%%%%%%
   \item[1.7.8]   Let $A$ be a nonempty set and let $k$ be a positive integer
                  with $k \le |A|$. The symmetric group $S_A$ acts on the set
                  $B$ consisting of all subsets of $A$ of cardinality $k$ by
                  $\sigma \cdot \{a_1, \ldots, a_k\} = \{\sigma(a_1), \ldots,
                   \sigma(a_k)\}$.
                  \begin{enumerate}
                     \item Prove that this is a group action.
                     \item Describe explicitly how the elements (1 2) and
                           (1 2 3) act on the six 2-element subsets of
                           $\{1, 2, 3, 4\}$.
                  \end{enumerate}

      \textbf{Solution.}
   
      \begin{enumerate}
         \item \textbf{Proof.} Let $\sigma, \alpha \in S_A$ and
               $\{a_1, \ldots, a_k\} \in B$. Since $\sigma$ is injective, it 
               follows that $\sigma(a_1)$, $\ldots$, $\sigma(a_k)$ are all 
               distinct, so that $\{\sigma(a_1), \ldots, \sigma(a_k)\} \in B$.
               Now $1 \cdot\nobreak \{a_1, \ldots, a_k\} = \{a_1, \ldots, a_k\}$
               and
               \begin{align*}
                  \sigma \cdot (\alpha \cdot \{a_1, \ldots, a_k\}) &=
                     \sigma \cdot \{\alpha(a_1), \ldots, \alpha(a_k)\} \\
                     &= \{\sigma(\alpha(a_1)), \ldots, \sigma(\alpha(a_k))\} \\
                     &= \{(\sigma\circ\alpha)(a_1), \ldots,
                          (\sigma\circ\alpha)(a_k)\} \\
                     &= (\sigma\circ\alpha) \cdot \{a_1, \ldots, a_k\}.
               \end{align*}
               
               Thus the operation is a group action. \qed
         \item $$
               \begin{tabular}{@{}|c|c|c|c|c|c|c|@{}} \hline
                  $\cdot$ & $\{1\;2\}$ & $\{1\;3\}$ & $\{1\;4\}$ & $\{2\;3\}$ &
                            $\{2\;4\}$ & $\{3\;4\}$ \\ \hline
                  (1 2)   & $\{1\;2\}$ & $\{2\;3\}$ & $\{2\;4\}$ & $\{1\;3\}$ &
                            $\{1\;4\}$ & $\{3\;4\}$ \\ \hline
                  (1 2 3) & $\{2\;3\}$ & $\{1\;2\}$ & $\{2\;4\}$ & $\{1\;3\}$ &
                            $\{3\;4\}$ & $\{1\;4\}$ \\ \hline
               \end{tabular}
               $$
      \end{enumerate}
%%%%%%%%%%%%%%%%%%%%%%%%%%%%%%%%%%%%%1.7.9%%%%%%%%%%%%%%%%%%%%%%%%%%%%%%%%%%%%%%
   \item[1.7.9]   Do both parts of the preceding exercise with ``ordered
                  $k$-tuples" in place of ``$k$-element subsets," where the
                  action on $k$-tuples is defined as above but with set braces
                  replaced by parentheses (note that, for example, the 2-tuples
                  (1, 2) and (2, 1) are different even though the sets
                  $\{1, 2\}$ and $\{2, 1\}$ are the same, so the sets being
                  acted upon are different).

      \textbf{Solution.}
   
      \begin{enumerate}
         \item The proof is similar to Exercise 1.7.8 (a).
         \item $$
               \begin{tabular}{@{}|c|c|c|c|c|c|c|c|c|c|c|c|c|@{}} \hline
                  $\cdot$ & (1 2) & (2 1) & (1 4) & (4 1) &
                            (2 4) & (4 2) \\ \hline
                  (1 2)   & (2 1) & (1 2) & (2 4) & (4 2) &
                            (1 4) & (4 1) \\ \hline
                  (1 2 3) & (2 3) & (2 1) & (2 4) & (3 1) &
                            (3 4) & (1 4) \\ \hline
               \end{tabular}
               $$
               and
               $$
               \begin{tabular}{@{}|c|c|c|c|c|c|c|c|c|c|c|c|c|@{}} \hline
                  $\cdot$ & (1 3) & (3 1) & (2 3) & (3 2) &
                            (3 4) & (4 3) \\ \hline
                  (1 2)   & (2 3) & (3 2) & (1 3) & (3 1) &
                            (3 4) & (4 3) \\ \hline
                  (1 2 3) & (2 1) & (1 2) & (3 1) & (1 3) &
                            (1 4) & (4 1) \\ \hline
               \end{tabular}
               $$
      \end{enumerate}
%%%%%%%%%%%%%%%%%%%%%%%%%%%%%%%%%%%%%1.7.10%%%%%%%%%%%%%%%%%%%%%%%%%%%%%%%%%%%%%
   \item[1.7.10]  With reference to the preceding two exercises determine:
                  \begin{enumerate}
                     \item for which values of $k$ the action of $S_n$ on
                           $k$-element subsets is faithful, and
                     \item for which values of $k$ the action of $S_n$ on
                           ordered $k$-tuples is faithful.
                  \end{enumerate}

      \textbf{Solution.}
   
      \begin{enumerate}
         \item \textit{If $k < n$, then this action is faithful}.

               \textbf{Proof.} By Exercise 1.7.6, it suffices to show that
               kernel of this action is trivial (Exercise 1.7.6). Let $\sigma$ 
               be a nonidentity element in $S_n$, so that $\sigma(i) = j$ for 
               some $i \neq j$. Let $B$ be any $k$-subset such that $i \in B$
               and $j \neq B$. Notice that $B \neq \sigma B$ since
               $j \in \sigma B$. Thus $\sigma$ is not in the kernel of this 
               action. Since $\sigma$ was an arbitrary nonidentity element, it 
               follows that the kernel of this action is trivial, so that this 
               action is faithful by Exercise 1.7.6. \qed \\
   
               \textit{If $k = n$, then this action is not faithful}.

               \textbf{Proof.} Observe that the set of all $n$-element subsets 
               contains a single element, $I_n = \{1, \ldots, n\}$. Thus every 
               element of $S_n$ must map $I_n$ to itself, so that every element 
               of $S_n$ induces the same permutation. \qed
         \item \textit{This action is faithful for all $k \le n$}.

               \textbf{Proof.} Let $\sigma$ be a nonidentity element in $S_n$.
               Then $\sigma(i) = j$ for some $i \neq j$. Let $B$ be any
               $k$-tuple with $i$ as its first element. Then the first element
               of $\sigma A$ is $j$; thus $A \neq \sigma A$; that is, the kernel 
               of this action is trivial, so that this action of $S_n$ on
               ordered $k$-tuples is faithful. \qed
      \end{enumerate}
%%%%%%%%%%%%%%%%%%%%%%%%%%%%%%%%%%%%%1.7.11%%%%%%%%%%%%%%%%%%%%%%%%%%%%%%%%%%%%%
   \item[1.7.11]  Write out the cycle decomposition of the eight permutations in
                  $S_4$ corresponding to the elements of $D_8$ given by the
                  action of $D_8$ on the vertices of a square (where the
                  vertices of the square are labelled as in Section 2).

      \textbf{Solution.}

      $$
         \begin{tabular}{@{}|c|c|@{}} \hline
            Element in $D_8$ & Corresponding element in $S_4$ \\ \hline         
            1      & (1) \\ \hline
            $r$    & (1 2 3 4) \\ \hline
            $r^2$  & (1 3)(2 4) \\ \hline
            $r^3$  & (1 4 3 2) \\ \hline        
            $s$    & (2 4) \\ \hline
            $sr$   & (1 4)(2 3) \\ \hline
            $sr^2$ & (1 3) \\ \hline
            $sr^3$ & (1 2)(3 4)\\ \hline
         \end{tabular}
      $$
%%%%%%%%%%%%%%%%%%%%%%%%%%%%%%%%%%%%%1.7.12%%%%%%%%%%%%%%%%%%%%%%%%%%%%%%%%%%%%%
   \item[1.7.12]  Assume $n$ is an even positive integer and show that $D_{2n}$
                  acts on the set consisting of pairs of opposite vertices of a
                  regular $n$-gon. Find the kernel of this action (label
                  vertices as usual).

      \textbf{Proof.} Assume that $n$ is an even integer greater than 3. So
      write $n = 2k$. Label the vertices of a regular $n$-gon as was done on
      Page 24 of the text. Let $I_n = \{1, 2, \ldots, n\}$ be the set of
      vertices of the $n$-gon. Also for the sake of convenience, we shall assume
      that two vertices $x$ and $y$ are equal if and only if $x \equiv y$ mod
      $n$; thus, for example, vertex 0 = vertex $n$ = vertex $-n$, and so on.
      A pair of vertices $\{a, b\}$ is said to be an opposite pair if and only 
      if $|a - b| = k$ mod $n$. Thus the set of opposite pairs, $P$, is
      $$\{\{x, x + k\} : 1 \le x \le k\} =
        \{\{1, 1 + k\}, \{2, 2 + k\}, \ldots, \{k, 2k\}\}.$$
      Let $r$, $s$ be the usual generators of $D_{2n}$. Viewing these generators
      as members of $S_n$, it follows that (in cycle notation)
      \begin{itemize}
         \item $r =$(1, 2, $\ldots$, $n$) and
         \item $s = \D\prod_{i=2}^k(i, 2k+2-i) = (2, 2k)(3, 2k-1)\cdots(k, k+2)$.
      \end{itemize}

      Now consider a pair of vertices $\{x, x + k\} \in P$. It follows that
      $$r \cdot \{x, x + k\} = \{(x + 1, x + k + 1\} \in P$$
      because $x + k + 1 - (x + 1) = k$. Assume $x = 1$. Since $s$ fixes 1 and
      $k + 1$, it must also fix $\{x, x + 1\}$. Now assume $2 \le x \le k$. Thus
      $$s \cdot \{x, x + k\} = \{2k+2 - x, 2k+2-x-k\} \in P.$$
      We have thus shown that $r$ and $s$ send every opposite pair to another
      opposite pair; since $r$ and $s$ generate it follows that every member of
      $D_{2n}$ maps an opposite pair to another opposite pair. Since we are
      regarding $r$ and $s$ as permutations in $S_n$, axiom 2 immediately
      follows by the composition of permutations. The kernel of this action is
      $\{1, r^k\}$. \qed
%%%%%%%%%%%%%%%%%%%%%%%%%%%%%%%%%%%%%1.7.13%%%%%%%%%%%%%%%%%%%%%%%%%%%%%%%%%%%%%
   \item[1.7.13]  Find the kernel of the left regular action.

      \textbf{Proof.} Let $(G, *)$ be a group. Recall that the left regular 
      action of $G$ on itself is defined by $g \cdot a = g * a$ for all
      $g, a \in G$. Let $x$ be in the kernel of this action. Then we must
      particularly have that $x = x * 1 = x \cdot 1 = 1$. Thus the kernel of the
      regular action of a group on itself is its trivial subgroup.
      \qed
%%%%%%%%%%%%%%%%%%%%%%%%%%%%%%%%%%%%%1.7.14%%%%%%%%%%%%%%%%%%%%%%%%%%%%%%%%%%%%%
   \item[1.7.14]  Let $G$ be a group and let $A = G$. Show that if $G$ is
                  non-abelian then the maps defined by $g \cdot a = ag$ for all
                  $g, a \in G$ \textit{do not} satisfy the axioms of a (left)
                  group action of $G$ on itself.

      \textbf{Proof.} Assume that $G$ is non-abelian. Suppose to the contrary 
      that the map defined by $g \cdot a = ag$ for all $g, a \in G$ satisfies
      the axioms of a (left) group action. Let $x, y, z \in G$. Then it follows
      by axiom 2 of a group action that $x \cdot (y \cdot z) = (xy) \cdot z$.
      Now
      \begin{align*}
         x \cdot (y \cdot z) &= (xy)\cdot z &\Longleftrightarrow \\
         x \cdot (zy) &= z(xy) &\Longleftrightarrow \\
         (zy)x &= z(xy) &\Longleftrightarrow \\
         z(yx) &= z(xy) &\Longleftrightarrow \\
         yx &= xy, &\Longleftrightarrow
      \end{align*}
      so that $G$ is abelian, a contradiction. Thus this map does not define a
      group action. \qed
%%%%%%%%%%%%%%%%%%%%%%%%%%%%%%%%%%%%%1.7.15%%%%%%%%%%%%%%%%%%%%%%%%%%%%%%%%%%%%%
   \item[1.7.15]  Let $G$ be any group and let $A = G$. Show that the maps
                  defined by $g \cdot a = ag^{-1}$ for all $g, a \in G$
                  \textit{do} satisfy the axioms of a (left) group action of $G$
                  on itself.

      \textbf{Proof.} We shall only verify the second axiom since the first one
      is trivial. Let $x, y, z \in G$. It follows that
      \begin{align*}
         x \cdot (y \cdot z) &= x \cdot (zy^{-1}) \\
            &= zy^{-1}x^{-1} \\
            &= z(xy)^{-1} \\
            &= (xy) \cdot z,
      \end{align*}
      so that axiom 2 holds. Thus this map defines a group action. \qed
%%%%%%%%%%%%%%%%%%%%%%%%%%%%%%%%%%%%%1.7.16%%%%%%%%%%%%%%%%%%%%%%%%%%%%%%%%%%%%%
   \item[1.7.16]  Let $G$ be any group and let $A = G$. Show that the maps
                  defined by $g \cdot a = gag^{-1}$ for all $g, a \in G$
                  \textit{do} satisfy the axioms of a (left) group action(this
                  action of $G$ on itself is called \textit{conjugation}).

      \textbf{Proof.} We shall only verify the second axiom since the first one
      is trivial. Let $x, y, z \in G$. It follows that
      \begin{align*}
         x \cdot (y \cdot z) &= x \cdot (yzy^{-1}) \\
            &= xyzy^{-1}x^{-1} \\
            &= xyz(xy)^{-1} \\
            &= (xy) \cdot z,
      \end{align*}
      so that axiom 2 holds. Thus this map defines a group action. \qed
%%%%%%%%%%%%%%%%%%%%%%%%%%%%%%%%%%%%%1.7.17%%%%%%%%%%%%%%%%%%%%%%%%%%%%%%%%%%%%%
   \item[1.7.17]  Let $G$ be a group and let $G$ act on itself by left 
                  conjugation, so each $g \in G$ maps $G$ to $G$ by
                  $$x \mapsto gxg^{-1}.$$
                  For fixed $g \in G$, prove that conjugation by $g$ is an
                  isomorphism from $G$ onto itself. Deduce that $x$ and
                  $gxg^{-1}$ have the same order for all $x \in G$ and that for
                  any subset $A$ of $G$, $|A| = |gAg^{-1}|$ (here
                  $gAg^{-1} = \{gag^{-1} : a \in A\})$.

      \textbf{Proof.} Let $g \in G$. Define $\sigma_g : G \rightarrow G$ by
      $x \mapsto gxg^{-1}$. This map is bijective since it has a 2-sided
      inverse, namely $\sigma_{g^{-1}}$. So it remains to show that $\sigma_g$
      is a homomorphism. This follows immediately because
      \begin{align*}
         \sigma_g(xy) = gxyg^{-1} = gxg^{-1}gyg^{-1} = \sigma_g(x)\sigma_g(y)
      \end{align*}
      for all $x, y \in G$. By Exercise 1.6.2, it follows that
      $|x| = |\sigma_g(x)| = |gxg^{-1}|$ for all $x \in G$. Let $A \subseteq G$.
      Then the map $\alpha_g : A \rightarrow gAg^{-1}$ defined by
      $a \mapsto gag^{-1}$ is also bijective (its 2-sided inverse is
      $\alpha_{g^{-1}}$); thus $|A| = |gAg^{-1}|$. \qed      
%%%%%%%%%%%%%%%%%%%%%%%%%%%%%%%%%%%%%1.7.18%%%%%%%%%%%%%%%%%%%%%%%%%%%%%%%%%%%%%
   \item[1.7.18]  Let $H$ be a group acting on a set $A$. Prove that the
                  relation $\sim$ on $A$ defined by
                  $$a \sim b \quad \text{if and only if} \quad
                    a = hb \quad \text{for some }h \in H$$
                  is an equivalence relation. (For each $x \in A$ the
                  equivalence class of $x$ under $\sim$ is called the
                  \textit{orbit} of $x$ under the action of $H$. The orbits
                  under the action of $H$ partition the set $A$.)

      \textbf{Proof.} Let $a, b, c \in H$.

      \textbf{Reflexivity.} Since $1a = a$, it follows that $a \sim a$, so that
      $\sim$ is reflexive on $A$.

      \textbf{Symmetry.} Suppose $a \sim b$. Then there exists $h \in H$ such
      that $a = hb$. Now $h^{-1}a = h^{-1}(hb) = (h^{-1}h)b = 1b = b$, so that
      $b \sim a$; thus $\sim$ is symmetric on $A$.

      \textbf{Transitivity.} Suppose $a \sim b$ and $b \sim c$. Then there exist
      $x, y \in H$ such that $a = xb$ and $b = yc$. Thus
      $a = xb = x(yc) = (xy)c$, so that $a \sim c$; thus $\sim$ is transitive on 
      $A$. We can thus conclude that $\sim$ is an equivalence relation on $A$. 
      \qed
%%%%%%%%%%%%%%%%%%%%%%%%%%%%%%%%%%%%%1.7.19%%%%%%%%%%%%%%%%%%%%%%%%%%%%%%%%%%%%%
   \item[1.7.19]  Let $H$ be a subgroup of the finite group $G$ and let $H$ act
                  on $G$ (here $A = G$) by left multiplication. Let $x \in G$
                  and let $\mathcal{O}$ be the orbit of $x$ under the action of
                  $H$. Prove that the map
                  $$H \rightarrow \mathcal{O}\quad \text{defined by} \quad
                    h \mapsto hx$$
                  is a bijection (hence all orbits have cardinality $|H|$). From
                  this and the preceding exercise deduce
                  $\textit{Lagrange's Theorem}:$
                  \begin{center}
                     \textit{if $G$ is a finite group and $H$ is a subgroup of
                     $G$ then $|H|$ divides $|G|$}.
                  \end{center}
                  
      \textbf{Proof.} Let $x \in G$ and let $\mathcal{O}$ be the orbit of $x$
      under the action of $H$ on $G$. Consider the map
      $\alpha : H \rightarrow \mathcal{O}$, $h \mapsto hx$. Suppose
      $\alpha(a) = \alpha(b)$. Then it follows that $ax = bx$, so that $a = b$,
      by cancellation. Thus $\alpha$ is injective. Since $G$ is finite, $H$
      must also be finite; thus $\alpha$ is necessarily surjective, and we 
      conclude that $\alpha$ is bijective. That is $|H| = |\mathcal{O}|$. Let
      $|G| = n$. By Exercise 1.7.18, the action of $H$ on $G$ partitions $G$
      into $m$ distinct orbits (there is a finite number of orbits because $G$
      is finite). Since $x$ was arbitrary, it follows that each orbit of this
      action has size $|H|$, so that $m\cdot|H| = |G|$. That is, $|H| \mid |G|$.
      \qed
%%%%%%%%%%%%%%%%%%%%%%%%%%%%%%%%%%%%%1.7.20%%%%%%%%%%%%%%%%%%%%%%%%%%%%%%%%%%%%%
   \item[1.7.20]  Show that the group of rigid motions of a tetrahedron is
                  isomorphic to a subgroup of $S_4$.
   
      \textbf{Proof.} Let $G$ be the group of rigid motions of a tetrahedron.
      Let $A = \{1, 2, 3, 4\}$ be the vertices of the tetrahedron. Each rigid
      motion in $G$ permutes the vertices of the tetrahedron, so let
      $\sigma_\alpha$ denote the permutation of $A$ by $\alpha \in G$. Now
      consider the action of $G$ on $A$ given by
      $\alpha \cdot a = \sigma_\alpha(a)$ for each $\alpha \in G$ and for each
      $a \in A$. And let $\varphi : G \rightarrow S_4$ be the permutation 
      representation of this action. Distinct rigid motions in $G$ induce 
      distinct permutations in $S_4$ because $\sigma_\alpha$(for each
      $\alpha \in G$) is a bijection. Thus $\varphi$ is injective. By Exercise 
      1.6.13, $G \cong \varphi(G)$ and $\varphi(G)$ is a subgroup of $S_4$. \qed
      %$$G = \{1, (2\;3\;4), (2\;4\;3), (1\;2\;3), (1\;3)(2\;4), (1\;4\;3),
      %        (1\;4\;2), (1\;2)(3\;4), (1\;3\;2), (1\;3\;4), (1\;4)(2\;3), 
      %        (1\;2\;4)\}$$
%%%%%%%%%%%%%%%%%%%%%%%%%%%%%%%%%%%%%1.7.21%%%%%%%%%%%%%%%%%%%%%%%%%%%%%%%%%%%%%
   \item[1.7.21]  Show that the group of rigid motions of a cube is isomorphic
                  to $S_4$. [This group acts on the set of four pairs of
                  opposite vertices.]

      \textbf{Proof.} Let $G$ be the group of rigid motions of a cube. We showed
      in Exercise 1.2.10 that $|G| = 24$. Using a right-handed coordinate
      system, let the vertices of the cube be at the following points:
      \begin{align*}
         \text{vertex 1} &= (-1, 1, 1) \\
         \text{vertex 2} &= (1, 1, 1) \\
         \text{vertex 3} &= (1, -1, 1) \\
         \text{vertex 4} &= (-1, -1, 1) \\
         \text{vertex 5} &= (-1, 1, -1) \\
         \text{vertex 6} &= (1, 1, -1) \\
         \text{vertex 7} &= (1, -1, -1) \\
         \text{vertex 8} &= (-1, -1, -1) \\
      \end{align*} 

      Let $\alpha_i$, $\beta_i$, $\gamma_i$ be the rigid motions that rotate 
      (counterclockwisely) the cube around the $z$-axis, $x$-axis, and $y$-axis,
      respectively, by an angle of $(\pi/2)i$, where $0 \le i \le 3$; for
      $i, j \in \{0, 1, 2, 3\}$, define $\alpha_{ij} = \alpha_{j}\beta_i$; and
      for $j \in \{0, 1, 2, 3\}$ define $\alpha_{4j} = \alpha_{j}\gamma_1$ and
      $\alpha_{5j} = \alpha_{j}\gamma_3$. Note, for example, that
      $\alpha_1\gamma_1$ is the rigid motion that results from first applying
      $\gamma_1$ to the cube and then applying $\alpha_1$. It follows that
      $G = \{\alpha_{ij} : 0 \le i \le 5, 0 \le j \le 3\}$, with identity
      $\alpha_{00}$. An element of $G$ induces a permutation of the 8 vertices
      of the cube; denote this permutation as $\sigma_\alpha$, for each
      $\alpha \in G$. Now let $B$ be the set of pairs of opposite vertices of
      the cube whose vertices are given above, so that
      $B = \{\{1, 7\}, \{2, 8\}, \{3, 5\}, \{4, 6\}\}$. Consider the map
      $f : G \times B \rightarrow A$,
      $(\alpha, \{u, v\}) \mapsto \{\sigma_\alpha(u), \sigma_\alpha(v)\}$. This
      map is a group action because for $\alpha, \beta \in G$, $\{u, v\} \in B$,
      we have
      $$\alpha_{00} \cdot \{u, v\} = \{1u, 1v\} = \{u, v\},$$
      and
      \begin{align*}
         \alpha \cdot (\beta \cdot \{u, v\}) &=
         \alpha \cdot \{\sigma_\beta(u), \sigma_\beta(v)\} \\
         &= \{\sigma_\alpha(\sigma_\beta(u)), \sigma_\alpha(\sigma_\beta(v))\}\\
         &= \{(\sigma_\alpha\circ\sigma_\beta)(u),
              (\sigma_\alpha\circ\sigma_\beta)(v)\} \\
                     &= (\sigma_\alpha\circ\sigma_\beta) \cdot \{u, v\} \\
                     &= (\alpha\circ\beta) \cdot \{u, v\}.
      \end{align*}

      For ease of computation, let us rename the elements of $B$ (as listed in
      the set, from left to right) as 1, 2, 3, and 4. Let
      $\varphi : G \rightarrow S_B$ be the permutation representation of this
      action. It follows that

      \begin{center}
         \begin{tabular}{@{}l c r c l c r c l c r@{}}
            $\varphi(\alpha_{00})$ & = & 1, & &
            $\varphi(\alpha_{20})$ & = & (1\;2)(3\;4), & &
            $\varphi(\alpha_{40})$ & = & (1\;2\;4\;3), \\
            $\varphi(\alpha_{01})$ & = & (1\;4\;3\;2), & &
            $\varphi(\alpha_{21})$ & = & (2\;4),  & &
            $\varphi(\alpha_{41})$ & = & (2\;3\;4), \\
            $\varphi(\alpha_{02})$ & = & (1\;3)(2\;4), & & 
            $\varphi(\alpha_{22})$ & = & (1\;4)(2\;3),  & &
            $\varphi(\alpha_{42})$ & = & (1\;4), \\
            $\varphi(\alpha_{03})$ & = & (1\;2\;3\;4), & & 
            $\varphi(\alpha_{23})$ & = & (1\;3),  & &
            $\varphi(\alpha_{43})$ & = & (1\;3\;2), \\
            $\varphi(\alpha_{10})$ & = & (1\;4\;2\;3), & &
            $\varphi(\alpha_{30})$ & = & (1\;3\;2\;4), & &
            $\varphi(\alpha_{50})$ & = & (1\;3\;4\;2), \\
            $\varphi(\alpha_{11})$ & = & (1\;3\;4), & &
            $\varphi(\alpha_{31})$ & = & (1\;2\;3),  & &
            $\varphi(\alpha_{51})$ & = & (1\;2\;4), \\
            $\varphi(\alpha_{12})$ & = & (1\;2), & & 
            $\varphi(\alpha_{32})$ & = & (3\;4),  & &
            $\varphi(\alpha_{52})$ & = & (2\;3), \\
            $\varphi(\alpha_{13})$ & = & (2\;4\;3), & & 
            $\varphi(\alpha_{33})$ & = & (1\;4\;2),  & &
            $\varphi(\alpha_{53})$ & = & (1\;4\;3).
         \end{tabular}
      \end{center}
      That is, $\varphi$ is an isomorphism. Thus $G \cong S_B \cong S_4$.
 \qed      
%%%%%%%%%%%%%%%%%%%%%%%%%%%%%%%%%%%%%1.7.22%%%%%%%%%%%%%%%%%%%%%%%%%%%%%%%%%%%%%
   \item[1.7.22]  Show that the group of rigid motions of an octahedron is
                  isomorphic to $S_4$. [This group acts on the set
                  of four pairs of opposite faces.] Deduce that the groups of
                  rigid motions of a cube and an octahedron are isomorphic.
                  (These groups are isomorphic because these solids are ``dual"
                  ---see \textit{Introduction to Geometry} by H.Coxeter, Wiley,
                  1961. We shall see later that the groups of rigid motions of
                  the dodecahedron and icosahedron are isomorphic as well---
                  these solids are also dual.)

      \textbf{Proof.} 
%%%%%%%%%%%%%%%%%%%%%%%%%%%%%%%%%%%%%1.7.23%%%%%%%%%%%%%%%%%%%%%%%%%%%%%%%%%%%%%
   \item[1.7.23]  Explain why the action of the group of rigid motions of a cube
                  on the set of three pairs of opposite faces is not faithful.
                  Find the kernel of this action.

      \textbf{Solution.} We will be using the cube (and symbols) from Exercise
      1.7.21. Let $\varphi : G \rightarrow S_F$, be the permutation
      representation of the action of $G$ on $F$, where the $F$ is the set of
      the pair of faces of the cube. Now $|G| = 24 > 6 = 3! = |S_F|$; that is,
      $\varphi$ is not injective, and it follows that some two elements of $G$
      must induce the same permutation of $F$. Hence, this action is not
      faithful. The kernel of the action is:
      $$\{\alpha_{00}, \alpha_{02}, \alpha_{20}, \alpha_{22}\}.$$

      
\end{enumerate}
