\begin{enumerate}
%%%%%%%%%%%%%%%%%%%%%%%%%%%%%%%%%%%%%1.7.1%%%%%%%%%%%%%%%%%%%%%%%%%%%%%%%%%%%%%%
   \item[1.7.1]   Let $F$ be a field. Show that the multiplicative group of
                  nonzero elements of $F$ (denoted by $F^\times$) acts on the
                  set $F$ by $g \cdot a = ga$, where $g \in F^\times$,
                  $a \in F$ and $ga$ is the usual product in $F$ of the two
                  field elements (state clearly which axioms in the definition
                  of a field are used).
%%%%%%%%%%%%%%%%%%%%%%%%%%%%%%%%%%%%%1.7.2%%%%%%%%%%%%%%%%%%%%%%%%%%%%%%%%%%%%%%
   \item[1.7.2]   Show that the additive group $\Z$ acts on itself by
                  $z \cdot a = z + a$ for all $z, a \in \Z$.
%%%%%%%%%%%%%%%%%%%%%%%%%%%%%%%%%%%%%1.7.3%%%%%%%%%%%%%%%%%%%%%%%%%%%%%%%%%%%%%%
   \item[1.7.3]   Show that the additive group $\R$ acts on the $x, y$ plane
                  $\R \times \R$ by $r \cdot (x, y) = (x + ry, y)$.
%%%%%%%%%%%%%%%%%%%%%%%%%%%%%%%%%%%%%1.7.4%%%%%%%%%%%%%%%%%%%%%%%%%%%%%%%%%%%%%%
   \item[1.7.4]   Let $G$ be a group acting on a set $A$ and fix some $a \in A$.
                  Show that the following sets are subgroups of $G$.
                  \begin{enumerate}
                     \item the kernel of the action,
                     \item $\{g \in G : ga = a\}$---this subgroup is called the
                           \textit{stablizer} of $a$ in $G$.
                  \end{enumerate}
%%%%%%%%%%%%%%%%%%%%%%%%%%%%%%%%%%%%%1.7.5%%%%%%%%%%%%%%%%%%%%%%%%%%%%%%%%%%%%%%
   \item[1.7.5]   Prove that the kernel of an action of the group $G$ on the set
                  $A$ is the same as the kernel of the corresponding permutation
                  representation $G \rightarrow S_A$.
%%%%%%%%%%%%%%%%%%%%%%%%%%%%%%%%%%%%%1.7.6%%%%%%%%%%%%%%%%%%%%%%%%%%%%%%%%%%%%%%
   \item[1.7.6]   Prove that a group $G$ acts faithfully on a set $A$ if and
                  only if the kernel of the action is the set consisting only of
                  the identity.
%%%%%%%%%%%%%%%%%%%%%%%%%%%%%%%%%%%%%1.7.7%%%%%%%%%%%%%%%%%%%%%%%%%%%%%%%%%%%%%%
   \item[1.7.7]   Prove that in Example 2 in this section the action is
                  faithful.
%%%%%%%%%%%%%%%%%%%%%%%%%%%%%%%%%%%%%1.7.8%%%%%%%%%%%%%%%%%%%%%%%%%%%%%%%%%%%%%%
   \item[1.7.8]   Let $A$ be a nonempty set and let $k$ be a positive integer
                  with $k \le |A|$. The symmetric group $S_A$ acts on the set
                  $B$ consisting of all subsets of $A$ of cardinality $k$ by
                  $\sigma \cdot \{a_1, \ldots, a_k\} = \{\sigma(a_1), \ldots,
                   \sigma(a_k)\}$.
                  \begin{enumerate}
                     \item Prove that this is a group action.
                     \item Describe explicitly how the elements (1 2) and
                           (1 2 3) act on the six 2-element subsets of
                           $\{1, 2, 3, 4\}$.
                  \end{enumerate}
%%%%%%%%%%%%%%%%%%%%%%%%%%%%%%%%%%%%%1.7.9%%%%%%%%%%%%%%%%%%%%%%%%%%%%%%%%%%%%%%
   \item[1.7.9]   Do both parts of the preceding exercise with ``ordered
                  $k$-tuples" in place of ``$k$-element subsets," where the
                  action on $k$-tuples is defined as above but with set braces
                  replaced by parentheses (note that, for example, the 2-tuples
                  (1, 2) and (2, 1) are different even though the sets
                  $\{1, 2\}$ and $\{2, 1\}$ are the same, so the sets being
                  acted upon are different).
%%%%%%%%%%%%%%%%%%%%%%%%%%%%%%%%%%%%%1.7.10%%%%%%%%%%%%%%%%%%%%%%%%%%%%%%%%%%%%%
   \item[1.7.10]  With reference to the preceding two exercises determine:
                  \begin{enumerate}
                     \item for which values of $k$ the action of $S_n$ on
                           $k$-element subsets if faithful, and
                     \item for which values of $k$ the action of $S_n$ on
                           ordered $k$-tuples is faithful.
                  \end{enumerate}
%%%%%%%%%%%%%%%%%%%%%%%%%%%%%%%%%%%%%1.7.11%%%%%%%%%%%%%%%%%%%%%%%%%%%%%%%%%%%%%
   \item[1.7.11]  Write out the cycle decomposition of the eight permutations in
                  $S_4$ corresponding to the elements of $D_8$ given by the
                  action of $D_8$ on the vertices of a square (where the
                  vertices of the square are labelled as in Section 2).
%%%%%%%%%%%%%%%%%%%%%%%%%%%%%%%%%%%%%1.7.12%%%%%%%%%%%%%%%%%%%%%%%%%%%%%%%%%%%%%
   \item[1.7.12]  Assume $n$ is an even positive integer and show that $D_{2n}$
                  acts on the set consisting of pairs of opposite vertices of a
                  regular $n$-gon. Find the kernel of this action (label
                  vertices as usual).
%%%%%%%%%%%%%%%%%%%%%%%%%%%%%%%%%%%%%1.7.13%%%%%%%%%%%%%%%%%%%%%%%%%%%%%%%%%%%%%
   \item[1.7.13]  Find the kernel of the left regular action.
%%%%%%%%%%%%%%%%%%%%%%%%%%%%%%%%%%%%%1.7.14%%%%%%%%%%%%%%%%%%%%%%%%%%%%%%%%%%%%%
   \item[1.7.14]  Let $G$ be a group and let $A = G$. Show that if $G$ is
                  non-abelian then the maps defined by $g \cdot a = ag$ for all
                  $g, a \in G$ \textit{do not} satisfy the axions of a (left)
                  group action of $G$ on itself.
%%%%%%%%%%%%%%%%%%%%%%%%%%%%%%%%%%%%%1.7.15%%%%%%%%%%%%%%%%%%%%%%%%%%%%%%%%%%%%%
   \item[1.7.15]  Let $G$ be any group and let $A = G$. Show that the maps
                  defined by $g \cdot a = ag^{-1}$ for all $g, a \in G$
                  \textit{do} satisfy the axioms of a (left) group action of $G$
                  on itself.
%%%%%%%%%%%%%%%%%%%%%%%%%%%%%%%%%%%%%1.7.16%%%%%%%%%%%%%%%%%%%%%%%%%%%%%%%%%%%%%
   \item[1.7.16]  Let $G$ be any group and let $A = G$. Show that the maps
                  defined by $g \cdot a = gag^{-1}$ for all $g, a \in G$
                  \textit{do} satisfy the axioms of a (left) group action(this
                  action of $G$ on itself is called \textit{conjugation}).
%%%%%%%%%%%%%%%%%%%%%%%%%%%%%%%%%%%%%1.7.17%%%%%%%%%%%%%%%%%%%%%%%%%%%%%%%%%%%%%
   \item[1.7.17]  Let $G$ be a group and let $G$ act on itself by left 
                  conjugation, so each $g \in G$ maps $G$ to $G$ by
                  $$x \mapsto gxg^{-1}.$$
                  For fixed $g \in G$, prove that conjugation by $g$ is an
                  isomorphism from $G$ onto itself. Deduce that $x$ and
                  $gxg^{-1}$ have the same order for all $x \in G$ and that for
                  any subset $A$ of $G$, $|A| = |gAg^{-1}|$ (here
                  $gAg^{-1} = \{gag^{-1} : a \in A\})$.
%%%%%%%%%%%%%%%%%%%%%%%%%%%%%%%%%%%%%1.7.18%%%%%%%%%%%%%%%%%%%%%%%%%%%%%%%%%%%%%
   \item[1.7.18]  Let $H$ be a group acting on a set $A$. Prove that the
                  relation $\sim$ on $A$ defined by
                  $$a \sim b \quad \text{if and only if} \quad
                    a = hb \quad \text{for some }h \in H$$
                  is an equivalence relation. (For each $x \in A$ the
                  equivalence class of $x$ under $\sim$ is called the
                  \textit{orbit} of $x$ under the action of $H$. The orbits
                  under the action of $H$ partition the set $A$.)
%%%%%%%%%%%%%%%%%%%%%%%%%%%%%%%%%%%%%1.7.19%%%%%%%%%%%%%%%%%%%%%%%%%%%%%%%%%%%%%
   \item[1.7.19]  Let $H$ be a subgroup of the finite group $G$ and let $H$ act
                  on $G$ (here $A = G$) by left multiplication. Let $x \in G$
                  and let $\mathcal{O}$ be the orbit of $x$ under the action of
                  $H$. Prove that the map
                  $$H \rightarrow \mathcal{O}\quad \text{defined by} \quad
                    h \mapsto hx$$
                  is a bijection (hence all orbits have cardinality $|H|$). From
                  this and the preceding exercise deduce
                  $\textit{Lagrange's Theorem}:$
                  \begin{center}
                     \textit{if $G$ is a finite group and $H$ is a subgroup of
                     $G$ then $|H|$ divides $|G|$}.
                  \end{center}
%%%%%%%%%%%%%%%%%%%%%%%%%%%%%%%%%%%%%1.7.20%%%%%%%%%%%%%%%%%%%%%%%%%%%%%%%%%%%%%
   \item[1.7.20]  Show that the group of rigid motions of a tetrahedron is
                  isomorphic to a subgroup of $S_4$.
%%%%%%%%%%%%%%%%%%%%%%%%%%%%%%%%%%%%%1.7.21%%%%%%%%%%%%%%%%%%%%%%%%%%%%%%%%%%%%%
   \item[1.7.21]  Show that the group of rigid motions of a cube is isomorphic
                  to $S_4$. [This group acts on the set of four pairs of
                  opposite vertices.]
%%%%%%%%%%%%%%%%%%%%%%%%%%%%%%%%%%%%%1.7.22%%%%%%%%%%%%%%%%%%%%%%%%%%%%%%%%%%%%%
   \item[1.7.22]  Show that the group of rigid motions of an octahedron is
                  isomorphic to a subgroup of $S_4$. [This group acts on the set
                  of four pairs of opposite faces.] Deduce that the groups of
                  rigid motions of a cube and an octahedron are isomorphic.
                  (These groups are isomorphic because these solids are ``dual"
                  ---see \textit{Introduction to Geometry} by H.Coxeter, Wiley,
                  1961. We shall see later that the groups of rigid motions of
                  the dodecahedron and icosahedron are isomorphic as well---
                  these solids are also dual.)
%%%%%%%%%%%%%%%%%%%%%%%%%%%%%%%%%%%%%1.7.23%%%%%%%%%%%%%%%%%%%%%%%%%%%%%%%%%%%%%
   \item[1.7.23]  Explain why the action of the group of rigid motions of a cube
                  on the set of three pairs of opposite faces is not faithful.
                  Find the kernel of this action.
\end{enumerate}
