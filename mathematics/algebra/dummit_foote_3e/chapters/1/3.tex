\begin{enumerate}
%%%%%%%%%%%%%%%%%%%%%%%%%%%%%%%%%%%%%1.3.1%%%%%%%%%%%%%%%%%%%%%%%%%%%%%%%%%%%%%%
   \item[1.3.1]   Let $\sigma$ be the permutation
                  $$1 \mapsto 3 \quad 2 \mapsto 4 \quad 3 \mapsto 5 \quad
                    4 \mapsto 2 \quad 5 \mapsto 1$$
                  and let $\tau$ be the permutation
                  $$1 \mapsto 5 \quad 2 \mapsto 3 \quad 3 \mapsto 2 \quad
                    4 \mapsto 4 \quad 5 \mapsto 1.$$
                  Find the cycle decompositions of each of the following
                  permutations: $\sigma$, $\tau$, $\sigma^2$, $\sigma\tau$,
                  $\tau\sigma$, and $\tau^2\sigma$.

      \textbf{Solution.}
      \begin{center}
         \begin{tabular}{@{}|c|c|@{}} \hline
            Permutation    & Cycle decomposition   \\ \hline
            $\sigma$       & (1 3 5)(2 4)          \\ \hline
            $\tau$         & (1 5)(2 3)            \\ \hline 
            $\sigma^2$     & (1 5 3)               \\ \hline
            $\sigma\tau$   & (2 5 3 4)             \\ \hline
            $\tau\sigma$   & (1 2 4 3)             \\ \hline
            $\tau^2\sigma$ & (1 3 5)(2 4)          \\ \hline
         \end{tabular}
      \end{center}
%%%%%%%%%%%%%%%%%%%%%%%%%%%%%%%%%%%%%1.3.2%%%%%%%%%%%%%%%%%%%%%%%%%%%%%%%%%%%%%%
   \item[1.3.2]   Let $\sigma$ be the permutation
                  \begin{center}
                     \begin{tabular}{@{}rclrclrclrclrcl@{}}
                        1 & $\mapsto$ & 13 & 2 & $\mapsto$ & 2 &
                        3 & $\mapsto$ & 15 & 4 & $\mapsto$ & 14 &
                        5 & $\mapsto$ & 10 \\
                        6 & $\mapsto$ & 6 & 7 & $\mapsto$ & 12 &
                        8 & $\mapsto$ & 3 & 9 & $\mapsto$ & 4 &
                        10 & $\mapsto$ & 1 \\
                        11 & $\mapsto$ & 7 & 12 & $\mapsto$ & 9 &
                        13 & $\mapsto$ & 5 & 14 & $\mapsto$ & 11 &
                        15 & $\mapsto$ & 8
                  \end{tabular}
                  \end{center}
                  and let $\tau$ be the permutation
                  \begin{center}
                     \begin{tabular}{@{}rclrclrclrclrcl@{}}
                        1 & $\mapsto$ & 14 & 2 & $\mapsto$ & 9 &
                        3 & $\mapsto$ & 10 & 4 & $\mapsto$ & 2 &
                        5 & $\mapsto$ & 12 \\
                        6 & $\mapsto$ & 6 & 7 & $\mapsto$ & 5 &
                        8 & $\mapsto$ & 11 & 9 & $\mapsto$ & 15 &
                        10 & $\mapsto$ & 3 \\
                        11 & $\mapsto$ & 8 & 12 & $\mapsto$ & 7 &
                        13 & $\mapsto$ & 4 & 14 & $\mapsto$ & 1 &
                        15 & $\mapsto$ & 13.
                  \end{tabular}
                  \end{center}
                  Find the cycle decompositions of the following permutations:
                  $\sigma$, $\tau$, $\sigma^2$, $\sigma\tau$, $\tau\sigma$, and
                  $\tau^2\sigma$.

      \textbf{Solution.}
      \begin{center}
         \begin{tabular}{@{}|c|c|@{}} \hline
            Permutation    & Cycle decomposition   \\ \hline
            $\sigma$       & (1 13 5 10)(3 15 8)(4 14 11 7 12 9)     \\ \hline
            $\tau$         & (1 14)(2 9 15 13 4)(3 10)(5 12 7)(8 11) \\ \hline 
            $\sigma^2$     & (1 5)(3 8 15)(4 11 12)(7 9 14)(10 13)   \\ \hline
            $\sigma\tau$   & (1 11 3)(2 4)(5 9 8 7 10 15)(13 14)     \\ \hline
            $\tau\sigma$   & (1 4)(2 9)(3 13 12 15 11 5)(8 10 14)\\ \hline
            $\tau^2\sigma$ & (1 2 15 8 3 4 14 11 12 13 7 5 10)\\ \hline
         \end{tabular}
      \end{center}
%%%%%%%%%%%%%%%%%%%%%%%%%%%%%%%%%%%%%1.3.3%%%%%%%%%%%%%%%%%%%%%%%%%%%%%%%%%%%%%%
   \item[1.3.3]   For each of the permutations whose cycle decompositions were
                  computed in the preceding two exercises compute its order.

      \textbf{Solution.} We shall use Exercise 1.3.15 to find the orders of the 
      elements in the preceding two exercises. 

      \textbf{From Ex. 1.3.1.}
      \begin{center}
         \begin{tabular}{@{}|c|c|@{}}   \hline
            Permutation    & Order   \\ \hline
            $\sigma$       & 6       \\ \hline
            $\tau$         & 2       \\ \hline 
            $\sigma^2$     & 3       \\ \hline
            $\sigma\tau$   & 4       \\ \hline
            $\tau\sigma$   & 4       \\ \hline
            $\tau^2\sigma$ & 6       \\ \hline
         \end{tabular}
      \end{center} 

      \textbf{From Ex. 1.3.2.}
      \begin{center}
         \begin{tabular}{@{}|c|c|@{}}     \hline
            Permutation    & Order     \\ \hline
            $\sigma$       & 12        \\ \hline
            $\tau$         & 30        \\ \hline 
            $\sigma^2$     & 6         \\ \hline
            $\sigma\tau$   & 6         \\ \hline
            $\tau\sigma$   & 6         \\ \hline
            $\tau^2\sigma$ & 13        \\ \hline
         \end{tabular}
      \end{center}
%%%%%%%%%%%%%%%%%%%%%%%%%%%%%%%%%%%%%1.3.4%%%%%%%%%%%%%%%%%%%%%%%%%%%%%%%%%%%%%%
   \item[1.3.4]   Compute the order of each of the elements in the following
                  groups: (a) $S_3$ \qquad (b) $S_4$.

      \textbf{Solution.} We shall use Exercise 1.3.15 to find the orders of the 
      elements in the preceding two exercises.

      \textbf{(a)}
      \begin{center}
         \begin{tabular}{@{}|c|c|@{}}   \hline
            Permutation in $S_3$    & Order   \\ \hline
            1        & 1       \\ \hline
            (1 2)    & 2       \\ \hline 
            (1 3)    & 2       \\ \hline
            (2 3)    & 2       \\ \hline
            (1 2 3)  & 3       \\ \hline
            (1 3 2)  & 3       \\ \hline
         \end{tabular}
      \end{center}

      \textbf{(b)}
      \begin{center}
         \begin{tabular}{@{}|c|c|c|@{}}   \hline
            \# & Permutation in $S_4$    & Order   \\ \hline
            1. &  1           & 1     \\ \hline
            2. &  (1 2)       & 2     \\ \hline 
            3. &  (1 3)       & 2     \\ \hline
            4. &  (1 4)       & 2     \\ \hline
            5. &  (2 3)       & 2     \\ \hline
            6. &  (2 4)       & 2     \\ \hline
            7. &  (3 4)       & 2     \\ \hline
            8. &  (1 2)(3 4)  & 2     \\ \hline
            9. &  (1 3)(2 4)  & 2     \\ \hline
            10.&  (1 4)(2 3)  & 2     \\ \hline
            11.&  (1 2 3)     & 3     \\ \hline
            12.&  (1 3 2)     & 3     \\ \hline
            13.&  (1 2 4)     & 3     \\ \hline
            14.&  (1 4 2)     & 3     \\ \hline
            15.&  (1 3 4)     & 3     \\ \hline
            16.&  (1 4 3)     & 3     \\ \hline
            17.&  (2 3 4)     & 3     \\ \hline
            18.&  (2 4 3)     & 3     \\ \hline
            19.&  (1 2 3 4)   & 4     \\ \hline
            20.&  (1 2 4 3)   & 4     \\ \hline
            21.&  (1 3 2 4)   & 4     \\ \hline
            22.&  (1 3 4 2)   & 4     \\ \hline
            23.&  (1 4 2 3)   & 4     \\ \hline
            24.&  (1 4 3 2)   & 4     \\ \hline
         \end{tabular}
      \end{center}
%%%%%%%%%%%%%%%%%%%%%%%%%%%%%%%%%%%%%1.3.5%%%%%%%%%%%%%%%%%%%%%%%%%%%%%%%%%%%%%%
   \item[1.3.5]   Find the order of $(1\;12\;8\;10\;4)(2\;13)(5\;11\;7)(6\;9)$.

      \textbf{Solution.} Since (1 12 8 10 4)(2 13)(5 11 7)(6 9) is a product of
      disjoint cycles, it follows by Exercise 1.3.15 that its order is
      lcm$(5, 2, 3, 2) = 30$.
%%%%%%%%%%%%%%%%%%%%%%%%%%%%%%%%%%%%%1.3.6%%%%%%%%%%%%%%%%%%%%%%%%%%%%%%%%%%%%%%
   \item[1.3.6]   Write out the cycle decomposition of each element of order 4
                  in $S_4$.

      \textbf{Solution.} See Exercise 1.3.4.
%%%%%%%%%%%%%%%%%%%%%%%%%%%%%%%%%%%%%1.3.7%%%%%%%%%%%%%%%%%%%%%%%%%%%%%%%%%%%%%%
   \item[1.3.7]   Write out the cycle decomposition of each element of order 2
                  in $S_4$.

      \textbf{Solution.} See Exercise 1.3.4.
%%%%%%%%%%%%%%%%%%%%%%%%%%%%%%%%%%%%%1.3.8%%%%%%%%%%%%%%%%%%%%%%%%%%%%%%%%%%%%%%
   \item[1.3.8]   Prove that if $\Omega = \{1, 2, 3, \ldots\}$ then $S_\Omega$
                  is an infinite group (do not say $\infty! = \infty$).

      \textbf{Proof.} Since $\Omega$ is infinite, it suffices to establish an
      injective function from $\Omega$ to $S_\Omega$. So consider
      $f : \Omega \rightarrow S_\Omega$, defined by $x \mapsto (1\;x)$. Now if
      $x, y \in \Omega$, where $x \neq y$, then
      $f(x) = (1\;x) \neq (1\;y) = f(y)$, so that $f$ is injective. That is,
      $S_\Omega$ is infinite. \qed
%%%%%%%%%%%%%%%%%%%%%%%%%%%%%%%%%%%%%1.3.9%%%%%%%%%%%%%%%%%%%%%%%%%%%%%%%%%%%%%%
   \item[1.3.9]   \begin{enumerate}
                     \item Let $\sigma$ be the 12-cycle
                           (1 2 3 4 5 6 7 8 9 10 11 12). For which positive
                           integers $i$ is $\sigma^i$ also a 12-cycle?
                     \item Let $\tau$ be the 8-cycle (1 2 3 4 5 6 7 8). For
                           which positive integers $i$ is $\tau^i$ also an
                           8-cycle?
                     \item Let $\omega$ be the 14-cycle
                           (1 2 3 4 5 6 7 8 9 10 11 12 13 14). For which
                           positive integers $i$ is $\omega^i$ also a 14-cycle?
                  \end{enumerate}

      \textbf{Solution.} See Exercise 1.3.11.
%%%%%%%%%%%%%%%%%%%%%%%%%%%%%%%%%%%%%1.3.10%%%%%%%%%%%%%%%%%%%%%%%%%%%%%%%%%%%%%
   \item[1.3.10]  Prove that if $\sigma$ is the $m$-cycle
                  ($a_1$ $a_2$ $\ldots$ $a_m$), then for all
                  $i \in \{1, 2, \ldots, m\}$, $\sigma^i(a_k) = a_{k+i}$,
                  where $k + i$ is replaced by its least positive residue mod
                  $m$. Deduce that $|\sigma| = m$.

      \textbf{Proof.} Let $\sigma$ be an $m$-cycle ($a_1$ $a_2$ $\ldots$ $a_m$).
      We want to show that if $i$ is a positive integer then
      \begin{equation} \label{1_3_10_1}
         \sigma^i(a_k) = a_{k+i}
      \end{equation}
      We shall proceed by induction on $i$.

      \textbf{Base Case.} $i = 1$. If $1 \le k < m$, then
      $\sigma^1(a_k) = a_{k+1}$. If $k = m$ then $k + 1 \equiv 1 \mod m$, so
      that $\sigma^1(a_k) = a_{k+1} = a_1$. Thus \eqref{1_3_10_1} holds for
      $i = 1$.

      \textbf{Inductive Hypothesis.} Suppose \eqref{1_3_10_1} holds for some
      positive integer $j$.

      So we have that
      \begin{align*}
         \sigma^{j+1}(a_k) &= \sigma(\sigma^j(a_k)) \\
            &= \sigma(a_{k+j}) &[\text{Inductive Hypothesis}] \\
            &= a_{k+j+1}.
      \end{align*}

      That is \eqref{1_3_10_1} holds for $j + 1$. It follows by Mathematical
      Induction that it holds for all positive $n$. Now we have that
      $\sigma^m(a_k) = a_{k + m} = a_k$, so that $\sigma^m = 1$ and thus
      $|\sigma| \le m$. Consider $1 \le s < m$. We have
      $\sigma^s(a_m) = a_{m + s} = a_s \neq a_m$ since $1 \le s < m$. Thus
      $\sigma^s \neq 1$. We can then conclude that $|\sigma| = m$. \qed
%%%%%%%%%%%%%%%%%%%%%%%%%%%%%%%%%%%%%1.3.11%%%%%%%%%%%%%%%%%%%%%%%%%%%%%%%%%%%%%
   \item[1.3.11]  Let $\sigma$ be the $m$-cycle (1 2 $\ldots$ $m$). Show that
                  $\sigma^i$ is also an $m$-cycle if and only if $i$ is
                  relatively prime to $m$.
                  
      \textbf{Proof.} ($\Leftarrow$) Suppose that $\gcd(i, m) = 1$. By
      definition, the elements, in order (and with possible repetition), of the
      first cycle in the cycle decomposition of $\sigma^i$ are
      $$1, (\sigma^{i})(1), (\sigma^{2i})(1), \ldots,
        (\sigma^{(m-1)i})(1).$$
      Now if $\sigma^i$ is not an $m$-cycle, then at least two of the $m$
      elements above will be equal. So it suffices to show that all the $m$
      elements above are unique. Using 1.3.10, we can rewrite these $m$ elements
      as
      $$1+0i, 1+1i, 1+2i, \ldots, 1+(m-1)i$$
      where the numbers above are taken mod $m$. Now suppose that
      $$1 + pi \equiv 1 + qi \mbox{ (mod $m$)}$$
      for some $p, q \in \{0, 1, \ldots, m-1\}$. It follows that
      $pi \equiv qi$ (mod $m$). Since $i$ and $m$ are relatively prime, it
      follows by the Bezout's Identity that there exist some integers $a$ and
      $b$ such that $ai + bm = 1$. That is, $ai \equiv 1$ (mod $m$). So if
      we multiply $pi \equiv qi$ (mod $m$) by $a$, we shall get $p \equiv q$
      (mod $m$). Since $0 \le p < m$ and $0 \le q < m$, it follows that $p = q$.
      Thus, all the $m$ elements are unique. Now observe that $\sigma^i$ cannot
      have more than one cycle in its cycle decomposition since the first cycle
      has exhausted all the possible $m$ elements. Thus
      $\sigma^i = (1\;\;1+i\;\;1+2i\;\;\ldots\;\;1+(m-1)i)$, an $m$-cycle.
      
      ($\Rightarrow$) Next suppose that $\sigma^i$ is an $m$-cycle. By 1.3.10,
      it follows that the order of $\sigma^i$ is $m$. Suppose to the contrary
      that $\gcd(i, m) > 1$. Thus
      $$\frac{m}{\gcd(i, m)} < m.$$
      Note that, by definition, both
      $$\frac{m}{\gcd(i, m)} \text{ and } \frac{i}{\gcd(i, m)}$$
      are positive integers. So
      \begin{align*}
         (\sigma^i)^{\frac{m}{\gcd(i,m)}} = (\sigma)^{\frac{im}{\gcd(i,m)}} =
         (\sigma^m)^{\frac{i}{\gcd(i,m)}} = 1^{\frac{i}{\gcd(i,m)}} = 1,
      \end{align*}
      so that $|\sigma^i| \le \frac{m}{\gcd(i, m)} < m$, contradicting the fact
      that $|\sigma^i| = m$. That is, we must have that $\gcd(i, m) = 1$, and
      the proof is done. \qed
%%%%%%%%%%%%%%%%%%%%%%%%%%%%%%%%%%%%%1.3.12%%%%%%%%%%%%%%%%%%%%%%%%%%%%%%%%%%%%%
   \item[1.3.12]  \begin{enumerate}
                     \item If $\tau$ = (1 2)(3 4)(5 6)(7 8)(9 10) determine
                           whether there is an $n$-cycle $\sigma$ ($n \ge 10$)
                           with $\tau = \sigma^k$ for some integer $k$.
                     \item If $\tau$ = (1 2)(3 4 5) determine whether there is
                           an $n$-cycle $\sigma (n \ge 5)$ with
                           $\tau = \sigma^k$ for some integer $k$.
                  \end{enumerate}

      \textbf{Solution.}

      \begin{enumerate}
         \item Let $\sigma =  $(1 3 5 7 9 2 4 6 8 10) and $k = 5$.
         \item There is no $n$-cycle $\sigma$ such that $\tau = \sigma^k$ for
               some integer $k$.

               \textbf{Proof.} Suppose to the contrary that there exists an
               $n$-cycle $\sigma$ = ($a_1$ $a_2$ $\ldots$ $a_n$) ($n \ge 5$) and
               positive integer $k$ such that $\tau = \sigma^k$. We can assume
               without loss of generality that $a_1 = 1$. Now we have
               $$a_{k+1} = \sigma^k(a_1) = \tau(a_1) = 2 \text{ so that }
                 a_{2k+1} = \sigma^k(a_{k+1}) = \tau(2) = a_1.$$
               Since $a_{2k+1} = a_1$, it follows that $2k \equiv 0$ mod $n$.
               We must now have that $a_i = 3$ for some $1 < i \le n$. So
               $a_{k+i} = \sigma^k(a_i) = \tau(a_i) = 4$. But
               $$5 = \tau(4) = \tau(a_{k+i}) = \sigma^k(a_{k+i}) = a_{2k+i} =
                 a_i = 3,$$
               a contradiction since $3 \neq 5$. The proof is done. \qed
      \end{enumerate}
%%%%%%%%%%%%%%%%%%%%%%%%%%%%%%%%%%%%%1.3.13%%%%%%%%%%%%%%%%%%%%%%%%%%%%%%%%%%%%%
   \item[1.3.13]  Show that an element has order 2 in $S_n$ if and only if its
                  cycle decomposition is a product of commuting 2-cycles.

      \textbf{Proof.} Use Exercise 1.3.14 (set $p = 2$).
%%%%%%%%%%%%%%%%%%%%%%%%%%%%%%%%%%%%%1.3.14%%%%%%%%%%%%%%%%%%%%%%%%%%%%%%%%%%%%%
   \item[1.3.14]  Let $p$ be a prime. Show that an element has order $p$ in
                  $S_n$ if and only if its cycle decomposition is a product of
                  commuting $p$-cycles. Show by an explicit example that this
                  need not be the case if $p$ is not prime.

      \textbf{Proof.} Let $p$ be a prime and let $\sigma \in S_n$.

      $(\Rightarrow)$ Suppose $|\sigma| = p$. Let
      $\sigma_1\sigma_2\cdots\sigma_m$ be the cycle decomposition of $\sigma$,
      with each cycle having length greater than 1. It follows by Exercise
      1.3.15 that $p = \text{lcm}(|\sigma_1|, \ldots, |\sigma_m|)$. Thus
      $|\sigma_i| \mid p$ for all $1 \le i \le m$. Since $|\sigma_i| > 1$ and
      since $p$ is prime it follows that $|\sigma_i| = p$, for all
      $1 \le i \le m$. The converse follows immediately from Exercise 1.3.15.
      \qed

      Consider $\sigma \in S_{10}$, where $\sigma = $(1 2 3 4)(5 6 7 8 9 10). By
      Exercise 1.3.15, $|\sigma| = \text{lcm}(4, 6) = 12$, but $\sigma$ is not a 
      product of commuting $12$-cycles.
%%%%%%%%%%%%%%%%%%%%%%%%%%%%%%%%%%%%%1.3.15%%%%%%%%%%%%%%%%%%%%%%%%%%%%%%%%%%%%%
   \item[1.3.15]  Prove that the order of an element in $S_n$ equals the least
                  common multiple of the lengths of the cycles in its cycle
                  decomposition.
                  [\textbf{Hint.} Use Exercises 1.3.10 and 1.1.24.]

      \textbf{Proof.} Let $\sigma \in S_n$. Consider the cycle decomposition of
      $\sigma$, say $\sigma = \sigma_1\cdots\sigma_m$. Let
      $r = \text{lcm}(|\sigma_1|, \ldots, |\sigma_m|)$, so that
      $|\sigma_i| \mid r$ (i.e there exists an integer $k_i$ such that
      $r = k_i \cdot |\sigma_i|$) for all $1 \le i \le m$. We can assume that
      $r > 1$ since the proof is trivial otherwise. Now
      \begin{align*}
         \sigma^r &= (\sigma_1\cdots\sigma_m)^r \\
               &= {\sigma_1}^r\cdots{\sigma_m}^r &[\text{Exercise 1.1.24}] \\
               &= ({\sigma_1}^{|\sigma_1|})^{k_1}\cdots
                  ({\sigma_m}^{|\sigma_m|})^{k_m} \\
               &= 1^{k_1}\cdots1^{k_m} = 1, &[\text{Exercise 1.3.10}]
      \end{align*}
      so that $|\sigma| \le r$. Now consider any positive integer $s < r$. Since
      $s$ is less than the least common multiple of the lengths of the cycles in
      the cycle decomposition of $\sigma$, it follows that $|\sigma_j| \nmid s$
      for some $1 \le j \le m$; by the Division Algorithm, it follows that
      $s = q\cdot|\sigma_j| + r'$, where $1 \le r' < |\sigma_j|$, so that 
      $${\sigma_j}^s = {\sigma_j}^{q\cdot|\sigma_j| + r'} = 
        ({\sigma_j}^{|\sigma_j|})^q{\sigma_j}^{r'} = {\sigma_j}^{r'} \neq 1;$$
      hence $\sigma^s \neq 1$, a contradiction and we can conclude that
      $|\sigma| = r$. \qed
%%%%%%%%%%%%%%%%%%%%%%%%%%%%%%%%%%%%%1.3.16%%%%%%%%%%%%%%%%%%%%%%%%%%%%%%%%%%%%%
   \item[1.3.16]  Show that if $n \ge m$ then the number of $m$-cycles in $S_n$
                  is given by
                  $$\frac{n(n - 1)(n - 2) \cdots (n - m + 1)}{m}.$$

      \textbf{Proof.} Note that if we fix an element of an $m$-cycle of $S_n$
      as the first element, then we can permute the remaining elements in
      $(m - 1)!$ ways. Thus there are $\D\binom{n}{m} \cdot (m - 1)!$ $m$-cycles
      in $S_n$. So
      $$\binom{n}{m} \cdot (m - 1)! = \frac{n!(m - 1)!}{m!(n - m)!} =
        \frac{n(n - 1)(n - 2) \cdots (n - m + 1)}{m}.$$ \qed
%%%%%%%%%%%%%%%%%%%%%%%%%%%%%%%%%%%%%1.3.17%%%%%%%%%%%%%%%%%%%%%%%%%%%%%%%%%%%%%
   \item[1.3.17]  Show that if $n \ge 4$ then the number of permutations in
                  $S_n$ which are the product of two disjoint 2-cycles is
                  $n(n-1)(n-2)(n-3)/8$.

      \textbf{Proof.} Let $n \ge 4$. To form a product of two disjoint 2-cycles 
      in $S_n$, we first select four elements in $\{1, 2, \ldots, n\}$, say $a$, 
      $b$, $c$, and $d$. There are exactly three products of two disjoint
      2-cycles in $S_n$ containing these numbers; they are:
      $$(a\;b)(c\;d), (a\;c)(b\;d), \text{ and } (a\;d)(b\;c).$$
      Thus, there are $\binom{n}{4} \cdot 3$ ways of forming a product of two
      disjoint 2-cycles in $S_n$, and the proof is done because
      $$\binom{n}{4} \cdot 3 = \frac{n(n-1)(n-2)(n-3)}{8}.$$ \qed
%%%%%%%%%%%%%%%%%%%%%%%%%%%%%%%%%%%%%1.3.18%%%%%%%%%%%%%%%%%%%%%%%%%%%%%%%%%%%%%
   \item[1.3.18]  Find all numbers $n$ such that $S_5$ contains an element of
                  order $n$.

      \textbf{Solution.} $S_5$ has no element of order greater than 6.
      \begin{center}
         \begin{tabular}{@{}|c|c|@{}}   \hline
            Permutation in $S_5$    & Order   \\ \hline
            (1)          & 1       \\ \hline
            (1 2)        & 2       \\ \hline 
            (1 2 3)      & 3       \\ \hline
            (1 2 3 4)    & 4       \\ \hline
            (1 2 3 4 5)  & 5       \\ \hline
            (1 2)(3 4 5) & 6       \\ \hline
         \end{tabular}
      \end{center}
%%%%%%%%%%%%%%%%%%%%%%%%%%%%%%%%%%%%%1.3.19%%%%%%%%%%%%%%%%%%%%%%%%%%%%%%%%%%%%%
   \item[1.3.19]  Find all numbers $n$ such that $S_7$ contains an element of
                  order $n$.

      \textbf{Solution.} $S_7$ has no element of order 8, 9, and greater than
      12.
      \begin{center}
         \begin{tabular}{@{}|c|c|@{}}   \hline
            Permutation in $S_7$    & Order   \\ \hline
            (1)              & 1       \\ \hline
            (1 2)            & 2       \\ \hline 
            (1 2 3)          & 3       \\ \hline
            (1 2 3 4)        & 4       \\ \hline
            (1 2 3 4 5)      & 5       \\ \hline
            (1 2 3 4 5 6)    & 6       \\ \hline
            (1 2 3 4 5 6 7)  & 7       \\ \hline
            (1 2)(3 4 5 6 7) & 10      \\ \hline
            (1 2 3)(4 5 6 7) & 12      \\ \hline
         \end{tabular}
      \end{center}
%%%%%%%%%%%%%%%%%%%%%%%%%%%%%%%%%%%%%1.3.20%%%%%%%%%%%%%%%%%%%%%%%%%%%%%%%%%%%%%
   \item[1.3.20]  Find a set of generators and relations for $S_3$.

      \textbf{Solution.} Let $\sigma$= (1 2 3) and $\alpha$ = (1 3). So we have 
      that
      $$S_3 = \cyc{\sigma, \alpha : \sigma^3 = \alpha^2 = 1, \;
                   \sigma\alpha = \alpha\sigma^{-1}}.$$
\end{enumerate}
