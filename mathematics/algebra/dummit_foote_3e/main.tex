\documentclass[9pt]{book}

\usepackage{amssymb}
\usepackage{amsthm}
\usepackage{amsmath}
\usepackage{amsfonts}
\usepackage{cancel}
\usepackage{comment}
\usepackage{fancyhdr}
\usepackage{mathrsfs}
\usepackage{xcolor}

\usepackage{array}
\usepackage[retainorgcmds]{IEEEtrantools}


\usepackage[utf8]{inputenc}
%\usepackage{cleveref}
\newtheorem{theorem}{Theorem}[section]
\newtheorem{lemma}[theorem]{Lemma}

%\crefname{lemma}{Lemma}{Lemmas}
%\usepackage{hyperref}
%\hypersetup{
%    colorlinks,
%    linkcolor={red!50!black},
%    citecolor={blue!50!black},
%    urlcolor={blue!80!black}
%}
%\hypersetup{
%    colorlinks=true,
%    linkcolor=blue,
%    filecolor=magenta,      
%    urlcolor=cyan,
%    pdftitle={Sharelatex Example},
%    bookmarks=true,
%    pdfpagemode=FullScreen,
%}

%\voffset = -50pt
%\textheight = 700pt
\addtolength{\textwidth}{60pt}
\addtolength{\evensidemargin}{-30pt}
\addtolength{\oddsidemargin}{-30pt}
%\setlength{\headheight}{44pt}

%\newcommand{\qed}{\hfill \ensuremath{\Box}}

\newcommand{\N}{\mathbb{N}}
\newcommand{\Z}{\mathbb{Z}}
\newcommand{\Q}{\mathbb{Q}}
\newcommand{\R}{\mathbb{R}}
\newcommand{\C}{\mathbb{C}}
\newcommand{\F}{\mathbb{F}}
\newcommand{\uline}{\underline{\hspace{3mm}}}
\newcommand{\cyc}[1]{\langle #1 \rangle}
\newcommand{\CYC}[1]{\left\langle #1 \right\rangle}
\newcommand{\gint}[1]{\left\lfloor #1 \right\rfloor}
\newcommand{\D}{\displaystyle}
\everymath{\displaystyle}
\setcounter{chapter}{-1}
\title{\vspace{-14cm}Abstract Algebra by Dummit and Foote \\Solutions Manual, By}
\author{Joseph Okonoboh}
\date{\today}

\begin{document}

   %\maketitle
   %\tableofcontents

   \chapter{Introduction To Groups}
      \section{Basic Axioms And Examples}
         \begin{enumerate}
   \item[]        Let $G$ be a group.
%%%%%%%%%%%%%%%%%%%%%%%%%%%%%%%%%%%Lemm1.1.1%%%%%%%%%%%%%%%%%%%%%%%%%%%%%%%%%%%%
   \item[]        \textbf{Lemma 1.1.1} Let $x \in G$ and let $m$ be an integer.
                  Then we have that
                  $$x^{m+1} = x^mx^1.$$

      \textbf{Proof.}  Consider the following cases:

      \textbf{Case 1:} \textit{$m = 0$}. It follows that
      $$x^{m+1} = x^{0+1} = x^1 = 1x^1 = x^0x^1 = x^mx^1.$$

      \textbf{Case 2:} \textit{$m$ is positive.} Then it follows that $m + 1$ is 
      positive, so that
      \begin{align*}
         x^{m + 1} &= \underbrace{x \cdot x \cdots x}_{m+1\text{ factors}} \\
            &= \underbrace{x \cdot x \cdots x}_{m\text{ factors}} \cdot x^1 \\
            &= x^mx^1.
      \end{align*}

      \textbf{Case 3:} \textit{$m$ is negative.} If $m = -1$, then we have that
      $$x^{m+1} = x^{-1+1} = x^0 = 1 = x^{-1}x^1 = x^mx^1.$$
      If $m < -1$, then $m + 1$ is negative, so that $-(m + 1) = -m - 1$ is 
      positive. Thus
      \begin{align*}
         x^mx^1 &= x^{-(-m)}x^1 \\
                &= \underbrace{x^{-1} \cdot x^{-1} \cdots x^{-1}}_{
                   -m\text{ factors}} \cdot x^1 \\
                &= \underbrace{x^{-1} \cdot x^{-1} \cdots x^{-1}}_{
                   -m-1\text{ factors}} \cdot (x^{-1} \cdot x^1) \\
                &= \underbrace{x^{-1} \cdot x^{-1} \cdots x^{-1}}_{
                   -m-1\text{ factors}} \\
                &= x^{-(-m-1)} \\
                &= x^{m+1}.
      \end{align*}

      In all cases, we can see that our assertion holds. \qed
%%%%%%%%%%%%%%%%%%%%%%%%%%%%%%%%%%%Lemm1.1.2%%%%%%%%%%%%%%%%%%%%%%%%%%%%%%%%%%%%
   \item[]        \textbf{Lemma 1.1.2} \textit{Let $x$ and $g$ be members of a 
                  group $G$, and let $n$ be a positive integer, then it follows 
                  that $(g^{-1}xg)^n = g^{-1}x^ng$.}

      \textbf{Proof.} We shall show by induction that the equation
      \begin{equation}
         (g^{-1}xg)^n = g^{-1}x^ng \label{l1_1_2_1}
      \end{equation}
      holds for every positive integer $n$. It is clear that equation
      \ref{l1_1_2_1} holds for $n = 1$. So assume that it also holds for some
      positive integer $k$. So we must now show that the equation also holds for 
      $k + 1$. Thus
      \begin{align*}
         (g^{-1}xg)^{k+1} &= (g^{-1}xg)^kg^{-1}xg &[\text{Execise 1.1.19}] \\
                     &= g^{-1}x^kgg^{-1}xg &[\text{Inductive hypothesis}] \\
                     &= g^{-1}x^kxg \\
                     &= g^{-1}x^{k+1}g,
      \end{align*}
      so that equation \eqref{l1_1_2_1} holds for $k+1$. Hence by the Principle 
      of Mathematical Induction, equation \eqref{l1_1_2_1} holds for every 
      positive integer $n$. \qed
%%%%%%%%%%%%%%%%%%%%%%%%%%%%%%%%%%%Lemm1.1.3%%%%%%%%%%%%%%%%%%%%%%%%%%%%%%%%%%%%
   \item[]        \textbf{Lemma 1.1.3} \textit{Let $x$ be an element of finite
                  order $n$ in $G$. If $x^m = 1$, then it follows that
                  $n \mid m$.}

      \textbf{Proof.} Suppose that $x^m = 1$. By the Division Algorithm, there
      exist unique integers $q$ and $r$ such that $m = qn + r$ and
      $0 \le r < n$. Now we have that
      $$1 = x^m = x^{qn+r} = x^{qn}x^r = (x^n)^qx^r = 1^qx^r = x^r.$$
      Since $|x| = n$, we cannot have $0 < r < n$; thus the only remaining
      possibility is $r = 0$, so that $n = qm$, as desired. \qed
%%%%%%%%%%%%%%%%%%%%%%%%%%%%%%%%%%%Lemm1.1.4%%%%%%%%%%%%%%%%%%%%%%%%%%%%%%%%%%%%
   \item[]        \textbf{Lemma 1.1.4} \textit{Let $(x, y)$ be an element of
                  $A \times B$ where $A$ and $B$ are groups. For any positive
                  integer $n$, we then have that $(x, y)^n = (x^n, y^n)$.}

      \textbf{Proof.} We shall induct on $n$. Our assertion clearly holds if
      $n$ is 1, so assume that it holds for some positive integer $k$. Thus we
      have that
      \begin{align*}
         (x, y)^{k+1} &= (x, y)(x, y)^k &[\text{Exercise 1.1.19}] \\
                      &= (x, y)(x^k, y^k) &[\text{Inductive hypothesis}] \\
                      &= (xx^k, yy^k) \\
                      &= (x^{k+1}, y^{k+1}). &[\text{Exercise 1.1.19}]
      \end{align*}
      The above shows that our assertion also holds for $k + 1$, so that by
      the Principle of Mathematical Induction it must holds for every integer
      $n$. \qed
%%%%%%%%%%%%%%%%%%%%%%%%%%%%%%%%%%%%%1.1.1%%%%%%%%%%%%%%%%%%%%%%%%%%%%%%%%%%%%%%
   \item[1.1.1]   Determine which of the following binary operations are
                  associative:
                  \begin{enumerate}
                     \item the operation $*$ on $\Z$ defined by $a * b = a - b$.
                     \item the operation $*$ on $\R$ defined by
                           $a * b = a + b + ab$.
                     \item the operation $*$ on $\Q$ defined by
                           $\displaystyle a * b = \frac{a + b}{5}$.
                     \item the operation $*$ on $\Z \times \Z$ defined by
                           $(a, b) * (c, d) = (ad + bc, bd)$.
                     \item the operation $*$ on $\Q - \{0\}$ defined by
                           $\displaystyle a * b = \frac{a}{b}$.
                  \end{enumerate}
                  
      \textbf{Solution.}
   
      \begin{enumerate}
         \item The binary operation $*$ on $\Z$ is not associative because
               $$(0 * 0) * 1 = -1 \neq 1 = 0 * (0 * 1).$$
         \item We claim that $*$ is associative on $\R$.
      
               \textbf{Proof.} Let $r_1, r_2, r_3 \in \R$. Then it follows that
               \begin{align*}
                  (r_1 * r_2) * r_3 &= (r_1 + r_2 + r_1r_2) * r_3 \\
                     &= (r_1 + r_2 + r_1r_2 + r_3) +
                        (r_1r_3 + r_2r_3 + r_1r_2r_3) \\
                     &= r_1 + r_2 + r_3 + r_1r_2 + r_2r_3 +
                        r_1r_3 + r_1r_2r_3 \\
                     &= (r_1 + r_2 + r_3 + r_2r_3) + r_1(r_2 + r_3 + r_2r_3) \\
                     &= r_1 + (r_2 * r_3) + r_1(r_2 * r_3) \\
                     &= r_1 * (r_2 * r_3),
               \end{align*}
               so that our claim holds. \qed
         \item The binary operation $*$ on $\Q$ is not associative because
               $$(0 * 0) * 25 = 5 \neq 1 = 0 * (0 * 25).$$
         \item We claim that $*$ is associative on $\Z \times \Z$.
      
               \textbf{Proof.} Let $(z_1, z_2)$, $(z_3, z_4)$,
               $(z_5, z_6) \in \Z \times \Z$. Then it follows that
               \begin{align*}
                  (z_1, z_2) * [(z_3, z_4) * (z_5, z_6)] 
                     &= (z_1, z_2) * [(z_3z_6 + z_4z_5, z_4z_6)] \\
                     &= (z_1z_4z_6 + z_2z_3z_6 + z_2z_4z_5, z_2z_4z_6) \\
                     &= ((z_1z_4 + z_2z_3) \cdot z_6 + z_2z_4 \cdot z_5,
                          z_2z_4 \cdot z_6) \\
                     &= (z_1z_4 + z_2z_3, z_2z_4) * (z_5, z_6) \\
                     &= [(z_1, z_2) * (z_3, z_4)] * (z_5, z_6),
               \end{align*}
               so that our claim holds. \qed
         \item The binary operation $*$ on $\Q - \{0\}$ is not associative
               because
               $$(4 * 1) * 2 = 2 \neq 8 = 4 * (1 * 2).$$
      \end{enumerate}
%%%%%%%%%%%%%%%%%%%%%%%%%%%%%%%%%%%%%1.1.2%%%%%%%%%%%%%%%%%%%%%%%%%%%%%%%%%%%%%%
   \item[1.1.2]   Decide which of the binary operations in the preceding
                  exercise are commutative.
                  
      \begin{enumerate}      
         \item The binary operation $*$ on $\Z$ is not commutative because
               $$1 * 0 = 1 \neq -1 = 0 * 1.$$
         \item The binary operation $*$ on $\R$ is commutative because addition
               and multiplication are commutative on $\R$.
         \item The binary operation $*$ on $\Q$ is commutative because addition
               is commutative on $\Q$.
         \item A quick check will show us that $*$ is commutative on
               $\Z \times \Z$. That is, for all $(z_1, z_2)$, $(z_3, z_4)$
               $\in \Z \times \Z$, we must have that
               \begin{align*}
                  (z_1, z_2) * (z_3, z_4) &= (z_1z_4 + z_2z_3, z_2z_4) \\
                                          &= (z_3z_2 + z_4z_1, z_4z_2) \\
                                          &= (z_3, z_4) * (z_1, z_2).
               \end{align*}
         \item The binary operation $*$ on $\Q - \{0\}$ is not commutative
               because
               $$1 * 2 = \frac{1}{2} \neq \frac{2}{1} = 2 * 1.$$
      \end{enumerate}
%%%%%%%%%%%%%%%%%%%%%%%%%%%%%%%%%%%%%%1.3%%%%%%%%%%%%%%%%%%%%%%%%%%%%%%%%%%%%%%%
   \item[1.1.3]   Prove that addition of residue classes in $\Z/n\Z$ is
                  associative (you may assume it is well defined).
                  
      \textbf{Proof.} Fix $n \in \Z^+$. Consider $\overline{a}$, $\overline{b}$,
      and $\overline{c}$ in $\Z/n\Z$. By Theorem 3, Pg. 9, we have that
      \begin{align*}
         \overline{a} + (\overline{b} + \overline{c})
            &= \overline{a} + \overline{b + c} \\
            &= \overline{a + b + c} \\
            &= \overline{a + b} + \overline{c} \\
            &= (\overline{a} + \overline{b}) + \overline{c},
      \end{align*}
      so that addition of residue classes in $\Z/n\Z$ is associative. \qed
%%%%%%%%%%%%%%%%%%%%%%%%%%%%%%%%%%%%%%1.4%%%%%%%%%%%%%%%%%%%%%%%%%%%%%%%%%%%%%%%
   \item[1.1.4]   Prove that multiplication of residue classes in $\Z/n\Z$ is
                  associative (you may assume it is well defined).
                  
      \textbf{Proof.} Fix $n \in \Z^+$. Consider $\overline{a}$, $\overline{b}$,
      and $\overline{c}$ in $\Z/n\Z$. By Theorem 3, Pg. 9, we have that
      \begin{align*}
         \overline{a} \cdot (\overline{b} \cdot \overline{c})
            &= \overline{a} \cdot \overline{bc} \\
            &= \overline{abc} \\
            &= \overline{ab} \cdot \overline{c} \\
            &= (\overline{a} \cdot \overline{b}) \cdot \overline{c},
      \end{align*}
      so that multiplication of residue classes in $\Z/n\Z$ is associative. \qed
%%%%%%%%%%%%%%%%%%%%%%%%%%%%%%%%%%%%%%1.5%%%%%%%%%%%%%%%%%%%%%%%%%%%%%%%%%%%%%%%
   \item[1.1.5]   Prove that for all $n > 1$ that $\Z/n\Z$ is not a group under
                  multiplication of residue classes.
                  
      \textbf{Proof.} Let $n$ be positive integer greater than 1. It follows
      that $\Z/n\Z$ is not a group under multiplication because $\overline{0}$
      has no multiplicative inverse. \qed
%%%%%%%%%%%%%%%%%%%%%%%%%%%%%%%%%%%%%%1.6%%%%%%%%%%%%%%%%%%%%%%%%%%%%%%%%%%%%%%%
   \item[1.1.6]   Determine which of the following sets are groups under
                  addition:
                  \begin{enumerate}
                     \item the set of rational numbers (including $0 = 0/1$) in
                           lowest terms whose denominators are odd.
                     \item the set of rational numbers (including $0 = 0/1$) in
                           lowest terms whose denominators are even.
                     \item the set of rational numbers of absolute value $< 1$.
                     \item the set of rational numbers of absolute value $\ge 1$
                           together with 0.
                     \item the set of rational numbers with denominators equal
                           to 1 or 2.
                     \item the set of rational numbers with denominators equal
                           to 1, 2, or 3.
                  \end{enumerate}

      \textbf{Solution.}

      \begin{enumerate}
         \item We claim that the set
               $$S = \left\{\frac{a}{b} \in \Q : b \text{ is odd} \text{ and }
                 \gcd(a, b) = 1\right\},$$
               is a group under addition.

               \textbf{Proof.} First we must show that $S$ is closed under 
               addition. Notice that $S$ is nonempty since it contains 7/5, so 
               let $r, s \in S$. By definition of $S$, we have that
               $r = a_1/b_1$ and $s = a_2/b_2$ for some integers $a_1$ and
               $a_2$, and nonzero integers $b_1$ and $b_2$, where $b_1$ and
               $b_2$ are odd and $\gcd(a_1, b_1) = \gcd(a_2, b_2) = 1$.
               It follows that
               \begin{align*}
                  r + s &= \frac{a_1}{b_1} + \frac{a_2}{b_2} \\
                        &= \frac{a_1b_2 + a_2b_1}{b_1b_2}.
               \end{align*}

               Since $b_1$ and $b_2$ are both odd, it must necessarily be the 
               case that $b_1b_2$ is also odd. In order words, $b_1b_2$ contains 
               no factor of 2, so that if we reduce $r + s$ to its lowest term, 
               the denominator of this lowest term will still be odd. Hence
               $r + s \in S$, so that $S$ is closed under addition. To complete 
               the proof we must now show that $S$ satisfies the group axioms. 
               We observe that $0/1$ is the identity element in $S$. Also, it is 
               clear that for all $s \in S$, we have $-s \in S$, so that every 
               element of $S$ has an inverse under addition. Since
               $S \subseteq \Q$, and since $\Q$ is associative under addition, 
               it follows that $S$ is also associative under addition. Thus $S$ 
               satisfies the group axioms, so that $(S, +)$ is a group. \qed
         \item The set
               $$S = \left\{\frac{a}{b} \in \Q : b \text{ is even} \text{ and }
                 \gcd(a, b) = 1\right\},$$
               is not a group under addition because it is not closed. Indeed,
               for $3/14 \in S$, we have $3/14 + 3/14 = 3/7 \notin S$.
         \item The set
               $$S = \left\{\frac{a}{b} \in \Q :
                     \left|\frac{a}{b}\right| < 1\right\},$$
               is not a group under addition because it is not closed. Indeed,
               for $9/10 \in S$, we have $9/10 + 9/10 = 18/10 \notin S$.
         \item The set
               $$S = \left\{\frac{a}{b} \in \Q : a = 0 \text{ or }
                     \left|\frac{a}{b}\right| \ge 1\right\},$$
               is not a group under addition because it is not closed. Indeed,
               for $-11/10, 10/10 \in S$, we have
               $-11/10 + 10/10 = -1/10 \notin S$.
         \item We claim that the set
               $$S = \left\{\frac{a}{b} \in \Q : b = 1 \text{ or }
                 b = 2\right\},$$
               is a group under addition.

               \textbf{Proof.} It is clear that 0 is the identity for $S$ under
               addition, that $S$ is associative under addition (because
               $S \subset \Q$ and $\Q$ is associative under addition, and that
               the inverse of an element in $S$ is its additive inverse in $\Q$.
               So to complete the proof, we need only show that $S$ is closed
               under addition. Let $a_1/b_1, a_2/b_2 \in \Q$. By observation, we
               note that $a_1/b_1 + a_2/b_2$ must have a denominator of 1 or 2,
               so that it is in $S$. Thus $S$ is closed under addition. \qed
         \item The set
               $$S = \left\{\frac{a}{b} \in \Q : b \in {1, 2, 3} \right\},$$
               is not a group under addition because it is not closed. Indeed,
               for $1/2, 1/3 \in S$, we have $1/2 + 1/3 = 5/6 \notin S$.
      \end{enumerate}
%%%%%%%%%%%%%%%%%%%%%%%%%%%%%%%%%%%%%%1.7%%%%%%%%%%%%%%%%%%%%%%%%%%%%%%%%%%%%%%%
   \item[1.1.7]   Let $G = \{x \in \R : 0 \le x < 1\}$ and for $x, y \in G$ let
                  $x * y$ be the fractional part of $x + y$ (i.e.,
                  $x * y = x + y = [x + y]$ where $[a]$ is the greatest integer
                  less than or equal to $a$). Prove that $*$ is a well defined
                  binary operation on $G$ and that $G$ is an abelian group under
                  $*$ (called the \textit{real numbers mod }1).
                  
      \textbf{Proof.} The set $G$ is clearly non-empty, so consider
      $x, y, z \in G$. To show that $G$ is a group, we shall now prove that it 
      is well defined, associative, has an identity, and is closed under taking
      inverses.

      \textbf{Well Defined:} To show that $*$ is well defined is tantamount to
      showing that $G$ is closed under $*$.  By definition, we have that
      $0 \le x < 1$ and $0 \le y < 1$, so that $0 \le x + y < 2$. If
      $0 \le x + y < 1$, so that $[x + y] = 0$, then we have that
      $$0 \le x + y = x + y - [x + y] = x * y = x + y < 1.$$
      However if $1 \le x + y < 2$, so that $[x + y] = 1$ and
      $0 \le x + y - 1 < 1$, we must have that
      $$0 \le x + y - 1 = x + y - [x + y] = x * y = x + y - 1 < 1.$$
      In either case, we have $0 \le x * y < 1$; i.e. $x * y \in G$, so that $G$ 
      is closed under $*$. Also we have that
      $$x * y = x + y - [x + y] = y + x - [y + x],$$
      so that $G$ is abelian.

      \textbf{Associativity:} We have that
      \begin{align*}
         x * (y * z) &= x * (y + z - [y + z]) \\
              &= x + y + z - [y + z] - [x + y + z - [y + z]], \text{ and} \\ \\
         (x * y) * z &= (x + y - [x + y]) * z \\
                     &=  x + y + z - [x + y] - [x + y + z - [x + y]].
      \end{align*}
      By definition, we have that $0 \le x < 1$, $0 \le y < 1$, and
      $0 \le z < 1$, so that $0 \le x + y < 2$ and $0 \le y + z < 2$. Let us 
      now investigate the following possible cases:

      \textit{Case 1:} \textit{$0 \le x + y  < 1$ and $0 \le y + z < 1$}. That
      is $[x + y] = [y + z] = 0$. It then follows that
      $$x * (y * z) = (x * y) * z = x + y + z - [x + y + z].$$

      \textit{Case 2:} \textit{$1 \le x + y  < 2$ and $1 \le y + z < 2$}. That
      is $[x + y] = [y + z] = 1$. It then follows that
      $$x * (y * z) = (x * y) * z = x + y + z - 1 - [x + y + z - 1].$$

      \textit{Case 3:} \textit{$0 \le x + y  < 1$ and $1 \le y + z < 2$}. That
      is $[x + y] = 0$, and $[y + z] = 1$. It then follows that
      $$(x * y) * z = x + y + z - [x + y + z].$$
      Since $0 \le x + y < 1$ and $0 \le z < 1$, we must have that
      $0 \le x + y + z < 2$. Similarly, since $1 \le y + z < 2$ and
      $0 \le x < 1$, we must have that $1 \le x + y + z < 3$, and since we 
      already showed that $0 \le x + y + z < 2$, it follows that
      $1 \le x + y + z < 2$. Hence $[x + y + z] = 1$. We can then conclude that 
      $(x * y) * z = x +y + z - 1$. Now we have that
      $$x * (y * z) = x + y + z - 1 - [x + y + z - 1].$$
      We already showed that $1 \le x + y + z < 2$; thus,
      $0 \le x + y + z - 1 < 1$, so that $[x + y + z - 1] = 0$; that is,
      $$x * (y * z) = x + y + z - 1 = (x * y) * z.$$
   
      \textit{Case 4:} \textit{$1 \le x + y  < 2$ and $0 \le y + z < 1$}. Apply
      Case 3, with the roles of $x + y$ and $y + z$ interchanged.

      We have thus shown that in all possible cases, we have
      $$x * (y * z) = (x * y) * z,$$
      so that $G$ is associative under $*$.


      \textbf{Identity:} We observe that $0 \in G$ is the identity element since
      $$x * 0 = x + 0 - [x + 0] = x - [x] = x - 0 = x.$$

      \textbf{Inverse:} Suppose $x \neq 0$, so that $0 < x < 1$, and thus
      $0 < 1 - x < 1$; that is $1 - x \in G$. It follows that
      $$x * (1 - x) = x + (1 - x) + [x + (1 - x)] = 1 - 1 = 0,$$
      so that $1 - x$ is the inverse of $x \in G$, with $x \neq 0$. Clearly, the 
      inverse of 0 is 0. \\

      We can now conclude that $(G, *)$ is a group. \qed
%%%%%%%%%%%%%%%%%%%%%%%%%%%%%%%%%%%%%%1.8%%%%%%%%%%%%%%%%%%%%%%%%%%%%%%%%%%%%%%%
   \item[1.1.8]   Let $G = \{z \in \C : z^n = 1 \text{ for some } n \in \Z^+\}$.
                  \begin{enumerate}
                     \item Prove that $G$ is a group under multiplication
                           (called the group of \textit{roots of unity} in
                           $\C$).
                     \item Prove that $G$ is not a group under addition.
                  \end{enumerate}
                  
      \textbf{Proof.}
      
      \begin{enumerate}
         \item We observe that 1 is the identity element of $G$, so that $G$ is
               not empty. So let $x, y, z \in G$.
               
               \textbf{Closure:} By definition, there exist positive integers
               $m$ and $n$ such that $x^m = y^n = 1$. Thus
               $(xy)^{mn} = (x^m)^n(y^n)^m = 1^n1^m = 1$. This says that $G$ is
               closed under multiplication.
               
               \textbf{Associativity:} Since $\C$ is associative under
               multiplication and since $G \subseteq \C$, it follows that $G$ is
               associative under multiplication.
               
               \textbf{Identity:} As state above, the identity of $G$ is clearly
               1.
               
               \textbf{Inverse:} Notice that since
               $(x^{m - 1})^m = (x^m)^{m - 1} = 1$, we must have that
               $x^{m - 1} \in G$. Thus we have $x^{m - 1}x = x^m = 1$; i.e., the
               inverse of $x$ is $x^{m - 1}$.
               
               We have thus shown that $G$ is a group under multiplication. \qed
         \item $G$ is not a group under addition because it is not closed under
               addition. In particular, we have $1 \in G$, but
               $1 + 1 = 2 \notin G$ because $2^n \neq 1$ for any positive
               integer.
      \end{enumerate}
%%%%%%%%%%%%%%%%%%%%%%%%%%%%%%%%%%%%%%1.9%%%%%%%%%%%%%%%%%%%%%%%%%%%%%%%%%%%%%%%
   \item[1.1.9]   Let $G = \{a + b\sqrt{2} \in \R : a, b \in \Q\}$.
                  \begin{enumerate}
                     \item Prove that $G$ is a group under addition.
                     \item Prove that the nonzero elements of $G$ are a group 
                           under multiplication. [``Rationalize the
                           denominators" to find multiplicative inverse.]
                  \end{enumerate}
                  
      \textbf{Proof.}
      
      \begin{enumerate}
         \item \textbf{Closure:} $G$ is clearly nonempty, so let $x, y \in G$.
               By definition of $G$, it follows that $x = a_1 + b_1\sqrt{2}$ and
               $y = a_2 + b_2\sqrt{2}$ for some rational numbers $a_1$, $b_1$,
               $a_2$, and $b_2$. Thus
               $$x + y = (a_1 + a_2) + (b_1 + b_2)\sqrt{2} \in G,$$
               so that $G$ is closed under addition.
               
               \textbf{Associativity:} Since $\R$ is associative under addition
               and since $G \subseteq \R$, it follows that $G$ is associative
               under addition.
               
               \textbf{Identity:} The identity of $G$ is 0.
               
               \textbf{Inverse:} For an element $x = a_1 + b_1\sqrt{2} \in G$,
               the additive inverse of $x$ is $-a_1 - b_1\sqrt{2} \in G$.
               
               We have thus shown that $G$ is a group under addition. \qed
         \item Let $G^{\times}$ denote the set of nonzero elements of $G$.
         
               \textbf{Closure:} Let $x, y \in G^{\times}$. By definition of
               $G$, it follows that $x = a_1 + b_1\sqrt{2}$ and
               $y = a_2 + b_2\sqrt{2}$ for some rational numbers $a_1$, $b_1$,
               $a_2$, and $b_2$, with $a_1$ and $b_1$ not both zero and $a_2$
               and $b_2$ not both zero. Thus
               $$xy = (a_1a_2 + 2b_1b_2) + (a_1b_2 + a_2b_1)\sqrt{2}.$$
               Since neither $x$ nor $y$ is zero, it must be the case that $xy$
               is not zero, so that $G^{\times}$ is closed under multiplication.
               
               \textbf{Associativity:} Since $\R$ is associative under
               multiplication and since $G^{\times} \subseteq \R$, it follows
               that $G^{\times}$ is associative under multiplication.
               
               \textbf{Identity:} The element $1 = 1 + 0\sqrt{2} \in G^{\times}$
               is the identity of $G^{\times}$.
               
               \textbf{Inverse:} Let $x = a_1 + b_1\sqrt{2} \in G^{\times}$.
               Since $x \neq 0$, the real number $1/x$ exists, and we have that
               $$\frac{1}{x} = \frac{1}{a_1 + b_1\sqrt{2}}
                 \frac{a_1 - b_1\sqrt{2}}{a_1 - b_1\sqrt{2}} =
                 \left(\frac{a_1}{{a_1}^2 - 2{b_1}^2} -
                 \frac{b_1}{{a_1}^2 - 2{b_1}^2}\sqrt{2}\right) \in G^{\times}.
               $$
               
               Since $1/x \in G^{\times}$ and since $x \cdot 1/x = 1$, we have
               that $1/x$ is the multiplicative inverse of $x$.
               
               We have thus shown that $G^{\times}$ is a group under
               multiplication. \qed
      \end{enumerate}
%%%%%%%%%%%%%%%%%%%%%%%%%%%%%%%%%%%%%%1.10%%%%%%%%%%%%%%%%%%%%%%%%%%%%%%%%%%%%%%
   \item[1.1.10]  Prove that a finite group is abelian if and only if its group
                  table is a symmetric matrix.
                  
      \textbf{Proof.} Let $G$ be a group such that $|G| = n \in \Z^+$, and let
      $(a_{ij})$ denote the matrix of the group table of $G$. Since $G$ is
      finite, we can enumerate the elements of $G$ like so:
      $$G = \{g_1, g_2, \ldots, g_n\}.$$      
      $(\Leftarrow)$ Suppose that $(a_{ij})$ is a symmetric matrix. Let
      $a, b \in G$. Then we have that $a = g_r$ and $b = g_s$ for some
      $r, s \in \{1, 2, \ldots, n\}$. Since $(a_{ij})$ is symmetric, we must
      have that
      $$ab = g_rg_s = a_{rs} = a_{sr} = g_sg_r = ba,$$
      so that $G$ is abelian.
      
      $(\Rightarrow)$ Now suppose that $G$ is abelian. Consider
      $a_{rs} \in (a_{ij})$. It follows that
      $$a_{rs} = g_rg_s = g_sg_r = a_{sr},$$
      so that $(a_{ij})$ is symmetric. \qed      
%%%%%%%%%%%%%%%%%%%%%%%%%%%%%%%%%%%%%%1.11%%%%%%%%%%%%%%%%%%%%%%%%%%%%%%%%%%%%%%
   \item[1.1.11]  Find the orders of each element of the additive group
                  $\Z/12\Z$.
                  
      \textbf{Solution.} The orders of the elements $\overline{0}$,
      $\overline{1}$, $\overline{2}$, $\overline{3}$, $\overline{4}$,
      $\overline{5}$, $\overline{6}$, $\overline{7}$, $\overline{8}$,
      $\overline{9}$, $\overline{10}$, and $\overline{11}$ in $\Z/12\Z$ are
      1, 12, 6, 4, 3, 12, 2, 12, 3, 4, 6, and 12.
%%%%%%%%%%%%%%%%%%%%%%%%%%%%%%%%%%%%%%1.12%%%%%%%%%%%%%%%%%%%%%%%%%%%%%%%%%%%%%%
   \item[1.1.12]  Find the orders of the following elements of the
                  multiplicative group $(\Z/12\Z)^\times: \overline{1},
                  \overline{-1}, \overline{5}, \overline{7}, \overline{-7}, 
                  \overline{13}$.
                  
      \textbf{Solution.} The orders of the elements $\overline{1}$,
      $\overline{-1}$, $\overline{5}$, $\overline{7}$, $\overline{-7}$,
      $\overline{13}$ in $(\Z/12\Z)^\times$ are 1, 11, 5, 7, 5, and 13.
%%%%%%%%%%%%%%%%%%%%%%%%%%%%%%%%%%%%%%1.13%%%%%%%%%%%%%%%%%%%%%%%%%%%%%%%%%%%%%%
   \item[1.1.13]  Find the orders of the following elements of the additive
                  group $\Z/36\Z: \overline{1}, \overline{2}, \overline{6}, 
                  \overline{9}, \overline{10}, \overline{12}, \overline{-1}, 
                  \overline{-10}, \overline{-18}$.
                  
      \textbf{Solution.} The orders of the elements $\overline{1}$,
      $\overline{2}$, $\overline{6}$, $\overline{9}$, $\overline{10}$,
      $\overline{12}$, $\overline{-1}$, $\overline{-10}$, and $\overline{-18}$
      in $\Z/36\Z$ are 1, 18, 6, 4, 18, 3, 36, 18, and 2.
%%%%%%%%%%%%%%%%%%%%%%%%%%%%%%%%%%%%%%1.14%%%%%%%%%%%%%%%%%%%%%%%%%%%%%%%%%%%%%%
   \item[1.1.14]  Find the orders of the following elements of the
                  multiplicative group $(\Z/36\Z)^\times: \overline{1},
                  \overline{-1}, \overline{5}, \overline{13}, \overline{-13},
                  \overline{17}$.
                  
      \textbf{Solution.} The orders of the elements $\overline{1}$,
      $\overline{-1}$, $\overline{5}$, $\overline{13}$, $\overline{-13}$,
      $\overline{17}$ in $(\Z/36\Z)^\times$ are 1, 35, 29, 25, 11, and 17.
%%%%%%%%%%%%%%%%%%%%%%%%%%%%%%%%%%%%%%1.15%%%%%%%%%%%%%%%%%%%%%%%%%%%%%%%%%%%%%%
   \item[1.1.15]  Prove that $(a_1a_2\cdots a_n)^{-1} =
                  {a_n}^{-1}{a_{n-1}}^{-1}\cdots {a_1}^{-1}$ for all
                  $a_1, a_2, \ldots, a_n \in G$.
                  
      \textbf{Proof.} We shall proceed by induction on $n$. The statement is
      trivial for $n = 1$. So assume that it also holds for some positive
      integer $k$. Let $b = a_1a_2\cdots a_k$. It then follows that
      \begin{align*}
         (a_1a_2\cdots a_ka_{k+1})^{-1} &= (b \cdot a_{k+1})^{-1} \\
            &= {a_{k+1}}^{-1}b^{-1} &[\text{By Proposition 1 (4)}] \\
            &= {a_{k+1}}^{-1}{a_k}^{-1}\cdots {a_1}^{-1}.
                  &[\text{Inductive hypothesis}]
      \end{align*}
      That is, our statement holds for $k + 1$, so that, by the Principle of
      Mathematical Induction, it holds for each positive integer $n$. \qed
%%%%%%%%%%%%%%%%%%%%%%%%%%%%%%%%%%%%%%1.16%%%%%%%%%%%%%%%%%%%%%%%%%%%%%%%%%%%%%%
   \item[1.1.16]  Let $x$ be an element of $G$. Prove that $x^2 = 1$ if and only
                  if $|x|$ is either 1 or 2.
                  
      \textbf{Proof.}
      
      $(\Leftarrow)$ Suppose that $x^2 = 1$. Now if $|x| > 2$, then by
      definition, $x^2 \neq 1$. The only remaining possibilities are $|x| = 1$
      or $|x| = 2$.
      
      $(\Rightarrow)$ Suppose that $|x| = 1$ or $|x| = 2$. It immediately
      follows that $x^2 = 1$. \qed
%%%%%%%%%%%%%%%%%%%%%%%%%%%%%%%%%%%%%%1.17%%%%%%%%%%%%%%%%%%%%%%%%%%%%%%%%%%%%%%
   \item[1.1.17]  Let $x$ be an element of $G$. Prove that if $|x| = n$ for some
                  positive integer $n$ then $x^{-1} = x^{n-1}$.
                  
      \textbf{Proof.} Suppose that $|x| = n \in \Z^+$. By Exercise 1.1.18, it
      follows that $x^{n-1}x^1 = x^{n-1+1} = x^n = 1$, so that
      $x^{-1} = x^{n-1}$. \qed      
%%%%%%%%%%%%%%%%%%%%%%%%%%%%%%%%%%%%%%1.18%%%%%%%%%%%%%%%%%%%%%%%%%%%%%%%%%%%%%%
   \item[1.1.18]  Let $x$ and $y$ be elements of $G$. Prove that $xy = yx$ if
                  and only if $y^{-1}xy =x$ if and only if $x^{-1}y^{-1}xy = 1$.
                  
      \textbf{Proof.} First assume that $xy = yx$. We then have that
      $yx = xy = 1xy = yy^{-1}xy$, so that $x = y^{-1}xy$ by left cancellation.
      Now assume that $y^{-1}xy = x$. Thus
      $x1 = x = y^{-1}xy = 1y^{-1}xy = xx^{-1}y^{-1}xy$, so that
      $1 = x^{-1}y^{-1}xy$ by left cancellation. Finally assume that
      $x^{-1}y^{-1}xy = 1$. Multiplying on the left by $yx$ will yield the
      equation $xy = yx$. \qed
%%%%%%%%%%%%%%%%%%%%%%%%%%%%%%%%%%%%%%1.19%%%%%%%%%%%%%%%%%%%%%%%%%%%%%%%%%%%%%%
   \item[1.1.19]  Let $x \in G$ and let $a, b \in \Z^+$.
                  \begin{enumerate}
                     \item Prove that $x^{a+b} = x^ax^b$.
                     \item Prove that $(x^a)^b = x^{ab}$.
                     \item Prove that $(x^a)^{-1} = x^{-a}$.
                     \item Establish part (a) for arbitrary integers $a$ and $b$
                           (positive, negative or zero).
                     \item Establish part (b) for arbitrary integers $a$ and $b$
                           (positive, negative or zero).
                  \end{enumerate}
               
      \textbf{Proof.}
      
      \begin{enumerate}
         \item We have that
               \begin{align*}
                  x^{a+b} &= \underbrace{xx\cdot x}_{a+b \text{ factors}} \\
                          &= \underbrace{xx\cdot x}_{a \text{ factors}}\mbox{ }
                             \underbrace{xx\cdot x}_{b \text{ factors}} \\
                          &= x^ax^b.
               \end{align*} \qed
         \item We have that
               \begin{align*}
                  (x^a)^b &= (\underbrace{xx\cdot x}_{a \text{ factors}})^b \\
                          &= \underbrace{xx\cdot x}_{ab \text{ factors}} \\
                          &= x^{ab}.
               \end{align*} \qed
         \item We have
               \begin{align*}
                  (x^a)^{-1}
                     &= (\underbrace{xx\cdot x}_{a \text{ factors}})^{-1} \\
                     &= \underbrace{x^{-1}x^{-1}\cdot x^{-1}}_{
                           a \text{ factors}} &[\text{Exercise 1.1.15}] \\
                     &= x^{-a}.
               \end{align*} \qed
         \item Now suppose that $a$ is an integer and $b$ is a positive integer.
               We shall induct on $b$ to show that
               \begin{equation}
                  x^{a+b} = x^ax^b. \label{1_1_19_1}
               \end{equation}
               By Lemma 1.1.1, \eqref{1_1_19_1} holds if $b$ equals 1. So assume
               that it also holds for some positive integer $k$. We now have
               that
               \begin{align*}
                  x^ax^{k+1} &= x^ax^kx^1 &[\text{Lemma 1.1.1}] \\
                             &= (x^ax^k)x^1 \\
                             &= x^{a+k}x^1 &[\text{Inductive hypothesis}] \\
                             &= x^{(a+k)+1} &[\text{Lemma 1.3.2}] \\
                             &= x^{a+(k+1)}, &[\text{Associativity of addition}]
               \end{align*}
               so that \eqref{1_1_19_1} holds for $k + 1$, and hence, by the 
               Principle of Mathematical Induction, it holds for each positive
               integer $n$. \\

               If $a$ is 0 or $b$ is 0, then Lemma 1.1.1 tells us that
               \eqref{1_1_19_1} holds, so the only remaining possibility is $a$ 
               and $b$ are negative.\footnote{If $a$ is positive and $b$ is
               negative, then interchange the roles of $a$ and $b$ in the 
               induction proof.} Now suppose that $a$ and $b$ are negative.
               Hence
               \begin{align*}
                  x^ax^b &= x^{-(-a)}x^{-(-b)} \\
                     &= (x^{-1})^{-a}(x^{-1})^{-b} &[\text{Definition}] \\
                     &= (x^{-1})^{(-a + (-b))} &[\text{Part (a)}] \\
                     &= x^{-(-a + (-b))} &[\text{Definition}] \\
                     &= x^{a+b}.
               \end{align*}

               Combining this result with part (a), we thus shown that
               \eqref{1_1_19_1} holds for all integers $a$ and $b$. \qed
         \item It is clear that part (b) holds if $a$ is 0 or $b$ is 0, so let
               us complete the proof for arbritrary integers $a$ and $b$.

               \textbf{Case 1:} \textit{$a$ is positive and $b$ is negative}. 
               Hence
               \begin{align*}
                  (x^a)^b &= (x^a)^{-(-b)} \\
                          &= [(x^a)^{-1}]^{-b} &[\text{Definition}] \\
                          &= (x^{-a})^{-b} &[\text{Part (c)}] \\
                          &= [(x^{-1})^a]^{-b} &[\text{Definition}] \\
                          &= (x^{-1})^{-ab} &[\text{Part (b)}] \\
                          &= x^{-(-ab)} &[\text{Definition}] \\
                          &= x^{ab}.
               \end{align*}

               \textbf{Case 2:} \textit{$a$ and $b$ are negative}. Thus
               \begin{align*}
                  (x^a)^b &= [x^{-(-a)}]^b \\
                          &= [(x^{-1})^{-a}]^b &[\text{Definition}] \\
                          &= (x^{-1})^{-ab} &[\text{Case 1}] \\
                          &= [(x^{-1})^{-1}]^{ab}. &[\text{Definition}] \\
                          &= x^{ab}. &[\text{Proposition 1 (3)}]
               \end{align*}

               \textbf{Case 3:} \textit{$a$ is negative and $b$ is positive}. 
               Thus
               \begin{align*}
                  (x^a)^b &= [x^{-(-a)}]^b \\
                          &= [(x^{-1})^{-a}]^b &[\text{Definition}] \\
                          &= (x^{-1})^{-ab} &[\text{Case 1}] \\
                          &= x^{-(-ab)} &[\text{Definition}] \\
                          &= x^{ab}.
               \end{align*}

               Combining these results with part (a), we can conclude that
               $(x^a)^b = x^{ab}$ holds for all integers $a$ and $b$ and
               $x \in G$. \qed
      \end{enumerate}
%%%%%%%%%%%%%%%%%%%%%%%%%%%%%%%%%%%%%%1.20%%%%%%%%%%%%%%%%%%%%%%%%%%%%%%%%%%%%%%
   \item[1.1.20]  For $x$ an element in $G$ show that $x$ and $x^{-1}$ have the
                  same order.

      \textbf{Proof.}

      \textbf{Case 1:} \textit{$|x| = n \in \Z^+$}. Since
      $(x^{-1})^n = (x^n)^{-1} = 1^{-1} = 1$, it follows that $|x^{-1}| \le n$,
      so suppose to the contrary that $|x^{-1}| = m < n$. Then we have that
      $$x^m = [(x^{-1})^{-1}]^m = [(x^{-1})^m]^{-1} = 1^{-1} = 1,$$
      a contradiction, so that $|x^{-1}| = n = |x|$.

      \textbf{Case 2:} \textit{$|x| = +\infty$}. Suppose to the contrary that
      $|x^{-1}| = n \in \Z^+$. As we argued in Case 1, it must be the case that
      $x^n = 1$, a contradiction. Thus $|x| = +\infty = |x^{-1}|$. \qed
%%%%%%%%%%%%%%%%%%%%%%%%%%%%%%%%%%%%%%1.20%%%%%%%%%%%%%%%%%%%%%%%%%%%%%%%%%%%%%%
   \item[1.1.21]  Let $G$ be a finite group and let $x$ be an element of $G$ of
                  order $n$. Prove that if $n$ is odd, then $x = (x^2)^k$ for
                  some $k$.

      \textbf{Proof.} Suppose that $n$ is odd. We can then write $n = 2k + 1$
      for some nonnegative integer $k$. By supposition, we have that
      $xx^{2k} = x^{2k+1} = 1 = x^{-2k}x^{2k}$, so that by right cancellation,
      we can conclude that $x = x^{-2k} = (x^2)^{-k}$. \qed
%%%%%%%%%%%%%%%%%%%%%%%%%%%%%%%%%%%%%%1.22%%%%%%%%%%%%%%%%%%%%%%%%%%%%%%%%%%%%%%
   \item[1.1.22]  If $x$ and $g$ are elements of the group $G$, prove that
                  $|x| = |g^{-1}xg|$. Deduce that $|ab| = |ba|$ for all
                  $a, b \in G$.

      \textbf{Proof.} Let $x, g \in G$.

      \textbf{Case 1:} \textit{$|x| = n \in \Z^+$}. By Lemma 1.1.2, it follows
      that $(g^{-1}xg)^n = g^{-1}x^ng = g^{-1}g =1$, so that $|g^{-1}xg| \le n$,
      so suppose to the contrary that $|g^{-1}xg| = m < n$. Then we have that
      $$g^{-1}1g = 1 = (g^{-1}xg)^m = g^{-1}x^mg,$$
      so that $x^m = 1$ by left and right cancellations, a contradiction; thus,  
      $|g^{-1}xg| = n = |x|$.

      \textbf{Case 2:} \textit{$|x| = +\infty$}. Suppose to the contrary that
      $|g^{-1}xg| = n \in \Z^+$. As we argued in Case 1, it must then be the 
      case that $x^n = 1$, a contradiction. Thus $|x| = +\infty = |g^{-1}xg|$.

      Now consider $a, b \in G$. Set $x = ab$ and $g = a$. Since 
      $|x| = |g^{-1}xg|$, it follows that $|ab| = |a^{-1}aba| = |ba|$. \qed
%%%%%%%%%%%%%%%%%%%%%%%%%%%%%%%%%%%%%%1.23%%%%%%%%%%%%%%%%%%%%%%%%%%%%%%%%%%%%%%
   \item[1.1.23]  Suppose $x \in G$ and $|x| = n < \infty$. If $n = st$ for some
                  positive integers $s$ and $t$, prove that $|x^s| = t$.

      \textbf{Proof.} Suppose $n = st$ for some positive integers $s$ and $t$.
      By supposition, we have that $1 = x^n = x^{st} = (x^s)^t$; i.e.,
      $|x^s| \le t$. Suppose to the contrary that $|x^s| = m < t$. Then we have
      that $1 = (x^s)^m = x^{sm}$. Since $0 < m < t$, it follows that
      $0 < sm < st = n$. However $|x| = n$ and we just showed that $x^{sm} = 1$, 
      so that we have a contradiction. Hence we can conclude that $|x^s| = |t|$.
      \qed
%%%%%%%%%%%%%%%%%%%%%%%%%%%%%%%%%%%%%%1.24%%%%%%%%%%%%%%%%%%%%%%%%%%%%%%%%%%%%%%
   \item[1.1.24]  If $a$ and $b$ are \textit{commuting} elements of $G$, prove 
                  that $(ab)^n = a^nb^n$ for all $n \in \Z$. [Do this by 
                  induction for positive $n$ first.]

      \textbf{Proof.} Let $R(n)$ be the statement that $(ab)^n = a^nb^n$, for
      commuting elements $a$ and $b$.
               
      We now want to show using induction that $R(n)$ holds for every positive 
      integer $n$. It is clear that $R(1)$ is true. So suppose that $R(k)$ is 
      true for some positive integer $k$. We must now show that $R(k + 1)$ is 
      also true. Now we have that
      \begin{align*}
         (ab)^{k+1} &= (ab)^k(ab)^1 &[\text{Exercise 1.1.19}] \\
                    &= a^kb^k(ab)^1 &[\text{Since }R(k) \text{ is true}] \\
                    &= a^kb^k(ba)^1 &[ab = ba] \\
                    &= a^kb^kba \\
                    &= a^kb^{k+1}a \\
                    &= a^kab^{k+1} &[\text{$a$ commutes with $b$}] \\
                    &= a^{k+1}b^{k+1}, \\
      \end{align*}
      so that $R(k + 1)$ holds. It follows by the Principle of Mathematical 
      Induction that $R(n)$ holds for every positive integer $n$. By inpsection 
      we can see that $R(0)$ also holds. To complete the proof, we must now show 
      that $(ab)^{m} = a^mb^m$, where $m$ is a negative integer. First we notice 
      that
      \begin{equation}
         a^{-1}b^{-1} = (ba)^{-1} = (ab)^{-1} = b^{-1}a^{-1},
         \label{1_1_24_1}
      \end{equation}
      so that $a^{-1}$ and $b^{-1}$ are commuting elements. Thus it follows that
      \begin{align*}
         (ab)^m &= (ab)^{-(-m)} \\
                &= [(ab)^{-1}]^{-m} &[\text{Definition}] \\
                &= (a^{-1}b^{-1})^{-m} &[\eqref{1_1_24_1}] \\
                &= (a^{-1})^{-m}(b^{-1})^{-m} &[\text{$R(-m)$ holds}] \\
                &= a^mb^m,
      \end{align*}
      as desired. \qed
%%%%%%%%%%%%%%%%%%%%%%%%%%%%%%%%%%%%%%1.25%%%%%%%%%%%%%%%%%%%%%%%%%%%%%%%%%%%%%%
   \item[1.1.25]  Prove that if $x^2 = 1$ for all $x \in G$ then $G$ is abelian.

      \textbf{Proof.} Let $G$ be a group. Suppose that $x^2 = 1$ for all
      $x \in G$. We want to show that $G$ is abelian; that is, we want to show 
      that $xy = yx$ for all $x, y \in G$. So let $x, y \in G$. By hypothesis, 
      we have that $x^2 = e$, $y^2 = e$, and $(xy)^2 = e$, so that according to 
      Proposition 2, we must have that $x = x^{-1}$, $y = y^{-1}$, and
      $xy = (xy)^{-1}$. Thus
      \begin{align*}
         xy &= (xy)^{-1}      &[\text{By Hypothesis}] \\
            &= y^{-1}x^{-1}   &[\text{Proposition 1}] \\
            &= yx.
      \end{align*}
      Thus $G$ is abelian. \qed
%%%%%%%%%%%%%%%%%%%%%%%%%%%%%%%%%%%%%%1.26%%%%%%%%%%%%%%%%%%%%%%%%%%%%%%%%%%%%%%
   \item[1.1.26]  Assume $H$ is a nonempty subset of $(G, *)$ which is closed 
                  under the binary operation on $G$ and is closed under
                  inverses, i.e., for all $h$ and
                  $k \in H$, $hk$ and $h^{-1} \in H$. Prove that $H$ is a group 
                  under the operation $*$ restricted to $H$ (such a subset $H$
                  is called a subgroup of $G$).

      \textbf{Proof.} We know that $H$ is closed under $*$ and under inverses, 
      so it suffices to show that $*$ is associative on $H$ and that $H$ has an 
      identity under $*$. The associativity of $H$ under $*$ follows because $H$ 
      is a subset of $G$ and $G$ is associative under $*$. Since $H$ is nonempty
      we pick an $h \in H$. Then by hypothesis, we have that
      $1 = hh^{-1} \in H$, so that $H$ contains the identity. (Note that
      $hh^{-1} = h^{-1}h = 1$ and $h1 = 1h = h$ because these equalities hold in
      $G$.) \qed
%%%%%%%%%%%%%%%%%%%%%%%%%%%%%%%%%%%%%%1.27%%%%%%%%%%%%%%%%%%%%%%%%%%%%%%%%%%%%%%
   \item[1.1.27]  Prove that if $x$ is an element of the group $G$ then
                  $\{x^n : n \in \Z\}$ is a subgroup of $G$ (called the
                  \textit{cyclic subgroup} of $G$ generated by $x$).

      \textbf{Proof.} Consider the set
      $$H = \{x^n : n \in \Z\}.$$
      $H$ is nonempty because it contains $1 = x^0$. So let $h_1, h_2 \in H$.
      Thus we have $h_1 = x^a$ and $h_2 = x^b$ for some integers $a$ and $b$, so
      that $h_1h_2 = x^ax^b = x^{a+b} \in H$; in other words, $H$ is closed
      under the operation of $G$. Since $h_1^{-1} = (x^a)^{-1} = x^{-a} \in H$, 
      it follows that $H$ is also closed under inverses, so that $H$ is a
      subgroup of $G$ by Exercise 1.1.26.
%%%%%%%%%%%%%%%%%%%%%%%%%%%%%%%%%%%%%%1.28%%%%%%%%%%%%%%%%%%%%%%%%%%%%%%%%%%%%%%
   \item[1.1.28]  Let $(A, *)$ and $(B, \diamond)$ be groups and let
                  $A \times B$ be their direct product (as defined in Example
                  6). Verify all the group axioms for $A \times B$.
                  \begin{enumerate}
                     \item prove that the associative law holds: for all
                           $(a_i, b_i) \in A \times B, i = 1, 2, 3$
                           $$(a_1, b_1)[(a_2, b_2)(a_3, b_3)] =
                            [(a_1, b_1)(a_2, b_2)](a_3, b_3),$$
                     \item prove that (1, 1) is the identity of $A \times B$,
                           and
                     \item prove that the inverse of $(a, b)$ is
                           $(a^{-1}, b^{-1})$.
                  \end{enumerate}

      \textbf{Proof.} Let $(a_1, b_1)$, $(a_2, b_2)$, and
      $(a_3, b_3) \in A \times B$.

      \begin{enumerate}
         \item The set $A \times B$ is associative under the component wise
               operations of $A$ and $B$ because
               \begin{align*}
                  (a_1, b_1)[(a_2, b_2)(a_3, b_3)]
                     &= (a_1, b_1)(a_2a_3, b_2b_3) \\
                     &= (a_1a_2a_3, b_1b_2b_3) \\
                     &= [(a_1a_2)a_3, (b_1b_2)b_3] &[\text{Associativity}] \\
                     &= (a_1a_2, b_1b_2)(a_3, b_3) \\
                     &= [(a_1, b_1)(a_2, b_2)](a_3, b_3).
               \end{align*}
         \item Consider $(1, 1) \in A \times B$. It follows that
               \begin{align*}
                  (1, 1)(a_1, b_1) &= (1a_1, 1b_1) \\
                                   &= (a_1, b_1) \\
                                   &= (a_11, b_11) \\
                                   &= (a_1, b_1)(1, 1),
               \end{align*}
               so that $(1, 1)$ is the identity of $A \times B$.
         \item Consider $(a, b) \in A \times B$. It 
               follows that
               \begin{align*}
                  (a, b)(a^{-1}, b^{-1}) &= (aa^{-1}, bb^{-1}) \\
                                   &= (1, 1) \\
                                   &= (a^{-1}a, b^{-1}b) \\
                                   &= (a^{-1}, b^{-1})(a, b),
               \end{align*}
               so that $(a^{-1}, b^{-1})$ is the inverse of $(a, b)$.
      \end{enumerate}
%%%%%%%%%%%%%%%%%%%%%%%%%%%%%%%%%%%%%%1.29%%%%%%%%%%%%%%%%%%%%%%%%%%%%%%%%%%%%%%
   \item[1.1.29]  Prove that $A \times B$ is an abelian group if and only if
                  both $A$ and $B$ are abelian.

      \textbf{Proof.} 

      $(\Leftarrow)$ Suppose that $A$ and $B$ are abelian. Let $(a_1, b_1)$ and
      $(a_2, b_2) \in A \times B$. It follows that $A \times B$ is abelian
      because
      \begin{align*}
         (a_1, b_1)(a_2, b_2) &= (a_1a_2, b_1b_2) \\
            &= (a_2a_1, b_2b_1) &[\text{$A$ and $B$ are abelian}] \\
            &= (a_2, b_2)(a_1, b_1).
      \end{align*}

      $(\Rightarrow)$ Now suppose that $A \times B$ is abelian. Let $a_1$ and
      $a_2$ be members of $A$ and let $b_1$ and $b_2$ be members of $B$. Then
      we have that
      \begin{align*}
         (a_1a_2, b_1b_2) = (a_1, b_1)(a_2, b_2) \\
            &= (a_2, b_2)(a_1, b_1) &[\text{$A \times B$ is abelian}] \\
            &= (a_2a_1, b_2b_1),
      \end{align*}
      so that $(a_1a_2, b_1b_2) = (a_2a_1, b_2b_1)$; i.e., $a_1a_2 = a_2a_1$ and
      $b_1b_2 = b_2b_1$. We can now conclude that $A$ and $B$ are both abelian.
      \qed
%%%%%%%%%%%%%%%%%%%%%%%%%%%%%%%%%%%%%%1.30%%%%%%%%%%%%%%%%%%%%%%%%%%%%%%%%%%%%%%
   \item[1.1.30]  Prove that the elements $(a, 1)$ and $(1, b)$ of $A \times B$
                  commute and deduce that the order of $(a, b)$ is the least 
                  common multiple of $|a|$ and $|b|$.

      \textbf{Proof.} Let $A$ and $B$ be groups, and let $a \in A$, $b \in B$.
      We shall be assuming that there exist positive integers $m$ and $n$ such 
      that $|a| = m$ and $|b| = n$, for the problem does not make sense if the
      order of $a$ or $b$ is not finite. Consider $(a, 1)$,
      $(1, b) \in A \times B$. We have that
      \begin{align*}
         (a, 1)(1, b) &= (a1, 1b) \\
                      &= (a, b) \\
                      &= (1a, b1) \\
                      &= (1, b)(a, 1),
      \end{align*}
      so that $(a, 1)$ and $(b, 1)$ commute. To complete the proof, we let
      $s = \text{lcm}(m, n)$. Thus we can write $s = mx = ny$ for positive 
      integers $x$ and $y$. Thus we have that
      \begin{align*}
         (a, b)^s &= (a^s, b^s) &[\text{Lemma 1.1.4}] \\
                  &= (a^{mx}, a^{ny}) \\
                  &= [(a^m)^x, (a^n)^y] \\
                  &= (1^x, 1^y) \\
                  &= (1, 1).
      \end{align*}
      This say that $|(a, b)| \le s$, so there exists a positive integer $q$ 
      such that $|(a, b)| = q$. By Lemma 1.1.4, we have that
      $(a, b)^q = (a^q, b^q) = (1, 1)$, so that $a^q = 1$ and $b^q = 1$. Thus by 
      Lemma 1.1.3, it follows that $m \mid q$ and $n \mid q$, so that $s \mid q$ 
      by definition of the lcm. Since $s \mid q$, we must have that $s \le q$.
      But we previously showed that $q \le s$. Thus we can conclude that
      $s = q$, as desired. \qed
%%%%%%%%%%%%%%%%%%%%%%%%%%%%%%%%%%%%%%1.31%%%%%%%%%%%%%%%%%%%%%%%%%%%%%%%%%%%%%%
   \item[1.1.31]  Prove that any finite group $G$ of even order contains an
                  element of order 2. [Let $t(G)$ be the set
                  $\{g \in G : g \neq g^{-1}\}$. Show that $t(G)$ has an even 
                  number of elements and every nonidentity element of $G - t(G)$ 
                  has order 2.]

      \textbf{Proof.} Let $G$ be a finite group of even order. We wish to show
      that there exists some $g \in G$ such that $|g| = 2$. Consider this subset
      of $G$:
      $$S = \{g \in G: g \neq g^{-1}\}.$$

      If $|S| = 0$, then the proof is done, so assume that $|S| > 0$. Now $|S|$ 
      is even, for if this were not the case, then if we pair up every element
      of $S$ with its inverse, then one element must be without an inverse, a 
      contradiction. Now let $S' = G\backslash S$. It follows that
      $|G| = |S| + |S'|$. Notice that $S'$ is not empty because $e \in S'$. 
      Since $G$ and $S$ are both even, it follows that $|S'|$ must also be even. 
      Since we already showed that $|S'| \ge 1$, we can conclude that
      $|S'| \ge 2$, so that $S'$ contains a non-identity $a$, where
      $a = a^{-1}$. That is, $|a| = 2$. \qed
%%%%%%%%%%%%%%%%%%%%%%%%%%%%%%%%%%%%%%1.32%%%%%%%%%%%%%%%%%%%%%%%%%%%%%%%%%%%%%%
   \item[1.1.32]  If $x$ is an element of finite order $n$ in $G$, prove that
                  the elements 1, $x$, $x^2$, $\ldots$, $x^{n-1}$ are all 
                  distinct. Deduce that $|x| \le |G|$.

      \textbf{Proof.} Suppose that $|x| = n \in \Z^+$ for some $x \in G$. 
      Suppose to the contrary that the elements $x^0$, $x^1$, $x^2$, $\ldots$, 
      $x^{n-1}$ are not distinct. Then we must have that $x^i = x^j$ for some
      integer $i$ and $j$ where $0 \le i < j \le n - 1$. That is, $x^{j-i} = 1$,
      a contradiction because $j - i$ is a positive integer less thatn $n$. It
      follows that the elements $x^0$, $x$, $x^2$, $\ldots$, $x^{n-1}$ are all 
      distinct. Since there are clearly $n$ of these elements and since they are
      all members of $G$, we can conclude that $|x| = n \le |G|$. \qed
%%%%%%%%%%%%%%%%%%%%%%%%%%%%%%%%%%%%%%1.33%%%%%%%%%%%%%%%%%%%%%%%%%%%%%%%%%%%%%%
   \item[1.1.33]  Let $x$ be an element of finite order $n$ in $G$.
                  \begin{enumerate}
                     \item Prove that if $n$ is odd then $x^i \neq x^{-i}$ for
                           all $i = 1, 2, \ldots, n - 1$,
                     \item Prove that if $n = 2k$ and $1 \le i < n$ then
                           $x^i = x^{-i}$ if and only if $i = k$.
                  \end{enumerate}

      \textbf{Proof.}

      \begin{enumerate}
         \item Suppose that $n$ is odd. Now we shall suppose to the contrary
               that $x^i = x^{-i}$ for some integer $1 \le i \le n - 1$. Since
               $x^i = x^{-i}$, it follows that $x^{2i} = 1$. By Lemma 1.1.3, we
               must have that $n \mid 2i$, a contradiction because an odd
               number cannot divide a positive even number, so we conclude that
               $x^i \neq x^{-i}$ for all $i = 1, 2, \ldots, n - 1$. \qed
         \item Suppose that $n$ is even and $1 \le i < n$. Write $n = 2k$ for
               some positive integer $k$.

               $(\Leftarrow)$ Suppose that $i = k$. Then we have that
               $1 = x^{2k} = x^{2i} = x^ix^i$, so that $x^i = x^{-i}$.

               $(\Rightarrow)$ Conversely suppose that $x^i = x^{-i}$, so that
               $x^{2i} =1$. Thus, by Lemma 1.1.3, $2k \mid 2i$, or equivalently,
               $k \mid i$, so that $i = mk$ for some positive integer $m$. 
               Recall that $i < n = 2k$ by hypothesis, so that $mk < 2k$. That 
               is $m < 2$. But $m$ is a positive integer and so the only
               possibility is therefore $m = 1$, so that $i = k$. \qed
      \end{enumerate}
%%%%%%%%%%%%%%%%%%%%%%%%%%%%%%%%%%%%%%1.34%%%%%%%%%%%%%%%%%%%%%%%%%%%%%%%%%%%%%%
   \item[1.1.34]  If $x$ is an element of infinite order in $G$, prove that the
                  elements $x^n$, $n \in \Z$ are all distinct.

      \textbf{Proof.} Assume that $x$ is an element of infinite order in $G$.
      Now suppose to the contrary that $x^i = x^j$ for some unequal integers
      $i$ and $j$. We can further assume without loss of generality that
      $i < j$. Thus $x^{j-i} = 1$, a contradiction because this says that
      $|x| \le j - i$. It follows that distinct integral powers of $x$ yield 
      distinct elements of $G$. \qed
%%%%%%%%%%%%%%%%%%%%%%%%%%%%%%%%%%%%%%1.35%%%%%%%%%%%%%%%%%%%%%%%%%%%%%%%%%%%%%%
   \item[1.1.35]  If $x$ is an element of finite order $n$ in $G$, use the 
                  Division Algorithm to show that any integral power of $x$ 
                  equals one of the elements in the set
                  $\{1, x, x^2, \ldots, x^{n-1}\}$ (so these are all the
                  distinct elements of the cyclic subgroup of $G$ generated by
                  $x$).

      \textbf{Proof.} Assume that $x$ is an element of finite order $n$ in $G$.
      Let $z \in \Z$. By the Division Algorithm, there exist unique integers
      $q$ and $r$ such that $z = qn + r$ and $0 \le r < n$. That is
      $$x^z = x^{qn+r} = x^{qn}x^r = (x^n)^qx^r = 1^qx^r = x^r.$$
      Since $r \in \{0, 1, \ldots, n - 1\}$ and since $x^z = x^r$, it follows
      that $x^z \in \{x^0, x^1, \ldots, x^{n-1}\}$. \qed
%%%%%%%%%%%%%%%%%%%%%%%%%%%%%%%%%%%%%%1.36%%%%%%%%%%%%%%%%%%%%%%%%%%%%%%%%%%%%%%
   \item[1.1.36]  Assume $G = \{1, a, b, c\}$ is a group of order 4 with
                  identity 1. Assume also that $G$ has no elements of order 4
                  (so by Exercise 32, every element has order $\le$ 3). Use the
                  cancellation laws to show that there is a unique group table
                  for $G$. Deduce that $G$ is abelian.

      \textbf{Proof.} Assume $G = \{1, a, b, c\}$. We can tentatively fill out 
      the group table for $G$ like so:      
      $$
         \begin{tabular}{@{}c | c | c | c | c@{}} 
                & $1$ & $a$ & $b$ & $c$ \\ \hline
            $1$ & $1$ & $a$ & $b$ & $c$ \\ \hline
            $a$ & $a$ & $ $ & $ $ & $ $ \\ \hline
            $b$ & $b$ & $ $ & $ $ & $ $ \\ \hline
            $c$ & $c$ & $ $ & $ $ & $ $
         \end{tabular}
      $$
      By the left cancellation law, the equality $ab = a$ will result in $b = e$
      and the equality $ab = b$ will result in $a = e$, both of which are
      contradictions. The only remaining possiblities are $ab = c$ or $ab = 1$.

      \textbf{Case 1:} $ab = c$. For the same reason as above, we cannot have
      $ac = a$ or $ac = c$, so that $ac = 1$ or $ac = b$. So suppose first that
      $ac = b$. Then our table will now look like so:
      $$
         \begin{tabular}{@{}c | c | c | c | c@{}} 
                & $1$ & $a$ & $b$ & $c$ \\ \hline
            $1$ & $1$ & $a$ & $b$ & $c$ \\ \hline
            $a$ & $a$ & $ $ & $c$ & $b$ \\ \hline
            $b$ & $b$ & $ $ & $ $ & $ $ \\ \hline
            $c$ & $c$ & $ $ & $ $ & $ $
         \end{tabular}
      $$
      From the table above, we see that $aa$ must be equal to 1, since that is
      the only remaining possibility. The cancellation laws tell us that every
      element in a column and row of a group table must be unique, so we must
      have that:
      $$
         \begin{tabular}{@{}c | c | c | c | c@{}} 
                & $1$ & $a$ & $b$ & $c$ \\ \hline
            $1$ & $1$ & $a$ & $b$ & $c$ \\ \hline
            $a$ & $a$ & $1$ & $c$ & $b$ \\ \hline
            $b$ & $b$ & $c$ & $ $ & $ $ \\ \hline
            $c$ & $c$ & $b$ & $ $ & $ $
         \end{tabular}
      $$
      Note that we cannot have $bb = a$ because that would imply that $bbb = c$,
      so that $|b| > 3$, contradicting our hypothesis. Thus we must have that
      $bb = 1$. The remaining positions are thus completely determined, so that
      we have
      $$
         \begin{tabular}{@{}c | c | c | c | c@{}} 
                & $1$ & $a$ & $b$ & $c$ \\ \hline
            $1$ & $1$ & $a$ & $b$ & $c$ \\ \hline
            $a$ & $a$ & $1$ & $c$ & $b$ \\ \hline
            $b$ & $b$ & $c$ & $1$ & $a$ \\ \hline
            $c$ & $c$ & $b$ & $a$ & $1$
         \end{tabular}
      $$
      Now suppose that $ac = 1$, then we would be forced to fill in the table
      like so:
      $$
         \begin{tabular}{@{}c | c | c | c | c@{}} 
                & $1$ & $a$ & $b$ & $c$ \\ \hline
            $1$ & $1$ & $a$ & $b$ & $c$ \\ \hline
            $a$ & $a$ & $b$ & $c$ & $1$ \\ \hline
            $b$ & $b$ & $c$ & $ $ & $ $ \\ \hline
            $c$ & $c$ & $1$ & $ $ & $ $
         \end{tabular}
      $$
      Since $a^2 = b$ and $a^3 = c$, we have that $|a| > 3$, contradicting our
      hypothesis, so this is a dead end.

      \textbf{Case 2:} $ab = 1$. For the same reason as above, we cannot have
      $ac = a$ or $ac = c$, so that $ac = 1$ or $ac = b$. So suppose first that
      $ac = b$. Then our table will now look like so:
      $$
         \begin{tabular}{@{}c | c | c | c | c@{}} 
                & $1$ & $a$ & $b$ & $c$ \\ \hline
            $1$ & $1$ & $a$ & $b$ & $c$ \\ \hline
            $a$ & $a$ & $c$ & $1$ & $b$ \\ \hline
            $b$ & $b$ & $1$ & $c$ & $a$ \\ \hline
            $c$ & $c$ & $b$ & $a$ & $1$
         \end{tabular}
      $$
      Since $a^2 = c$ and $a^3 = b$, we have that $|a| > 3$, contradicting our
      hypothesis, so this is another dead end. From our arguments above, we see
      that the only viable and legal table is thus:
      $$
         \begin{tabular}{@{}c | c | c | c | c@{}} 
                & $1$ & $a$ & $b$ & $c$ \\ \hline
            $1$ & $1$ & $a$ & $b$ & $c$ \\ \hline
            $a$ & $a$ & $1$ & $c$ & $b$ \\ \hline
            $b$ & $b$ & $c$ & $1$ & $a$ \\ \hline
            $c$ & $c$ & $b$ & $a$ & $1$
         \end{tabular}
      $$
      This table is unique, and since it is symmeteric it follows that $G$ is
      abelian. \qed
\end{enumerate}

      \section{Dihedral Groups}
         \begin{enumerate}
%%%%%%%%%%%%%%%%%%%%%%%%%%%%%%%%%%%Prob1.2_1%%%%%%%%%%%%%%%%%%%%%%%%%%%%%%%%%%%%
   \item[1.2.1]   For each of the following statements, determine whether it is 
                  true or false and justify your answer.
                  \begin{enumerate}
                     \item The set $\Z$ of integers is dense in $\R$.
                     \item The set of positive real numbers is dense in $\R$.
                     \item The set $\Q\backslash \Z$ of rational numbers that 
                           are not integers is dense in $\R$.
                  \end{enumerate}  

      \textbf{Solution:} 

      \begin{enumerate}
         \item False. Proposition 1.6 states that there is no integer in the
               interval (0, 1).
         \item False. The interval $(-1, 0)$ contains no positive real number.
         \item True. Let $a$ and $b$ be real numbers. Then we shall investigate
               the following two cases:
               
               \textbf{Case I:} $a < a + 1 \le b$. Theorem 1.8 says that there
               exists a unique integer $k$ in $[a, a + 1)$. Thus there is no
               integer in the interval $(k, a + 1)$. By the density of $\Q$ in
               $\R$, there exists a rational $q \in (k, a + 1)$. Since
               $(k, a + 1)$ contains no integer, then $q$ must be a member of
               $\Q\backslash\Z$. We observe that $q \in (a, b)$.
               
               \textbf{Case II:} $a < b < a + 1$. Theorem 1.8 says that there
               exists a unique integer $k$ in $[a, a + 1)$. If $k  \le b$, then
               $(a, k)$ has no integer, so there exists a noninteger rational
               in $(a, k) \subseteq (a, b)$ by the density of $\Q$ in $\R$. If,
               however, $k > b$, then the interval $(a, b)$ contains no integer,
               so that there exists a noninteger rational in $(a, b)$ by the
               density of $\Q$ in $\R$.
      \end{enumerate}
%%%%%%%%%%%%%%%%%%%%%%%%%%%%%%%%%%Prob1.2_2%%%%%%%%%%%%%%%%%%%%%%%%%%%%%%%%%%%%%
   \item[1.2.2]   Suppose that $S$ is a nonempty set of integers that is bounded
                  below. Show that $S$ has a minimum. In particular, conclude
                  that every nonempty set of natural numbers has a minimum.  

      \textbf{Proof:}

      Let $S$ be a nonempty set of integers bounded below. Then there exists
      some $r \in \R$ such that for every $a \in S$, we have that $r \le a$.
      Consider the set $S' = \{-s: s \in S\}$, the set of the additive inverses
      of the elements of $S$. Note that $S'$ is also a nonempty set of integers.
      So let $-d \in S'$ where $d \in S$. Hence $r \le d$, so that $-d \le -r$;
      that is $S'$ is bounded above. By Proposition 1.7 $S'$ has a maximum, say
      $-b$, where $b \in S$. It suffices to show that $b$ is the minimum in $S$.
      Let $c \in S$. Then we have that $-c \le -b$, so that $b \le c$; that is,
      $b$ is the minimum element in $S$. In paritcular, we can see that the
      Well Ordering Principle follows. \qed
%%%%%%%%%%%%%%%%%%%%%%%%%%%%%%%%%%Prob1.2_3%%%%%%%%%%%%%%%%%%%%%%%%%%%%%%%%%%%%%
   \item[1.2.3]   Let $S$ be a nonempty set of real numbers that is bounded
                  below. Prove that the set $S$ has a minimum if and only if the
                  number $\inf S$ belongs to $S$.
			
		\textbf{Proof:} Let $S$ be a nonempty set of real numbers that is bounded
      below.

      $(\Leftarrow)$ Suppose $\inf S$ belongs in $S$; then it immediately
      follows by definition that $\inf S$ is the minimum element of $S$. \\
      $(\Rightarrow)$ Now suppose that $S$ has a minimum, say $s$. By the
      Completeness Axiom, we have that $\inf S$ exists; since $s \in S$, we must
      have that $\inf S \le s$. But $s$ is also a lower bound for $S$ and since
      every lower bound of $S$ cannot exceed $\inf S$, we must have that
      $s \le \sup S$; we have shown that $\inf S \le s$ and $s \le \inf S$ so
      that $s = \inf S$. \qed
%%%%%%%%%%%%%%%%%%%%%%%%%%%%%%%%%%Prob1.2_4%%%%%%%%%%%%%%%%%%%%%%%%%%%%%%%%%%%%%
   \item[1.2.4]   For each of the following two sets, find the maximum, minimum,
                  infimum, and supremum if they are defined. Justify your
                  conclusions.
                  \begin{enumerate}
                     \item $S = \{1/n : n \in \N\}$.
                     \item $T = \{x \in \R : x^2 < 2\}$.
                  \end{enumerate}

      \textbf{Solution:}

      \begin{enumerate}
         \item The \textbf{maximum} is 1. To show this consider any natural
               number $n$; then we have $n \ge 1$. Multiply this inequality by
               the positive number $1/n$ to give us $1/n \le 1$. Since
               $1 = 1/1 \in S$, we are done. $S$ has no \textbf{minimum}. Assume
               by way of contradiction that $\min S$ exists. Then by definition
               of $S$, we know that $\min S$ must be positive. So by the
               Archimedean Property, there exists a natural number $n_1$(so that
               $1/n_1 \in S$) such that $1/n_1 < \min S$, a contradiction. So
               $\min S$ doesn't exist. Since the Archimedean Property enables us
               to find a member of $S$ that is less than any positive number, no
               positive number can be a lower bound for $S$. Thus $S$ can only 
               be bounded below by negative numbers and 0. It follows that the
               \textbf{infimum} of  $S$ is 0. Since $S$ has a maximum, this
               maximum. By Problem 1.1.15, we have that the \textbf{supremum} of
               $S = 1$. If we consider $-T$, the set of the additive inverses of
               the elements of $T$.
         \item It is trivial to show that%%%%%%%%%%%%%%%%%%%%%%%%%%%%%%%%%%%%%%%%%%%%%%%%%%%%Show true
               $T = \{x \in \R: -\sqrt{2} < x < \sqrt{2}\}$. We claim that the
               \textbf{infimum} and \textbf{supremum} of $T$ are $-\sqrt{2}$ and
               $\sqrt{2}$. Suppose by contradiction that this is false; then 
               there exist $a > -\sqrt{2}$ and $b < \sqrt{2}$ such that $a$ and 
               $b$ are the infimum and supremum of $T$. Then by the density of
               $\Q$ in $\R$, there exist rationals $p$ and $q$ such that
               $-\sqrt{2} < p < a$ and $b < q < \sqrt{2}$; that is, $p$ and $q$ 
               are members of $T$. But since $p$ is less than $a$ and $q > b$, 
               we have contradictions. Thus our claim holds. Since the infimum 
               and supremum are not members of $T$, it follows that $T$ has 
               neither a \textbf{maximum} nor a \textbf{minimum}.
      \end{enumerate}
%%%%%%%%%%%%%%%%%%%%%%%%%%%%%%%%%%Prob1.2_5%%%%%%%%%%%%%%%%%%%%%%%%%%%%%%%%%%%%%
   \item[1.2.5]   Suppose that the number $a$ has the property that for every
                  natural number $n$, $a \le 1/n$. Prove that $a \le 0$.

      \textbf{Proof:} Assume by way of contradiction that $a > 0$. By The
      Archimedean Property there exists a natural number $k$ such that
      $a > 1/k$, a contradiction. Thus $a \le 0$. \qed

%%%%%%%%%%%%%%%%%%%%%%%%%%%%%%%%%%Prob1.2_6%%%%%%%%%%%%%%%%%%%%%%%%%%%%%%%%%%%%%
   \item[1.2.6]   Given a real number $a$, define
                  $S \equiv \{x : x \in \Q, x < a\}$. Prove that $a = \sup S$.

      \textbf{Proof:} By the density of $\Q$ in $\R$, there exists a rational
      $q \in (a - 1, a)$, so that $q \in S$. Thus $S$ is nonempty. By 
      definition, $S$ is bounded above by $a$; since $S$ is also nonempty, the
      Completeness Axiom says that $\sup S$ exists. So we must have that
      $\sup S \le a$. Now suppose that $\sup S < a$, then the density of $\Q$ in
      $\R$ guarantees that we have a rational $q$ in $(\sup S, a)$, so that $q$
      is also a member of $S$, a contradiction since we cannot have a member of
      $S$ that is greater than $\sup S$. Thus $a = \sup S$. \qed

%%%%%%%%%%%%%%%%%%%%%%%%%%%%%%%%%%Prob1.2_7%%%%%%%%%%%%%%%%%%%%%%%%%%%%%%%%%%%%%
   \item[1.2.7]   Show that for any real number $c$, there is exactly one 
                  integer in the interval $(c, c+1]$.

      \textbf{Proof:} Let $c$ be a real number. According to Theorem 1.8, there
      exists a unique integer $k$ in the interval $[-(c + 1), -c)$. So we have
      $-(c + 1) \le k < -c$, so that $c < -k \le c + 1$. Hence we have an
      integer $-k$ in the interval $(c, c + 1]$. We can see that $-k$ is unique
      because if another integer $h$ exists in $(c, c + 1]$, then $-h$ would
      also be in $[-(c + 1), -c)$, and since $k$ is unique, we must have
      $-h = k$, so that $h = -k$. \qed
   
%%%%%%%%%%%%%%%%%%%%%%%%%%%%%%%%%%Prob1.2_8%%%%%%%%%%%%%%%%%%%%%%%%%%%%%%%%%%%%%
   \item[1.2.8]   Show that the Archimedean Property is a consequence of the
                  assertion that for any real number $c$, there is an integer in
                  the interval $[c, c + 1)$.

      \textbf{Proof:} Let $\epsilon$ be a positive real number. It suffices to
      show that there exists a natural number greater than $\epsilon$. By our
      assertion, there exists an integer $k$ in the interval
      $[\epsilon+ 1, \epsilon + 2)$. So we have $\epsilon < \epsilon + 1 \le k$,
      so that $k$ is a positive integer. \qed
%%%%%%%%%%%%%%%%%%%%%%%%%%%%%%%%%%Prob1.2_9%%%%%%%%%%%%%%%%%%%%%%%%%%%%%%%%%%%%%
   \item[1.2.9]   Show that the Archimedean Property is a consequence of the
                  assertion that every interval $(a, b)$ contains a rational
                  number.

      \textbf{Proof:} Let $\epsilon$ be a positive real number. It suffices to
      show that there exists a natural number greater than $\epsilon$. By our
      assertion, there exist positive integers $p$ and $q$ such that
      $p/q \in (\epsilon, \epsilon + 1)$. Since $p$ and $q$ are positive, we 
      have that $q \ge 1$ so that $pq \ge p$; that is $p/q \le p$. We have now
      shown that $\varepsilon < p/q \le p$. Particularly $p > \varepsilon$,
      which is what we wanted to prove. \qed

      
\end{enumerate}

      \section{Symmetric Groups}
         \begin{enumerate}
%%%%%%%%%%%%%%%%%%%%%%%%%%%%%%%%%%%%%2.3.1%%%%%%%%%%%%%%%%%%%%%%%%%%%%%%%%%%%%%%
   \item[2.3.1]   Find all subgroups of $Z_{45} = \cyc{x}$, giving a generator
                  for each. Describe the containments between these subgroups.
                  
      \textbf{Solution.} Since the positive divisors of 45 are: 1, 3, 5, 9, 15,
      and 45, it follows that the subgroups of $Z_{45}$ are
      $$\cyc{x}, \cyc{x^3}, \cyc{x^5}, \cyc{x^9}, \cyc{x^{15}}, \text{ and }
        \cyc{x^{45}}.$$
        
      We have the following containments:
      $$
         \begin{tabular}{>{$}c<{$}>{$}c<{$}>{$}c<{$}>{$}c<{$}>{$}c<{$}>{$}c<{$}>{$}c<{$}}
            \cyc{x^{45}} & \le & \cyc{x^{15}} & \le & \cyc{x^5} & \le & \cyc{x} \\
            \cyc{x^{15}} & \le &  \cyc{x^3} & \le & \cyc{x} \\
            \cyc{x^9} & \le &  \cyc{x^3} & \le & \cyc{x}
         \end{tabular}
      $$
%%%%%%%%%%%%%%%%%%%%%%%%%%%%%%%%%%%%%2.3.2%%%%%%%%%%%%%%%%%%%%%%%%%%%%%%%%%%%%%%
   \item[2.3.2]   If $x$ is an element of the finite group $G$ and $|x| = |G|$,
                  prove that $G = \cyc{x}$. Give an explicit example to show 
                  that this result need not be true if $G$ is an infinite group.
                  
      \textbf{Proof.} Let $G$ be a finite group, so that $|G| = n \in \Z^+$.
      Suppose that there exists $x \in G$ such that $|x| = n$. Clearly
      $\cyc{x} \subseteq G$. But $|\cyc{x}| = n$ since $|x| = n$; thus
      $G \subseteq \cyc{x}$ so that $G = \cyc{x}$. Now let $G = \Z$. We have
      that $|\cyc{2}| = |G|$ but $G \neq \cyc{2}$. \qed
%%%%%%%%%%%%%%%%%%%%%%%%%%%%%%%%%%%%%2.3.3%%%%%%%%%%%%%%%%%%%%%%%%%%%%%%%%%%%%%%
   \item[2.3.3]   Find all generators for $\Z/48\Z$.
   
      \textbf{Solution.} The generators for $\Z/48\Z$ are: $\cyc{\overline{1}}$,
      $\cyc{\overline{5}}$, $\cyc{\overline{7}}$, $\cyc{\overline{11}}$,
      $\cyc{\overline{13}}$, $\cyc{\overline{17}}$, $\cyc{\overline{19}}$,
      $\cyc{\overline{23}}$, $\cyc{\overline{25}}$, $\cyc{\overline{29}}$,
      $\cyc{\overline{31}}$, $\cyc{\overline{35}}$, $\cyc{\overline{37}}$,
      $\cyc{\overline{41}}$, $\cyc{\overline{43}}$, and $\cyc{\overline{47}}$.
%%%%%%%%%%%%%%%%%%%%%%%%%%%%%%%%%%%%%2.3.4%%%%%%%%%%%%%%%%%%%%%%%%%%%%%%%%%%%%%%
   \item[2.3.4]   Find all generators for $\Z/202\Z$.
   
      \textbf{Solution.} Let $S$ be the set of generators for $\Z/202\Z$. Then
      $|S| = 100$ since
      $$S = \{\cyc{x} : x \text{ is odd and positive}, x \neq 101, \text{ and } x < 202\}.$$
%%%%%%%%%%%%%%%%%%%%%%%%%%%%%%%%%%%%%2.3.5%%%%%%%%%%%%%%%%%%%%%%%%%%%%%%%%%%%%%%
   \item[2.3.5]   Find the number of generators for $\Z/49000\Z$.
   
      \textbf{Solution.} For a positive integer $n$ let $\varphi(n)$ be the
      number of positive integers---less than or equal to $n$---that are
      relatively prime to $n$. Then the number of generators for $\Z/49000\Z$ is
      $\varphi(49000) = \varphi(2^35^37^2) =
      \varphi(2^3)\varphi(5^3)\varphi(7^2) = 16800$. 
%%%%%%%%%%%%%%%%%%%%%%%%%%%%%%%%%%%%%2.3.6%%%%%%%%%%%%%%%%%%%%%%%%%%%%%%%%%%%%%%
   \item[2.3.6]   In $\Z/48\Z$ write out all elements of $\cyc{\overline{a}}$
                  for every $\overline{a}$. Find all inclusions between
                  subgroups in $\Z/48\Z$.
      
      \textbf{Solution.}
      $$
         \begin{tabular}{|c|c|} \hline
            \textbf{Generators} & \textbf{Subgroups in} $\Z/48\Z$ \\ \hline
            0 & $\{0\}$ \\ \hline
            24 & $\{0, 24\}$ \\ \hline
            16, 32 & $\{0, 16, 32\}$ \\ \hline
            12, 36 & $\{0, 12, 24, 36\}$ \\ \hline
            8, 40 & $\{0, 8, 16, 24, 32, 40\}$ \\ \hline
            6, 18, 30, 42 & $\{0, 6, 12, 18, 24, 30, 36, 42\}$ \\ \hline
            4,20,28,44 & $\{0,4,8,12,16, 20, 24, 28, 32, 36, 40, 44\}$ \\ \hline
            3, 9, 15, 21, 27, 33, 39, 45 & $\{0, 3, 6, 9, 12, 15, 18, 21, 24,
            27, 30, 33, 36, 39, 42, 45\}$ \\ \hline            
            2, 10, 14, 22, 26, 34, 38, 46 & $\{x : 0 \le x \le 46,
            x \text{ is even}\}$ \\ \hline
            \text{See Exercise } 2.3.3 & $\Z/48\Z$ \\ \hline
         \end{tabular}
      $$
%%%%%%%%%%%%%%%%%%%%%%%%%%%%%%%%%%%%%2.3.7%%%%%%%%%%%%%%%%%%%%%%%%%%%%%%%%%%%%%%
   \item[2.3.7]   Let $Z_{48} = \cyc{x}$ and use the isomorphism
                  $\Z/48\Z \cong Z_{48}$ given by $\overline{1} \mapsto x$ to
                  list all subgroups of $Z_{48}$ as computed in the preceding
                  exercise.
                  
      \textbf{Solution.}
      $$
         \begin{tabular}{|c|} \hline
            \textbf{Subgroups in} $Z_{48}$ \\ \hline
            $\{1\}$ \\ \hline
            $\{1, x^{24}\}$ \\ \hline
            $\{1, x^{16}, x^{32}\}$ \\ \hline
            $\{1, x^{12}, x^{24}, x^{36}\}$ \\ \hline
            $\{1, x^8, x^{16}, x^{24}, x^{32}, x^{40}\}$ \\ \hline
            $\{1, x^6, x^{12}, x^{18}, x^{24}, x^{30},x^{36},x^{42}\}$ \\ \hline
            $\{1,x^4,x^8,x^{12},x^{16}, x^{20}, x^{24}, x^{28}, x^{32}, x^{36},
               x^{40}, x^{44}\}$ \\ \hline
            $\{1, x^3, x^6, x^9, x^{12}, x^{15}, x^{18}, x^{21}, x^{24},
            x^{27}, x^{30}, x^{33}, x^{36}, x^{39}, x^{42}, x^{45}\}$ \\ \hline
            $\{x^y : 0 \le y \le 46, y \text{ is even}\}$ \\ \hline
            $Z_{48}$ \\ \hline
         \end{tabular}
      $$
%%%%%%%%%%%%%%%%%%%%%%%%%%%%%%%%%%%%%2.3.8%%%%%%%%%%%%%%%%%%%%%%%%%%%%%%%%%%%%%%
   \item[2.3.8]   Let $Z_{48} = \cyc{x}$. For which integers $a$ does the map
                  $\varphi_a$ defined by $\varphi_a : \overline{1} \mapsto x^a$
                  extend to an \textit{isomorphism} from $\Z/48\Z$ onto
                  $Z_{48}$.
                  
      \textbf{Solution.} Suppose that $(a, 48) = 1$. Then it follows that $x^a$
      generates $Z_{48}$. Thus $\varphi_a$ is an isomorphism by Theorem 4 (Page
      56). Now suppose that $a$ is not relatively prime to 48. Then $x^a$ does
      not generate $Z_{48}$, so that the image of $\varphi_a$ is not $Z_{48}$.
      Hence $\varphi_a$ is an isomorphism if and only if $(a, 48) = 1$.
%%%%%%%%%%%%%%%%%%%%%%%%%%%%%%%%%%%%%2.3.9%%%%%%%%%%%%%%%%%%%%%%%%%%%%%%%%%%%%%%
   \item[2.3.9]   Let $Z_{36} = \cyc{x}$. For which integers $a$ does the map
                  $\psi_a$ defined by $\psi_a : \overline{1} \mapsto x^a$ extend
                  to a \textit{well defined homomorphism} from $\Z/48\Z$ into
                  $Z_{36}$. Can $\psi_a$ ever be a surjective homomorphism?
                  
      \textbf{Solution.} First we shall find the restriction(s) on $a$ such that
      $\psi_a$ is well defined. Suppose $b = c$ for some $b, c \in \Z/48\Z$. It
      suffices to show that $\psi_a(b) = \psi_a(c)$. Since $b = c$, there exists
      an integer $k$ such that $b = c + 48k$. Thus $\psi_a(b) = \psi_a(c+48k)$,
      so that
      $\psi_a(b)=(x^a)^{c+48k}=x^{ac + 48ak}= x^{ac}x^{48ak}=\psi_a(c)x^{12ak}$.
      So we must require $x^{12ak} = 1$ for all $k \in \Z$. Now $x^{12ak} = 1$
      for all $k \in \Z$ if and only if $3 \mid a$ if and only if $\psi_a$ is
      well defined. It follows immediately that
      $\psi_a$ is an homomorphism since
      \begin{align*}
         \psi_a(p + q) &= (x^a)^{p+q} \\
            &= x^{ap+aq} \\
            &= x^{ap}x^{aq} \\
            &= (x^a)^p(x^a)^q \\
            &= \psi_a(p)\psi_a(q)
      \end{align*}      
      for all $p, q \in \Z/48\Z$.
      
      \textit{Can $\psi_a$ ever be a surjective homomorphism?} No!
      
      \textbf{Proof.} Suppose to the contrary that $\psi_a$ is surjective. Then
      there exists $y \in \Z/48\Z$ such that $\psi_a(y) = x$. That is
      $x^{ay} = x$, so that $x^{ay-1} = 1$; thus $ay - 1 = 36m$ for some integer
      $m$. Rearrange the equality $ay - 1 = 36m$ to get $1 = ay - 36m$. Recall
      that $3 \mid a$; since $3$ also divides 36, it follows that 3 must divide
      1, a contradiction. Thus $\psi_a$ can never be surjective. \qed
%%%%%%%%%%%%%%%%%%%%%%%%%%%%%%%%%%%%%2.3.10%%%%%%%%%%%%%%%%%%%%%%%%%%%%%%%%%%%%%
   \item[2.3.10]  What is the order of $\overline{30}$ in $\Z/54\Z$? Write out
                  all the elements and their orders in $\cyc{\overline{30}}$.
                  
      \textbf{Solution.} The order of $30$ in $\Z/54\Z$ is
      $$\frac{54}{(30, 54)} = 9.$$
      The elements of $\cyc{30}$ and their respective orders are:
      $$
         \begin{tabular}{|c|c|} \hline
            Element of $\cyc{30}$ & Order \\ \hline
            30 & 9 \\ \hline
             6 & 9 \\ \hline
            36 & 3 \\ \hline
            12 & 9 \\ \hline
            42 & 9 \\ \hline
            18 & 3 \\ \hline
            48 & 9 \\ \hline
            24 & 9 \\ \hline
             0 & 1 \\ \hline
         \end{tabular}
      $$
%%%%%%%%%%%%%%%%%%%%%%%%%%%%%%%%%%%%%2.3.11%%%%%%%%%%%%%%%%%%%%%%%%%%%%%%%%%%%%%
   \item[2.3.11]  Find all cyclic subgroups of $D_8$. Find a proper subgroup of
                  $D_8$ which is not cyclic.
                  
      \textbf{Solution.} In $D_8$, only $r$ and $r^4$ have order 4. Thus
      $\{1, r, r^2, r^3\}$ is the only cyclic subgroup of order 4. The trivial
      subgroup is the only cyclic subgroup of order 1. Finally there are 5
      cyclic subgroups of order 2 and they are of the form $\{1, x\}$ where
      $x \in \{r^2, s, sr, sr^2, sr^3\}$. The set $\{1, s, r^2, sr^2\}$ is a
      non-cyclic proper subgroup of $D_8$.
%%%%%%%%%%%%%%%%%%%%%%%%%%%%%%%%%%%%%2.3.12%%%%%%%%%%%%%%%%%%%%%%%%%%%%%%%%%%%%%
   \item[2.3.12]  Prove that the following groups are \textit{not} cyclic:
                  \begin{enumerate}
                     \item $Z_2 \times Z_2$
                     \item $Z_2 \times \Z$
                     \item $\Z \times \Z$.
                  \end{enumerate}
      
      \textbf{Proof.}
      \begin{enumerate}
         \item The order of $Z_2 \times Z_2$ is 4, but no element in this group
               has order 4; thus $Z_2 \times Z_2$ is not cyclic.
         \item Let $Z_2 = \cyc{x}$. Observe that $Z_2 \times \Z$ is not finite,
               so in order for it to be cyclic it must be isomorphic to $\Z$.
               But this is not the case since $Z_2 \times \Z$ has two elements
               of finite order(namely $(1, 0)$ and $(x, 0)$) while $\Z$ has
               exactly 1 element of finite order.
         \item Suppose to the contrary that $\Z \times \Z$ is cyclic. Then there
               exist nonzero integers $a$ and $b$ such that
               $$\Z \times \Z = \cyc{(a,b)} = \{(na, nb) : n \in \Z\}.$$
               Thus there exists an integer $m$ such that
               $(ma, mb) = (0, 1)$. That is, $ma = 0$ and $mb = 1$. Since
               $ma = 0$, we must have $m = 0$ or $a = 0$. If $m$ is 0, then
               $(ma, mb) = (0, 0) \neq (0, 1)$, a contradiction; thus we must
               have $a = 0$, contradicting our assumption that $a$ is nonzero.
               Thus $\Z \times \Z$ is not cyclic.
      \end{enumerate} \qed
%%%%%%%%%%%%%%%%%%%%%%%%%%%%%%%%%%%%%2.3.13%%%%%%%%%%%%%%%%%%%%%%%%%%%%%%%%%%%%%
   \item[2.3.13]  Prove that the following pairs of groups are \textit{not}
                  isomorphic:
                  \begin{enumerate}
                     \item $\Z \times Z_2$ and $\Z$
                     \item $\Q \times Z_2$ and $\Q$.
                  \end{enumerate}
      
      \textbf{Proof.}
      \begin{enumerate}
         \item By Exercise 1.6.11, we know that $\Z \times Z_2$ is isomorphic to
               $Z_2 \times \Z$. By Exercise 2.3.12, $Z_2 \times \Z$ is not
               cyclic; thus $\Z \times Z_2$ is not cyclic. That is,
               $\Z \times Z_2$ is not isomorphic to $\Z$.
         \item Let $Z_2 = \cyc{x}$. It immediately follows that
               $\Q \times Z_2$ and $\Q$ are not isomorphic since $\Q \times Z_2$
               has two elements of finite order(namely $(0, 1)$ and $(0, x)$)
               while $\Q$ has exactly 1 element of finite order.
      \end{enumerate} \qed
%%%%%%%%%%%%%%%%%%%%%%%%%%%%%%%%%%%%%2.3.14%%%%%%%%%%%%%%%%%%%%%%%%%%%%%%%%%%%%%
   \item[2.3.14]  Let $\sigma =$ (1 2 3 4 5 6 7 8 9 10 11 12). For each of the
                  following integers $a$ compute $\sigma^a$:
                  $$a = 13, 65, 626, 1195, -6, -81, -570,\text{ and } {-1211}.$$
                  
      \textbf{Solution.}
      
      \begin{alignat*}{4}
         &\sigma^{13}   &&= \sigma &&\text{ } \\
         &\sigma^{65}   &&= \sigma^5 &&=
            (1\;6\;11\;4\;9\;2\;7\;12\;5\;10\;3\;8) \\
         &\sigma^{626}  &&= \sigma^2 &&= (1\;3\;5\;7\;9\;11) \\
         &\sigma^{1195} &&= \sigma^7 &&=
            (1\;8\;3\;10\;5\;12\;7\;2\;9\;4\;11\;6\;13) \\
         &\sigma^{-6} &&= \sigma^6 &&= (1\;7)
            (1\;8\;3\;10\;5\;12\;7\;2\;9\;4\;11\;6\;13) \\
         &\sigma^{-81} &&= \sigma^3 &&= (1\;4\;7\;10) \\
         &\sigma^{-570} &&= \sigma^6 &&= (1\;7) \\
         &\sigma^{-1211} &&= \sigma
      \end{alignat*}
%%%%%%%%%%%%%%%%%%%%%%%%%%%%%%%%%%%%%2.3.15%%%%%%%%%%%%%%%%%%%%%%%%%%%%%%%%%%%%%
   \item[2.3.15]  Prove that $\Q \times \Q$ is not cyclic.
   
      \textbf{Proof.} Since $\Q$ is infinite and, by Exercise 1.6.6, $\Q$ is not
      isomorphic to $\Z$, it follows that $\Q$ is not cyclic. We know that the
      subgroup of every cyclic group is cyclic; since $\Q \times\{1\} \cong \Q$,
      it follows that $\Q \times \{1\}$ is not cyclic; thus $\Q \times \Q$ is
      not cyclic because it has a noncyclic subgroup, namely $\Q \times \{1\}$.
      \qed
%%%%%%%%%%%%%%%%%%%%%%%%%%%%%%%%%%%%%2.3.16%%%%%%%%%%%%%%%%%%%%%%%%%%%%%%%%%%%%%
   \item[2.3.16]  Assume $|x| = n$ and $|y| = m$. Suppose that $x$ and $y$
                  \textit{commute}: $xy = yx$. Prove that $|xy|$ divides the
                  least common multiple of $m$ and $n$. Need this be true if $x$
                  and $y$ do \textit{not} commute? Give an example of commuting
                  elements $x$, $y$ such that the order of $xy$ is not equal to
                  the least common multiple of $|x|$ and $|y|$.
                  
      \textbf{Proof.} Let $l = \text{lcm}(m, n)$. So there exist integers
      $m'$ and $n'$ such that $mm' = nn' = l$. So we have that
      $$(xy)^l = x^ly^l = x^{nn'}y^{mm'} = (x^n)^{n'}(y^m)^{m'} = 1.$$
      That is $|xy|$ divides $l$ (by Proposition 3, Page 55).
      
      \textit{Need this be true if $x$ and $y$ do not commute?} No! Let
      $$
         A = \left(\begin{tabular}{@{}cc@{}}
            0 & 1/2 \\
            2 & 0
         \end{tabular}\right) \text{ and }
         B = \left(\begin{tabular}{@{}cc@{}}
            0 & 1 \\
            1 & 0
         \end{tabular}\right).
      $$
      A simple computation will show us that although $|A| = |B| = 2$, we have
      that $|AB| = \infty$.
      
      \textbf{Example.} Consider $\Z/2\Z = \{0, 1\}$. Let $x = y = 1$. Then we
      have $|x| = |y| = 2$, so that lcm($|x|, |y|) = 2 \neq |x + y| = |0| = 1$.
      \qed
%%%%%%%%%%%%%%%%%%%%%%%%%%%%%%%%%%%%%2.3.17%%%%%%%%%%%%%%%%%%%%%%%%%%%%%%%%%%%%%
   \item[2.3.17]  Find a presentation for $Z_n$ with one generator.
   
      \textbf{Solution.} $Z_n = \cyc{x : x^n = 1}$.
%%%%%%%%%%%%%%%%%%%%%%%%%%%%%%%%%%%%%2.3.18%%%%%%%%%%%%%%%%%%%%%%%%%%%%%%%%%%%%%
   \item[2.3.18]  Show that if $H$ is any group and $h$ is an element of $H$
                  with $h^n = 1$, then there is a unique homomorphism from
                  $Z_n = \cyc{x}$ to $H$ such that $x \mapsto h$.
                  
      \textbf{Proof.} Let $n \in \Z^+$, $Z_n = \cyc{x}$, $H$ a group, and
      $h^n  = 1$ for some $h \in H$. First we shall show the existence of a
      homomorphism from $Z_n$ to $H$ such that $x \mapsto h$. So consider the
      map $\alpha : \cyc{x} \rightarrow H$ defined by $\alpha(x^a) = h^a$.
      Clearly $\alpha(x) = h$. Now we will show that $\alpha$ is well defined.
      Suppose $x^w = x^y$ for some $x^w, x^y \in Z_n$. Thus $w = y + nk$ for
      some integer $k$. Thus
      $$\alpha(x^w) = \alpha(x^{y+nk})=h^{y+nk}=h^{y}{h^n}^k =h^y=\alpha(x^y),$$
      so that $\alpha$ is well defined. Now we have that
      $$\alpha(x^px^q)=\alpha(x^{p+q})=h^{p+q}=h^ph^q=\alpha(x^p)\alpha(x^q),$$
      so that $\alpha$ is an homomorphism. Now to show uniqueness, we suppose
      that $\phi : \cyc{x} \rightarrow H$ is an homommorphism such that
      $\phi(x) = h$. Since $\phi$ is a homomorphism, it follows that
      $\phi(x^a) = h^a$. Thus $\phi = \alpha$, as desired. \qed
%%%%%%%%%%%%%%%%%%%%%%%%%%%%%%%%%%%%%2.3.19%%%%%%%%%%%%%%%%%%%%%%%%%%%%%%%%%%%%%
   \item[2.3.19]  Show that if $H$ is any group and $h$ is an element of $H$,
                  then there is a unique homomorphism from $\Z$ to $H$ such that
                  $1 \mapsto h$.
                  
      \textbf{Proof.} Let $H$ be a group and let $h \in H$. First we shall show
      that there exists a homomorphism from $\Z$ to $H$ such that $1 \mapsto h$.
      So consider the map $\alpha : \Z \rightarrow H$ defined by
      $n \mapsto h^n$. Clearly $\alpha(1) = h$ and
      $$\alpha(x+y) = h^{x+y} = h^xh^y = \alpha(x)\alpha(y) \text{ for all }
        x, y \in \Z^+,$$
      so that $\alpha$ is a homomorphism. To show uniqueness, suppose that
      $\alpha' : \Z \rightarrow H$ is an homomorphism such that
      $\alpha'(1) = h$. Then according to Exercise 1.6.1, we have that
      $\alpha'(n) = \alpha'(n\cdot1) = \alpha'(1)^n = h^n$ for all $n \in \Z$;
      that is, $\alpha' = \alpha$, as desired. \qed
%%%%%%%%%%%%%%%%%%%%%%%%%%%%%%%%%%%%%2.3.20%%%%%%%%%%%%%%%%%%%%%%%%%%%%%%%%%%%%%
   \item[2.3.20]  Let $p$ be a prime and let $n$ be a positive integer. Show
                  that if $x$ is an element of the group $G$ such that
                  $x^{p^n} = 1$ then $|x| = p^m$ for some $m \le n$.
                  
      \textbf{Proof.} Suppose that $x \in G$ such that $x^{p^n} = 1$. Then it
      follows by Proposition 3 (Page 55) that $|x|$ divides $p^n$. Since $p$ is
      a prime, its factors are $p^i$, $0 \le i \le n$. Thus $|x| = p^m$ for
      some nonnegative $m$ not greater than $n$. \qed
%%%%%%%%%%%%%%%%%%%%%%%%%%%%%%%%%%%%%2.3.21%%%%%%%%%%%%%%%%%%%%%%%%%%%%%%%%%%%%%
   \item[2.3.21]  Let $p$ be an odd prime and let $n$ be a positive integer
                  $\ge 2$. Use the Binomial Theorem to show that
                  $(1+p)^{p^{n-1}} \equiv 1$ (mod $p^n$) but
                  $(1+p)^{p^{n-2}} \not\equiv 1$ (mod $p^n$). Deduce that $1+p$
                  is an element of order $p^{n-1}$ in the multiplicative group
                  $(\Z/p^n\Z)^\times$.

      \textbf{Lemma 2.3.1.} \textit{For an integer $n \ge 2$ and an odd prime
      $p$, let $f_p(n)$ be the number of $p$ factors of $n!$ (i.e., the greatest
      nonnegative integer $j$ such that $p^j \mid i!$), then it follows that
      $f_p(n) < \D\frac{n}{2}$}.

      \textbf{Proof.} Let $n \ge 2$ be an integer and $p$ an odd prime. For a
      a positive integer $r$, let $g_p(n, r)$ be the number of positive
      integers, less than or equal to $n$, that have at least $r$ number of $p$ 
      factors. It follows that $g_p(n, r) = \D\gint{\frac{n}{p^r}}$, where
      $\gint{x}$ is the greatest integer less than or equal to $x$. Finally let
      $k_n$ be the maximum nonnegative integer such that $p^{k_n}$ is a multiple
      of some positive integer not greater than $n$. Thus we have that
      \begin{align*}
         f_p(n) &= g_p(n, 1) + g_p(n, 2) + \cdots + g_p(n, k_n) \\
            &= \sum_{i=1}^{k_n} g_p(n, i)
            = \sum_{i=1}^{k_n} \gint{\frac{n}{p^i}} \\
            &\le \sum_{i=1}^{k_n} \frac{n}{p^i}
            < \sum_{i=1}^\infty \frac{n}{p^i} \\
            &= \frac{n}{p-1} &[\text{Sum of Geometric Series}] \\
            &< \frac{n}{2}. &[\text{Since }p \ge 3]
      \end{align*}

      So we can write $n! = p^{f_p(n)} h_n$ for some $h_n \in \Z^+$, so that
      $(h_n, p) = 1$.

      Now we are ready to commence the proof of the problem. By the Binomial
      Theorem, we have that
      \begin{align*}
         (1+p)^{p^{n-1}} &= \sum_{i=0}^{p^{n-1}}\binom{p^{n-1}}{i}p^i \\
            &= \sum_{i=0}^{p^{n-1}}p^i\frac{p^{n-1}(p^{n-1}-1)(p^{n-1}-2)
               \cdots(p^{n-1}-i+1)}{i!} \\
            &= \sum_{i=0}^{p^{n-1}}p^i\frac{p^{n-1}(p^{n-1}-1)(p^{n-1}-2)
               \cdots(p^{n-1}-i+1)}{p^{f_p(i)} h_i} \\
            &= 1 + p^n + p^n\sum_{i=2}^{p^{n-1}}\frac{p^{i-1}(p^{n-1}-1)
               (p^{n-1}-2) \cdots(p^{n-1}-i+1)}{p^{f_p(i)} h_i}.
      \end{align*}
      Now $f_p(i) < i / 2 \le i - 1$ for $i \ge 2$. Thus $i - 1 - f_p(i) \ge 0$
      (so that $p^{i - 1 - f_p(i)}$ is an integer) if $i \ge 2$. We then have
      \begin{equation} \label{2_3_21_1}
         (1+p)^{p^{n-1}} = 1 + p^n + p^n\sum_{i=2}^{p^{n-1}}\frac{p^{i-1-f_p(i)}
        (p^{n-1}-1)(p^{n-1}-2) \cdots(p^{n-1}-i+1)}{h_i}
      \end{equation}
      Since $(h_i, p) = 1$, it follows that $h_i$ must divide
      $p^{i-1}(p^{n-1}-1)(p^{n-1}-2) \cdots(p^{n-1}-i+1)$. Hence
      $$\sum_{i=2}^{p^{n-1}}\frac{p^{i-1-f_p(i)}
        (p^{n-1}-1)(p^{n-1}-2) \cdots(p^{n-1}-i+1)}{h_i}$$
      is an integer and we can conclude from \eqref{2_3_21_1} that
      $(1+p)^{p^{n-1}} \equiv 1$ (mod $p^n$). Now we have that
      \begin{align*}
         (1+p)^{p^{n-2}} &= \sum_{i=0}^{p^{n-2}}\binom{p^{n-2}}{i}p^i \\
            &= \sum_{i=0}^{p^{n-2}}p^i\frac{p^{n-2}(p^{n-2}-1)(p^{n-2}-2)
               \cdots(p^{n-2}-i+1)}{i!} \\
            &= 1 + p^{n-1} + p^n\frac{p^{n-2}-1}{2} + p^n\frac{p(p^{n-2}-1)(p^{n-2}-2)}{3!} +\sum_{i=4}^{p^{n-1}}p^i\frac{p^{n-2}(p^{n-2}-1)(p^{n-2}-2)
               \cdots(p^{n-2}-i+1)}{p^{f_p(i)} h_i} \\
            &= 1 + p^n + p^n\sum_{i=2}^{p^{n-1}}\frac{p^{i-1}(p^{n-1}-1)
               (p^{n-1}-2) \cdots(p^{n-1}-i+1)}{p^{f_p(i)} h_i}.
      \end{align*}
      
%%%%%%%%%%%%%%%%%%%%%%%%%%%%%%%%%%%%%2.3.22%%%%%%%%%%%%%%%%%%%%%%%%%%%%%%%%%%%%%
   \item[2.3.22]  Let $n$ be an integer $\ge 3$. Use the Binomial Theorem to
                  show that $(1+2^2)^{2^{n-2}} \equiv 1$ (mod $2^n$) but
                  $(1+2^2)^{2^{n-3}} \not\equiv 1$ (mod $2^n$). Deduce that 5 is
                  an element of order $2^{n-2}$ in the multiplicative group
                  $(\Z/2^n\Z)^\times$.

      \textbf{Proof.}
%%%%%%%%%%%%%%%%%%%%%%%%%%%%%%%%%%%%%2.3.23%%%%%%%%%%%%%%%%%%%%%%%%%%%%%%%%%%%%%
   \item[2.3.23]  Show that $(\Z/2^n\Z)^\times$ is not cyclic for any $n \ge 3$.
                  [Find two distinct subgroups of order 2.]
%%%%%%%%%%%%%%%%%%%%%%%%%%%%%%%%%%%%%2.3.24%%%%%%%%%%%%%%%%%%%%%%%%%%%%%%%%%%%%%
   \item[2.3.24]  Let $G$ be a finite group and let $x \in G$.
                  \begin{enumerate}
                     \item Prove that if $g \in N_G(\cyc{x})$ then
                           $gxg^{-1} = x^a$ for some $a \in \Z$. 
                     \item Prove conversely that if $gxg^{-1} = x^a$ for some
                           $a \in \Z$ then $g \in N_G(\cyc{x})$. [Show first
                           that $gx^kg^{-1} = (gxg^{-1})^k = x^{ak}$ for any
                           integer $k$, so that $g\cyc{x}g^{-1} \le \cyc{x}$.
                           If $x$ has order $n$, show the elements $gx^ig^{-1}$,
                           $i = 0, 1, \ldots, n-1$ are distinct, so that
                           $|g\cyc{x}g^{-1}| = |\cyc{x}| = n$ and conclude that
                           $g\cyc{x}g^{-1} = \cyc{x}$.]
                  \end{enumerate}
                  Note that this cuts down some of the work in computing
                  normalizers of cyclic subgroups since one does not have to
                  check $ghg^{-1} \in \cyc{x}$ for every $h \in \cyc{x}$.
%%%%%%%%%%%%%%%%%%%%%%%%%%%%%%%%%%%%%2.3.25%%%%%%%%%%%%%%%%%%%%%%%%%%%%%%%%%%%%%
   \item[2.3.25]  Let $G$ be a cyclic group of order $n$ and let $k$ be an
                  integer relatively prime to $n$. Prove that the map
                  $x \mapsto x^k$ is surjective. Use Lagrange's Theorem
                  (Exercise 1.7.19) to prove the same is true for any finite
                  group of order $n$. (For such $k$ each element has a
                  $k^{\text{th}}$ root in $G$. It follows from Cauchy's Theorem
                  in Section 3.2 that if $k$ is not relatively prime to the
                  order of $G$ then the map $x \mapsto x^k$ is not surjective.)
%%%%%%%%%%%%%%%%%%%%%%%%%%%%%%%%%%%%%2.3.26%%%%%%%%%%%%%%%%%%%%%%%%%%%%%%%%%%%%%
   \item[2.3.26]  Let $Z_n$ be a cyclic group of order $n$ and for each integer
                  $a$ let
                  $$\sigma_a : Z_n \mapsto Z_n \qquad by \qquad \sigma_a(x) =
                  x^a \quad \text{for all } x \in Z_n.$$
                  \begin{enumerate}
                     \item Prove that $\sigma_a$ is an automorphism of $Z_n$ if
                           and only if $a$ and $n$ are relatively prime(
                           automorphisms were introduced in Exercise 1.6.20).
                     \item Prove that $\sigma_a = \sigma_b$ if and only if
                           $a \equiv b$ (mod $n$).
                     \item Prove that \textit{every} automorphism of $Z_n$ is
                           equal to $\sigma_a$ for some integer $a$.
                     \item Prove that $\sigma_a\circ\sigma_b=\sigma_{ab}$.
                           Deduce that the map $\overline{a} \mapsto \sigma_a$
                           is an isomorphism of $(\Z/n\Z)^\times$ onto the
                           automorphism group of $Z_n$ (so Aut($Z_n$) is an
                           abelian group of order $\varphi(n)$).
                  \end{enumerate}
                  %%%%%MISSING CONTAINMENT%%%%%%%%
\end{enumerate}


































      \section{Matrix Groups}
         Let $F$ be a field and let $n \in \Z^+$.
\begin{enumerate}
%%%%%%%%%%%%%%%%%%%%%%%%%%%%%%%%%%%%%1.4.1%%%%%%%%%%%%%%%%%%%%%%%%%%%%%%%%%%%%%%
   \item[1.4.1]   Prove that $|GL_2(\F_2)| = 6$.
%%%%%%%%%%%%%%%%%%%%%%%%%%%%%%%%%%%%%1.4.2%%%%%%%%%%%%%%%%%%%%%%%%%%%%%%%%%%%%%%
   \item[1.4.2]   Write out all the elements of $GL_2(\F_2)$ and compute the
                  order of each element.
%%%%%%%%%%%%%%%%%%%%%%%%%%%%%%%%%%%%%1.4.3%%%%%%%%%%%%%%%%%%%%%%%%%%%%%%%%%%%%%%
   \item[1.4.3]   Show that $GL_2(\F_2)$ is non-abelian.
%%%%%%%%%%%%%%%%%%%%%%%%%%%%%%%%%%%%%1.4.4%%%%%%%%%%%%%%%%%%%%%%%%%%%%%%%%%%%%%%
   \item[1.4.4]   Show that if $n$ is not prime then $\Z/n\Z$ is not a field.
%%%%%%%%%%%%%%%%%%%%%%%%%%%%%%%%%%%%%1.4.5%%%%%%%%%%%%%%%%%%%%%%%%%%%%%%%%%%%%%%
   \item[1.4.5]   Show that $GL_n(F)$ is a finite group if and only if $F$ has a
                  finite number of elements.
%%%%%%%%%%%%%%%%%%%%%%%%%%%%%%%%%%%%%1.4.6%%%%%%%%%%%%%%%%%%%%%%%%%%%%%%%%%%%%%%
   \item[1.4.6]   If $|F| = q$ is finite prove that $|GL_n(F)| < q^{n^2}$.
%%%%%%%%%%%%%%%%%%%%%%%%%%%%%%%%%%%%%1.4.7%%%%%%%%%%%%%%%%%%%%%%%%%%%%%%%%%%%%%%
   \item[1.4.7]   Let $p$ be a prime. Prove that the order of $GL_2(\F_p)$ is
                  $p^4 - p^3 - p^2 + p$.
%%%%%%%%%%%%%%%%%%%%%%%%%%%%%%%%%%%%%1.4.8%%%%%%%%%%%%%%%%%%%%%%%%%%%%%%%%%%%%%%
   \item[1.4.8]   Show that $GL_n(F)$ is non-abelian for any $n \ge 2$ and any
                  $F$.
%%%%%%%%%%%%%%%%%%%%%%%%%%%%%%%%%%%%%1.4.9%%%%%%%%%%%%%%%%%%%%%%%%%%%%%%%%%%%%%%
   \item[1.4.9]   Prove that the binary operation of matrix multiplication of
                  $2 \times 2$ matrices with real number entries is associative.
%%%%%%%%%%%%%%%%%%%%%%%%%%%%%%%%%%%%%1.4.10%%%%%%%%%%%%%%%%%%%%%%%%%%%%%%%%%%%%%
   \item[1.4.10]  Let $\left\{\left(\begin{tabular}{@{}cc@{}}
                     $a$ & $b$ \\
                      0  & $c$
                  \end{tabular}\right) : a, b, c \in \R, a \neq 0, c \neq 0
                  \right\}$.

                  \begin{enumerate}
                     \item Compute the product of
                           $\left(\begin{tabular}{@{}cc@{}}
                              $a_1$ & $b_1$ \\
                              0  & $c_1$
                           \end{tabular}\right)$ and
                           $\left(\begin{tabular}{@{}cc@{}}
                              $a_2$ & $b_2$ \\
                              0  & $c_2$
                           \end{tabular}\right)$ to show that $G$ is closed under
                           matrix multiplication.
                     \item Find the matrix inverse of
                           $\left(\begin{tabular}{@{}cc@{}}
                              $a$ & $b$ \\
                              0  & $c$
                           \end{tabular}\right)$ and deduce that $G$ is closed 
                           under inverses.
                     \item Deduce that $G$ is a subgroup of $GL_2(\R)$.
                     \item Prove that the set of elements of $G$ whose two
                           diagonal entries are equal is also a subgroup of
                           $GL_2(\R)$.
                  \end{enumerate}
\end{enumerate}

The next exercise introduces the \textit{Heisenberg group} over the field $F$
and develops some of its basic properties. When $F = \R$ this groups plays an
important role in quantum mechanics and signal theory by giving a group
theoretic interpretation (due to H. Weyl) of Heisenberg's Uncertainty Principle.
Note also that the Heisenberg group may be defined more generally---for example,
with entries in $\Z$.

\begin{enumerate}
%%%%%%%%%%%%%%%%%%%%%%%%%%%%%%%%%%%%%1.4.11%%%%%%%%%%%%%%%%%%%%%%%%%%%%%%%%%%%%%
   \item[1.4.11]  Let $H(F) = \left\{\left(\begin{tabular}{@{}ccc@{}}
                     1 & $a$ & $b$ \\
                     0 & 1 & $c$ \\
                     0 & 0 & 1
                  \end{tabular}\right) : a, b, c \in F\right\}$---called the
                  \textit{Heisenberg group} over $F$. Let
                  $X = \left(\begin{tabular}{@{}ccc@{}}
                     1 & $a$ & $b$ \\
                     0 & 1 & $c$ \\
                     0 & 0 & 1
                  \end{tabular}\right)$ and $Y =\left(\begin{tabular}{@{}ccc@{}}
                     1 & $d$ & $e$ \\
                     0 & 1 & $f$ \\
                     0 & 0 & 1
                  \end{tabular}\right)$ be elements of $H(F)$.

                  \begin{enumerate}
                     \item Compute the matrix product $XY$ and deduce that
                           $H(F)$ is closed under matrix multiplication. Exhibit
                           explicit matrices such that $XY \neq YX$ (so that
                           $H(F)$ is always non-abelian).
                     \item Find an explicit formula for the matrix inverse
                           $X^{-1}$ and deduce that $H(F)$ is closed under
                           inverses.
                     \item Prove the associative law for $H(F)$ and deduce that
                           $H(F)$ is a group of order $|F|^3$. 

                           (Do not assume that matrix multiplication is 
                           associative).
                     \item Find the order of each element of the finite group
                           $H(\Z/2\Z)$.
                     \item Prove that every nonidentity element of the group
                           $H(\R)$ has infinite order.
                  \end{enumerate}
\end{enumerate}

      \section{The Quaternion Group}
         \begin{enumerate}
%%%%%%%%%%%%%%%%%%%%%%%%%%%%%%%%%%%%%2.5.1%%%%%%%%%%%%%%%%%%%%%%%%%%%%%%%%%%%%%%
   \item[2.5.1]   Let $H$ and $K$ be subgroups of $G$. Exhibit all possible
                  sublattices which show only $G$, 1, $H$, $K$, and their joins
                  and intersections. What distinguishes the different drawings?
%%%%%%%%%%%%%%%%%%%%%%%%%%%%%%%%%%%%%2.5.2%%%%%%%%%%%%%%%%%%%%%%%%%%%%%%%%%%%%%%
   \item[2.5.2]   In each of (a) to (d) list all subgroups of $D_{16}$ that
                  satisfy the given condition.
                  \begin{enumerate}
                     \item Subgroups that are contained in $\cyc{sr^2, r^4}$
                     \item Subgroups that are contained in $\cyc{sr^7, r^4}$
                     \item Subgroups that contain $\cyc{r^4}$
                     \item Subgroups that contain $\cyc{s}$.
                  \end{enumerate}
                  
      \textbf{Solution.}
      
      \begin{enumerate}
         \item The subgroups that are contained in $\cyc{sr^2, r^4}$ are those
               that have an upward path to the join of $sr^2$ and $r^4$. Thus
               these subgroups are: $1$, $\cyc{r^4}$, $\cyc{sr^2}$,
               $\cyc{sr^6}$, and $\cyc{sr^2, r^4}$.
         \item The subgroups that are contained in $\cyc{sr^7, r^4}$ are those
               that have an upward path to the join of $sr^3$ and $r^4$. Since
               the join of $sr^7$ and $r^4$ is $\cyc{sr^3, r^4}$. It follows
               that these subgroups are: $1$, $\cyc{r^4}$, $\cyc{sr^3}$,
               $\cyc{sr^7}$, and $\cyc{sr^7, r^4} = \cyc{sr^3, r^4}$.
         \item The subgroups that contain $r^4$ are: $\cyc{r^4}$,
               $\cyc{sr^2, r^4}$, $\cyc{s, r^4}$, $\cyc{r^2}$,
               $\cyc{sr^3, r^4}$, $\cyc{sr^5, r^4}$, $\cyc{s, r^2}$, $\cyc{r}$,
               $\cyc{sr, r^2}$, and $D_{16}$.
         \item The subgroups that contain $s$ are: $\cyc{s}$, $\cyc{s, r^4}$,
               $\cyc{s, r^2}$, and $D_{16}$.
      \end{enumerate}
%%%%%%%%%%%%%%%%%%%%%%%%%%%%%%%%%%%%%2.5.3%%%%%%%%%%%%%%%%%%%%%%%%%%%%%%%%%%%%%%
   \item[2.5.3]   Show that the subgroup $\cyc{s, r^2}$ of $D_8$ is isomorphic
                  to $V_4$.
                  
      \textbf{Proof.} By Exercise 1.1.36, there is exactly one group, say $K$,
      of order 4 that has no element of order 4. Since
      $\cyc{s, r^2} = \{1, s, r^2, sr^2\}$, it follows that every non-identity
      element of $\cyc{s, r^2}$ has order 2, so that $\cyc{s, r^2} \cong K$.
      Similarly, the Klein-4 group has no element of order 4. Thus
      $V_4 \cong K$, and we conclude that $V_4 \cong \cyc{s, r^2}$. \qed
%%%%%%%%%%%%%%%%%%%%%%%%%%%%%%%%%%%%%2.5.4%%%%%%%%%%%%%%%%%%%%%%%%%%%%%%%%%%%%%%
   \item[2.5.4]   Use the given lattice to find all pairs of elements that
                  generate $D_8$ (there are 12 pairs).
                  
      \textbf{Solution.} It suffices to find all pairs of elements whose join is
      $D_8$. They are: $\cyc{s, r}$, $\cyc{s, rs}$, $\cyc{s, r^3s}$,
      $\cyc{r^2s, rs}$, $\cyc{r^2s, r^3s}$, $\cyc{r^2s, r}$, $\cyc{r^2s, r^3}$,
      $\cyc{r, rs}$, $\cyc{r, r^3s}$,  $\cyc{r^3, s}$, $\cyc{r^3, rs}$, and
      $\cyc{r^3, r^3s}$,
%%%%%%%%%%%%%%%%%%%%%%%%%%%%%%%%%%%%%2.5.5%%%%%%%%%%%%%%%%%%%%%%%%%%%%%%%%%%%%%%
   \item[2.5.5]   Use the given lattice to find all elements $x \in D_{16}$
                  such that $D_{16} = \cyc{x, s}$ (there are 8 such elements
                  $x$).
                  
      \textbf{Solution.} By observing the given lattice of $D_{16}$, we find
      that      
      $$x \in \{r, r^3, r^5, r^7, sr^3, sr^7, sr^5, sr\}.$$
%%%%%%%%%%%%%%%%%%%%%%%%%%%%%%%%%%%%%2.5.6%%%%%%%%%%%%%%%%%%%%%%%%%%%%%%%%%%%%%%
   \item[2.5.6]   Use the given lattices to help find the centralizers of every
                  element in the following groups:

                  (a) $D_8$ \qquad (b) $Q_8$ \qquad
                  (c) $S_3$ \qquad (d) $D_{16}$.
                  
      \begin{enumerate}
         \item $$
                  \begin{tabular}{@{}|c|c|@{}} \hline
                     Elements in $D_8$ & Centralizer \\ \hline
                     1, $r^2$ & $D_8$ \\ \hline
                     $r, r^3$ & $\cyc{r}$ \\ \hline
                     $s, r^2s$ & $\cyc{s, r^2}$ \\ \hline
                     $rs, r^3s$ & $\cyc{rs, r^2}$ \\ \hline
                  \end{tabular}
               $$
         \item $$
                  \begin{tabular}{@{}|c|c|@{}} \hline
                     Elements in $Q_8$ & Centralizer \\ \hline
                     $\pm1$ & $Q_8$ \\ \hline
                     $\pm i$ & $\cyc{i}$ \\ \hline
                     $\pm j$ & $\cyc{j}$ \\ \hline
                     $\pm k$ & $\cyc{k}$ \\ \hline
                  \end{tabular}
               $$
         \item $$
                  \begin{tabular}{@{}|c|c|@{}} \hline
                     Element(s) in $Q_8$ & Centralizer \\ \hline
                     1 & $S_3$ \\ \hline
                     (1 2) & $\cyc{(1\;2)}$ \\ \hline
                     (1 3) & $\cyc{(1\;3)}$ \\ \hline
                     (2 3) & $\cyc{(2\;3)}$ \\ \hline
                     (1 2 3), (1 3 2) & $\cyc{(1\;2\;3)}$ \\ \hline
                  \end{tabular}
               $$
         \item $$
                  \begin{tabular}{@{}|c|c|@{}} \hline
                     Elements in $D_{16}$ & Centralizer \\ \hline
                     1, $r^4$ & $D_{16}$ \\ \hline
                     $r, r^2, r^3, r^5, r^6, r^7$ & $\cyc{r}$ \\ \hline
                     $s, sr^4$ & $\cyc{s, r^4}$ \\ \hline
                     $sr, sr^5$ & $\cyc{sr^5, r^4}$ \\ \hline
                     $sr^2, sr^6$ & $\cyc{sr^2, r^4}$ \\ \hline
                     $sr^3, sr^7$ & $\cyc{sr^3, r^4}$ \\ \hline
                  \end{tabular}
               $$
      \end{enumerate}
%%%%%%%%%%%%%%%%%%%%%%%%%%%%%%%%%%%%%2.5.7%%%%%%%%%%%%%%%%%%%%%%%%%%%%%%%%%%%%%%
   \item[2.5.7]   Find the center of $D_{16}$.
   
      \textbf{Solution.} From Exercise 2.5.6(d), we see that only 1 and $r^4$
      are in the all the centralizers of the elements of $D_{16}$. Thus
      $Z(D_{16}) = \cyc{r_4}$.
%%%%%%%%%%%%%%%%%%%%%%%%%%%%%%%%%%%%%2.5.8%%%%%%%%%%%%%%%%%%%%%%%%%%%%%%%%%%%%%%
   \item[2.5.8]   In each of the following groups find the normalizer of each
                  subgroup:

                  (a) $S_3$ \qquad (b) $Q_8$.
%%%%%%%%%%%%%%%%%%%%%%%%%%%%%%%%%%%%%2.5.9%%%%%%%%%%%%%%%%%%%%%%%%%%%%%%%%%%%%%%
   \item[2.5.9]   Draw the lattices of subgroups of the following groups:

                  (a) $\Z/16\Z$ \qquad (b) $\Z/24\Z$ \qquad
                  (c) $\Z/48\Z$. [See Exercise 6 in Section 3.]
%%%%%%%%%%%%%%%%%%%%%%%%%%%%%%%%%%%%%2.5.10%%%%%%%%%%%%%%%%%%%%%%%%%%%%%%%%%%%%%
   \item[2.5.10]  Classify groups of order 4 by proving that if $|G| = 4$ then
                  $G \cong Z_4$ or $G\cong V_4$. [See Exercise 36, Section 1.1.]
%%%%%%%%%%%%%%%%%%%%%%%%%%%%%%%%%%%%%2.5.11%%%%%%%%%%%%%%%%%%%%%%%%%%%%%%%%%%%%%
   \item[2.5.11]  Consider the group of order 16 with the following
                  presentation:

                  $$QD_{16} = \cyc{\sigma, \tau : \sigma^8 = \tau^2 = 1,
                    \sigma\tau = \tau\sigma^3}$$
                  (called the \textit{quasidihedral} or \textit{semidihedral}
                  group of order 16). This group has three subgroups of order 8:
                  $\cyc{\tau, \sigma^2} \cong D_8, \cyc{\sigma} \cong Z_8$ and
                  $\cyc{\sigma^2, \sigma\tau} \cong Q_8$ and every proper
                  subgroup is contained in one of these three subgroups. Fill in
                  the missing subgroups in the lattice of all subgroups of the 
                  quasidiheral group on the following page, exhibiting each
                  subgroup with at most two generators. (This is another example
                  of a nonplanar lattice.)
\end{enumerate}

\noindent The next three examples lead to two nonisomorphic groups that have the 
          same lattice of subgroups.

\begin{enumerate}
%%%%%%%%%%%%%%%%%%%%%%%%%%%%%%%%%%%%%2.5.12%%%%%%%%%%%%%%%%%%%%%%%%%%%%%%%%%%%%%
   \item[2.5.12]  The group
                  $A = Z_2 \times Z_4 = \cyc{a, b : a^2 = b^4 = 1, ab = ba}$ has
                  order 8 and has three subgroups of order 4:
                  $\cyc{a, b^2} \cong V_4$, $\cyc{b} \cong Z_4$ and
                  \begin{verbatim}
                     *
                     *
                     *
                     *
                     *
                     *
                     *
                     *
                     *
                  \end{verbatim}
                  $\cyc{ab} \cong Z_4$ and every proper subgroup is contained in
                  one of these three. Draw the lattice of all subgroups of $A$,
                  giving each subgroup in terms of at most two generators.
%%%%%%%%%%%%%%%%%%%%%%%%%%%%%%%%%%%%%2.5.13%%%%%%%%%%%%%%%%%%%%%%%%%%%%%%%%%%%%%
   \item[2.5.13]  The group
                  $G = Z_2 \times Z_8 = \cyc{x, y : x^2 = y^8 = 1, xy = yx}$ has
                  order 16 and has three subgroups of order 8:
                  $\cyc{x, y^2} \cong Z_2 \times Z_4$, $\cyc{y} \cong Z_8$ and
                  $\cyc{xy} \cong Z_8$ and every proper subgroup is contained in
                  one of these three. Draw the lattice of all subgroups of $G$,
                  giving each subgroup in terms of at most two generators.
%%%%%%%%%%%%%%%%%%%%%%%%%%%%%%%%%%%%%2.5.14%%%%%%%%%%%%%%%%%%%%%%%%%%%%%%%%%%%%%
   \item[2.5.14]  Let $M$ be the group of order 16 with the following 
                  presentation:
                  $$\cyc{u, v : u^2 v^8 = 1, vu = uv^5}$$
                  (sometimes called the \textit{modular} group of order 16). It
                  has three subgroups of order 8: $\cyc{u, v^2}$, $\cyc{v}$, and
                  $\cyc{uv}$ and every proper subgroup is contained in one of
                  these three. Prove that $\cyc{u, v^2} \cong Z_2 \times Z_4$,
                  $\cyc{v} \cong Z_8$ and $\cyc{uv} \cong Z_8$. Show that the
                  lattice of subgroups of $M$ is the same as the lattice of
                  subgroups of $Z_2 \times Z_8$ (cf. Exercise 13) but that these
                  two groups are not isomorphic.
%%%%%%%%%%%%%%%%%%%%%%%%%%%%%%%%%%%%%2.5.15%%%%%%%%%%%%%%%%%%%%%%%%%%%%%%%%%%%%%
   \item[2.5.15]  Describe the isomorphism type of each of the three subgroups
                  of $D_{16}$ of order 8.
%%%%%%%%%%%%%%%%%%%%%%%%%%%%%%%%%%%%%2.5.16%%%%%%%%%%%%%%%%%%%%%%%%%%%%%%%%%%%%%
   \item[2.5.16]  Use the lattice of subgroups of the quasidihedral group of
                  order 16 to show that every element of order 2 is contained in
                  the proper subgroup $\cyc{\tau, \sigma^2}$.
%%%%%%%%%%%%%%%%%%%%%%%%%%%%%%%%%%%%%2.5.17%%%%%%%%%%%%%%%%%%%%%%%%%%%%%%%%%%%%%
   \item[2.5.17]  Use the lattice of subgroups of the modular group $M$ of order
                  16 to show that the set $\{x \in M : x^2 = 1\}$ is a subgroup
                  of $M$ isomorphic to the Klein 4-group.
%%%%%%%%%%%%%%%%%%%%%%%%%%%%%%%%%%%%%2.5.18%%%%%%%%%%%%%%%%%%%%%%%%%%%%%%%%%%%%%
   \item[2.5.18]  Use the lattice to help find the centralizer of every element
                  of $QD_{16}$.
%%%%%%%%%%%%%%%%%%%%%%%%%%%%%%%%%%%%%2.5.19%%%%%%%%%%%%%%%%%%%%%%%%%%%%%%%%%%%%%
   \item[2.5.19]  Use the lattice to help find $N_{D_{16}}(\cyc{s, r^4})$.
%%%%%%%%%%%%%%%%%%%%%%%%%%%%%%%%%%%%%2.5.20%%%%%%%%%%%%%%%%%%%%%%%%%%%%%%%%%%%%%
   \item[2.5.20]  Use the lattice of subgroups of $QD_{16}$ to help find the
                  normalizers.

                  (a) $N_{QD_{16}}(\cyc{\tau\sigma})$ \qquad
                  (b) $N_{QD_{16}}(\cyc{\tau, \sigma^4})$.
\end{enumerate}

      \section{Homomorphisms And Isomorphisms}
         Let $G$ and $H$ be groups.
\begin{enumerate}
%%%%%%%%%%%%%%%%%%%%%%%%%%%%%%%%%%%%%1.6.1%%%%%%%%%%%%%%%%%%%%%%%%%%%%%%%%%%%%%%
   \item[1.6.1]   Let $\varphi : G \rightarrow H$ be a homomorphism.
                  \begin{enumerate}
                     \item Prove that $\varphi(x^n) = \varphi(x)^n$ for all
                           $n \in \Z^+$.
                     \item Do part (a) for $n = -1$ and deduce that
                           $\varphi(x^n) = \varphi(x)^n$ for all $n \in \Z$.
                  \end{enumerate}

      \textbf{Solution.}

      \begin{enumerate}
         \item \textbf{Proof.} We shall proceed by induction on $n$. It is clear
               that $\varphi(x^1) = \varphi(x)^1$. Now suppose that
               $\varphi(x^k) = \varphi(x)^k$ for some integer $k$. Thus
               \begin{align*}
                  \varphi(x^{k+1}) &= \varphi(x^kx) \\
                     &= \varphi(x^k)\varphi(x)
                        &[\varphi\text{ is a homomorphism}] \\
                     &= \varphi(x)^k\varphi(x) &[\text{Inductive hypothesis}] \\
                     &= \varphi(x)^{k+1},
               \end{align*}
               so that, by Mathematical Induction, $\varphi(x^n) = \varphi(x)^n$ 
               for all $n \in \Z^+$. \qed
         \item Since
               $$1 \cdot \varphi(1) = \varphi(1) = \varphi(1 \cdot 1) =
                 \varphi(1)\cdot\varphi(1),$$
               it follows by cancellation that $\varphi(1) = 1$. Thus
               $$\varphi(x)\varphi(x^{-1}) = \varphi(xx^{-1}) =\varphi(1) = 1,$$
               so that $\varphi(x^{-1}) = \varphi(x)^{-1}$. Now let $n$ be a
               positive integer. Then it follows that
               \begin{align*}
                  \varphi(x^{-n}) &= \varphi((x^{-1})^n) \\
                     &= \varphi(x^{-1})^n &[\text{1.6.1(a)}] \\
                     &= (\varphi(x)^{-1})^n \\
                     &= \varphi(x)^{-n}.
               \end{align*}
               Moreover $\varphi(x^0) = 1 = \varphi(x)^0$; thus we can conclude 
               that $\varphi(x^n) = \varphi(x)^n$ for all $n \in \Z$.
      \end{enumerate}
%%%%%%%%%%%%%%%%%%%%%%%%%%%%%%%%%%%%%1.6.2%%%%%%%%%%%%%%%%%%%%%%%%%%%%%%%%%%%%%%
   \item[1.6.2]   If $\varphi : G \rightarrow H$ is an isomorphism, prove that
                  $|\varphi(x)| = |x|$ for all $x \in G$. Deduce that any two
                  isomorphic groups have the same number of elements of order
                  $n$ for each $n \in \Z^+$. Is the result true if $\varphi$ is
                  only assumed to be a homomorphism?

      \textbf{Proof.} Assume that $\varphi : G \rightarrow H$ is a group
      isomorphism. Let $x \in G$. Suppose $|x| = n$. By the preceding exercise, 
      we have that $\varphi(x)^n = \varphi(x^n) = \varphi(1) = 1$, so that
      $|\varphi(x)| \le n$. Now suppose that $|\varphi(x)| = m < n$. Then we
      must have that $\varphi(1) = 1 = \varphi(x)^m = \varphi(x^m)$. That is,
      $x^m = 1$ (since $\varphi$ is injective), a contradiction since $|x| = n$.
      Thus $|\varphi(x)| = |x| = n$. Finally suppose that $|x| = \infty$ and
      $|\varphi(x)| = r < \infty$. Then, as previously argued, we must have
      that $x^r = 1$, a contradiction. Thus if $y \in G$, it must follow that
      $|y| = |\varphi(y)|$. \qed

      For a positive integer $n$, we can now exhibit a bijection (using
      $\varphi$) between the elements of $G$ of order $n$ and the elements of
      $H$ of order $n$. Thus any two isomorphic groups must have the same number 
      of elements of order $n$ for each $n \in \Z^+$. If $\varphi$ is only 
      assumed to be a homomorphism then the result is not generally true. 
      Consider the homomorphism
      $$\alpha : S_3 \rightarrow \{1\}.$$
      Although $S_3$ has an element of order 2, the trivial group $\{1\}$ has no 
      element of order 2.
%%%%%%%%%%%%%%%%%%%%%%%%%%%%%%%%%%%%%1.6.3%%%%%%%%%%%%%%%%%%%%%%%%%%%%%%%%%%%%%%
   \item[1.6.3]   If $\varphi : G \rightarrow H$ is an isomorphism, prove that
                  $G$ is abelian if and only if $H$ is abelian. If
                  $\varphi : G \rightarrow H$ is a homomorphism, what additional
                  conditions on $\varphi$ (if any) are sufficient to ensure that
                  if $G$ is abelian, then so is $H$?

      \textbf{Proof.} Suppose $\varphi : G \rightarrow H$ is an isomorphism.

      ($\Rightarrow$) Assume $G$ is abelian. Consider $h_1$ and $h_2$ in $H$.
      Since $\varphi$ is surjective, there exist $g_1$, $g_2 \in G$ such that
      $\varphi(g_1) = h_1$ and $\varphi(g_2) = h_2$. Since $G$ is abelian, we
      have that
      $$h_1h_2 = \varphi(g_1)\varphi(g_2) = \varphi(g_1g_2) = \varphi(g_2g_1) = 
        \varphi(g_2)\varphi(g_1) = h_2h_1,$$
      so that $H$ is also abelian.

      ($\Rightarrow$) Now assume that $H$ is abelian. Consider $g_3$ and $g_4$ 
      in $G$. Since $H$ is abelian, we have that
      $$\varphi(g_3g_4) = \varphi(g_3)\varphi(g_4) = \varphi(g_4)\varphi(g_3) =
        \varphi(g_4g_3).$$
      We thus conclude that $g_3g_4 = g_4g_3$ since $\varphi$ is injective. That
      is, $G$ is abelian. \qed

      Finally suppose $\varphi$ is an homomorphism and $G$ is abelian. Looking
      at the first direction of our proof above, we see that restricting
      $\varphi$ to be surjective is sufficient to make $H$ abelian.
%%%%%%%%%%%%%%%%%%%%%%%%%%%%%%%%%%%%%1.6.4%%%%%%%%%%%%%%%%%%%%%%%%%%%%%%%%%%%%%%
   \item[1.6.4]   Prove that the multiplicative groups $\R - \{0\}$ and
                  $\C - \{0\}$ are not isomorphic.

      \textbf{Proof.} Since $\R - \{0\}$ has no element of order 4, and since
      $\C - \{0\}$ has 2 elements ($i$ and $-i$) of order 4, it follows that
      these two multiplicative groups are not isomorphic. \qed
%%%%%%%%%%%%%%%%%%%%%%%%%%%%%%%%%%%%%1.6.5%%%%%%%%%%%%%%%%%%%%%%%%%%%%%%%%%%%%%%
   \item[1.6.5]   Prove that the additive groups of $\R$ and $\Q$ are not
                  isomorphic.

      \textbf{Proof.} Since $|\R| \neq |\Q|$, it follows that $(\R, +)$ and
      $(\Q, +)$ are not isomorphic. \qed
%%%%%%%%%%%%%%%%%%%%%%%%%%%%%%%%%%%%%1.6.6%%%%%%%%%%%%%%%%%%%%%%%%%%%%%%%%%%%%%%
   \item[1.6.6]   Prove that the additive groups of $\Z$ and $\Q$ are not
                  isomorphic.

      \textbf{Proof.} First we shall show that $(\Q, +)$ is not cyclic. So
      suppose that $(\Q, +) = \cyc{c}$. Thus there exists an integer $n$ such 
      that $nc = \D\frac{c}{2}$, so that $\D n = \frac{1}{2}$, a contradiction. 
      That is, $(\Q, +)$ is not cyclic. Now suppose to the contrary that
      $\varphi : \Z \rightarrow \Q$ is a homomorphism from $(\Z, +)$ to
      $(\Q, +)$. Let $q \in \Q$ and let $z$ be the unique preimage of $q$ under
      $\varphi$. Thus
      \begin{align*}
         q &= \varphi(z) \\
           &= \varphi(z \cdot 1) \\
           &= z \cdot \varphi(1),  &[1.6.1(a)]
      \end{align*}
      so that $\Q = \cyc{\varphi(1)}$, a contradiction. Thus $(\Z, +)$ and
      $(\Q, +)$ are not isomorphic. \qed
%%%%%%%%%%%%%%%%%%%%%%%%%%%%%%%%%%%%%1.6.7%%%%%%%%%%%%%%%%%%%%%%%%%%%%%%%%%%%%%%
   \item[1.6.7]   Prove that $D_8$ and $Q_8$ are not isomorphic.

      \textbf{Proof.} Since $D_8$ has exactly 5 elements of order 2 and $Q_8$
      has exactly 1 element of order 2, it follows that these two groups are not
      isomorphic. \qed
%%%%%%%%%%%%%%%%%%%%%%%%%%%%%%%%%%%%%1.6.8%%%%%%%%%%%%%%%%%%%%%%%%%%%%%%%%%%%%%%
   \item[1.6.8]   Prove that if $n \neq m$, $S_n$ and $S_m$ are not isomorphic.

      \textbf{Proof.} Let $n$ and $m$ be unequal positive integers so that
      $n! \neq m!$; i.e $n! = |S_n| \neq |S_m| = m!$. Thus no bijective map can
      exist between $S_n$ and $S_m$, so that these two groups are not
      isomorphic. \qed
%%%%%%%%%%%%%%%%%%%%%%%%%%%%%%%%%%%%%1.6.9%%%%%%%%%%%%%%%%%%%%%%%%%%%%%%%%%%%%%%
   \item[1.6.9]   Prove that $D_{24}$ and $S_4$ are not isomorphic.

      \textbf{Proof.} Since $D_{24}$ has exactly 2 elements of order 4 and $S_4$
      has exactly 6 elements of order 4, it follows that these two groups are
      not isomorphic. \qed
%%%%%%%%%%%%%%%%%%%%%%%%%%%%%%%%%%%%%1.6.10%%%%%%%%%%%%%%%%%%%%%%%%%%%%%%%%%%%%%
   \item[1.6.10]  Fill in the details of the proof that the symmetric groups
                  $S_\triangle$ and $S_\Omega$ are isomorphic if
                  $|\triangle| = |\Omega|$ as follows: let
                  $\theta : \triangle \rightarrow \Omega$ be a bijection. Define
                  $$\varphi : S_\triangle \rightarrow S_\Omega \qquad
                    \text{by} \qquad \varphi(\sigma) = \theta \circ \sigma \circ 
                    \theta^{-1} \text{ for all } \sigma \in S_\triangle$$
                  and prove the following:
                  \begin{enumerate}
                     \item $\varphi$ is well defined, that is, if $\sigma$ is a
                           permutation of $\triangle$ then
                           $\theta\circ\sigma\circ\theta^{-1}$ is a permutation
                           of $\Omega$.
                     \item $\varphi$ is a bijection from $S_\triangle$ onto
                           $S_\Omega$. [Find a 2-sided inverse for $\varphi$.]
                     \item $\varphi$ is a homomorphism, that is,
                           $\varphi(\sigma\circ\tau) =
                            \varphi(\sigma)\circ\varphi(\tau)$.
                  \end{enumerate}
                  Note the similarity to the \textit{change of basis} or
                  \textit{similarity} transformations for matrices.

      \textbf{Proof.}

      \begin{enumerate}
         \item Let $\sigma \in S_\triangle$. Notice that
               $\varphi(\sigma) = \theta\circ\sigma\circ\theta^{-1}$ maps
               $\Omega$ into $\Omega$, and is also a bijecton since it is a
               composition of bijective maps. Thus
               $\varphi(\sigma) \in S_\Omega$, so that $\varphi$ is well
               defined.
         \item Consider the map
               $$\alpha : S_\Omega \rightarrow S_\triangle \qquad
                 \text{by} \qquad \alpha(\sigma) = \theta \circ \sigma^{-1} 
                 \circ \theta^{-1} \text{ for all } \sigma \in S_\Omega.$$
               A trivial computation will show us that $\varphi\circ\alpha$ is
               the identity map on $S_\Omega$ and $\alpha\circ\varphi$ is the
               identity map on $S_\triangle$. Thus $\varphi$ is a bijection.
         \item Let $\sigma, \tau \in S_\triangle$. So
               \begin{align*}
                  \varphi(\sigma\circ\tau) &=
                     \theta\circ\sigma\circ\tau\circ\theta^{-1} \\
                     &= \theta\circ\sigma\circ\theta^{-1}\circ\theta\circ
                        \tau\circ\theta^{-1} \\
                     &= \varphi(\sigma) \circ \varphi(\tau),
               \end{align*}
               as desired.
      \end{enumerate} \qed
%%%%%%%%%%%%%%%%%%%%%%%%%%%%%%%%%%%%%1.6.11%%%%%%%%%%%%%%%%%%%%%%%%%%%%%%%%%%%%%
   \item[1.6.11]  Let $A$ and $B$ be groups. Prove that
                  $A \times B \cong B\times A$.

      \textbf{Proof.} Consider the map $f : A \times B \rightarrow B \times A$,
      $(a, b) \mapsto (b, a)$. Now define
      $$g : B \times A\rightarrow A\times B, \quad (b, a) \mapsto (a, b).$$
      So $f$ is bijective since $g$ is its two-sided inverse. Now $f$ is a
      homomorphism because
      \begin{align*}
         f((a_1, b_1)(a_2, b_2)) &= f((a_1a_2, b_1b_2)) \\
            &= (b_1b_2, a_1a_2) \\
            &= (b_1, a_1)(b_2, a_2) \\
            &= f((a_1, b_1))f((a_2, b_2)).
      \end{align*} \qed
%%%%%%%%%%%%%%%%%%%%%%%%%%%%%%%%%%%%%1.6.12%%%%%%%%%%%%%%%%%%%%%%%%%%%%%%%%%%%%%
   \item[1.6.12]  Let $A$, $B$, and $C$ be groups and let $G = A \times B$ and
                  $H = B \times C$. Prove that $G \times C \cong A \times H$.

      \textbf{Proof.} Proceed as we did in Exercise 1.6.11 with the following
      modification:
      $$f : G \times C \rightarrow A \times H,
        \quad((a, b), c) \mapsto (a, (b, c))$$ and 
      $$g : A \times H\rightarrow G\times C,
        \quad (a, (b, c)) \mapsto ((a, b), c).$$ \qed
%%%%%%%%%%%%%%%%%%%%%%%%%%%%%%%%%%%%%1.6.13%%%%%%%%%%%%%%%%%%%%%%%%%%%%%%%%%%%%%
   \item[1.6.13]  Let $G$ and $H$ be groups and let $\varphi : G \rightarrow H$
                  be a homomorphism. Prove that the image of $\varphi$,
                  $\varphi(G)$, is a subgroup of $H$. Prove that if $\varphi$ is
                  injective then $G \cong \varphi(G)$.

      \textbf{Proof.} The set $\varphi(G)$ is nonempty since it contains
      $\varphi(1) = 1$. So let $h_1, h_2 \in \varphi(G)$. Thus there exist
      $g_1, g_2 \in G$ such that $\varphi(g_1) = h_1$ and $\varphi(g_2) = h_2$. 
      So
      $$\varphi(g_1g_2^{-1}) = \varphi(g_1)\varphi(g_2)^{-1} = h_1h_2^{-1} \in
        \varphi(G).$$
      That is $\varphi(G)$ is a subgroup of $H$. Now suppose that $\varphi$ is
      injective and consider the map $\alpha : G \rightarrow \varphi(G)$,
      $g \mapsto \varphi(g)$. Since $\varphi$ is an injective homomorphism, it
      follows that $\alpha$ is also an injective homomorphism. Also it is clear
      that $\alpha$ is onto. Thus $\alpha$ is an isomorphism and we have that
      $G \cong \varphi(G)$. \qed
%%%%%%%%%%%%%%%%%%%%%%%%%%%%%%%%%%%%%1.6.14%%%%%%%%%%%%%%%%%%%%%%%%%%%%%%%%%%%%%
   \item[1.6.14]  Let $G$ and $H$ be groups and let $\varphi : G \rightarrow H$
                  be a homomorphism. Define the \text{kernel} of $\varphi$ to be
                  $\{g \in G : \varphi(g) = 1_H\}$. Prove that the kernel of
                  $\varphi$ is a subgroup of $G$. Prove that $\varphi$ is
                  injective if and only if the kernel of $\varphi$ is the
                  identity subgroup of $G$.

      \textbf{Proof 1.} Let ker($\varphi$) be the kernel of $\varphi$. The set
      ker($\varphi$) is not empty because $1_G \in \text{ker}(\varphi)$. Now let
      $x, y \in \text{ker}(\varphi)$. Then we have that
      $$\varphi(xy^{-1}) = \varphi(x)\varphi(y)^{-1} = 1_H{1_H}^{-1} = 1_H,$$
      so that $xy^{-1} \in \text{ker}(\varphi)$. That is ker($\varphi$) is a
      subgroup of $G$. \qed

      \textbf{Proof 2.} ($\Rightarrow$) Assume that $\varphi$ is injective. Let
      $x \in \text{ker}(\varphi)$. Thus we have that
      $\varphi(1_G) = \varphi(x) = 1_H$, so that $x = 1_G$ since $\varphi$ is
      one to one. That is ker($\varphi$) = $\{1_G\}$, the identity subgroup of
      $G$.

      ($\Leftarrow$) Assume that ker($\varphi$) = $\{1_G\}$. Suppose that
      $\varphi(g_1) = \varphi(g_2)$ for some $g_1, g_2 \in G$. Then it follows
      that $\varphi(g_1)\varphi(g_2)^{-1} = 1_H$, so that
      $\varphi(g_1{g_2}^{-1}) = 1_H$. Since ker($\varphi$) = $\{1_G\}$, we can
      conclude that $g_1{g_2}^{-1} = 1_G$, so that $g_1 = g_2$; i.e., $\varphi$
      is injective. \qed      
%%%%%%%%%%%%%%%%%%%%%%%%%%%%%%%%%%%%%1.6.15%%%%%%%%%%%%%%%%%%%%%%%%%%%%%%%%%%%%%
   \item[1.6.15]  Define a map $\pi : \R^2 \rightarrow \R$ by $\pi((x, y)) = x$.
                  Prove that $\pi$ is a homomorphism and find the kernel of
                  $\pi$.

      \textbf{Proof.} Let $(a, b), (c, d) \in \R^2$. Then it follows immediately
      that $\pi$ is a homomorphism since
      $$\pi((a, b) + (c, d)) = \pi((a + c, b + d)) =
         a + c = \pi(a, b) + \pi(c, d).$$
      The kernel of $\pi$ is the set $\{(0, y) : y \in \R\}$. \qed
%%%%%%%%%%%%%%%%%%%%%%%%%%%%%%%%%%%%%1.6.16%%%%%%%%%%%%%%%%%%%%%%%%%%%%%%%%%%%%%
   \item[1.6.16]  Let $A$ and $B$ be groups and let $G$ be their direct product,
                  $A \times B$. Prove that the maps $\pi_1 : G \rightarrow A$
                  and $\pi_2 : G \rightarrow B$ defined by $\pi_1((a, b)) = a$
                  and $\pi_2((a, b)) = b$ are homomorphisms and find their
                  kernels.

      \textbf{Proof.} Let $(a_1, b_1), (a_2, b_2) \in G$. Then it follows 
      immediately that $\pi_1$ and $\pi_2$ are homomorphisms since
      $$\pi_1((a_1, b_1)(a_2, b_2)) = \pi_1((a_1a_2, b_1b_2)) =
         a_1a_2 = \pi_1(a_1, b_1)\pi(a_2, b_2)$$
      and
      $$\pi_2((a_1, b_1)(a_2, b_2)) = \pi_2((a_1a_2, b_1b_2)) =
         b_1b_2 = \pi_1(a_1, b_1)\pi(a_2, b_2).$$
      The kernel of $\pi_1 = \{(1, b) : b \in B\}$ and
      the kernel of $\pi_2 = \{(a, 1) : a \in A\}$. \qed
%%%%%%%%%%%%%%%%%%%%%%%%%%%%%%%%%%%%%1.6.17%%%%%%%%%%%%%%%%%%%%%%%%%%%%%%%%%%%%%
   \item[1.6.17]  Let $G$ be any group. Prove that the map from $G$ to itself
                  defined by $g \mapsto g^{-1}$ is a homomorphism if and only if
                  $G$ is abelian.

      \textbf{Proof.} Let $x, y \in G$. Consider the map
      $\alpha : G \rightarrow G$, $g \mapsto g^{-1}$. 

      ($\Leftarrow$) Assume that $G$ is abelian. Then it
      follows that
      \begin{align*}
         \alpha(xy) &= (xy)^{-1} \\
            &= y^{-1}x^{-1} \\
            &= x^{-1}y^{-1} &[G \text{ is abelian}] \\
            &= \alpha(x)\alpha(y),
      \end{align*}
      so that $\alpha$ is a homomorphism.

      ($\Rightarrow$) Assume that $\alpha$ is a homomorphism. Then it
      follows that
      \begin{align*}
         xy &= \alpha(x^{-1})\alpha(y^{-1}) \\
            &= \alpha(x^{-1}y^{-1}) \\
            &= \alpha((yx)^{-1}) \\
            &= yx,
      \end{align*}
      so that $G$ is abelian. \qed
%%%%%%%%%%%%%%%%%%%%%%%%%%%%%%%%%%%%%1.6.18%%%%%%%%%%%%%%%%%%%%%%%%%%%%%%%%%%%%%
   \item[1.6.18]  Let $G$ be any group. Prove that the map from $G$ to itself
                  defined by $g \mapsto g^2$ is a homomorphism if and only if
                  $G$ is abelian.

      \textbf{Proof.} Let $x, y \in G$. Consider the map
      $\alpha : G \rightarrow G$, $g \mapsto g^2$. 

      ($\Leftarrow$) Assume that $G$ is abelian. Then it
      follows that
      \begin{align*}
         \alpha(xy) &= (xy)^2 \\
            &= x^2y^2 &[G \text{ is abelian}] \\
            &= \alpha(x)\alpha(y),
      \end{align*}
      so that $\alpha$ is a homomorphism.

      ($\Rightarrow$) Assume that $\alpha$ is a homomorphism. Then it
      follows that
      \begin{align*}
         x^2y^2 &= \alpha(x)\alpha(y) \\
            &= \alpha(xy) \\
            &= (xy)^2 \\
            &= xyxy,
      \end{align*}
      so that $xxyy = xyxy$. By cancellation we thus have $xy = yx$; i.e, $G$ is
      abelian. \qed
%%%%%%%%%%%%%%%%%%%%%%%%%%%%%%%%%%%%%1.6.19%%%%%%%%%%%%%%%%%%%%%%%%%%%%%%%%%%%%%
   \item[1.6.19]  Let $G = \{z \in \C : z^n = 1 \text{ for some }n \in \Z^+\}$.
                  Prove that for any fixed integer $k > 1$ the map from $G$ to
                  itself defined by $z \mapsto z^k$ is a surjective homomorphism
                  but is not an isomorphism.

      \textbf{Proof.} Consider an integer $k > 1$ and the map
      $\alpha : G \rightarrow G$, $g \mapsto g^k$. Let $x, y \in G$. The map
      $\alpha$ is a homomorphism since
      $\alpha(xy) = (xy)^k = x^ky^k = \alpha(x)\alpha(y)$. Since $x \in G$, it
      follows that $x^m = 1$ for some positive integer $m$. Notice that
      $x^{1/k} \in G$ since $(x^{1/k})^{km} = 1$. Thus $\alpha$ is onto because
      $\alpha(x^{1/k}) = x$. Now consider the complex number
      $e^{2\pi/k} = \cos2\pi/k + i \sin2\pi/k$. Notice that $e^{2\pi/k} \in G$
      since $(e^{2\pi/k})^k = 1$. Also notice that $e^{2\pi/k} \neq 1$ (since
      $2\pi/k$ is not a multiple of $2\pi$), but we have that
      $\alpha(e^{2\pi/k}) = \alpha(1) = 1$, so that $\alpha$ is not injective; 
      i.e, $\alpha$ is not an isomorphism. \qed
%%%%%%%%%%%%%%%%%%%%%%%%%%%%%%%%%%%%%1.6.20%%%%%%%%%%%%%%%%%%%%%%%%%%%%%%%%%%%%%
   \item[1.6.20]  Let $G$ be a group and let Aut($G$) be the set of all
                  isomorphisms from $G$ onto $G$. Prove that Aut($G$) is a
                  group under function composition (called the
                  \text{automorphism group} of $G$ and the elements of Aut($G$)
                  are called \text{automorphisms} of $G$).

      \textbf{Proof.}

      \textbf{Closure.} Let $\alpha, \gamma \in \text{Aut}(G)$. Since the
      composition of two bijective functions is also bijective, it follows that
      $\alpha \circ \gamma$ is bijective. Now let $x, y \in G$. It follows that
      $$(\alpha\circ\gamma)(xy) = \alpha(\gamma(xy)) =
         \alpha(\gamma(x)\gamma(y)) = \alpha(\gamma(x))\alpha(\gamma(y)) =
         ((\alpha\circ\gamma)(x))((\alpha\circ\gamma)(y)),$$
      so that $\alpha\circ\gamma)$ is also an isomorphism on $G$, and thus,
      Aut($G$) is closed.

      \textbf{Associativity.} This follows from the associativity of functions.

      \textbf{Identity.} The identity map is the identity of Aut($G$).

      \textbf{Inverse.} Since every map in Aut($G$) is bijective, it follows 
      that every map has a 2-sided inverse.

      Thus we have shown that Aut($G$) is a group under composition. \qed      
%%%%%%%%%%%%%%%%%%%%%%%%%%%%%%%%%%%%%1.6.21%%%%%%%%%%%%%%%%%%%%%%%%%%%%%%%%%%%%%
   \item[1.6.21]  Prove that for each fixed nonzero $k \in \Q$ the map from $\Q$
                  to itself defined by $q \mapsto kq$ is an automorphism of
                  $\Q$.

      \textbf{Proof.} Let $k$ be a nonzero rational number. Consider the map
      $f : \Q \rightarrow \Q$, $q \mapsto kq$. Let $x, y \in Q$. We have that
      $f(x + y) = k(x + y) = kx + ky = f(x) + f(y)$, so that $f$ is a 
      homomorphism. Now suppose $f(x) = f(y)$, so that $kx = ky$. Since
      $k \neq 0$, we shall multiply the equality $kx = ky$ by $1/k$ to get
      $x = y$; i.e, $f$ is injective. Since $f(x/k) = x$, it follows that $f$ is
      onto, so that $f$ is an automorphism of $\Q$. \qed
%%%%%%%%%%%%%%%%%%%%%%%%%%%%%%%%%%%%%1.6.22%%%%%%%%%%%%%%%%%%%%%%%%%%%%%%%%%%%%%
   \item[1.6.22]  Let $A$ be an abelian group and fix some $k \in \Z$. Prove
                  that the map $a \mapsto a^k$ is a homomorphism from $A$ to 
                  itself. If $k = -1$ prove that this homomorphism is an
                  isomorphism.

      \textbf{Proof.} Let $k$ be an integer, $x, y \in A$. Consider the map
      $f : A \rightarrow A$, $a \mapsto a^k$. Since $A$ is abelian we have that
      $f(xy) = (xy)^k = x^ky^k = f(x)f(y)$, so that $f$ is a homomorphism. Now 
      assume that $k = -1$. In this case, notice that the map $f$ is also the
      2-sided inverse of $f$; thus $f$ is bijective, so that $f$ is an 
      isomorphism. \qed
%%%%%%%%%%%%%%%%%%%%%%%%%%%%%%%%%%%%%1.6.23%%%%%%%%%%%%%%%%%%%%%%%%%%%%%%%%%%%%%
   \item[1.6.23]  Let $G$ be a finite group which possesses an automorphism
                  $\sigma$ such that $\sigma(g) = g$ if and only if $g = 1$. If
                  $\sigma^2$ is the identity map from $G$ to $G$, prove that $G$
                  is abelian (such an automorphism $\sigma$ is called
                  \text{fixed point free} of order 2). [Hint. Show that every
                  element of $G$ can be written in the form $x^{-1}\sigma(x)$
                  and apply $\sigma$ to such an expression.]

      \textbf{Proof.} Consider the map $\alpha : G \rightarrow G$,
      $g \mapsto g^{-1}\sigma(g)$. Suppose that for some $x, y \in G$, we have
      that $\alpha(x) = \alpha(y)$. Then it follows that
      $x^{-1}\sigma(x) = y^{-1}\sigma(y)$, so that
      $yx^{-1} =\nobreak \sigma(y)\sigma(x)^{-1} = \sigma(yx^{-1})$. Since
      $\sigma(g) = g$ if and only if $g = 1$, we must then have that
      $yx^{-1} = 1$; thus $y = x$, so that $\alpha$ is injective; $\alpha$ is
      also surjective since $G$ is finite. So let $z \in G$. Then we must have 
      that $z = h^{-1}\sigma(h)$ for some $h \in G$. So
      $\sigma(z) = \sigma(h^{-1}\sigma(h)) = \sigma(h)^{-1}\sigma^2(h) =
       \sigma(h)^{-1}h = z^{-1}$. We can then conclude by Exercise 1.6.17 that
      $G$ is abelian. \qed
%%%%%%%%%%%%%%%%%%%%%%%%%%%%%%%%%%%%%1.6.24%%%%%%%%%%%%%%%%%%%%%%%%%%%%%%%%%%%%%
   \item[1.6.24]  Let $G$ be a finite group and let $x$ and $y$ be distinct
                  elements of order 2 in $G$ that generate $G$. Prove that
                  $G \cong D_{2n}$, where $n = |xy|$. [See Exercise 1.2.6]
                  
      \textbf{Proof.} Let $a = xy$. Since $G$ is finite, $|a|$ must also be
      finite. So write $|a| = n$. By Exercise 1.2.6, we have that
      $ax = xa^{-1}$. Also note since $y = x^2y = x(xy) = xa$, the elements
      $a$ and $x$ generate $G$. Thus we have that
      $$G = \cyc{a, x : a^n = x^2 = 1, ax = xa^{-1}}.$$
      By the discussion on Page 38-39 of the Textbook, the map
      $\varphi : D_{2n} \rightarrow G$, given by $\varphi(r) = a$ and
      $\varphi(s) = x$ is an isomorphism. Hence $G \cong D_{2n}$. \qed      
%%%%%%%%%%%%%%%%%%%%%%%%%%%%%%%%%%%%%1.6.25%%%%%%%%%%%%%%%%%%%%%%%%%%%%%%%%%%%%%
   \item[1.6.25]  Let $n \in \Z^+$, let $r$ and $s$ be the usual generators of
                  $D_{2n}$ and let $\theta = 2\pi/n$.
                  \begin{enumerate}
                     \item Prove that the matrix
                           $\left(\begin{tabular}{@{}cr@{}}
                              $\cos\theta$ & $-\sin\theta$ \\
                              $\sin\theta$ & $\cos\theta$
                           \end{tabular}\right)$ is the matrix of the linear
                           transformation which rotates the $x$, $y$ plane about
                           the origin in a counterclockwise direction by
                           $\theta$ radians.
                     \item Prove that the map
                           $\varphi : D_{2n} \rightarrow GL_2(\R)$ defined on
                           generators by
                           $$\varphi(r) = \left(\begin{tabular}{@{}cr@{}}
                              $\cos\theta$ & $-\sin\theta$ \\
                              $\sin\theta$ & $\cos\theta$
                           \end{tabular}\right) \quad\text{and}\quad\varphi(s) =
                           \left(\begin{tabular}{@{}cc@{}}
                              0 & 1 \\
                              1 & 0
                           \end{tabular}\right)$$
                           extends to a homomorphism of $D_{2n}$ into
                           $GL_2(\R)$.
                     \item Prove that the homomorphism $\varphi$ in part (b) is
                           injective.
                  \end{enumerate}
                  
      \textbf{Proof.}
      
      \begin{enumerate}
         \item Consider $T_\theta : \R^2 \rightarrow \R^2$,
               $(x, y) \mapsto (x\cos\theta - y\sin\theta,
               x\sin\theta + y\cos\theta)$, the linear transformation which
               rotates the cartesian plane about the origin in a
               counterclockwise direction by $\theta$ radians. Consider the
               standard basis $\{(1, 0), (0, 1)\}$ for $\R^2$. A simple
               calculation will show us that the matrix of $T_\theta$ with
               respect to the standard basis is $\left(\begin{tabular}{@{}cr@{}}
                  $\cos\theta$ & $-\sin\theta$ \\
                  $\sin\theta$ & $\cos\theta$
               \end{tabular}\right)$. 
         \item It suffices to show that $\varphi(r)$ and $\varphi(s)$ satisfy
               (in $GL_2(\R)$) the relations satisfied by $r$ and $s$
               (in $D_{2n}$). It is clear that $\varphi(s)^2$ is the identity
               matrix. Also $\varphi(r)^n$ is the identity matrix because the
               least (positive) number of rotations we need to get back to the
               starting point is $n$. Finally we have that
               $$\varphi(r)\varphi(s) = \left(\begin{tabular}{@{}rc@{}}
                  $-\sin\theta$ & $\cos\theta$ \\
                  $\cos\theta$ & $\sin\theta$
               \end{tabular}\right) = \varphi(s)\varphi(r)^{-1},$$
               as desired.
         \item First note that (using induction)
               $$\varphi(r^p) = \left(\begin{tabular}{@{}cr@{}}
                  $\cos p\theta$ & $-\sin p\theta$ \\
                  $\sin p\theta$ & $\cos p\theta$
               \end{tabular}\right) \qquad \text{for all } p \in \Z.$$
               Now suppose that $\varphi(r^is^j) = \varphi(r^xs^y)$. That is,
               $\varphi(r^i)\varphi(s^j) = \varphi(r^x)\varphi(s^y)$, so that
               $\varphi(r^{i-x}) = \varphi(s^{y-j})$.

               \textbf{Case 1.} $y - j \equiv 0$ mod 2. That is $s^y = s^j$,
               so that $$\left(\begin{tabular}{@{}cr@{}}
                  $\cos [(i-x)\theta]$ & $-\sin[(i-x)\theta]$ \\
                  $\sin[(i-x)\theta]$ & $\cos[(i-x)\theta]$
               \end{tabular}\right) = \varphi(r^{i-x}) = \varphi(s^{y-j}) =
               \varphi(1) = \left(\begin{tabular}{@{}cc@{}}
                  1 & 0 \\
                  0 & 1
               \end{tabular}\right).$$
               It follows that $(i - x)\theta = 2\pi k$ for some integer $k$.
               Recall that $\theta = 2\pi/n$; thus $i - x = nk$ and we have that
               $r^{i-x} = r^{nk} = 1$, so we can conclude that $r^i = r^x$. We
               have thus shown that $r^is^j = r^xs^y$.

               \textbf{Case 2.} $y - j \equiv 1$ mod 2. Now
               $$\left(\begin{tabular}{@{}cr@{}}
                  $\cos [(i-x)\theta]$ & $-\sin[(i-x)\theta]$ \\
                  $\sin[(i-x)\theta]$ & $\cos[(i-x)\theta]$
               \end{tabular}\right) = \varphi(r^{i-x}) = \varphi(s^{y-j}) =
               \varphi(s) = \left(\begin{tabular}{@{}cc@{}}
                  0 & 1 \\
                  1 & 0
               \end{tabular}\right),$$
               a contradiction because $-1 = \sin[(i-x)\theta] = 1$. Hence the
               only possibility is Case 1, wherein we showed that
               $r^is^j = r^xs^y$, so that $\varphi$ is injective.
      \end{enumerate} \qed
%%%%%%%%%%%%%%%%%%%%%%%%%%%%%%%%%%%%%1.6.26%%%%%%%%%%%%%%%%%%%%%%%%%%%%%%%%%%%%%
   \item[1.6.26]  Let $i$ and $j$ be the generators of $Q_8$ described in
                  Section 5. Prove that the map $\varphi$ from $Q_8$ to
                  $GL_2(\C)$ defined on generators by
                  $$\varphi(i) = \left(\begin{tabular}{@{}cc@{}}
                        $\sqrt{-1}$ & 0 \\
                        0         &  $-\sqrt{-1}$
                    \end{tabular}\right) \qquad \text{and} \qquad
                    \varphi(j) = \left(\begin{tabular}{@{}cr@{}}
                        0 & $-1$ \\
                        1 & 0
                    \end{tabular}\right)$$
                  extends to a homomorphism. Prove that $\varphi$ is injective.

      \textbf{Proof.} To show that $\varphi$ extends to an homomorphism, it
      suffices to show that $\varphi(i)$ and $\varphi(j)$ satisfy
      (in $GL_2(\C)$) the relations satisfied by $i$ and $j$ (in $Q_8$). Using
      Exercise 1.5.3, we must then show that $\varphi(i)^2 = \varphi(j)^2$ and
      $\varphi(i)^4 = 1$. These equalities immediately follow using matrix
      multiplication. Now suppose that $\varphi(i^xj^y) = \varphi(i^mj^n$). Then
      it follows that $\varphi(i)^{x-m} = \varphi(j)^{n-y}$. Notice that
      $|\varphi(i)| = |\varphi(j)| = 4$ and that $\varphi(i)^p = \varphi(j)^q$
      if and only if $p \equiv q \equiv 2$ (or 0) mod 4. So we must have that
      $x - m \equiv n - y \equiv 2 $ (or 0) mod 4. In either case, it follows
      that $i^{x - m} = j^{n - y}$. That is $i^xj^y = i^mj^n$, so that
      $\varphi$ is injective. \qed
\end{enumerate}

      \section{Group Actions}
         \begin{enumerate}
%%%%%%%%%%%%%%%%%%%%%%%%%%%%%%%%%%%%%1.7.1%%%%%%%%%%%%%%%%%%%%%%%%%%%%%%%%%%%%%%
   \item[1.7.1]   Let $F$ be a field. Show that the multiplicative group of
                  nonzero elements of $F$ (denoted by $F^\times$) acts on the
                  set $F$ by $g \cdot a = ga$, where $g \in F^\times$,
                  $a \in F$ and $ga$ is the usual product in $F$ of the two
                  field elements (state clearly which axioms in the definition
                  of a field are used).
%%%%%%%%%%%%%%%%%%%%%%%%%%%%%%%%%%%%%1.7.2%%%%%%%%%%%%%%%%%%%%%%%%%%%%%%%%%%%%%%
   \item[1.7.2]   Show that the additive group $\Z$ acts on itself by
                  $z \cdot a = z + a$ for all $z, a \in \Z$.
%%%%%%%%%%%%%%%%%%%%%%%%%%%%%%%%%%%%%1.7.3%%%%%%%%%%%%%%%%%%%%%%%%%%%%%%%%%%%%%%
   \item[1.7.3]   Show that the additive group $\R$ acts on the $x, y$ plane
                  $\R \times \R$ by $r \cdot (x, y) = (x + ry, y)$.
%%%%%%%%%%%%%%%%%%%%%%%%%%%%%%%%%%%%%1.7.4%%%%%%%%%%%%%%%%%%%%%%%%%%%%%%%%%%%%%%
   \item[1.7.4]   Let $G$ be a group acting on a set $A$ and fix some $a \in A$.
                  Show that the following sets are subgroups of $G$.
                  \begin{enumerate}
                     \item the kernel of the action,
                     \item $\{g \in G : ga = a\}$---this subgroup is called the
                           \textit{stablizer} of $a$ in $G$.
                  \end{enumerate}
%%%%%%%%%%%%%%%%%%%%%%%%%%%%%%%%%%%%%1.7.5%%%%%%%%%%%%%%%%%%%%%%%%%%%%%%%%%%%%%%
   \item[1.7.5]   Prove that the kernel of an action of the group $G$ on the set
                  $A$ is the same as the kernel of the corresponding permutation
                  representation $G \rightarrow S_A$.
%%%%%%%%%%%%%%%%%%%%%%%%%%%%%%%%%%%%%1.7.6%%%%%%%%%%%%%%%%%%%%%%%%%%%%%%%%%%%%%%
   \item[1.7.6]   Prove that a group $G$ acts faithfully on a set $A$ if and
                  only if the kernel of the action is the set consisting only of
                  the identity.
%%%%%%%%%%%%%%%%%%%%%%%%%%%%%%%%%%%%%1.7.7%%%%%%%%%%%%%%%%%%%%%%%%%%%%%%%%%%%%%%
   \item[1.7.7]   Prove that in Example 2 in this section the action is
                  faithful.
%%%%%%%%%%%%%%%%%%%%%%%%%%%%%%%%%%%%%1.7.8%%%%%%%%%%%%%%%%%%%%%%%%%%%%%%%%%%%%%%
   \item[1.7.8]   Let $A$ be a nonempty set and let $k$ be a positive integer
                  with $k \le |A|$. The symmetric group $S_A$ acts on the set
                  $B$ consisting of all subsets of $A$ of cardinality $k$ by
                  $\sigma \cdot \{a_1, \ldots, a_k\} = \{\sigma(a_1), \ldots,
                   \sigma(a_k)\}$.
                  \begin{enumerate}
                     \item Prove that this is a group action.
                     \item Describe explicitly how the elements (1 2) and
                           (1 2 3) act on the six 2-element subsets of
                           $\{1, 2, 3, 4\}$.
                  \end{enumerate}
%%%%%%%%%%%%%%%%%%%%%%%%%%%%%%%%%%%%%1.7.9%%%%%%%%%%%%%%%%%%%%%%%%%%%%%%%%%%%%%%
   \item[1.7.9]   Do both parts of the preceding exercise with ``ordered
                  $k$-tuples" in place of ``$k$-element subsets," where the
                  action on $k$-tuples is defined as above but with set braces
                  replaced by parentheses (note that, for example, the 2-tuples
                  (1, 2) and (2, 1) are different even though the sets
                  $\{1, 2\}$ and $\{2, 1\}$ are the same, so the sets being
                  acted upon are different).
%%%%%%%%%%%%%%%%%%%%%%%%%%%%%%%%%%%%%1.7.10%%%%%%%%%%%%%%%%%%%%%%%%%%%%%%%%%%%%%
   \item[1.7.10]  With reference to the preceding two exercises determine:
                  \begin{enumerate}
                     \item for which values of $k$ the action of $S_n$ on
                           $k$-element subsets if faithful, and
                     \item for which values of $k$ the action of $S_n$ on
                           ordered $k$-tuples is faithful.
                  \end{enumerate}
%%%%%%%%%%%%%%%%%%%%%%%%%%%%%%%%%%%%%1.7.11%%%%%%%%%%%%%%%%%%%%%%%%%%%%%%%%%%%%%
   \item[1.7.11]  Write out the cycle decomposition of the eight permutations in
                  $S_4$ corresponding to the elements of $D_8$ given by the
                  action of $D_8$ on the vertices of a square (where the
                  vertices of the square are labelled as in Section 2).
%%%%%%%%%%%%%%%%%%%%%%%%%%%%%%%%%%%%%1.7.12%%%%%%%%%%%%%%%%%%%%%%%%%%%%%%%%%%%%%
   \item[1.7.12]  Assume $n$ is an even positive integer and show that $D_{2n}$
                  acts on the set consisting of pairs of opposite vertices of a
                  regular $n$-gon. Find the kernel of this action (label
                  vertices as usual).
%%%%%%%%%%%%%%%%%%%%%%%%%%%%%%%%%%%%%1.7.13%%%%%%%%%%%%%%%%%%%%%%%%%%%%%%%%%%%%%
   \item[1.7.13]  Find the kernel of the left regular action.
%%%%%%%%%%%%%%%%%%%%%%%%%%%%%%%%%%%%%1.7.14%%%%%%%%%%%%%%%%%%%%%%%%%%%%%%%%%%%%%
   \item[1.7.14]  Let $G$ be a group and let $A = G$. Show that if $G$ is
                  non-abelian then the maps defined by $g \cdot a = ag$ for all
                  $g, a \in G$ \textit{do not} satisfy the axions of a (left)
                  group action of $G$ on itself.
%%%%%%%%%%%%%%%%%%%%%%%%%%%%%%%%%%%%%1.7.15%%%%%%%%%%%%%%%%%%%%%%%%%%%%%%%%%%%%%
   \item[1.7.15]  Let $G$ be any group and let $A = G$. Show that the maps
                  defined by $g \cdot a = ag^{-1}$ for all $g, a \in G$
                  \textit{do} satisfy the axioms of a (left) group action of $G$
                  on itself.
%%%%%%%%%%%%%%%%%%%%%%%%%%%%%%%%%%%%%1.7.16%%%%%%%%%%%%%%%%%%%%%%%%%%%%%%%%%%%%%
   \item[1.7.16]  Let $G$ be any group and let $A = G$. Show that the maps
                  defined by $g \cdot a = gag^{-1}$ for all $g, a \in G$
                  \textit{do} satisfy the axioms of a (left) group action(this
                  action of $G$ on itself is called \textit{conjugation}).
%%%%%%%%%%%%%%%%%%%%%%%%%%%%%%%%%%%%%1.7.17%%%%%%%%%%%%%%%%%%%%%%%%%%%%%%%%%%%%%
   \item[1.7.17]  Let $G$ be a group and let $G$ act on itself by left 
                  conjugation, so each $g \in G$ maps $G$ to $G$ by
                  $$x \mapsto gxg^{-1}.$$
                  For fixed $g \in G$, prove that conjugation by $g$ is an
                  isomorphism from $G$ onto itself. Deduce that $x$ and
                  $gxg^{-1}$ have the same order for all $x \in G$ and that for
                  any subset $A$ of $G$, $|A| = |gAg^{-1}|$ (here
                  $gAg^{-1} = \{gag^{-1} : a \in A\})$.
%%%%%%%%%%%%%%%%%%%%%%%%%%%%%%%%%%%%%1.7.18%%%%%%%%%%%%%%%%%%%%%%%%%%%%%%%%%%%%%
   \item[1.7.18]  Let $H$ be a group acting on a set $A$. Prove that the
                  relation $\sim$ on $A$ defined by
                  $$a \sim b \quad \text{if and only if} \quad
                    a = hb \quad \text{for some }h \in H$$
                  is an equivalence relation. (For each $x \in A$ the
                  equivalence class of $x$ under $\sim$ is called the
                  \textit{orbit} of $x$ under the action of $H$. The orbits
                  under the action of $H$ partition the set $A$.)
%%%%%%%%%%%%%%%%%%%%%%%%%%%%%%%%%%%%%1.7.19%%%%%%%%%%%%%%%%%%%%%%%%%%%%%%%%%%%%%
   \item[1.7.19]  Let $H$ be a subgroup of the finite group $G$ and let $H$ act
                  on $G$ (here $A = G$) by left multiplication. Let $x \in G$
                  and let $\mathcal{O}$ be the orbit of $x$ under the action of
                  $H$. Prove that the map
                  $$H \rightarrow \mathcal{O}\quad \text{defined by} \quad
                    h \mapsto hx$$
                  is a bijection (hence all orbits have cardinality $|H|$). From
                  this and the preceding exercise deduce
                  $\textit{Lagrange's Theorem}:$
                  \begin{center}
                     \textit{if $G$ is a finite group and $H$ is a subgroup of
                     $G$ then $|H|$ divides $|G|$}.
                  \end{center}
%%%%%%%%%%%%%%%%%%%%%%%%%%%%%%%%%%%%%1.7.20%%%%%%%%%%%%%%%%%%%%%%%%%%%%%%%%%%%%%
   \item[1.7.20]  Show that the group of rigid motions of a tetrahedron is
                  isomorphic to a subgroup of $S_4$.
%%%%%%%%%%%%%%%%%%%%%%%%%%%%%%%%%%%%%1.7.21%%%%%%%%%%%%%%%%%%%%%%%%%%%%%%%%%%%%%
   \item[1.7.21]  Show that the group of rigid motions of a cube is isomorphic
                  to $S_4$. [This group acts on the set of four pairs of
                  opposite vertices.]
%%%%%%%%%%%%%%%%%%%%%%%%%%%%%%%%%%%%%1.7.22%%%%%%%%%%%%%%%%%%%%%%%%%%%%%%%%%%%%%
   \item[1.7.22]  Show that the group of rigid motions of an octahedron is
                  isomorphic to a subgroup of $S_4$. [This group acts on the set
                  of four pairs of opposite faces.] Deduce that the groups of
                  rigid motions of a cube and an octahedron are isomorphic.
                  (These groups are isomorphic because these solids are ``dual"
                  ---see \textit{Introduction to Geometry} by H.Coxeter, Wiley,
                  1961. We shall see later that the groups of rigid motions of
                  the dodecahedron and icosahedron are isomorphic as well---
                  these solids are also dual.)
%%%%%%%%%%%%%%%%%%%%%%%%%%%%%%%%%%%%%1.7.23%%%%%%%%%%%%%%%%%%%%%%%%%%%%%%%%%%%%%
   \item[1.7.23]  Explain why the action of the group of rigid motions of a cube
                  on the set of three pairs of opposite faces is not faithful.
                  Find the kernel of this action.
\end{enumerate}

\begin{comment}
   \chapter{Preliminaries}
      \section{Basics}
         \begin{enumerate}
   \item[]        Let $G$ be a group.
%%%%%%%%%%%%%%%%%%%%%%%%%%%%%%%%%%%Lemm1.1.1%%%%%%%%%%%%%%%%%%%%%%%%%%%%%%%%%%%%
   \item[]        \textbf{Lemma 1.1.1} Let $x \in G$ and let $m$ be an integer.
                  Then we have that
                  $$x^{m+1} = x^mx^1.$$

      \textbf{Proof.}  Consider the following cases:

      \textbf{Case 1:} \textit{$m = 0$}. It follows that
      $$x^{m+1} = x^{0+1} = x^1 = 1x^1 = x^0x^1 = x^mx^1.$$

      \textbf{Case 2:} \textit{$m$ is positive.} Then it follows that $m + 1$ is 
      positive, so that
      \begin{align*}
         x^{m + 1} &= \underbrace{x \cdot x \cdots x}_{m+1\text{ factors}} \\
            &= \underbrace{x \cdot x \cdots x}_{m\text{ factors}} \cdot x^1 \\
            &= x^mx^1.
      \end{align*}

      \textbf{Case 3:} \textit{$m$ is negative.} If $m = -1$, then we have that
      $$x^{m+1} = x^{-1+1} = x^0 = 1 = x^{-1}x^1 = x^mx^1.$$
      If $m < -1$, then $m + 1$ is negative, so that $-(m + 1) = -m - 1$ is 
      positive. Thus
      \begin{align*}
         x^mx^1 &= x^{-(-m)}x^1 \\
                &= \underbrace{x^{-1} \cdot x^{-1} \cdots x^{-1}}_{
                   -m\text{ factors}} \cdot x^1 \\
                &= \underbrace{x^{-1} \cdot x^{-1} \cdots x^{-1}}_{
                   -m-1\text{ factors}} \cdot (x^{-1} \cdot x^1) \\
                &= \underbrace{x^{-1} \cdot x^{-1} \cdots x^{-1}}_{
                   -m-1\text{ factors}} \\
                &= x^{-(-m-1)} \\
                &= x^{m+1}.
      \end{align*}

      In all cases, we can see that our assertion holds. \qed
%%%%%%%%%%%%%%%%%%%%%%%%%%%%%%%%%%%Lemm1.1.2%%%%%%%%%%%%%%%%%%%%%%%%%%%%%%%%%%%%
   \item[]        \textbf{Lemma 1.1.2} \textit{Let $x$ and $g$ be members of a 
                  group $G$, and let $n$ be a positive integer, then it follows 
                  that $(g^{-1}xg)^n = g^{-1}x^ng$.}

      \textbf{Proof.} We shall show by induction that the equation
      \begin{equation}
         (g^{-1}xg)^n = g^{-1}x^ng \label{l1_1_2_1}
      \end{equation}
      holds for every positive integer $n$. It is clear that equation
      \ref{l1_1_2_1} holds for $n = 1$. So assume that it also holds for some
      positive integer $k$. So we must now show that the equation also holds for 
      $k + 1$. Thus
      \begin{align*}
         (g^{-1}xg)^{k+1} &= (g^{-1}xg)^kg^{-1}xg &[\text{Execise 1.1.19}] \\
                     &= g^{-1}x^kgg^{-1}xg &[\text{Inductive hypothesis}] \\
                     &= g^{-1}x^kxg \\
                     &= g^{-1}x^{k+1}g,
      \end{align*}
      so that equation \eqref{l1_1_2_1} holds for $k+1$. Hence by the Principle 
      of Mathematical Induction, equation \eqref{l1_1_2_1} holds for every 
      positive integer $n$. \qed
%%%%%%%%%%%%%%%%%%%%%%%%%%%%%%%%%%%Lemm1.1.3%%%%%%%%%%%%%%%%%%%%%%%%%%%%%%%%%%%%
   \item[]        \textbf{Lemma 1.1.3} \textit{Let $x$ be an element of finite
                  order $n$ in $G$. If $x^m = 1$, then it follows that
                  $n \mid m$.}

      \textbf{Proof.} Suppose that $x^m = 1$. By the Division Algorithm, there
      exist unique integers $q$ and $r$ such that $m = qn + r$ and
      $0 \le r < n$. Now we have that
      $$1 = x^m = x^{qn+r} = x^{qn}x^r = (x^n)^qx^r = 1^qx^r = x^r.$$
      Since $|x| = n$, we cannot have $0 < r < n$; thus the only remaining
      possibility is $r = 0$, so that $n = qm$, as desired. \qed
%%%%%%%%%%%%%%%%%%%%%%%%%%%%%%%%%%%Lemm1.1.4%%%%%%%%%%%%%%%%%%%%%%%%%%%%%%%%%%%%
   \item[]        \textbf{Lemma 1.1.4} \textit{Let $(x, y)$ be an element of
                  $A \times B$ where $A$ and $B$ are groups. For any positive
                  integer $n$, we then have that $(x, y)^n = (x^n, y^n)$.}

      \textbf{Proof.} We shall induct on $n$. Our assertion clearly holds if
      $n$ is 1, so assume that it holds for some positive integer $k$. Thus we
      have that
      \begin{align*}
         (x, y)^{k+1} &= (x, y)(x, y)^k &[\text{Exercise 1.1.19}] \\
                      &= (x, y)(x^k, y^k) &[\text{Inductive hypothesis}] \\
                      &= (xx^k, yy^k) \\
                      &= (x^{k+1}, y^{k+1}). &[\text{Exercise 1.1.19}]
      \end{align*}
      The above shows that our assertion also holds for $k + 1$, so that by
      the Principle of Mathematical Induction it must holds for every integer
      $n$. \qed
%%%%%%%%%%%%%%%%%%%%%%%%%%%%%%%%%%%%%1.1.1%%%%%%%%%%%%%%%%%%%%%%%%%%%%%%%%%%%%%%
   \item[1.1.1]   Determine which of the following binary operations are
                  associative:
                  \begin{enumerate}
                     \item the operation $*$ on $\Z$ defined by $a * b = a - b$.
                     \item the operation $*$ on $\R$ defined by
                           $a * b = a + b + ab$.
                     \item the operation $*$ on $\Q$ defined by
                           $\displaystyle a * b = \frac{a + b}{5}$.
                     \item the operation $*$ on $\Z \times \Z$ defined by
                           $(a, b) * (c, d) = (ad + bc, bd)$.
                     \item the operation $*$ on $\Q - \{0\}$ defined by
                           $\displaystyle a * b = \frac{a}{b}$.
                  \end{enumerate}
                  
      \textbf{Solution.}
   
      \begin{enumerate}
         \item The binary operation $*$ on $\Z$ is not associative because
               $$(0 * 0) * 1 = -1 \neq 1 = 0 * (0 * 1).$$
         \item We claim that $*$ is associative on $\R$.
      
               \textbf{Proof.} Let $r_1, r_2, r_3 \in \R$. Then it follows that
               \begin{align*}
                  (r_1 * r_2) * r_3 &= (r_1 + r_2 + r_1r_2) * r_3 \\
                     &= (r_1 + r_2 + r_1r_2 + r_3) +
                        (r_1r_3 + r_2r_3 + r_1r_2r_3) \\
                     &= r_1 + r_2 + r_3 + r_1r_2 + r_2r_3 +
                        r_1r_3 + r_1r_2r_3 \\
                     &= (r_1 + r_2 + r_3 + r_2r_3) + r_1(r_2 + r_3 + r_2r_3) \\
                     &= r_1 + (r_2 * r_3) + r_1(r_2 * r_3) \\
                     &= r_1 * (r_2 * r_3),
               \end{align*}
               so that our claim holds. \qed
         \item The binary operation $*$ on $\Q$ is not associative because
               $$(0 * 0) * 25 = 5 \neq 1 = 0 * (0 * 25).$$
         \item We claim that $*$ is associative on $\Z \times \Z$.
      
               \textbf{Proof.} Let $(z_1, z_2)$, $(z_3, z_4)$,
               $(z_5, z_6) \in \Z \times \Z$. Then it follows that
               \begin{align*}
                  (z_1, z_2) * [(z_3, z_4) * (z_5, z_6)] 
                     &= (z_1, z_2) * [(z_3z_6 + z_4z_5, z_4z_6)] \\
                     &= (z_1z_4z_6 + z_2z_3z_6 + z_2z_4z_5, z_2z_4z_6) \\
                     &= ((z_1z_4 + z_2z_3) \cdot z_6 + z_2z_4 \cdot z_5,
                          z_2z_4 \cdot z_6) \\
                     &= (z_1z_4 + z_2z_3, z_2z_4) * (z_5, z_6) \\
                     &= [(z_1, z_2) * (z_3, z_4)] * (z_5, z_6),
               \end{align*}
               so that our claim holds. \qed
         \item The binary operation $*$ on $\Q - \{0\}$ is not associative
               because
               $$(4 * 1) * 2 = 2 \neq 8 = 4 * (1 * 2).$$
      \end{enumerate}
%%%%%%%%%%%%%%%%%%%%%%%%%%%%%%%%%%%%%1.1.2%%%%%%%%%%%%%%%%%%%%%%%%%%%%%%%%%%%%%%
   \item[1.1.2]   Decide which of the binary operations in the preceding
                  exercise are commutative.
                  
      \begin{enumerate}      
         \item The binary operation $*$ on $\Z$ is not commutative because
               $$1 * 0 = 1 \neq -1 = 0 * 1.$$
         \item The binary operation $*$ on $\R$ is commutative because addition
               and multiplication are commutative on $\R$.
         \item The binary operation $*$ on $\Q$ is commutative because addition
               is commutative on $\Q$.
         \item A quick check will show us that $*$ is commutative on
               $\Z \times \Z$. That is, for all $(z_1, z_2)$, $(z_3, z_4)$
               $\in \Z \times \Z$, we must have that
               \begin{align*}
                  (z_1, z_2) * (z_3, z_4) &= (z_1z_4 + z_2z_3, z_2z_4) \\
                                          &= (z_3z_2 + z_4z_1, z_4z_2) \\
                                          &= (z_3, z_4) * (z_1, z_2).
               \end{align*}
         \item The binary operation $*$ on $\Q - \{0\}$ is not commutative
               because
               $$1 * 2 = \frac{1}{2} \neq \frac{2}{1} = 2 * 1.$$
      \end{enumerate}
%%%%%%%%%%%%%%%%%%%%%%%%%%%%%%%%%%%%%%1.3%%%%%%%%%%%%%%%%%%%%%%%%%%%%%%%%%%%%%%%
   \item[1.1.3]   Prove that addition of residue classes in $\Z/n\Z$ is
                  associative (you may assume it is well defined).
                  
      \textbf{Proof.} Fix $n \in \Z^+$. Consider $\overline{a}$, $\overline{b}$,
      and $\overline{c}$ in $\Z/n\Z$. By Theorem 3, Pg. 9, we have that
      \begin{align*}
         \overline{a} + (\overline{b} + \overline{c})
            &= \overline{a} + \overline{b + c} \\
            &= \overline{a + b + c} \\
            &= \overline{a + b} + \overline{c} \\
            &= (\overline{a} + \overline{b}) + \overline{c},
      \end{align*}
      so that addition of residue classes in $\Z/n\Z$ is associative. \qed
%%%%%%%%%%%%%%%%%%%%%%%%%%%%%%%%%%%%%%1.4%%%%%%%%%%%%%%%%%%%%%%%%%%%%%%%%%%%%%%%
   \item[1.1.4]   Prove that multiplication of residue classes in $\Z/n\Z$ is
                  associative (you may assume it is well defined).
                  
      \textbf{Proof.} Fix $n \in \Z^+$. Consider $\overline{a}$, $\overline{b}$,
      and $\overline{c}$ in $\Z/n\Z$. By Theorem 3, Pg. 9, we have that
      \begin{align*}
         \overline{a} \cdot (\overline{b} \cdot \overline{c})
            &= \overline{a} \cdot \overline{bc} \\
            &= \overline{abc} \\
            &= \overline{ab} \cdot \overline{c} \\
            &= (\overline{a} \cdot \overline{b}) \cdot \overline{c},
      \end{align*}
      so that multiplication of residue classes in $\Z/n\Z$ is associative. \qed
%%%%%%%%%%%%%%%%%%%%%%%%%%%%%%%%%%%%%%1.5%%%%%%%%%%%%%%%%%%%%%%%%%%%%%%%%%%%%%%%
   \item[1.1.5]   Prove that for all $n > 1$ that $\Z/n\Z$ is not a group under
                  multiplication of residue classes.
                  
      \textbf{Proof.} Let $n$ be positive integer greater than 1. It follows
      that $\Z/n\Z$ is not a group under multiplication because $\overline{0}$
      has no multiplicative inverse. \qed
%%%%%%%%%%%%%%%%%%%%%%%%%%%%%%%%%%%%%%1.6%%%%%%%%%%%%%%%%%%%%%%%%%%%%%%%%%%%%%%%
   \item[1.1.6]   Determine which of the following sets are groups under
                  addition:
                  \begin{enumerate}
                     \item the set of rational numbers (including $0 = 0/1$) in
                           lowest terms whose denominators are odd.
                     \item the set of rational numbers (including $0 = 0/1$) in
                           lowest terms whose denominators are even.
                     \item the set of rational numbers of absolute value $< 1$.
                     \item the set of rational numbers of absolute value $\ge 1$
                           together with 0.
                     \item the set of rational numbers with denominators equal
                           to 1 or 2.
                     \item the set of rational numbers with denominators equal
                           to 1, 2, or 3.
                  \end{enumerate}

      \textbf{Solution.}

      \begin{enumerate}
         \item We claim that the set
               $$S = \left\{\frac{a}{b} \in \Q : b \text{ is odd} \text{ and }
                 \gcd(a, b) = 1\right\},$$
               is a group under addition.

               \textbf{Proof.} First we must show that $S$ is closed under 
               addition. Notice that $S$ is nonempty since it contains 7/5, so 
               let $r, s \in S$. By definition of $S$, we have that
               $r = a_1/b_1$ and $s = a_2/b_2$ for some integers $a_1$ and
               $a_2$, and nonzero integers $b_1$ and $b_2$, where $b_1$ and
               $b_2$ are odd and $\gcd(a_1, b_1) = \gcd(a_2, b_2) = 1$.
               It follows that
               \begin{align*}
                  r + s &= \frac{a_1}{b_1} + \frac{a_2}{b_2} \\
                        &= \frac{a_1b_2 + a_2b_1}{b_1b_2}.
               \end{align*}

               Since $b_1$ and $b_2$ are both odd, it must necessarily be the 
               case that $b_1b_2$ is also odd. In order words, $b_1b_2$ contains 
               no factor of 2, so that if we reduce $r + s$ to its lowest term, 
               the denominator of this lowest term will still be odd. Hence
               $r + s \in S$, so that $S$ is closed under addition. To complete 
               the proof we must now show that $S$ satisfies the group axioms. 
               We observe that $0/1$ is the identity element in $S$. Also, it is 
               clear that for all $s \in S$, we have $-s \in S$, so that every 
               element of $S$ has an inverse under addition. Since
               $S \subseteq \Q$, and since $\Q$ is associative under addition, 
               it follows that $S$ is also associative under addition. Thus $S$ 
               satisfies the group axioms, so that $(S, +)$ is a group. \qed
         \item The set
               $$S = \left\{\frac{a}{b} \in \Q : b \text{ is even} \text{ and }
                 \gcd(a, b) = 1\right\},$$
               is not a group under addition because it is not closed. Indeed,
               for $3/14 \in S$, we have $3/14 + 3/14 = 3/7 \notin S$.
         \item The set
               $$S = \left\{\frac{a}{b} \in \Q :
                     \left|\frac{a}{b}\right| < 1\right\},$$
               is not a group under addition because it is not closed. Indeed,
               for $9/10 \in S$, we have $9/10 + 9/10 = 18/10 \notin S$.
         \item The set
               $$S = \left\{\frac{a}{b} \in \Q : a = 0 \text{ or }
                     \left|\frac{a}{b}\right| \ge 1\right\},$$
               is not a group under addition because it is not closed. Indeed,
               for $-11/10, 10/10 \in S$, we have
               $-11/10 + 10/10 = -1/10 \notin S$.
         \item We claim that the set
               $$S = \left\{\frac{a}{b} \in \Q : b = 1 \text{ or }
                 b = 2\right\},$$
               is a group under addition.

               \textbf{Proof.} It is clear that 0 is the identity for $S$ under
               addition, that $S$ is associative under addition (because
               $S \subset \Q$ and $\Q$ is associative under addition, and that
               the inverse of an element in $S$ is its additive inverse in $\Q$.
               So to complete the proof, we need only show that $S$ is closed
               under addition. Let $a_1/b_1, a_2/b_2 \in \Q$. By observation, we
               note that $a_1/b_1 + a_2/b_2$ must have a denominator of 1 or 2,
               so that it is in $S$. Thus $S$ is closed under addition. \qed
         \item The set
               $$S = \left\{\frac{a}{b} \in \Q : b \in {1, 2, 3} \right\},$$
               is not a group under addition because it is not closed. Indeed,
               for $1/2, 1/3 \in S$, we have $1/2 + 1/3 = 5/6 \notin S$.
      \end{enumerate}
%%%%%%%%%%%%%%%%%%%%%%%%%%%%%%%%%%%%%%1.7%%%%%%%%%%%%%%%%%%%%%%%%%%%%%%%%%%%%%%%
   \item[1.1.7]   Let $G = \{x \in \R : 0 \le x < 1\}$ and for $x, y \in G$ let
                  $x * y$ be the fractional part of $x + y$ (i.e.,
                  $x * y = x + y = [x + y]$ where $[a]$ is the greatest integer
                  less than or equal to $a$). Prove that $*$ is a well defined
                  binary operation on $G$ and that $G$ is an abelian group under
                  $*$ (called the \textit{real numbers mod }1).
                  
      \textbf{Proof.} The set $G$ is clearly non-empty, so consider
      $x, y, z \in G$. To show that $G$ is a group, we shall now prove that it 
      is well defined, associative, has an identity, and is closed under taking
      inverses.

      \textbf{Well Defined:} To show that $*$ is well defined is tantamount to
      showing that $G$ is closed under $*$.  By definition, we have that
      $0 \le x < 1$ and $0 \le y < 1$, so that $0 \le x + y < 2$. If
      $0 \le x + y < 1$, so that $[x + y] = 0$, then we have that
      $$0 \le x + y = x + y - [x + y] = x * y = x + y < 1.$$
      However if $1 \le x + y < 2$, so that $[x + y] = 1$ and
      $0 \le x + y - 1 < 1$, we must have that
      $$0 \le x + y - 1 = x + y - [x + y] = x * y = x + y - 1 < 1.$$
      In either case, we have $0 \le x * y < 1$; i.e. $x * y \in G$, so that $G$ 
      is closed under $*$. Also we have that
      $$x * y = x + y - [x + y] = y + x - [y + x],$$
      so that $G$ is abelian.

      \textbf{Associativity:} We have that
      \begin{align*}
         x * (y * z) &= x * (y + z - [y + z]) \\
              &= x + y + z - [y + z] - [x + y + z - [y + z]], \text{ and} \\ \\
         (x * y) * z &= (x + y - [x + y]) * z \\
                     &=  x + y + z - [x + y] - [x + y + z - [x + y]].
      \end{align*}
      By definition, we have that $0 \le x < 1$, $0 \le y < 1$, and
      $0 \le z < 1$, so that $0 \le x + y < 2$ and $0 \le y + z < 2$. Let us 
      now investigate the following possible cases:

      \textit{Case 1:} \textit{$0 \le x + y  < 1$ and $0 \le y + z < 1$}. That
      is $[x + y] = [y + z] = 0$. It then follows that
      $$x * (y * z) = (x * y) * z = x + y + z - [x + y + z].$$

      \textit{Case 2:} \textit{$1 \le x + y  < 2$ and $1 \le y + z < 2$}. That
      is $[x + y] = [y + z] = 1$. It then follows that
      $$x * (y * z) = (x * y) * z = x + y + z - 1 - [x + y + z - 1].$$

      \textit{Case 3:} \textit{$0 \le x + y  < 1$ and $1 \le y + z < 2$}. That
      is $[x + y] = 0$, and $[y + z] = 1$. It then follows that
      $$(x * y) * z = x + y + z - [x + y + z].$$
      Since $0 \le x + y < 1$ and $0 \le z < 1$, we must have that
      $0 \le x + y + z < 2$. Similarly, since $1 \le y + z < 2$ and
      $0 \le x < 1$, we must have that $1 \le x + y + z < 3$, and since we 
      already showed that $0 \le x + y + z < 2$, it follows that
      $1 \le x + y + z < 2$. Hence $[x + y + z] = 1$. We can then conclude that 
      $(x * y) * z = x +y + z - 1$. Now we have that
      $$x * (y * z) = x + y + z - 1 - [x + y + z - 1].$$
      We already showed that $1 \le x + y + z < 2$; thus,
      $0 \le x + y + z - 1 < 1$, so that $[x + y + z - 1] = 0$; that is,
      $$x * (y * z) = x + y + z - 1 = (x * y) * z.$$
   
      \textit{Case 4:} \textit{$1 \le x + y  < 2$ and $0 \le y + z < 1$}. Apply
      Case 3, with the roles of $x + y$ and $y + z$ interchanged.

      We have thus shown that in all possible cases, we have
      $$x * (y * z) = (x * y) * z,$$
      so that $G$ is associative under $*$.


      \textbf{Identity:} We observe that $0 \in G$ is the identity element since
      $$x * 0 = x + 0 - [x + 0] = x - [x] = x - 0 = x.$$

      \textbf{Inverse:} Suppose $x \neq 0$, so that $0 < x < 1$, and thus
      $0 < 1 - x < 1$; that is $1 - x \in G$. It follows that
      $$x * (1 - x) = x + (1 - x) + [x + (1 - x)] = 1 - 1 = 0,$$
      so that $1 - x$ is the inverse of $x \in G$, with $x \neq 0$. Clearly, the 
      inverse of 0 is 0. \\

      We can now conclude that $(G, *)$ is a group. \qed
%%%%%%%%%%%%%%%%%%%%%%%%%%%%%%%%%%%%%%1.8%%%%%%%%%%%%%%%%%%%%%%%%%%%%%%%%%%%%%%%
   \item[1.1.8]   Let $G = \{z \in \C : z^n = 1 \text{ for some } n \in \Z^+\}$.
                  \begin{enumerate}
                     \item Prove that $G$ is a group under multiplication
                           (called the group of \textit{roots of unity} in
                           $\C$).
                     \item Prove that $G$ is not a group under addition.
                  \end{enumerate}
                  
      \textbf{Proof.}
      
      \begin{enumerate}
         \item We observe that 1 is the identity element of $G$, so that $G$ is
               not empty. So let $x, y, z \in G$.
               
               \textbf{Closure:} By definition, there exist positive integers
               $m$ and $n$ such that $x^m = y^n = 1$. Thus
               $(xy)^{mn} = (x^m)^n(y^n)^m = 1^n1^m = 1$. This says that $G$ is
               closed under multiplication.
               
               \textbf{Associativity:} Since $\C$ is associative under
               multiplication and since $G \subseteq \C$, it follows that $G$ is
               associative under multiplication.
               
               \textbf{Identity:} As state above, the identity of $G$ is clearly
               1.
               
               \textbf{Inverse:} Notice that since
               $(x^{m - 1})^m = (x^m)^{m - 1} = 1$, we must have that
               $x^{m - 1} \in G$. Thus we have $x^{m - 1}x = x^m = 1$; i.e., the
               inverse of $x$ is $x^{m - 1}$.
               
               We have thus shown that $G$ is a group under multiplication. \qed
         \item $G$ is not a group under addition because it is not closed under
               addition. In particular, we have $1 \in G$, but
               $1 + 1 = 2 \notin G$ because $2^n \neq 1$ for any positive
               integer.
      \end{enumerate}
%%%%%%%%%%%%%%%%%%%%%%%%%%%%%%%%%%%%%%1.9%%%%%%%%%%%%%%%%%%%%%%%%%%%%%%%%%%%%%%%
   \item[1.1.9]   Let $G = \{a + b\sqrt{2} \in \R : a, b \in \Q\}$.
                  \begin{enumerate}
                     \item Prove that $G$ is a group under addition.
                     \item Prove that the nonzero elements of $G$ are a group 
                           under multiplication. [``Rationalize the
                           denominators" to find multiplicative inverse.]
                  \end{enumerate}
                  
      \textbf{Proof.}
      
      \begin{enumerate}
         \item \textbf{Closure:} $G$ is clearly nonempty, so let $x, y \in G$.
               By definition of $G$, it follows that $x = a_1 + b_1\sqrt{2}$ and
               $y = a_2 + b_2\sqrt{2}$ for some rational numbers $a_1$, $b_1$,
               $a_2$, and $b_2$. Thus
               $$x + y = (a_1 + a_2) + (b_1 + b_2)\sqrt{2} \in G,$$
               so that $G$ is closed under addition.
               
               \textbf{Associativity:} Since $\R$ is associative under addition
               and since $G \subseteq \R$, it follows that $G$ is associative
               under addition.
               
               \textbf{Identity:} The identity of $G$ is 0.
               
               \textbf{Inverse:} For an element $x = a_1 + b_1\sqrt{2} \in G$,
               the additive inverse of $x$ is $-a_1 - b_1\sqrt{2} \in G$.
               
               We have thus shown that $G$ is a group under addition. \qed
         \item Let $G^{\times}$ denote the set of nonzero elements of $G$.
         
               \textbf{Closure:} Let $x, y \in G^{\times}$. By definition of
               $G$, it follows that $x = a_1 + b_1\sqrt{2}$ and
               $y = a_2 + b_2\sqrt{2}$ for some rational numbers $a_1$, $b_1$,
               $a_2$, and $b_2$, with $a_1$ and $b_1$ not both zero and $a_2$
               and $b_2$ not both zero. Thus
               $$xy = (a_1a_2 + 2b_1b_2) + (a_1b_2 + a_2b_1)\sqrt{2}.$$
               Since neither $x$ nor $y$ is zero, it must be the case that $xy$
               is not zero, so that $G^{\times}$ is closed under multiplication.
               
               \textbf{Associativity:} Since $\R$ is associative under
               multiplication and since $G^{\times} \subseteq \R$, it follows
               that $G^{\times}$ is associative under multiplication.
               
               \textbf{Identity:} The element $1 = 1 + 0\sqrt{2} \in G^{\times}$
               is the identity of $G^{\times}$.
               
               \textbf{Inverse:} Let $x = a_1 + b_1\sqrt{2} \in G^{\times}$.
               Since $x \neq 0$, the real number $1/x$ exists, and we have that
               $$\frac{1}{x} = \frac{1}{a_1 + b_1\sqrt{2}}
                 \frac{a_1 - b_1\sqrt{2}}{a_1 - b_1\sqrt{2}} =
                 \left(\frac{a_1}{{a_1}^2 - 2{b_1}^2} -
                 \frac{b_1}{{a_1}^2 - 2{b_1}^2}\sqrt{2}\right) \in G^{\times}.
               $$
               
               Since $1/x \in G^{\times}$ and since $x \cdot 1/x = 1$, we have
               that $1/x$ is the multiplicative inverse of $x$.
               
               We have thus shown that $G^{\times}$ is a group under
               multiplication. \qed
      \end{enumerate}
%%%%%%%%%%%%%%%%%%%%%%%%%%%%%%%%%%%%%%1.10%%%%%%%%%%%%%%%%%%%%%%%%%%%%%%%%%%%%%%
   \item[1.1.10]  Prove that a finite group is abelian if and only if its group
                  table is a symmetric matrix.
                  
      \textbf{Proof.} Let $G$ be a group such that $|G| = n \in \Z^+$, and let
      $(a_{ij})$ denote the matrix of the group table of $G$. Since $G$ is
      finite, we can enumerate the elements of $G$ like so:
      $$G = \{g_1, g_2, \ldots, g_n\}.$$      
      $(\Leftarrow)$ Suppose that $(a_{ij})$ is a symmetric matrix. Let
      $a, b \in G$. Then we have that $a = g_r$ and $b = g_s$ for some
      $r, s \in \{1, 2, \ldots, n\}$. Since $(a_{ij})$ is symmetric, we must
      have that
      $$ab = g_rg_s = a_{rs} = a_{sr} = g_sg_r = ba,$$
      so that $G$ is abelian.
      
      $(\Rightarrow)$ Now suppose that $G$ is abelian. Consider
      $a_{rs} \in (a_{ij})$. It follows that
      $$a_{rs} = g_rg_s = g_sg_r = a_{sr},$$
      so that $(a_{ij})$ is symmetric. \qed      
%%%%%%%%%%%%%%%%%%%%%%%%%%%%%%%%%%%%%%1.11%%%%%%%%%%%%%%%%%%%%%%%%%%%%%%%%%%%%%%
   \item[1.1.11]  Find the orders of each element of the additive group
                  $\Z/12\Z$.
                  
      \textbf{Solution.} The orders of the elements $\overline{0}$,
      $\overline{1}$, $\overline{2}$, $\overline{3}$, $\overline{4}$,
      $\overline{5}$, $\overline{6}$, $\overline{7}$, $\overline{8}$,
      $\overline{9}$, $\overline{10}$, and $\overline{11}$ in $\Z/12\Z$ are
      1, 12, 6, 4, 3, 12, 2, 12, 3, 4, 6, and 12.
%%%%%%%%%%%%%%%%%%%%%%%%%%%%%%%%%%%%%%1.12%%%%%%%%%%%%%%%%%%%%%%%%%%%%%%%%%%%%%%
   \item[1.1.12]  Find the orders of the following elements of the
                  multiplicative group $(\Z/12\Z)^\times: \overline{1},
                  \overline{-1}, \overline{5}, \overline{7}, \overline{-7}, 
                  \overline{13}$.
                  
      \textbf{Solution.} The orders of the elements $\overline{1}$,
      $\overline{-1}$, $\overline{5}$, $\overline{7}$, $\overline{-7}$,
      $\overline{13}$ in $(\Z/12\Z)^\times$ are 1, 11, 5, 7, 5, and 13.
%%%%%%%%%%%%%%%%%%%%%%%%%%%%%%%%%%%%%%1.13%%%%%%%%%%%%%%%%%%%%%%%%%%%%%%%%%%%%%%
   \item[1.1.13]  Find the orders of the following elements of the additive
                  group $\Z/36\Z: \overline{1}, \overline{2}, \overline{6}, 
                  \overline{9}, \overline{10}, \overline{12}, \overline{-1}, 
                  \overline{-10}, \overline{-18}$.
                  
      \textbf{Solution.} The orders of the elements $\overline{1}$,
      $\overline{2}$, $\overline{6}$, $\overline{9}$, $\overline{10}$,
      $\overline{12}$, $\overline{-1}$, $\overline{-10}$, and $\overline{-18}$
      in $\Z/36\Z$ are 1, 18, 6, 4, 18, 3, 36, 18, and 2.
%%%%%%%%%%%%%%%%%%%%%%%%%%%%%%%%%%%%%%1.14%%%%%%%%%%%%%%%%%%%%%%%%%%%%%%%%%%%%%%
   \item[1.1.14]  Find the orders of the following elements of the
                  multiplicative group $(\Z/36\Z)^\times: \overline{1},
                  \overline{-1}, \overline{5}, \overline{13}, \overline{-13},
                  \overline{17}$.
                  
      \textbf{Solution.} The orders of the elements $\overline{1}$,
      $\overline{-1}$, $\overline{5}$, $\overline{13}$, $\overline{-13}$,
      $\overline{17}$ in $(\Z/36\Z)^\times$ are 1, 35, 29, 25, 11, and 17.
%%%%%%%%%%%%%%%%%%%%%%%%%%%%%%%%%%%%%%1.15%%%%%%%%%%%%%%%%%%%%%%%%%%%%%%%%%%%%%%
   \item[1.1.15]  Prove that $(a_1a_2\cdots a_n)^{-1} =
                  {a_n}^{-1}{a_{n-1}}^{-1}\cdots {a_1}^{-1}$ for all
                  $a_1, a_2, \ldots, a_n \in G$.
                  
      \textbf{Proof.} We shall proceed by induction on $n$. The statement is
      trivial for $n = 1$. So assume that it also holds for some positive
      integer $k$. Let $b = a_1a_2\cdots a_k$. It then follows that
      \begin{align*}
         (a_1a_2\cdots a_ka_{k+1})^{-1} &= (b \cdot a_{k+1})^{-1} \\
            &= {a_{k+1}}^{-1}b^{-1} &[\text{By Proposition 1 (4)}] \\
            &= {a_{k+1}}^{-1}{a_k}^{-1}\cdots {a_1}^{-1}.
                  &[\text{Inductive hypothesis}]
      \end{align*}
      That is, our statement holds for $k + 1$, so that, by the Principle of
      Mathematical Induction, it holds for each positive integer $n$. \qed
%%%%%%%%%%%%%%%%%%%%%%%%%%%%%%%%%%%%%%1.16%%%%%%%%%%%%%%%%%%%%%%%%%%%%%%%%%%%%%%
   \item[1.1.16]  Let $x$ be an element of $G$. Prove that $x^2 = 1$ if and only
                  if $|x|$ is either 1 or 2.
                  
      \textbf{Proof.}
      
      $(\Leftarrow)$ Suppose that $x^2 = 1$. Now if $|x| > 2$, then by
      definition, $x^2 \neq 1$. The only remaining possibilities are $|x| = 1$
      or $|x| = 2$.
      
      $(\Rightarrow)$ Suppose that $|x| = 1$ or $|x| = 2$. It immediately
      follows that $x^2 = 1$. \qed
%%%%%%%%%%%%%%%%%%%%%%%%%%%%%%%%%%%%%%1.17%%%%%%%%%%%%%%%%%%%%%%%%%%%%%%%%%%%%%%
   \item[1.1.17]  Let $x$ be an element of $G$. Prove that if $|x| = n$ for some
                  positive integer $n$ then $x^{-1} = x^{n-1}$.
                  
      \textbf{Proof.} Suppose that $|x| = n \in \Z^+$. By Exercise 1.1.18, it
      follows that $x^{n-1}x^1 = x^{n-1+1} = x^n = 1$, so that
      $x^{-1} = x^{n-1}$. \qed      
%%%%%%%%%%%%%%%%%%%%%%%%%%%%%%%%%%%%%%1.18%%%%%%%%%%%%%%%%%%%%%%%%%%%%%%%%%%%%%%
   \item[1.1.18]  Let $x$ and $y$ be elements of $G$. Prove that $xy = yx$ if
                  and only if $y^{-1}xy =x$ if and only if $x^{-1}y^{-1}xy = 1$.
                  
      \textbf{Proof.} First assume that $xy = yx$. We then have that
      $yx = xy = 1xy = yy^{-1}xy$, so that $x = y^{-1}xy$ by left cancellation.
      Now assume that $y^{-1}xy = x$. Thus
      $x1 = x = y^{-1}xy = 1y^{-1}xy = xx^{-1}y^{-1}xy$, so that
      $1 = x^{-1}y^{-1}xy$ by left cancellation. Finally assume that
      $x^{-1}y^{-1}xy = 1$. Multiplying on the left by $yx$ will yield the
      equation $xy = yx$. \qed
%%%%%%%%%%%%%%%%%%%%%%%%%%%%%%%%%%%%%%1.19%%%%%%%%%%%%%%%%%%%%%%%%%%%%%%%%%%%%%%
   \item[1.1.19]  Let $x \in G$ and let $a, b \in \Z^+$.
                  \begin{enumerate}
                     \item Prove that $x^{a+b} = x^ax^b$.
                     \item Prove that $(x^a)^b = x^{ab}$.
                     \item Prove that $(x^a)^{-1} = x^{-a}$.
                     \item Establish part (a) for arbitrary integers $a$ and $b$
                           (positive, negative or zero).
                     \item Establish part (b) for arbitrary integers $a$ and $b$
                           (positive, negative or zero).
                  \end{enumerate}
               
      \textbf{Proof.}
      
      \begin{enumerate}
         \item We have that
               \begin{align*}
                  x^{a+b} &= \underbrace{xx\cdot x}_{a+b \text{ factors}} \\
                          &= \underbrace{xx\cdot x}_{a \text{ factors}}\mbox{ }
                             \underbrace{xx\cdot x}_{b \text{ factors}} \\
                          &= x^ax^b.
               \end{align*} \qed
         \item We have that
               \begin{align*}
                  (x^a)^b &= (\underbrace{xx\cdot x}_{a \text{ factors}})^b \\
                          &= \underbrace{xx\cdot x}_{ab \text{ factors}} \\
                          &= x^{ab}.
               \end{align*} \qed
         \item We have
               \begin{align*}
                  (x^a)^{-1}
                     &= (\underbrace{xx\cdot x}_{a \text{ factors}})^{-1} \\
                     &= \underbrace{x^{-1}x^{-1}\cdot x^{-1}}_{
                           a \text{ factors}} &[\text{Exercise 1.1.15}] \\
                     &= x^{-a}.
               \end{align*} \qed
         \item Now suppose that $a$ is an integer and $b$ is a positive integer.
               We shall induct on $b$ to show that
               \begin{equation}
                  x^{a+b} = x^ax^b. \label{1_1_19_1}
               \end{equation}
               By Lemma 1.1.1, \eqref{1_1_19_1} holds if $b$ equals 1. So assume
               that it also holds for some positive integer $k$. We now have
               that
               \begin{align*}
                  x^ax^{k+1} &= x^ax^kx^1 &[\text{Lemma 1.1.1}] \\
                             &= (x^ax^k)x^1 \\
                             &= x^{a+k}x^1 &[\text{Inductive hypothesis}] \\
                             &= x^{(a+k)+1} &[\text{Lemma 1.3.2}] \\
                             &= x^{a+(k+1)}, &[\text{Associativity of addition}]
               \end{align*}
               so that \eqref{1_1_19_1} holds for $k + 1$, and hence, by the 
               Principle of Mathematical Induction, it holds for each positive
               integer $n$. \\

               If $a$ is 0 or $b$ is 0, then Lemma 1.1.1 tells us that
               \eqref{1_1_19_1} holds, so the only remaining possibility is $a$ 
               and $b$ are negative.\footnote{If $a$ is positive and $b$ is
               negative, then interchange the roles of $a$ and $b$ in the 
               induction proof.} Now suppose that $a$ and $b$ are negative.
               Hence
               \begin{align*}
                  x^ax^b &= x^{-(-a)}x^{-(-b)} \\
                     &= (x^{-1})^{-a}(x^{-1})^{-b} &[\text{Definition}] \\
                     &= (x^{-1})^{(-a + (-b))} &[\text{Part (a)}] \\
                     &= x^{-(-a + (-b))} &[\text{Definition}] \\
                     &= x^{a+b}.
               \end{align*}

               Combining this result with part (a), we thus shown that
               \eqref{1_1_19_1} holds for all integers $a$ and $b$. \qed
         \item It is clear that part (b) holds if $a$ is 0 or $b$ is 0, so let
               us complete the proof for arbritrary integers $a$ and $b$.

               \textbf{Case 1:} \textit{$a$ is positive and $b$ is negative}. 
               Hence
               \begin{align*}
                  (x^a)^b &= (x^a)^{-(-b)} \\
                          &= [(x^a)^{-1}]^{-b} &[\text{Definition}] \\
                          &= (x^{-a})^{-b} &[\text{Part (c)}] \\
                          &= [(x^{-1})^a]^{-b} &[\text{Definition}] \\
                          &= (x^{-1})^{-ab} &[\text{Part (b)}] \\
                          &= x^{-(-ab)} &[\text{Definition}] \\
                          &= x^{ab}.
               \end{align*}

               \textbf{Case 2:} \textit{$a$ and $b$ are negative}. Thus
               \begin{align*}
                  (x^a)^b &= [x^{-(-a)}]^b \\
                          &= [(x^{-1})^{-a}]^b &[\text{Definition}] \\
                          &= (x^{-1})^{-ab} &[\text{Case 1}] \\
                          &= [(x^{-1})^{-1}]^{ab}. &[\text{Definition}] \\
                          &= x^{ab}. &[\text{Proposition 1 (3)}]
               \end{align*}

               \textbf{Case 3:} \textit{$a$ is negative and $b$ is positive}. 
               Thus
               \begin{align*}
                  (x^a)^b &= [x^{-(-a)}]^b \\
                          &= [(x^{-1})^{-a}]^b &[\text{Definition}] \\
                          &= (x^{-1})^{-ab} &[\text{Case 1}] \\
                          &= x^{-(-ab)} &[\text{Definition}] \\
                          &= x^{ab}.
               \end{align*}

               Combining these results with part (a), we can conclude that
               $(x^a)^b = x^{ab}$ holds for all integers $a$ and $b$ and
               $x \in G$. \qed
      \end{enumerate}
%%%%%%%%%%%%%%%%%%%%%%%%%%%%%%%%%%%%%%1.20%%%%%%%%%%%%%%%%%%%%%%%%%%%%%%%%%%%%%%
   \item[1.1.20]  For $x$ an element in $G$ show that $x$ and $x^{-1}$ have the
                  same order.

      \textbf{Proof.}

      \textbf{Case 1:} \textit{$|x| = n \in \Z^+$}. Since
      $(x^{-1})^n = (x^n)^{-1} = 1^{-1} = 1$, it follows that $|x^{-1}| \le n$,
      so suppose to the contrary that $|x^{-1}| = m < n$. Then we have that
      $$x^m = [(x^{-1})^{-1}]^m = [(x^{-1})^m]^{-1} = 1^{-1} = 1,$$
      a contradiction, so that $|x^{-1}| = n = |x|$.

      \textbf{Case 2:} \textit{$|x| = +\infty$}. Suppose to the contrary that
      $|x^{-1}| = n \in \Z^+$. As we argued in Case 1, it must be the case that
      $x^n = 1$, a contradiction. Thus $|x| = +\infty = |x^{-1}|$. \qed
%%%%%%%%%%%%%%%%%%%%%%%%%%%%%%%%%%%%%%1.20%%%%%%%%%%%%%%%%%%%%%%%%%%%%%%%%%%%%%%
   \item[1.1.21]  Let $G$ be a finite group and let $x$ be an element of $G$ of
                  order $n$. Prove that if $n$ is odd, then $x = (x^2)^k$ for
                  some $k$.

      \textbf{Proof.} Suppose that $n$ is odd. We can then write $n = 2k + 1$
      for some nonnegative integer $k$. By supposition, we have that
      $xx^{2k} = x^{2k+1} = 1 = x^{-2k}x^{2k}$, so that by right cancellation,
      we can conclude that $x = x^{-2k} = (x^2)^{-k}$. \qed
%%%%%%%%%%%%%%%%%%%%%%%%%%%%%%%%%%%%%%1.22%%%%%%%%%%%%%%%%%%%%%%%%%%%%%%%%%%%%%%
   \item[1.1.22]  If $x$ and $g$ are elements of the group $G$, prove that
                  $|x| = |g^{-1}xg|$. Deduce that $|ab| = |ba|$ for all
                  $a, b \in G$.

      \textbf{Proof.} Let $x, g \in G$.

      \textbf{Case 1:} \textit{$|x| = n \in \Z^+$}. By Lemma 1.1.2, it follows
      that $(g^{-1}xg)^n = g^{-1}x^ng = g^{-1}g =1$, so that $|g^{-1}xg| \le n$,
      so suppose to the contrary that $|g^{-1}xg| = m < n$. Then we have that
      $$g^{-1}1g = 1 = (g^{-1}xg)^m = g^{-1}x^mg,$$
      so that $x^m = 1$ by left and right cancellations, a contradiction; thus,  
      $|g^{-1}xg| = n = |x|$.

      \textbf{Case 2:} \textit{$|x| = +\infty$}. Suppose to the contrary that
      $|g^{-1}xg| = n \in \Z^+$. As we argued in Case 1, it must then be the 
      case that $x^n = 1$, a contradiction. Thus $|x| = +\infty = |g^{-1}xg|$.

      Now consider $a, b \in G$. Set $x = ab$ and $g = a$. Since 
      $|x| = |g^{-1}xg|$, it follows that $|ab| = |a^{-1}aba| = |ba|$. \qed
%%%%%%%%%%%%%%%%%%%%%%%%%%%%%%%%%%%%%%1.23%%%%%%%%%%%%%%%%%%%%%%%%%%%%%%%%%%%%%%
   \item[1.1.23]  Suppose $x \in G$ and $|x| = n < \infty$. If $n = st$ for some
                  positive integers $s$ and $t$, prove that $|x^s| = t$.

      \textbf{Proof.} Suppose $n = st$ for some positive integers $s$ and $t$.
      By supposition, we have that $1 = x^n = x^{st} = (x^s)^t$; i.e.,
      $|x^s| \le t$. Suppose to the contrary that $|x^s| = m < t$. Then we have
      that $1 = (x^s)^m = x^{sm}$. Since $0 < m < t$, it follows that
      $0 < sm < st = n$. However $|x| = n$ and we just showed that $x^{sm} = 1$, 
      so that we have a contradiction. Hence we can conclude that $|x^s| = |t|$.
      \qed
%%%%%%%%%%%%%%%%%%%%%%%%%%%%%%%%%%%%%%1.24%%%%%%%%%%%%%%%%%%%%%%%%%%%%%%%%%%%%%%
   \item[1.1.24]  If $a$ and $b$ are \textit{commuting} elements of $G$, prove 
                  that $(ab)^n = a^nb^n$ for all $n \in \Z$. [Do this by 
                  induction for positive $n$ first.]

      \textbf{Proof.} Let $R(n)$ be the statement that $(ab)^n = a^nb^n$, for
      commuting elements $a$ and $b$.
               
      We now want to show using induction that $R(n)$ holds for every positive 
      integer $n$. It is clear that $R(1)$ is true. So suppose that $R(k)$ is 
      true for some positive integer $k$. We must now show that $R(k + 1)$ is 
      also true. Now we have that
      \begin{align*}
         (ab)^{k+1} &= (ab)^k(ab)^1 &[\text{Exercise 1.1.19}] \\
                    &= a^kb^k(ab)^1 &[\text{Since }R(k) \text{ is true}] \\
                    &= a^kb^k(ba)^1 &[ab = ba] \\
                    &= a^kb^kba \\
                    &= a^kb^{k+1}a \\
                    &= a^kab^{k+1} &[\text{$a$ commutes with $b$}] \\
                    &= a^{k+1}b^{k+1}, \\
      \end{align*}
      so that $R(k + 1)$ holds. It follows by the Principle of Mathematical 
      Induction that $R(n)$ holds for every positive integer $n$. By inpsection 
      we can see that $R(0)$ also holds. To complete the proof, we must now show 
      that $(ab)^{m} = a^mb^m$, where $m$ is a negative integer. First we notice 
      that
      \begin{equation}
         a^{-1}b^{-1} = (ba)^{-1} = (ab)^{-1} = b^{-1}a^{-1},
         \label{1_1_24_1}
      \end{equation}
      so that $a^{-1}$ and $b^{-1}$ are commuting elements. Thus it follows that
      \begin{align*}
         (ab)^m &= (ab)^{-(-m)} \\
                &= [(ab)^{-1}]^{-m} &[\text{Definition}] \\
                &= (a^{-1}b^{-1})^{-m} &[\eqref{1_1_24_1}] \\
                &= (a^{-1})^{-m}(b^{-1})^{-m} &[\text{$R(-m)$ holds}] \\
                &= a^mb^m,
      \end{align*}
      as desired. \qed
%%%%%%%%%%%%%%%%%%%%%%%%%%%%%%%%%%%%%%1.25%%%%%%%%%%%%%%%%%%%%%%%%%%%%%%%%%%%%%%
   \item[1.1.25]  Prove that if $x^2 = 1$ for all $x \in G$ then $G$ is abelian.

      \textbf{Proof.} Let $G$ be a group. Suppose that $x^2 = 1$ for all
      $x \in G$. We want to show that $G$ is abelian; that is, we want to show 
      that $xy = yx$ for all $x, y \in G$. So let $x, y \in G$. By hypothesis, 
      we have that $x^2 = e$, $y^2 = e$, and $(xy)^2 = e$, so that according to 
      Proposition 2, we must have that $x = x^{-1}$, $y = y^{-1}$, and
      $xy = (xy)^{-1}$. Thus
      \begin{align*}
         xy &= (xy)^{-1}      &[\text{By Hypothesis}] \\
            &= y^{-1}x^{-1}   &[\text{Proposition 1}] \\
            &= yx.
      \end{align*}
      Thus $G$ is abelian. \qed
%%%%%%%%%%%%%%%%%%%%%%%%%%%%%%%%%%%%%%1.26%%%%%%%%%%%%%%%%%%%%%%%%%%%%%%%%%%%%%%
   \item[1.1.26]  Assume $H$ is a nonempty subset of $(G, *)$ which is closed 
                  under the binary operation on $G$ and is closed under
                  inverses, i.e., for all $h$ and
                  $k \in H$, $hk$ and $h^{-1} \in H$. Prove that $H$ is a group 
                  under the operation $*$ restricted to $H$ (such a subset $H$
                  is called a subgroup of $G$).

      \textbf{Proof.} We know that $H$ is closed under $*$ and under inverses, 
      so it suffices to show that $*$ is associative on $H$ and that $H$ has an 
      identity under $*$. The associativity of $H$ under $*$ follows because $H$ 
      is a subset of $G$ and $G$ is associative under $*$. Since $H$ is nonempty
      we pick an $h \in H$. Then by hypothesis, we have that
      $1 = hh^{-1} \in H$, so that $H$ contains the identity. (Note that
      $hh^{-1} = h^{-1}h = 1$ and $h1 = 1h = h$ because these equalities hold in
      $G$.) \qed
%%%%%%%%%%%%%%%%%%%%%%%%%%%%%%%%%%%%%%1.27%%%%%%%%%%%%%%%%%%%%%%%%%%%%%%%%%%%%%%
   \item[1.1.27]  Prove that if $x$ is an element of the group $G$ then
                  $\{x^n : n \in \Z\}$ is a subgroup of $G$ (called the
                  \textit{cyclic subgroup} of $G$ generated by $x$).

      \textbf{Proof.} Consider the set
      $$H = \{x^n : n \in \Z\}.$$
      $H$ is nonempty because it contains $1 = x^0$. So let $h_1, h_2 \in H$.
      Thus we have $h_1 = x^a$ and $h_2 = x^b$ for some integers $a$ and $b$, so
      that $h_1h_2 = x^ax^b = x^{a+b} \in H$; in other words, $H$ is closed
      under the operation of $G$. Since $h_1^{-1} = (x^a)^{-1} = x^{-a} \in H$, 
      it follows that $H$ is also closed under inverses, so that $H$ is a
      subgroup of $G$ by Exercise 1.1.26.
%%%%%%%%%%%%%%%%%%%%%%%%%%%%%%%%%%%%%%1.28%%%%%%%%%%%%%%%%%%%%%%%%%%%%%%%%%%%%%%
   \item[1.1.28]  Let $(A, *)$ and $(B, \diamond)$ be groups and let
                  $A \times B$ be their direct product (as defined in Example
                  6). Verify all the group axioms for $A \times B$.
                  \begin{enumerate}
                     \item prove that the associative law holds: for all
                           $(a_i, b_i) \in A \times B, i = 1, 2, 3$
                           $$(a_1, b_1)[(a_2, b_2)(a_3, b_3)] =
                            [(a_1, b_1)(a_2, b_2)](a_3, b_3),$$
                     \item prove that (1, 1) is the identity of $A \times B$,
                           and
                     \item prove that the inverse of $(a, b)$ is
                           $(a^{-1}, b^{-1})$.
                  \end{enumerate}

      \textbf{Proof.} Let $(a_1, b_1)$, $(a_2, b_2)$, and
      $(a_3, b_3) \in A \times B$.

      \begin{enumerate}
         \item The set $A \times B$ is associative under the component wise
               operations of $A$ and $B$ because
               \begin{align*}
                  (a_1, b_1)[(a_2, b_2)(a_3, b_3)]
                     &= (a_1, b_1)(a_2a_3, b_2b_3) \\
                     &= (a_1a_2a_3, b_1b_2b_3) \\
                     &= [(a_1a_2)a_3, (b_1b_2)b_3] &[\text{Associativity}] \\
                     &= (a_1a_2, b_1b_2)(a_3, b_3) \\
                     &= [(a_1, b_1)(a_2, b_2)](a_3, b_3).
               \end{align*}
         \item Consider $(1, 1) \in A \times B$. It follows that
               \begin{align*}
                  (1, 1)(a_1, b_1) &= (1a_1, 1b_1) \\
                                   &= (a_1, b_1) \\
                                   &= (a_11, b_11) \\
                                   &= (a_1, b_1)(1, 1),
               \end{align*}
               so that $(1, 1)$ is the identity of $A \times B$.
         \item Consider $(a, b) \in A \times B$. It 
               follows that
               \begin{align*}
                  (a, b)(a^{-1}, b^{-1}) &= (aa^{-1}, bb^{-1}) \\
                                   &= (1, 1) \\
                                   &= (a^{-1}a, b^{-1}b) \\
                                   &= (a^{-1}, b^{-1})(a, b),
               \end{align*}
               so that $(a^{-1}, b^{-1})$ is the inverse of $(a, b)$.
      \end{enumerate}
%%%%%%%%%%%%%%%%%%%%%%%%%%%%%%%%%%%%%%1.29%%%%%%%%%%%%%%%%%%%%%%%%%%%%%%%%%%%%%%
   \item[1.1.29]  Prove that $A \times B$ is an abelian group if and only if
                  both $A$ and $B$ are abelian.

      \textbf{Proof.} 

      $(\Leftarrow)$ Suppose that $A$ and $B$ are abelian. Let $(a_1, b_1)$ and
      $(a_2, b_2) \in A \times B$. It follows that $A \times B$ is abelian
      because
      \begin{align*}
         (a_1, b_1)(a_2, b_2) &= (a_1a_2, b_1b_2) \\
            &= (a_2a_1, b_2b_1) &[\text{$A$ and $B$ are abelian}] \\
            &= (a_2, b_2)(a_1, b_1).
      \end{align*}

      $(\Rightarrow)$ Now suppose that $A \times B$ is abelian. Let $a_1$ and
      $a_2$ be members of $A$ and let $b_1$ and $b_2$ be members of $B$. Then
      we have that
      \begin{align*}
         (a_1a_2, b_1b_2) = (a_1, b_1)(a_2, b_2) \\
            &= (a_2, b_2)(a_1, b_1) &[\text{$A \times B$ is abelian}] \\
            &= (a_2a_1, b_2b_1),
      \end{align*}
      so that $(a_1a_2, b_1b_2) = (a_2a_1, b_2b_1)$; i.e., $a_1a_2 = a_2a_1$ and
      $b_1b_2 = b_2b_1$. We can now conclude that $A$ and $B$ are both abelian.
      \qed
%%%%%%%%%%%%%%%%%%%%%%%%%%%%%%%%%%%%%%1.30%%%%%%%%%%%%%%%%%%%%%%%%%%%%%%%%%%%%%%
   \item[1.1.30]  Prove that the elements $(a, 1)$ and $(1, b)$ of $A \times B$
                  commute and deduce that the order of $(a, b)$ is the least 
                  common multiple of $|a|$ and $|b|$.

      \textbf{Proof.} Let $A$ and $B$ be groups, and let $a \in A$, $b \in B$.
      We shall be assuming that there exist positive integers $m$ and $n$ such 
      that $|a| = m$ and $|b| = n$, for the problem does not make sense if the
      order of $a$ or $b$ is not finite. Consider $(a, 1)$,
      $(1, b) \in A \times B$. We have that
      \begin{align*}
         (a, 1)(1, b) &= (a1, 1b) \\
                      &= (a, b) \\
                      &= (1a, b1) \\
                      &= (1, b)(a, 1),
      \end{align*}
      so that $(a, 1)$ and $(b, 1)$ commute. To complete the proof, we let
      $s = \text{lcm}(m, n)$. Thus we can write $s = mx = ny$ for positive 
      integers $x$ and $y$. Thus we have that
      \begin{align*}
         (a, b)^s &= (a^s, b^s) &[\text{Lemma 1.1.4}] \\
                  &= (a^{mx}, a^{ny}) \\
                  &= [(a^m)^x, (a^n)^y] \\
                  &= (1^x, 1^y) \\
                  &= (1, 1).
      \end{align*}
      This say that $|(a, b)| \le s$, so there exists a positive integer $q$ 
      such that $|(a, b)| = q$. By Lemma 1.1.4, we have that
      $(a, b)^q = (a^q, b^q) = (1, 1)$, so that $a^q = 1$ and $b^q = 1$. Thus by 
      Lemma 1.1.3, it follows that $m \mid q$ and $n \mid q$, so that $s \mid q$ 
      by definition of the lcm. Since $s \mid q$, we must have that $s \le q$.
      But we previously showed that $q \le s$. Thus we can conclude that
      $s = q$, as desired. \qed
%%%%%%%%%%%%%%%%%%%%%%%%%%%%%%%%%%%%%%1.31%%%%%%%%%%%%%%%%%%%%%%%%%%%%%%%%%%%%%%
   \item[1.1.31]  Prove that any finite group $G$ of even order contains an
                  element of order 2. [Let $t(G)$ be the set
                  $\{g \in G : g \neq g^{-1}\}$. Show that $t(G)$ has an even 
                  number of elements and every nonidentity element of $G - t(G)$ 
                  has order 2.]

      \textbf{Proof.} Let $G$ be a finite group of even order. We wish to show
      that there exists some $g \in G$ such that $|g| = 2$. Consider this subset
      of $G$:
      $$S = \{g \in G: g \neq g^{-1}\}.$$

      If $|S| = 0$, then the proof is done, so assume that $|S| > 0$. Now $|S|$ 
      is even, for if this were not the case, then if we pair up every element
      of $S$ with its inverse, then one element must be without an inverse, a 
      contradiction. Now let $S' = G\backslash S$. It follows that
      $|G| = |S| + |S'|$. Notice that $S'$ is not empty because $e \in S'$. 
      Since $G$ and $S$ are both even, it follows that $|S'|$ must also be even. 
      Since we already showed that $|S'| \ge 1$, we can conclude that
      $|S'| \ge 2$, so that $S'$ contains a non-identity $a$, where
      $a = a^{-1}$. That is, $|a| = 2$. \qed
%%%%%%%%%%%%%%%%%%%%%%%%%%%%%%%%%%%%%%1.32%%%%%%%%%%%%%%%%%%%%%%%%%%%%%%%%%%%%%%
   \item[1.1.32]  If $x$ is an element of finite order $n$ in $G$, prove that
                  the elements 1, $x$, $x^2$, $\ldots$, $x^{n-1}$ are all 
                  distinct. Deduce that $|x| \le |G|$.

      \textbf{Proof.} Suppose that $|x| = n \in \Z^+$ for some $x \in G$. 
      Suppose to the contrary that the elements $x^0$, $x^1$, $x^2$, $\ldots$, 
      $x^{n-1}$ are not distinct. Then we must have that $x^i = x^j$ for some
      integer $i$ and $j$ where $0 \le i < j \le n - 1$. That is, $x^{j-i} = 1$,
      a contradiction because $j - i$ is a positive integer less thatn $n$. It
      follows that the elements $x^0$, $x$, $x^2$, $\ldots$, $x^{n-1}$ are all 
      distinct. Since there are clearly $n$ of these elements and since they are
      all members of $G$, we can conclude that $|x| = n \le |G|$. \qed
%%%%%%%%%%%%%%%%%%%%%%%%%%%%%%%%%%%%%%1.33%%%%%%%%%%%%%%%%%%%%%%%%%%%%%%%%%%%%%%
   \item[1.1.33]  Let $x$ be an element of finite order $n$ in $G$.
                  \begin{enumerate}
                     \item Prove that if $n$ is odd then $x^i \neq x^{-i}$ for
                           all $i = 1, 2, \ldots, n - 1$,
                     \item Prove that if $n = 2k$ and $1 \le i < n$ then
                           $x^i = x^{-i}$ if and only if $i = k$.
                  \end{enumerate}

      \textbf{Proof.}

      \begin{enumerate}
         \item Suppose that $n$ is odd. Now we shall suppose to the contrary
               that $x^i = x^{-i}$ for some integer $1 \le i \le n - 1$. Since
               $x^i = x^{-i}$, it follows that $x^{2i} = 1$. By Lemma 1.1.3, we
               must have that $n \mid 2i$, a contradiction because an odd
               number cannot divide a positive even number, so we conclude that
               $x^i \neq x^{-i}$ for all $i = 1, 2, \ldots, n - 1$. \qed
         \item Suppose that $n$ is even and $1 \le i < n$. Write $n = 2k$ for
               some positive integer $k$.

               $(\Leftarrow)$ Suppose that $i = k$. Then we have that
               $1 = x^{2k} = x^{2i} = x^ix^i$, so that $x^i = x^{-i}$.

               $(\Rightarrow)$ Conversely suppose that $x^i = x^{-i}$, so that
               $x^{2i} =1$. Thus, by Lemma 1.1.3, $2k \mid 2i$, or equivalently,
               $k \mid i$, so that $i = mk$ for some positive integer $m$. 
               Recall that $i < n = 2k$ by hypothesis, so that $mk < 2k$. That 
               is $m < 2$. But $m$ is a positive integer and so the only
               possibility is therefore $m = 1$, so that $i = k$. \qed
      \end{enumerate}
%%%%%%%%%%%%%%%%%%%%%%%%%%%%%%%%%%%%%%1.34%%%%%%%%%%%%%%%%%%%%%%%%%%%%%%%%%%%%%%
   \item[1.1.34]  If $x$ is an element of infinite order in $G$, prove that the
                  elements $x^n$, $n \in \Z$ are all distinct.

      \textbf{Proof.} Assume that $x$ is an element of infinite order in $G$.
      Now suppose to the contrary that $x^i = x^j$ for some unequal integers
      $i$ and $j$. We can further assume without loss of generality that
      $i < j$. Thus $x^{j-i} = 1$, a contradiction because this says that
      $|x| \le j - i$. It follows that distinct integral powers of $x$ yield 
      distinct elements of $G$. \qed
%%%%%%%%%%%%%%%%%%%%%%%%%%%%%%%%%%%%%%1.35%%%%%%%%%%%%%%%%%%%%%%%%%%%%%%%%%%%%%%
   \item[1.1.35]  If $x$ is an element of finite order $n$ in $G$, use the 
                  Division Algorithm to show that any integral power of $x$ 
                  equals one of the elements in the set
                  $\{1, x, x^2, \ldots, x^{n-1}\}$ (so these are all the
                  distinct elements of the cyclic subgroup of $G$ generated by
                  $x$).

      \textbf{Proof.} Assume that $x$ is an element of finite order $n$ in $G$.
      Let $z \in \Z$. By the Division Algorithm, there exist unique integers
      $q$ and $r$ such that $z = qn + r$ and $0 \le r < n$. That is
      $$x^z = x^{qn+r} = x^{qn}x^r = (x^n)^qx^r = 1^qx^r = x^r.$$
      Since $r \in \{0, 1, \ldots, n - 1\}$ and since $x^z = x^r$, it follows
      that $x^z \in \{x^0, x^1, \ldots, x^{n-1}\}$. \qed
%%%%%%%%%%%%%%%%%%%%%%%%%%%%%%%%%%%%%%1.36%%%%%%%%%%%%%%%%%%%%%%%%%%%%%%%%%%%%%%
   \item[1.1.36]  Assume $G = \{1, a, b, c\}$ is a group of order 4 with
                  identity 1. Assume also that $G$ has no elements of order 4
                  (so by Exercise 32, every element has order $\le$ 3). Use the
                  cancellation laws to show that there is a unique group table
                  for $G$. Deduce that $G$ is abelian.

      \textbf{Proof.} Assume $G = \{1, a, b, c\}$. We can tentatively fill out 
      the group table for $G$ like so:      
      $$
         \begin{tabular}{@{}c | c | c | c | c@{}} 
                & $1$ & $a$ & $b$ & $c$ \\ \hline
            $1$ & $1$ & $a$ & $b$ & $c$ \\ \hline
            $a$ & $a$ & $ $ & $ $ & $ $ \\ \hline
            $b$ & $b$ & $ $ & $ $ & $ $ \\ \hline
            $c$ & $c$ & $ $ & $ $ & $ $
         \end{tabular}
      $$
      By the left cancellation law, the equality $ab = a$ will result in $b = e$
      and the equality $ab = b$ will result in $a = e$, both of which are
      contradictions. The only remaining possiblities are $ab = c$ or $ab = 1$.

      \textbf{Case 1:} $ab = c$. For the same reason as above, we cannot have
      $ac = a$ or $ac = c$, so that $ac = 1$ or $ac = b$. So suppose first that
      $ac = b$. Then our table will now look like so:
      $$
         \begin{tabular}{@{}c | c | c | c | c@{}} 
                & $1$ & $a$ & $b$ & $c$ \\ \hline
            $1$ & $1$ & $a$ & $b$ & $c$ \\ \hline
            $a$ & $a$ & $ $ & $c$ & $b$ \\ \hline
            $b$ & $b$ & $ $ & $ $ & $ $ \\ \hline
            $c$ & $c$ & $ $ & $ $ & $ $
         \end{tabular}
      $$
      From the table above, we see that $aa$ must be equal to 1, since that is
      the only remaining possibility. The cancellation laws tell us that every
      element in a column and row of a group table must be unique, so we must
      have that:
      $$
         \begin{tabular}{@{}c | c | c | c | c@{}} 
                & $1$ & $a$ & $b$ & $c$ \\ \hline
            $1$ & $1$ & $a$ & $b$ & $c$ \\ \hline
            $a$ & $a$ & $1$ & $c$ & $b$ \\ \hline
            $b$ & $b$ & $c$ & $ $ & $ $ \\ \hline
            $c$ & $c$ & $b$ & $ $ & $ $
         \end{tabular}
      $$
      Note that we cannot have $bb = a$ because that would imply that $bbb = c$,
      so that $|b| > 3$, contradicting our hypothesis. Thus we must have that
      $bb = 1$. The remaining positions are thus completely determined, so that
      we have
      $$
         \begin{tabular}{@{}c | c | c | c | c@{}} 
                & $1$ & $a$ & $b$ & $c$ \\ \hline
            $1$ & $1$ & $a$ & $b$ & $c$ \\ \hline
            $a$ & $a$ & $1$ & $c$ & $b$ \\ \hline
            $b$ & $b$ & $c$ & $1$ & $a$ \\ \hline
            $c$ & $c$ & $b$ & $a$ & $1$
         \end{tabular}
      $$
      Now suppose that $ac = 1$, then we would be forced to fill in the table
      like so:
      $$
         \begin{tabular}{@{}c | c | c | c | c@{}} 
                & $1$ & $a$ & $b$ & $c$ \\ \hline
            $1$ & $1$ & $a$ & $b$ & $c$ \\ \hline
            $a$ & $a$ & $b$ & $c$ & $1$ \\ \hline
            $b$ & $b$ & $c$ & $ $ & $ $ \\ \hline
            $c$ & $c$ & $1$ & $ $ & $ $
         \end{tabular}
      $$
      Since $a^2 = b$ and $a^3 = c$, we have that $|a| > 3$, contradicting our
      hypothesis, so this is a dead end.

      \textbf{Case 2:} $ab = 1$. For the same reason as above, we cannot have
      $ac = a$ or $ac = c$, so that $ac = 1$ or $ac = b$. So suppose first that
      $ac = b$. Then our table will now look like so:
      $$
         \begin{tabular}{@{}c | c | c | c | c@{}} 
                & $1$ & $a$ & $b$ & $c$ \\ \hline
            $1$ & $1$ & $a$ & $b$ & $c$ \\ \hline
            $a$ & $a$ & $c$ & $1$ & $b$ \\ \hline
            $b$ & $b$ & $1$ & $c$ & $a$ \\ \hline
            $c$ & $c$ & $b$ & $a$ & $1$
         \end{tabular}
      $$
      Since $a^2 = c$ and $a^3 = b$, we have that $|a| > 3$, contradicting our
      hypothesis, so this is another dead end. From our arguments above, we see
      that the only viable and legal table is thus:
      $$
         \begin{tabular}{@{}c | c | c | c | c@{}} 
                & $1$ & $a$ & $b$ & $c$ \\ \hline
            $1$ & $1$ & $a$ & $b$ & $c$ \\ \hline
            $a$ & $a$ & $1$ & $c$ & $b$ \\ \hline
            $b$ & $b$ & $c$ & $1$ & $a$ \\ \hline
            $c$ & $c$ & $b$ & $a$ & $1$
         \end{tabular}
      $$
      This table is unique, and since it is symmeteric it follows that $G$ is
      abelian. \qed
\end{enumerate}

      \section{Properties Of The Integers}
         \begin{enumerate}
%%%%%%%%%%%%%%%%%%%%%%%%%%%%%%%%%%%Prob1.2_1%%%%%%%%%%%%%%%%%%%%%%%%%%%%%%%%%%%%
   \item[1.2.1]   For each of the following statements, determine whether it is 
                  true or false and justify your answer.
                  \begin{enumerate}
                     \item The set $\Z$ of integers is dense in $\R$.
                     \item The set of positive real numbers is dense in $\R$.
                     \item The set $\Q\backslash \Z$ of rational numbers that 
                           are not integers is dense in $\R$.
                  \end{enumerate}  

      \textbf{Solution:} 

      \begin{enumerate}
         \item False. Proposition 1.6 states that there is no integer in the
               interval (0, 1).
         \item False. The interval $(-1, 0)$ contains no positive real number.
         \item True. Let $a$ and $b$ be real numbers. Then we shall investigate
               the following two cases:
               
               \textbf{Case I:} $a < a + 1 \le b$. Theorem 1.8 says that there
               exists a unique integer $k$ in $[a, a + 1)$. Thus there is no
               integer in the interval $(k, a + 1)$. By the density of $\Q$ in
               $\R$, there exists a rational $q \in (k, a + 1)$. Since
               $(k, a + 1)$ contains no integer, then $q$ must be a member of
               $\Q\backslash\Z$. We observe that $q \in (a, b)$.
               
               \textbf{Case II:} $a < b < a + 1$. Theorem 1.8 says that there
               exists a unique integer $k$ in $[a, a + 1)$. If $k  \le b$, then
               $(a, k)$ has no integer, so there exists a noninteger rational
               in $(a, k) \subseteq (a, b)$ by the density of $\Q$ in $\R$. If,
               however, $k > b$, then the interval $(a, b)$ contains no integer,
               so that there exists a noninteger rational in $(a, b)$ by the
               density of $\Q$ in $\R$.
      \end{enumerate}
%%%%%%%%%%%%%%%%%%%%%%%%%%%%%%%%%%Prob1.2_2%%%%%%%%%%%%%%%%%%%%%%%%%%%%%%%%%%%%%
   \item[1.2.2]   Suppose that $S$ is a nonempty set of integers that is bounded
                  below. Show that $S$ has a minimum. In particular, conclude
                  that every nonempty set of natural numbers has a minimum.  

      \textbf{Proof:}

      Let $S$ be a nonempty set of integers bounded below. Then there exists
      some $r \in \R$ such that for every $a \in S$, we have that $r \le a$.
      Consider the set $S' = \{-s: s \in S\}$, the set of the additive inverses
      of the elements of $S$. Note that $S'$ is also a nonempty set of integers.
      So let $-d \in S'$ where $d \in S$. Hence $r \le d$, so that $-d \le -r$;
      that is $S'$ is bounded above. By Proposition 1.7 $S'$ has a maximum, say
      $-b$, where $b \in S$. It suffices to show that $b$ is the minimum in $S$.
      Let $c \in S$. Then we have that $-c \le -b$, so that $b \le c$; that is,
      $b$ is the minimum element in $S$. In paritcular, we can see that the
      Well Ordering Principle follows. \qed
%%%%%%%%%%%%%%%%%%%%%%%%%%%%%%%%%%Prob1.2_3%%%%%%%%%%%%%%%%%%%%%%%%%%%%%%%%%%%%%
   \item[1.2.3]   Let $S$ be a nonempty set of real numbers that is bounded
                  below. Prove that the set $S$ has a minimum if and only if the
                  number $\inf S$ belongs to $S$.
			
		\textbf{Proof:} Let $S$ be a nonempty set of real numbers that is bounded
      below.

      $(\Leftarrow)$ Suppose $\inf S$ belongs in $S$; then it immediately
      follows by definition that $\inf S$ is the minimum element of $S$. \\
      $(\Rightarrow)$ Now suppose that $S$ has a minimum, say $s$. By the
      Completeness Axiom, we have that $\inf S$ exists; since $s \in S$, we must
      have that $\inf S \le s$. But $s$ is also a lower bound for $S$ and since
      every lower bound of $S$ cannot exceed $\inf S$, we must have that
      $s \le \sup S$; we have shown that $\inf S \le s$ and $s \le \inf S$ so
      that $s = \inf S$. \qed
%%%%%%%%%%%%%%%%%%%%%%%%%%%%%%%%%%Prob1.2_4%%%%%%%%%%%%%%%%%%%%%%%%%%%%%%%%%%%%%
   \item[1.2.4]   For each of the following two sets, find the maximum, minimum,
                  infimum, and supremum if they are defined. Justify your
                  conclusions.
                  \begin{enumerate}
                     \item $S = \{1/n : n \in \N\}$.
                     \item $T = \{x \in \R : x^2 < 2\}$.
                  \end{enumerate}

      \textbf{Solution:}

      \begin{enumerate}
         \item The \textbf{maximum} is 1. To show this consider any natural
               number $n$; then we have $n \ge 1$. Multiply this inequality by
               the positive number $1/n$ to give us $1/n \le 1$. Since
               $1 = 1/1 \in S$, we are done. $S$ has no \textbf{minimum}. Assume
               by way of contradiction that $\min S$ exists. Then by definition
               of $S$, we know that $\min S$ must be positive. So by the
               Archimedean Property, there exists a natural number $n_1$(so that
               $1/n_1 \in S$) such that $1/n_1 < \min S$, a contradiction. So
               $\min S$ doesn't exist. Since the Archimedean Property enables us
               to find a member of $S$ that is less than any positive number, no
               positive number can be a lower bound for $S$. Thus $S$ can only 
               be bounded below by negative numbers and 0. It follows that the
               \textbf{infimum} of  $S$ is 0. Since $S$ has a maximum, this
               maximum. By Problem 1.1.15, we have that the \textbf{supremum} of
               $S = 1$. If we consider $-T$, the set of the additive inverses of
               the elements of $T$.
         \item It is trivial to show that%%%%%%%%%%%%%%%%%%%%%%%%%%%%%%%%%%%%%%%%%%%%%%%%%%%%Show true
               $T = \{x \in \R: -\sqrt{2} < x < \sqrt{2}\}$. We claim that the
               \textbf{infimum} and \textbf{supremum} of $T$ are $-\sqrt{2}$ and
               $\sqrt{2}$. Suppose by contradiction that this is false; then 
               there exist $a > -\sqrt{2}$ and $b < \sqrt{2}$ such that $a$ and 
               $b$ are the infimum and supremum of $T$. Then by the density of
               $\Q$ in $\R$, there exist rationals $p$ and $q$ such that
               $-\sqrt{2} < p < a$ and $b < q < \sqrt{2}$; that is, $p$ and $q$ 
               are members of $T$. But since $p$ is less than $a$ and $q > b$, 
               we have contradictions. Thus our claim holds. Since the infimum 
               and supremum are not members of $T$, it follows that $T$ has 
               neither a \textbf{maximum} nor a \textbf{minimum}.
      \end{enumerate}
%%%%%%%%%%%%%%%%%%%%%%%%%%%%%%%%%%Prob1.2_5%%%%%%%%%%%%%%%%%%%%%%%%%%%%%%%%%%%%%
   \item[1.2.5]   Suppose that the number $a$ has the property that for every
                  natural number $n$, $a \le 1/n$. Prove that $a \le 0$.

      \textbf{Proof:} Assume by way of contradiction that $a > 0$. By The
      Archimedean Property there exists a natural number $k$ such that
      $a > 1/k$, a contradiction. Thus $a \le 0$. \qed

%%%%%%%%%%%%%%%%%%%%%%%%%%%%%%%%%%Prob1.2_6%%%%%%%%%%%%%%%%%%%%%%%%%%%%%%%%%%%%%
   \item[1.2.6]   Given a real number $a$, define
                  $S \equiv \{x : x \in \Q, x < a\}$. Prove that $a = \sup S$.

      \textbf{Proof:} By the density of $\Q$ in $\R$, there exists a rational
      $q \in (a - 1, a)$, so that $q \in S$. Thus $S$ is nonempty. By 
      definition, $S$ is bounded above by $a$; since $S$ is also nonempty, the
      Completeness Axiom says that $\sup S$ exists. So we must have that
      $\sup S \le a$. Now suppose that $\sup S < a$, then the density of $\Q$ in
      $\R$ guarantees that we have a rational $q$ in $(\sup S, a)$, so that $q$
      is also a member of $S$, a contradiction since we cannot have a member of
      $S$ that is greater than $\sup S$. Thus $a = \sup S$. \qed

%%%%%%%%%%%%%%%%%%%%%%%%%%%%%%%%%%Prob1.2_7%%%%%%%%%%%%%%%%%%%%%%%%%%%%%%%%%%%%%
   \item[1.2.7]   Show that for any real number $c$, there is exactly one 
                  integer in the interval $(c, c+1]$.

      \textbf{Proof:} Let $c$ be a real number. According to Theorem 1.8, there
      exists a unique integer $k$ in the interval $[-(c + 1), -c)$. So we have
      $-(c + 1) \le k < -c$, so that $c < -k \le c + 1$. Hence we have an
      integer $-k$ in the interval $(c, c + 1]$. We can see that $-k$ is unique
      because if another integer $h$ exists in $(c, c + 1]$, then $-h$ would
      also be in $[-(c + 1), -c)$, and since $k$ is unique, we must have
      $-h = k$, so that $h = -k$. \qed
   
%%%%%%%%%%%%%%%%%%%%%%%%%%%%%%%%%%Prob1.2_8%%%%%%%%%%%%%%%%%%%%%%%%%%%%%%%%%%%%%
   \item[1.2.8]   Show that the Archimedean Property is a consequence of the
                  assertion that for any real number $c$, there is an integer in
                  the interval $[c, c + 1)$.

      \textbf{Proof:} Let $\epsilon$ be a positive real number. It suffices to
      show that there exists a natural number greater than $\epsilon$. By our
      assertion, there exists an integer $k$ in the interval
      $[\epsilon+ 1, \epsilon + 2)$. So we have $\epsilon < \epsilon + 1 \le k$,
      so that $k$ is a positive integer. \qed
%%%%%%%%%%%%%%%%%%%%%%%%%%%%%%%%%%Prob1.2_9%%%%%%%%%%%%%%%%%%%%%%%%%%%%%%%%%%%%%
   \item[1.2.9]   Show that the Archimedean Property is a consequence of the
                  assertion that every interval $(a, b)$ contains a rational
                  number.

      \textbf{Proof:} Let $\epsilon$ be a positive real number. It suffices to
      show that there exists a natural number greater than $\epsilon$. By our
      assertion, there exist positive integers $p$ and $q$ such that
      $p/q \in (\epsilon, \epsilon + 1)$. Since $p$ and $q$ are positive, we 
      have that $q \ge 1$ so that $pq \ge p$; that is $p/q \le p$. We have now
      shown that $\varepsilon < p/q \le p$. Particularly $p > \varepsilon$,
      which is what we wanted to prove. \qed

      
\end{enumerate}

      \section{$\Z/n\Z$ : The Integers Modulo $n$}
         \begin{enumerate}
%%%%%%%%%%%%%%%%%%%%%%%%%%%%%%%%%%%%%2.3.1%%%%%%%%%%%%%%%%%%%%%%%%%%%%%%%%%%%%%%
   \item[2.3.1]   Find all subgroups of $Z_{45} = \cyc{x}$, giving a generator
                  for each. Describe the containments between these subgroups.
                  
      \textbf{Solution.} Since the positive divisors of 45 are: 1, 3, 5, 9, 15,
      and 45, it follows that the subgroups of $Z_{45}$ are
      $$\cyc{x}, \cyc{x^3}, \cyc{x^5}, \cyc{x^9}, \cyc{x^{15}}, \text{ and }
        \cyc{x^{45}}.$$
        
      We have the following containments:
      $$
         \begin{tabular}{>{$}c<{$}>{$}c<{$}>{$}c<{$}>{$}c<{$}>{$}c<{$}>{$}c<{$}>{$}c<{$}}
            \cyc{x^{45}} & \le & \cyc{x^{15}} & \le & \cyc{x^5} & \le & \cyc{x} \\
            \cyc{x^{15}} & \le &  \cyc{x^3} & \le & \cyc{x} \\
            \cyc{x^9} & \le &  \cyc{x^3} & \le & \cyc{x}
         \end{tabular}
      $$
%%%%%%%%%%%%%%%%%%%%%%%%%%%%%%%%%%%%%2.3.2%%%%%%%%%%%%%%%%%%%%%%%%%%%%%%%%%%%%%%
   \item[2.3.2]   If $x$ is an element of the finite group $G$ and $|x| = |G|$,
                  prove that $G = \cyc{x}$. Give an explicit example to show 
                  that this result need not be true if $G$ is an infinite group.
                  
      \textbf{Proof.} Let $G$ be a finite group, so that $|G| = n \in \Z^+$.
      Suppose that there exists $x \in G$ such that $|x| = n$. Clearly
      $\cyc{x} \subseteq G$. But $|\cyc{x}| = n$ since $|x| = n$; thus
      $G \subseteq \cyc{x}$ so that $G = \cyc{x}$. Now let $G = \Z$. We have
      that $|\cyc{2}| = |G|$ but $G \neq \cyc{2}$. \qed
%%%%%%%%%%%%%%%%%%%%%%%%%%%%%%%%%%%%%2.3.3%%%%%%%%%%%%%%%%%%%%%%%%%%%%%%%%%%%%%%
   \item[2.3.3]   Find all generators for $\Z/48\Z$.
   
      \textbf{Solution.} The generators for $\Z/48\Z$ are: $\cyc{\overline{1}}$,
      $\cyc{\overline{5}}$, $\cyc{\overline{7}}$, $\cyc{\overline{11}}$,
      $\cyc{\overline{13}}$, $\cyc{\overline{17}}$, $\cyc{\overline{19}}$,
      $\cyc{\overline{23}}$, $\cyc{\overline{25}}$, $\cyc{\overline{29}}$,
      $\cyc{\overline{31}}$, $\cyc{\overline{35}}$, $\cyc{\overline{37}}$,
      $\cyc{\overline{41}}$, $\cyc{\overline{43}}$, and $\cyc{\overline{47}}$.
%%%%%%%%%%%%%%%%%%%%%%%%%%%%%%%%%%%%%2.3.4%%%%%%%%%%%%%%%%%%%%%%%%%%%%%%%%%%%%%%
   \item[2.3.4]   Find all generators for $\Z/202\Z$.
   
      \textbf{Solution.} Let $S$ be the set of generators for $\Z/202\Z$. Then
      $|S| = 100$ since
      $$S = \{\cyc{x} : x \text{ is odd and positive}, x \neq 101, \text{ and } x < 202\}.$$
%%%%%%%%%%%%%%%%%%%%%%%%%%%%%%%%%%%%%2.3.5%%%%%%%%%%%%%%%%%%%%%%%%%%%%%%%%%%%%%%
   \item[2.3.5]   Find the number of generators for $\Z/49000\Z$.
   
      \textbf{Solution.} For a positive integer $n$ let $\varphi(n)$ be the
      number of positive integers---less than or equal to $n$---that are
      relatively prime to $n$. Then the number of generators for $\Z/49000\Z$ is
      $\varphi(49000) = \varphi(2^35^37^2) =
      \varphi(2^3)\varphi(5^3)\varphi(7^2) = 16800$. 
%%%%%%%%%%%%%%%%%%%%%%%%%%%%%%%%%%%%%2.3.6%%%%%%%%%%%%%%%%%%%%%%%%%%%%%%%%%%%%%%
   \item[2.3.6]   In $\Z/48\Z$ write out all elements of $\cyc{\overline{a}}$
                  for every $\overline{a}$. Find all inclusions between
                  subgroups in $\Z/48\Z$.
      
      \textbf{Solution.}
      $$
         \begin{tabular}{|c|c|} \hline
            \textbf{Generators} & \textbf{Subgroups in} $\Z/48\Z$ \\ \hline
            0 & $\{0\}$ \\ \hline
            24 & $\{0, 24\}$ \\ \hline
            16, 32 & $\{0, 16, 32\}$ \\ \hline
            12, 36 & $\{0, 12, 24, 36\}$ \\ \hline
            8, 40 & $\{0, 8, 16, 24, 32, 40\}$ \\ \hline
            6, 18, 30, 42 & $\{0, 6, 12, 18, 24, 30, 36, 42\}$ \\ \hline
            4,20,28,44 & $\{0,4,8,12,16, 20, 24, 28, 32, 36, 40, 44\}$ \\ \hline
            3, 9, 15, 21, 27, 33, 39, 45 & $\{0, 3, 6, 9, 12, 15, 18, 21, 24,
            27, 30, 33, 36, 39, 42, 45\}$ \\ \hline            
            2, 10, 14, 22, 26, 34, 38, 46 & $\{x : 0 \le x \le 46,
            x \text{ is even}\}$ \\ \hline
            \text{See Exercise } 2.3.3 & $\Z/48\Z$ \\ \hline
         \end{tabular}
      $$
%%%%%%%%%%%%%%%%%%%%%%%%%%%%%%%%%%%%%2.3.7%%%%%%%%%%%%%%%%%%%%%%%%%%%%%%%%%%%%%%
   \item[2.3.7]   Let $Z_{48} = \cyc{x}$ and use the isomorphism
                  $\Z/48\Z \cong Z_{48}$ given by $\overline{1} \mapsto x$ to
                  list all subgroups of $Z_{48}$ as computed in the preceding
                  exercise.
                  
      \textbf{Solution.}
      $$
         \begin{tabular}{|c|} \hline
            \textbf{Subgroups in} $Z_{48}$ \\ \hline
            $\{1\}$ \\ \hline
            $\{1, x^{24}\}$ \\ \hline
            $\{1, x^{16}, x^{32}\}$ \\ \hline
            $\{1, x^{12}, x^{24}, x^{36}\}$ \\ \hline
            $\{1, x^8, x^{16}, x^{24}, x^{32}, x^{40}\}$ \\ \hline
            $\{1, x^6, x^{12}, x^{18}, x^{24}, x^{30},x^{36},x^{42}\}$ \\ \hline
            $\{1,x^4,x^8,x^{12},x^{16}, x^{20}, x^{24}, x^{28}, x^{32}, x^{36},
               x^{40}, x^{44}\}$ \\ \hline
            $\{1, x^3, x^6, x^9, x^{12}, x^{15}, x^{18}, x^{21}, x^{24},
            x^{27}, x^{30}, x^{33}, x^{36}, x^{39}, x^{42}, x^{45}\}$ \\ \hline
            $\{x^y : 0 \le y \le 46, y \text{ is even}\}$ \\ \hline
            $Z_{48}$ \\ \hline
         \end{tabular}
      $$
%%%%%%%%%%%%%%%%%%%%%%%%%%%%%%%%%%%%%2.3.8%%%%%%%%%%%%%%%%%%%%%%%%%%%%%%%%%%%%%%
   \item[2.3.8]   Let $Z_{48} = \cyc{x}$. For which integers $a$ does the map
                  $\varphi_a$ defined by $\varphi_a : \overline{1} \mapsto x^a$
                  extend to an \textit{isomorphism} from $\Z/48\Z$ onto
                  $Z_{48}$.
                  
      \textbf{Solution.} Suppose that $(a, 48) = 1$. Then it follows that $x^a$
      generates $Z_{48}$. Thus $\varphi_a$ is an isomorphism by Theorem 4 (Page
      56). Now suppose that $a$ is not relatively prime to 48. Then $x^a$ does
      not generate $Z_{48}$, so that the image of $\varphi_a$ is not $Z_{48}$.
      Hence $\varphi_a$ is an isomorphism if and only if $(a, 48) = 1$.
%%%%%%%%%%%%%%%%%%%%%%%%%%%%%%%%%%%%%2.3.9%%%%%%%%%%%%%%%%%%%%%%%%%%%%%%%%%%%%%%
   \item[2.3.9]   Let $Z_{36} = \cyc{x}$. For which integers $a$ does the map
                  $\psi_a$ defined by $\psi_a : \overline{1} \mapsto x^a$ extend
                  to a \textit{well defined homomorphism} from $\Z/48\Z$ into
                  $Z_{36}$. Can $\psi_a$ ever be a surjective homomorphism?
                  
      \textbf{Solution.} First we shall find the restriction(s) on $a$ such that
      $\psi_a$ is well defined. Suppose $b = c$ for some $b, c \in \Z/48\Z$. It
      suffices to show that $\psi_a(b) = \psi_a(c)$. Since $b = c$, there exists
      an integer $k$ such that $b = c + 48k$. Thus $\psi_a(b) = \psi_a(c+48k)$,
      so that
      $\psi_a(b)=(x^a)^{c+48k}=x^{ac + 48ak}= x^{ac}x^{48ak}=\psi_a(c)x^{12ak}$.
      So we must require $x^{12ak} = 1$ for all $k \in \Z$. Now $x^{12ak} = 1$
      for all $k \in \Z$ if and only if $3 \mid a$ if and only if $\psi_a$ is
      well defined. It follows immediately that
      $\psi_a$ is an homomorphism since
      \begin{align*}
         \psi_a(p + q) &= (x^a)^{p+q} \\
            &= x^{ap+aq} \\
            &= x^{ap}x^{aq} \\
            &= (x^a)^p(x^a)^q \\
            &= \psi_a(p)\psi_a(q)
      \end{align*}      
      for all $p, q \in \Z/48\Z$.
      
      \textit{Can $\psi_a$ ever be a surjective homomorphism?} No!
      
      \textbf{Proof.} Suppose to the contrary that $\psi_a$ is surjective. Then
      there exists $y \in \Z/48\Z$ such that $\psi_a(y) = x$. That is
      $x^{ay} = x$, so that $x^{ay-1} = 1$; thus $ay - 1 = 36m$ for some integer
      $m$. Rearrange the equality $ay - 1 = 36m$ to get $1 = ay - 36m$. Recall
      that $3 \mid a$; since $3$ also divides 36, it follows that 3 must divide
      1, a contradiction. Thus $\psi_a$ can never be surjective. \qed
%%%%%%%%%%%%%%%%%%%%%%%%%%%%%%%%%%%%%2.3.10%%%%%%%%%%%%%%%%%%%%%%%%%%%%%%%%%%%%%
   \item[2.3.10]  What is the order of $\overline{30}$ in $\Z/54\Z$? Write out
                  all the elements and their orders in $\cyc{\overline{30}}$.
                  
      \textbf{Solution.} The order of $30$ in $\Z/54\Z$ is
      $$\frac{54}{(30, 54)} = 9.$$
      The elements of $\cyc{30}$ and their respective orders are:
      $$
         \begin{tabular}{|c|c|} \hline
            Element of $\cyc{30}$ & Order \\ \hline
            30 & 9 \\ \hline
             6 & 9 \\ \hline
            36 & 3 \\ \hline
            12 & 9 \\ \hline
            42 & 9 \\ \hline
            18 & 3 \\ \hline
            48 & 9 \\ \hline
            24 & 9 \\ \hline
             0 & 1 \\ \hline
         \end{tabular}
      $$
%%%%%%%%%%%%%%%%%%%%%%%%%%%%%%%%%%%%%2.3.11%%%%%%%%%%%%%%%%%%%%%%%%%%%%%%%%%%%%%
   \item[2.3.11]  Find all cyclic subgroups of $D_8$. Find a proper subgroup of
                  $D_8$ which is not cyclic.
                  
      \textbf{Solution.} In $D_8$, only $r$ and $r^4$ have order 4. Thus
      $\{1, r, r^2, r^3\}$ is the only cyclic subgroup of order 4. The trivial
      subgroup is the only cyclic subgroup of order 1. Finally there are 5
      cyclic subgroups of order 2 and they are of the form $\{1, x\}$ where
      $x \in \{r^2, s, sr, sr^2, sr^3\}$. The set $\{1, s, r^2, sr^2\}$ is a
      non-cyclic proper subgroup of $D_8$.
%%%%%%%%%%%%%%%%%%%%%%%%%%%%%%%%%%%%%2.3.12%%%%%%%%%%%%%%%%%%%%%%%%%%%%%%%%%%%%%
   \item[2.3.12]  Prove that the following groups are \textit{not} cyclic:
                  \begin{enumerate}
                     \item $Z_2 \times Z_2$
                     \item $Z_2 \times \Z$
                     \item $\Z \times \Z$.
                  \end{enumerate}
      
      \textbf{Proof.}
      \begin{enumerate}
         \item The order of $Z_2 \times Z_2$ is 4, but no element in this group
               has order 4; thus $Z_2 \times Z_2$ is not cyclic.
         \item Let $Z_2 = \cyc{x}$. Observe that $Z_2 \times \Z$ is not finite,
               so in order for it to be cyclic it must be isomorphic to $\Z$.
               But this is not the case since $Z_2 \times \Z$ has two elements
               of finite order(namely $(1, 0)$ and $(x, 0)$) while $\Z$ has
               exactly 1 element of finite order.
         \item Suppose to the contrary that $\Z \times \Z$ is cyclic. Then there
               exist nonzero integers $a$ and $b$ such that
               $$\Z \times \Z = \cyc{(a,b)} = \{(na, nb) : n \in \Z\}.$$
               Thus there exists an integer $m$ such that
               $(ma, mb) = (0, 1)$. That is, $ma = 0$ and $mb = 1$. Since
               $ma = 0$, we must have $m = 0$ or $a = 0$. If $m$ is 0, then
               $(ma, mb) = (0, 0) \neq (0, 1)$, a contradiction; thus we must
               have $a = 0$, contradicting our assumption that $a$ is nonzero.
               Thus $\Z \times \Z$ is not cyclic.
      \end{enumerate} \qed
%%%%%%%%%%%%%%%%%%%%%%%%%%%%%%%%%%%%%2.3.13%%%%%%%%%%%%%%%%%%%%%%%%%%%%%%%%%%%%%
   \item[2.3.13]  Prove that the following pairs of groups are \textit{not}
                  isomorphic:
                  \begin{enumerate}
                     \item $\Z \times Z_2$ and $\Z$
                     \item $\Q \times Z_2$ and $\Q$.
                  \end{enumerate}
      
      \textbf{Proof.}
      \begin{enumerate}
         \item By Exercise 1.6.11, we know that $\Z \times Z_2$ is isomorphic to
               $Z_2 \times \Z$. By Exercise 2.3.12, $Z_2 \times \Z$ is not
               cyclic; thus $\Z \times Z_2$ is not cyclic. That is,
               $\Z \times Z_2$ is not isomorphic to $\Z$.
         \item Let $Z_2 = \cyc{x}$. It immediately follows that
               $\Q \times Z_2$ and $\Q$ are not isomorphic since $\Q \times Z_2$
               has two elements of finite order(namely $(0, 1)$ and $(0, x)$)
               while $\Q$ has exactly 1 element of finite order.
      \end{enumerate} \qed
%%%%%%%%%%%%%%%%%%%%%%%%%%%%%%%%%%%%%2.3.14%%%%%%%%%%%%%%%%%%%%%%%%%%%%%%%%%%%%%
   \item[2.3.14]  Let $\sigma =$ (1 2 3 4 5 6 7 8 9 10 11 12). For each of the
                  following integers $a$ compute $\sigma^a$:
                  $$a = 13, 65, 626, 1195, -6, -81, -570,\text{ and } {-1211}.$$
                  
      \textbf{Solution.}
      
      \begin{alignat*}{4}
         &\sigma^{13}   &&= \sigma &&\text{ } \\
         &\sigma^{65}   &&= \sigma^5 &&=
            (1\;6\;11\;4\;9\;2\;7\;12\;5\;10\;3\;8) \\
         &\sigma^{626}  &&= \sigma^2 &&= (1\;3\;5\;7\;9\;11) \\
         &\sigma^{1195} &&= \sigma^7 &&=
            (1\;8\;3\;10\;5\;12\;7\;2\;9\;4\;11\;6\;13) \\
         &\sigma^{-6} &&= \sigma^6 &&= (1\;7)
            (1\;8\;3\;10\;5\;12\;7\;2\;9\;4\;11\;6\;13) \\
         &\sigma^{-81} &&= \sigma^3 &&= (1\;4\;7\;10) \\
         &\sigma^{-570} &&= \sigma^6 &&= (1\;7) \\
         &\sigma^{-1211} &&= \sigma
      \end{alignat*}
%%%%%%%%%%%%%%%%%%%%%%%%%%%%%%%%%%%%%2.3.15%%%%%%%%%%%%%%%%%%%%%%%%%%%%%%%%%%%%%
   \item[2.3.15]  Prove that $\Q \times \Q$ is not cyclic.
   
      \textbf{Proof.} Since $\Q$ is infinite and, by Exercise 1.6.6, $\Q$ is not
      isomorphic to $\Z$, it follows that $\Q$ is not cyclic. We know that the
      subgroup of every cyclic group is cyclic; since $\Q \times\{1\} \cong \Q$,
      it follows that $\Q \times \{1\}$ is not cyclic; thus $\Q \times \Q$ is
      not cyclic because it has a noncyclic subgroup, namely $\Q \times \{1\}$.
      \qed
%%%%%%%%%%%%%%%%%%%%%%%%%%%%%%%%%%%%%2.3.16%%%%%%%%%%%%%%%%%%%%%%%%%%%%%%%%%%%%%
   \item[2.3.16]  Assume $|x| = n$ and $|y| = m$. Suppose that $x$ and $y$
                  \textit{commute}: $xy = yx$. Prove that $|xy|$ divides the
                  least common multiple of $m$ and $n$. Need this be true if $x$
                  and $y$ do \textit{not} commute? Give an example of commuting
                  elements $x$, $y$ such that the order of $xy$ is not equal to
                  the least common multiple of $|x|$ and $|y|$.
                  
      \textbf{Proof.} Let $l = \text{lcm}(m, n)$. So there exist integers
      $m'$ and $n'$ such that $mm' = nn' = l$. So we have that
      $$(xy)^l = x^ly^l = x^{nn'}y^{mm'} = (x^n)^{n'}(y^m)^{m'} = 1.$$
      That is $|xy|$ divides $l$ (by Proposition 3, Page 55).
      
      \textit{Need this be true if $x$ and $y$ do not commute?} No! Let
      $$
         A = \left(\begin{tabular}{@{}cc@{}}
            0 & 1/2 \\
            2 & 0
         \end{tabular}\right) \text{ and }
         B = \left(\begin{tabular}{@{}cc@{}}
            0 & 1 \\
            1 & 0
         \end{tabular}\right).
      $$
      A simple computation will show us that although $|A| = |B| = 2$, we have
      that $|AB| = \infty$.
      
      \textbf{Example.} Consider $\Z/2\Z = \{0, 1\}$. Let $x = y = 1$. Then we
      have $|x| = |y| = 2$, so that lcm($|x|, |y|) = 2 \neq |x + y| = |0| = 1$.
      \qed
%%%%%%%%%%%%%%%%%%%%%%%%%%%%%%%%%%%%%2.3.17%%%%%%%%%%%%%%%%%%%%%%%%%%%%%%%%%%%%%
   \item[2.3.17]  Find a presentation for $Z_n$ with one generator.
   
      \textbf{Solution.} $Z_n = \cyc{x : x^n = 1}$.
%%%%%%%%%%%%%%%%%%%%%%%%%%%%%%%%%%%%%2.3.18%%%%%%%%%%%%%%%%%%%%%%%%%%%%%%%%%%%%%
   \item[2.3.18]  Show that if $H$ is any group and $h$ is an element of $H$
                  with $h^n = 1$, then there is a unique homomorphism from
                  $Z_n = \cyc{x}$ to $H$ such that $x \mapsto h$.
                  
      \textbf{Proof.} Let $n \in \Z^+$, $Z_n = \cyc{x}$, $H$ a group, and
      $h^n  = 1$ for some $h \in H$. First we shall show the existence of a
      homomorphism from $Z_n$ to $H$ such that $x \mapsto h$. So consider the
      map $\alpha : \cyc{x} \rightarrow H$ defined by $\alpha(x^a) = h^a$.
      Clearly $\alpha(x) = h$. Now we will show that $\alpha$ is well defined.
      Suppose $x^w = x^y$ for some $x^w, x^y \in Z_n$. Thus $w = y + nk$ for
      some integer $k$. Thus
      $$\alpha(x^w) = \alpha(x^{y+nk})=h^{y+nk}=h^{y}{h^n}^k =h^y=\alpha(x^y),$$
      so that $\alpha$ is well defined. Now we have that
      $$\alpha(x^px^q)=\alpha(x^{p+q})=h^{p+q}=h^ph^q=\alpha(x^p)\alpha(x^q),$$
      so that $\alpha$ is an homomorphism. Now to show uniqueness, we suppose
      that $\phi : \cyc{x} \rightarrow H$ is an homommorphism such that
      $\phi(x) = h$. Since $\phi$ is a homomorphism, it follows that
      $\phi(x^a) = h^a$. Thus $\phi = \alpha$, as desired. \qed
%%%%%%%%%%%%%%%%%%%%%%%%%%%%%%%%%%%%%2.3.19%%%%%%%%%%%%%%%%%%%%%%%%%%%%%%%%%%%%%
   \item[2.3.19]  Show that if $H$ is any group and $h$ is an element of $H$,
                  then there is a unique homomorphism from $\Z$ to $H$ such that
                  $1 \mapsto h$.
                  
      \textbf{Proof.} Let $H$ be a group and let $h \in H$. First we shall show
      that there exists a homomorphism from $\Z$ to $H$ such that $1 \mapsto h$.
      So consider the map $\alpha : \Z \rightarrow H$ defined by
      $n \mapsto h^n$. Clearly $\alpha(1) = h$ and
      $$\alpha(x+y) = h^{x+y} = h^xh^y = \alpha(x)\alpha(y) \text{ for all }
        x, y \in \Z^+,$$
      so that $\alpha$ is a homomorphism. To show uniqueness, suppose that
      $\alpha' : \Z \rightarrow H$ is an homomorphism such that
      $\alpha'(1) = h$. Then according to Exercise 1.6.1, we have that
      $\alpha'(n) = \alpha'(n\cdot1) = \alpha'(1)^n = h^n$ for all $n \in \Z$;
      that is, $\alpha' = \alpha$, as desired. \qed
%%%%%%%%%%%%%%%%%%%%%%%%%%%%%%%%%%%%%2.3.20%%%%%%%%%%%%%%%%%%%%%%%%%%%%%%%%%%%%%
   \item[2.3.20]  Let $p$ be a prime and let $n$ be a positive integer. Show
                  that if $x$ is an element of the group $G$ such that
                  $x^{p^n} = 1$ then $|x| = p^m$ for some $m \le n$.
                  
      \textbf{Proof.} Suppose that $x \in G$ such that $x^{p^n} = 1$. Then it
      follows by Proposition 3 (Page 55) that $|x|$ divides $p^n$. Since $p$ is
      a prime, its factors are $p^i$, $0 \le i \le n$. Thus $|x| = p^m$ for
      some nonnegative $m$ not greater than $n$. \qed
%%%%%%%%%%%%%%%%%%%%%%%%%%%%%%%%%%%%%2.3.21%%%%%%%%%%%%%%%%%%%%%%%%%%%%%%%%%%%%%
   \item[2.3.21]  Let $p$ be an odd prime and let $n$ be a positive integer
                  $\ge 2$. Use the Binomial Theorem to show that
                  $(1+p)^{p^{n-1}} \equiv 1$ (mod $p^n$) but
                  $(1+p)^{p^{n-2}} \not\equiv 1$ (mod $p^n$). Deduce that $1+p$
                  is an element of order $p^{n-1}$ in the multiplicative group
                  $(\Z/p^n\Z)^\times$.

      \textbf{Lemma 2.3.1.} \textit{For an integer $n \ge 2$ and an odd prime
      $p$, let $f_p(n)$ be the number of $p$ factors of $n!$ (i.e., the greatest
      nonnegative integer $j$ such that $p^j \mid i!$), then it follows that
      $f_p(n) < \D\frac{n}{2}$}.

      \textbf{Proof.} Let $n \ge 2$ be an integer and $p$ an odd prime. For a
      a positive integer $r$, let $g_p(n, r)$ be the number of positive
      integers, less than or equal to $n$, that have at least $r$ number of $p$ 
      factors. It follows that $g_p(n, r) = \D\gint{\frac{n}{p^r}}$, where
      $\gint{x}$ is the greatest integer less than or equal to $x$. Finally let
      $k_n$ be the maximum nonnegative integer such that $p^{k_n}$ is a multiple
      of some positive integer not greater than $n$. Thus we have that
      \begin{align*}
         f_p(n) &= g_p(n, 1) + g_p(n, 2) + \cdots + g_p(n, k_n) \\
            &= \sum_{i=1}^{k_n} g_p(n, i)
            = \sum_{i=1}^{k_n} \gint{\frac{n}{p^i}} \\
            &\le \sum_{i=1}^{k_n} \frac{n}{p^i}
            < \sum_{i=1}^\infty \frac{n}{p^i} \\
            &= \frac{n}{p-1} &[\text{Sum of Geometric Series}] \\
            &< \frac{n}{2}. &[\text{Since }p \ge 3]
      \end{align*}

      So we can write $n! = p^{f_p(n)} h_n$ for some $h_n \in \Z^+$, so that
      $(h_n, p) = 1$.

      Now we are ready to commence the proof of the problem. By the Binomial
      Theorem, we have that
      \begin{align*}
         (1+p)^{p^{n-1}} &= \sum_{i=0}^{p^{n-1}}\binom{p^{n-1}}{i}p^i \\
            &= \sum_{i=0}^{p^{n-1}}p^i\frac{p^{n-1}(p^{n-1}-1)(p^{n-1}-2)
               \cdots(p^{n-1}-i+1)}{i!} \\
            &= \sum_{i=0}^{p^{n-1}}p^i\frac{p^{n-1}(p^{n-1}-1)(p^{n-1}-2)
               \cdots(p^{n-1}-i+1)}{p^{f_p(i)} h_i} \\
            &= 1 + p^n + p^n\sum_{i=2}^{p^{n-1}}\frac{p^{i-1}(p^{n-1}-1)
               (p^{n-1}-2) \cdots(p^{n-1}-i+1)}{p^{f_p(i)} h_i}.
      \end{align*}
      Now $f_p(i) < i / 2 \le i - 1$ for $i \ge 2$. Thus $i - 1 - f_p(i) \ge 0$
      (so that $p^{i - 1 - f_p(i)}$ is an integer) if $i \ge 2$. We then have
      \begin{equation} \label{2_3_21_1}
         (1+p)^{p^{n-1}} = 1 + p^n + p^n\sum_{i=2}^{p^{n-1}}\frac{p^{i-1-f_p(i)}
        (p^{n-1}-1)(p^{n-1}-2) \cdots(p^{n-1}-i+1)}{h_i}
      \end{equation}
      Since $(h_i, p) = 1$, it follows that $h_i$ must divide
      $p^{i-1}(p^{n-1}-1)(p^{n-1}-2) \cdots(p^{n-1}-i+1)$. Hence
      $$\sum_{i=2}^{p^{n-1}}\frac{p^{i-1-f_p(i)}
        (p^{n-1}-1)(p^{n-1}-2) \cdots(p^{n-1}-i+1)}{h_i}$$
      is an integer and we can conclude from \eqref{2_3_21_1} that
      $(1+p)^{p^{n-1}} \equiv 1$ (mod $p^n$). Now we have that
      \begin{align*}
         (1+p)^{p^{n-2}} &= \sum_{i=0}^{p^{n-2}}\binom{p^{n-2}}{i}p^i \\
            &= \sum_{i=0}^{p^{n-2}}p^i\frac{p^{n-2}(p^{n-2}-1)(p^{n-2}-2)
               \cdots(p^{n-2}-i+1)}{i!} \\
            &= 1 + p^{n-1} + p^n\frac{p^{n-2}-1}{2} + p^n\frac{p(p^{n-2}-1)(p^{n-2}-2)}{3!} +\sum_{i=4}^{p^{n-1}}p^i\frac{p^{n-2}(p^{n-2}-1)(p^{n-2}-2)
               \cdots(p^{n-2}-i+1)}{p^{f_p(i)} h_i} \\
            &= 1 + p^n + p^n\sum_{i=2}^{p^{n-1}}\frac{p^{i-1}(p^{n-1}-1)
               (p^{n-1}-2) \cdots(p^{n-1}-i+1)}{p^{f_p(i)} h_i}.
      \end{align*}
      
%%%%%%%%%%%%%%%%%%%%%%%%%%%%%%%%%%%%%2.3.22%%%%%%%%%%%%%%%%%%%%%%%%%%%%%%%%%%%%%
   \item[2.3.22]  Let $n$ be an integer $\ge 3$. Use the Binomial Theorem to
                  show that $(1+2^2)^{2^{n-2}} \equiv 1$ (mod $2^n$) but
                  $(1+2^2)^{2^{n-3}} \not\equiv 1$ (mod $2^n$). Deduce that 5 is
                  an element of order $2^{n-2}$ in the multiplicative group
                  $(\Z/2^n\Z)^\times$.

      \textbf{Proof.}
%%%%%%%%%%%%%%%%%%%%%%%%%%%%%%%%%%%%%2.3.23%%%%%%%%%%%%%%%%%%%%%%%%%%%%%%%%%%%%%
   \item[2.3.23]  Show that $(\Z/2^n\Z)^\times$ is not cyclic for any $n \ge 3$.
                  [Find two distinct subgroups of order 2.]
%%%%%%%%%%%%%%%%%%%%%%%%%%%%%%%%%%%%%2.3.24%%%%%%%%%%%%%%%%%%%%%%%%%%%%%%%%%%%%%
   \item[2.3.24]  Let $G$ be a finite group and let $x \in G$.
                  \begin{enumerate}
                     \item Prove that if $g \in N_G(\cyc{x})$ then
                           $gxg^{-1} = x^a$ for some $a \in \Z$. 
                     \item Prove conversely that if $gxg^{-1} = x^a$ for some
                           $a \in \Z$ then $g \in N_G(\cyc{x})$. [Show first
                           that $gx^kg^{-1} = (gxg^{-1})^k = x^{ak}$ for any
                           integer $k$, so that $g\cyc{x}g^{-1} \le \cyc{x}$.
                           If $x$ has order $n$, show the elements $gx^ig^{-1}$,
                           $i = 0, 1, \ldots, n-1$ are distinct, so that
                           $|g\cyc{x}g^{-1}| = |\cyc{x}| = n$ and conclude that
                           $g\cyc{x}g^{-1} = \cyc{x}$.]
                  \end{enumerate}
                  Note that this cuts down some of the work in computing
                  normalizers of cyclic subgroups since one does not have to
                  check $ghg^{-1} \in \cyc{x}$ for every $h \in \cyc{x}$.
%%%%%%%%%%%%%%%%%%%%%%%%%%%%%%%%%%%%%2.3.25%%%%%%%%%%%%%%%%%%%%%%%%%%%%%%%%%%%%%
   \item[2.3.25]  Let $G$ be a cyclic group of order $n$ and let $k$ be an
                  integer relatively prime to $n$. Prove that the map
                  $x \mapsto x^k$ is surjective. Use Lagrange's Theorem
                  (Exercise 1.7.19) to prove the same is true for any finite
                  group of order $n$. (For such $k$ each element has a
                  $k^{\text{th}}$ root in $G$. It follows from Cauchy's Theorem
                  in Section 3.2 that if $k$ is not relatively prime to the
                  order of $G$ then the map $x \mapsto x^k$ is not surjective.)
%%%%%%%%%%%%%%%%%%%%%%%%%%%%%%%%%%%%%2.3.26%%%%%%%%%%%%%%%%%%%%%%%%%%%%%%%%%%%%%
   \item[2.3.26]  Let $Z_n$ be a cyclic group of order $n$ and for each integer
                  $a$ let
                  $$\sigma_a : Z_n \mapsto Z_n \qquad by \qquad \sigma_a(x) =
                  x^a \quad \text{for all } x \in Z_n.$$
                  \begin{enumerate}
                     \item Prove that $\sigma_a$ is an automorphism of $Z_n$ if
                           and only if $a$ and $n$ are relatively prime(
                           automorphisms were introduced in Exercise 1.6.20).
                     \item Prove that $\sigma_a = \sigma_b$ if and only if
                           $a \equiv b$ (mod $n$).
                     \item Prove that \textit{every} automorphism of $Z_n$ is
                           equal to $\sigma_a$ for some integer $a$.
                     \item Prove that $\sigma_a\circ\sigma_b=\sigma_{ab}$.
                           Deduce that the map $\overline{a} \mapsto \sigma_a$
                           is an isomorphism of $(\Z/n\Z)^\times$ onto the
                           automorphism group of $Z_n$ (so Aut($Z_n$) is an
                           abelian group of order $\varphi(n)$).
                  \end{enumerate}
                  %%%%%MISSING CONTAINMENT%%%%%%%%
\end{enumerate}


































         
   \part{}

   \chapter{Introduction To Groups}
      \section{Basic Axioms And Examples}
         \begin{enumerate}
   \item[]        Let $G$ be a group.
%%%%%%%%%%%%%%%%%%%%%%%%%%%%%%%%%%%Lemm1.1.1%%%%%%%%%%%%%%%%%%%%%%%%%%%%%%%%%%%%
   \item[]        \textbf{Lemma 1.1.1} Let $x \in G$ and let $m$ be an integer.
                  Then we have that
                  $$x^{m+1} = x^mx^1.$$

      \textbf{Proof.}  Consider the following cases:

      \textbf{Case 1:} \textit{$m = 0$}. It follows that
      $$x^{m+1} = x^{0+1} = x^1 = 1x^1 = x^0x^1 = x^mx^1.$$

      \textbf{Case 2:} \textit{$m$ is positive.} Then it follows that $m + 1$ is 
      positive, so that
      \begin{align*}
         x^{m + 1} &= \underbrace{x \cdot x \cdots x}_{m+1\text{ factors}} \\
            &= \underbrace{x \cdot x \cdots x}_{m\text{ factors}} \cdot x^1 \\
            &= x^mx^1.
      \end{align*}

      \textbf{Case 3:} \textit{$m$ is negative.} If $m = -1$, then we have that
      $$x^{m+1} = x^{-1+1} = x^0 = 1 = x^{-1}x^1 = x^mx^1.$$
      If $m < -1$, then $m + 1$ is negative, so that $-(m + 1) = -m - 1$ is 
      positive. Thus
      \begin{align*}
         x^mx^1 &= x^{-(-m)}x^1 \\
                &= \underbrace{x^{-1} \cdot x^{-1} \cdots x^{-1}}_{
                   -m\text{ factors}} \cdot x^1 \\
                &= \underbrace{x^{-1} \cdot x^{-1} \cdots x^{-1}}_{
                   -m-1\text{ factors}} \cdot (x^{-1} \cdot x^1) \\
                &= \underbrace{x^{-1} \cdot x^{-1} \cdots x^{-1}}_{
                   -m-1\text{ factors}} \\
                &= x^{-(-m-1)} \\
                &= x^{m+1}.
      \end{align*}

      In all cases, we can see that our assertion holds. \qed
%%%%%%%%%%%%%%%%%%%%%%%%%%%%%%%%%%%Lemm1.1.2%%%%%%%%%%%%%%%%%%%%%%%%%%%%%%%%%%%%
   \item[]        \textbf{Lemma 1.1.2} \textit{Let $x$ and $g$ be members of a 
                  group $G$, and let $n$ be a positive integer, then it follows 
                  that $(g^{-1}xg)^n = g^{-1}x^ng$.}

      \textbf{Proof.} We shall show by induction that the equation
      \begin{equation}
         (g^{-1}xg)^n = g^{-1}x^ng \label{l1_1_2_1}
      \end{equation}
      holds for every positive integer $n$. It is clear that equation
      \ref{l1_1_2_1} holds for $n = 1$. So assume that it also holds for some
      positive integer $k$. So we must now show that the equation also holds for 
      $k + 1$. Thus
      \begin{align*}
         (g^{-1}xg)^{k+1} &= (g^{-1}xg)^kg^{-1}xg &[\text{Execise 1.1.19}] \\
                     &= g^{-1}x^kgg^{-1}xg &[\text{Inductive hypothesis}] \\
                     &= g^{-1}x^kxg \\
                     &= g^{-1}x^{k+1}g,
      \end{align*}
      so that equation \eqref{l1_1_2_1} holds for $k+1$. Hence by the Principle 
      of Mathematical Induction, equation \eqref{l1_1_2_1} holds for every 
      positive integer $n$. \qed
%%%%%%%%%%%%%%%%%%%%%%%%%%%%%%%%%%%Lemm1.1.3%%%%%%%%%%%%%%%%%%%%%%%%%%%%%%%%%%%%
   \item[]        \textbf{Lemma 1.1.3} \textit{Let $x$ be an element of finite
                  order $n$ in $G$. If $x^m = 1$, then it follows that
                  $n \mid m$.}

      \textbf{Proof.} Suppose that $x^m = 1$. By the Division Algorithm, there
      exist unique integers $q$ and $r$ such that $m = qn + r$ and
      $0 \le r < n$. Now we have that
      $$1 = x^m = x^{qn+r} = x^{qn}x^r = (x^n)^qx^r = 1^qx^r = x^r.$$
      Since $|x| = n$, we cannot have $0 < r < n$; thus the only remaining
      possibility is $r = 0$, so that $n = qm$, as desired. \qed
%%%%%%%%%%%%%%%%%%%%%%%%%%%%%%%%%%%Lemm1.1.4%%%%%%%%%%%%%%%%%%%%%%%%%%%%%%%%%%%%
   \item[]        \textbf{Lemma 1.1.4} \textit{Let $(x, y)$ be an element of
                  $A \times B$ where $A$ and $B$ are groups. For any positive
                  integer $n$, we then have that $(x, y)^n = (x^n, y^n)$.}

      \textbf{Proof.} We shall induct on $n$. Our assertion clearly holds if
      $n$ is 1, so assume that it holds for some positive integer $k$. Thus we
      have that
      \begin{align*}
         (x, y)^{k+1} &= (x, y)(x, y)^k &[\text{Exercise 1.1.19}] \\
                      &= (x, y)(x^k, y^k) &[\text{Inductive hypothesis}] \\
                      &= (xx^k, yy^k) \\
                      &= (x^{k+1}, y^{k+1}). &[\text{Exercise 1.1.19}]
      \end{align*}
      The above shows that our assertion also holds for $k + 1$, so that by
      the Principle of Mathematical Induction it must holds for every integer
      $n$. \qed
%%%%%%%%%%%%%%%%%%%%%%%%%%%%%%%%%%%%%1.1.1%%%%%%%%%%%%%%%%%%%%%%%%%%%%%%%%%%%%%%
   \item[1.1.1]   Determine which of the following binary operations are
                  associative:
                  \begin{enumerate}
                     \item the operation $*$ on $\Z$ defined by $a * b = a - b$.
                     \item the operation $*$ on $\R$ defined by
                           $a * b = a + b + ab$.
                     \item the operation $*$ on $\Q$ defined by
                           $\displaystyle a * b = \frac{a + b}{5}$.
                     \item the operation $*$ on $\Z \times \Z$ defined by
                           $(a, b) * (c, d) = (ad + bc, bd)$.
                     \item the operation $*$ on $\Q - \{0\}$ defined by
                           $\displaystyle a * b = \frac{a}{b}$.
                  \end{enumerate}
                  
      \textbf{Solution.}
   
      \begin{enumerate}
         \item The binary operation $*$ on $\Z$ is not associative because
               $$(0 * 0) * 1 = -1 \neq 1 = 0 * (0 * 1).$$
         \item We claim that $*$ is associative on $\R$.
      
               \textbf{Proof.} Let $r_1, r_2, r_3 \in \R$. Then it follows that
               \begin{align*}
                  (r_1 * r_2) * r_3 &= (r_1 + r_2 + r_1r_2) * r_3 \\
                     &= (r_1 + r_2 + r_1r_2 + r_3) +
                        (r_1r_3 + r_2r_3 + r_1r_2r_3) \\
                     &= r_1 + r_2 + r_3 + r_1r_2 + r_2r_3 +
                        r_1r_3 + r_1r_2r_3 \\
                     &= (r_1 + r_2 + r_3 + r_2r_3) + r_1(r_2 + r_3 + r_2r_3) \\
                     &= r_1 + (r_2 * r_3) + r_1(r_2 * r_3) \\
                     &= r_1 * (r_2 * r_3),
               \end{align*}
               so that our claim holds. \qed
         \item The binary operation $*$ on $\Q$ is not associative because
               $$(0 * 0) * 25 = 5 \neq 1 = 0 * (0 * 25).$$
         \item We claim that $*$ is associative on $\Z \times \Z$.
      
               \textbf{Proof.} Let $(z_1, z_2)$, $(z_3, z_4)$,
               $(z_5, z_6) \in \Z \times \Z$. Then it follows that
               \begin{align*}
                  (z_1, z_2) * [(z_3, z_4) * (z_5, z_6)] 
                     &= (z_1, z_2) * [(z_3z_6 + z_4z_5, z_4z_6)] \\
                     &= (z_1z_4z_6 + z_2z_3z_6 + z_2z_4z_5, z_2z_4z_6) \\
                     &= ((z_1z_4 + z_2z_3) \cdot z_6 + z_2z_4 \cdot z_5,
                          z_2z_4 \cdot z_6) \\
                     &= (z_1z_4 + z_2z_3, z_2z_4) * (z_5, z_6) \\
                     &= [(z_1, z_2) * (z_3, z_4)] * (z_5, z_6),
               \end{align*}
               so that our claim holds. \qed
         \item The binary operation $*$ on $\Q - \{0\}$ is not associative
               because
               $$(4 * 1) * 2 = 2 \neq 8 = 4 * (1 * 2).$$
      \end{enumerate}
%%%%%%%%%%%%%%%%%%%%%%%%%%%%%%%%%%%%%1.1.2%%%%%%%%%%%%%%%%%%%%%%%%%%%%%%%%%%%%%%
   \item[1.1.2]   Decide which of the binary operations in the preceding
                  exercise are commutative.
                  
      \begin{enumerate}      
         \item The binary operation $*$ on $\Z$ is not commutative because
               $$1 * 0 = 1 \neq -1 = 0 * 1.$$
         \item The binary operation $*$ on $\R$ is commutative because addition
               and multiplication are commutative on $\R$.
         \item The binary operation $*$ on $\Q$ is commutative because addition
               is commutative on $\Q$.
         \item A quick check will show us that $*$ is commutative on
               $\Z \times \Z$. That is, for all $(z_1, z_2)$, $(z_3, z_4)$
               $\in \Z \times \Z$, we must have that
               \begin{align*}
                  (z_1, z_2) * (z_3, z_4) &= (z_1z_4 + z_2z_3, z_2z_4) \\
                                          &= (z_3z_2 + z_4z_1, z_4z_2) \\
                                          &= (z_3, z_4) * (z_1, z_2).
               \end{align*}
         \item The binary operation $*$ on $\Q - \{0\}$ is not commutative
               because
               $$1 * 2 = \frac{1}{2} \neq \frac{2}{1} = 2 * 1.$$
      \end{enumerate}
%%%%%%%%%%%%%%%%%%%%%%%%%%%%%%%%%%%%%%1.3%%%%%%%%%%%%%%%%%%%%%%%%%%%%%%%%%%%%%%%
   \item[1.1.3]   Prove that addition of residue classes in $\Z/n\Z$ is
                  associative (you may assume it is well defined).
                  
      \textbf{Proof.} Fix $n \in \Z^+$. Consider $\overline{a}$, $\overline{b}$,
      and $\overline{c}$ in $\Z/n\Z$. By Theorem 3, Pg. 9, we have that
      \begin{align*}
         \overline{a} + (\overline{b} + \overline{c})
            &= \overline{a} + \overline{b + c} \\
            &= \overline{a + b + c} \\
            &= \overline{a + b} + \overline{c} \\
            &= (\overline{a} + \overline{b}) + \overline{c},
      \end{align*}
      so that addition of residue classes in $\Z/n\Z$ is associative. \qed
%%%%%%%%%%%%%%%%%%%%%%%%%%%%%%%%%%%%%%1.4%%%%%%%%%%%%%%%%%%%%%%%%%%%%%%%%%%%%%%%
   \item[1.1.4]   Prove that multiplication of residue classes in $\Z/n\Z$ is
                  associative (you may assume it is well defined).
                  
      \textbf{Proof.} Fix $n \in \Z^+$. Consider $\overline{a}$, $\overline{b}$,
      and $\overline{c}$ in $\Z/n\Z$. By Theorem 3, Pg. 9, we have that
      \begin{align*}
         \overline{a} \cdot (\overline{b} \cdot \overline{c})
            &= \overline{a} \cdot \overline{bc} \\
            &= \overline{abc} \\
            &= \overline{ab} \cdot \overline{c} \\
            &= (\overline{a} \cdot \overline{b}) \cdot \overline{c},
      \end{align*}
      so that multiplication of residue classes in $\Z/n\Z$ is associative. \qed
%%%%%%%%%%%%%%%%%%%%%%%%%%%%%%%%%%%%%%1.5%%%%%%%%%%%%%%%%%%%%%%%%%%%%%%%%%%%%%%%
   \item[1.1.5]   Prove that for all $n > 1$ that $\Z/n\Z$ is not a group under
                  multiplication of residue classes.
                  
      \textbf{Proof.} Let $n$ be positive integer greater than 1. It follows
      that $\Z/n\Z$ is not a group under multiplication because $\overline{0}$
      has no multiplicative inverse. \qed
%%%%%%%%%%%%%%%%%%%%%%%%%%%%%%%%%%%%%%1.6%%%%%%%%%%%%%%%%%%%%%%%%%%%%%%%%%%%%%%%
   \item[1.1.6]   Determine which of the following sets are groups under
                  addition:
                  \begin{enumerate}
                     \item the set of rational numbers (including $0 = 0/1$) in
                           lowest terms whose denominators are odd.
                     \item the set of rational numbers (including $0 = 0/1$) in
                           lowest terms whose denominators are even.
                     \item the set of rational numbers of absolute value $< 1$.
                     \item the set of rational numbers of absolute value $\ge 1$
                           together with 0.
                     \item the set of rational numbers with denominators equal
                           to 1 or 2.
                     \item the set of rational numbers with denominators equal
                           to 1, 2, or 3.
                  \end{enumerate}

      \textbf{Solution.}

      \begin{enumerate}
         \item We claim that the set
               $$S = \left\{\frac{a}{b} \in \Q : b \text{ is odd} \text{ and }
                 \gcd(a, b) = 1\right\},$$
               is a group under addition.

               \textbf{Proof.} First we must show that $S$ is closed under 
               addition. Notice that $S$ is nonempty since it contains 7/5, so 
               let $r, s \in S$. By definition of $S$, we have that
               $r = a_1/b_1$ and $s = a_2/b_2$ for some integers $a_1$ and
               $a_2$, and nonzero integers $b_1$ and $b_2$, where $b_1$ and
               $b_2$ are odd and $\gcd(a_1, b_1) = \gcd(a_2, b_2) = 1$.
               It follows that
               \begin{align*}
                  r + s &= \frac{a_1}{b_1} + \frac{a_2}{b_2} \\
                        &= \frac{a_1b_2 + a_2b_1}{b_1b_2}.
               \end{align*}

               Since $b_1$ and $b_2$ are both odd, it must necessarily be the 
               case that $b_1b_2$ is also odd. In order words, $b_1b_2$ contains 
               no factor of 2, so that if we reduce $r + s$ to its lowest term, 
               the denominator of this lowest term will still be odd. Hence
               $r + s \in S$, so that $S$ is closed under addition. To complete 
               the proof we must now show that $S$ satisfies the group axioms. 
               We observe that $0/1$ is the identity element in $S$. Also, it is 
               clear that for all $s \in S$, we have $-s \in S$, so that every 
               element of $S$ has an inverse under addition. Since
               $S \subseteq \Q$, and since $\Q$ is associative under addition, 
               it follows that $S$ is also associative under addition. Thus $S$ 
               satisfies the group axioms, so that $(S, +)$ is a group. \qed
         \item The set
               $$S = \left\{\frac{a}{b} \in \Q : b \text{ is even} \text{ and }
                 \gcd(a, b) = 1\right\},$$
               is not a group under addition because it is not closed. Indeed,
               for $3/14 \in S$, we have $3/14 + 3/14 = 3/7 \notin S$.
         \item The set
               $$S = \left\{\frac{a}{b} \in \Q :
                     \left|\frac{a}{b}\right| < 1\right\},$$
               is not a group under addition because it is not closed. Indeed,
               for $9/10 \in S$, we have $9/10 + 9/10 = 18/10 \notin S$.
         \item The set
               $$S = \left\{\frac{a}{b} \in \Q : a = 0 \text{ or }
                     \left|\frac{a}{b}\right| \ge 1\right\},$$
               is not a group under addition because it is not closed. Indeed,
               for $-11/10, 10/10 \in S$, we have
               $-11/10 + 10/10 = -1/10 \notin S$.
         \item We claim that the set
               $$S = \left\{\frac{a}{b} \in \Q : b = 1 \text{ or }
                 b = 2\right\},$$
               is a group under addition.

               \textbf{Proof.} It is clear that 0 is the identity for $S$ under
               addition, that $S$ is associative under addition (because
               $S \subset \Q$ and $\Q$ is associative under addition, and that
               the inverse of an element in $S$ is its additive inverse in $\Q$.
               So to complete the proof, we need only show that $S$ is closed
               under addition. Let $a_1/b_1, a_2/b_2 \in \Q$. By observation, we
               note that $a_1/b_1 + a_2/b_2$ must have a denominator of 1 or 2,
               so that it is in $S$. Thus $S$ is closed under addition. \qed
         \item The set
               $$S = \left\{\frac{a}{b} \in \Q : b \in {1, 2, 3} \right\},$$
               is not a group under addition because it is not closed. Indeed,
               for $1/2, 1/3 \in S$, we have $1/2 + 1/3 = 5/6 \notin S$.
      \end{enumerate}
%%%%%%%%%%%%%%%%%%%%%%%%%%%%%%%%%%%%%%1.7%%%%%%%%%%%%%%%%%%%%%%%%%%%%%%%%%%%%%%%
   \item[1.1.7]   Let $G = \{x \in \R : 0 \le x < 1\}$ and for $x, y \in G$ let
                  $x * y$ be the fractional part of $x + y$ (i.e.,
                  $x * y = x + y = [x + y]$ where $[a]$ is the greatest integer
                  less than or equal to $a$). Prove that $*$ is a well defined
                  binary operation on $G$ and that $G$ is an abelian group under
                  $*$ (called the \textit{real numbers mod }1).
                  
      \textbf{Proof.} The set $G$ is clearly non-empty, so consider
      $x, y, z \in G$. To show that $G$ is a group, we shall now prove that it 
      is well defined, associative, has an identity, and is closed under taking
      inverses.

      \textbf{Well Defined:} To show that $*$ is well defined is tantamount to
      showing that $G$ is closed under $*$.  By definition, we have that
      $0 \le x < 1$ and $0 \le y < 1$, so that $0 \le x + y < 2$. If
      $0 \le x + y < 1$, so that $[x + y] = 0$, then we have that
      $$0 \le x + y = x + y - [x + y] = x * y = x + y < 1.$$
      However if $1 \le x + y < 2$, so that $[x + y] = 1$ and
      $0 \le x + y - 1 < 1$, we must have that
      $$0 \le x + y - 1 = x + y - [x + y] = x * y = x + y - 1 < 1.$$
      In either case, we have $0 \le x * y < 1$; i.e. $x * y \in G$, so that $G$ 
      is closed under $*$. Also we have that
      $$x * y = x + y - [x + y] = y + x - [y + x],$$
      so that $G$ is abelian.

      \textbf{Associativity:} We have that
      \begin{align*}
         x * (y * z) &= x * (y + z - [y + z]) \\
              &= x + y + z - [y + z] - [x + y + z - [y + z]], \text{ and} \\ \\
         (x * y) * z &= (x + y - [x + y]) * z \\
                     &=  x + y + z - [x + y] - [x + y + z - [x + y]].
      \end{align*}
      By definition, we have that $0 \le x < 1$, $0 \le y < 1$, and
      $0 \le z < 1$, so that $0 \le x + y < 2$ and $0 \le y + z < 2$. Let us 
      now investigate the following possible cases:

      \textit{Case 1:} \textit{$0 \le x + y  < 1$ and $0 \le y + z < 1$}. That
      is $[x + y] = [y + z] = 0$. It then follows that
      $$x * (y * z) = (x * y) * z = x + y + z - [x + y + z].$$

      \textit{Case 2:} \textit{$1 \le x + y  < 2$ and $1 \le y + z < 2$}. That
      is $[x + y] = [y + z] = 1$. It then follows that
      $$x * (y * z) = (x * y) * z = x + y + z - 1 - [x + y + z - 1].$$

      \textit{Case 3:} \textit{$0 \le x + y  < 1$ and $1 \le y + z < 2$}. That
      is $[x + y] = 0$, and $[y + z] = 1$. It then follows that
      $$(x * y) * z = x + y + z - [x + y + z].$$
      Since $0 \le x + y < 1$ and $0 \le z < 1$, we must have that
      $0 \le x + y + z < 2$. Similarly, since $1 \le y + z < 2$ and
      $0 \le x < 1$, we must have that $1 \le x + y + z < 3$, and since we 
      already showed that $0 \le x + y + z < 2$, it follows that
      $1 \le x + y + z < 2$. Hence $[x + y + z] = 1$. We can then conclude that 
      $(x * y) * z = x +y + z - 1$. Now we have that
      $$x * (y * z) = x + y + z - 1 - [x + y + z - 1].$$
      We already showed that $1 \le x + y + z < 2$; thus,
      $0 \le x + y + z - 1 < 1$, so that $[x + y + z - 1] = 0$; that is,
      $$x * (y * z) = x + y + z - 1 = (x * y) * z.$$
   
      \textit{Case 4:} \textit{$1 \le x + y  < 2$ and $0 \le y + z < 1$}. Apply
      Case 3, with the roles of $x + y$ and $y + z$ interchanged.

      We have thus shown that in all possible cases, we have
      $$x * (y * z) = (x * y) * z,$$
      so that $G$ is associative under $*$.


      \textbf{Identity:} We observe that $0 \in G$ is the identity element since
      $$x * 0 = x + 0 - [x + 0] = x - [x] = x - 0 = x.$$

      \textbf{Inverse:} Suppose $x \neq 0$, so that $0 < x < 1$, and thus
      $0 < 1 - x < 1$; that is $1 - x \in G$. It follows that
      $$x * (1 - x) = x + (1 - x) + [x + (1 - x)] = 1 - 1 = 0,$$
      so that $1 - x$ is the inverse of $x \in G$, with $x \neq 0$. Clearly, the 
      inverse of 0 is 0. \\

      We can now conclude that $(G, *)$ is a group. \qed
%%%%%%%%%%%%%%%%%%%%%%%%%%%%%%%%%%%%%%1.8%%%%%%%%%%%%%%%%%%%%%%%%%%%%%%%%%%%%%%%
   \item[1.1.8]   Let $G = \{z \in \C : z^n = 1 \text{ for some } n \in \Z^+\}$.
                  \begin{enumerate}
                     \item Prove that $G$ is a group under multiplication
                           (called the group of \textit{roots of unity} in
                           $\C$).
                     \item Prove that $G$ is not a group under addition.
                  \end{enumerate}
                  
      \textbf{Proof.}
      
      \begin{enumerate}
         \item We observe that 1 is the identity element of $G$, so that $G$ is
               not empty. So let $x, y, z \in G$.
               
               \textbf{Closure:} By definition, there exist positive integers
               $m$ and $n$ such that $x^m = y^n = 1$. Thus
               $(xy)^{mn} = (x^m)^n(y^n)^m = 1^n1^m = 1$. This says that $G$ is
               closed under multiplication.
               
               \textbf{Associativity:} Since $\C$ is associative under
               multiplication and since $G \subseteq \C$, it follows that $G$ is
               associative under multiplication.
               
               \textbf{Identity:} As state above, the identity of $G$ is clearly
               1.
               
               \textbf{Inverse:} Notice that since
               $(x^{m - 1})^m = (x^m)^{m - 1} = 1$, we must have that
               $x^{m - 1} \in G$. Thus we have $x^{m - 1}x = x^m = 1$; i.e., the
               inverse of $x$ is $x^{m - 1}$.
               
               We have thus shown that $G$ is a group under multiplication. \qed
         \item $G$ is not a group under addition because it is not closed under
               addition. In particular, we have $1 \in G$, but
               $1 + 1 = 2 \notin G$ because $2^n \neq 1$ for any positive
               integer.
      \end{enumerate}
%%%%%%%%%%%%%%%%%%%%%%%%%%%%%%%%%%%%%%1.9%%%%%%%%%%%%%%%%%%%%%%%%%%%%%%%%%%%%%%%
   \item[1.1.9]   Let $G = \{a + b\sqrt{2} \in \R : a, b \in \Q\}$.
                  \begin{enumerate}
                     \item Prove that $G$ is a group under addition.
                     \item Prove that the nonzero elements of $G$ are a group 
                           under multiplication. [``Rationalize the
                           denominators" to find multiplicative inverse.]
                  \end{enumerate}
                  
      \textbf{Proof.}
      
      \begin{enumerate}
         \item \textbf{Closure:} $G$ is clearly nonempty, so let $x, y \in G$.
               By definition of $G$, it follows that $x = a_1 + b_1\sqrt{2}$ and
               $y = a_2 + b_2\sqrt{2}$ for some rational numbers $a_1$, $b_1$,
               $a_2$, and $b_2$. Thus
               $$x + y = (a_1 + a_2) + (b_1 + b_2)\sqrt{2} \in G,$$
               so that $G$ is closed under addition.
               
               \textbf{Associativity:} Since $\R$ is associative under addition
               and since $G \subseteq \R$, it follows that $G$ is associative
               under addition.
               
               \textbf{Identity:} The identity of $G$ is 0.
               
               \textbf{Inverse:} For an element $x = a_1 + b_1\sqrt{2} \in G$,
               the additive inverse of $x$ is $-a_1 - b_1\sqrt{2} \in G$.
               
               We have thus shown that $G$ is a group under addition. \qed
         \item Let $G^{\times}$ denote the set of nonzero elements of $G$.
         
               \textbf{Closure:} Let $x, y \in G^{\times}$. By definition of
               $G$, it follows that $x = a_1 + b_1\sqrt{2}$ and
               $y = a_2 + b_2\sqrt{2}$ for some rational numbers $a_1$, $b_1$,
               $a_2$, and $b_2$, with $a_1$ and $b_1$ not both zero and $a_2$
               and $b_2$ not both zero. Thus
               $$xy = (a_1a_2 + 2b_1b_2) + (a_1b_2 + a_2b_1)\sqrt{2}.$$
               Since neither $x$ nor $y$ is zero, it must be the case that $xy$
               is not zero, so that $G^{\times}$ is closed under multiplication.
               
               \textbf{Associativity:} Since $\R$ is associative under
               multiplication and since $G^{\times} \subseteq \R$, it follows
               that $G^{\times}$ is associative under multiplication.
               
               \textbf{Identity:} The element $1 = 1 + 0\sqrt{2} \in G^{\times}$
               is the identity of $G^{\times}$.
               
               \textbf{Inverse:} Let $x = a_1 + b_1\sqrt{2} \in G^{\times}$.
               Since $x \neq 0$, the real number $1/x$ exists, and we have that
               $$\frac{1}{x} = \frac{1}{a_1 + b_1\sqrt{2}}
                 \frac{a_1 - b_1\sqrt{2}}{a_1 - b_1\sqrt{2}} =
                 \left(\frac{a_1}{{a_1}^2 - 2{b_1}^2} -
                 \frac{b_1}{{a_1}^2 - 2{b_1}^2}\sqrt{2}\right) \in G^{\times}.
               $$
               
               Since $1/x \in G^{\times}$ and since $x \cdot 1/x = 1$, we have
               that $1/x$ is the multiplicative inverse of $x$.
               
               We have thus shown that $G^{\times}$ is a group under
               multiplication. \qed
      \end{enumerate}
%%%%%%%%%%%%%%%%%%%%%%%%%%%%%%%%%%%%%%1.10%%%%%%%%%%%%%%%%%%%%%%%%%%%%%%%%%%%%%%
   \item[1.1.10]  Prove that a finite group is abelian if and only if its group
                  table is a symmetric matrix.
                  
      \textbf{Proof.} Let $G$ be a group such that $|G| = n \in \Z^+$, and let
      $(a_{ij})$ denote the matrix of the group table of $G$. Since $G$ is
      finite, we can enumerate the elements of $G$ like so:
      $$G = \{g_1, g_2, \ldots, g_n\}.$$      
      $(\Leftarrow)$ Suppose that $(a_{ij})$ is a symmetric matrix. Let
      $a, b \in G$. Then we have that $a = g_r$ and $b = g_s$ for some
      $r, s \in \{1, 2, \ldots, n\}$. Since $(a_{ij})$ is symmetric, we must
      have that
      $$ab = g_rg_s = a_{rs} = a_{sr} = g_sg_r = ba,$$
      so that $G$ is abelian.
      
      $(\Rightarrow)$ Now suppose that $G$ is abelian. Consider
      $a_{rs} \in (a_{ij})$. It follows that
      $$a_{rs} = g_rg_s = g_sg_r = a_{sr},$$
      so that $(a_{ij})$ is symmetric. \qed      
%%%%%%%%%%%%%%%%%%%%%%%%%%%%%%%%%%%%%%1.11%%%%%%%%%%%%%%%%%%%%%%%%%%%%%%%%%%%%%%
   \item[1.1.11]  Find the orders of each element of the additive group
                  $\Z/12\Z$.
                  
      \textbf{Solution.} The orders of the elements $\overline{0}$,
      $\overline{1}$, $\overline{2}$, $\overline{3}$, $\overline{4}$,
      $\overline{5}$, $\overline{6}$, $\overline{7}$, $\overline{8}$,
      $\overline{9}$, $\overline{10}$, and $\overline{11}$ in $\Z/12\Z$ are
      1, 12, 6, 4, 3, 12, 2, 12, 3, 4, 6, and 12.
%%%%%%%%%%%%%%%%%%%%%%%%%%%%%%%%%%%%%%1.12%%%%%%%%%%%%%%%%%%%%%%%%%%%%%%%%%%%%%%
   \item[1.1.12]  Find the orders of the following elements of the
                  multiplicative group $(\Z/12\Z)^\times: \overline{1},
                  \overline{-1}, \overline{5}, \overline{7}, \overline{-7}, 
                  \overline{13}$.
                  
      \textbf{Solution.} The orders of the elements $\overline{1}$,
      $\overline{-1}$, $\overline{5}$, $\overline{7}$, $\overline{-7}$,
      $\overline{13}$ in $(\Z/12\Z)^\times$ are 1, 11, 5, 7, 5, and 13.
%%%%%%%%%%%%%%%%%%%%%%%%%%%%%%%%%%%%%%1.13%%%%%%%%%%%%%%%%%%%%%%%%%%%%%%%%%%%%%%
   \item[1.1.13]  Find the orders of the following elements of the additive
                  group $\Z/36\Z: \overline{1}, \overline{2}, \overline{6}, 
                  \overline{9}, \overline{10}, \overline{12}, \overline{-1}, 
                  \overline{-10}, \overline{-18}$.
                  
      \textbf{Solution.} The orders of the elements $\overline{1}$,
      $\overline{2}$, $\overline{6}$, $\overline{9}$, $\overline{10}$,
      $\overline{12}$, $\overline{-1}$, $\overline{-10}$, and $\overline{-18}$
      in $\Z/36\Z$ are 1, 18, 6, 4, 18, 3, 36, 18, and 2.
%%%%%%%%%%%%%%%%%%%%%%%%%%%%%%%%%%%%%%1.14%%%%%%%%%%%%%%%%%%%%%%%%%%%%%%%%%%%%%%
   \item[1.1.14]  Find the orders of the following elements of the
                  multiplicative group $(\Z/36\Z)^\times: \overline{1},
                  \overline{-1}, \overline{5}, \overline{13}, \overline{-13},
                  \overline{17}$.
                  
      \textbf{Solution.} The orders of the elements $\overline{1}$,
      $\overline{-1}$, $\overline{5}$, $\overline{13}$, $\overline{-13}$,
      $\overline{17}$ in $(\Z/36\Z)^\times$ are 1, 35, 29, 25, 11, and 17.
%%%%%%%%%%%%%%%%%%%%%%%%%%%%%%%%%%%%%%1.15%%%%%%%%%%%%%%%%%%%%%%%%%%%%%%%%%%%%%%
   \item[1.1.15]  Prove that $(a_1a_2\cdots a_n)^{-1} =
                  {a_n}^{-1}{a_{n-1}}^{-1}\cdots {a_1}^{-1}$ for all
                  $a_1, a_2, \ldots, a_n \in G$.
                  
      \textbf{Proof.} We shall proceed by induction on $n$. The statement is
      trivial for $n = 1$. So assume that it also holds for some positive
      integer $k$. Let $b = a_1a_2\cdots a_k$. It then follows that
      \begin{align*}
         (a_1a_2\cdots a_ka_{k+1})^{-1} &= (b \cdot a_{k+1})^{-1} \\
            &= {a_{k+1}}^{-1}b^{-1} &[\text{By Proposition 1 (4)}] \\
            &= {a_{k+1}}^{-1}{a_k}^{-1}\cdots {a_1}^{-1}.
                  &[\text{Inductive hypothesis}]
      \end{align*}
      That is, our statement holds for $k + 1$, so that, by the Principle of
      Mathematical Induction, it holds for each positive integer $n$. \qed
%%%%%%%%%%%%%%%%%%%%%%%%%%%%%%%%%%%%%%1.16%%%%%%%%%%%%%%%%%%%%%%%%%%%%%%%%%%%%%%
   \item[1.1.16]  Let $x$ be an element of $G$. Prove that $x^2 = 1$ if and only
                  if $|x|$ is either 1 or 2.
                  
      \textbf{Proof.}
      
      $(\Leftarrow)$ Suppose that $x^2 = 1$. Now if $|x| > 2$, then by
      definition, $x^2 \neq 1$. The only remaining possibilities are $|x| = 1$
      or $|x| = 2$.
      
      $(\Rightarrow)$ Suppose that $|x| = 1$ or $|x| = 2$. It immediately
      follows that $x^2 = 1$. \qed
%%%%%%%%%%%%%%%%%%%%%%%%%%%%%%%%%%%%%%1.17%%%%%%%%%%%%%%%%%%%%%%%%%%%%%%%%%%%%%%
   \item[1.1.17]  Let $x$ be an element of $G$. Prove that if $|x| = n$ for some
                  positive integer $n$ then $x^{-1} = x^{n-1}$.
                  
      \textbf{Proof.} Suppose that $|x| = n \in \Z^+$. By Exercise 1.1.18, it
      follows that $x^{n-1}x^1 = x^{n-1+1} = x^n = 1$, so that
      $x^{-1} = x^{n-1}$. \qed      
%%%%%%%%%%%%%%%%%%%%%%%%%%%%%%%%%%%%%%1.18%%%%%%%%%%%%%%%%%%%%%%%%%%%%%%%%%%%%%%
   \item[1.1.18]  Let $x$ and $y$ be elements of $G$. Prove that $xy = yx$ if
                  and only if $y^{-1}xy =x$ if and only if $x^{-1}y^{-1}xy = 1$.
                  
      \textbf{Proof.} First assume that $xy = yx$. We then have that
      $yx = xy = 1xy = yy^{-1}xy$, so that $x = y^{-1}xy$ by left cancellation.
      Now assume that $y^{-1}xy = x$. Thus
      $x1 = x = y^{-1}xy = 1y^{-1}xy = xx^{-1}y^{-1}xy$, so that
      $1 = x^{-1}y^{-1}xy$ by left cancellation. Finally assume that
      $x^{-1}y^{-1}xy = 1$. Multiplying on the left by $yx$ will yield the
      equation $xy = yx$. \qed
%%%%%%%%%%%%%%%%%%%%%%%%%%%%%%%%%%%%%%1.19%%%%%%%%%%%%%%%%%%%%%%%%%%%%%%%%%%%%%%
   \item[1.1.19]  Let $x \in G$ and let $a, b \in \Z^+$.
                  \begin{enumerate}
                     \item Prove that $x^{a+b} = x^ax^b$.
                     \item Prove that $(x^a)^b = x^{ab}$.
                     \item Prove that $(x^a)^{-1} = x^{-a}$.
                     \item Establish part (a) for arbitrary integers $a$ and $b$
                           (positive, negative or zero).
                     \item Establish part (b) for arbitrary integers $a$ and $b$
                           (positive, negative or zero).
                  \end{enumerate}
               
      \textbf{Proof.}
      
      \begin{enumerate}
         \item We have that
               \begin{align*}
                  x^{a+b} &= \underbrace{xx\cdot x}_{a+b \text{ factors}} \\
                          &= \underbrace{xx\cdot x}_{a \text{ factors}}\mbox{ }
                             \underbrace{xx\cdot x}_{b \text{ factors}} \\
                          &= x^ax^b.
               \end{align*} \qed
         \item We have that
               \begin{align*}
                  (x^a)^b &= (\underbrace{xx\cdot x}_{a \text{ factors}})^b \\
                          &= \underbrace{xx\cdot x}_{ab \text{ factors}} \\
                          &= x^{ab}.
               \end{align*} \qed
         \item We have
               \begin{align*}
                  (x^a)^{-1}
                     &= (\underbrace{xx\cdot x}_{a \text{ factors}})^{-1} \\
                     &= \underbrace{x^{-1}x^{-1}\cdot x^{-1}}_{
                           a \text{ factors}} &[\text{Exercise 1.1.15}] \\
                     &= x^{-a}.
               \end{align*} \qed
         \item Now suppose that $a$ is an integer and $b$ is a positive integer.
               We shall induct on $b$ to show that
               \begin{equation}
                  x^{a+b} = x^ax^b. \label{1_1_19_1}
               \end{equation}
               By Lemma 1.1.1, \eqref{1_1_19_1} holds if $b$ equals 1. So assume
               that it also holds for some positive integer $k$. We now have
               that
               \begin{align*}
                  x^ax^{k+1} &= x^ax^kx^1 &[\text{Lemma 1.1.1}] \\
                             &= (x^ax^k)x^1 \\
                             &= x^{a+k}x^1 &[\text{Inductive hypothesis}] \\
                             &= x^{(a+k)+1} &[\text{Lemma 1.3.2}] \\
                             &= x^{a+(k+1)}, &[\text{Associativity of addition}]
               \end{align*}
               so that \eqref{1_1_19_1} holds for $k + 1$, and hence, by the 
               Principle of Mathematical Induction, it holds for each positive
               integer $n$. \\

               If $a$ is 0 or $b$ is 0, then Lemma 1.1.1 tells us that
               \eqref{1_1_19_1} holds, so the only remaining possibility is $a$ 
               and $b$ are negative.\footnote{If $a$ is positive and $b$ is
               negative, then interchange the roles of $a$ and $b$ in the 
               induction proof.} Now suppose that $a$ and $b$ are negative.
               Hence
               \begin{align*}
                  x^ax^b &= x^{-(-a)}x^{-(-b)} \\
                     &= (x^{-1})^{-a}(x^{-1})^{-b} &[\text{Definition}] \\
                     &= (x^{-1})^{(-a + (-b))} &[\text{Part (a)}] \\
                     &= x^{-(-a + (-b))} &[\text{Definition}] \\
                     &= x^{a+b}.
               \end{align*}

               Combining this result with part (a), we thus shown that
               \eqref{1_1_19_1} holds for all integers $a$ and $b$. \qed
         \item It is clear that part (b) holds if $a$ is 0 or $b$ is 0, so let
               us complete the proof for arbritrary integers $a$ and $b$.

               \textbf{Case 1:} \textit{$a$ is positive and $b$ is negative}. 
               Hence
               \begin{align*}
                  (x^a)^b &= (x^a)^{-(-b)} \\
                          &= [(x^a)^{-1}]^{-b} &[\text{Definition}] \\
                          &= (x^{-a})^{-b} &[\text{Part (c)}] \\
                          &= [(x^{-1})^a]^{-b} &[\text{Definition}] \\
                          &= (x^{-1})^{-ab} &[\text{Part (b)}] \\
                          &= x^{-(-ab)} &[\text{Definition}] \\
                          &= x^{ab}.
               \end{align*}

               \textbf{Case 2:} \textit{$a$ and $b$ are negative}. Thus
               \begin{align*}
                  (x^a)^b &= [x^{-(-a)}]^b \\
                          &= [(x^{-1})^{-a}]^b &[\text{Definition}] \\
                          &= (x^{-1})^{-ab} &[\text{Case 1}] \\
                          &= [(x^{-1})^{-1}]^{ab}. &[\text{Definition}] \\
                          &= x^{ab}. &[\text{Proposition 1 (3)}]
               \end{align*}

               \textbf{Case 3:} \textit{$a$ is negative and $b$ is positive}. 
               Thus
               \begin{align*}
                  (x^a)^b &= [x^{-(-a)}]^b \\
                          &= [(x^{-1})^{-a}]^b &[\text{Definition}] \\
                          &= (x^{-1})^{-ab} &[\text{Case 1}] \\
                          &= x^{-(-ab)} &[\text{Definition}] \\
                          &= x^{ab}.
               \end{align*}

               Combining these results with part (a), we can conclude that
               $(x^a)^b = x^{ab}$ holds for all integers $a$ and $b$ and
               $x \in G$. \qed
      \end{enumerate}
%%%%%%%%%%%%%%%%%%%%%%%%%%%%%%%%%%%%%%1.20%%%%%%%%%%%%%%%%%%%%%%%%%%%%%%%%%%%%%%
   \item[1.1.20]  For $x$ an element in $G$ show that $x$ and $x^{-1}$ have the
                  same order.

      \textbf{Proof.}

      \textbf{Case 1:} \textit{$|x| = n \in \Z^+$}. Since
      $(x^{-1})^n = (x^n)^{-1} = 1^{-1} = 1$, it follows that $|x^{-1}| \le n$,
      so suppose to the contrary that $|x^{-1}| = m < n$. Then we have that
      $$x^m = [(x^{-1})^{-1}]^m = [(x^{-1})^m]^{-1} = 1^{-1} = 1,$$
      a contradiction, so that $|x^{-1}| = n = |x|$.

      \textbf{Case 2:} \textit{$|x| = +\infty$}. Suppose to the contrary that
      $|x^{-1}| = n \in \Z^+$. As we argued in Case 1, it must be the case that
      $x^n = 1$, a contradiction. Thus $|x| = +\infty = |x^{-1}|$. \qed
%%%%%%%%%%%%%%%%%%%%%%%%%%%%%%%%%%%%%%1.20%%%%%%%%%%%%%%%%%%%%%%%%%%%%%%%%%%%%%%
   \item[1.1.21]  Let $G$ be a finite group and let $x$ be an element of $G$ of
                  order $n$. Prove that if $n$ is odd, then $x = (x^2)^k$ for
                  some $k$.

      \textbf{Proof.} Suppose that $n$ is odd. We can then write $n = 2k + 1$
      for some nonnegative integer $k$. By supposition, we have that
      $xx^{2k} = x^{2k+1} = 1 = x^{-2k}x^{2k}$, so that by right cancellation,
      we can conclude that $x = x^{-2k} = (x^2)^{-k}$. \qed
%%%%%%%%%%%%%%%%%%%%%%%%%%%%%%%%%%%%%%1.22%%%%%%%%%%%%%%%%%%%%%%%%%%%%%%%%%%%%%%
   \item[1.1.22]  If $x$ and $g$ are elements of the group $G$, prove that
                  $|x| = |g^{-1}xg|$. Deduce that $|ab| = |ba|$ for all
                  $a, b \in G$.

      \textbf{Proof.} Let $x, g \in G$.

      \textbf{Case 1:} \textit{$|x| = n \in \Z^+$}. By Lemma 1.1.2, it follows
      that $(g^{-1}xg)^n = g^{-1}x^ng = g^{-1}g =1$, so that $|g^{-1}xg| \le n$,
      so suppose to the contrary that $|g^{-1}xg| = m < n$. Then we have that
      $$g^{-1}1g = 1 = (g^{-1}xg)^m = g^{-1}x^mg,$$
      so that $x^m = 1$ by left and right cancellations, a contradiction; thus,  
      $|g^{-1}xg| = n = |x|$.

      \textbf{Case 2:} \textit{$|x| = +\infty$}. Suppose to the contrary that
      $|g^{-1}xg| = n \in \Z^+$. As we argued in Case 1, it must then be the 
      case that $x^n = 1$, a contradiction. Thus $|x| = +\infty = |g^{-1}xg|$.

      Now consider $a, b \in G$. Set $x = ab$ and $g = a$. Since 
      $|x| = |g^{-1}xg|$, it follows that $|ab| = |a^{-1}aba| = |ba|$. \qed
%%%%%%%%%%%%%%%%%%%%%%%%%%%%%%%%%%%%%%1.23%%%%%%%%%%%%%%%%%%%%%%%%%%%%%%%%%%%%%%
   \item[1.1.23]  Suppose $x \in G$ and $|x| = n < \infty$. If $n = st$ for some
                  positive integers $s$ and $t$, prove that $|x^s| = t$.

      \textbf{Proof.} Suppose $n = st$ for some positive integers $s$ and $t$.
      By supposition, we have that $1 = x^n = x^{st} = (x^s)^t$; i.e.,
      $|x^s| \le t$. Suppose to the contrary that $|x^s| = m < t$. Then we have
      that $1 = (x^s)^m = x^{sm}$. Since $0 < m < t$, it follows that
      $0 < sm < st = n$. However $|x| = n$ and we just showed that $x^{sm} = 1$, 
      so that we have a contradiction. Hence we can conclude that $|x^s| = |t|$.
      \qed
%%%%%%%%%%%%%%%%%%%%%%%%%%%%%%%%%%%%%%1.24%%%%%%%%%%%%%%%%%%%%%%%%%%%%%%%%%%%%%%
   \item[1.1.24]  If $a$ and $b$ are \textit{commuting} elements of $G$, prove 
                  that $(ab)^n = a^nb^n$ for all $n \in \Z$. [Do this by 
                  induction for positive $n$ first.]

      \textbf{Proof.} Let $R(n)$ be the statement that $(ab)^n = a^nb^n$, for
      commuting elements $a$ and $b$.
               
      We now want to show using induction that $R(n)$ holds for every positive 
      integer $n$. It is clear that $R(1)$ is true. So suppose that $R(k)$ is 
      true for some positive integer $k$. We must now show that $R(k + 1)$ is 
      also true. Now we have that
      \begin{align*}
         (ab)^{k+1} &= (ab)^k(ab)^1 &[\text{Exercise 1.1.19}] \\
                    &= a^kb^k(ab)^1 &[\text{Since }R(k) \text{ is true}] \\
                    &= a^kb^k(ba)^1 &[ab = ba] \\
                    &= a^kb^kba \\
                    &= a^kb^{k+1}a \\
                    &= a^kab^{k+1} &[\text{$a$ commutes with $b$}] \\
                    &= a^{k+1}b^{k+1}, \\
      \end{align*}
      so that $R(k + 1)$ holds. It follows by the Principle of Mathematical 
      Induction that $R(n)$ holds for every positive integer $n$. By inpsection 
      we can see that $R(0)$ also holds. To complete the proof, we must now show 
      that $(ab)^{m} = a^mb^m$, where $m$ is a negative integer. First we notice 
      that
      \begin{equation}
         a^{-1}b^{-1} = (ba)^{-1} = (ab)^{-1} = b^{-1}a^{-1},
         \label{1_1_24_1}
      \end{equation}
      so that $a^{-1}$ and $b^{-1}$ are commuting elements. Thus it follows that
      \begin{align*}
         (ab)^m &= (ab)^{-(-m)} \\
                &= [(ab)^{-1}]^{-m} &[\text{Definition}] \\
                &= (a^{-1}b^{-1})^{-m} &[\eqref{1_1_24_1}] \\
                &= (a^{-1})^{-m}(b^{-1})^{-m} &[\text{$R(-m)$ holds}] \\
                &= a^mb^m,
      \end{align*}
      as desired. \qed
%%%%%%%%%%%%%%%%%%%%%%%%%%%%%%%%%%%%%%1.25%%%%%%%%%%%%%%%%%%%%%%%%%%%%%%%%%%%%%%
   \item[1.1.25]  Prove that if $x^2 = 1$ for all $x \in G$ then $G$ is abelian.

      \textbf{Proof.} Let $G$ be a group. Suppose that $x^2 = 1$ for all
      $x \in G$. We want to show that $G$ is abelian; that is, we want to show 
      that $xy = yx$ for all $x, y \in G$. So let $x, y \in G$. By hypothesis, 
      we have that $x^2 = e$, $y^2 = e$, and $(xy)^2 = e$, so that according to 
      Proposition 2, we must have that $x = x^{-1}$, $y = y^{-1}$, and
      $xy = (xy)^{-1}$. Thus
      \begin{align*}
         xy &= (xy)^{-1}      &[\text{By Hypothesis}] \\
            &= y^{-1}x^{-1}   &[\text{Proposition 1}] \\
            &= yx.
      \end{align*}
      Thus $G$ is abelian. \qed
%%%%%%%%%%%%%%%%%%%%%%%%%%%%%%%%%%%%%%1.26%%%%%%%%%%%%%%%%%%%%%%%%%%%%%%%%%%%%%%
   \item[1.1.26]  Assume $H$ is a nonempty subset of $(G, *)$ which is closed 
                  under the binary operation on $G$ and is closed under
                  inverses, i.e., for all $h$ and
                  $k \in H$, $hk$ and $h^{-1} \in H$. Prove that $H$ is a group 
                  under the operation $*$ restricted to $H$ (such a subset $H$
                  is called a subgroup of $G$).

      \textbf{Proof.} We know that $H$ is closed under $*$ and under inverses, 
      so it suffices to show that $*$ is associative on $H$ and that $H$ has an 
      identity under $*$. The associativity of $H$ under $*$ follows because $H$ 
      is a subset of $G$ and $G$ is associative under $*$. Since $H$ is nonempty
      we pick an $h \in H$. Then by hypothesis, we have that
      $1 = hh^{-1} \in H$, so that $H$ contains the identity. (Note that
      $hh^{-1} = h^{-1}h = 1$ and $h1 = 1h = h$ because these equalities hold in
      $G$.) \qed
%%%%%%%%%%%%%%%%%%%%%%%%%%%%%%%%%%%%%%1.27%%%%%%%%%%%%%%%%%%%%%%%%%%%%%%%%%%%%%%
   \item[1.1.27]  Prove that if $x$ is an element of the group $G$ then
                  $\{x^n : n \in \Z\}$ is a subgroup of $G$ (called the
                  \textit{cyclic subgroup} of $G$ generated by $x$).

      \textbf{Proof.} Consider the set
      $$H = \{x^n : n \in \Z\}.$$
      $H$ is nonempty because it contains $1 = x^0$. So let $h_1, h_2 \in H$.
      Thus we have $h_1 = x^a$ and $h_2 = x^b$ for some integers $a$ and $b$, so
      that $h_1h_2 = x^ax^b = x^{a+b} \in H$; in other words, $H$ is closed
      under the operation of $G$. Since $h_1^{-1} = (x^a)^{-1} = x^{-a} \in H$, 
      it follows that $H$ is also closed under inverses, so that $H$ is a
      subgroup of $G$ by Exercise 1.1.26.
%%%%%%%%%%%%%%%%%%%%%%%%%%%%%%%%%%%%%%1.28%%%%%%%%%%%%%%%%%%%%%%%%%%%%%%%%%%%%%%
   \item[1.1.28]  Let $(A, *)$ and $(B, \diamond)$ be groups and let
                  $A \times B$ be their direct product (as defined in Example
                  6). Verify all the group axioms for $A \times B$.
                  \begin{enumerate}
                     \item prove that the associative law holds: for all
                           $(a_i, b_i) \in A \times B, i = 1, 2, 3$
                           $$(a_1, b_1)[(a_2, b_2)(a_3, b_3)] =
                            [(a_1, b_1)(a_2, b_2)](a_3, b_3),$$
                     \item prove that (1, 1) is the identity of $A \times B$,
                           and
                     \item prove that the inverse of $(a, b)$ is
                           $(a^{-1}, b^{-1})$.
                  \end{enumerate}

      \textbf{Proof.} Let $(a_1, b_1)$, $(a_2, b_2)$, and
      $(a_3, b_3) \in A \times B$.

      \begin{enumerate}
         \item The set $A \times B$ is associative under the component wise
               operations of $A$ and $B$ because
               \begin{align*}
                  (a_1, b_1)[(a_2, b_2)(a_3, b_3)]
                     &= (a_1, b_1)(a_2a_3, b_2b_3) \\
                     &= (a_1a_2a_3, b_1b_2b_3) \\
                     &= [(a_1a_2)a_3, (b_1b_2)b_3] &[\text{Associativity}] \\
                     &= (a_1a_2, b_1b_2)(a_3, b_3) \\
                     &= [(a_1, b_1)(a_2, b_2)](a_3, b_3).
               \end{align*}
         \item Consider $(1, 1) \in A \times B$. It follows that
               \begin{align*}
                  (1, 1)(a_1, b_1) &= (1a_1, 1b_1) \\
                                   &= (a_1, b_1) \\
                                   &= (a_11, b_11) \\
                                   &= (a_1, b_1)(1, 1),
               \end{align*}
               so that $(1, 1)$ is the identity of $A \times B$.
         \item Consider $(a, b) \in A \times B$. It 
               follows that
               \begin{align*}
                  (a, b)(a^{-1}, b^{-1}) &= (aa^{-1}, bb^{-1}) \\
                                   &= (1, 1) \\
                                   &= (a^{-1}a, b^{-1}b) \\
                                   &= (a^{-1}, b^{-1})(a, b),
               \end{align*}
               so that $(a^{-1}, b^{-1})$ is the inverse of $(a, b)$.
      \end{enumerate}
%%%%%%%%%%%%%%%%%%%%%%%%%%%%%%%%%%%%%%1.29%%%%%%%%%%%%%%%%%%%%%%%%%%%%%%%%%%%%%%
   \item[1.1.29]  Prove that $A \times B$ is an abelian group if and only if
                  both $A$ and $B$ are abelian.

      \textbf{Proof.} 

      $(\Leftarrow)$ Suppose that $A$ and $B$ are abelian. Let $(a_1, b_1)$ and
      $(a_2, b_2) \in A \times B$. It follows that $A \times B$ is abelian
      because
      \begin{align*}
         (a_1, b_1)(a_2, b_2) &= (a_1a_2, b_1b_2) \\
            &= (a_2a_1, b_2b_1) &[\text{$A$ and $B$ are abelian}] \\
            &= (a_2, b_2)(a_1, b_1).
      \end{align*}

      $(\Rightarrow)$ Now suppose that $A \times B$ is abelian. Let $a_1$ and
      $a_2$ be members of $A$ and let $b_1$ and $b_2$ be members of $B$. Then
      we have that
      \begin{align*}
         (a_1a_2, b_1b_2) = (a_1, b_1)(a_2, b_2) \\
            &= (a_2, b_2)(a_1, b_1) &[\text{$A \times B$ is abelian}] \\
            &= (a_2a_1, b_2b_1),
      \end{align*}
      so that $(a_1a_2, b_1b_2) = (a_2a_1, b_2b_1)$; i.e., $a_1a_2 = a_2a_1$ and
      $b_1b_2 = b_2b_1$. We can now conclude that $A$ and $B$ are both abelian.
      \qed
%%%%%%%%%%%%%%%%%%%%%%%%%%%%%%%%%%%%%%1.30%%%%%%%%%%%%%%%%%%%%%%%%%%%%%%%%%%%%%%
   \item[1.1.30]  Prove that the elements $(a, 1)$ and $(1, b)$ of $A \times B$
                  commute and deduce that the order of $(a, b)$ is the least 
                  common multiple of $|a|$ and $|b|$.

      \textbf{Proof.} Let $A$ and $B$ be groups, and let $a \in A$, $b \in B$.
      We shall be assuming that there exist positive integers $m$ and $n$ such 
      that $|a| = m$ and $|b| = n$, for the problem does not make sense if the
      order of $a$ or $b$ is not finite. Consider $(a, 1)$,
      $(1, b) \in A \times B$. We have that
      \begin{align*}
         (a, 1)(1, b) &= (a1, 1b) \\
                      &= (a, b) \\
                      &= (1a, b1) \\
                      &= (1, b)(a, 1),
      \end{align*}
      so that $(a, 1)$ and $(b, 1)$ commute. To complete the proof, we let
      $s = \text{lcm}(m, n)$. Thus we can write $s = mx = ny$ for positive 
      integers $x$ and $y$. Thus we have that
      \begin{align*}
         (a, b)^s &= (a^s, b^s) &[\text{Lemma 1.1.4}] \\
                  &= (a^{mx}, a^{ny}) \\
                  &= [(a^m)^x, (a^n)^y] \\
                  &= (1^x, 1^y) \\
                  &= (1, 1).
      \end{align*}
      This say that $|(a, b)| \le s$, so there exists a positive integer $q$ 
      such that $|(a, b)| = q$. By Lemma 1.1.4, we have that
      $(a, b)^q = (a^q, b^q) = (1, 1)$, so that $a^q = 1$ and $b^q = 1$. Thus by 
      Lemma 1.1.3, it follows that $m \mid q$ and $n \mid q$, so that $s \mid q$ 
      by definition of the lcm. Since $s \mid q$, we must have that $s \le q$.
      But we previously showed that $q \le s$. Thus we can conclude that
      $s = q$, as desired. \qed
%%%%%%%%%%%%%%%%%%%%%%%%%%%%%%%%%%%%%%1.31%%%%%%%%%%%%%%%%%%%%%%%%%%%%%%%%%%%%%%
   \item[1.1.31]  Prove that any finite group $G$ of even order contains an
                  element of order 2. [Let $t(G)$ be the set
                  $\{g \in G : g \neq g^{-1}\}$. Show that $t(G)$ has an even 
                  number of elements and every nonidentity element of $G - t(G)$ 
                  has order 2.]

      \textbf{Proof.} Let $G$ be a finite group of even order. We wish to show
      that there exists some $g \in G$ such that $|g| = 2$. Consider this subset
      of $G$:
      $$S = \{g \in G: g \neq g^{-1}\}.$$

      If $|S| = 0$, then the proof is done, so assume that $|S| > 0$. Now $|S|$ 
      is even, for if this were not the case, then if we pair up every element
      of $S$ with its inverse, then one element must be without an inverse, a 
      contradiction. Now let $S' = G\backslash S$. It follows that
      $|G| = |S| + |S'|$. Notice that $S'$ is not empty because $e \in S'$. 
      Since $G$ and $S$ are both even, it follows that $|S'|$ must also be even. 
      Since we already showed that $|S'| \ge 1$, we can conclude that
      $|S'| \ge 2$, so that $S'$ contains a non-identity $a$, where
      $a = a^{-1}$. That is, $|a| = 2$. \qed
%%%%%%%%%%%%%%%%%%%%%%%%%%%%%%%%%%%%%%1.32%%%%%%%%%%%%%%%%%%%%%%%%%%%%%%%%%%%%%%
   \item[1.1.32]  If $x$ is an element of finite order $n$ in $G$, prove that
                  the elements 1, $x$, $x^2$, $\ldots$, $x^{n-1}$ are all 
                  distinct. Deduce that $|x| \le |G|$.

      \textbf{Proof.} Suppose that $|x| = n \in \Z^+$ for some $x \in G$. 
      Suppose to the contrary that the elements $x^0$, $x^1$, $x^2$, $\ldots$, 
      $x^{n-1}$ are not distinct. Then we must have that $x^i = x^j$ for some
      integer $i$ and $j$ where $0 \le i < j \le n - 1$. That is, $x^{j-i} = 1$,
      a contradiction because $j - i$ is a positive integer less thatn $n$. It
      follows that the elements $x^0$, $x$, $x^2$, $\ldots$, $x^{n-1}$ are all 
      distinct. Since there are clearly $n$ of these elements and since they are
      all members of $G$, we can conclude that $|x| = n \le |G|$. \qed
%%%%%%%%%%%%%%%%%%%%%%%%%%%%%%%%%%%%%%1.33%%%%%%%%%%%%%%%%%%%%%%%%%%%%%%%%%%%%%%
   \item[1.1.33]  Let $x$ be an element of finite order $n$ in $G$.
                  \begin{enumerate}
                     \item Prove that if $n$ is odd then $x^i \neq x^{-i}$ for
                           all $i = 1, 2, \ldots, n - 1$,
                     \item Prove that if $n = 2k$ and $1 \le i < n$ then
                           $x^i = x^{-i}$ if and only if $i = k$.
                  \end{enumerate}

      \textbf{Proof.}

      \begin{enumerate}
         \item Suppose that $n$ is odd. Now we shall suppose to the contrary
               that $x^i = x^{-i}$ for some integer $1 \le i \le n - 1$. Since
               $x^i = x^{-i}$, it follows that $x^{2i} = 1$. By Lemma 1.1.3, we
               must have that $n \mid 2i$, a contradiction because an odd
               number cannot divide a positive even number, so we conclude that
               $x^i \neq x^{-i}$ for all $i = 1, 2, \ldots, n - 1$. \qed
         \item Suppose that $n$ is even and $1 \le i < n$. Write $n = 2k$ for
               some positive integer $k$.

               $(\Leftarrow)$ Suppose that $i = k$. Then we have that
               $1 = x^{2k} = x^{2i} = x^ix^i$, so that $x^i = x^{-i}$.

               $(\Rightarrow)$ Conversely suppose that $x^i = x^{-i}$, so that
               $x^{2i} =1$. Thus, by Lemma 1.1.3, $2k \mid 2i$, or equivalently,
               $k \mid i$, so that $i = mk$ for some positive integer $m$. 
               Recall that $i < n = 2k$ by hypothesis, so that $mk < 2k$. That 
               is $m < 2$. But $m$ is a positive integer and so the only
               possibility is therefore $m = 1$, so that $i = k$. \qed
      \end{enumerate}
%%%%%%%%%%%%%%%%%%%%%%%%%%%%%%%%%%%%%%1.34%%%%%%%%%%%%%%%%%%%%%%%%%%%%%%%%%%%%%%
   \item[1.1.34]  If $x$ is an element of infinite order in $G$, prove that the
                  elements $x^n$, $n \in \Z$ are all distinct.

      \textbf{Proof.} Assume that $x$ is an element of infinite order in $G$.
      Now suppose to the contrary that $x^i = x^j$ for some unequal integers
      $i$ and $j$. We can further assume without loss of generality that
      $i < j$. Thus $x^{j-i} = 1$, a contradiction because this says that
      $|x| \le j - i$. It follows that distinct integral powers of $x$ yield 
      distinct elements of $G$. \qed
%%%%%%%%%%%%%%%%%%%%%%%%%%%%%%%%%%%%%%1.35%%%%%%%%%%%%%%%%%%%%%%%%%%%%%%%%%%%%%%
   \item[1.1.35]  If $x$ is an element of finite order $n$ in $G$, use the 
                  Division Algorithm to show that any integral power of $x$ 
                  equals one of the elements in the set
                  $\{1, x, x^2, \ldots, x^{n-1}\}$ (so these are all the
                  distinct elements of the cyclic subgroup of $G$ generated by
                  $x$).

      \textbf{Proof.} Assume that $x$ is an element of finite order $n$ in $G$.
      Let $z \in \Z$. By the Division Algorithm, there exist unique integers
      $q$ and $r$ such that $z = qn + r$ and $0 \le r < n$. That is
      $$x^z = x^{qn+r} = x^{qn}x^r = (x^n)^qx^r = 1^qx^r = x^r.$$
      Since $r \in \{0, 1, \ldots, n - 1\}$ and since $x^z = x^r$, it follows
      that $x^z \in \{x^0, x^1, \ldots, x^{n-1}\}$. \qed
%%%%%%%%%%%%%%%%%%%%%%%%%%%%%%%%%%%%%%1.36%%%%%%%%%%%%%%%%%%%%%%%%%%%%%%%%%%%%%%
   \item[1.1.36]  Assume $G = \{1, a, b, c\}$ is a group of order 4 with
                  identity 1. Assume also that $G$ has no elements of order 4
                  (so by Exercise 32, every element has order $\le$ 3). Use the
                  cancellation laws to show that there is a unique group table
                  for $G$. Deduce that $G$ is abelian.

      \textbf{Proof.} Assume $G = \{1, a, b, c\}$. We can tentatively fill out 
      the group table for $G$ like so:      
      $$
         \begin{tabular}{@{}c | c | c | c | c@{}} 
                & $1$ & $a$ & $b$ & $c$ \\ \hline
            $1$ & $1$ & $a$ & $b$ & $c$ \\ \hline
            $a$ & $a$ & $ $ & $ $ & $ $ \\ \hline
            $b$ & $b$ & $ $ & $ $ & $ $ \\ \hline
            $c$ & $c$ & $ $ & $ $ & $ $
         \end{tabular}
      $$
      By the left cancellation law, the equality $ab = a$ will result in $b = e$
      and the equality $ab = b$ will result in $a = e$, both of which are
      contradictions. The only remaining possiblities are $ab = c$ or $ab = 1$.

      \textbf{Case 1:} $ab = c$. For the same reason as above, we cannot have
      $ac = a$ or $ac = c$, so that $ac = 1$ or $ac = b$. So suppose first that
      $ac = b$. Then our table will now look like so:
      $$
         \begin{tabular}{@{}c | c | c | c | c@{}} 
                & $1$ & $a$ & $b$ & $c$ \\ \hline
            $1$ & $1$ & $a$ & $b$ & $c$ \\ \hline
            $a$ & $a$ & $ $ & $c$ & $b$ \\ \hline
            $b$ & $b$ & $ $ & $ $ & $ $ \\ \hline
            $c$ & $c$ & $ $ & $ $ & $ $
         \end{tabular}
      $$
      From the table above, we see that $aa$ must be equal to 1, since that is
      the only remaining possibility. The cancellation laws tell us that every
      element in a column and row of a group table must be unique, so we must
      have that:
      $$
         \begin{tabular}{@{}c | c | c | c | c@{}} 
                & $1$ & $a$ & $b$ & $c$ \\ \hline
            $1$ & $1$ & $a$ & $b$ & $c$ \\ \hline
            $a$ & $a$ & $1$ & $c$ & $b$ \\ \hline
            $b$ & $b$ & $c$ & $ $ & $ $ \\ \hline
            $c$ & $c$ & $b$ & $ $ & $ $
         \end{tabular}
      $$
      Note that we cannot have $bb = a$ because that would imply that $bbb = c$,
      so that $|b| > 3$, contradicting our hypothesis. Thus we must have that
      $bb = 1$. The remaining positions are thus completely determined, so that
      we have
      $$
         \begin{tabular}{@{}c | c | c | c | c@{}} 
                & $1$ & $a$ & $b$ & $c$ \\ \hline
            $1$ & $1$ & $a$ & $b$ & $c$ \\ \hline
            $a$ & $a$ & $1$ & $c$ & $b$ \\ \hline
            $b$ & $b$ & $c$ & $1$ & $a$ \\ \hline
            $c$ & $c$ & $b$ & $a$ & $1$
         \end{tabular}
      $$
      Now suppose that $ac = 1$, then we would be forced to fill in the table
      like so:
      $$
         \begin{tabular}{@{}c | c | c | c | c@{}} 
                & $1$ & $a$ & $b$ & $c$ \\ \hline
            $1$ & $1$ & $a$ & $b$ & $c$ \\ \hline
            $a$ & $a$ & $b$ & $c$ & $1$ \\ \hline
            $b$ & $b$ & $c$ & $ $ & $ $ \\ \hline
            $c$ & $c$ & $1$ & $ $ & $ $
         \end{tabular}
      $$
      Since $a^2 = b$ and $a^3 = c$, we have that $|a| > 3$, contradicting our
      hypothesis, so this is a dead end.

      \textbf{Case 2:} $ab = 1$. For the same reason as above, we cannot have
      $ac = a$ or $ac = c$, so that $ac = 1$ or $ac = b$. So suppose first that
      $ac = b$. Then our table will now look like so:
      $$
         \begin{tabular}{@{}c | c | c | c | c@{}} 
                & $1$ & $a$ & $b$ & $c$ \\ \hline
            $1$ & $1$ & $a$ & $b$ & $c$ \\ \hline
            $a$ & $a$ & $c$ & $1$ & $b$ \\ \hline
            $b$ & $b$ & $1$ & $c$ & $a$ \\ \hline
            $c$ & $c$ & $b$ & $a$ & $1$
         \end{tabular}
      $$
      Since $a^2 = c$ and $a^3 = b$, we have that $|a| > 3$, contradicting our
      hypothesis, so this is another dead end. From our arguments above, we see
      that the only viable and legal table is thus:
      $$
         \begin{tabular}{@{}c | c | c | c | c@{}} 
                & $1$ & $a$ & $b$ & $c$ \\ \hline
            $1$ & $1$ & $a$ & $b$ & $c$ \\ \hline
            $a$ & $a$ & $1$ & $c$ & $b$ \\ \hline
            $b$ & $b$ & $c$ & $1$ & $a$ \\ \hline
            $c$ & $c$ & $b$ & $a$ & $1$
         \end{tabular}
      $$
      This table is unique, and since it is symmeteric it follows that $G$ is
      abelian. \qed
\end{enumerate}

      \section{Dihedral Groups}
         \begin{enumerate}
%%%%%%%%%%%%%%%%%%%%%%%%%%%%%%%%%%%Prob1.2_1%%%%%%%%%%%%%%%%%%%%%%%%%%%%%%%%%%%%
   \item[1.2.1]   For each of the following statements, determine whether it is 
                  true or false and justify your answer.
                  \begin{enumerate}
                     \item The set $\Z$ of integers is dense in $\R$.
                     \item The set of positive real numbers is dense in $\R$.
                     \item The set $\Q\backslash \Z$ of rational numbers that 
                           are not integers is dense in $\R$.
                  \end{enumerate}  

      \textbf{Solution:} 

      \begin{enumerate}
         \item False. Proposition 1.6 states that there is no integer in the
               interval (0, 1).
         \item False. The interval $(-1, 0)$ contains no positive real number.
         \item True. Let $a$ and $b$ be real numbers. Then we shall investigate
               the following two cases:
               
               \textbf{Case I:} $a < a + 1 \le b$. Theorem 1.8 says that there
               exists a unique integer $k$ in $[a, a + 1)$. Thus there is no
               integer in the interval $(k, a + 1)$. By the density of $\Q$ in
               $\R$, there exists a rational $q \in (k, a + 1)$. Since
               $(k, a + 1)$ contains no integer, then $q$ must be a member of
               $\Q\backslash\Z$. We observe that $q \in (a, b)$.
               
               \textbf{Case II:} $a < b < a + 1$. Theorem 1.8 says that there
               exists a unique integer $k$ in $[a, a + 1)$. If $k  \le b$, then
               $(a, k)$ has no integer, so there exists a noninteger rational
               in $(a, k) \subseteq (a, b)$ by the density of $\Q$ in $\R$. If,
               however, $k > b$, then the interval $(a, b)$ contains no integer,
               so that there exists a noninteger rational in $(a, b)$ by the
               density of $\Q$ in $\R$.
      \end{enumerate}
%%%%%%%%%%%%%%%%%%%%%%%%%%%%%%%%%%Prob1.2_2%%%%%%%%%%%%%%%%%%%%%%%%%%%%%%%%%%%%%
   \item[1.2.2]   Suppose that $S$ is a nonempty set of integers that is bounded
                  below. Show that $S$ has a minimum. In particular, conclude
                  that every nonempty set of natural numbers has a minimum.  

      \textbf{Proof:}

      Let $S$ be a nonempty set of integers bounded below. Then there exists
      some $r \in \R$ such that for every $a \in S$, we have that $r \le a$.
      Consider the set $S' = \{-s: s \in S\}$, the set of the additive inverses
      of the elements of $S$. Note that $S'$ is also a nonempty set of integers.
      So let $-d \in S'$ where $d \in S$. Hence $r \le d$, so that $-d \le -r$;
      that is $S'$ is bounded above. By Proposition 1.7 $S'$ has a maximum, say
      $-b$, where $b \in S$. It suffices to show that $b$ is the minimum in $S$.
      Let $c \in S$. Then we have that $-c \le -b$, so that $b \le c$; that is,
      $b$ is the minimum element in $S$. In paritcular, we can see that the
      Well Ordering Principle follows. \qed
%%%%%%%%%%%%%%%%%%%%%%%%%%%%%%%%%%Prob1.2_3%%%%%%%%%%%%%%%%%%%%%%%%%%%%%%%%%%%%%
   \item[1.2.3]   Let $S$ be a nonempty set of real numbers that is bounded
                  below. Prove that the set $S$ has a minimum if and only if the
                  number $\inf S$ belongs to $S$.
			
		\textbf{Proof:} Let $S$ be a nonempty set of real numbers that is bounded
      below.

      $(\Leftarrow)$ Suppose $\inf S$ belongs in $S$; then it immediately
      follows by definition that $\inf S$ is the minimum element of $S$. \\
      $(\Rightarrow)$ Now suppose that $S$ has a minimum, say $s$. By the
      Completeness Axiom, we have that $\inf S$ exists; since $s \in S$, we must
      have that $\inf S \le s$. But $s$ is also a lower bound for $S$ and since
      every lower bound of $S$ cannot exceed $\inf S$, we must have that
      $s \le \sup S$; we have shown that $\inf S \le s$ and $s \le \inf S$ so
      that $s = \inf S$. \qed
%%%%%%%%%%%%%%%%%%%%%%%%%%%%%%%%%%Prob1.2_4%%%%%%%%%%%%%%%%%%%%%%%%%%%%%%%%%%%%%
   \item[1.2.4]   For each of the following two sets, find the maximum, minimum,
                  infimum, and supremum if they are defined. Justify your
                  conclusions.
                  \begin{enumerate}
                     \item $S = \{1/n : n \in \N\}$.
                     \item $T = \{x \in \R : x^2 < 2\}$.
                  \end{enumerate}

      \textbf{Solution:}

      \begin{enumerate}
         \item The \textbf{maximum} is 1. To show this consider any natural
               number $n$; then we have $n \ge 1$. Multiply this inequality by
               the positive number $1/n$ to give us $1/n \le 1$. Since
               $1 = 1/1 \in S$, we are done. $S$ has no \textbf{minimum}. Assume
               by way of contradiction that $\min S$ exists. Then by definition
               of $S$, we know that $\min S$ must be positive. So by the
               Archimedean Property, there exists a natural number $n_1$(so that
               $1/n_1 \in S$) such that $1/n_1 < \min S$, a contradiction. So
               $\min S$ doesn't exist. Since the Archimedean Property enables us
               to find a member of $S$ that is less than any positive number, no
               positive number can be a lower bound for $S$. Thus $S$ can only 
               be bounded below by negative numbers and 0. It follows that the
               \textbf{infimum} of  $S$ is 0. Since $S$ has a maximum, this
               maximum. By Problem 1.1.15, we have that the \textbf{supremum} of
               $S = 1$. If we consider $-T$, the set of the additive inverses of
               the elements of $T$.
         \item It is trivial to show that%%%%%%%%%%%%%%%%%%%%%%%%%%%%%%%%%%%%%%%%%%%%%%%%%%%%Show true
               $T = \{x \in \R: -\sqrt{2} < x < \sqrt{2}\}$. We claim that the
               \textbf{infimum} and \textbf{supremum} of $T$ are $-\sqrt{2}$ and
               $\sqrt{2}$. Suppose by contradiction that this is false; then 
               there exist $a > -\sqrt{2}$ and $b < \sqrt{2}$ such that $a$ and 
               $b$ are the infimum and supremum of $T$. Then by the density of
               $\Q$ in $\R$, there exist rationals $p$ and $q$ such that
               $-\sqrt{2} < p < a$ and $b < q < \sqrt{2}$; that is, $p$ and $q$ 
               are members of $T$. But since $p$ is less than $a$ and $q > b$, 
               we have contradictions. Thus our claim holds. Since the infimum 
               and supremum are not members of $T$, it follows that $T$ has 
               neither a \textbf{maximum} nor a \textbf{minimum}.
      \end{enumerate}
%%%%%%%%%%%%%%%%%%%%%%%%%%%%%%%%%%Prob1.2_5%%%%%%%%%%%%%%%%%%%%%%%%%%%%%%%%%%%%%
   \item[1.2.5]   Suppose that the number $a$ has the property that for every
                  natural number $n$, $a \le 1/n$. Prove that $a \le 0$.

      \textbf{Proof:} Assume by way of contradiction that $a > 0$. By The
      Archimedean Property there exists a natural number $k$ such that
      $a > 1/k$, a contradiction. Thus $a \le 0$. \qed

%%%%%%%%%%%%%%%%%%%%%%%%%%%%%%%%%%Prob1.2_6%%%%%%%%%%%%%%%%%%%%%%%%%%%%%%%%%%%%%
   \item[1.2.6]   Given a real number $a$, define
                  $S \equiv \{x : x \in \Q, x < a\}$. Prove that $a = \sup S$.

      \textbf{Proof:} By the density of $\Q$ in $\R$, there exists a rational
      $q \in (a - 1, a)$, so that $q \in S$. Thus $S$ is nonempty. By 
      definition, $S$ is bounded above by $a$; since $S$ is also nonempty, the
      Completeness Axiom says that $\sup S$ exists. So we must have that
      $\sup S \le a$. Now suppose that $\sup S < a$, then the density of $\Q$ in
      $\R$ guarantees that we have a rational $q$ in $(\sup S, a)$, so that $q$
      is also a member of $S$, a contradiction since we cannot have a member of
      $S$ that is greater than $\sup S$. Thus $a = \sup S$. \qed

%%%%%%%%%%%%%%%%%%%%%%%%%%%%%%%%%%Prob1.2_7%%%%%%%%%%%%%%%%%%%%%%%%%%%%%%%%%%%%%
   \item[1.2.7]   Show that for any real number $c$, there is exactly one 
                  integer in the interval $(c, c+1]$.

      \textbf{Proof:} Let $c$ be a real number. According to Theorem 1.8, there
      exists a unique integer $k$ in the interval $[-(c + 1), -c)$. So we have
      $-(c + 1) \le k < -c$, so that $c < -k \le c + 1$. Hence we have an
      integer $-k$ in the interval $(c, c + 1]$. We can see that $-k$ is unique
      because if another integer $h$ exists in $(c, c + 1]$, then $-h$ would
      also be in $[-(c + 1), -c)$, and since $k$ is unique, we must have
      $-h = k$, so that $h = -k$. \qed
   
%%%%%%%%%%%%%%%%%%%%%%%%%%%%%%%%%%Prob1.2_8%%%%%%%%%%%%%%%%%%%%%%%%%%%%%%%%%%%%%
   \item[1.2.8]   Show that the Archimedean Property is a consequence of the
                  assertion that for any real number $c$, there is an integer in
                  the interval $[c, c + 1)$.

      \textbf{Proof:} Let $\epsilon$ be a positive real number. It suffices to
      show that there exists a natural number greater than $\epsilon$. By our
      assertion, there exists an integer $k$ in the interval
      $[\epsilon+ 1, \epsilon + 2)$. So we have $\epsilon < \epsilon + 1 \le k$,
      so that $k$ is a positive integer. \qed
%%%%%%%%%%%%%%%%%%%%%%%%%%%%%%%%%%Prob1.2_9%%%%%%%%%%%%%%%%%%%%%%%%%%%%%%%%%%%%%
   \item[1.2.9]   Show that the Archimedean Property is a consequence of the
                  assertion that every interval $(a, b)$ contains a rational
                  number.

      \textbf{Proof:} Let $\epsilon$ be a positive real number. It suffices to
      show that there exists a natural number greater than $\epsilon$. By our
      assertion, there exist positive integers $p$ and $q$ such that
      $p/q \in (\epsilon, \epsilon + 1)$. Since $p$ and $q$ are positive, we 
      have that $q \ge 1$ so that $pq \ge p$; that is $p/q \le p$. We have now
      shown that $\varepsilon < p/q \le p$. Particularly $p > \varepsilon$,
      which is what we wanted to prove. \qed

      
\end{enumerate}

      \section{Symmetric Groups}
         \begin{enumerate}
%%%%%%%%%%%%%%%%%%%%%%%%%%%%%%%%%%%%%2.3.1%%%%%%%%%%%%%%%%%%%%%%%%%%%%%%%%%%%%%%
   \item[2.3.1]   Find all subgroups of $Z_{45} = \cyc{x}$, giving a generator
                  for each. Describe the containments between these subgroups.
                  
      \textbf{Solution.} Since the positive divisors of 45 are: 1, 3, 5, 9, 15,
      and 45, it follows that the subgroups of $Z_{45}$ are
      $$\cyc{x}, \cyc{x^3}, \cyc{x^5}, \cyc{x^9}, \cyc{x^{15}}, \text{ and }
        \cyc{x^{45}}.$$
        
      We have the following containments:
      $$
         \begin{tabular}{>{$}c<{$}>{$}c<{$}>{$}c<{$}>{$}c<{$}>{$}c<{$}>{$}c<{$}>{$}c<{$}}
            \cyc{x^{45}} & \le & \cyc{x^{15}} & \le & \cyc{x^5} & \le & \cyc{x} \\
            \cyc{x^{15}} & \le &  \cyc{x^3} & \le & \cyc{x} \\
            \cyc{x^9} & \le &  \cyc{x^3} & \le & \cyc{x}
         \end{tabular}
      $$
%%%%%%%%%%%%%%%%%%%%%%%%%%%%%%%%%%%%%2.3.2%%%%%%%%%%%%%%%%%%%%%%%%%%%%%%%%%%%%%%
   \item[2.3.2]   If $x$ is an element of the finite group $G$ and $|x| = |G|$,
                  prove that $G = \cyc{x}$. Give an explicit example to show 
                  that this result need not be true if $G$ is an infinite group.
                  
      \textbf{Proof.} Let $G$ be a finite group, so that $|G| = n \in \Z^+$.
      Suppose that there exists $x \in G$ such that $|x| = n$. Clearly
      $\cyc{x} \subseteq G$. But $|\cyc{x}| = n$ since $|x| = n$; thus
      $G \subseteq \cyc{x}$ so that $G = \cyc{x}$. Now let $G = \Z$. We have
      that $|\cyc{2}| = |G|$ but $G \neq \cyc{2}$. \qed
%%%%%%%%%%%%%%%%%%%%%%%%%%%%%%%%%%%%%2.3.3%%%%%%%%%%%%%%%%%%%%%%%%%%%%%%%%%%%%%%
   \item[2.3.3]   Find all generators for $\Z/48\Z$.
   
      \textbf{Solution.} The generators for $\Z/48\Z$ are: $\cyc{\overline{1}}$,
      $\cyc{\overline{5}}$, $\cyc{\overline{7}}$, $\cyc{\overline{11}}$,
      $\cyc{\overline{13}}$, $\cyc{\overline{17}}$, $\cyc{\overline{19}}$,
      $\cyc{\overline{23}}$, $\cyc{\overline{25}}$, $\cyc{\overline{29}}$,
      $\cyc{\overline{31}}$, $\cyc{\overline{35}}$, $\cyc{\overline{37}}$,
      $\cyc{\overline{41}}$, $\cyc{\overline{43}}$, and $\cyc{\overline{47}}$.
%%%%%%%%%%%%%%%%%%%%%%%%%%%%%%%%%%%%%2.3.4%%%%%%%%%%%%%%%%%%%%%%%%%%%%%%%%%%%%%%
   \item[2.3.4]   Find all generators for $\Z/202\Z$.
   
      \textbf{Solution.} Let $S$ be the set of generators for $\Z/202\Z$. Then
      $|S| = 100$ since
      $$S = \{\cyc{x} : x \text{ is odd and positive}, x \neq 101, \text{ and } x < 202\}.$$
%%%%%%%%%%%%%%%%%%%%%%%%%%%%%%%%%%%%%2.3.5%%%%%%%%%%%%%%%%%%%%%%%%%%%%%%%%%%%%%%
   \item[2.3.5]   Find the number of generators for $\Z/49000\Z$.
   
      \textbf{Solution.} For a positive integer $n$ let $\varphi(n)$ be the
      number of positive integers---less than or equal to $n$---that are
      relatively prime to $n$. Then the number of generators for $\Z/49000\Z$ is
      $\varphi(49000) = \varphi(2^35^37^2) =
      \varphi(2^3)\varphi(5^3)\varphi(7^2) = 16800$. 
%%%%%%%%%%%%%%%%%%%%%%%%%%%%%%%%%%%%%2.3.6%%%%%%%%%%%%%%%%%%%%%%%%%%%%%%%%%%%%%%
   \item[2.3.6]   In $\Z/48\Z$ write out all elements of $\cyc{\overline{a}}$
                  for every $\overline{a}$. Find all inclusions between
                  subgroups in $\Z/48\Z$.
      
      \textbf{Solution.}
      $$
         \begin{tabular}{|c|c|} \hline
            \textbf{Generators} & \textbf{Subgroups in} $\Z/48\Z$ \\ \hline
            0 & $\{0\}$ \\ \hline
            24 & $\{0, 24\}$ \\ \hline
            16, 32 & $\{0, 16, 32\}$ \\ \hline
            12, 36 & $\{0, 12, 24, 36\}$ \\ \hline
            8, 40 & $\{0, 8, 16, 24, 32, 40\}$ \\ \hline
            6, 18, 30, 42 & $\{0, 6, 12, 18, 24, 30, 36, 42\}$ \\ \hline
            4,20,28,44 & $\{0,4,8,12,16, 20, 24, 28, 32, 36, 40, 44\}$ \\ \hline
            3, 9, 15, 21, 27, 33, 39, 45 & $\{0, 3, 6, 9, 12, 15, 18, 21, 24,
            27, 30, 33, 36, 39, 42, 45\}$ \\ \hline            
            2, 10, 14, 22, 26, 34, 38, 46 & $\{x : 0 \le x \le 46,
            x \text{ is even}\}$ \\ \hline
            \text{See Exercise } 2.3.3 & $\Z/48\Z$ \\ \hline
         \end{tabular}
      $$
%%%%%%%%%%%%%%%%%%%%%%%%%%%%%%%%%%%%%2.3.7%%%%%%%%%%%%%%%%%%%%%%%%%%%%%%%%%%%%%%
   \item[2.3.7]   Let $Z_{48} = \cyc{x}$ and use the isomorphism
                  $\Z/48\Z \cong Z_{48}$ given by $\overline{1} \mapsto x$ to
                  list all subgroups of $Z_{48}$ as computed in the preceding
                  exercise.
                  
      \textbf{Solution.}
      $$
         \begin{tabular}{|c|} \hline
            \textbf{Subgroups in} $Z_{48}$ \\ \hline
            $\{1\}$ \\ \hline
            $\{1, x^{24}\}$ \\ \hline
            $\{1, x^{16}, x^{32}\}$ \\ \hline
            $\{1, x^{12}, x^{24}, x^{36}\}$ \\ \hline
            $\{1, x^8, x^{16}, x^{24}, x^{32}, x^{40}\}$ \\ \hline
            $\{1, x^6, x^{12}, x^{18}, x^{24}, x^{30},x^{36},x^{42}\}$ \\ \hline
            $\{1,x^4,x^8,x^{12},x^{16}, x^{20}, x^{24}, x^{28}, x^{32}, x^{36},
               x^{40}, x^{44}\}$ \\ \hline
            $\{1, x^3, x^6, x^9, x^{12}, x^{15}, x^{18}, x^{21}, x^{24},
            x^{27}, x^{30}, x^{33}, x^{36}, x^{39}, x^{42}, x^{45}\}$ \\ \hline
            $\{x^y : 0 \le y \le 46, y \text{ is even}\}$ \\ \hline
            $Z_{48}$ \\ \hline
         \end{tabular}
      $$
%%%%%%%%%%%%%%%%%%%%%%%%%%%%%%%%%%%%%2.3.8%%%%%%%%%%%%%%%%%%%%%%%%%%%%%%%%%%%%%%
   \item[2.3.8]   Let $Z_{48} = \cyc{x}$. For which integers $a$ does the map
                  $\varphi_a$ defined by $\varphi_a : \overline{1} \mapsto x^a$
                  extend to an \textit{isomorphism} from $\Z/48\Z$ onto
                  $Z_{48}$.
                  
      \textbf{Solution.} Suppose that $(a, 48) = 1$. Then it follows that $x^a$
      generates $Z_{48}$. Thus $\varphi_a$ is an isomorphism by Theorem 4 (Page
      56). Now suppose that $a$ is not relatively prime to 48. Then $x^a$ does
      not generate $Z_{48}$, so that the image of $\varphi_a$ is not $Z_{48}$.
      Hence $\varphi_a$ is an isomorphism if and only if $(a, 48) = 1$.
%%%%%%%%%%%%%%%%%%%%%%%%%%%%%%%%%%%%%2.3.9%%%%%%%%%%%%%%%%%%%%%%%%%%%%%%%%%%%%%%
   \item[2.3.9]   Let $Z_{36} = \cyc{x}$. For which integers $a$ does the map
                  $\psi_a$ defined by $\psi_a : \overline{1} \mapsto x^a$ extend
                  to a \textit{well defined homomorphism} from $\Z/48\Z$ into
                  $Z_{36}$. Can $\psi_a$ ever be a surjective homomorphism?
                  
      \textbf{Solution.} First we shall find the restriction(s) on $a$ such that
      $\psi_a$ is well defined. Suppose $b = c$ for some $b, c \in \Z/48\Z$. It
      suffices to show that $\psi_a(b) = \psi_a(c)$. Since $b = c$, there exists
      an integer $k$ such that $b = c + 48k$. Thus $\psi_a(b) = \psi_a(c+48k)$,
      so that
      $\psi_a(b)=(x^a)^{c+48k}=x^{ac + 48ak}= x^{ac}x^{48ak}=\psi_a(c)x^{12ak}$.
      So we must require $x^{12ak} = 1$ for all $k \in \Z$. Now $x^{12ak} = 1$
      for all $k \in \Z$ if and only if $3 \mid a$ if and only if $\psi_a$ is
      well defined. It follows immediately that
      $\psi_a$ is an homomorphism since
      \begin{align*}
         \psi_a(p + q) &= (x^a)^{p+q} \\
            &= x^{ap+aq} \\
            &= x^{ap}x^{aq} \\
            &= (x^a)^p(x^a)^q \\
            &= \psi_a(p)\psi_a(q)
      \end{align*}      
      for all $p, q \in \Z/48\Z$.
      
      \textit{Can $\psi_a$ ever be a surjective homomorphism?} No!
      
      \textbf{Proof.} Suppose to the contrary that $\psi_a$ is surjective. Then
      there exists $y \in \Z/48\Z$ such that $\psi_a(y) = x$. That is
      $x^{ay} = x$, so that $x^{ay-1} = 1$; thus $ay - 1 = 36m$ for some integer
      $m$. Rearrange the equality $ay - 1 = 36m$ to get $1 = ay - 36m$. Recall
      that $3 \mid a$; since $3$ also divides 36, it follows that 3 must divide
      1, a contradiction. Thus $\psi_a$ can never be surjective. \qed
%%%%%%%%%%%%%%%%%%%%%%%%%%%%%%%%%%%%%2.3.10%%%%%%%%%%%%%%%%%%%%%%%%%%%%%%%%%%%%%
   \item[2.3.10]  What is the order of $\overline{30}$ in $\Z/54\Z$? Write out
                  all the elements and their orders in $\cyc{\overline{30}}$.
                  
      \textbf{Solution.} The order of $30$ in $\Z/54\Z$ is
      $$\frac{54}{(30, 54)} = 9.$$
      The elements of $\cyc{30}$ and their respective orders are:
      $$
         \begin{tabular}{|c|c|} \hline
            Element of $\cyc{30}$ & Order \\ \hline
            30 & 9 \\ \hline
             6 & 9 \\ \hline
            36 & 3 \\ \hline
            12 & 9 \\ \hline
            42 & 9 \\ \hline
            18 & 3 \\ \hline
            48 & 9 \\ \hline
            24 & 9 \\ \hline
             0 & 1 \\ \hline
         \end{tabular}
      $$
%%%%%%%%%%%%%%%%%%%%%%%%%%%%%%%%%%%%%2.3.11%%%%%%%%%%%%%%%%%%%%%%%%%%%%%%%%%%%%%
   \item[2.3.11]  Find all cyclic subgroups of $D_8$. Find a proper subgroup of
                  $D_8$ which is not cyclic.
                  
      \textbf{Solution.} In $D_8$, only $r$ and $r^4$ have order 4. Thus
      $\{1, r, r^2, r^3\}$ is the only cyclic subgroup of order 4. The trivial
      subgroup is the only cyclic subgroup of order 1. Finally there are 5
      cyclic subgroups of order 2 and they are of the form $\{1, x\}$ where
      $x \in \{r^2, s, sr, sr^2, sr^3\}$. The set $\{1, s, r^2, sr^2\}$ is a
      non-cyclic proper subgroup of $D_8$.
%%%%%%%%%%%%%%%%%%%%%%%%%%%%%%%%%%%%%2.3.12%%%%%%%%%%%%%%%%%%%%%%%%%%%%%%%%%%%%%
   \item[2.3.12]  Prove that the following groups are \textit{not} cyclic:
                  \begin{enumerate}
                     \item $Z_2 \times Z_2$
                     \item $Z_2 \times \Z$
                     \item $\Z \times \Z$.
                  \end{enumerate}
      
      \textbf{Proof.}
      \begin{enumerate}
         \item The order of $Z_2 \times Z_2$ is 4, but no element in this group
               has order 4; thus $Z_2 \times Z_2$ is not cyclic.
         \item Let $Z_2 = \cyc{x}$. Observe that $Z_2 \times \Z$ is not finite,
               so in order for it to be cyclic it must be isomorphic to $\Z$.
               But this is not the case since $Z_2 \times \Z$ has two elements
               of finite order(namely $(1, 0)$ and $(x, 0)$) while $\Z$ has
               exactly 1 element of finite order.
         \item Suppose to the contrary that $\Z \times \Z$ is cyclic. Then there
               exist nonzero integers $a$ and $b$ such that
               $$\Z \times \Z = \cyc{(a,b)} = \{(na, nb) : n \in \Z\}.$$
               Thus there exists an integer $m$ such that
               $(ma, mb) = (0, 1)$. That is, $ma = 0$ and $mb = 1$. Since
               $ma = 0$, we must have $m = 0$ or $a = 0$. If $m$ is 0, then
               $(ma, mb) = (0, 0) \neq (0, 1)$, a contradiction; thus we must
               have $a = 0$, contradicting our assumption that $a$ is nonzero.
               Thus $\Z \times \Z$ is not cyclic.
      \end{enumerate} \qed
%%%%%%%%%%%%%%%%%%%%%%%%%%%%%%%%%%%%%2.3.13%%%%%%%%%%%%%%%%%%%%%%%%%%%%%%%%%%%%%
   \item[2.3.13]  Prove that the following pairs of groups are \textit{not}
                  isomorphic:
                  \begin{enumerate}
                     \item $\Z \times Z_2$ and $\Z$
                     \item $\Q \times Z_2$ and $\Q$.
                  \end{enumerate}
      
      \textbf{Proof.}
      \begin{enumerate}
         \item By Exercise 1.6.11, we know that $\Z \times Z_2$ is isomorphic to
               $Z_2 \times \Z$. By Exercise 2.3.12, $Z_2 \times \Z$ is not
               cyclic; thus $\Z \times Z_2$ is not cyclic. That is,
               $\Z \times Z_2$ is not isomorphic to $\Z$.
         \item Let $Z_2 = \cyc{x}$. It immediately follows that
               $\Q \times Z_2$ and $\Q$ are not isomorphic since $\Q \times Z_2$
               has two elements of finite order(namely $(0, 1)$ and $(0, x)$)
               while $\Q$ has exactly 1 element of finite order.
      \end{enumerate} \qed
%%%%%%%%%%%%%%%%%%%%%%%%%%%%%%%%%%%%%2.3.14%%%%%%%%%%%%%%%%%%%%%%%%%%%%%%%%%%%%%
   \item[2.3.14]  Let $\sigma =$ (1 2 3 4 5 6 7 8 9 10 11 12). For each of the
                  following integers $a$ compute $\sigma^a$:
                  $$a = 13, 65, 626, 1195, -6, -81, -570,\text{ and } {-1211}.$$
                  
      \textbf{Solution.}
      
      \begin{alignat*}{4}
         &\sigma^{13}   &&= \sigma &&\text{ } \\
         &\sigma^{65}   &&= \sigma^5 &&=
            (1\;6\;11\;4\;9\;2\;7\;12\;5\;10\;3\;8) \\
         &\sigma^{626}  &&= \sigma^2 &&= (1\;3\;5\;7\;9\;11) \\
         &\sigma^{1195} &&= \sigma^7 &&=
            (1\;8\;3\;10\;5\;12\;7\;2\;9\;4\;11\;6\;13) \\
         &\sigma^{-6} &&= \sigma^6 &&= (1\;7)
            (1\;8\;3\;10\;5\;12\;7\;2\;9\;4\;11\;6\;13) \\
         &\sigma^{-81} &&= \sigma^3 &&= (1\;4\;7\;10) \\
         &\sigma^{-570} &&= \sigma^6 &&= (1\;7) \\
         &\sigma^{-1211} &&= \sigma
      \end{alignat*}
%%%%%%%%%%%%%%%%%%%%%%%%%%%%%%%%%%%%%2.3.15%%%%%%%%%%%%%%%%%%%%%%%%%%%%%%%%%%%%%
   \item[2.3.15]  Prove that $\Q \times \Q$ is not cyclic.
   
      \textbf{Proof.} Since $\Q$ is infinite and, by Exercise 1.6.6, $\Q$ is not
      isomorphic to $\Z$, it follows that $\Q$ is not cyclic. We know that the
      subgroup of every cyclic group is cyclic; since $\Q \times\{1\} \cong \Q$,
      it follows that $\Q \times \{1\}$ is not cyclic; thus $\Q \times \Q$ is
      not cyclic because it has a noncyclic subgroup, namely $\Q \times \{1\}$.
      \qed
%%%%%%%%%%%%%%%%%%%%%%%%%%%%%%%%%%%%%2.3.16%%%%%%%%%%%%%%%%%%%%%%%%%%%%%%%%%%%%%
   \item[2.3.16]  Assume $|x| = n$ and $|y| = m$. Suppose that $x$ and $y$
                  \textit{commute}: $xy = yx$. Prove that $|xy|$ divides the
                  least common multiple of $m$ and $n$. Need this be true if $x$
                  and $y$ do \textit{not} commute? Give an example of commuting
                  elements $x$, $y$ such that the order of $xy$ is not equal to
                  the least common multiple of $|x|$ and $|y|$.
                  
      \textbf{Proof.} Let $l = \text{lcm}(m, n)$. So there exist integers
      $m'$ and $n'$ such that $mm' = nn' = l$. So we have that
      $$(xy)^l = x^ly^l = x^{nn'}y^{mm'} = (x^n)^{n'}(y^m)^{m'} = 1.$$
      That is $|xy|$ divides $l$ (by Proposition 3, Page 55).
      
      \textit{Need this be true if $x$ and $y$ do not commute?} No! Let
      $$
         A = \left(\begin{tabular}{@{}cc@{}}
            0 & 1/2 \\
            2 & 0
         \end{tabular}\right) \text{ and }
         B = \left(\begin{tabular}{@{}cc@{}}
            0 & 1 \\
            1 & 0
         \end{tabular}\right).
      $$
      A simple computation will show us that although $|A| = |B| = 2$, we have
      that $|AB| = \infty$.
      
      \textbf{Example.} Consider $\Z/2\Z = \{0, 1\}$. Let $x = y = 1$. Then we
      have $|x| = |y| = 2$, so that lcm($|x|, |y|) = 2 \neq |x + y| = |0| = 1$.
      \qed
%%%%%%%%%%%%%%%%%%%%%%%%%%%%%%%%%%%%%2.3.17%%%%%%%%%%%%%%%%%%%%%%%%%%%%%%%%%%%%%
   \item[2.3.17]  Find a presentation for $Z_n$ with one generator.
   
      \textbf{Solution.} $Z_n = \cyc{x : x^n = 1}$.
%%%%%%%%%%%%%%%%%%%%%%%%%%%%%%%%%%%%%2.3.18%%%%%%%%%%%%%%%%%%%%%%%%%%%%%%%%%%%%%
   \item[2.3.18]  Show that if $H$ is any group and $h$ is an element of $H$
                  with $h^n = 1$, then there is a unique homomorphism from
                  $Z_n = \cyc{x}$ to $H$ such that $x \mapsto h$.
                  
      \textbf{Proof.} Let $n \in \Z^+$, $Z_n = \cyc{x}$, $H$ a group, and
      $h^n  = 1$ for some $h \in H$. First we shall show the existence of a
      homomorphism from $Z_n$ to $H$ such that $x \mapsto h$. So consider the
      map $\alpha : \cyc{x} \rightarrow H$ defined by $\alpha(x^a) = h^a$.
      Clearly $\alpha(x) = h$. Now we will show that $\alpha$ is well defined.
      Suppose $x^w = x^y$ for some $x^w, x^y \in Z_n$. Thus $w = y + nk$ for
      some integer $k$. Thus
      $$\alpha(x^w) = \alpha(x^{y+nk})=h^{y+nk}=h^{y}{h^n}^k =h^y=\alpha(x^y),$$
      so that $\alpha$ is well defined. Now we have that
      $$\alpha(x^px^q)=\alpha(x^{p+q})=h^{p+q}=h^ph^q=\alpha(x^p)\alpha(x^q),$$
      so that $\alpha$ is an homomorphism. Now to show uniqueness, we suppose
      that $\phi : \cyc{x} \rightarrow H$ is an homommorphism such that
      $\phi(x) = h$. Since $\phi$ is a homomorphism, it follows that
      $\phi(x^a) = h^a$. Thus $\phi = \alpha$, as desired. \qed
%%%%%%%%%%%%%%%%%%%%%%%%%%%%%%%%%%%%%2.3.19%%%%%%%%%%%%%%%%%%%%%%%%%%%%%%%%%%%%%
   \item[2.3.19]  Show that if $H$ is any group and $h$ is an element of $H$,
                  then there is a unique homomorphism from $\Z$ to $H$ such that
                  $1 \mapsto h$.
                  
      \textbf{Proof.} Let $H$ be a group and let $h \in H$. First we shall show
      that there exists a homomorphism from $\Z$ to $H$ such that $1 \mapsto h$.
      So consider the map $\alpha : \Z \rightarrow H$ defined by
      $n \mapsto h^n$. Clearly $\alpha(1) = h$ and
      $$\alpha(x+y) = h^{x+y} = h^xh^y = \alpha(x)\alpha(y) \text{ for all }
        x, y \in \Z^+,$$
      so that $\alpha$ is a homomorphism. To show uniqueness, suppose that
      $\alpha' : \Z \rightarrow H$ is an homomorphism such that
      $\alpha'(1) = h$. Then according to Exercise 1.6.1, we have that
      $\alpha'(n) = \alpha'(n\cdot1) = \alpha'(1)^n = h^n$ for all $n \in \Z$;
      that is, $\alpha' = \alpha$, as desired. \qed
%%%%%%%%%%%%%%%%%%%%%%%%%%%%%%%%%%%%%2.3.20%%%%%%%%%%%%%%%%%%%%%%%%%%%%%%%%%%%%%
   \item[2.3.20]  Let $p$ be a prime and let $n$ be a positive integer. Show
                  that if $x$ is an element of the group $G$ such that
                  $x^{p^n} = 1$ then $|x| = p^m$ for some $m \le n$.
                  
      \textbf{Proof.} Suppose that $x \in G$ such that $x^{p^n} = 1$. Then it
      follows by Proposition 3 (Page 55) that $|x|$ divides $p^n$. Since $p$ is
      a prime, its factors are $p^i$, $0 \le i \le n$. Thus $|x| = p^m$ for
      some nonnegative $m$ not greater than $n$. \qed
%%%%%%%%%%%%%%%%%%%%%%%%%%%%%%%%%%%%%2.3.21%%%%%%%%%%%%%%%%%%%%%%%%%%%%%%%%%%%%%
   \item[2.3.21]  Let $p$ be an odd prime and let $n$ be a positive integer
                  $\ge 2$. Use the Binomial Theorem to show that
                  $(1+p)^{p^{n-1}} \equiv 1$ (mod $p^n$) but
                  $(1+p)^{p^{n-2}} \not\equiv 1$ (mod $p^n$). Deduce that $1+p$
                  is an element of order $p^{n-1}$ in the multiplicative group
                  $(\Z/p^n\Z)^\times$.

      \textbf{Lemma 2.3.1.} \textit{For an integer $n \ge 2$ and an odd prime
      $p$, let $f_p(n)$ be the number of $p$ factors of $n!$ (i.e., the greatest
      nonnegative integer $j$ such that $p^j \mid i!$), then it follows that
      $f_p(n) < \D\frac{n}{2}$}.

      \textbf{Proof.} Let $n \ge 2$ be an integer and $p$ an odd prime. For a
      a positive integer $r$, let $g_p(n, r)$ be the number of positive
      integers, less than or equal to $n$, that have at least $r$ number of $p$ 
      factors. It follows that $g_p(n, r) = \D\gint{\frac{n}{p^r}}$, where
      $\gint{x}$ is the greatest integer less than or equal to $x$. Finally let
      $k_n$ be the maximum nonnegative integer such that $p^{k_n}$ is a multiple
      of some positive integer not greater than $n$. Thus we have that
      \begin{align*}
         f_p(n) &= g_p(n, 1) + g_p(n, 2) + \cdots + g_p(n, k_n) \\
            &= \sum_{i=1}^{k_n} g_p(n, i)
            = \sum_{i=1}^{k_n} \gint{\frac{n}{p^i}} \\
            &\le \sum_{i=1}^{k_n} \frac{n}{p^i}
            < \sum_{i=1}^\infty \frac{n}{p^i} \\
            &= \frac{n}{p-1} &[\text{Sum of Geometric Series}] \\
            &< \frac{n}{2}. &[\text{Since }p \ge 3]
      \end{align*}

      So we can write $n! = p^{f_p(n)} h_n$ for some $h_n \in \Z^+$, so that
      $(h_n, p) = 1$.

      Now we are ready to commence the proof of the problem. By the Binomial
      Theorem, we have that
      \begin{align*}
         (1+p)^{p^{n-1}} &= \sum_{i=0}^{p^{n-1}}\binom{p^{n-1}}{i}p^i \\
            &= \sum_{i=0}^{p^{n-1}}p^i\frac{p^{n-1}(p^{n-1}-1)(p^{n-1}-2)
               \cdots(p^{n-1}-i+1)}{i!} \\
            &= \sum_{i=0}^{p^{n-1}}p^i\frac{p^{n-1}(p^{n-1}-1)(p^{n-1}-2)
               \cdots(p^{n-1}-i+1)}{p^{f_p(i)} h_i} \\
            &= 1 + p^n + p^n\sum_{i=2}^{p^{n-1}}\frac{p^{i-1}(p^{n-1}-1)
               (p^{n-1}-2) \cdots(p^{n-1}-i+1)}{p^{f_p(i)} h_i}.
      \end{align*}
      Now $f_p(i) < i / 2 \le i - 1$ for $i \ge 2$. Thus $i - 1 - f_p(i) \ge 0$
      (so that $p^{i - 1 - f_p(i)}$ is an integer) if $i \ge 2$. We then have
      \begin{equation} \label{2_3_21_1}
         (1+p)^{p^{n-1}} = 1 + p^n + p^n\sum_{i=2}^{p^{n-1}}\frac{p^{i-1-f_p(i)}
        (p^{n-1}-1)(p^{n-1}-2) \cdots(p^{n-1}-i+1)}{h_i}
      \end{equation}
      Since $(h_i, p) = 1$, it follows that $h_i$ must divide
      $p^{i-1}(p^{n-1}-1)(p^{n-1}-2) \cdots(p^{n-1}-i+1)$. Hence
      $$\sum_{i=2}^{p^{n-1}}\frac{p^{i-1-f_p(i)}
        (p^{n-1}-1)(p^{n-1}-2) \cdots(p^{n-1}-i+1)}{h_i}$$
      is an integer and we can conclude from \eqref{2_3_21_1} that
      $(1+p)^{p^{n-1}} \equiv 1$ (mod $p^n$). Now we have that
      \begin{align*}
         (1+p)^{p^{n-2}} &= \sum_{i=0}^{p^{n-2}}\binom{p^{n-2}}{i}p^i \\
            &= \sum_{i=0}^{p^{n-2}}p^i\frac{p^{n-2}(p^{n-2}-1)(p^{n-2}-2)
               \cdots(p^{n-2}-i+1)}{i!} \\
            &= 1 + p^{n-1} + p^n\frac{p^{n-2}-1}{2} + p^n\frac{p(p^{n-2}-1)(p^{n-2}-2)}{3!} +\sum_{i=4}^{p^{n-1}}p^i\frac{p^{n-2}(p^{n-2}-1)(p^{n-2}-2)
               \cdots(p^{n-2}-i+1)}{p^{f_p(i)} h_i} \\
            &= 1 + p^n + p^n\sum_{i=2}^{p^{n-1}}\frac{p^{i-1}(p^{n-1}-1)
               (p^{n-1}-2) \cdots(p^{n-1}-i+1)}{p^{f_p(i)} h_i}.
      \end{align*}
      
%%%%%%%%%%%%%%%%%%%%%%%%%%%%%%%%%%%%%2.3.22%%%%%%%%%%%%%%%%%%%%%%%%%%%%%%%%%%%%%
   \item[2.3.22]  Let $n$ be an integer $\ge 3$. Use the Binomial Theorem to
                  show that $(1+2^2)^{2^{n-2}} \equiv 1$ (mod $2^n$) but
                  $(1+2^2)^{2^{n-3}} \not\equiv 1$ (mod $2^n$). Deduce that 5 is
                  an element of order $2^{n-2}$ in the multiplicative group
                  $(\Z/2^n\Z)^\times$.

      \textbf{Proof.}
%%%%%%%%%%%%%%%%%%%%%%%%%%%%%%%%%%%%%2.3.23%%%%%%%%%%%%%%%%%%%%%%%%%%%%%%%%%%%%%
   \item[2.3.23]  Show that $(\Z/2^n\Z)^\times$ is not cyclic for any $n \ge 3$.
                  [Find two distinct subgroups of order 2.]
%%%%%%%%%%%%%%%%%%%%%%%%%%%%%%%%%%%%%2.3.24%%%%%%%%%%%%%%%%%%%%%%%%%%%%%%%%%%%%%
   \item[2.3.24]  Let $G$ be a finite group and let $x \in G$.
                  \begin{enumerate}
                     \item Prove that if $g \in N_G(\cyc{x})$ then
                           $gxg^{-1} = x^a$ for some $a \in \Z$. 
                     \item Prove conversely that if $gxg^{-1} = x^a$ for some
                           $a \in \Z$ then $g \in N_G(\cyc{x})$. [Show first
                           that $gx^kg^{-1} = (gxg^{-1})^k = x^{ak}$ for any
                           integer $k$, so that $g\cyc{x}g^{-1} \le \cyc{x}$.
                           If $x$ has order $n$, show the elements $gx^ig^{-1}$,
                           $i = 0, 1, \ldots, n-1$ are distinct, so that
                           $|g\cyc{x}g^{-1}| = |\cyc{x}| = n$ and conclude that
                           $g\cyc{x}g^{-1} = \cyc{x}$.]
                  \end{enumerate}
                  Note that this cuts down some of the work in computing
                  normalizers of cyclic subgroups since one does not have to
                  check $ghg^{-1} \in \cyc{x}$ for every $h \in \cyc{x}$.
%%%%%%%%%%%%%%%%%%%%%%%%%%%%%%%%%%%%%2.3.25%%%%%%%%%%%%%%%%%%%%%%%%%%%%%%%%%%%%%
   \item[2.3.25]  Let $G$ be a cyclic group of order $n$ and let $k$ be an
                  integer relatively prime to $n$. Prove that the map
                  $x \mapsto x^k$ is surjective. Use Lagrange's Theorem
                  (Exercise 1.7.19) to prove the same is true for any finite
                  group of order $n$. (For such $k$ each element has a
                  $k^{\text{th}}$ root in $G$. It follows from Cauchy's Theorem
                  in Section 3.2 that if $k$ is not relatively prime to the
                  order of $G$ then the map $x \mapsto x^k$ is not surjective.)
%%%%%%%%%%%%%%%%%%%%%%%%%%%%%%%%%%%%%2.3.26%%%%%%%%%%%%%%%%%%%%%%%%%%%%%%%%%%%%%
   \item[2.3.26]  Let $Z_n$ be a cyclic group of order $n$ and for each integer
                  $a$ let
                  $$\sigma_a : Z_n \mapsto Z_n \qquad by \qquad \sigma_a(x) =
                  x^a \quad \text{for all } x \in Z_n.$$
                  \begin{enumerate}
                     \item Prove that $\sigma_a$ is an automorphism of $Z_n$ if
                           and only if $a$ and $n$ are relatively prime(
                           automorphisms were introduced in Exercise 1.6.20).
                     \item Prove that $\sigma_a = \sigma_b$ if and only if
                           $a \equiv b$ (mod $n$).
                     \item Prove that \textit{every} automorphism of $Z_n$ is
                           equal to $\sigma_a$ for some integer $a$.
                     \item Prove that $\sigma_a\circ\sigma_b=\sigma_{ab}$.
                           Deduce that the map $\overline{a} \mapsto \sigma_a$
                           is an isomorphism of $(\Z/n\Z)^\times$ onto the
                           automorphism group of $Z_n$ (so Aut($Z_n$) is an
                           abelian group of order $\varphi(n)$).
                  \end{enumerate}
                  %%%%%MISSING CONTAINMENT%%%%%%%%
\end{enumerate}


































      \section{Matrix Groups}
         Let $F$ be a field and let $n \in \Z^+$.
\begin{enumerate}
%%%%%%%%%%%%%%%%%%%%%%%%%%%%%%%%%%%%%1.4.1%%%%%%%%%%%%%%%%%%%%%%%%%%%%%%%%%%%%%%
   \item[1.4.1]   Prove that $|GL_2(\F_2)| = 6$.
%%%%%%%%%%%%%%%%%%%%%%%%%%%%%%%%%%%%%1.4.2%%%%%%%%%%%%%%%%%%%%%%%%%%%%%%%%%%%%%%
   \item[1.4.2]   Write out all the elements of $GL_2(\F_2)$ and compute the
                  order of each element.
%%%%%%%%%%%%%%%%%%%%%%%%%%%%%%%%%%%%%1.4.3%%%%%%%%%%%%%%%%%%%%%%%%%%%%%%%%%%%%%%
   \item[1.4.3]   Show that $GL_2(\F_2)$ is non-abelian.
%%%%%%%%%%%%%%%%%%%%%%%%%%%%%%%%%%%%%1.4.4%%%%%%%%%%%%%%%%%%%%%%%%%%%%%%%%%%%%%%
   \item[1.4.4]   Show that if $n$ is not prime then $\Z/n\Z$ is not a field.
%%%%%%%%%%%%%%%%%%%%%%%%%%%%%%%%%%%%%1.4.5%%%%%%%%%%%%%%%%%%%%%%%%%%%%%%%%%%%%%%
   \item[1.4.5]   Show that $GL_n(F)$ is a finite group if and only if $F$ has a
                  finite number of elements.
%%%%%%%%%%%%%%%%%%%%%%%%%%%%%%%%%%%%%1.4.6%%%%%%%%%%%%%%%%%%%%%%%%%%%%%%%%%%%%%%
   \item[1.4.6]   If $|F| = q$ is finite prove that $|GL_n(F)| < q^{n^2}$.
%%%%%%%%%%%%%%%%%%%%%%%%%%%%%%%%%%%%%1.4.7%%%%%%%%%%%%%%%%%%%%%%%%%%%%%%%%%%%%%%
   \item[1.4.7]   Let $p$ be a prime. Prove that the order of $GL_2(\F_p)$ is
                  $p^4 - p^3 - p^2 + p$.
%%%%%%%%%%%%%%%%%%%%%%%%%%%%%%%%%%%%%1.4.8%%%%%%%%%%%%%%%%%%%%%%%%%%%%%%%%%%%%%%
   \item[1.4.8]   Show that $GL_n(F)$ is non-abelian for any $n \ge 2$ and any
                  $F$.
%%%%%%%%%%%%%%%%%%%%%%%%%%%%%%%%%%%%%1.4.9%%%%%%%%%%%%%%%%%%%%%%%%%%%%%%%%%%%%%%
   \item[1.4.9]   Prove that the binary operation of matrix multiplication of
                  $2 \times 2$ matrices with real number entries is associative.
%%%%%%%%%%%%%%%%%%%%%%%%%%%%%%%%%%%%%1.4.10%%%%%%%%%%%%%%%%%%%%%%%%%%%%%%%%%%%%%
   \item[1.4.10]  Let $\left\{\left(\begin{tabular}{@{}cc@{}}
                     $a$ & $b$ \\
                      0  & $c$
                  \end{tabular}\right) : a, b, c \in \R, a \neq 0, c \neq 0
                  \right\}$.

                  \begin{enumerate}
                     \item Compute the product of
                           $\left(\begin{tabular}{@{}cc@{}}
                              $a_1$ & $b_1$ \\
                              0  & $c_1$
                           \end{tabular}\right)$ and
                           $\left(\begin{tabular}{@{}cc@{}}
                              $a_2$ & $b_2$ \\
                              0  & $c_2$
                           \end{tabular}\right)$ to show that $G$ is closed under
                           matrix multiplication.
                     \item Find the matrix inverse of
                           $\left(\begin{tabular}{@{}cc@{}}
                              $a$ & $b$ \\
                              0  & $c$
                           \end{tabular}\right)$ and deduce that $G$ is closed 
                           under inverses.
                     \item Deduce that $G$ is a subgroup of $GL_2(\R)$.
                     \item Prove that the set of elements of $G$ whose two
                           diagonal entries are equal is also a subgroup of
                           $GL_2(\R)$.
                  \end{enumerate}
\end{enumerate}

The next exercise introduces the \textit{Heisenberg group} over the field $F$
and develops some of its basic properties. When $F = \R$ this groups plays an
important role in quantum mechanics and signal theory by giving a group
theoretic interpretation (due to H. Weyl) of Heisenberg's Uncertainty Principle.
Note also that the Heisenberg group may be defined more generally---for example,
with entries in $\Z$.

\begin{enumerate}
%%%%%%%%%%%%%%%%%%%%%%%%%%%%%%%%%%%%%1.4.11%%%%%%%%%%%%%%%%%%%%%%%%%%%%%%%%%%%%%
   \item[1.4.11]  Let $H(F) = \left\{\left(\begin{tabular}{@{}ccc@{}}
                     1 & $a$ & $b$ \\
                     0 & 1 & $c$ \\
                     0 & 0 & 1
                  \end{tabular}\right) : a, b, c \in F\right\}$---called the
                  \textit{Heisenberg group} over $F$. Let
                  $X = \left(\begin{tabular}{@{}ccc@{}}
                     1 & $a$ & $b$ \\
                     0 & 1 & $c$ \\
                     0 & 0 & 1
                  \end{tabular}\right)$ and $Y =\left(\begin{tabular}{@{}ccc@{}}
                     1 & $d$ & $e$ \\
                     0 & 1 & $f$ \\
                     0 & 0 & 1
                  \end{tabular}\right)$ be elements of $H(F)$.

                  \begin{enumerate}
                     \item Compute the matrix product $XY$ and deduce that
                           $H(F)$ is closed under matrix multiplication. Exhibit
                           explicit matrices such that $XY \neq YX$ (so that
                           $H(F)$ is always non-abelian).
                     \item Find an explicit formula for the matrix inverse
                           $X^{-1}$ and deduce that $H(F)$ is closed under
                           inverses.
                     \item Prove the associative law for $H(F)$ and deduce that
                           $H(F)$ is a group of order $|F|^3$. 

                           (Do not assume that matrix multiplication is 
                           associative).
                     \item Find the order of each element of the finite group
                           $H(\Z/2\Z)$.
                     \item Prove that every nonidentity element of the group
                           $H(\R)$ has infinite order.
                  \end{enumerate}
\end{enumerate}

      \section{The Quaternion Group}
         \begin{enumerate}
%%%%%%%%%%%%%%%%%%%%%%%%%%%%%%%%%%%%%2.5.1%%%%%%%%%%%%%%%%%%%%%%%%%%%%%%%%%%%%%%
   \item[2.5.1]   Let $H$ and $K$ be subgroups of $G$. Exhibit all possible
                  sublattices which show only $G$, 1, $H$, $K$, and their joins
                  and intersections. What distinguishes the different drawings?
%%%%%%%%%%%%%%%%%%%%%%%%%%%%%%%%%%%%%2.5.2%%%%%%%%%%%%%%%%%%%%%%%%%%%%%%%%%%%%%%
   \item[2.5.2]   In each of (a) to (d) list all subgroups of $D_{16}$ that
                  satisfy the given condition.
                  \begin{enumerate}
                     \item Subgroups that are contained in $\cyc{sr^2, r^4}$
                     \item Subgroups that are contained in $\cyc{sr^7, r^4}$
                     \item Subgroups that contain $\cyc{r^4}$
                     \item Subgroups that contain $\cyc{s}$.
                  \end{enumerate}
                  
      \textbf{Solution.}
      
      \begin{enumerate}
         \item The subgroups that are contained in $\cyc{sr^2, r^4}$ are those
               that have an upward path to the join of $sr^2$ and $r^4$. Thus
               these subgroups are: $1$, $\cyc{r^4}$, $\cyc{sr^2}$,
               $\cyc{sr^6}$, and $\cyc{sr^2, r^4}$.
         \item The subgroups that are contained in $\cyc{sr^7, r^4}$ are those
               that have an upward path to the join of $sr^3$ and $r^4$. Since
               the join of $sr^7$ and $r^4$ is $\cyc{sr^3, r^4}$. It follows
               that these subgroups are: $1$, $\cyc{r^4}$, $\cyc{sr^3}$,
               $\cyc{sr^7}$, and $\cyc{sr^7, r^4} = \cyc{sr^3, r^4}$.
         \item The subgroups that contain $r^4$ are: $\cyc{r^4}$,
               $\cyc{sr^2, r^4}$, $\cyc{s, r^4}$, $\cyc{r^2}$,
               $\cyc{sr^3, r^4}$, $\cyc{sr^5, r^4}$, $\cyc{s, r^2}$, $\cyc{r}$,
               $\cyc{sr, r^2}$, and $D_{16}$.
         \item The subgroups that contain $s$ are: $\cyc{s}$, $\cyc{s, r^4}$,
               $\cyc{s, r^2}$, and $D_{16}$.
      \end{enumerate}
%%%%%%%%%%%%%%%%%%%%%%%%%%%%%%%%%%%%%2.5.3%%%%%%%%%%%%%%%%%%%%%%%%%%%%%%%%%%%%%%
   \item[2.5.3]   Show that the subgroup $\cyc{s, r^2}$ of $D_8$ is isomorphic
                  to $V_4$.
                  
      \textbf{Proof.} By Exercise 1.1.36, there is exactly one group, say $K$,
      of order 4 that has no element of order 4. Since
      $\cyc{s, r^2} = \{1, s, r^2, sr^2\}$, it follows that every non-identity
      element of $\cyc{s, r^2}$ has order 2, so that $\cyc{s, r^2} \cong K$.
      Similarly, the Klein-4 group has no element of order 4. Thus
      $V_4 \cong K$, and we conclude that $V_4 \cong \cyc{s, r^2}$. \qed
%%%%%%%%%%%%%%%%%%%%%%%%%%%%%%%%%%%%%2.5.4%%%%%%%%%%%%%%%%%%%%%%%%%%%%%%%%%%%%%%
   \item[2.5.4]   Use the given lattice to find all pairs of elements that
                  generate $D_8$ (there are 12 pairs).
                  
      \textbf{Solution.} It suffices to find all pairs of elements whose join is
      $D_8$. They are: $\cyc{s, r}$, $\cyc{s, rs}$, $\cyc{s, r^3s}$,
      $\cyc{r^2s, rs}$, $\cyc{r^2s, r^3s}$, $\cyc{r^2s, r}$, $\cyc{r^2s, r^3}$,
      $\cyc{r, rs}$, $\cyc{r, r^3s}$,  $\cyc{r^3, s}$, $\cyc{r^3, rs}$, and
      $\cyc{r^3, r^3s}$,
%%%%%%%%%%%%%%%%%%%%%%%%%%%%%%%%%%%%%2.5.5%%%%%%%%%%%%%%%%%%%%%%%%%%%%%%%%%%%%%%
   \item[2.5.5]   Use the given lattice to find all elements $x \in D_{16}$
                  such that $D_{16} = \cyc{x, s}$ (there are 8 such elements
                  $x$).
                  
      \textbf{Solution.} By observing the given lattice of $D_{16}$, we find
      that      
      $$x \in \{r, r^3, r^5, r^7, sr^3, sr^7, sr^5, sr\}.$$
%%%%%%%%%%%%%%%%%%%%%%%%%%%%%%%%%%%%%2.5.6%%%%%%%%%%%%%%%%%%%%%%%%%%%%%%%%%%%%%%
   \item[2.5.6]   Use the given lattices to help find the centralizers of every
                  element in the following groups:

                  (a) $D_8$ \qquad (b) $Q_8$ \qquad
                  (c) $S_3$ \qquad (d) $D_{16}$.
                  
      \begin{enumerate}
         \item $$
                  \begin{tabular}{@{}|c|c|@{}} \hline
                     Elements in $D_8$ & Centralizer \\ \hline
                     1, $r^2$ & $D_8$ \\ \hline
                     $r, r^3$ & $\cyc{r}$ \\ \hline
                     $s, r^2s$ & $\cyc{s, r^2}$ \\ \hline
                     $rs, r^3s$ & $\cyc{rs, r^2}$ \\ \hline
                  \end{tabular}
               $$
         \item $$
                  \begin{tabular}{@{}|c|c|@{}} \hline
                     Elements in $Q_8$ & Centralizer \\ \hline
                     $\pm1$ & $Q_8$ \\ \hline
                     $\pm i$ & $\cyc{i}$ \\ \hline
                     $\pm j$ & $\cyc{j}$ \\ \hline
                     $\pm k$ & $\cyc{k}$ \\ \hline
                  \end{tabular}
               $$
         \item $$
                  \begin{tabular}{@{}|c|c|@{}} \hline
                     Element(s) in $Q_8$ & Centralizer \\ \hline
                     1 & $S_3$ \\ \hline
                     (1 2) & $\cyc{(1\;2)}$ \\ \hline
                     (1 3) & $\cyc{(1\;3)}$ \\ \hline
                     (2 3) & $\cyc{(2\;3)}$ \\ \hline
                     (1 2 3), (1 3 2) & $\cyc{(1\;2\;3)}$ \\ \hline
                  \end{tabular}
               $$
         \item $$
                  \begin{tabular}{@{}|c|c|@{}} \hline
                     Elements in $D_{16}$ & Centralizer \\ \hline
                     1, $r^4$ & $D_{16}$ \\ \hline
                     $r, r^2, r^3, r^5, r^6, r^7$ & $\cyc{r}$ \\ \hline
                     $s, sr^4$ & $\cyc{s, r^4}$ \\ \hline
                     $sr, sr^5$ & $\cyc{sr^5, r^4}$ \\ \hline
                     $sr^2, sr^6$ & $\cyc{sr^2, r^4}$ \\ \hline
                     $sr^3, sr^7$ & $\cyc{sr^3, r^4}$ \\ \hline
                  \end{tabular}
               $$
      \end{enumerate}
%%%%%%%%%%%%%%%%%%%%%%%%%%%%%%%%%%%%%2.5.7%%%%%%%%%%%%%%%%%%%%%%%%%%%%%%%%%%%%%%
   \item[2.5.7]   Find the center of $D_{16}$.
   
      \textbf{Solution.} From Exercise 2.5.6(d), we see that only 1 and $r^4$
      are in the all the centralizers of the elements of $D_{16}$. Thus
      $Z(D_{16}) = \cyc{r_4}$.
%%%%%%%%%%%%%%%%%%%%%%%%%%%%%%%%%%%%%2.5.8%%%%%%%%%%%%%%%%%%%%%%%%%%%%%%%%%%%%%%
   \item[2.5.8]   In each of the following groups find the normalizer of each
                  subgroup:

                  (a) $S_3$ \qquad (b) $Q_8$.
%%%%%%%%%%%%%%%%%%%%%%%%%%%%%%%%%%%%%2.5.9%%%%%%%%%%%%%%%%%%%%%%%%%%%%%%%%%%%%%%
   \item[2.5.9]   Draw the lattices of subgroups of the following groups:

                  (a) $\Z/16\Z$ \qquad (b) $\Z/24\Z$ \qquad
                  (c) $\Z/48\Z$. [See Exercise 6 in Section 3.]
%%%%%%%%%%%%%%%%%%%%%%%%%%%%%%%%%%%%%2.5.10%%%%%%%%%%%%%%%%%%%%%%%%%%%%%%%%%%%%%
   \item[2.5.10]  Classify groups of order 4 by proving that if $|G| = 4$ then
                  $G \cong Z_4$ or $G\cong V_4$. [See Exercise 36, Section 1.1.]
%%%%%%%%%%%%%%%%%%%%%%%%%%%%%%%%%%%%%2.5.11%%%%%%%%%%%%%%%%%%%%%%%%%%%%%%%%%%%%%
   \item[2.5.11]  Consider the group of order 16 with the following
                  presentation:

                  $$QD_{16} = \cyc{\sigma, \tau : \sigma^8 = \tau^2 = 1,
                    \sigma\tau = \tau\sigma^3}$$
                  (called the \textit{quasidihedral} or \textit{semidihedral}
                  group of order 16). This group has three subgroups of order 8:
                  $\cyc{\tau, \sigma^2} \cong D_8, \cyc{\sigma} \cong Z_8$ and
                  $\cyc{\sigma^2, \sigma\tau} \cong Q_8$ and every proper
                  subgroup is contained in one of these three subgroups. Fill in
                  the missing subgroups in the lattice of all subgroups of the 
                  quasidiheral group on the following page, exhibiting each
                  subgroup with at most two generators. (This is another example
                  of a nonplanar lattice.)
\end{enumerate}

\noindent The next three examples lead to two nonisomorphic groups that have the 
          same lattice of subgroups.

\begin{enumerate}
%%%%%%%%%%%%%%%%%%%%%%%%%%%%%%%%%%%%%2.5.12%%%%%%%%%%%%%%%%%%%%%%%%%%%%%%%%%%%%%
   \item[2.5.12]  The group
                  $A = Z_2 \times Z_4 = \cyc{a, b : a^2 = b^4 = 1, ab = ba}$ has
                  order 8 and has three subgroups of order 4:
                  $\cyc{a, b^2} \cong V_4$, $\cyc{b} \cong Z_4$ and
                  \begin{verbatim}
                     *
                     *
                     *
                     *
                     *
                     *
                     *
                     *
                     *
                  \end{verbatim}
                  $\cyc{ab} \cong Z_4$ and every proper subgroup is contained in
                  one of these three. Draw the lattice of all subgroups of $A$,
                  giving each subgroup in terms of at most two generators.
%%%%%%%%%%%%%%%%%%%%%%%%%%%%%%%%%%%%%2.5.13%%%%%%%%%%%%%%%%%%%%%%%%%%%%%%%%%%%%%
   \item[2.5.13]  The group
                  $G = Z_2 \times Z_8 = \cyc{x, y : x^2 = y^8 = 1, xy = yx}$ has
                  order 16 and has three subgroups of order 8:
                  $\cyc{x, y^2} \cong Z_2 \times Z_4$, $\cyc{y} \cong Z_8$ and
                  $\cyc{xy} \cong Z_8$ and every proper subgroup is contained in
                  one of these three. Draw the lattice of all subgroups of $G$,
                  giving each subgroup in terms of at most two generators.
%%%%%%%%%%%%%%%%%%%%%%%%%%%%%%%%%%%%%2.5.14%%%%%%%%%%%%%%%%%%%%%%%%%%%%%%%%%%%%%
   \item[2.5.14]  Let $M$ be the group of order 16 with the following 
                  presentation:
                  $$\cyc{u, v : u^2 v^8 = 1, vu = uv^5}$$
                  (sometimes called the \textit{modular} group of order 16). It
                  has three subgroups of order 8: $\cyc{u, v^2}$, $\cyc{v}$, and
                  $\cyc{uv}$ and every proper subgroup is contained in one of
                  these three. Prove that $\cyc{u, v^2} \cong Z_2 \times Z_4$,
                  $\cyc{v} \cong Z_8$ and $\cyc{uv} \cong Z_8$. Show that the
                  lattice of subgroups of $M$ is the same as the lattice of
                  subgroups of $Z_2 \times Z_8$ (cf. Exercise 13) but that these
                  two groups are not isomorphic.
%%%%%%%%%%%%%%%%%%%%%%%%%%%%%%%%%%%%%2.5.15%%%%%%%%%%%%%%%%%%%%%%%%%%%%%%%%%%%%%
   \item[2.5.15]  Describe the isomorphism type of each of the three subgroups
                  of $D_{16}$ of order 8.
%%%%%%%%%%%%%%%%%%%%%%%%%%%%%%%%%%%%%2.5.16%%%%%%%%%%%%%%%%%%%%%%%%%%%%%%%%%%%%%
   \item[2.5.16]  Use the lattice of subgroups of the quasidihedral group of
                  order 16 to show that every element of order 2 is contained in
                  the proper subgroup $\cyc{\tau, \sigma^2}$.
%%%%%%%%%%%%%%%%%%%%%%%%%%%%%%%%%%%%%2.5.17%%%%%%%%%%%%%%%%%%%%%%%%%%%%%%%%%%%%%
   \item[2.5.17]  Use the lattice of subgroups of the modular group $M$ of order
                  16 to show that the set $\{x \in M : x^2 = 1\}$ is a subgroup
                  of $M$ isomorphic to the Klein 4-group.
%%%%%%%%%%%%%%%%%%%%%%%%%%%%%%%%%%%%%2.5.18%%%%%%%%%%%%%%%%%%%%%%%%%%%%%%%%%%%%%
   \item[2.5.18]  Use the lattice to help find the centralizer of every element
                  of $QD_{16}$.
%%%%%%%%%%%%%%%%%%%%%%%%%%%%%%%%%%%%%2.5.19%%%%%%%%%%%%%%%%%%%%%%%%%%%%%%%%%%%%%
   \item[2.5.19]  Use the lattice to help find $N_{D_{16}}(\cyc{s, r^4})$.
%%%%%%%%%%%%%%%%%%%%%%%%%%%%%%%%%%%%%2.5.20%%%%%%%%%%%%%%%%%%%%%%%%%%%%%%%%%%%%%
   \item[2.5.20]  Use the lattice of subgroups of $QD_{16}$ to help find the
                  normalizers.

                  (a) $N_{QD_{16}}(\cyc{\tau\sigma})$ \qquad
                  (b) $N_{QD_{16}}(\cyc{\tau, \sigma^4})$.
\end{enumerate}

      \section{Homomorphisms And Isomorphisms}
         Let $G$ and $H$ be groups.
\begin{enumerate}
%%%%%%%%%%%%%%%%%%%%%%%%%%%%%%%%%%%%%1.6.1%%%%%%%%%%%%%%%%%%%%%%%%%%%%%%%%%%%%%%
   \item[1.6.1]   Let $\varphi : G \rightarrow H$ be a homomorphism.
                  \begin{enumerate}
                     \item Prove that $\varphi(x^n) = \varphi(x)^n$ for all
                           $n \in \Z^+$.
                     \item Do part (a) for $n = -1$ and deduce that
                           $\varphi(x^n) = \varphi(x)^n$ for all $n \in \Z$.
                  \end{enumerate}

      \textbf{Solution.}

      \begin{enumerate}
         \item \textbf{Proof.} We shall proceed by induction on $n$. It is clear
               that $\varphi(x^1) = \varphi(x)^1$. Now suppose that
               $\varphi(x^k) = \varphi(x)^k$ for some integer $k$. Thus
               \begin{align*}
                  \varphi(x^{k+1}) &= \varphi(x^kx) \\
                     &= \varphi(x^k)\varphi(x)
                        &[\varphi\text{ is a homomorphism}] \\
                     &= \varphi(x)^k\varphi(x) &[\text{Inductive hypothesis}] \\
                     &= \varphi(x)^{k+1},
               \end{align*}
               so that, by Mathematical Induction, $\varphi(x^n) = \varphi(x)^n$ 
               for all $n \in \Z^+$. \qed
         \item Since
               $$1 \cdot \varphi(1) = \varphi(1) = \varphi(1 \cdot 1) =
                 \varphi(1)\cdot\varphi(1),$$
               it follows by cancellation that $\varphi(1) = 1$. Thus
               $$\varphi(x)\varphi(x^{-1}) = \varphi(xx^{-1}) =\varphi(1) = 1,$$
               so that $\varphi(x^{-1}) = \varphi(x)^{-1}$. Now let $n$ be a
               positive integer. Then it follows that
               \begin{align*}
                  \varphi(x^{-n}) &= \varphi((x^{-1})^n) \\
                     &= \varphi(x^{-1})^n &[\text{1.6.1(a)}] \\
                     &= (\varphi(x)^{-1})^n \\
                     &= \varphi(x)^{-n}.
               \end{align*}
               Moreover $\varphi(x^0) = 1 = \varphi(x)^0$; thus we can conclude 
               that $\varphi(x^n) = \varphi(x)^n$ for all $n \in \Z$.
      \end{enumerate}
%%%%%%%%%%%%%%%%%%%%%%%%%%%%%%%%%%%%%1.6.2%%%%%%%%%%%%%%%%%%%%%%%%%%%%%%%%%%%%%%
   \item[1.6.2]   If $\varphi : G \rightarrow H$ is an isomorphism, prove that
                  $|\varphi(x)| = |x|$ for all $x \in G$. Deduce that any two
                  isomorphic groups have the same number of elements of order
                  $n$ for each $n \in \Z^+$. Is the result true if $\varphi$ is
                  only assumed to be a homomorphism?

      \textbf{Proof.} Assume that $\varphi : G \rightarrow H$ is a group
      isomorphism. Let $x \in G$. Suppose $|x| = n$. By the preceding exercise, 
      we have that $\varphi(x)^n = \varphi(x^n) = \varphi(1) = 1$, so that
      $|\varphi(x)| \le n$. Now suppose that $|\varphi(x)| = m < n$. Then we
      must have that $\varphi(1) = 1 = \varphi(x)^m = \varphi(x^m)$. That is,
      $x^m = 1$ (since $\varphi$ is injective), a contradiction since $|x| = n$.
      Thus $|\varphi(x)| = |x| = n$. Finally suppose that $|x| = \infty$ and
      $|\varphi(x)| = r < \infty$. Then, as previously argued, we must have
      that $x^r = 1$, a contradiction. Thus if $y \in G$, it must follow that
      $|y| = |\varphi(y)|$. \qed

      For a positive integer $n$, we can now exhibit a bijection (using
      $\varphi$) between the elements of $G$ of order $n$ and the elements of
      $H$ of order $n$. Thus any two isomorphic groups must have the same number 
      of elements of order $n$ for each $n \in \Z^+$. If $\varphi$ is only 
      assumed to be a homomorphism then the result is not generally true. 
      Consider the homomorphism
      $$\alpha : S_3 \rightarrow \{1\}.$$
      Although $S_3$ has an element of order 2, the trivial group $\{1\}$ has no 
      element of order 2.
%%%%%%%%%%%%%%%%%%%%%%%%%%%%%%%%%%%%%1.6.3%%%%%%%%%%%%%%%%%%%%%%%%%%%%%%%%%%%%%%
   \item[1.6.3]   If $\varphi : G \rightarrow H$ is an isomorphism, prove that
                  $G$ is abelian if and only if $H$ is abelian. If
                  $\varphi : G \rightarrow H$ is a homomorphism, what additional
                  conditions on $\varphi$ (if any) are sufficient to ensure that
                  if $G$ is abelian, then so is $H$?

      \textbf{Proof.} Suppose $\varphi : G \rightarrow H$ is an isomorphism.

      ($\Rightarrow$) Assume $G$ is abelian. Consider $h_1$ and $h_2$ in $H$.
      Since $\varphi$ is surjective, there exist $g_1$, $g_2 \in G$ such that
      $\varphi(g_1) = h_1$ and $\varphi(g_2) = h_2$. Since $G$ is abelian, we
      have that
      $$h_1h_2 = \varphi(g_1)\varphi(g_2) = \varphi(g_1g_2) = \varphi(g_2g_1) = 
        \varphi(g_2)\varphi(g_1) = h_2h_1,$$
      so that $H$ is also abelian.

      ($\Rightarrow$) Now assume that $H$ is abelian. Consider $g_3$ and $g_4$ 
      in $G$. Since $H$ is abelian, we have that
      $$\varphi(g_3g_4) = \varphi(g_3)\varphi(g_4) = \varphi(g_4)\varphi(g_3) =
        \varphi(g_4g_3).$$
      We thus conclude that $g_3g_4 = g_4g_3$ since $\varphi$ is injective. That
      is, $G$ is abelian. \qed

      Finally suppose $\varphi$ is an homomorphism and $G$ is abelian. Looking
      at the first direction of our proof above, we see that restricting
      $\varphi$ to be surjective is sufficient to make $H$ abelian.
%%%%%%%%%%%%%%%%%%%%%%%%%%%%%%%%%%%%%1.6.4%%%%%%%%%%%%%%%%%%%%%%%%%%%%%%%%%%%%%%
   \item[1.6.4]   Prove that the multiplicative groups $\R - \{0\}$ and
                  $\C - \{0\}$ are not isomorphic.

      \textbf{Proof.} Since $\R - \{0\}$ has no element of order 4, and since
      $\C - \{0\}$ has 2 elements ($i$ and $-i$) of order 4, it follows that
      these two multiplicative groups are not isomorphic. \qed
%%%%%%%%%%%%%%%%%%%%%%%%%%%%%%%%%%%%%1.6.5%%%%%%%%%%%%%%%%%%%%%%%%%%%%%%%%%%%%%%
   \item[1.6.5]   Prove that the additive groups of $\R$ and $\Q$ are not
                  isomorphic.

      \textbf{Proof.} Since $|\R| \neq |\Q|$, it follows that $(\R, +)$ and
      $(\Q, +)$ are not isomorphic. \qed
%%%%%%%%%%%%%%%%%%%%%%%%%%%%%%%%%%%%%1.6.6%%%%%%%%%%%%%%%%%%%%%%%%%%%%%%%%%%%%%%
   \item[1.6.6]   Prove that the additive groups of $\Z$ and $\Q$ are not
                  isomorphic.

      \textbf{Proof.} First we shall show that $(\Q, +)$ is not cyclic. So
      suppose that $(\Q, +) = \cyc{c}$. Thus there exists an integer $n$ such 
      that $nc = \D\frac{c}{2}$, so that $\D n = \frac{1}{2}$, a contradiction. 
      That is, $(\Q, +)$ is not cyclic. Now suppose to the contrary that
      $\varphi : \Z \rightarrow \Q$ is a homomorphism from $(\Z, +)$ to
      $(\Q, +)$. Let $q \in \Q$ and let $z$ be the unique preimage of $q$ under
      $\varphi$. Thus
      \begin{align*}
         q &= \varphi(z) \\
           &= \varphi(z \cdot 1) \\
           &= z \cdot \varphi(1),  &[1.6.1(a)]
      \end{align*}
      so that $\Q = \cyc{\varphi(1)}$, a contradiction. Thus $(\Z, +)$ and
      $(\Q, +)$ are not isomorphic. \qed
%%%%%%%%%%%%%%%%%%%%%%%%%%%%%%%%%%%%%1.6.7%%%%%%%%%%%%%%%%%%%%%%%%%%%%%%%%%%%%%%
   \item[1.6.7]   Prove that $D_8$ and $Q_8$ are not isomorphic.

      \textbf{Proof.} Since $D_8$ has exactly 5 elements of order 2 and $Q_8$
      has exactly 1 element of order 2, it follows that these two groups are not
      isomorphic. \qed
%%%%%%%%%%%%%%%%%%%%%%%%%%%%%%%%%%%%%1.6.8%%%%%%%%%%%%%%%%%%%%%%%%%%%%%%%%%%%%%%
   \item[1.6.8]   Prove that if $n \neq m$, $S_n$ and $S_m$ are not isomorphic.

      \textbf{Proof.} Let $n$ and $m$ be unequal positive integers so that
      $n! \neq m!$; i.e $n! = |S_n| \neq |S_m| = m!$. Thus no bijective map can
      exist between $S_n$ and $S_m$, so that these two groups are not
      isomorphic. \qed
%%%%%%%%%%%%%%%%%%%%%%%%%%%%%%%%%%%%%1.6.9%%%%%%%%%%%%%%%%%%%%%%%%%%%%%%%%%%%%%%
   \item[1.6.9]   Prove that $D_{24}$ and $S_4$ are not isomorphic.

      \textbf{Proof.} Since $D_{24}$ has exactly 2 elements of order 4 and $S_4$
      has exactly 6 elements of order 4, it follows that these two groups are
      not isomorphic. \qed
%%%%%%%%%%%%%%%%%%%%%%%%%%%%%%%%%%%%%1.6.10%%%%%%%%%%%%%%%%%%%%%%%%%%%%%%%%%%%%%
   \item[1.6.10]  Fill in the details of the proof that the symmetric groups
                  $S_\triangle$ and $S_\Omega$ are isomorphic if
                  $|\triangle| = |\Omega|$ as follows: let
                  $\theta : \triangle \rightarrow \Omega$ be a bijection. Define
                  $$\varphi : S_\triangle \rightarrow S_\Omega \qquad
                    \text{by} \qquad \varphi(\sigma) = \theta \circ \sigma \circ 
                    \theta^{-1} \text{ for all } \sigma \in S_\triangle$$
                  and prove the following:
                  \begin{enumerate}
                     \item $\varphi$ is well defined, that is, if $\sigma$ is a
                           permutation of $\triangle$ then
                           $\theta\circ\sigma\circ\theta^{-1}$ is a permutation
                           of $\Omega$.
                     \item $\varphi$ is a bijection from $S_\triangle$ onto
                           $S_\Omega$. [Find a 2-sided inverse for $\varphi$.]
                     \item $\varphi$ is a homomorphism, that is,
                           $\varphi(\sigma\circ\tau) =
                            \varphi(\sigma)\circ\varphi(\tau)$.
                  \end{enumerate}
                  Note the similarity to the \textit{change of basis} or
                  \textit{similarity} transformations for matrices.

      \textbf{Proof.}

      \begin{enumerate}
         \item Let $\sigma \in S_\triangle$. Notice that
               $\varphi(\sigma) = \theta\circ\sigma\circ\theta^{-1}$ maps
               $\Omega$ into $\Omega$, and is also a bijecton since it is a
               composition of bijective maps. Thus
               $\varphi(\sigma) \in S_\Omega$, so that $\varphi$ is well
               defined.
         \item Consider the map
               $$\alpha : S_\Omega \rightarrow S_\triangle \qquad
                 \text{by} \qquad \alpha(\sigma) = \theta \circ \sigma^{-1} 
                 \circ \theta^{-1} \text{ for all } \sigma \in S_\Omega.$$
               A trivial computation will show us that $\varphi\circ\alpha$ is
               the identity map on $S_\Omega$ and $\alpha\circ\varphi$ is the
               identity map on $S_\triangle$. Thus $\varphi$ is a bijection.
         \item Let $\sigma, \tau \in S_\triangle$. So
               \begin{align*}
                  \varphi(\sigma\circ\tau) &=
                     \theta\circ\sigma\circ\tau\circ\theta^{-1} \\
                     &= \theta\circ\sigma\circ\theta^{-1}\circ\theta\circ
                        \tau\circ\theta^{-1} \\
                     &= \varphi(\sigma) \circ \varphi(\tau),
               \end{align*}
               as desired.
      \end{enumerate} \qed
%%%%%%%%%%%%%%%%%%%%%%%%%%%%%%%%%%%%%1.6.11%%%%%%%%%%%%%%%%%%%%%%%%%%%%%%%%%%%%%
   \item[1.6.11]  Let $A$ and $B$ be groups. Prove that
                  $A \times B \cong B\times A$.

      \textbf{Proof.} Consider the map $f : A \times B \rightarrow B \times A$,
      $(a, b) \mapsto (b, a)$. Now define
      $$g : B \times A\rightarrow A\times B, \quad (b, a) \mapsto (a, b).$$
      So $f$ is bijective since $g$ is its two-sided inverse. Now $f$ is a
      homomorphism because
      \begin{align*}
         f((a_1, b_1)(a_2, b_2)) &= f((a_1a_2, b_1b_2)) \\
            &= (b_1b_2, a_1a_2) \\
            &= (b_1, a_1)(b_2, a_2) \\
            &= f((a_1, b_1))f((a_2, b_2)).
      \end{align*} \qed
%%%%%%%%%%%%%%%%%%%%%%%%%%%%%%%%%%%%%1.6.12%%%%%%%%%%%%%%%%%%%%%%%%%%%%%%%%%%%%%
   \item[1.6.12]  Let $A$, $B$, and $C$ be groups and let $G = A \times B$ and
                  $H = B \times C$. Prove that $G \times C \cong A \times H$.

      \textbf{Proof.} Proceed as we did in Exercise 1.6.11 with the following
      modification:
      $$f : G \times C \rightarrow A \times H,
        \quad((a, b), c) \mapsto (a, (b, c))$$ and 
      $$g : A \times H\rightarrow G\times C,
        \quad (a, (b, c)) \mapsto ((a, b), c).$$ \qed
%%%%%%%%%%%%%%%%%%%%%%%%%%%%%%%%%%%%%1.6.13%%%%%%%%%%%%%%%%%%%%%%%%%%%%%%%%%%%%%
   \item[1.6.13]  Let $G$ and $H$ be groups and let $\varphi : G \rightarrow H$
                  be a homomorphism. Prove that the image of $\varphi$,
                  $\varphi(G)$, is a subgroup of $H$. Prove that if $\varphi$ is
                  injective then $G \cong \varphi(G)$.

      \textbf{Proof.} The set $\varphi(G)$ is nonempty since it contains
      $\varphi(1) = 1$. So let $h_1, h_2 \in \varphi(G)$. Thus there exist
      $g_1, g_2 \in G$ such that $\varphi(g_1) = h_1$ and $\varphi(g_2) = h_2$. 
      So
      $$\varphi(g_1g_2^{-1}) = \varphi(g_1)\varphi(g_2)^{-1} = h_1h_2^{-1} \in
        \varphi(G).$$
      That is $\varphi(G)$ is a subgroup of $H$. Now suppose that $\varphi$ is
      injective and consider the map $\alpha : G \rightarrow \varphi(G)$,
      $g \mapsto \varphi(g)$. Since $\varphi$ is an injective homomorphism, it
      follows that $\alpha$ is also an injective homomorphism. Also it is clear
      that $\alpha$ is onto. Thus $\alpha$ is an isomorphism and we have that
      $G \cong \varphi(G)$. \qed
%%%%%%%%%%%%%%%%%%%%%%%%%%%%%%%%%%%%%1.6.14%%%%%%%%%%%%%%%%%%%%%%%%%%%%%%%%%%%%%
   \item[1.6.14]  Let $G$ and $H$ be groups and let $\varphi : G \rightarrow H$
                  be a homomorphism. Define the \text{kernel} of $\varphi$ to be
                  $\{g \in G : \varphi(g) = 1_H\}$. Prove that the kernel of
                  $\varphi$ is a subgroup of $G$. Prove that $\varphi$ is
                  injective if and only if the kernel of $\varphi$ is the
                  identity subgroup of $G$.

      \textbf{Proof 1.} Let ker($\varphi$) be the kernel of $\varphi$. The set
      ker($\varphi$) is not empty because $1_G \in \text{ker}(\varphi)$. Now let
      $x, y \in \text{ker}(\varphi)$. Then we have that
      $$\varphi(xy^{-1}) = \varphi(x)\varphi(y)^{-1} = 1_H{1_H}^{-1} = 1_H,$$
      so that $xy^{-1} \in \text{ker}(\varphi)$. That is ker($\varphi$) is a
      subgroup of $G$. \qed

      \textbf{Proof 2.} ($\Rightarrow$) Assume that $\varphi$ is injective. Let
      $x \in \text{ker}(\varphi)$. Thus we have that
      $\varphi(1_G) = \varphi(x) = 1_H$, so that $x = 1_G$ since $\varphi$ is
      one to one. That is ker($\varphi$) = $\{1_G\}$, the identity subgroup of
      $G$.

      ($\Leftarrow$) Assume that ker($\varphi$) = $\{1_G\}$. Suppose that
      $\varphi(g_1) = \varphi(g_2)$ for some $g_1, g_2 \in G$. Then it follows
      that $\varphi(g_1)\varphi(g_2)^{-1} = 1_H$, so that
      $\varphi(g_1{g_2}^{-1}) = 1_H$. Since ker($\varphi$) = $\{1_G\}$, we can
      conclude that $g_1{g_2}^{-1} = 1_G$, so that $g_1 = g_2$; i.e., $\varphi$
      is injective. \qed      
%%%%%%%%%%%%%%%%%%%%%%%%%%%%%%%%%%%%%1.6.15%%%%%%%%%%%%%%%%%%%%%%%%%%%%%%%%%%%%%
   \item[1.6.15]  Define a map $\pi : \R^2 \rightarrow \R$ by $\pi((x, y)) = x$.
                  Prove that $\pi$ is a homomorphism and find the kernel of
                  $\pi$.

      \textbf{Proof.} Let $(a, b), (c, d) \in \R^2$. Then it follows immediately
      that $\pi$ is a homomorphism since
      $$\pi((a, b) + (c, d)) = \pi((a + c, b + d)) =
         a + c = \pi(a, b) + \pi(c, d).$$
      The kernel of $\pi$ is the set $\{(0, y) : y \in \R\}$. \qed
%%%%%%%%%%%%%%%%%%%%%%%%%%%%%%%%%%%%%1.6.16%%%%%%%%%%%%%%%%%%%%%%%%%%%%%%%%%%%%%
   \item[1.6.16]  Let $A$ and $B$ be groups and let $G$ be their direct product,
                  $A \times B$. Prove that the maps $\pi_1 : G \rightarrow A$
                  and $\pi_2 : G \rightarrow B$ defined by $\pi_1((a, b)) = a$
                  and $\pi_2((a, b)) = b$ are homomorphisms and find their
                  kernels.

      \textbf{Proof.} Let $(a_1, b_1), (a_2, b_2) \in G$. Then it follows 
      immediately that $\pi_1$ and $\pi_2$ are homomorphisms since
      $$\pi_1((a_1, b_1)(a_2, b_2)) = \pi_1((a_1a_2, b_1b_2)) =
         a_1a_2 = \pi_1(a_1, b_1)\pi(a_2, b_2)$$
      and
      $$\pi_2((a_1, b_1)(a_2, b_2)) = \pi_2((a_1a_2, b_1b_2)) =
         b_1b_2 = \pi_1(a_1, b_1)\pi(a_2, b_2).$$
      The kernel of $\pi_1 = \{(1, b) : b \in B\}$ and
      the kernel of $\pi_2 = \{(a, 1) : a \in A\}$. \qed
%%%%%%%%%%%%%%%%%%%%%%%%%%%%%%%%%%%%%1.6.17%%%%%%%%%%%%%%%%%%%%%%%%%%%%%%%%%%%%%
   \item[1.6.17]  Let $G$ be any group. Prove that the map from $G$ to itself
                  defined by $g \mapsto g^{-1}$ is a homomorphism if and only if
                  $G$ is abelian.

      \textbf{Proof.} Let $x, y \in G$. Consider the map
      $\alpha : G \rightarrow G$, $g \mapsto g^{-1}$. 

      ($\Leftarrow$) Assume that $G$ is abelian. Then it
      follows that
      \begin{align*}
         \alpha(xy) &= (xy)^{-1} \\
            &= y^{-1}x^{-1} \\
            &= x^{-1}y^{-1} &[G \text{ is abelian}] \\
            &= \alpha(x)\alpha(y),
      \end{align*}
      so that $\alpha$ is a homomorphism.

      ($\Rightarrow$) Assume that $\alpha$ is a homomorphism. Then it
      follows that
      \begin{align*}
         xy &= \alpha(x^{-1})\alpha(y^{-1}) \\
            &= \alpha(x^{-1}y^{-1}) \\
            &= \alpha((yx)^{-1}) \\
            &= yx,
      \end{align*}
      so that $G$ is abelian. \qed
%%%%%%%%%%%%%%%%%%%%%%%%%%%%%%%%%%%%%1.6.18%%%%%%%%%%%%%%%%%%%%%%%%%%%%%%%%%%%%%
   \item[1.6.18]  Let $G$ be any group. Prove that the map from $G$ to itself
                  defined by $g \mapsto g^2$ is a homomorphism if and only if
                  $G$ is abelian.

      \textbf{Proof.} Let $x, y \in G$. Consider the map
      $\alpha : G \rightarrow G$, $g \mapsto g^2$. 

      ($\Leftarrow$) Assume that $G$ is abelian. Then it
      follows that
      \begin{align*}
         \alpha(xy) &= (xy)^2 \\
            &= x^2y^2 &[G \text{ is abelian}] \\
            &= \alpha(x)\alpha(y),
      \end{align*}
      so that $\alpha$ is a homomorphism.

      ($\Rightarrow$) Assume that $\alpha$ is a homomorphism. Then it
      follows that
      \begin{align*}
         x^2y^2 &= \alpha(x)\alpha(y) \\
            &= \alpha(xy) \\
            &= (xy)^2 \\
            &= xyxy,
      \end{align*}
      so that $xxyy = xyxy$. By cancellation we thus have $xy = yx$; i.e, $G$ is
      abelian. \qed
%%%%%%%%%%%%%%%%%%%%%%%%%%%%%%%%%%%%%1.6.19%%%%%%%%%%%%%%%%%%%%%%%%%%%%%%%%%%%%%
   \item[1.6.19]  Let $G = \{z \in \C : z^n = 1 \text{ for some }n \in \Z^+\}$.
                  Prove that for any fixed integer $k > 1$ the map from $G$ to
                  itself defined by $z \mapsto z^k$ is a surjective homomorphism
                  but is not an isomorphism.

      \textbf{Proof.} Consider an integer $k > 1$ and the map
      $\alpha : G \rightarrow G$, $g \mapsto g^k$. Let $x, y \in G$. The map
      $\alpha$ is a homomorphism since
      $\alpha(xy) = (xy)^k = x^ky^k = \alpha(x)\alpha(y)$. Since $x \in G$, it
      follows that $x^m = 1$ for some positive integer $m$. Notice that
      $x^{1/k} \in G$ since $(x^{1/k})^{km} = 1$. Thus $\alpha$ is onto because
      $\alpha(x^{1/k}) = x$. Now consider the complex number
      $e^{2\pi/k} = \cos2\pi/k + i \sin2\pi/k$. Notice that $e^{2\pi/k} \in G$
      since $(e^{2\pi/k})^k = 1$. Also notice that $e^{2\pi/k} \neq 1$ (since
      $2\pi/k$ is not a multiple of $2\pi$), but we have that
      $\alpha(e^{2\pi/k}) = \alpha(1) = 1$, so that $\alpha$ is not injective; 
      i.e, $\alpha$ is not an isomorphism. \qed
%%%%%%%%%%%%%%%%%%%%%%%%%%%%%%%%%%%%%1.6.20%%%%%%%%%%%%%%%%%%%%%%%%%%%%%%%%%%%%%
   \item[1.6.20]  Let $G$ be a group and let Aut($G$) be the set of all
                  isomorphisms from $G$ onto $G$. Prove that Aut($G$) is a
                  group under function composition (called the
                  \text{automorphism group} of $G$ and the elements of Aut($G$)
                  are called \text{automorphisms} of $G$).

      \textbf{Proof.}

      \textbf{Closure.} Let $\alpha, \gamma \in \text{Aut}(G)$. Since the
      composition of two bijective functions is also bijective, it follows that
      $\alpha \circ \gamma$ is bijective. Now let $x, y \in G$. It follows that
      $$(\alpha\circ\gamma)(xy) = \alpha(\gamma(xy)) =
         \alpha(\gamma(x)\gamma(y)) = \alpha(\gamma(x))\alpha(\gamma(y)) =
         ((\alpha\circ\gamma)(x))((\alpha\circ\gamma)(y)),$$
      so that $\alpha\circ\gamma)$ is also an isomorphism on $G$, and thus,
      Aut($G$) is closed.

      \textbf{Associativity.} This follows from the associativity of functions.

      \textbf{Identity.} The identity map is the identity of Aut($G$).

      \textbf{Inverse.} Since every map in Aut($G$) is bijective, it follows 
      that every map has a 2-sided inverse.

      Thus we have shown that Aut($G$) is a group under composition. \qed      
%%%%%%%%%%%%%%%%%%%%%%%%%%%%%%%%%%%%%1.6.21%%%%%%%%%%%%%%%%%%%%%%%%%%%%%%%%%%%%%
   \item[1.6.21]  Prove that for each fixed nonzero $k \in \Q$ the map from $\Q$
                  to itself defined by $q \mapsto kq$ is an automorphism of
                  $\Q$.

      \textbf{Proof.} Let $k$ be a nonzero rational number. Consider the map
      $f : \Q \rightarrow \Q$, $q \mapsto kq$. Let $x, y \in Q$. We have that
      $f(x + y) = k(x + y) = kx + ky = f(x) + f(y)$, so that $f$ is a 
      homomorphism. Now suppose $f(x) = f(y)$, so that $kx = ky$. Since
      $k \neq 0$, we shall multiply the equality $kx = ky$ by $1/k$ to get
      $x = y$; i.e, $f$ is injective. Since $f(x/k) = x$, it follows that $f$ is
      onto, so that $f$ is an automorphism of $\Q$. \qed
%%%%%%%%%%%%%%%%%%%%%%%%%%%%%%%%%%%%%1.6.22%%%%%%%%%%%%%%%%%%%%%%%%%%%%%%%%%%%%%
   \item[1.6.22]  Let $A$ be an abelian group and fix some $k \in \Z$. Prove
                  that the map $a \mapsto a^k$ is a homomorphism from $A$ to 
                  itself. If $k = -1$ prove that this homomorphism is an
                  isomorphism.

      \textbf{Proof.} Let $k$ be an integer, $x, y \in A$. Consider the map
      $f : A \rightarrow A$, $a \mapsto a^k$. Since $A$ is abelian we have that
      $f(xy) = (xy)^k = x^ky^k = f(x)f(y)$, so that $f$ is a homomorphism. Now 
      assume that $k = -1$. In this case, notice that the map $f$ is also the
      2-sided inverse of $f$; thus $f$ is bijective, so that $f$ is an 
      isomorphism. \qed
%%%%%%%%%%%%%%%%%%%%%%%%%%%%%%%%%%%%%1.6.23%%%%%%%%%%%%%%%%%%%%%%%%%%%%%%%%%%%%%
   \item[1.6.23]  Let $G$ be a finite group which possesses an automorphism
                  $\sigma$ such that $\sigma(g) = g$ if and only if $g = 1$. If
                  $\sigma^2$ is the identity map from $G$ to $G$, prove that $G$
                  is abelian (such an automorphism $\sigma$ is called
                  \text{fixed point free} of order 2). [Hint. Show that every
                  element of $G$ can be written in the form $x^{-1}\sigma(x)$
                  and apply $\sigma$ to such an expression.]

      \textbf{Proof.} Consider the map $\alpha : G \rightarrow G$,
      $g \mapsto g^{-1}\sigma(g)$. Suppose that for some $x, y \in G$, we have
      that $\alpha(x) = \alpha(y)$. Then it follows that
      $x^{-1}\sigma(x) = y^{-1}\sigma(y)$, so that
      $yx^{-1} =\nobreak \sigma(y)\sigma(x)^{-1} = \sigma(yx^{-1})$. Since
      $\sigma(g) = g$ if and only if $g = 1$, we must then have that
      $yx^{-1} = 1$; thus $y = x$, so that $\alpha$ is injective; $\alpha$ is
      also surjective since $G$ is finite. So let $z \in G$. Then we must have 
      that $z = h^{-1}\sigma(h)$ for some $h \in G$. So
      $\sigma(z) = \sigma(h^{-1}\sigma(h)) = \sigma(h)^{-1}\sigma^2(h) =
       \sigma(h)^{-1}h = z^{-1}$. We can then conclude by Exercise 1.6.17 that
      $G$ is abelian. \qed
%%%%%%%%%%%%%%%%%%%%%%%%%%%%%%%%%%%%%1.6.24%%%%%%%%%%%%%%%%%%%%%%%%%%%%%%%%%%%%%
   \item[1.6.24]  Let $G$ be a finite group and let $x$ and $y$ be distinct
                  elements of order 2 in $G$ that generate $G$. Prove that
                  $G \cong D_{2n}$, where $n = |xy|$. [See Exercise 1.2.6]
                  
      \textbf{Proof.} Let $a = xy$. Since $G$ is finite, $|a|$ must also be
      finite. So write $|a| = n$. By Exercise 1.2.6, we have that
      $ax = xa^{-1}$. Also note since $y = x^2y = x(xy) = xa$, the elements
      $a$ and $x$ generate $G$. Thus we have that
      $$G = \cyc{a, x : a^n = x^2 = 1, ax = xa^{-1}}.$$
      By the discussion on Page 38-39 of the Textbook, the map
      $\varphi : D_{2n} \rightarrow G$, given by $\varphi(r) = a$ and
      $\varphi(s) = x$ is an isomorphism. Hence $G \cong D_{2n}$. \qed      
%%%%%%%%%%%%%%%%%%%%%%%%%%%%%%%%%%%%%1.6.25%%%%%%%%%%%%%%%%%%%%%%%%%%%%%%%%%%%%%
   \item[1.6.25]  Let $n \in \Z^+$, let $r$ and $s$ be the usual generators of
                  $D_{2n}$ and let $\theta = 2\pi/n$.
                  \begin{enumerate}
                     \item Prove that the matrix
                           $\left(\begin{tabular}{@{}cr@{}}
                              $\cos\theta$ & $-\sin\theta$ \\
                              $\sin\theta$ & $\cos\theta$
                           \end{tabular}\right)$ is the matrix of the linear
                           transformation which rotates the $x$, $y$ plane about
                           the origin in a counterclockwise direction by
                           $\theta$ radians.
                     \item Prove that the map
                           $\varphi : D_{2n} \rightarrow GL_2(\R)$ defined on
                           generators by
                           $$\varphi(r) = \left(\begin{tabular}{@{}cr@{}}
                              $\cos\theta$ & $-\sin\theta$ \\
                              $\sin\theta$ & $\cos\theta$
                           \end{tabular}\right) \quad\text{and}\quad\varphi(s) =
                           \left(\begin{tabular}{@{}cc@{}}
                              0 & 1 \\
                              1 & 0
                           \end{tabular}\right)$$
                           extends to a homomorphism of $D_{2n}$ into
                           $GL_2(\R)$.
                     \item Prove that the homomorphism $\varphi$ in part (b) is
                           injective.
                  \end{enumerate}
                  
      \textbf{Proof.}
      
      \begin{enumerate}
         \item Consider $T_\theta : \R^2 \rightarrow \R^2$,
               $(x, y) \mapsto (x\cos\theta - y\sin\theta,
               x\sin\theta + y\cos\theta)$, the linear transformation which
               rotates the cartesian plane about the origin in a
               counterclockwise direction by $\theta$ radians. Consider the
               standard basis $\{(1, 0), (0, 1)\}$ for $\R^2$. A simple
               calculation will show us that the matrix of $T_\theta$ with
               respect to the standard basis is $\left(\begin{tabular}{@{}cr@{}}
                  $\cos\theta$ & $-\sin\theta$ \\
                  $\sin\theta$ & $\cos\theta$
               \end{tabular}\right)$. 
         \item It suffices to show that $\varphi(r)$ and $\varphi(s)$ satisfy
               (in $GL_2(\R)$) the relations satisfied by $r$ and $s$
               (in $D_{2n}$). It is clear that $\varphi(s)^2$ is the identity
               matrix. Also $\varphi(r)^n$ is the identity matrix because the
               least (positive) number of rotations we need to get back to the
               starting point is $n$. Finally we have that
               $$\varphi(r)\varphi(s) = \left(\begin{tabular}{@{}rc@{}}
                  $-\sin\theta$ & $\cos\theta$ \\
                  $\cos\theta$ & $\sin\theta$
               \end{tabular}\right) = \varphi(s)\varphi(r)^{-1},$$
               as desired.
         \item First note that (using induction)
               $$\varphi(r^p) = \left(\begin{tabular}{@{}cr@{}}
                  $\cos p\theta$ & $-\sin p\theta$ \\
                  $\sin p\theta$ & $\cos p\theta$
               \end{tabular}\right) \qquad \text{for all } p \in \Z.$$
               Now suppose that $\varphi(r^is^j) = \varphi(r^xs^y)$. That is,
               $\varphi(r^i)\varphi(s^j) = \varphi(r^x)\varphi(s^y)$, so that
               $\varphi(r^{i-x}) = \varphi(s^{y-j})$.

               \textbf{Case 1.} $y - j \equiv 0$ mod 2. That is $s^y = s^j$,
               so that $$\left(\begin{tabular}{@{}cr@{}}
                  $\cos [(i-x)\theta]$ & $-\sin[(i-x)\theta]$ \\
                  $\sin[(i-x)\theta]$ & $\cos[(i-x)\theta]$
               \end{tabular}\right) = \varphi(r^{i-x}) = \varphi(s^{y-j}) =
               \varphi(1) = \left(\begin{tabular}{@{}cc@{}}
                  1 & 0 \\
                  0 & 1
               \end{tabular}\right).$$
               It follows that $(i - x)\theta = 2\pi k$ for some integer $k$.
               Recall that $\theta = 2\pi/n$; thus $i - x = nk$ and we have that
               $r^{i-x} = r^{nk} = 1$, so we can conclude that $r^i = r^x$. We
               have thus shown that $r^is^j = r^xs^y$.

               \textbf{Case 2.} $y - j \equiv 1$ mod 2. Now
               $$\left(\begin{tabular}{@{}cr@{}}
                  $\cos [(i-x)\theta]$ & $-\sin[(i-x)\theta]$ \\
                  $\sin[(i-x)\theta]$ & $\cos[(i-x)\theta]$
               \end{tabular}\right) = \varphi(r^{i-x}) = \varphi(s^{y-j}) =
               \varphi(s) = \left(\begin{tabular}{@{}cc@{}}
                  0 & 1 \\
                  1 & 0
               \end{tabular}\right),$$
               a contradiction because $-1 = \sin[(i-x)\theta] = 1$. Hence the
               only possibility is Case 1, wherein we showed that
               $r^is^j = r^xs^y$, so that $\varphi$ is injective.
      \end{enumerate} \qed
%%%%%%%%%%%%%%%%%%%%%%%%%%%%%%%%%%%%%1.6.26%%%%%%%%%%%%%%%%%%%%%%%%%%%%%%%%%%%%%
   \item[1.6.26]  Let $i$ and $j$ be the generators of $Q_8$ described in
                  Section 5. Prove that the map $\varphi$ from $Q_8$ to
                  $GL_2(\C)$ defined on generators by
                  $$\varphi(i) = \left(\begin{tabular}{@{}cc@{}}
                        $\sqrt{-1}$ & 0 \\
                        0         &  $-\sqrt{-1}$
                    \end{tabular}\right) \qquad \text{and} \qquad
                    \varphi(j) = \left(\begin{tabular}{@{}cr@{}}
                        0 & $-1$ \\
                        1 & 0
                    \end{tabular}\right)$$
                  extends to a homomorphism. Prove that $\varphi$ is injective.

      \textbf{Proof.} To show that $\varphi$ extends to an homomorphism, it
      suffices to show that $\varphi(i)$ and $\varphi(j)$ satisfy
      (in $GL_2(\C)$) the relations satisfied by $i$ and $j$ (in $Q_8$). Using
      Exercise 1.5.3, we must then show that $\varphi(i)^2 = \varphi(j)^2$ and
      $\varphi(i)^4 = 1$. These equalities immediately follow using matrix
      multiplication. Now suppose that $\varphi(i^xj^y) = \varphi(i^mj^n$). Then
      it follows that $\varphi(i)^{x-m} = \varphi(j)^{n-y}$. Notice that
      $|\varphi(i)| = |\varphi(j)| = 4$ and that $\varphi(i)^p = \varphi(j)^q$
      if and only if $p \equiv q \equiv 2$ (or 0) mod 4. So we must have that
      $x - m \equiv n - y \equiv 2 $ (or 0) mod 4. In either case, it follows
      that $i^{x - m} = j^{n - y}$. That is $i^xj^y = i^mj^n$, so that
      $\varphi$ is injective. \qed
\end{enumerate}

      \section{Group Actions}
         \begin{enumerate}
%%%%%%%%%%%%%%%%%%%%%%%%%%%%%%%%%%%%%1.7.1%%%%%%%%%%%%%%%%%%%%%%%%%%%%%%%%%%%%%%
   \item[1.7.1]   Let $F$ be a field. Show that the multiplicative group of
                  nonzero elements of $F$ (denoted by $F^\times$) acts on the
                  set $F$ by $g \cdot a = ga$, where $g \in F^\times$,
                  $a \in F$ and $ga$ is the usual product in $F$ of the two
                  field elements (state clearly which axioms in the definition
                  of a field are used).
%%%%%%%%%%%%%%%%%%%%%%%%%%%%%%%%%%%%%1.7.2%%%%%%%%%%%%%%%%%%%%%%%%%%%%%%%%%%%%%%
   \item[1.7.2]   Show that the additive group $\Z$ acts on itself by
                  $z \cdot a = z + a$ for all $z, a \in \Z$.
%%%%%%%%%%%%%%%%%%%%%%%%%%%%%%%%%%%%%1.7.3%%%%%%%%%%%%%%%%%%%%%%%%%%%%%%%%%%%%%%
   \item[1.7.3]   Show that the additive group $\R$ acts on the $x, y$ plane
                  $\R \times \R$ by $r \cdot (x, y) = (x + ry, y)$.
%%%%%%%%%%%%%%%%%%%%%%%%%%%%%%%%%%%%%1.7.4%%%%%%%%%%%%%%%%%%%%%%%%%%%%%%%%%%%%%%
   \item[1.7.4]   Let $G$ be a group acting on a set $A$ and fix some $a \in A$.
                  Show that the following sets are subgroups of $G$.
                  \begin{enumerate}
                     \item the kernel of the action,
                     \item $\{g \in G : ga = a\}$---this subgroup is called the
                           \textit{stablizer} of $a$ in $G$.
                  \end{enumerate}
%%%%%%%%%%%%%%%%%%%%%%%%%%%%%%%%%%%%%1.7.5%%%%%%%%%%%%%%%%%%%%%%%%%%%%%%%%%%%%%%
   \item[1.7.5]   Prove that the kernel of an action of the group $G$ on the set
                  $A$ is the same as the kernel of the corresponding permutation
                  representation $G \rightarrow S_A$.
%%%%%%%%%%%%%%%%%%%%%%%%%%%%%%%%%%%%%1.7.6%%%%%%%%%%%%%%%%%%%%%%%%%%%%%%%%%%%%%%
   \item[1.7.6]   Prove that a group $G$ acts faithfully on a set $A$ if and
                  only if the kernel of the action is the set consisting only of
                  the identity.
%%%%%%%%%%%%%%%%%%%%%%%%%%%%%%%%%%%%%1.7.7%%%%%%%%%%%%%%%%%%%%%%%%%%%%%%%%%%%%%%
   \item[1.7.7]   Prove that in Example 2 in this section the action is
                  faithful.
%%%%%%%%%%%%%%%%%%%%%%%%%%%%%%%%%%%%%1.7.8%%%%%%%%%%%%%%%%%%%%%%%%%%%%%%%%%%%%%%
   \item[1.7.8]   Let $A$ be a nonempty set and let $k$ be a positive integer
                  with $k \le |A|$. The symmetric group $S_A$ acts on the set
                  $B$ consisting of all subsets of $A$ of cardinality $k$ by
                  $\sigma \cdot \{a_1, \ldots, a_k\} = \{\sigma(a_1), \ldots,
                   \sigma(a_k)\}$.
                  \begin{enumerate}
                     \item Prove that this is a group action.
                     \item Describe explicitly how the elements (1 2) and
                           (1 2 3) act on the six 2-element subsets of
                           $\{1, 2, 3, 4\}$.
                  \end{enumerate}
%%%%%%%%%%%%%%%%%%%%%%%%%%%%%%%%%%%%%1.7.9%%%%%%%%%%%%%%%%%%%%%%%%%%%%%%%%%%%%%%
   \item[1.7.9]   Do both parts of the preceding exercise with ``ordered
                  $k$-tuples" in place of ``$k$-element subsets," where the
                  action on $k$-tuples is defined as above but with set braces
                  replaced by parentheses (note that, for example, the 2-tuples
                  (1, 2) and (2, 1) are different even though the sets
                  $\{1, 2\}$ and $\{2, 1\}$ are the same, so the sets being
                  acted upon are different).
%%%%%%%%%%%%%%%%%%%%%%%%%%%%%%%%%%%%%1.7.10%%%%%%%%%%%%%%%%%%%%%%%%%%%%%%%%%%%%%
   \item[1.7.10]  With reference to the preceding two exercises determine:
                  \begin{enumerate}
                     \item for which values of $k$ the action of $S_n$ on
                           $k$-element subsets if faithful, and
                     \item for which values of $k$ the action of $S_n$ on
                           ordered $k$-tuples is faithful.
                  \end{enumerate}
%%%%%%%%%%%%%%%%%%%%%%%%%%%%%%%%%%%%%1.7.11%%%%%%%%%%%%%%%%%%%%%%%%%%%%%%%%%%%%%
   \item[1.7.11]  Write out the cycle decomposition of the eight permutations in
                  $S_4$ corresponding to the elements of $D_8$ given by the
                  action of $D_8$ on the vertices of a square (where the
                  vertices of the square are labelled as in Section 2).
%%%%%%%%%%%%%%%%%%%%%%%%%%%%%%%%%%%%%1.7.12%%%%%%%%%%%%%%%%%%%%%%%%%%%%%%%%%%%%%
   \item[1.7.12]  Assume $n$ is an even positive integer and show that $D_{2n}$
                  acts on the set consisting of pairs of opposite vertices of a
                  regular $n$-gon. Find the kernel of this action (label
                  vertices as usual).
%%%%%%%%%%%%%%%%%%%%%%%%%%%%%%%%%%%%%1.7.13%%%%%%%%%%%%%%%%%%%%%%%%%%%%%%%%%%%%%
   \item[1.7.13]  Find the kernel of the left regular action.
%%%%%%%%%%%%%%%%%%%%%%%%%%%%%%%%%%%%%1.7.14%%%%%%%%%%%%%%%%%%%%%%%%%%%%%%%%%%%%%
   \item[1.7.14]  Let $G$ be a group and let $A = G$. Show that if $G$ is
                  non-abelian then the maps defined by $g \cdot a = ag$ for all
                  $g, a \in G$ \textit{do not} satisfy the axions of a (left)
                  group action of $G$ on itself.
%%%%%%%%%%%%%%%%%%%%%%%%%%%%%%%%%%%%%1.7.15%%%%%%%%%%%%%%%%%%%%%%%%%%%%%%%%%%%%%
   \item[1.7.15]  Let $G$ be any group and let $A = G$. Show that the maps
                  defined by $g \cdot a = ag^{-1}$ for all $g, a \in G$
                  \textit{do} satisfy the axioms of a (left) group action of $G$
                  on itself.
%%%%%%%%%%%%%%%%%%%%%%%%%%%%%%%%%%%%%1.7.16%%%%%%%%%%%%%%%%%%%%%%%%%%%%%%%%%%%%%
   \item[1.7.16]  Let $G$ be any group and let $A = G$. Show that the maps
                  defined by $g \cdot a = gag^{-1}$ for all $g, a \in G$
                  \textit{do} satisfy the axioms of a (left) group action(this
                  action of $G$ on itself is called \textit{conjugation}).
%%%%%%%%%%%%%%%%%%%%%%%%%%%%%%%%%%%%%1.7.17%%%%%%%%%%%%%%%%%%%%%%%%%%%%%%%%%%%%%
   \item[1.7.17]  Let $G$ be a group and let $G$ act on itself by left 
                  conjugation, so each $g \in G$ maps $G$ to $G$ by
                  $$x \mapsto gxg^{-1}.$$
                  For fixed $g \in G$, prove that conjugation by $g$ is an
                  isomorphism from $G$ onto itself. Deduce that $x$ and
                  $gxg^{-1}$ have the same order for all $x \in G$ and that for
                  any subset $A$ of $G$, $|A| = |gAg^{-1}|$ (here
                  $gAg^{-1} = \{gag^{-1} : a \in A\})$.
%%%%%%%%%%%%%%%%%%%%%%%%%%%%%%%%%%%%%1.7.18%%%%%%%%%%%%%%%%%%%%%%%%%%%%%%%%%%%%%
   \item[1.7.18]  Let $H$ be a group acting on a set $A$. Prove that the
                  relation $\sim$ on $A$ defined by
                  $$a \sim b \quad \text{if and only if} \quad
                    a = hb \quad \text{for some }h \in H$$
                  is an equivalence relation. (For each $x \in A$ the
                  equivalence class of $x$ under $\sim$ is called the
                  \textit{orbit} of $x$ under the action of $H$. The orbits
                  under the action of $H$ partition the set $A$.)
%%%%%%%%%%%%%%%%%%%%%%%%%%%%%%%%%%%%%1.7.19%%%%%%%%%%%%%%%%%%%%%%%%%%%%%%%%%%%%%
   \item[1.7.19]  Let $H$ be a subgroup of the finite group $G$ and let $H$ act
                  on $G$ (here $A = G$) by left multiplication. Let $x \in G$
                  and let $\mathcal{O}$ be the orbit of $x$ under the action of
                  $H$. Prove that the map
                  $$H \rightarrow \mathcal{O}\quad \text{defined by} \quad
                    h \mapsto hx$$
                  is a bijection (hence all orbits have cardinality $|H|$). From
                  this and the preceding exercise deduce
                  $\textit{Lagrange's Theorem}:$
                  \begin{center}
                     \textit{if $G$ is a finite group and $H$ is a subgroup of
                     $G$ then $|H|$ divides $|G|$}.
                  \end{center}
%%%%%%%%%%%%%%%%%%%%%%%%%%%%%%%%%%%%%1.7.20%%%%%%%%%%%%%%%%%%%%%%%%%%%%%%%%%%%%%
   \item[1.7.20]  Show that the group of rigid motions of a tetrahedron is
                  isomorphic to a subgroup of $S_4$.
%%%%%%%%%%%%%%%%%%%%%%%%%%%%%%%%%%%%%1.7.21%%%%%%%%%%%%%%%%%%%%%%%%%%%%%%%%%%%%%
   \item[1.7.21]  Show that the group of rigid motions of a cube is isomorphic
                  to $S_4$. [This group acts on the set of four pairs of
                  opposite vertices.]
%%%%%%%%%%%%%%%%%%%%%%%%%%%%%%%%%%%%%1.7.22%%%%%%%%%%%%%%%%%%%%%%%%%%%%%%%%%%%%%
   \item[1.7.22]  Show that the group of rigid motions of an octahedron is
                  isomorphic to a subgroup of $S_4$. [This group acts on the set
                  of four pairs of opposite faces.] Deduce that the groups of
                  rigid motions of a cube and an octahedron are isomorphic.
                  (These groups are isomorphic because these solids are ``dual"
                  ---see \textit{Introduction to Geometry} by H.Coxeter, Wiley,
                  1961. We shall see later that the groups of rigid motions of
                  the dodecahedron and icosahedron are isomorphic as well---
                  these solids are also dual.)
%%%%%%%%%%%%%%%%%%%%%%%%%%%%%%%%%%%%%1.7.23%%%%%%%%%%%%%%%%%%%%%%%%%%%%%%%%%%%%%
   \item[1.7.23]  Explain why the action of the group of rigid motions of a cube
                  on the set of three pairs of opposite faces is not faithful.
                  Find the kernel of this action.
\end{enumerate}

         
   \chapter{Subgroups}
      \section{Definition And Examples}
         \begin{enumerate}
   \item[]        Let $G$ be a group.
%%%%%%%%%%%%%%%%%%%%%%%%%%%%%%%%%%%Lemm1.1.1%%%%%%%%%%%%%%%%%%%%%%%%%%%%%%%%%%%%
   \item[]        \textbf{Lemma 1.1.1} Let $x \in G$ and let $m$ be an integer.
                  Then we have that
                  $$x^{m+1} = x^mx^1.$$

      \textbf{Proof.}  Consider the following cases:

      \textbf{Case 1:} \textit{$m = 0$}. It follows that
      $$x^{m+1} = x^{0+1} = x^1 = 1x^1 = x^0x^1 = x^mx^1.$$

      \textbf{Case 2:} \textit{$m$ is positive.} Then it follows that $m + 1$ is 
      positive, so that
      \begin{align*}
         x^{m + 1} &= \underbrace{x \cdot x \cdots x}_{m+1\text{ factors}} \\
            &= \underbrace{x \cdot x \cdots x}_{m\text{ factors}} \cdot x^1 \\
            &= x^mx^1.
      \end{align*}

      \textbf{Case 3:} \textit{$m$ is negative.} If $m = -1$, then we have that
      $$x^{m+1} = x^{-1+1} = x^0 = 1 = x^{-1}x^1 = x^mx^1.$$
      If $m < -1$, then $m + 1$ is negative, so that $-(m + 1) = -m - 1$ is 
      positive. Thus
      \begin{align*}
         x^mx^1 &= x^{-(-m)}x^1 \\
                &= \underbrace{x^{-1} \cdot x^{-1} \cdots x^{-1}}_{
                   -m\text{ factors}} \cdot x^1 \\
                &= \underbrace{x^{-1} \cdot x^{-1} \cdots x^{-1}}_{
                   -m-1\text{ factors}} \cdot (x^{-1} \cdot x^1) \\
                &= \underbrace{x^{-1} \cdot x^{-1} \cdots x^{-1}}_{
                   -m-1\text{ factors}} \\
                &= x^{-(-m-1)} \\
                &= x^{m+1}.
      \end{align*}

      In all cases, we can see that our assertion holds. \qed
%%%%%%%%%%%%%%%%%%%%%%%%%%%%%%%%%%%Lemm1.1.2%%%%%%%%%%%%%%%%%%%%%%%%%%%%%%%%%%%%
   \item[]        \textbf{Lemma 1.1.2} \textit{Let $x$ and $g$ be members of a 
                  group $G$, and let $n$ be a positive integer, then it follows 
                  that $(g^{-1}xg)^n = g^{-1}x^ng$.}

      \textbf{Proof.} We shall show by induction that the equation
      \begin{equation}
         (g^{-1}xg)^n = g^{-1}x^ng \label{l1_1_2_1}
      \end{equation}
      holds for every positive integer $n$. It is clear that equation
      \ref{l1_1_2_1} holds for $n = 1$. So assume that it also holds for some
      positive integer $k$. So we must now show that the equation also holds for 
      $k + 1$. Thus
      \begin{align*}
         (g^{-1}xg)^{k+1} &= (g^{-1}xg)^kg^{-1}xg &[\text{Execise 1.1.19}] \\
                     &= g^{-1}x^kgg^{-1}xg &[\text{Inductive hypothesis}] \\
                     &= g^{-1}x^kxg \\
                     &= g^{-1}x^{k+1}g,
      \end{align*}
      so that equation \eqref{l1_1_2_1} holds for $k+1$. Hence by the Principle 
      of Mathematical Induction, equation \eqref{l1_1_2_1} holds for every 
      positive integer $n$. \qed
%%%%%%%%%%%%%%%%%%%%%%%%%%%%%%%%%%%Lemm1.1.3%%%%%%%%%%%%%%%%%%%%%%%%%%%%%%%%%%%%
   \item[]        \textbf{Lemma 1.1.3} \textit{Let $x$ be an element of finite
                  order $n$ in $G$. If $x^m = 1$, then it follows that
                  $n \mid m$.}

      \textbf{Proof.} Suppose that $x^m = 1$. By the Division Algorithm, there
      exist unique integers $q$ and $r$ such that $m = qn + r$ and
      $0 \le r < n$. Now we have that
      $$1 = x^m = x^{qn+r} = x^{qn}x^r = (x^n)^qx^r = 1^qx^r = x^r.$$
      Since $|x| = n$, we cannot have $0 < r < n$; thus the only remaining
      possibility is $r = 0$, so that $n = qm$, as desired. \qed
%%%%%%%%%%%%%%%%%%%%%%%%%%%%%%%%%%%Lemm1.1.4%%%%%%%%%%%%%%%%%%%%%%%%%%%%%%%%%%%%
   \item[]        \textbf{Lemma 1.1.4} \textit{Let $(x, y)$ be an element of
                  $A \times B$ where $A$ and $B$ are groups. For any positive
                  integer $n$, we then have that $(x, y)^n = (x^n, y^n)$.}

      \textbf{Proof.} We shall induct on $n$. Our assertion clearly holds if
      $n$ is 1, so assume that it holds for some positive integer $k$. Thus we
      have that
      \begin{align*}
         (x, y)^{k+1} &= (x, y)(x, y)^k &[\text{Exercise 1.1.19}] \\
                      &= (x, y)(x^k, y^k) &[\text{Inductive hypothesis}] \\
                      &= (xx^k, yy^k) \\
                      &= (x^{k+1}, y^{k+1}). &[\text{Exercise 1.1.19}]
      \end{align*}
      The above shows that our assertion also holds for $k + 1$, so that by
      the Principle of Mathematical Induction it must holds for every integer
      $n$. \qed
%%%%%%%%%%%%%%%%%%%%%%%%%%%%%%%%%%%%%1.1.1%%%%%%%%%%%%%%%%%%%%%%%%%%%%%%%%%%%%%%
   \item[1.1.1]   Determine which of the following binary operations are
                  associative:
                  \begin{enumerate}
                     \item the operation $*$ on $\Z$ defined by $a * b = a - b$.
                     \item the operation $*$ on $\R$ defined by
                           $a * b = a + b + ab$.
                     \item the operation $*$ on $\Q$ defined by
                           $\displaystyle a * b = \frac{a + b}{5}$.
                     \item the operation $*$ on $\Z \times \Z$ defined by
                           $(a, b) * (c, d) = (ad + bc, bd)$.
                     \item the operation $*$ on $\Q - \{0\}$ defined by
                           $\displaystyle a * b = \frac{a}{b}$.
                  \end{enumerate}
                  
      \textbf{Solution.}
   
      \begin{enumerate}
         \item The binary operation $*$ on $\Z$ is not associative because
               $$(0 * 0) * 1 = -1 \neq 1 = 0 * (0 * 1).$$
         \item We claim that $*$ is associative on $\R$.
      
               \textbf{Proof.} Let $r_1, r_2, r_3 \in \R$. Then it follows that
               \begin{align*}
                  (r_1 * r_2) * r_3 &= (r_1 + r_2 + r_1r_2) * r_3 \\
                     &= (r_1 + r_2 + r_1r_2 + r_3) +
                        (r_1r_3 + r_2r_3 + r_1r_2r_3) \\
                     &= r_1 + r_2 + r_3 + r_1r_2 + r_2r_3 +
                        r_1r_3 + r_1r_2r_3 \\
                     &= (r_1 + r_2 + r_3 + r_2r_3) + r_1(r_2 + r_3 + r_2r_3) \\
                     &= r_1 + (r_2 * r_3) + r_1(r_2 * r_3) \\
                     &= r_1 * (r_2 * r_3),
               \end{align*}
               so that our claim holds. \qed
         \item The binary operation $*$ on $\Q$ is not associative because
               $$(0 * 0) * 25 = 5 \neq 1 = 0 * (0 * 25).$$
         \item We claim that $*$ is associative on $\Z \times \Z$.
      
               \textbf{Proof.} Let $(z_1, z_2)$, $(z_3, z_4)$,
               $(z_5, z_6) \in \Z \times \Z$. Then it follows that
               \begin{align*}
                  (z_1, z_2) * [(z_3, z_4) * (z_5, z_6)] 
                     &= (z_1, z_2) * [(z_3z_6 + z_4z_5, z_4z_6)] \\
                     &= (z_1z_4z_6 + z_2z_3z_6 + z_2z_4z_5, z_2z_4z_6) \\
                     &= ((z_1z_4 + z_2z_3) \cdot z_6 + z_2z_4 \cdot z_5,
                          z_2z_4 \cdot z_6) \\
                     &= (z_1z_4 + z_2z_3, z_2z_4) * (z_5, z_6) \\
                     &= [(z_1, z_2) * (z_3, z_4)] * (z_5, z_6),
               \end{align*}
               so that our claim holds. \qed
         \item The binary operation $*$ on $\Q - \{0\}$ is not associative
               because
               $$(4 * 1) * 2 = 2 \neq 8 = 4 * (1 * 2).$$
      \end{enumerate}
%%%%%%%%%%%%%%%%%%%%%%%%%%%%%%%%%%%%%1.1.2%%%%%%%%%%%%%%%%%%%%%%%%%%%%%%%%%%%%%%
   \item[1.1.2]   Decide which of the binary operations in the preceding
                  exercise are commutative.
                  
      \begin{enumerate}      
         \item The binary operation $*$ on $\Z$ is not commutative because
               $$1 * 0 = 1 \neq -1 = 0 * 1.$$
         \item The binary operation $*$ on $\R$ is commutative because addition
               and multiplication are commutative on $\R$.
         \item The binary operation $*$ on $\Q$ is commutative because addition
               is commutative on $\Q$.
         \item A quick check will show us that $*$ is commutative on
               $\Z \times \Z$. That is, for all $(z_1, z_2)$, $(z_3, z_4)$
               $\in \Z \times \Z$, we must have that
               \begin{align*}
                  (z_1, z_2) * (z_3, z_4) &= (z_1z_4 + z_2z_3, z_2z_4) \\
                                          &= (z_3z_2 + z_4z_1, z_4z_2) \\
                                          &= (z_3, z_4) * (z_1, z_2).
               \end{align*}
         \item The binary operation $*$ on $\Q - \{0\}$ is not commutative
               because
               $$1 * 2 = \frac{1}{2} \neq \frac{2}{1} = 2 * 1.$$
      \end{enumerate}
%%%%%%%%%%%%%%%%%%%%%%%%%%%%%%%%%%%%%%1.3%%%%%%%%%%%%%%%%%%%%%%%%%%%%%%%%%%%%%%%
   \item[1.1.3]   Prove that addition of residue classes in $\Z/n\Z$ is
                  associative (you may assume it is well defined).
                  
      \textbf{Proof.} Fix $n \in \Z^+$. Consider $\overline{a}$, $\overline{b}$,
      and $\overline{c}$ in $\Z/n\Z$. By Theorem 3, Pg. 9, we have that
      \begin{align*}
         \overline{a} + (\overline{b} + \overline{c})
            &= \overline{a} + \overline{b + c} \\
            &= \overline{a + b + c} \\
            &= \overline{a + b} + \overline{c} \\
            &= (\overline{a} + \overline{b}) + \overline{c},
      \end{align*}
      so that addition of residue classes in $\Z/n\Z$ is associative. \qed
%%%%%%%%%%%%%%%%%%%%%%%%%%%%%%%%%%%%%%1.4%%%%%%%%%%%%%%%%%%%%%%%%%%%%%%%%%%%%%%%
   \item[1.1.4]   Prove that multiplication of residue classes in $\Z/n\Z$ is
                  associative (you may assume it is well defined).
                  
      \textbf{Proof.} Fix $n \in \Z^+$. Consider $\overline{a}$, $\overline{b}$,
      and $\overline{c}$ in $\Z/n\Z$. By Theorem 3, Pg. 9, we have that
      \begin{align*}
         \overline{a} \cdot (\overline{b} \cdot \overline{c})
            &= \overline{a} \cdot \overline{bc} \\
            &= \overline{abc} \\
            &= \overline{ab} \cdot \overline{c} \\
            &= (\overline{a} \cdot \overline{b}) \cdot \overline{c},
      \end{align*}
      so that multiplication of residue classes in $\Z/n\Z$ is associative. \qed
%%%%%%%%%%%%%%%%%%%%%%%%%%%%%%%%%%%%%%1.5%%%%%%%%%%%%%%%%%%%%%%%%%%%%%%%%%%%%%%%
   \item[1.1.5]   Prove that for all $n > 1$ that $\Z/n\Z$ is not a group under
                  multiplication of residue classes.
                  
      \textbf{Proof.} Let $n$ be positive integer greater than 1. It follows
      that $\Z/n\Z$ is not a group under multiplication because $\overline{0}$
      has no multiplicative inverse. \qed
%%%%%%%%%%%%%%%%%%%%%%%%%%%%%%%%%%%%%%1.6%%%%%%%%%%%%%%%%%%%%%%%%%%%%%%%%%%%%%%%
   \item[1.1.6]   Determine which of the following sets are groups under
                  addition:
                  \begin{enumerate}
                     \item the set of rational numbers (including $0 = 0/1$) in
                           lowest terms whose denominators are odd.
                     \item the set of rational numbers (including $0 = 0/1$) in
                           lowest terms whose denominators are even.
                     \item the set of rational numbers of absolute value $< 1$.
                     \item the set of rational numbers of absolute value $\ge 1$
                           together with 0.
                     \item the set of rational numbers with denominators equal
                           to 1 or 2.
                     \item the set of rational numbers with denominators equal
                           to 1, 2, or 3.
                  \end{enumerate}

      \textbf{Solution.}

      \begin{enumerate}
         \item We claim that the set
               $$S = \left\{\frac{a}{b} \in \Q : b \text{ is odd} \text{ and }
                 \gcd(a, b) = 1\right\},$$
               is a group under addition.

               \textbf{Proof.} First we must show that $S$ is closed under 
               addition. Notice that $S$ is nonempty since it contains 7/5, so 
               let $r, s \in S$. By definition of $S$, we have that
               $r = a_1/b_1$ and $s = a_2/b_2$ for some integers $a_1$ and
               $a_2$, and nonzero integers $b_1$ and $b_2$, where $b_1$ and
               $b_2$ are odd and $\gcd(a_1, b_1) = \gcd(a_2, b_2) = 1$.
               It follows that
               \begin{align*}
                  r + s &= \frac{a_1}{b_1} + \frac{a_2}{b_2} \\
                        &= \frac{a_1b_2 + a_2b_1}{b_1b_2}.
               \end{align*}

               Since $b_1$ and $b_2$ are both odd, it must necessarily be the 
               case that $b_1b_2$ is also odd. In order words, $b_1b_2$ contains 
               no factor of 2, so that if we reduce $r + s$ to its lowest term, 
               the denominator of this lowest term will still be odd. Hence
               $r + s \in S$, so that $S$ is closed under addition. To complete 
               the proof we must now show that $S$ satisfies the group axioms. 
               We observe that $0/1$ is the identity element in $S$. Also, it is 
               clear that for all $s \in S$, we have $-s \in S$, so that every 
               element of $S$ has an inverse under addition. Since
               $S \subseteq \Q$, and since $\Q$ is associative under addition, 
               it follows that $S$ is also associative under addition. Thus $S$ 
               satisfies the group axioms, so that $(S, +)$ is a group. \qed
         \item The set
               $$S = \left\{\frac{a}{b} \in \Q : b \text{ is even} \text{ and }
                 \gcd(a, b) = 1\right\},$$
               is not a group under addition because it is not closed. Indeed,
               for $3/14 \in S$, we have $3/14 + 3/14 = 3/7 \notin S$.
         \item The set
               $$S = \left\{\frac{a}{b} \in \Q :
                     \left|\frac{a}{b}\right| < 1\right\},$$
               is not a group under addition because it is not closed. Indeed,
               for $9/10 \in S$, we have $9/10 + 9/10 = 18/10 \notin S$.
         \item The set
               $$S = \left\{\frac{a}{b} \in \Q : a = 0 \text{ or }
                     \left|\frac{a}{b}\right| \ge 1\right\},$$
               is not a group under addition because it is not closed. Indeed,
               for $-11/10, 10/10 \in S$, we have
               $-11/10 + 10/10 = -1/10 \notin S$.
         \item We claim that the set
               $$S = \left\{\frac{a}{b} \in \Q : b = 1 \text{ or }
                 b = 2\right\},$$
               is a group under addition.

               \textbf{Proof.} It is clear that 0 is the identity for $S$ under
               addition, that $S$ is associative under addition (because
               $S \subset \Q$ and $\Q$ is associative under addition, and that
               the inverse of an element in $S$ is its additive inverse in $\Q$.
               So to complete the proof, we need only show that $S$ is closed
               under addition. Let $a_1/b_1, a_2/b_2 \in \Q$. By observation, we
               note that $a_1/b_1 + a_2/b_2$ must have a denominator of 1 or 2,
               so that it is in $S$. Thus $S$ is closed under addition. \qed
         \item The set
               $$S = \left\{\frac{a}{b} \in \Q : b \in {1, 2, 3} \right\},$$
               is not a group under addition because it is not closed. Indeed,
               for $1/2, 1/3 \in S$, we have $1/2 + 1/3 = 5/6 \notin S$.
      \end{enumerate}
%%%%%%%%%%%%%%%%%%%%%%%%%%%%%%%%%%%%%%1.7%%%%%%%%%%%%%%%%%%%%%%%%%%%%%%%%%%%%%%%
   \item[1.1.7]   Let $G = \{x \in \R : 0 \le x < 1\}$ and for $x, y \in G$ let
                  $x * y$ be the fractional part of $x + y$ (i.e.,
                  $x * y = x + y = [x + y]$ where $[a]$ is the greatest integer
                  less than or equal to $a$). Prove that $*$ is a well defined
                  binary operation on $G$ and that $G$ is an abelian group under
                  $*$ (called the \textit{real numbers mod }1).
                  
      \textbf{Proof.} The set $G$ is clearly non-empty, so consider
      $x, y, z \in G$. To show that $G$ is a group, we shall now prove that it 
      is well defined, associative, has an identity, and is closed under taking
      inverses.

      \textbf{Well Defined:} To show that $*$ is well defined is tantamount to
      showing that $G$ is closed under $*$.  By definition, we have that
      $0 \le x < 1$ and $0 \le y < 1$, so that $0 \le x + y < 2$. If
      $0 \le x + y < 1$, so that $[x + y] = 0$, then we have that
      $$0 \le x + y = x + y - [x + y] = x * y = x + y < 1.$$
      However if $1 \le x + y < 2$, so that $[x + y] = 1$ and
      $0 \le x + y - 1 < 1$, we must have that
      $$0 \le x + y - 1 = x + y - [x + y] = x * y = x + y - 1 < 1.$$
      In either case, we have $0 \le x * y < 1$; i.e. $x * y \in G$, so that $G$ 
      is closed under $*$. Also we have that
      $$x * y = x + y - [x + y] = y + x - [y + x],$$
      so that $G$ is abelian.

      \textbf{Associativity:} We have that
      \begin{align*}
         x * (y * z) &= x * (y + z - [y + z]) \\
              &= x + y + z - [y + z] - [x + y + z - [y + z]], \text{ and} \\ \\
         (x * y) * z &= (x + y - [x + y]) * z \\
                     &=  x + y + z - [x + y] - [x + y + z - [x + y]].
      \end{align*}
      By definition, we have that $0 \le x < 1$, $0 \le y < 1$, and
      $0 \le z < 1$, so that $0 \le x + y < 2$ and $0 \le y + z < 2$. Let us 
      now investigate the following possible cases:

      \textit{Case 1:} \textit{$0 \le x + y  < 1$ and $0 \le y + z < 1$}. That
      is $[x + y] = [y + z] = 0$. It then follows that
      $$x * (y * z) = (x * y) * z = x + y + z - [x + y + z].$$

      \textit{Case 2:} \textit{$1 \le x + y  < 2$ and $1 \le y + z < 2$}. That
      is $[x + y] = [y + z] = 1$. It then follows that
      $$x * (y * z) = (x * y) * z = x + y + z - 1 - [x + y + z - 1].$$

      \textit{Case 3:} \textit{$0 \le x + y  < 1$ and $1 \le y + z < 2$}. That
      is $[x + y] = 0$, and $[y + z] = 1$. It then follows that
      $$(x * y) * z = x + y + z - [x + y + z].$$
      Since $0 \le x + y < 1$ and $0 \le z < 1$, we must have that
      $0 \le x + y + z < 2$. Similarly, since $1 \le y + z < 2$ and
      $0 \le x < 1$, we must have that $1 \le x + y + z < 3$, and since we 
      already showed that $0 \le x + y + z < 2$, it follows that
      $1 \le x + y + z < 2$. Hence $[x + y + z] = 1$. We can then conclude that 
      $(x * y) * z = x +y + z - 1$. Now we have that
      $$x * (y * z) = x + y + z - 1 - [x + y + z - 1].$$
      We already showed that $1 \le x + y + z < 2$; thus,
      $0 \le x + y + z - 1 < 1$, so that $[x + y + z - 1] = 0$; that is,
      $$x * (y * z) = x + y + z - 1 = (x * y) * z.$$
   
      \textit{Case 4:} \textit{$1 \le x + y  < 2$ and $0 \le y + z < 1$}. Apply
      Case 3, with the roles of $x + y$ and $y + z$ interchanged.

      We have thus shown that in all possible cases, we have
      $$x * (y * z) = (x * y) * z,$$
      so that $G$ is associative under $*$.


      \textbf{Identity:} We observe that $0 \in G$ is the identity element since
      $$x * 0 = x + 0 - [x + 0] = x - [x] = x - 0 = x.$$

      \textbf{Inverse:} Suppose $x \neq 0$, so that $0 < x < 1$, and thus
      $0 < 1 - x < 1$; that is $1 - x \in G$. It follows that
      $$x * (1 - x) = x + (1 - x) + [x + (1 - x)] = 1 - 1 = 0,$$
      so that $1 - x$ is the inverse of $x \in G$, with $x \neq 0$. Clearly, the 
      inverse of 0 is 0. \\

      We can now conclude that $(G, *)$ is a group. \qed
%%%%%%%%%%%%%%%%%%%%%%%%%%%%%%%%%%%%%%1.8%%%%%%%%%%%%%%%%%%%%%%%%%%%%%%%%%%%%%%%
   \item[1.1.8]   Let $G = \{z \in \C : z^n = 1 \text{ for some } n \in \Z^+\}$.
                  \begin{enumerate}
                     \item Prove that $G$ is a group under multiplication
                           (called the group of \textit{roots of unity} in
                           $\C$).
                     \item Prove that $G$ is not a group under addition.
                  \end{enumerate}
                  
      \textbf{Proof.}
      
      \begin{enumerate}
         \item We observe that 1 is the identity element of $G$, so that $G$ is
               not empty. So let $x, y, z \in G$.
               
               \textbf{Closure:} By definition, there exist positive integers
               $m$ and $n$ such that $x^m = y^n = 1$. Thus
               $(xy)^{mn} = (x^m)^n(y^n)^m = 1^n1^m = 1$. This says that $G$ is
               closed under multiplication.
               
               \textbf{Associativity:} Since $\C$ is associative under
               multiplication and since $G \subseteq \C$, it follows that $G$ is
               associative under multiplication.
               
               \textbf{Identity:} As state above, the identity of $G$ is clearly
               1.
               
               \textbf{Inverse:} Notice that since
               $(x^{m - 1})^m = (x^m)^{m - 1} = 1$, we must have that
               $x^{m - 1} \in G$. Thus we have $x^{m - 1}x = x^m = 1$; i.e., the
               inverse of $x$ is $x^{m - 1}$.
               
               We have thus shown that $G$ is a group under multiplication. \qed
         \item $G$ is not a group under addition because it is not closed under
               addition. In particular, we have $1 \in G$, but
               $1 + 1 = 2 \notin G$ because $2^n \neq 1$ for any positive
               integer.
      \end{enumerate}
%%%%%%%%%%%%%%%%%%%%%%%%%%%%%%%%%%%%%%1.9%%%%%%%%%%%%%%%%%%%%%%%%%%%%%%%%%%%%%%%
   \item[1.1.9]   Let $G = \{a + b\sqrt{2} \in \R : a, b \in \Q\}$.
                  \begin{enumerate}
                     \item Prove that $G$ is a group under addition.
                     \item Prove that the nonzero elements of $G$ are a group 
                           under multiplication. [``Rationalize the
                           denominators" to find multiplicative inverse.]
                  \end{enumerate}
                  
      \textbf{Proof.}
      
      \begin{enumerate}
         \item \textbf{Closure:} $G$ is clearly nonempty, so let $x, y \in G$.
               By definition of $G$, it follows that $x = a_1 + b_1\sqrt{2}$ and
               $y = a_2 + b_2\sqrt{2}$ for some rational numbers $a_1$, $b_1$,
               $a_2$, and $b_2$. Thus
               $$x + y = (a_1 + a_2) + (b_1 + b_2)\sqrt{2} \in G,$$
               so that $G$ is closed under addition.
               
               \textbf{Associativity:} Since $\R$ is associative under addition
               and since $G \subseteq \R$, it follows that $G$ is associative
               under addition.
               
               \textbf{Identity:} The identity of $G$ is 0.
               
               \textbf{Inverse:} For an element $x = a_1 + b_1\sqrt{2} \in G$,
               the additive inverse of $x$ is $-a_1 - b_1\sqrt{2} \in G$.
               
               We have thus shown that $G$ is a group under addition. \qed
         \item Let $G^{\times}$ denote the set of nonzero elements of $G$.
         
               \textbf{Closure:} Let $x, y \in G^{\times}$. By definition of
               $G$, it follows that $x = a_1 + b_1\sqrt{2}$ and
               $y = a_2 + b_2\sqrt{2}$ for some rational numbers $a_1$, $b_1$,
               $a_2$, and $b_2$, with $a_1$ and $b_1$ not both zero and $a_2$
               and $b_2$ not both zero. Thus
               $$xy = (a_1a_2 + 2b_1b_2) + (a_1b_2 + a_2b_1)\sqrt{2}.$$
               Since neither $x$ nor $y$ is zero, it must be the case that $xy$
               is not zero, so that $G^{\times}$ is closed under multiplication.
               
               \textbf{Associativity:} Since $\R$ is associative under
               multiplication and since $G^{\times} \subseteq \R$, it follows
               that $G^{\times}$ is associative under multiplication.
               
               \textbf{Identity:} The element $1 = 1 + 0\sqrt{2} \in G^{\times}$
               is the identity of $G^{\times}$.
               
               \textbf{Inverse:} Let $x = a_1 + b_1\sqrt{2} \in G^{\times}$.
               Since $x \neq 0$, the real number $1/x$ exists, and we have that
               $$\frac{1}{x} = \frac{1}{a_1 + b_1\sqrt{2}}
                 \frac{a_1 - b_1\sqrt{2}}{a_1 - b_1\sqrt{2}} =
                 \left(\frac{a_1}{{a_1}^2 - 2{b_1}^2} -
                 \frac{b_1}{{a_1}^2 - 2{b_1}^2}\sqrt{2}\right) \in G^{\times}.
               $$
               
               Since $1/x \in G^{\times}$ and since $x \cdot 1/x = 1$, we have
               that $1/x$ is the multiplicative inverse of $x$.
               
               We have thus shown that $G^{\times}$ is a group under
               multiplication. \qed
      \end{enumerate}
%%%%%%%%%%%%%%%%%%%%%%%%%%%%%%%%%%%%%%1.10%%%%%%%%%%%%%%%%%%%%%%%%%%%%%%%%%%%%%%
   \item[1.1.10]  Prove that a finite group is abelian if and only if its group
                  table is a symmetric matrix.
                  
      \textbf{Proof.} Let $G$ be a group such that $|G| = n \in \Z^+$, and let
      $(a_{ij})$ denote the matrix of the group table of $G$. Since $G$ is
      finite, we can enumerate the elements of $G$ like so:
      $$G = \{g_1, g_2, \ldots, g_n\}.$$      
      $(\Leftarrow)$ Suppose that $(a_{ij})$ is a symmetric matrix. Let
      $a, b \in G$. Then we have that $a = g_r$ and $b = g_s$ for some
      $r, s \in \{1, 2, \ldots, n\}$. Since $(a_{ij})$ is symmetric, we must
      have that
      $$ab = g_rg_s = a_{rs} = a_{sr} = g_sg_r = ba,$$
      so that $G$ is abelian.
      
      $(\Rightarrow)$ Now suppose that $G$ is abelian. Consider
      $a_{rs} \in (a_{ij})$. It follows that
      $$a_{rs} = g_rg_s = g_sg_r = a_{sr},$$
      so that $(a_{ij})$ is symmetric. \qed      
%%%%%%%%%%%%%%%%%%%%%%%%%%%%%%%%%%%%%%1.11%%%%%%%%%%%%%%%%%%%%%%%%%%%%%%%%%%%%%%
   \item[1.1.11]  Find the orders of each element of the additive group
                  $\Z/12\Z$.
                  
      \textbf{Solution.} The orders of the elements $\overline{0}$,
      $\overline{1}$, $\overline{2}$, $\overline{3}$, $\overline{4}$,
      $\overline{5}$, $\overline{6}$, $\overline{7}$, $\overline{8}$,
      $\overline{9}$, $\overline{10}$, and $\overline{11}$ in $\Z/12\Z$ are
      1, 12, 6, 4, 3, 12, 2, 12, 3, 4, 6, and 12.
%%%%%%%%%%%%%%%%%%%%%%%%%%%%%%%%%%%%%%1.12%%%%%%%%%%%%%%%%%%%%%%%%%%%%%%%%%%%%%%
   \item[1.1.12]  Find the orders of the following elements of the
                  multiplicative group $(\Z/12\Z)^\times: \overline{1},
                  \overline{-1}, \overline{5}, \overline{7}, \overline{-7}, 
                  \overline{13}$.
                  
      \textbf{Solution.} The orders of the elements $\overline{1}$,
      $\overline{-1}$, $\overline{5}$, $\overline{7}$, $\overline{-7}$,
      $\overline{13}$ in $(\Z/12\Z)^\times$ are 1, 11, 5, 7, 5, and 13.
%%%%%%%%%%%%%%%%%%%%%%%%%%%%%%%%%%%%%%1.13%%%%%%%%%%%%%%%%%%%%%%%%%%%%%%%%%%%%%%
   \item[1.1.13]  Find the orders of the following elements of the additive
                  group $\Z/36\Z: \overline{1}, \overline{2}, \overline{6}, 
                  \overline{9}, \overline{10}, \overline{12}, \overline{-1}, 
                  \overline{-10}, \overline{-18}$.
                  
      \textbf{Solution.} The orders of the elements $\overline{1}$,
      $\overline{2}$, $\overline{6}$, $\overline{9}$, $\overline{10}$,
      $\overline{12}$, $\overline{-1}$, $\overline{-10}$, and $\overline{-18}$
      in $\Z/36\Z$ are 1, 18, 6, 4, 18, 3, 36, 18, and 2.
%%%%%%%%%%%%%%%%%%%%%%%%%%%%%%%%%%%%%%1.14%%%%%%%%%%%%%%%%%%%%%%%%%%%%%%%%%%%%%%
   \item[1.1.14]  Find the orders of the following elements of the
                  multiplicative group $(\Z/36\Z)^\times: \overline{1},
                  \overline{-1}, \overline{5}, \overline{13}, \overline{-13},
                  \overline{17}$.
                  
      \textbf{Solution.} The orders of the elements $\overline{1}$,
      $\overline{-1}$, $\overline{5}$, $\overline{13}$, $\overline{-13}$,
      $\overline{17}$ in $(\Z/36\Z)^\times$ are 1, 35, 29, 25, 11, and 17.
%%%%%%%%%%%%%%%%%%%%%%%%%%%%%%%%%%%%%%1.15%%%%%%%%%%%%%%%%%%%%%%%%%%%%%%%%%%%%%%
   \item[1.1.15]  Prove that $(a_1a_2\cdots a_n)^{-1} =
                  {a_n}^{-1}{a_{n-1}}^{-1}\cdots {a_1}^{-1}$ for all
                  $a_1, a_2, \ldots, a_n \in G$.
                  
      \textbf{Proof.} We shall proceed by induction on $n$. The statement is
      trivial for $n = 1$. So assume that it also holds for some positive
      integer $k$. Let $b = a_1a_2\cdots a_k$. It then follows that
      \begin{align*}
         (a_1a_2\cdots a_ka_{k+1})^{-1} &= (b \cdot a_{k+1})^{-1} \\
            &= {a_{k+1}}^{-1}b^{-1} &[\text{By Proposition 1 (4)}] \\
            &= {a_{k+1}}^{-1}{a_k}^{-1}\cdots {a_1}^{-1}.
                  &[\text{Inductive hypothesis}]
      \end{align*}
      That is, our statement holds for $k + 1$, so that, by the Principle of
      Mathematical Induction, it holds for each positive integer $n$. \qed
%%%%%%%%%%%%%%%%%%%%%%%%%%%%%%%%%%%%%%1.16%%%%%%%%%%%%%%%%%%%%%%%%%%%%%%%%%%%%%%
   \item[1.1.16]  Let $x$ be an element of $G$. Prove that $x^2 = 1$ if and only
                  if $|x|$ is either 1 or 2.
                  
      \textbf{Proof.}
      
      $(\Leftarrow)$ Suppose that $x^2 = 1$. Now if $|x| > 2$, then by
      definition, $x^2 \neq 1$. The only remaining possibilities are $|x| = 1$
      or $|x| = 2$.
      
      $(\Rightarrow)$ Suppose that $|x| = 1$ or $|x| = 2$. It immediately
      follows that $x^2 = 1$. \qed
%%%%%%%%%%%%%%%%%%%%%%%%%%%%%%%%%%%%%%1.17%%%%%%%%%%%%%%%%%%%%%%%%%%%%%%%%%%%%%%
   \item[1.1.17]  Let $x$ be an element of $G$. Prove that if $|x| = n$ for some
                  positive integer $n$ then $x^{-1} = x^{n-1}$.
                  
      \textbf{Proof.} Suppose that $|x| = n \in \Z^+$. By Exercise 1.1.18, it
      follows that $x^{n-1}x^1 = x^{n-1+1} = x^n = 1$, so that
      $x^{-1} = x^{n-1}$. \qed      
%%%%%%%%%%%%%%%%%%%%%%%%%%%%%%%%%%%%%%1.18%%%%%%%%%%%%%%%%%%%%%%%%%%%%%%%%%%%%%%
   \item[1.1.18]  Let $x$ and $y$ be elements of $G$. Prove that $xy = yx$ if
                  and only if $y^{-1}xy =x$ if and only if $x^{-1}y^{-1}xy = 1$.
                  
      \textbf{Proof.} First assume that $xy = yx$. We then have that
      $yx = xy = 1xy = yy^{-1}xy$, so that $x = y^{-1}xy$ by left cancellation.
      Now assume that $y^{-1}xy = x$. Thus
      $x1 = x = y^{-1}xy = 1y^{-1}xy = xx^{-1}y^{-1}xy$, so that
      $1 = x^{-1}y^{-1}xy$ by left cancellation. Finally assume that
      $x^{-1}y^{-1}xy = 1$. Multiplying on the left by $yx$ will yield the
      equation $xy = yx$. \qed
%%%%%%%%%%%%%%%%%%%%%%%%%%%%%%%%%%%%%%1.19%%%%%%%%%%%%%%%%%%%%%%%%%%%%%%%%%%%%%%
   \item[1.1.19]  Let $x \in G$ and let $a, b \in \Z^+$.
                  \begin{enumerate}
                     \item Prove that $x^{a+b} = x^ax^b$.
                     \item Prove that $(x^a)^b = x^{ab}$.
                     \item Prove that $(x^a)^{-1} = x^{-a}$.
                     \item Establish part (a) for arbitrary integers $a$ and $b$
                           (positive, negative or zero).
                     \item Establish part (b) for arbitrary integers $a$ and $b$
                           (positive, negative or zero).
                  \end{enumerate}
               
      \textbf{Proof.}
      
      \begin{enumerate}
         \item We have that
               \begin{align*}
                  x^{a+b} &= \underbrace{xx\cdot x}_{a+b \text{ factors}} \\
                          &= \underbrace{xx\cdot x}_{a \text{ factors}}\mbox{ }
                             \underbrace{xx\cdot x}_{b \text{ factors}} \\
                          &= x^ax^b.
               \end{align*} \qed
         \item We have that
               \begin{align*}
                  (x^a)^b &= (\underbrace{xx\cdot x}_{a \text{ factors}})^b \\
                          &= \underbrace{xx\cdot x}_{ab \text{ factors}} \\
                          &= x^{ab}.
               \end{align*} \qed
         \item We have
               \begin{align*}
                  (x^a)^{-1}
                     &= (\underbrace{xx\cdot x}_{a \text{ factors}})^{-1} \\
                     &= \underbrace{x^{-1}x^{-1}\cdot x^{-1}}_{
                           a \text{ factors}} &[\text{Exercise 1.1.15}] \\
                     &= x^{-a}.
               \end{align*} \qed
         \item Now suppose that $a$ is an integer and $b$ is a positive integer.
               We shall induct on $b$ to show that
               \begin{equation}
                  x^{a+b} = x^ax^b. \label{1_1_19_1}
               \end{equation}
               By Lemma 1.1.1, \eqref{1_1_19_1} holds if $b$ equals 1. So assume
               that it also holds for some positive integer $k$. We now have
               that
               \begin{align*}
                  x^ax^{k+1} &= x^ax^kx^1 &[\text{Lemma 1.1.1}] \\
                             &= (x^ax^k)x^1 \\
                             &= x^{a+k}x^1 &[\text{Inductive hypothesis}] \\
                             &= x^{(a+k)+1} &[\text{Lemma 1.3.2}] \\
                             &= x^{a+(k+1)}, &[\text{Associativity of addition}]
               \end{align*}
               so that \eqref{1_1_19_1} holds for $k + 1$, and hence, by the 
               Principle of Mathematical Induction, it holds for each positive
               integer $n$. \\

               If $a$ is 0 or $b$ is 0, then Lemma 1.1.1 tells us that
               \eqref{1_1_19_1} holds, so the only remaining possibility is $a$ 
               and $b$ are negative.\footnote{If $a$ is positive and $b$ is
               negative, then interchange the roles of $a$ and $b$ in the 
               induction proof.} Now suppose that $a$ and $b$ are negative.
               Hence
               \begin{align*}
                  x^ax^b &= x^{-(-a)}x^{-(-b)} \\
                     &= (x^{-1})^{-a}(x^{-1})^{-b} &[\text{Definition}] \\
                     &= (x^{-1})^{(-a + (-b))} &[\text{Part (a)}] \\
                     &= x^{-(-a + (-b))} &[\text{Definition}] \\
                     &= x^{a+b}.
               \end{align*}

               Combining this result with part (a), we thus shown that
               \eqref{1_1_19_1} holds for all integers $a$ and $b$. \qed
         \item It is clear that part (b) holds if $a$ is 0 or $b$ is 0, so let
               us complete the proof for arbritrary integers $a$ and $b$.

               \textbf{Case 1:} \textit{$a$ is positive and $b$ is negative}. 
               Hence
               \begin{align*}
                  (x^a)^b &= (x^a)^{-(-b)} \\
                          &= [(x^a)^{-1}]^{-b} &[\text{Definition}] \\
                          &= (x^{-a})^{-b} &[\text{Part (c)}] \\
                          &= [(x^{-1})^a]^{-b} &[\text{Definition}] \\
                          &= (x^{-1})^{-ab} &[\text{Part (b)}] \\
                          &= x^{-(-ab)} &[\text{Definition}] \\
                          &= x^{ab}.
               \end{align*}

               \textbf{Case 2:} \textit{$a$ and $b$ are negative}. Thus
               \begin{align*}
                  (x^a)^b &= [x^{-(-a)}]^b \\
                          &= [(x^{-1})^{-a}]^b &[\text{Definition}] \\
                          &= (x^{-1})^{-ab} &[\text{Case 1}] \\
                          &= [(x^{-1})^{-1}]^{ab}. &[\text{Definition}] \\
                          &= x^{ab}. &[\text{Proposition 1 (3)}]
               \end{align*}

               \textbf{Case 3:} \textit{$a$ is negative and $b$ is positive}. 
               Thus
               \begin{align*}
                  (x^a)^b &= [x^{-(-a)}]^b \\
                          &= [(x^{-1})^{-a}]^b &[\text{Definition}] \\
                          &= (x^{-1})^{-ab} &[\text{Case 1}] \\
                          &= x^{-(-ab)} &[\text{Definition}] \\
                          &= x^{ab}.
               \end{align*}

               Combining these results with part (a), we can conclude that
               $(x^a)^b = x^{ab}$ holds for all integers $a$ and $b$ and
               $x \in G$. \qed
      \end{enumerate}
%%%%%%%%%%%%%%%%%%%%%%%%%%%%%%%%%%%%%%1.20%%%%%%%%%%%%%%%%%%%%%%%%%%%%%%%%%%%%%%
   \item[1.1.20]  For $x$ an element in $G$ show that $x$ and $x^{-1}$ have the
                  same order.

      \textbf{Proof.}

      \textbf{Case 1:} \textit{$|x| = n \in \Z^+$}. Since
      $(x^{-1})^n = (x^n)^{-1} = 1^{-1} = 1$, it follows that $|x^{-1}| \le n$,
      so suppose to the contrary that $|x^{-1}| = m < n$. Then we have that
      $$x^m = [(x^{-1})^{-1}]^m = [(x^{-1})^m]^{-1} = 1^{-1} = 1,$$
      a contradiction, so that $|x^{-1}| = n = |x|$.

      \textbf{Case 2:} \textit{$|x| = +\infty$}. Suppose to the contrary that
      $|x^{-1}| = n \in \Z^+$. As we argued in Case 1, it must be the case that
      $x^n = 1$, a contradiction. Thus $|x| = +\infty = |x^{-1}|$. \qed
%%%%%%%%%%%%%%%%%%%%%%%%%%%%%%%%%%%%%%1.20%%%%%%%%%%%%%%%%%%%%%%%%%%%%%%%%%%%%%%
   \item[1.1.21]  Let $G$ be a finite group and let $x$ be an element of $G$ of
                  order $n$. Prove that if $n$ is odd, then $x = (x^2)^k$ for
                  some $k$.

      \textbf{Proof.} Suppose that $n$ is odd. We can then write $n = 2k + 1$
      for some nonnegative integer $k$. By supposition, we have that
      $xx^{2k} = x^{2k+1} = 1 = x^{-2k}x^{2k}$, so that by right cancellation,
      we can conclude that $x = x^{-2k} = (x^2)^{-k}$. \qed
%%%%%%%%%%%%%%%%%%%%%%%%%%%%%%%%%%%%%%1.22%%%%%%%%%%%%%%%%%%%%%%%%%%%%%%%%%%%%%%
   \item[1.1.22]  If $x$ and $g$ are elements of the group $G$, prove that
                  $|x| = |g^{-1}xg|$. Deduce that $|ab| = |ba|$ for all
                  $a, b \in G$.

      \textbf{Proof.} Let $x, g \in G$.

      \textbf{Case 1:} \textit{$|x| = n \in \Z^+$}. By Lemma 1.1.2, it follows
      that $(g^{-1}xg)^n = g^{-1}x^ng = g^{-1}g =1$, so that $|g^{-1}xg| \le n$,
      so suppose to the contrary that $|g^{-1}xg| = m < n$. Then we have that
      $$g^{-1}1g = 1 = (g^{-1}xg)^m = g^{-1}x^mg,$$
      so that $x^m = 1$ by left and right cancellations, a contradiction; thus,  
      $|g^{-1}xg| = n = |x|$.

      \textbf{Case 2:} \textit{$|x| = +\infty$}. Suppose to the contrary that
      $|g^{-1}xg| = n \in \Z^+$. As we argued in Case 1, it must then be the 
      case that $x^n = 1$, a contradiction. Thus $|x| = +\infty = |g^{-1}xg|$.

      Now consider $a, b \in G$. Set $x = ab$ and $g = a$. Since 
      $|x| = |g^{-1}xg|$, it follows that $|ab| = |a^{-1}aba| = |ba|$. \qed
%%%%%%%%%%%%%%%%%%%%%%%%%%%%%%%%%%%%%%1.23%%%%%%%%%%%%%%%%%%%%%%%%%%%%%%%%%%%%%%
   \item[1.1.23]  Suppose $x \in G$ and $|x| = n < \infty$. If $n = st$ for some
                  positive integers $s$ and $t$, prove that $|x^s| = t$.

      \textbf{Proof.} Suppose $n = st$ for some positive integers $s$ and $t$.
      By supposition, we have that $1 = x^n = x^{st} = (x^s)^t$; i.e.,
      $|x^s| \le t$. Suppose to the contrary that $|x^s| = m < t$. Then we have
      that $1 = (x^s)^m = x^{sm}$. Since $0 < m < t$, it follows that
      $0 < sm < st = n$. However $|x| = n$ and we just showed that $x^{sm} = 1$, 
      so that we have a contradiction. Hence we can conclude that $|x^s| = |t|$.
      \qed
%%%%%%%%%%%%%%%%%%%%%%%%%%%%%%%%%%%%%%1.24%%%%%%%%%%%%%%%%%%%%%%%%%%%%%%%%%%%%%%
   \item[1.1.24]  If $a$ and $b$ are \textit{commuting} elements of $G$, prove 
                  that $(ab)^n = a^nb^n$ for all $n \in \Z$. [Do this by 
                  induction for positive $n$ first.]

      \textbf{Proof.} Let $R(n)$ be the statement that $(ab)^n = a^nb^n$, for
      commuting elements $a$ and $b$.
               
      We now want to show using induction that $R(n)$ holds for every positive 
      integer $n$. It is clear that $R(1)$ is true. So suppose that $R(k)$ is 
      true for some positive integer $k$. We must now show that $R(k + 1)$ is 
      also true. Now we have that
      \begin{align*}
         (ab)^{k+1} &= (ab)^k(ab)^1 &[\text{Exercise 1.1.19}] \\
                    &= a^kb^k(ab)^1 &[\text{Since }R(k) \text{ is true}] \\
                    &= a^kb^k(ba)^1 &[ab = ba] \\
                    &= a^kb^kba \\
                    &= a^kb^{k+1}a \\
                    &= a^kab^{k+1} &[\text{$a$ commutes with $b$}] \\
                    &= a^{k+1}b^{k+1}, \\
      \end{align*}
      so that $R(k + 1)$ holds. It follows by the Principle of Mathematical 
      Induction that $R(n)$ holds for every positive integer $n$. By inpsection 
      we can see that $R(0)$ also holds. To complete the proof, we must now show 
      that $(ab)^{m} = a^mb^m$, where $m$ is a negative integer. First we notice 
      that
      \begin{equation}
         a^{-1}b^{-1} = (ba)^{-1} = (ab)^{-1} = b^{-1}a^{-1},
         \label{1_1_24_1}
      \end{equation}
      so that $a^{-1}$ and $b^{-1}$ are commuting elements. Thus it follows that
      \begin{align*}
         (ab)^m &= (ab)^{-(-m)} \\
                &= [(ab)^{-1}]^{-m} &[\text{Definition}] \\
                &= (a^{-1}b^{-1})^{-m} &[\eqref{1_1_24_1}] \\
                &= (a^{-1})^{-m}(b^{-1})^{-m} &[\text{$R(-m)$ holds}] \\
                &= a^mb^m,
      \end{align*}
      as desired. \qed
%%%%%%%%%%%%%%%%%%%%%%%%%%%%%%%%%%%%%%1.25%%%%%%%%%%%%%%%%%%%%%%%%%%%%%%%%%%%%%%
   \item[1.1.25]  Prove that if $x^2 = 1$ for all $x \in G$ then $G$ is abelian.

      \textbf{Proof.} Let $G$ be a group. Suppose that $x^2 = 1$ for all
      $x \in G$. We want to show that $G$ is abelian; that is, we want to show 
      that $xy = yx$ for all $x, y \in G$. So let $x, y \in G$. By hypothesis, 
      we have that $x^2 = e$, $y^2 = e$, and $(xy)^2 = e$, so that according to 
      Proposition 2, we must have that $x = x^{-1}$, $y = y^{-1}$, and
      $xy = (xy)^{-1}$. Thus
      \begin{align*}
         xy &= (xy)^{-1}      &[\text{By Hypothesis}] \\
            &= y^{-1}x^{-1}   &[\text{Proposition 1}] \\
            &= yx.
      \end{align*}
      Thus $G$ is abelian. \qed
%%%%%%%%%%%%%%%%%%%%%%%%%%%%%%%%%%%%%%1.26%%%%%%%%%%%%%%%%%%%%%%%%%%%%%%%%%%%%%%
   \item[1.1.26]  Assume $H$ is a nonempty subset of $(G, *)$ which is closed 
                  under the binary operation on $G$ and is closed under
                  inverses, i.e., for all $h$ and
                  $k \in H$, $hk$ and $h^{-1} \in H$. Prove that $H$ is a group 
                  under the operation $*$ restricted to $H$ (such a subset $H$
                  is called a subgroup of $G$).

      \textbf{Proof.} We know that $H$ is closed under $*$ and under inverses, 
      so it suffices to show that $*$ is associative on $H$ and that $H$ has an 
      identity under $*$. The associativity of $H$ under $*$ follows because $H$ 
      is a subset of $G$ and $G$ is associative under $*$. Since $H$ is nonempty
      we pick an $h \in H$. Then by hypothesis, we have that
      $1 = hh^{-1} \in H$, so that $H$ contains the identity. (Note that
      $hh^{-1} = h^{-1}h = 1$ and $h1 = 1h = h$ because these equalities hold in
      $G$.) \qed
%%%%%%%%%%%%%%%%%%%%%%%%%%%%%%%%%%%%%%1.27%%%%%%%%%%%%%%%%%%%%%%%%%%%%%%%%%%%%%%
   \item[1.1.27]  Prove that if $x$ is an element of the group $G$ then
                  $\{x^n : n \in \Z\}$ is a subgroup of $G$ (called the
                  \textit{cyclic subgroup} of $G$ generated by $x$).

      \textbf{Proof.} Consider the set
      $$H = \{x^n : n \in \Z\}.$$
      $H$ is nonempty because it contains $1 = x^0$. So let $h_1, h_2 \in H$.
      Thus we have $h_1 = x^a$ and $h_2 = x^b$ for some integers $a$ and $b$, so
      that $h_1h_2 = x^ax^b = x^{a+b} \in H$; in other words, $H$ is closed
      under the operation of $G$. Since $h_1^{-1} = (x^a)^{-1} = x^{-a} \in H$, 
      it follows that $H$ is also closed under inverses, so that $H$ is a
      subgroup of $G$ by Exercise 1.1.26.
%%%%%%%%%%%%%%%%%%%%%%%%%%%%%%%%%%%%%%1.28%%%%%%%%%%%%%%%%%%%%%%%%%%%%%%%%%%%%%%
   \item[1.1.28]  Let $(A, *)$ and $(B, \diamond)$ be groups and let
                  $A \times B$ be their direct product (as defined in Example
                  6). Verify all the group axioms for $A \times B$.
                  \begin{enumerate}
                     \item prove that the associative law holds: for all
                           $(a_i, b_i) \in A \times B, i = 1, 2, 3$
                           $$(a_1, b_1)[(a_2, b_2)(a_3, b_3)] =
                            [(a_1, b_1)(a_2, b_2)](a_3, b_3),$$
                     \item prove that (1, 1) is the identity of $A \times B$,
                           and
                     \item prove that the inverse of $(a, b)$ is
                           $(a^{-1}, b^{-1})$.
                  \end{enumerate}

      \textbf{Proof.} Let $(a_1, b_1)$, $(a_2, b_2)$, and
      $(a_3, b_3) \in A \times B$.

      \begin{enumerate}
         \item The set $A \times B$ is associative under the component wise
               operations of $A$ and $B$ because
               \begin{align*}
                  (a_1, b_1)[(a_2, b_2)(a_3, b_3)]
                     &= (a_1, b_1)(a_2a_3, b_2b_3) \\
                     &= (a_1a_2a_3, b_1b_2b_3) \\
                     &= [(a_1a_2)a_3, (b_1b_2)b_3] &[\text{Associativity}] \\
                     &= (a_1a_2, b_1b_2)(a_3, b_3) \\
                     &= [(a_1, b_1)(a_2, b_2)](a_3, b_3).
               \end{align*}
         \item Consider $(1, 1) \in A \times B$. It follows that
               \begin{align*}
                  (1, 1)(a_1, b_1) &= (1a_1, 1b_1) \\
                                   &= (a_1, b_1) \\
                                   &= (a_11, b_11) \\
                                   &= (a_1, b_1)(1, 1),
               \end{align*}
               so that $(1, 1)$ is the identity of $A \times B$.
         \item Consider $(a, b) \in A \times B$. It 
               follows that
               \begin{align*}
                  (a, b)(a^{-1}, b^{-1}) &= (aa^{-1}, bb^{-1}) \\
                                   &= (1, 1) \\
                                   &= (a^{-1}a, b^{-1}b) \\
                                   &= (a^{-1}, b^{-1})(a, b),
               \end{align*}
               so that $(a^{-1}, b^{-1})$ is the inverse of $(a, b)$.
      \end{enumerate}
%%%%%%%%%%%%%%%%%%%%%%%%%%%%%%%%%%%%%%1.29%%%%%%%%%%%%%%%%%%%%%%%%%%%%%%%%%%%%%%
   \item[1.1.29]  Prove that $A \times B$ is an abelian group if and only if
                  both $A$ and $B$ are abelian.

      \textbf{Proof.} 

      $(\Leftarrow)$ Suppose that $A$ and $B$ are abelian. Let $(a_1, b_1)$ and
      $(a_2, b_2) \in A \times B$. It follows that $A \times B$ is abelian
      because
      \begin{align*}
         (a_1, b_1)(a_2, b_2) &= (a_1a_2, b_1b_2) \\
            &= (a_2a_1, b_2b_1) &[\text{$A$ and $B$ are abelian}] \\
            &= (a_2, b_2)(a_1, b_1).
      \end{align*}

      $(\Rightarrow)$ Now suppose that $A \times B$ is abelian. Let $a_1$ and
      $a_2$ be members of $A$ and let $b_1$ and $b_2$ be members of $B$. Then
      we have that
      \begin{align*}
         (a_1a_2, b_1b_2) = (a_1, b_1)(a_2, b_2) \\
            &= (a_2, b_2)(a_1, b_1) &[\text{$A \times B$ is abelian}] \\
            &= (a_2a_1, b_2b_1),
      \end{align*}
      so that $(a_1a_2, b_1b_2) = (a_2a_1, b_2b_1)$; i.e., $a_1a_2 = a_2a_1$ and
      $b_1b_2 = b_2b_1$. We can now conclude that $A$ and $B$ are both abelian.
      \qed
%%%%%%%%%%%%%%%%%%%%%%%%%%%%%%%%%%%%%%1.30%%%%%%%%%%%%%%%%%%%%%%%%%%%%%%%%%%%%%%
   \item[1.1.30]  Prove that the elements $(a, 1)$ and $(1, b)$ of $A \times B$
                  commute and deduce that the order of $(a, b)$ is the least 
                  common multiple of $|a|$ and $|b|$.

      \textbf{Proof.} Let $A$ and $B$ be groups, and let $a \in A$, $b \in B$.
      We shall be assuming that there exist positive integers $m$ and $n$ such 
      that $|a| = m$ and $|b| = n$, for the problem does not make sense if the
      order of $a$ or $b$ is not finite. Consider $(a, 1)$,
      $(1, b) \in A \times B$. We have that
      \begin{align*}
         (a, 1)(1, b) &= (a1, 1b) \\
                      &= (a, b) \\
                      &= (1a, b1) \\
                      &= (1, b)(a, 1),
      \end{align*}
      so that $(a, 1)$ and $(b, 1)$ commute. To complete the proof, we let
      $s = \text{lcm}(m, n)$. Thus we can write $s = mx = ny$ for positive 
      integers $x$ and $y$. Thus we have that
      \begin{align*}
         (a, b)^s &= (a^s, b^s) &[\text{Lemma 1.1.4}] \\
                  &= (a^{mx}, a^{ny}) \\
                  &= [(a^m)^x, (a^n)^y] \\
                  &= (1^x, 1^y) \\
                  &= (1, 1).
      \end{align*}
      This say that $|(a, b)| \le s$, so there exists a positive integer $q$ 
      such that $|(a, b)| = q$. By Lemma 1.1.4, we have that
      $(a, b)^q = (a^q, b^q) = (1, 1)$, so that $a^q = 1$ and $b^q = 1$. Thus by 
      Lemma 1.1.3, it follows that $m \mid q$ and $n \mid q$, so that $s \mid q$ 
      by definition of the lcm. Since $s \mid q$, we must have that $s \le q$.
      But we previously showed that $q \le s$. Thus we can conclude that
      $s = q$, as desired. \qed
%%%%%%%%%%%%%%%%%%%%%%%%%%%%%%%%%%%%%%1.31%%%%%%%%%%%%%%%%%%%%%%%%%%%%%%%%%%%%%%
   \item[1.1.31]  Prove that any finite group $G$ of even order contains an
                  element of order 2. [Let $t(G)$ be the set
                  $\{g \in G : g \neq g^{-1}\}$. Show that $t(G)$ has an even 
                  number of elements and every nonidentity element of $G - t(G)$ 
                  has order 2.]

      \textbf{Proof.} Let $G$ be a finite group of even order. We wish to show
      that there exists some $g \in G$ such that $|g| = 2$. Consider this subset
      of $G$:
      $$S = \{g \in G: g \neq g^{-1}\}.$$

      If $|S| = 0$, then the proof is done, so assume that $|S| > 0$. Now $|S|$ 
      is even, for if this were not the case, then if we pair up every element
      of $S$ with its inverse, then one element must be without an inverse, a 
      contradiction. Now let $S' = G\backslash S$. It follows that
      $|G| = |S| + |S'|$. Notice that $S'$ is not empty because $e \in S'$. 
      Since $G$ and $S$ are both even, it follows that $|S'|$ must also be even. 
      Since we already showed that $|S'| \ge 1$, we can conclude that
      $|S'| \ge 2$, so that $S'$ contains a non-identity $a$, where
      $a = a^{-1}$. That is, $|a| = 2$. \qed
%%%%%%%%%%%%%%%%%%%%%%%%%%%%%%%%%%%%%%1.32%%%%%%%%%%%%%%%%%%%%%%%%%%%%%%%%%%%%%%
   \item[1.1.32]  If $x$ is an element of finite order $n$ in $G$, prove that
                  the elements 1, $x$, $x^2$, $\ldots$, $x^{n-1}$ are all 
                  distinct. Deduce that $|x| \le |G|$.

      \textbf{Proof.} Suppose that $|x| = n \in \Z^+$ for some $x \in G$. 
      Suppose to the contrary that the elements $x^0$, $x^1$, $x^2$, $\ldots$, 
      $x^{n-1}$ are not distinct. Then we must have that $x^i = x^j$ for some
      integer $i$ and $j$ where $0 \le i < j \le n - 1$. That is, $x^{j-i} = 1$,
      a contradiction because $j - i$ is a positive integer less thatn $n$. It
      follows that the elements $x^0$, $x$, $x^2$, $\ldots$, $x^{n-1}$ are all 
      distinct. Since there are clearly $n$ of these elements and since they are
      all members of $G$, we can conclude that $|x| = n \le |G|$. \qed
%%%%%%%%%%%%%%%%%%%%%%%%%%%%%%%%%%%%%%1.33%%%%%%%%%%%%%%%%%%%%%%%%%%%%%%%%%%%%%%
   \item[1.1.33]  Let $x$ be an element of finite order $n$ in $G$.
                  \begin{enumerate}
                     \item Prove that if $n$ is odd then $x^i \neq x^{-i}$ for
                           all $i = 1, 2, \ldots, n - 1$,
                     \item Prove that if $n = 2k$ and $1 \le i < n$ then
                           $x^i = x^{-i}$ if and only if $i = k$.
                  \end{enumerate}

      \textbf{Proof.}

      \begin{enumerate}
         \item Suppose that $n$ is odd. Now we shall suppose to the contrary
               that $x^i = x^{-i}$ for some integer $1 \le i \le n - 1$. Since
               $x^i = x^{-i}$, it follows that $x^{2i} = 1$. By Lemma 1.1.3, we
               must have that $n \mid 2i$, a contradiction because an odd
               number cannot divide a positive even number, so we conclude that
               $x^i \neq x^{-i}$ for all $i = 1, 2, \ldots, n - 1$. \qed
         \item Suppose that $n$ is even and $1 \le i < n$. Write $n = 2k$ for
               some positive integer $k$.

               $(\Leftarrow)$ Suppose that $i = k$. Then we have that
               $1 = x^{2k} = x^{2i} = x^ix^i$, so that $x^i = x^{-i}$.

               $(\Rightarrow)$ Conversely suppose that $x^i = x^{-i}$, so that
               $x^{2i} =1$. Thus, by Lemma 1.1.3, $2k \mid 2i$, or equivalently,
               $k \mid i$, so that $i = mk$ for some positive integer $m$. 
               Recall that $i < n = 2k$ by hypothesis, so that $mk < 2k$. That 
               is $m < 2$. But $m$ is a positive integer and so the only
               possibility is therefore $m = 1$, so that $i = k$. \qed
      \end{enumerate}
%%%%%%%%%%%%%%%%%%%%%%%%%%%%%%%%%%%%%%1.34%%%%%%%%%%%%%%%%%%%%%%%%%%%%%%%%%%%%%%
   \item[1.1.34]  If $x$ is an element of infinite order in $G$, prove that the
                  elements $x^n$, $n \in \Z$ are all distinct.

      \textbf{Proof.} Assume that $x$ is an element of infinite order in $G$.
      Now suppose to the contrary that $x^i = x^j$ for some unequal integers
      $i$ and $j$. We can further assume without loss of generality that
      $i < j$. Thus $x^{j-i} = 1$, a contradiction because this says that
      $|x| \le j - i$. It follows that distinct integral powers of $x$ yield 
      distinct elements of $G$. \qed
%%%%%%%%%%%%%%%%%%%%%%%%%%%%%%%%%%%%%%1.35%%%%%%%%%%%%%%%%%%%%%%%%%%%%%%%%%%%%%%
   \item[1.1.35]  If $x$ is an element of finite order $n$ in $G$, use the 
                  Division Algorithm to show that any integral power of $x$ 
                  equals one of the elements in the set
                  $\{1, x, x^2, \ldots, x^{n-1}\}$ (so these are all the
                  distinct elements of the cyclic subgroup of $G$ generated by
                  $x$).

      \textbf{Proof.} Assume that $x$ is an element of finite order $n$ in $G$.
      Let $z \in \Z$. By the Division Algorithm, there exist unique integers
      $q$ and $r$ such that $z = qn + r$ and $0 \le r < n$. That is
      $$x^z = x^{qn+r} = x^{qn}x^r = (x^n)^qx^r = 1^qx^r = x^r.$$
      Since $r \in \{0, 1, \ldots, n - 1\}$ and since $x^z = x^r$, it follows
      that $x^z \in \{x^0, x^1, \ldots, x^{n-1}\}$. \qed
%%%%%%%%%%%%%%%%%%%%%%%%%%%%%%%%%%%%%%1.36%%%%%%%%%%%%%%%%%%%%%%%%%%%%%%%%%%%%%%
   \item[1.1.36]  Assume $G = \{1, a, b, c\}$ is a group of order 4 with
                  identity 1. Assume also that $G$ has no elements of order 4
                  (so by Exercise 32, every element has order $\le$ 3). Use the
                  cancellation laws to show that there is a unique group table
                  for $G$. Deduce that $G$ is abelian.

      \textbf{Proof.} Assume $G = \{1, a, b, c\}$. We can tentatively fill out 
      the group table for $G$ like so:      
      $$
         \begin{tabular}{@{}c | c | c | c | c@{}} 
                & $1$ & $a$ & $b$ & $c$ \\ \hline
            $1$ & $1$ & $a$ & $b$ & $c$ \\ \hline
            $a$ & $a$ & $ $ & $ $ & $ $ \\ \hline
            $b$ & $b$ & $ $ & $ $ & $ $ \\ \hline
            $c$ & $c$ & $ $ & $ $ & $ $
         \end{tabular}
      $$
      By the left cancellation law, the equality $ab = a$ will result in $b = e$
      and the equality $ab = b$ will result in $a = e$, both of which are
      contradictions. The only remaining possiblities are $ab = c$ or $ab = 1$.

      \textbf{Case 1:} $ab = c$. For the same reason as above, we cannot have
      $ac = a$ or $ac = c$, so that $ac = 1$ or $ac = b$. So suppose first that
      $ac = b$. Then our table will now look like so:
      $$
         \begin{tabular}{@{}c | c | c | c | c@{}} 
                & $1$ & $a$ & $b$ & $c$ \\ \hline
            $1$ & $1$ & $a$ & $b$ & $c$ \\ \hline
            $a$ & $a$ & $ $ & $c$ & $b$ \\ \hline
            $b$ & $b$ & $ $ & $ $ & $ $ \\ \hline
            $c$ & $c$ & $ $ & $ $ & $ $
         \end{tabular}
      $$
      From the table above, we see that $aa$ must be equal to 1, since that is
      the only remaining possibility. The cancellation laws tell us that every
      element in a column and row of a group table must be unique, so we must
      have that:
      $$
         \begin{tabular}{@{}c | c | c | c | c@{}} 
                & $1$ & $a$ & $b$ & $c$ \\ \hline
            $1$ & $1$ & $a$ & $b$ & $c$ \\ \hline
            $a$ & $a$ & $1$ & $c$ & $b$ \\ \hline
            $b$ & $b$ & $c$ & $ $ & $ $ \\ \hline
            $c$ & $c$ & $b$ & $ $ & $ $
         \end{tabular}
      $$
      Note that we cannot have $bb = a$ because that would imply that $bbb = c$,
      so that $|b| > 3$, contradicting our hypothesis. Thus we must have that
      $bb = 1$. The remaining positions are thus completely determined, so that
      we have
      $$
         \begin{tabular}{@{}c | c | c | c | c@{}} 
                & $1$ & $a$ & $b$ & $c$ \\ \hline
            $1$ & $1$ & $a$ & $b$ & $c$ \\ \hline
            $a$ & $a$ & $1$ & $c$ & $b$ \\ \hline
            $b$ & $b$ & $c$ & $1$ & $a$ \\ \hline
            $c$ & $c$ & $b$ & $a$ & $1$
         \end{tabular}
      $$
      Now suppose that $ac = 1$, then we would be forced to fill in the table
      like so:
      $$
         \begin{tabular}{@{}c | c | c | c | c@{}} 
                & $1$ & $a$ & $b$ & $c$ \\ \hline
            $1$ & $1$ & $a$ & $b$ & $c$ \\ \hline
            $a$ & $a$ & $b$ & $c$ & $1$ \\ \hline
            $b$ & $b$ & $c$ & $ $ & $ $ \\ \hline
            $c$ & $c$ & $1$ & $ $ & $ $
         \end{tabular}
      $$
      Since $a^2 = b$ and $a^3 = c$, we have that $|a| > 3$, contradicting our
      hypothesis, so this is a dead end.

      \textbf{Case 2:} $ab = 1$. For the same reason as above, we cannot have
      $ac = a$ or $ac = c$, so that $ac = 1$ or $ac = b$. So suppose first that
      $ac = b$. Then our table will now look like so:
      $$
         \begin{tabular}{@{}c | c | c | c | c@{}} 
                & $1$ & $a$ & $b$ & $c$ \\ \hline
            $1$ & $1$ & $a$ & $b$ & $c$ \\ \hline
            $a$ & $a$ & $c$ & $1$ & $b$ \\ \hline
            $b$ & $b$ & $1$ & $c$ & $a$ \\ \hline
            $c$ & $c$ & $b$ & $a$ & $1$
         \end{tabular}
      $$
      Since $a^2 = c$ and $a^3 = b$, we have that $|a| > 3$, contradicting our
      hypothesis, so this is another dead end. From our arguments above, we see
      that the only viable and legal table is thus:
      $$
         \begin{tabular}{@{}c | c | c | c | c@{}} 
                & $1$ & $a$ & $b$ & $c$ \\ \hline
            $1$ & $1$ & $a$ & $b$ & $c$ \\ \hline
            $a$ & $a$ & $1$ & $c$ & $b$ \\ \hline
            $b$ & $b$ & $c$ & $1$ & $a$ \\ \hline
            $c$ & $c$ & $b$ & $a$ & $1$
         \end{tabular}
      $$
      This table is unique, and since it is symmeteric it follows that $G$ is
      abelian. \qed
\end{enumerate}

      \section{Centralizers And Normalizers, Stabilizes And Kernels}
         \begin{enumerate}
%%%%%%%%%%%%%%%%%%%%%%%%%%%%%%%%%%%Prob1.2_1%%%%%%%%%%%%%%%%%%%%%%%%%%%%%%%%%%%%
   \item[1.2.1]   For each of the following statements, determine whether it is 
                  true or false and justify your answer.
                  \begin{enumerate}
                     \item The set $\Z$ of integers is dense in $\R$.
                     \item The set of positive real numbers is dense in $\R$.
                     \item The set $\Q\backslash \Z$ of rational numbers that 
                           are not integers is dense in $\R$.
                  \end{enumerate}  

      \textbf{Solution:} 

      \begin{enumerate}
         \item False. Proposition 1.6 states that there is no integer in the
               interval (0, 1).
         \item False. The interval $(-1, 0)$ contains no positive real number.
         \item True. Let $a$ and $b$ be real numbers. Then we shall investigate
               the following two cases:
               
               \textbf{Case I:} $a < a + 1 \le b$. Theorem 1.8 says that there
               exists a unique integer $k$ in $[a, a + 1)$. Thus there is no
               integer in the interval $(k, a + 1)$. By the density of $\Q$ in
               $\R$, there exists a rational $q \in (k, a + 1)$. Since
               $(k, a + 1)$ contains no integer, then $q$ must be a member of
               $\Q\backslash\Z$. We observe that $q \in (a, b)$.
               
               \textbf{Case II:} $a < b < a + 1$. Theorem 1.8 says that there
               exists a unique integer $k$ in $[a, a + 1)$. If $k  \le b$, then
               $(a, k)$ has no integer, so there exists a noninteger rational
               in $(a, k) \subseteq (a, b)$ by the density of $\Q$ in $\R$. If,
               however, $k > b$, then the interval $(a, b)$ contains no integer,
               so that there exists a noninteger rational in $(a, b)$ by the
               density of $\Q$ in $\R$.
      \end{enumerate}
%%%%%%%%%%%%%%%%%%%%%%%%%%%%%%%%%%Prob1.2_2%%%%%%%%%%%%%%%%%%%%%%%%%%%%%%%%%%%%%
   \item[1.2.2]   Suppose that $S$ is a nonempty set of integers that is bounded
                  below. Show that $S$ has a minimum. In particular, conclude
                  that every nonempty set of natural numbers has a minimum.  

      \textbf{Proof:}

      Let $S$ be a nonempty set of integers bounded below. Then there exists
      some $r \in \R$ such that for every $a \in S$, we have that $r \le a$.
      Consider the set $S' = \{-s: s \in S\}$, the set of the additive inverses
      of the elements of $S$. Note that $S'$ is also a nonempty set of integers.
      So let $-d \in S'$ where $d \in S$. Hence $r \le d$, so that $-d \le -r$;
      that is $S'$ is bounded above. By Proposition 1.7 $S'$ has a maximum, say
      $-b$, where $b \in S$. It suffices to show that $b$ is the minimum in $S$.
      Let $c \in S$. Then we have that $-c \le -b$, so that $b \le c$; that is,
      $b$ is the minimum element in $S$. In paritcular, we can see that the
      Well Ordering Principle follows. \qed
%%%%%%%%%%%%%%%%%%%%%%%%%%%%%%%%%%Prob1.2_3%%%%%%%%%%%%%%%%%%%%%%%%%%%%%%%%%%%%%
   \item[1.2.3]   Let $S$ be a nonempty set of real numbers that is bounded
                  below. Prove that the set $S$ has a minimum if and only if the
                  number $\inf S$ belongs to $S$.
			
		\textbf{Proof:} Let $S$ be a nonempty set of real numbers that is bounded
      below.

      $(\Leftarrow)$ Suppose $\inf S$ belongs in $S$; then it immediately
      follows by definition that $\inf S$ is the minimum element of $S$. \\
      $(\Rightarrow)$ Now suppose that $S$ has a minimum, say $s$. By the
      Completeness Axiom, we have that $\inf S$ exists; since $s \in S$, we must
      have that $\inf S \le s$. But $s$ is also a lower bound for $S$ and since
      every lower bound of $S$ cannot exceed $\inf S$, we must have that
      $s \le \sup S$; we have shown that $\inf S \le s$ and $s \le \inf S$ so
      that $s = \inf S$. \qed
%%%%%%%%%%%%%%%%%%%%%%%%%%%%%%%%%%Prob1.2_4%%%%%%%%%%%%%%%%%%%%%%%%%%%%%%%%%%%%%
   \item[1.2.4]   For each of the following two sets, find the maximum, minimum,
                  infimum, and supremum if they are defined. Justify your
                  conclusions.
                  \begin{enumerate}
                     \item $S = \{1/n : n \in \N\}$.
                     \item $T = \{x \in \R : x^2 < 2\}$.
                  \end{enumerate}

      \textbf{Solution:}

      \begin{enumerate}
         \item The \textbf{maximum} is 1. To show this consider any natural
               number $n$; then we have $n \ge 1$. Multiply this inequality by
               the positive number $1/n$ to give us $1/n \le 1$. Since
               $1 = 1/1 \in S$, we are done. $S$ has no \textbf{minimum}. Assume
               by way of contradiction that $\min S$ exists. Then by definition
               of $S$, we know that $\min S$ must be positive. So by the
               Archimedean Property, there exists a natural number $n_1$(so that
               $1/n_1 \in S$) such that $1/n_1 < \min S$, a contradiction. So
               $\min S$ doesn't exist. Since the Archimedean Property enables us
               to find a member of $S$ that is less than any positive number, no
               positive number can be a lower bound for $S$. Thus $S$ can only 
               be bounded below by negative numbers and 0. It follows that the
               \textbf{infimum} of  $S$ is 0. Since $S$ has a maximum, this
               maximum. By Problem 1.1.15, we have that the \textbf{supremum} of
               $S = 1$. If we consider $-T$, the set of the additive inverses of
               the elements of $T$.
         \item It is trivial to show that%%%%%%%%%%%%%%%%%%%%%%%%%%%%%%%%%%%%%%%%%%%%%%%%%%%%Show true
               $T = \{x \in \R: -\sqrt{2} < x < \sqrt{2}\}$. We claim that the
               \textbf{infimum} and \textbf{supremum} of $T$ are $-\sqrt{2}$ and
               $\sqrt{2}$. Suppose by contradiction that this is false; then 
               there exist $a > -\sqrt{2}$ and $b < \sqrt{2}$ such that $a$ and 
               $b$ are the infimum and supremum of $T$. Then by the density of
               $\Q$ in $\R$, there exist rationals $p$ and $q$ such that
               $-\sqrt{2} < p < a$ and $b < q < \sqrt{2}$; that is, $p$ and $q$ 
               are members of $T$. But since $p$ is less than $a$ and $q > b$, 
               we have contradictions. Thus our claim holds. Since the infimum 
               and supremum are not members of $T$, it follows that $T$ has 
               neither a \textbf{maximum} nor a \textbf{minimum}.
      \end{enumerate}
%%%%%%%%%%%%%%%%%%%%%%%%%%%%%%%%%%Prob1.2_5%%%%%%%%%%%%%%%%%%%%%%%%%%%%%%%%%%%%%
   \item[1.2.5]   Suppose that the number $a$ has the property that for every
                  natural number $n$, $a \le 1/n$. Prove that $a \le 0$.

      \textbf{Proof:} Assume by way of contradiction that $a > 0$. By The
      Archimedean Property there exists a natural number $k$ such that
      $a > 1/k$, a contradiction. Thus $a \le 0$. \qed

%%%%%%%%%%%%%%%%%%%%%%%%%%%%%%%%%%Prob1.2_6%%%%%%%%%%%%%%%%%%%%%%%%%%%%%%%%%%%%%
   \item[1.2.6]   Given a real number $a$, define
                  $S \equiv \{x : x \in \Q, x < a\}$. Prove that $a = \sup S$.

      \textbf{Proof:} By the density of $\Q$ in $\R$, there exists a rational
      $q \in (a - 1, a)$, so that $q \in S$. Thus $S$ is nonempty. By 
      definition, $S$ is bounded above by $a$; since $S$ is also nonempty, the
      Completeness Axiom says that $\sup S$ exists. So we must have that
      $\sup S \le a$. Now suppose that $\sup S < a$, then the density of $\Q$ in
      $\R$ guarantees that we have a rational $q$ in $(\sup S, a)$, so that $q$
      is also a member of $S$, a contradiction since we cannot have a member of
      $S$ that is greater than $\sup S$. Thus $a = \sup S$. \qed

%%%%%%%%%%%%%%%%%%%%%%%%%%%%%%%%%%Prob1.2_7%%%%%%%%%%%%%%%%%%%%%%%%%%%%%%%%%%%%%
   \item[1.2.7]   Show that for any real number $c$, there is exactly one 
                  integer in the interval $(c, c+1]$.

      \textbf{Proof:} Let $c$ be a real number. According to Theorem 1.8, there
      exists a unique integer $k$ in the interval $[-(c + 1), -c)$. So we have
      $-(c + 1) \le k < -c$, so that $c < -k \le c + 1$. Hence we have an
      integer $-k$ in the interval $(c, c + 1]$. We can see that $-k$ is unique
      because if another integer $h$ exists in $(c, c + 1]$, then $-h$ would
      also be in $[-(c + 1), -c)$, and since $k$ is unique, we must have
      $-h = k$, so that $h = -k$. \qed
   
%%%%%%%%%%%%%%%%%%%%%%%%%%%%%%%%%%Prob1.2_8%%%%%%%%%%%%%%%%%%%%%%%%%%%%%%%%%%%%%
   \item[1.2.8]   Show that the Archimedean Property is a consequence of the
                  assertion that for any real number $c$, there is an integer in
                  the interval $[c, c + 1)$.

      \textbf{Proof:} Let $\epsilon$ be a positive real number. It suffices to
      show that there exists a natural number greater than $\epsilon$. By our
      assertion, there exists an integer $k$ in the interval
      $[\epsilon+ 1, \epsilon + 2)$. So we have $\epsilon < \epsilon + 1 \le k$,
      so that $k$ is a positive integer. \qed
%%%%%%%%%%%%%%%%%%%%%%%%%%%%%%%%%%Prob1.2_9%%%%%%%%%%%%%%%%%%%%%%%%%%%%%%%%%%%%%
   \item[1.2.9]   Show that the Archimedean Property is a consequence of the
                  assertion that every interval $(a, b)$ contains a rational
                  number.

      \textbf{Proof:} Let $\epsilon$ be a positive real number. It suffices to
      show that there exists a natural number greater than $\epsilon$. By our
      assertion, there exist positive integers $p$ and $q$ such that
      $p/q \in (\epsilon, \epsilon + 1)$. Since $p$ and $q$ are positive, we 
      have that $q \ge 1$ so that $pq \ge p$; that is $p/q \le p$. We have now
      shown that $\varepsilon < p/q \le p$. Particularly $p > \varepsilon$,
      which is what we wanted to prove. \qed

      
\end{enumerate}

      \section{Cyclic Groups And Cyclic Subgroups}
         \begin{enumerate}
%%%%%%%%%%%%%%%%%%%%%%%%%%%%%%%%%%%%%2.3.1%%%%%%%%%%%%%%%%%%%%%%%%%%%%%%%%%%%%%%
   \item[2.3.1]   Find all subgroups of $Z_{45} = \cyc{x}$, giving a generator
                  for each. Describe the containments between these subgroups.
                  
      \textbf{Solution.} Since the positive divisors of 45 are: 1, 3, 5, 9, 15,
      and 45, it follows that the subgroups of $Z_{45}$ are
      $$\cyc{x}, \cyc{x^3}, \cyc{x^5}, \cyc{x^9}, \cyc{x^{15}}, \text{ and }
        \cyc{x^{45}}.$$
        
      We have the following containments:
      $$
         \begin{tabular}{>{$}c<{$}>{$}c<{$}>{$}c<{$}>{$}c<{$}>{$}c<{$}>{$}c<{$}>{$}c<{$}}
            \cyc{x^{45}} & \le & \cyc{x^{15}} & \le & \cyc{x^5} & \le & \cyc{x} \\
            \cyc{x^{15}} & \le &  \cyc{x^3} & \le & \cyc{x} \\
            \cyc{x^9} & \le &  \cyc{x^3} & \le & \cyc{x}
         \end{tabular}
      $$
%%%%%%%%%%%%%%%%%%%%%%%%%%%%%%%%%%%%%2.3.2%%%%%%%%%%%%%%%%%%%%%%%%%%%%%%%%%%%%%%
   \item[2.3.2]   If $x$ is an element of the finite group $G$ and $|x| = |G|$,
                  prove that $G = \cyc{x}$. Give an explicit example to show 
                  that this result need not be true if $G$ is an infinite group.
                  
      \textbf{Proof.} Let $G$ be a finite group, so that $|G| = n \in \Z^+$.
      Suppose that there exists $x \in G$ such that $|x| = n$. Clearly
      $\cyc{x} \subseteq G$. But $|\cyc{x}| = n$ since $|x| = n$; thus
      $G \subseteq \cyc{x}$ so that $G = \cyc{x}$. Now let $G = \Z$. We have
      that $|\cyc{2}| = |G|$ but $G \neq \cyc{2}$. \qed
%%%%%%%%%%%%%%%%%%%%%%%%%%%%%%%%%%%%%2.3.3%%%%%%%%%%%%%%%%%%%%%%%%%%%%%%%%%%%%%%
   \item[2.3.3]   Find all generators for $\Z/48\Z$.
   
      \textbf{Solution.} The generators for $\Z/48\Z$ are: $\cyc{\overline{1}}$,
      $\cyc{\overline{5}}$, $\cyc{\overline{7}}$, $\cyc{\overline{11}}$,
      $\cyc{\overline{13}}$, $\cyc{\overline{17}}$, $\cyc{\overline{19}}$,
      $\cyc{\overline{23}}$, $\cyc{\overline{25}}$, $\cyc{\overline{29}}$,
      $\cyc{\overline{31}}$, $\cyc{\overline{35}}$, $\cyc{\overline{37}}$,
      $\cyc{\overline{41}}$, $\cyc{\overline{43}}$, and $\cyc{\overline{47}}$.
%%%%%%%%%%%%%%%%%%%%%%%%%%%%%%%%%%%%%2.3.4%%%%%%%%%%%%%%%%%%%%%%%%%%%%%%%%%%%%%%
   \item[2.3.4]   Find all generators for $\Z/202\Z$.
   
      \textbf{Solution.} Let $S$ be the set of generators for $\Z/202\Z$. Then
      $|S| = 100$ since
      $$S = \{\cyc{x} : x \text{ is odd and positive}, x \neq 101, \text{ and } x < 202\}.$$
%%%%%%%%%%%%%%%%%%%%%%%%%%%%%%%%%%%%%2.3.5%%%%%%%%%%%%%%%%%%%%%%%%%%%%%%%%%%%%%%
   \item[2.3.5]   Find the number of generators for $\Z/49000\Z$.
   
      \textbf{Solution.} For a positive integer $n$ let $\varphi(n)$ be the
      number of positive integers---less than or equal to $n$---that are
      relatively prime to $n$. Then the number of generators for $\Z/49000\Z$ is
      $\varphi(49000) = \varphi(2^35^37^2) =
      \varphi(2^3)\varphi(5^3)\varphi(7^2) = 16800$. 
%%%%%%%%%%%%%%%%%%%%%%%%%%%%%%%%%%%%%2.3.6%%%%%%%%%%%%%%%%%%%%%%%%%%%%%%%%%%%%%%
   \item[2.3.6]   In $\Z/48\Z$ write out all elements of $\cyc{\overline{a}}$
                  for every $\overline{a}$. Find all inclusions between
                  subgroups in $\Z/48\Z$.
      
      \textbf{Solution.}
      $$
         \begin{tabular}{|c|c|} \hline
            \textbf{Generators} & \textbf{Subgroups in} $\Z/48\Z$ \\ \hline
            0 & $\{0\}$ \\ \hline
            24 & $\{0, 24\}$ \\ \hline
            16, 32 & $\{0, 16, 32\}$ \\ \hline
            12, 36 & $\{0, 12, 24, 36\}$ \\ \hline
            8, 40 & $\{0, 8, 16, 24, 32, 40\}$ \\ \hline
            6, 18, 30, 42 & $\{0, 6, 12, 18, 24, 30, 36, 42\}$ \\ \hline
            4,20,28,44 & $\{0,4,8,12,16, 20, 24, 28, 32, 36, 40, 44\}$ \\ \hline
            3, 9, 15, 21, 27, 33, 39, 45 & $\{0, 3, 6, 9, 12, 15, 18, 21, 24,
            27, 30, 33, 36, 39, 42, 45\}$ \\ \hline            
            2, 10, 14, 22, 26, 34, 38, 46 & $\{x : 0 \le x \le 46,
            x \text{ is even}\}$ \\ \hline
            \text{See Exercise } 2.3.3 & $\Z/48\Z$ \\ \hline
         \end{tabular}
      $$
%%%%%%%%%%%%%%%%%%%%%%%%%%%%%%%%%%%%%2.3.7%%%%%%%%%%%%%%%%%%%%%%%%%%%%%%%%%%%%%%
   \item[2.3.7]   Let $Z_{48} = \cyc{x}$ and use the isomorphism
                  $\Z/48\Z \cong Z_{48}$ given by $\overline{1} \mapsto x$ to
                  list all subgroups of $Z_{48}$ as computed in the preceding
                  exercise.
                  
      \textbf{Solution.}
      $$
         \begin{tabular}{|c|} \hline
            \textbf{Subgroups in} $Z_{48}$ \\ \hline
            $\{1\}$ \\ \hline
            $\{1, x^{24}\}$ \\ \hline
            $\{1, x^{16}, x^{32}\}$ \\ \hline
            $\{1, x^{12}, x^{24}, x^{36}\}$ \\ \hline
            $\{1, x^8, x^{16}, x^{24}, x^{32}, x^{40}\}$ \\ \hline
            $\{1, x^6, x^{12}, x^{18}, x^{24}, x^{30},x^{36},x^{42}\}$ \\ \hline
            $\{1,x^4,x^8,x^{12},x^{16}, x^{20}, x^{24}, x^{28}, x^{32}, x^{36},
               x^{40}, x^{44}\}$ \\ \hline
            $\{1, x^3, x^6, x^9, x^{12}, x^{15}, x^{18}, x^{21}, x^{24},
            x^{27}, x^{30}, x^{33}, x^{36}, x^{39}, x^{42}, x^{45}\}$ \\ \hline
            $\{x^y : 0 \le y \le 46, y \text{ is even}\}$ \\ \hline
            $Z_{48}$ \\ \hline
         \end{tabular}
      $$
%%%%%%%%%%%%%%%%%%%%%%%%%%%%%%%%%%%%%2.3.8%%%%%%%%%%%%%%%%%%%%%%%%%%%%%%%%%%%%%%
   \item[2.3.8]   Let $Z_{48} = \cyc{x}$. For which integers $a$ does the map
                  $\varphi_a$ defined by $\varphi_a : \overline{1} \mapsto x^a$
                  extend to an \textit{isomorphism} from $\Z/48\Z$ onto
                  $Z_{48}$.
                  
      \textbf{Solution.} Suppose that $(a, 48) = 1$. Then it follows that $x^a$
      generates $Z_{48}$. Thus $\varphi_a$ is an isomorphism by Theorem 4 (Page
      56). Now suppose that $a$ is not relatively prime to 48. Then $x^a$ does
      not generate $Z_{48}$, so that the image of $\varphi_a$ is not $Z_{48}$.
      Hence $\varphi_a$ is an isomorphism if and only if $(a, 48) = 1$.
%%%%%%%%%%%%%%%%%%%%%%%%%%%%%%%%%%%%%2.3.9%%%%%%%%%%%%%%%%%%%%%%%%%%%%%%%%%%%%%%
   \item[2.3.9]   Let $Z_{36} = \cyc{x}$. For which integers $a$ does the map
                  $\psi_a$ defined by $\psi_a : \overline{1} \mapsto x^a$ extend
                  to a \textit{well defined homomorphism} from $\Z/48\Z$ into
                  $Z_{36}$. Can $\psi_a$ ever be a surjective homomorphism?
                  
      \textbf{Solution.} First we shall find the restriction(s) on $a$ such that
      $\psi_a$ is well defined. Suppose $b = c$ for some $b, c \in \Z/48\Z$. It
      suffices to show that $\psi_a(b) = \psi_a(c)$. Since $b = c$, there exists
      an integer $k$ such that $b = c + 48k$. Thus $\psi_a(b) = \psi_a(c+48k)$,
      so that
      $\psi_a(b)=(x^a)^{c+48k}=x^{ac + 48ak}= x^{ac}x^{48ak}=\psi_a(c)x^{12ak}$.
      So we must require $x^{12ak} = 1$ for all $k \in \Z$. Now $x^{12ak} = 1$
      for all $k \in \Z$ if and only if $3 \mid a$ if and only if $\psi_a$ is
      well defined. It follows immediately that
      $\psi_a$ is an homomorphism since
      \begin{align*}
         \psi_a(p + q) &= (x^a)^{p+q} \\
            &= x^{ap+aq} \\
            &= x^{ap}x^{aq} \\
            &= (x^a)^p(x^a)^q \\
            &= \psi_a(p)\psi_a(q)
      \end{align*}      
      for all $p, q \in \Z/48\Z$.
      
      \textit{Can $\psi_a$ ever be a surjective homomorphism?} No!
      
      \textbf{Proof.} Suppose to the contrary that $\psi_a$ is surjective. Then
      there exists $y \in \Z/48\Z$ such that $\psi_a(y) = x$. That is
      $x^{ay} = x$, so that $x^{ay-1} = 1$; thus $ay - 1 = 36m$ for some integer
      $m$. Rearrange the equality $ay - 1 = 36m$ to get $1 = ay - 36m$. Recall
      that $3 \mid a$; since $3$ also divides 36, it follows that 3 must divide
      1, a contradiction. Thus $\psi_a$ can never be surjective. \qed
%%%%%%%%%%%%%%%%%%%%%%%%%%%%%%%%%%%%%2.3.10%%%%%%%%%%%%%%%%%%%%%%%%%%%%%%%%%%%%%
   \item[2.3.10]  What is the order of $\overline{30}$ in $\Z/54\Z$? Write out
                  all the elements and their orders in $\cyc{\overline{30}}$.
                  
      \textbf{Solution.} The order of $30$ in $\Z/54\Z$ is
      $$\frac{54}{(30, 54)} = 9.$$
      The elements of $\cyc{30}$ and their respective orders are:
      $$
         \begin{tabular}{|c|c|} \hline
            Element of $\cyc{30}$ & Order \\ \hline
            30 & 9 \\ \hline
             6 & 9 \\ \hline
            36 & 3 \\ \hline
            12 & 9 \\ \hline
            42 & 9 \\ \hline
            18 & 3 \\ \hline
            48 & 9 \\ \hline
            24 & 9 \\ \hline
             0 & 1 \\ \hline
         \end{tabular}
      $$
%%%%%%%%%%%%%%%%%%%%%%%%%%%%%%%%%%%%%2.3.11%%%%%%%%%%%%%%%%%%%%%%%%%%%%%%%%%%%%%
   \item[2.3.11]  Find all cyclic subgroups of $D_8$. Find a proper subgroup of
                  $D_8$ which is not cyclic.
                  
      \textbf{Solution.} In $D_8$, only $r$ and $r^4$ have order 4. Thus
      $\{1, r, r^2, r^3\}$ is the only cyclic subgroup of order 4. The trivial
      subgroup is the only cyclic subgroup of order 1. Finally there are 5
      cyclic subgroups of order 2 and they are of the form $\{1, x\}$ where
      $x \in \{r^2, s, sr, sr^2, sr^3\}$. The set $\{1, s, r^2, sr^2\}$ is a
      non-cyclic proper subgroup of $D_8$.
%%%%%%%%%%%%%%%%%%%%%%%%%%%%%%%%%%%%%2.3.12%%%%%%%%%%%%%%%%%%%%%%%%%%%%%%%%%%%%%
   \item[2.3.12]  Prove that the following groups are \textit{not} cyclic:
                  \begin{enumerate}
                     \item $Z_2 \times Z_2$
                     \item $Z_2 \times \Z$
                     \item $\Z \times \Z$.
                  \end{enumerate}
      
      \textbf{Proof.}
      \begin{enumerate}
         \item The order of $Z_2 \times Z_2$ is 4, but no element in this group
               has order 4; thus $Z_2 \times Z_2$ is not cyclic.
         \item Let $Z_2 = \cyc{x}$. Observe that $Z_2 \times \Z$ is not finite,
               so in order for it to be cyclic it must be isomorphic to $\Z$.
               But this is not the case since $Z_2 \times \Z$ has two elements
               of finite order(namely $(1, 0)$ and $(x, 0)$) while $\Z$ has
               exactly 1 element of finite order.
         \item Suppose to the contrary that $\Z \times \Z$ is cyclic. Then there
               exist nonzero integers $a$ and $b$ such that
               $$\Z \times \Z = \cyc{(a,b)} = \{(na, nb) : n \in \Z\}.$$
               Thus there exists an integer $m$ such that
               $(ma, mb) = (0, 1)$. That is, $ma = 0$ and $mb = 1$. Since
               $ma = 0$, we must have $m = 0$ or $a = 0$. If $m$ is 0, then
               $(ma, mb) = (0, 0) \neq (0, 1)$, a contradiction; thus we must
               have $a = 0$, contradicting our assumption that $a$ is nonzero.
               Thus $\Z \times \Z$ is not cyclic.
      \end{enumerate} \qed
%%%%%%%%%%%%%%%%%%%%%%%%%%%%%%%%%%%%%2.3.13%%%%%%%%%%%%%%%%%%%%%%%%%%%%%%%%%%%%%
   \item[2.3.13]  Prove that the following pairs of groups are \textit{not}
                  isomorphic:
                  \begin{enumerate}
                     \item $\Z \times Z_2$ and $\Z$
                     \item $\Q \times Z_2$ and $\Q$.
                  \end{enumerate}
      
      \textbf{Proof.}
      \begin{enumerate}
         \item By Exercise 1.6.11, we know that $\Z \times Z_2$ is isomorphic to
               $Z_2 \times \Z$. By Exercise 2.3.12, $Z_2 \times \Z$ is not
               cyclic; thus $\Z \times Z_2$ is not cyclic. That is,
               $\Z \times Z_2$ is not isomorphic to $\Z$.
         \item Let $Z_2 = \cyc{x}$. It immediately follows that
               $\Q \times Z_2$ and $\Q$ are not isomorphic since $\Q \times Z_2$
               has two elements of finite order(namely $(0, 1)$ and $(0, x)$)
               while $\Q$ has exactly 1 element of finite order.
      \end{enumerate} \qed
%%%%%%%%%%%%%%%%%%%%%%%%%%%%%%%%%%%%%2.3.14%%%%%%%%%%%%%%%%%%%%%%%%%%%%%%%%%%%%%
   \item[2.3.14]  Let $\sigma =$ (1 2 3 4 5 6 7 8 9 10 11 12). For each of the
                  following integers $a$ compute $\sigma^a$:
                  $$a = 13, 65, 626, 1195, -6, -81, -570,\text{ and } {-1211}.$$
                  
      \textbf{Solution.}
      
      \begin{alignat*}{4}
         &\sigma^{13}   &&= \sigma &&\text{ } \\
         &\sigma^{65}   &&= \sigma^5 &&=
            (1\;6\;11\;4\;9\;2\;7\;12\;5\;10\;3\;8) \\
         &\sigma^{626}  &&= \sigma^2 &&= (1\;3\;5\;7\;9\;11) \\
         &\sigma^{1195} &&= \sigma^7 &&=
            (1\;8\;3\;10\;5\;12\;7\;2\;9\;4\;11\;6\;13) \\
         &\sigma^{-6} &&= \sigma^6 &&= (1\;7)
            (1\;8\;3\;10\;5\;12\;7\;2\;9\;4\;11\;6\;13) \\
         &\sigma^{-81} &&= \sigma^3 &&= (1\;4\;7\;10) \\
         &\sigma^{-570} &&= \sigma^6 &&= (1\;7) \\
         &\sigma^{-1211} &&= \sigma
      \end{alignat*}
%%%%%%%%%%%%%%%%%%%%%%%%%%%%%%%%%%%%%2.3.15%%%%%%%%%%%%%%%%%%%%%%%%%%%%%%%%%%%%%
   \item[2.3.15]  Prove that $\Q \times \Q$ is not cyclic.
   
      \textbf{Proof.} Since $\Q$ is infinite and, by Exercise 1.6.6, $\Q$ is not
      isomorphic to $\Z$, it follows that $\Q$ is not cyclic. We know that the
      subgroup of every cyclic group is cyclic; since $\Q \times\{1\} \cong \Q$,
      it follows that $\Q \times \{1\}$ is not cyclic; thus $\Q \times \Q$ is
      not cyclic because it has a noncyclic subgroup, namely $\Q \times \{1\}$.
      \qed
%%%%%%%%%%%%%%%%%%%%%%%%%%%%%%%%%%%%%2.3.16%%%%%%%%%%%%%%%%%%%%%%%%%%%%%%%%%%%%%
   \item[2.3.16]  Assume $|x| = n$ and $|y| = m$. Suppose that $x$ and $y$
                  \textit{commute}: $xy = yx$. Prove that $|xy|$ divides the
                  least common multiple of $m$ and $n$. Need this be true if $x$
                  and $y$ do \textit{not} commute? Give an example of commuting
                  elements $x$, $y$ such that the order of $xy$ is not equal to
                  the least common multiple of $|x|$ and $|y|$.
                  
      \textbf{Proof.} Let $l = \text{lcm}(m, n)$. So there exist integers
      $m'$ and $n'$ such that $mm' = nn' = l$. So we have that
      $$(xy)^l = x^ly^l = x^{nn'}y^{mm'} = (x^n)^{n'}(y^m)^{m'} = 1.$$
      That is $|xy|$ divides $l$ (by Proposition 3, Page 55).
      
      \textit{Need this be true if $x$ and $y$ do not commute?} No! Let
      $$
         A = \left(\begin{tabular}{@{}cc@{}}
            0 & 1/2 \\
            2 & 0
         \end{tabular}\right) \text{ and }
         B = \left(\begin{tabular}{@{}cc@{}}
            0 & 1 \\
            1 & 0
         \end{tabular}\right).
      $$
      A simple computation will show us that although $|A| = |B| = 2$, we have
      that $|AB| = \infty$.
      
      \textbf{Example.} Consider $\Z/2\Z = \{0, 1\}$. Let $x = y = 1$. Then we
      have $|x| = |y| = 2$, so that lcm($|x|, |y|) = 2 \neq |x + y| = |0| = 1$.
      \qed
%%%%%%%%%%%%%%%%%%%%%%%%%%%%%%%%%%%%%2.3.17%%%%%%%%%%%%%%%%%%%%%%%%%%%%%%%%%%%%%
   \item[2.3.17]  Find a presentation for $Z_n$ with one generator.
   
      \textbf{Solution.} $Z_n = \cyc{x : x^n = 1}$.
%%%%%%%%%%%%%%%%%%%%%%%%%%%%%%%%%%%%%2.3.18%%%%%%%%%%%%%%%%%%%%%%%%%%%%%%%%%%%%%
   \item[2.3.18]  Show that if $H$ is any group and $h$ is an element of $H$
                  with $h^n = 1$, then there is a unique homomorphism from
                  $Z_n = \cyc{x}$ to $H$ such that $x \mapsto h$.
                  
      \textbf{Proof.} Let $n \in \Z^+$, $Z_n = \cyc{x}$, $H$ a group, and
      $h^n  = 1$ for some $h \in H$. First we shall show the existence of a
      homomorphism from $Z_n$ to $H$ such that $x \mapsto h$. So consider the
      map $\alpha : \cyc{x} \rightarrow H$ defined by $\alpha(x^a) = h^a$.
      Clearly $\alpha(x) = h$. Now we will show that $\alpha$ is well defined.
      Suppose $x^w = x^y$ for some $x^w, x^y \in Z_n$. Thus $w = y + nk$ for
      some integer $k$. Thus
      $$\alpha(x^w) = \alpha(x^{y+nk})=h^{y+nk}=h^{y}{h^n}^k =h^y=\alpha(x^y),$$
      so that $\alpha$ is well defined. Now we have that
      $$\alpha(x^px^q)=\alpha(x^{p+q})=h^{p+q}=h^ph^q=\alpha(x^p)\alpha(x^q),$$
      so that $\alpha$ is an homomorphism. Now to show uniqueness, we suppose
      that $\phi : \cyc{x} \rightarrow H$ is an homommorphism such that
      $\phi(x) = h$. Since $\phi$ is a homomorphism, it follows that
      $\phi(x^a) = h^a$. Thus $\phi = \alpha$, as desired. \qed
%%%%%%%%%%%%%%%%%%%%%%%%%%%%%%%%%%%%%2.3.19%%%%%%%%%%%%%%%%%%%%%%%%%%%%%%%%%%%%%
   \item[2.3.19]  Show that if $H$ is any group and $h$ is an element of $H$,
                  then there is a unique homomorphism from $\Z$ to $H$ such that
                  $1 \mapsto h$.
                  
      \textbf{Proof.} Let $H$ be a group and let $h \in H$. First we shall show
      that there exists a homomorphism from $\Z$ to $H$ such that $1 \mapsto h$.
      So consider the map $\alpha : \Z \rightarrow H$ defined by
      $n \mapsto h^n$. Clearly $\alpha(1) = h$ and
      $$\alpha(x+y) = h^{x+y} = h^xh^y = \alpha(x)\alpha(y) \text{ for all }
        x, y \in \Z^+,$$
      so that $\alpha$ is a homomorphism. To show uniqueness, suppose that
      $\alpha' : \Z \rightarrow H$ is an homomorphism such that
      $\alpha'(1) = h$. Then according to Exercise 1.6.1, we have that
      $\alpha'(n) = \alpha'(n\cdot1) = \alpha'(1)^n = h^n$ for all $n \in \Z$;
      that is, $\alpha' = \alpha$, as desired. \qed
%%%%%%%%%%%%%%%%%%%%%%%%%%%%%%%%%%%%%2.3.20%%%%%%%%%%%%%%%%%%%%%%%%%%%%%%%%%%%%%
   \item[2.3.20]  Let $p$ be a prime and let $n$ be a positive integer. Show
                  that if $x$ is an element of the group $G$ such that
                  $x^{p^n} = 1$ then $|x| = p^m$ for some $m \le n$.
                  
      \textbf{Proof.} Suppose that $x \in G$ such that $x^{p^n} = 1$. Then it
      follows by Proposition 3 (Page 55) that $|x|$ divides $p^n$. Since $p$ is
      a prime, its factors are $p^i$, $0 \le i \le n$. Thus $|x| = p^m$ for
      some nonnegative $m$ not greater than $n$. \qed
%%%%%%%%%%%%%%%%%%%%%%%%%%%%%%%%%%%%%2.3.21%%%%%%%%%%%%%%%%%%%%%%%%%%%%%%%%%%%%%
   \item[2.3.21]  Let $p$ be an odd prime and let $n$ be a positive integer
                  $\ge 2$. Use the Binomial Theorem to show that
                  $(1+p)^{p^{n-1}} \equiv 1$ (mod $p^n$) but
                  $(1+p)^{p^{n-2}} \not\equiv 1$ (mod $p^n$). Deduce that $1+p$
                  is an element of order $p^{n-1}$ in the multiplicative group
                  $(\Z/p^n\Z)^\times$.

      \textbf{Lemma 2.3.1.} \textit{For an integer $n \ge 2$ and an odd prime
      $p$, let $f_p(n)$ be the number of $p$ factors of $n!$ (i.e., the greatest
      nonnegative integer $j$ such that $p^j \mid i!$), then it follows that
      $f_p(n) < \D\frac{n}{2}$}.

      \textbf{Proof.} Let $n \ge 2$ be an integer and $p$ an odd prime. For a
      a positive integer $r$, let $g_p(n, r)$ be the number of positive
      integers, less than or equal to $n$, that have at least $r$ number of $p$ 
      factors. It follows that $g_p(n, r) = \D\gint{\frac{n}{p^r}}$, where
      $\gint{x}$ is the greatest integer less than or equal to $x$. Finally let
      $k_n$ be the maximum nonnegative integer such that $p^{k_n}$ is a multiple
      of some positive integer not greater than $n$. Thus we have that
      \begin{align*}
         f_p(n) &= g_p(n, 1) + g_p(n, 2) + \cdots + g_p(n, k_n) \\
            &= \sum_{i=1}^{k_n} g_p(n, i)
            = \sum_{i=1}^{k_n} \gint{\frac{n}{p^i}} \\
            &\le \sum_{i=1}^{k_n} \frac{n}{p^i}
            < \sum_{i=1}^\infty \frac{n}{p^i} \\
            &= \frac{n}{p-1} &[\text{Sum of Geometric Series}] \\
            &< \frac{n}{2}. &[\text{Since }p \ge 3]
      \end{align*}

      So we can write $n! = p^{f_p(n)} h_n$ for some $h_n \in \Z^+$, so that
      $(h_n, p) = 1$.

      Now we are ready to commence the proof of the problem. By the Binomial
      Theorem, we have that
      \begin{align*}
         (1+p)^{p^{n-1}} &= \sum_{i=0}^{p^{n-1}}\binom{p^{n-1}}{i}p^i \\
            &= \sum_{i=0}^{p^{n-1}}p^i\frac{p^{n-1}(p^{n-1}-1)(p^{n-1}-2)
               \cdots(p^{n-1}-i+1)}{i!} \\
            &= \sum_{i=0}^{p^{n-1}}p^i\frac{p^{n-1}(p^{n-1}-1)(p^{n-1}-2)
               \cdots(p^{n-1}-i+1)}{p^{f_p(i)} h_i} \\
            &= 1 + p^n + p^n\sum_{i=2}^{p^{n-1}}\frac{p^{i-1}(p^{n-1}-1)
               (p^{n-1}-2) \cdots(p^{n-1}-i+1)}{p^{f_p(i)} h_i}.
      \end{align*}
      Now $f_p(i) < i / 2 \le i - 1$ for $i \ge 2$. Thus $i - 1 - f_p(i) \ge 0$
      (so that $p^{i - 1 - f_p(i)}$ is an integer) if $i \ge 2$. We then have
      \begin{equation} \label{2_3_21_1}
         (1+p)^{p^{n-1}} = 1 + p^n + p^n\sum_{i=2}^{p^{n-1}}\frac{p^{i-1-f_p(i)}
        (p^{n-1}-1)(p^{n-1}-2) \cdots(p^{n-1}-i+1)}{h_i}
      \end{equation}
      Since $(h_i, p) = 1$, it follows that $h_i$ must divide
      $p^{i-1}(p^{n-1}-1)(p^{n-1}-2) \cdots(p^{n-1}-i+1)$. Hence
      $$\sum_{i=2}^{p^{n-1}}\frac{p^{i-1-f_p(i)}
        (p^{n-1}-1)(p^{n-1}-2) \cdots(p^{n-1}-i+1)}{h_i}$$
      is an integer and we can conclude from \eqref{2_3_21_1} that
      $(1+p)^{p^{n-1}} \equiv 1$ (mod $p^n$). Now we have that
      \begin{align*}
         (1+p)^{p^{n-2}} &= \sum_{i=0}^{p^{n-2}}\binom{p^{n-2}}{i}p^i \\
            &= \sum_{i=0}^{p^{n-2}}p^i\frac{p^{n-2}(p^{n-2}-1)(p^{n-2}-2)
               \cdots(p^{n-2}-i+1)}{i!} \\
            &= 1 + p^{n-1} + p^n\frac{p^{n-2}-1}{2} + p^n\frac{p(p^{n-2}-1)(p^{n-2}-2)}{3!} +\sum_{i=4}^{p^{n-1}}p^i\frac{p^{n-2}(p^{n-2}-1)(p^{n-2}-2)
               \cdots(p^{n-2}-i+1)}{p^{f_p(i)} h_i} \\
            &= 1 + p^n + p^n\sum_{i=2}^{p^{n-1}}\frac{p^{i-1}(p^{n-1}-1)
               (p^{n-1}-2) \cdots(p^{n-1}-i+1)}{p^{f_p(i)} h_i}.
      \end{align*}
      
%%%%%%%%%%%%%%%%%%%%%%%%%%%%%%%%%%%%%2.3.22%%%%%%%%%%%%%%%%%%%%%%%%%%%%%%%%%%%%%
   \item[2.3.22]  Let $n$ be an integer $\ge 3$. Use the Binomial Theorem to
                  show that $(1+2^2)^{2^{n-2}} \equiv 1$ (mod $2^n$) but
                  $(1+2^2)^{2^{n-3}} \not\equiv 1$ (mod $2^n$). Deduce that 5 is
                  an element of order $2^{n-2}$ in the multiplicative group
                  $(\Z/2^n\Z)^\times$.

      \textbf{Proof.}
%%%%%%%%%%%%%%%%%%%%%%%%%%%%%%%%%%%%%2.3.23%%%%%%%%%%%%%%%%%%%%%%%%%%%%%%%%%%%%%
   \item[2.3.23]  Show that $(\Z/2^n\Z)^\times$ is not cyclic for any $n \ge 3$.
                  [Find two distinct subgroups of order 2.]
%%%%%%%%%%%%%%%%%%%%%%%%%%%%%%%%%%%%%2.3.24%%%%%%%%%%%%%%%%%%%%%%%%%%%%%%%%%%%%%
   \item[2.3.24]  Let $G$ be a finite group and let $x \in G$.
                  \begin{enumerate}
                     \item Prove that if $g \in N_G(\cyc{x})$ then
                           $gxg^{-1} = x^a$ for some $a \in \Z$. 
                     \item Prove conversely that if $gxg^{-1} = x^a$ for some
                           $a \in \Z$ then $g \in N_G(\cyc{x})$. [Show first
                           that $gx^kg^{-1} = (gxg^{-1})^k = x^{ak}$ for any
                           integer $k$, so that $g\cyc{x}g^{-1} \le \cyc{x}$.
                           If $x$ has order $n$, show the elements $gx^ig^{-1}$,
                           $i = 0, 1, \ldots, n-1$ are distinct, so that
                           $|g\cyc{x}g^{-1}| = |\cyc{x}| = n$ and conclude that
                           $g\cyc{x}g^{-1} = \cyc{x}$.]
                  \end{enumerate}
                  Note that this cuts down some of the work in computing
                  normalizers of cyclic subgroups since one does not have to
                  check $ghg^{-1} \in \cyc{x}$ for every $h \in \cyc{x}$.
%%%%%%%%%%%%%%%%%%%%%%%%%%%%%%%%%%%%%2.3.25%%%%%%%%%%%%%%%%%%%%%%%%%%%%%%%%%%%%%
   \item[2.3.25]  Let $G$ be a cyclic group of order $n$ and let $k$ be an
                  integer relatively prime to $n$. Prove that the map
                  $x \mapsto x^k$ is surjective. Use Lagrange's Theorem
                  (Exercise 1.7.19) to prove the same is true for any finite
                  group of order $n$. (For such $k$ each element has a
                  $k^{\text{th}}$ root in $G$. It follows from Cauchy's Theorem
                  in Section 3.2 that if $k$ is not relatively prime to the
                  order of $G$ then the map $x \mapsto x^k$ is not surjective.)
%%%%%%%%%%%%%%%%%%%%%%%%%%%%%%%%%%%%%2.3.26%%%%%%%%%%%%%%%%%%%%%%%%%%%%%%%%%%%%%
   \item[2.3.26]  Let $Z_n$ be a cyclic group of order $n$ and for each integer
                  $a$ let
                  $$\sigma_a : Z_n \mapsto Z_n \qquad by \qquad \sigma_a(x) =
                  x^a \quad \text{for all } x \in Z_n.$$
                  \begin{enumerate}
                     \item Prove that $\sigma_a$ is an automorphism of $Z_n$ if
                           and only if $a$ and $n$ are relatively prime(
                           automorphisms were introduced in Exercise 1.6.20).
                     \item Prove that $\sigma_a = \sigma_b$ if and only if
                           $a \equiv b$ (mod $n$).
                     \item Prove that \textit{every} automorphism of $Z_n$ is
                           equal to $\sigma_a$ for some integer $a$.
                     \item Prove that $\sigma_a\circ\sigma_b=\sigma_{ab}$.
                           Deduce that the map $\overline{a} \mapsto \sigma_a$
                           is an isomorphism of $(\Z/n\Z)^\times$ onto the
                           automorphism group of $Z_n$ (so Aut($Z_n$) is an
                           abelian group of order $\varphi(n)$).
                  \end{enumerate}
                  %%%%%MISSING CONTAINMENT%%%%%%%%
\end{enumerate}


































      \section{Subgroups Generated By Subsets Of A Group}
         Let $F$ be a field and let $n \in \Z^+$.
\begin{enumerate}
%%%%%%%%%%%%%%%%%%%%%%%%%%%%%%%%%%%%%1.4.1%%%%%%%%%%%%%%%%%%%%%%%%%%%%%%%%%%%%%%
   \item[1.4.1]   Prove that $|GL_2(\F_2)| = 6$.
%%%%%%%%%%%%%%%%%%%%%%%%%%%%%%%%%%%%%1.4.2%%%%%%%%%%%%%%%%%%%%%%%%%%%%%%%%%%%%%%
   \item[1.4.2]   Write out all the elements of $GL_2(\F_2)$ and compute the
                  order of each element.
%%%%%%%%%%%%%%%%%%%%%%%%%%%%%%%%%%%%%1.4.3%%%%%%%%%%%%%%%%%%%%%%%%%%%%%%%%%%%%%%
   \item[1.4.3]   Show that $GL_2(\F_2)$ is non-abelian.
%%%%%%%%%%%%%%%%%%%%%%%%%%%%%%%%%%%%%1.4.4%%%%%%%%%%%%%%%%%%%%%%%%%%%%%%%%%%%%%%
   \item[1.4.4]   Show that if $n$ is not prime then $\Z/n\Z$ is not a field.
%%%%%%%%%%%%%%%%%%%%%%%%%%%%%%%%%%%%%1.4.5%%%%%%%%%%%%%%%%%%%%%%%%%%%%%%%%%%%%%%
   \item[1.4.5]   Show that $GL_n(F)$ is a finite group if and only if $F$ has a
                  finite number of elements.
%%%%%%%%%%%%%%%%%%%%%%%%%%%%%%%%%%%%%1.4.6%%%%%%%%%%%%%%%%%%%%%%%%%%%%%%%%%%%%%%
   \item[1.4.6]   If $|F| = q$ is finite prove that $|GL_n(F)| < q^{n^2}$.
%%%%%%%%%%%%%%%%%%%%%%%%%%%%%%%%%%%%%1.4.7%%%%%%%%%%%%%%%%%%%%%%%%%%%%%%%%%%%%%%
   \item[1.4.7]   Let $p$ be a prime. Prove that the order of $GL_2(\F_p)$ is
                  $p^4 - p^3 - p^2 + p$.
%%%%%%%%%%%%%%%%%%%%%%%%%%%%%%%%%%%%%1.4.8%%%%%%%%%%%%%%%%%%%%%%%%%%%%%%%%%%%%%%
   \item[1.4.8]   Show that $GL_n(F)$ is non-abelian for any $n \ge 2$ and any
                  $F$.
%%%%%%%%%%%%%%%%%%%%%%%%%%%%%%%%%%%%%1.4.9%%%%%%%%%%%%%%%%%%%%%%%%%%%%%%%%%%%%%%
   \item[1.4.9]   Prove that the binary operation of matrix multiplication of
                  $2 \times 2$ matrices with real number entries is associative.
%%%%%%%%%%%%%%%%%%%%%%%%%%%%%%%%%%%%%1.4.10%%%%%%%%%%%%%%%%%%%%%%%%%%%%%%%%%%%%%
   \item[1.4.10]  Let $\left\{\left(\begin{tabular}{@{}cc@{}}
                     $a$ & $b$ \\
                      0  & $c$
                  \end{tabular}\right) : a, b, c \in \R, a \neq 0, c \neq 0
                  \right\}$.

                  \begin{enumerate}
                     \item Compute the product of
                           $\left(\begin{tabular}{@{}cc@{}}
                              $a_1$ & $b_1$ \\
                              0  & $c_1$
                           \end{tabular}\right)$ and
                           $\left(\begin{tabular}{@{}cc@{}}
                              $a_2$ & $b_2$ \\
                              0  & $c_2$
                           \end{tabular}\right)$ to show that $G$ is closed under
                           matrix multiplication.
                     \item Find the matrix inverse of
                           $\left(\begin{tabular}{@{}cc@{}}
                              $a$ & $b$ \\
                              0  & $c$
                           \end{tabular}\right)$ and deduce that $G$ is closed 
                           under inverses.
                     \item Deduce that $G$ is a subgroup of $GL_2(\R)$.
                     \item Prove that the set of elements of $G$ whose two
                           diagonal entries are equal is also a subgroup of
                           $GL_2(\R)$.
                  \end{enumerate}
\end{enumerate}

The next exercise introduces the \textit{Heisenberg group} over the field $F$
and develops some of its basic properties. When $F = \R$ this groups plays an
important role in quantum mechanics and signal theory by giving a group
theoretic interpretation (due to H. Weyl) of Heisenberg's Uncertainty Principle.
Note also that the Heisenberg group may be defined more generally---for example,
with entries in $\Z$.

\begin{enumerate}
%%%%%%%%%%%%%%%%%%%%%%%%%%%%%%%%%%%%%1.4.11%%%%%%%%%%%%%%%%%%%%%%%%%%%%%%%%%%%%%
   \item[1.4.11]  Let $H(F) = \left\{\left(\begin{tabular}{@{}ccc@{}}
                     1 & $a$ & $b$ \\
                     0 & 1 & $c$ \\
                     0 & 0 & 1
                  \end{tabular}\right) : a, b, c \in F\right\}$---called the
                  \textit{Heisenberg group} over $F$. Let
                  $X = \left(\begin{tabular}{@{}ccc@{}}
                     1 & $a$ & $b$ \\
                     0 & 1 & $c$ \\
                     0 & 0 & 1
                  \end{tabular}\right)$ and $Y =\left(\begin{tabular}{@{}ccc@{}}
                     1 & $d$ & $e$ \\
                     0 & 1 & $f$ \\
                     0 & 0 & 1
                  \end{tabular}\right)$ be elements of $H(F)$.

                  \begin{enumerate}
                     \item Compute the matrix product $XY$ and deduce that
                           $H(F)$ is closed under matrix multiplication. Exhibit
                           explicit matrices such that $XY \neq YX$ (so that
                           $H(F)$ is always non-abelian).
                     \item Find an explicit formula for the matrix inverse
                           $X^{-1}$ and deduce that $H(F)$ is closed under
                           inverses.
                     \item Prove the associative law for $H(F)$ and deduce that
                           $H(F)$ is a group of order $|F|^3$. 

                           (Do not assume that matrix multiplication is 
                           associative).
                     \item Find the order of each element of the finite group
                           $H(\Z/2\Z)$.
                     \item Prove that every nonidentity element of the group
                           $H(\R)$ has infinite order.
                  \end{enumerate}
\end{enumerate}

      \section{The Lattice Of Subgroups Of A Subgroup}
         \begin{enumerate}
%%%%%%%%%%%%%%%%%%%%%%%%%%%%%%%%%%%%%2.5.1%%%%%%%%%%%%%%%%%%%%%%%%%%%%%%%%%%%%%%
   \item[2.5.1]   Let $H$ and $K$ be subgroups of $G$. Exhibit all possible
                  sublattices which show only $G$, 1, $H$, $K$, and their joins
                  and intersections. What distinguishes the different drawings?
%%%%%%%%%%%%%%%%%%%%%%%%%%%%%%%%%%%%%2.5.2%%%%%%%%%%%%%%%%%%%%%%%%%%%%%%%%%%%%%%
   \item[2.5.2]   In each of (a) to (d) list all subgroups of $D_{16}$ that
                  satisfy the given condition.
                  \begin{enumerate}
                     \item Subgroups that are contained in $\cyc{sr^2, r^4}$
                     \item Subgroups that are contained in $\cyc{sr^7, r^4}$
                     \item Subgroups that contain $\cyc{r^4}$
                     \item Subgroups that contain $\cyc{s}$.
                  \end{enumerate}
                  
      \textbf{Solution.}
      
      \begin{enumerate}
         \item The subgroups that are contained in $\cyc{sr^2, r^4}$ are those
               that have an upward path to the join of $sr^2$ and $r^4$. Thus
               these subgroups are: $1$, $\cyc{r^4}$, $\cyc{sr^2}$,
               $\cyc{sr^6}$, and $\cyc{sr^2, r^4}$.
         \item The subgroups that are contained in $\cyc{sr^7, r^4}$ are those
               that have an upward path to the join of $sr^3$ and $r^4$. Since
               the join of $sr^7$ and $r^4$ is $\cyc{sr^3, r^4}$. It follows
               that these subgroups are: $1$, $\cyc{r^4}$, $\cyc{sr^3}$,
               $\cyc{sr^7}$, and $\cyc{sr^7, r^4} = \cyc{sr^3, r^4}$.
         \item The subgroups that contain $r^4$ are: $\cyc{r^4}$,
               $\cyc{sr^2, r^4}$, $\cyc{s, r^4}$, $\cyc{r^2}$,
               $\cyc{sr^3, r^4}$, $\cyc{sr^5, r^4}$, $\cyc{s, r^2}$, $\cyc{r}$,
               $\cyc{sr, r^2}$, and $D_{16}$.
         \item The subgroups that contain $s$ are: $\cyc{s}$, $\cyc{s, r^4}$,
               $\cyc{s, r^2}$, and $D_{16}$.
      \end{enumerate}
%%%%%%%%%%%%%%%%%%%%%%%%%%%%%%%%%%%%%2.5.3%%%%%%%%%%%%%%%%%%%%%%%%%%%%%%%%%%%%%%
   \item[2.5.3]   Show that the subgroup $\cyc{s, r^2}$ of $D_8$ is isomorphic
                  to $V_4$.
                  
      \textbf{Proof.} By Exercise 1.1.36, there is exactly one group, say $K$,
      of order 4 that has no element of order 4. Since
      $\cyc{s, r^2} = \{1, s, r^2, sr^2\}$, it follows that every non-identity
      element of $\cyc{s, r^2}$ has order 2, so that $\cyc{s, r^2} \cong K$.
      Similarly, the Klein-4 group has no element of order 4. Thus
      $V_4 \cong K$, and we conclude that $V_4 \cong \cyc{s, r^2}$. \qed
%%%%%%%%%%%%%%%%%%%%%%%%%%%%%%%%%%%%%2.5.4%%%%%%%%%%%%%%%%%%%%%%%%%%%%%%%%%%%%%%
   \item[2.5.4]   Use the given lattice to find all pairs of elements that
                  generate $D_8$ (there are 12 pairs).
                  
      \textbf{Solution.} It suffices to find all pairs of elements whose join is
      $D_8$. They are: $\cyc{s, r}$, $\cyc{s, rs}$, $\cyc{s, r^3s}$,
      $\cyc{r^2s, rs}$, $\cyc{r^2s, r^3s}$, $\cyc{r^2s, r}$, $\cyc{r^2s, r^3}$,
      $\cyc{r, rs}$, $\cyc{r, r^3s}$,  $\cyc{r^3, s}$, $\cyc{r^3, rs}$, and
      $\cyc{r^3, r^3s}$,
%%%%%%%%%%%%%%%%%%%%%%%%%%%%%%%%%%%%%2.5.5%%%%%%%%%%%%%%%%%%%%%%%%%%%%%%%%%%%%%%
   \item[2.5.5]   Use the given lattice to find all elements $x \in D_{16}$
                  such that $D_{16} = \cyc{x, s}$ (there are 8 such elements
                  $x$).
                  
      \textbf{Solution.} By observing the given lattice of $D_{16}$, we find
      that      
      $$x \in \{r, r^3, r^5, r^7, sr^3, sr^7, sr^5, sr\}.$$
%%%%%%%%%%%%%%%%%%%%%%%%%%%%%%%%%%%%%2.5.6%%%%%%%%%%%%%%%%%%%%%%%%%%%%%%%%%%%%%%
   \item[2.5.6]   Use the given lattices to help find the centralizers of every
                  element in the following groups:

                  (a) $D_8$ \qquad (b) $Q_8$ \qquad
                  (c) $S_3$ \qquad (d) $D_{16}$.
                  
      \begin{enumerate}
         \item $$
                  \begin{tabular}{@{}|c|c|@{}} \hline
                     Elements in $D_8$ & Centralizer \\ \hline
                     1, $r^2$ & $D_8$ \\ \hline
                     $r, r^3$ & $\cyc{r}$ \\ \hline
                     $s, r^2s$ & $\cyc{s, r^2}$ \\ \hline
                     $rs, r^3s$ & $\cyc{rs, r^2}$ \\ \hline
                  \end{tabular}
               $$
         \item $$
                  \begin{tabular}{@{}|c|c|@{}} \hline
                     Elements in $Q_8$ & Centralizer \\ \hline
                     $\pm1$ & $Q_8$ \\ \hline
                     $\pm i$ & $\cyc{i}$ \\ \hline
                     $\pm j$ & $\cyc{j}$ \\ \hline
                     $\pm k$ & $\cyc{k}$ \\ \hline
                  \end{tabular}
               $$
         \item $$
                  \begin{tabular}{@{}|c|c|@{}} \hline
                     Element(s) in $Q_8$ & Centralizer \\ \hline
                     1 & $S_3$ \\ \hline
                     (1 2) & $\cyc{(1\;2)}$ \\ \hline
                     (1 3) & $\cyc{(1\;3)}$ \\ \hline
                     (2 3) & $\cyc{(2\;3)}$ \\ \hline
                     (1 2 3), (1 3 2) & $\cyc{(1\;2\;3)}$ \\ \hline
                  \end{tabular}
               $$
         \item $$
                  \begin{tabular}{@{}|c|c|@{}} \hline
                     Elements in $D_{16}$ & Centralizer \\ \hline
                     1, $r^4$ & $D_{16}$ \\ \hline
                     $r, r^2, r^3, r^5, r^6, r^7$ & $\cyc{r}$ \\ \hline
                     $s, sr^4$ & $\cyc{s, r^4}$ \\ \hline
                     $sr, sr^5$ & $\cyc{sr^5, r^4}$ \\ \hline
                     $sr^2, sr^6$ & $\cyc{sr^2, r^4}$ \\ \hline
                     $sr^3, sr^7$ & $\cyc{sr^3, r^4}$ \\ \hline
                  \end{tabular}
               $$
      \end{enumerate}
%%%%%%%%%%%%%%%%%%%%%%%%%%%%%%%%%%%%%2.5.7%%%%%%%%%%%%%%%%%%%%%%%%%%%%%%%%%%%%%%
   \item[2.5.7]   Find the center of $D_{16}$.
   
      \textbf{Solution.} From Exercise 2.5.6(d), we see that only 1 and $r^4$
      are in the all the centralizers of the elements of $D_{16}$. Thus
      $Z(D_{16}) = \cyc{r_4}$.
%%%%%%%%%%%%%%%%%%%%%%%%%%%%%%%%%%%%%2.5.8%%%%%%%%%%%%%%%%%%%%%%%%%%%%%%%%%%%%%%
   \item[2.5.8]   In each of the following groups find the normalizer of each
                  subgroup:

                  (a) $S_3$ \qquad (b) $Q_8$.
%%%%%%%%%%%%%%%%%%%%%%%%%%%%%%%%%%%%%2.5.9%%%%%%%%%%%%%%%%%%%%%%%%%%%%%%%%%%%%%%
   \item[2.5.9]   Draw the lattices of subgroups of the following groups:

                  (a) $\Z/16\Z$ \qquad (b) $\Z/24\Z$ \qquad
                  (c) $\Z/48\Z$. [See Exercise 6 in Section 3.]
%%%%%%%%%%%%%%%%%%%%%%%%%%%%%%%%%%%%%2.5.10%%%%%%%%%%%%%%%%%%%%%%%%%%%%%%%%%%%%%
   \item[2.5.10]  Classify groups of order 4 by proving that if $|G| = 4$ then
                  $G \cong Z_4$ or $G\cong V_4$. [See Exercise 36, Section 1.1.]
%%%%%%%%%%%%%%%%%%%%%%%%%%%%%%%%%%%%%2.5.11%%%%%%%%%%%%%%%%%%%%%%%%%%%%%%%%%%%%%
   \item[2.5.11]  Consider the group of order 16 with the following
                  presentation:

                  $$QD_{16} = \cyc{\sigma, \tau : \sigma^8 = \tau^2 = 1,
                    \sigma\tau = \tau\sigma^3}$$
                  (called the \textit{quasidihedral} or \textit{semidihedral}
                  group of order 16). This group has three subgroups of order 8:
                  $\cyc{\tau, \sigma^2} \cong D_8, \cyc{\sigma} \cong Z_8$ and
                  $\cyc{\sigma^2, \sigma\tau} \cong Q_8$ and every proper
                  subgroup is contained in one of these three subgroups. Fill in
                  the missing subgroups in the lattice of all subgroups of the 
                  quasidiheral group on the following page, exhibiting each
                  subgroup with at most two generators. (This is another example
                  of a nonplanar lattice.)
\end{enumerate}

\noindent The next three examples lead to two nonisomorphic groups that have the 
          same lattice of subgroups.

\begin{enumerate}
%%%%%%%%%%%%%%%%%%%%%%%%%%%%%%%%%%%%%2.5.12%%%%%%%%%%%%%%%%%%%%%%%%%%%%%%%%%%%%%
   \item[2.5.12]  The group
                  $A = Z_2 \times Z_4 = \cyc{a, b : a^2 = b^4 = 1, ab = ba}$ has
                  order 8 and has three subgroups of order 4:
                  $\cyc{a, b^2} \cong V_4$, $\cyc{b} \cong Z_4$ and
                  \begin{verbatim}
                     *
                     *
                     *
                     *
                     *
                     *
                     *
                     *
                     *
                  \end{verbatim}
                  $\cyc{ab} \cong Z_4$ and every proper subgroup is contained in
                  one of these three. Draw the lattice of all subgroups of $A$,
                  giving each subgroup in terms of at most two generators.
%%%%%%%%%%%%%%%%%%%%%%%%%%%%%%%%%%%%%2.5.13%%%%%%%%%%%%%%%%%%%%%%%%%%%%%%%%%%%%%
   \item[2.5.13]  The group
                  $G = Z_2 \times Z_8 = \cyc{x, y : x^2 = y^8 = 1, xy = yx}$ has
                  order 16 and has three subgroups of order 8:
                  $\cyc{x, y^2} \cong Z_2 \times Z_4$, $\cyc{y} \cong Z_8$ and
                  $\cyc{xy} \cong Z_8$ and every proper subgroup is contained in
                  one of these three. Draw the lattice of all subgroups of $G$,
                  giving each subgroup in terms of at most two generators.
%%%%%%%%%%%%%%%%%%%%%%%%%%%%%%%%%%%%%2.5.14%%%%%%%%%%%%%%%%%%%%%%%%%%%%%%%%%%%%%
   \item[2.5.14]  Let $M$ be the group of order 16 with the following 
                  presentation:
                  $$\cyc{u, v : u^2 v^8 = 1, vu = uv^5}$$
                  (sometimes called the \textit{modular} group of order 16). It
                  has three subgroups of order 8: $\cyc{u, v^2}$, $\cyc{v}$, and
                  $\cyc{uv}$ and every proper subgroup is contained in one of
                  these three. Prove that $\cyc{u, v^2} \cong Z_2 \times Z_4$,
                  $\cyc{v} \cong Z_8$ and $\cyc{uv} \cong Z_8$. Show that the
                  lattice of subgroups of $M$ is the same as the lattice of
                  subgroups of $Z_2 \times Z_8$ (cf. Exercise 13) but that these
                  two groups are not isomorphic.
%%%%%%%%%%%%%%%%%%%%%%%%%%%%%%%%%%%%%2.5.15%%%%%%%%%%%%%%%%%%%%%%%%%%%%%%%%%%%%%
   \item[2.5.15]  Describe the isomorphism type of each of the three subgroups
                  of $D_{16}$ of order 8.
%%%%%%%%%%%%%%%%%%%%%%%%%%%%%%%%%%%%%2.5.16%%%%%%%%%%%%%%%%%%%%%%%%%%%%%%%%%%%%%
   \item[2.5.16]  Use the lattice of subgroups of the quasidihedral group of
                  order 16 to show that every element of order 2 is contained in
                  the proper subgroup $\cyc{\tau, \sigma^2}$.
%%%%%%%%%%%%%%%%%%%%%%%%%%%%%%%%%%%%%2.5.17%%%%%%%%%%%%%%%%%%%%%%%%%%%%%%%%%%%%%
   \item[2.5.17]  Use the lattice of subgroups of the modular group $M$ of order
                  16 to show that the set $\{x \in M : x^2 = 1\}$ is a subgroup
                  of $M$ isomorphic to the Klein 4-group.
%%%%%%%%%%%%%%%%%%%%%%%%%%%%%%%%%%%%%2.5.18%%%%%%%%%%%%%%%%%%%%%%%%%%%%%%%%%%%%%
   \item[2.5.18]  Use the lattice to help find the centralizer of every element
                  of $QD_{16}$.
%%%%%%%%%%%%%%%%%%%%%%%%%%%%%%%%%%%%%2.5.19%%%%%%%%%%%%%%%%%%%%%%%%%%%%%%%%%%%%%
   \item[2.5.19]  Use the lattice to help find $N_{D_{16}}(\cyc{s, r^4})$.
%%%%%%%%%%%%%%%%%%%%%%%%%%%%%%%%%%%%%2.5.20%%%%%%%%%%%%%%%%%%%%%%%%%%%%%%%%%%%%%
   \item[2.5.20]  Use the lattice of subgroups of $QD_{16}$ to help find the
                  normalizers.

                  (a) $N_{QD_{16}}(\cyc{\tau\sigma})$ \qquad
                  (b) $N_{QD_{16}}(\cyc{\tau, \sigma^4})$.
\end{enumerate}

         
   \chapter{Introduction To Rings}
      \section{Basic Definition And Examples}
         \begin{enumerate}
   \item[]        Let $G$ be a group.
%%%%%%%%%%%%%%%%%%%%%%%%%%%%%%%%%%%Lemm1.1.1%%%%%%%%%%%%%%%%%%%%%%%%%%%%%%%%%%%%
   \item[]        \textbf{Lemma 1.1.1} Let $x \in G$ and let $m$ be an integer.
                  Then we have that
                  $$x^{m+1} = x^mx^1.$$

      \textbf{Proof.}  Consider the following cases:

      \textbf{Case 1:} \textit{$m = 0$}. It follows that
      $$x^{m+1} = x^{0+1} = x^1 = 1x^1 = x^0x^1 = x^mx^1.$$

      \textbf{Case 2:} \textit{$m$ is positive.} Then it follows that $m + 1$ is 
      positive, so that
      \begin{align*}
         x^{m + 1} &= \underbrace{x \cdot x \cdots x}_{m+1\text{ factors}} \\
            &= \underbrace{x \cdot x \cdots x}_{m\text{ factors}} \cdot x^1 \\
            &= x^mx^1.
      \end{align*}

      \textbf{Case 3:} \textit{$m$ is negative.} If $m = -1$, then we have that
      $$x^{m+1} = x^{-1+1} = x^0 = 1 = x^{-1}x^1 = x^mx^1.$$
      If $m < -1$, then $m + 1$ is negative, so that $-(m + 1) = -m - 1$ is 
      positive. Thus
      \begin{align*}
         x^mx^1 &= x^{-(-m)}x^1 \\
                &= \underbrace{x^{-1} \cdot x^{-1} \cdots x^{-1}}_{
                   -m\text{ factors}} \cdot x^1 \\
                &= \underbrace{x^{-1} \cdot x^{-1} \cdots x^{-1}}_{
                   -m-1\text{ factors}} \cdot (x^{-1} \cdot x^1) \\
                &= \underbrace{x^{-1} \cdot x^{-1} \cdots x^{-1}}_{
                   -m-1\text{ factors}} \\
                &= x^{-(-m-1)} \\
                &= x^{m+1}.
      \end{align*}

      In all cases, we can see that our assertion holds. \qed
%%%%%%%%%%%%%%%%%%%%%%%%%%%%%%%%%%%Lemm1.1.2%%%%%%%%%%%%%%%%%%%%%%%%%%%%%%%%%%%%
   \item[]        \textbf{Lemma 1.1.2} \textit{Let $x$ and $g$ be members of a 
                  group $G$, and let $n$ be a positive integer, then it follows 
                  that $(g^{-1}xg)^n = g^{-1}x^ng$.}

      \textbf{Proof.} We shall show by induction that the equation
      \begin{equation}
         (g^{-1}xg)^n = g^{-1}x^ng \label{l1_1_2_1}
      \end{equation}
      holds for every positive integer $n$. It is clear that equation
      \ref{l1_1_2_1} holds for $n = 1$. So assume that it also holds for some
      positive integer $k$. So we must now show that the equation also holds for 
      $k + 1$. Thus
      \begin{align*}
         (g^{-1}xg)^{k+1} &= (g^{-1}xg)^kg^{-1}xg &[\text{Execise 1.1.19}] \\
                     &= g^{-1}x^kgg^{-1}xg &[\text{Inductive hypothesis}] \\
                     &= g^{-1}x^kxg \\
                     &= g^{-1}x^{k+1}g,
      \end{align*}
      so that equation \eqref{l1_1_2_1} holds for $k+1$. Hence by the Principle 
      of Mathematical Induction, equation \eqref{l1_1_2_1} holds for every 
      positive integer $n$. \qed
%%%%%%%%%%%%%%%%%%%%%%%%%%%%%%%%%%%Lemm1.1.3%%%%%%%%%%%%%%%%%%%%%%%%%%%%%%%%%%%%
   \item[]        \textbf{Lemma 1.1.3} \textit{Let $x$ be an element of finite
                  order $n$ in $G$. If $x^m = 1$, then it follows that
                  $n \mid m$.}

      \textbf{Proof.} Suppose that $x^m = 1$. By the Division Algorithm, there
      exist unique integers $q$ and $r$ such that $m = qn + r$ and
      $0 \le r < n$. Now we have that
      $$1 = x^m = x^{qn+r} = x^{qn}x^r = (x^n)^qx^r = 1^qx^r = x^r.$$
      Since $|x| = n$, we cannot have $0 < r < n$; thus the only remaining
      possibility is $r = 0$, so that $n = qm$, as desired. \qed
%%%%%%%%%%%%%%%%%%%%%%%%%%%%%%%%%%%Lemm1.1.4%%%%%%%%%%%%%%%%%%%%%%%%%%%%%%%%%%%%
   \item[]        \textbf{Lemma 1.1.4} \textit{Let $(x, y)$ be an element of
                  $A \times B$ where $A$ and $B$ are groups. For any positive
                  integer $n$, we then have that $(x, y)^n = (x^n, y^n)$.}

      \textbf{Proof.} We shall induct on $n$. Our assertion clearly holds if
      $n$ is 1, so assume that it holds for some positive integer $k$. Thus we
      have that
      \begin{align*}
         (x, y)^{k+1} &= (x, y)(x, y)^k &[\text{Exercise 1.1.19}] \\
                      &= (x, y)(x^k, y^k) &[\text{Inductive hypothesis}] \\
                      &= (xx^k, yy^k) \\
                      &= (x^{k+1}, y^{k+1}). &[\text{Exercise 1.1.19}]
      \end{align*}
      The above shows that our assertion also holds for $k + 1$, so that by
      the Principle of Mathematical Induction it must holds for every integer
      $n$. \qed
%%%%%%%%%%%%%%%%%%%%%%%%%%%%%%%%%%%%%1.1.1%%%%%%%%%%%%%%%%%%%%%%%%%%%%%%%%%%%%%%
   \item[1.1.1]   Determine which of the following binary operations are
                  associative:
                  \begin{enumerate}
                     \item the operation $*$ on $\Z$ defined by $a * b = a - b$.
                     \item the operation $*$ on $\R$ defined by
                           $a * b = a + b + ab$.
                     \item the operation $*$ on $\Q$ defined by
                           $\displaystyle a * b = \frac{a + b}{5}$.
                     \item the operation $*$ on $\Z \times \Z$ defined by
                           $(a, b) * (c, d) = (ad + bc, bd)$.
                     \item the operation $*$ on $\Q - \{0\}$ defined by
                           $\displaystyle a * b = \frac{a}{b}$.
                  \end{enumerate}
                  
      \textbf{Solution.}
   
      \begin{enumerate}
         \item The binary operation $*$ on $\Z$ is not associative because
               $$(0 * 0) * 1 = -1 \neq 1 = 0 * (0 * 1).$$
         \item We claim that $*$ is associative on $\R$.
      
               \textbf{Proof.} Let $r_1, r_2, r_3 \in \R$. Then it follows that
               \begin{align*}
                  (r_1 * r_2) * r_3 &= (r_1 + r_2 + r_1r_2) * r_3 \\
                     &= (r_1 + r_2 + r_1r_2 + r_3) +
                        (r_1r_3 + r_2r_3 + r_1r_2r_3) \\
                     &= r_1 + r_2 + r_3 + r_1r_2 + r_2r_3 +
                        r_1r_3 + r_1r_2r_3 \\
                     &= (r_1 + r_2 + r_3 + r_2r_3) + r_1(r_2 + r_3 + r_2r_3) \\
                     &= r_1 + (r_2 * r_3) + r_1(r_2 * r_3) \\
                     &= r_1 * (r_2 * r_3),
               \end{align*}
               so that our claim holds. \qed
         \item The binary operation $*$ on $\Q$ is not associative because
               $$(0 * 0) * 25 = 5 \neq 1 = 0 * (0 * 25).$$
         \item We claim that $*$ is associative on $\Z \times \Z$.
      
               \textbf{Proof.} Let $(z_1, z_2)$, $(z_3, z_4)$,
               $(z_5, z_6) \in \Z \times \Z$. Then it follows that
               \begin{align*}
                  (z_1, z_2) * [(z_3, z_4) * (z_5, z_6)] 
                     &= (z_1, z_2) * [(z_3z_6 + z_4z_5, z_4z_6)] \\
                     &= (z_1z_4z_6 + z_2z_3z_6 + z_2z_4z_5, z_2z_4z_6) \\
                     &= ((z_1z_4 + z_2z_3) \cdot z_6 + z_2z_4 \cdot z_5,
                          z_2z_4 \cdot z_6) \\
                     &= (z_1z_4 + z_2z_3, z_2z_4) * (z_5, z_6) \\
                     &= [(z_1, z_2) * (z_3, z_4)] * (z_5, z_6),
               \end{align*}
               so that our claim holds. \qed
         \item The binary operation $*$ on $\Q - \{0\}$ is not associative
               because
               $$(4 * 1) * 2 = 2 \neq 8 = 4 * (1 * 2).$$
      \end{enumerate}
%%%%%%%%%%%%%%%%%%%%%%%%%%%%%%%%%%%%%1.1.2%%%%%%%%%%%%%%%%%%%%%%%%%%%%%%%%%%%%%%
   \item[1.1.2]   Decide which of the binary operations in the preceding
                  exercise are commutative.
                  
      \begin{enumerate}      
         \item The binary operation $*$ on $\Z$ is not commutative because
               $$1 * 0 = 1 \neq -1 = 0 * 1.$$
         \item The binary operation $*$ on $\R$ is commutative because addition
               and multiplication are commutative on $\R$.
         \item The binary operation $*$ on $\Q$ is commutative because addition
               is commutative on $\Q$.
         \item A quick check will show us that $*$ is commutative on
               $\Z \times \Z$. That is, for all $(z_1, z_2)$, $(z_3, z_4)$
               $\in \Z \times \Z$, we must have that
               \begin{align*}
                  (z_1, z_2) * (z_3, z_4) &= (z_1z_4 + z_2z_3, z_2z_4) \\
                                          &= (z_3z_2 + z_4z_1, z_4z_2) \\
                                          &= (z_3, z_4) * (z_1, z_2).
               \end{align*}
         \item The binary operation $*$ on $\Q - \{0\}$ is not commutative
               because
               $$1 * 2 = \frac{1}{2} \neq \frac{2}{1} = 2 * 1.$$
      \end{enumerate}
%%%%%%%%%%%%%%%%%%%%%%%%%%%%%%%%%%%%%%1.3%%%%%%%%%%%%%%%%%%%%%%%%%%%%%%%%%%%%%%%
   \item[1.1.3]   Prove that addition of residue classes in $\Z/n\Z$ is
                  associative (you may assume it is well defined).
                  
      \textbf{Proof.} Fix $n \in \Z^+$. Consider $\overline{a}$, $\overline{b}$,
      and $\overline{c}$ in $\Z/n\Z$. By Theorem 3, Pg. 9, we have that
      \begin{align*}
         \overline{a} + (\overline{b} + \overline{c})
            &= \overline{a} + \overline{b + c} \\
            &= \overline{a + b + c} \\
            &= \overline{a + b} + \overline{c} \\
            &= (\overline{a} + \overline{b}) + \overline{c},
      \end{align*}
      so that addition of residue classes in $\Z/n\Z$ is associative. \qed
%%%%%%%%%%%%%%%%%%%%%%%%%%%%%%%%%%%%%%1.4%%%%%%%%%%%%%%%%%%%%%%%%%%%%%%%%%%%%%%%
   \item[1.1.4]   Prove that multiplication of residue classes in $\Z/n\Z$ is
                  associative (you may assume it is well defined).
                  
      \textbf{Proof.} Fix $n \in \Z^+$. Consider $\overline{a}$, $\overline{b}$,
      and $\overline{c}$ in $\Z/n\Z$. By Theorem 3, Pg. 9, we have that
      \begin{align*}
         \overline{a} \cdot (\overline{b} \cdot \overline{c})
            &= \overline{a} \cdot \overline{bc} \\
            &= \overline{abc} \\
            &= \overline{ab} \cdot \overline{c} \\
            &= (\overline{a} \cdot \overline{b}) \cdot \overline{c},
      \end{align*}
      so that multiplication of residue classes in $\Z/n\Z$ is associative. \qed
%%%%%%%%%%%%%%%%%%%%%%%%%%%%%%%%%%%%%%1.5%%%%%%%%%%%%%%%%%%%%%%%%%%%%%%%%%%%%%%%
   \item[1.1.5]   Prove that for all $n > 1$ that $\Z/n\Z$ is not a group under
                  multiplication of residue classes.
                  
      \textbf{Proof.} Let $n$ be positive integer greater than 1. It follows
      that $\Z/n\Z$ is not a group under multiplication because $\overline{0}$
      has no multiplicative inverse. \qed
%%%%%%%%%%%%%%%%%%%%%%%%%%%%%%%%%%%%%%1.6%%%%%%%%%%%%%%%%%%%%%%%%%%%%%%%%%%%%%%%
   \item[1.1.6]   Determine which of the following sets are groups under
                  addition:
                  \begin{enumerate}
                     \item the set of rational numbers (including $0 = 0/1$) in
                           lowest terms whose denominators are odd.
                     \item the set of rational numbers (including $0 = 0/1$) in
                           lowest terms whose denominators are even.
                     \item the set of rational numbers of absolute value $< 1$.
                     \item the set of rational numbers of absolute value $\ge 1$
                           together with 0.
                     \item the set of rational numbers with denominators equal
                           to 1 or 2.
                     \item the set of rational numbers with denominators equal
                           to 1, 2, or 3.
                  \end{enumerate}

      \textbf{Solution.}

      \begin{enumerate}
         \item We claim that the set
               $$S = \left\{\frac{a}{b} \in \Q : b \text{ is odd} \text{ and }
                 \gcd(a, b) = 1\right\},$$
               is a group under addition.

               \textbf{Proof.} First we must show that $S$ is closed under 
               addition. Notice that $S$ is nonempty since it contains 7/5, so 
               let $r, s \in S$. By definition of $S$, we have that
               $r = a_1/b_1$ and $s = a_2/b_2$ for some integers $a_1$ and
               $a_2$, and nonzero integers $b_1$ and $b_2$, where $b_1$ and
               $b_2$ are odd and $\gcd(a_1, b_1) = \gcd(a_2, b_2) = 1$.
               It follows that
               \begin{align*}
                  r + s &= \frac{a_1}{b_1} + \frac{a_2}{b_2} \\
                        &= \frac{a_1b_2 + a_2b_1}{b_1b_2}.
               \end{align*}

               Since $b_1$ and $b_2$ are both odd, it must necessarily be the 
               case that $b_1b_2$ is also odd. In order words, $b_1b_2$ contains 
               no factor of 2, so that if we reduce $r + s$ to its lowest term, 
               the denominator of this lowest term will still be odd. Hence
               $r + s \in S$, so that $S$ is closed under addition. To complete 
               the proof we must now show that $S$ satisfies the group axioms. 
               We observe that $0/1$ is the identity element in $S$. Also, it is 
               clear that for all $s \in S$, we have $-s \in S$, so that every 
               element of $S$ has an inverse under addition. Since
               $S \subseteq \Q$, and since $\Q$ is associative under addition, 
               it follows that $S$ is also associative under addition. Thus $S$ 
               satisfies the group axioms, so that $(S, +)$ is a group. \qed
         \item The set
               $$S = \left\{\frac{a}{b} \in \Q : b \text{ is even} \text{ and }
                 \gcd(a, b) = 1\right\},$$
               is not a group under addition because it is not closed. Indeed,
               for $3/14 \in S$, we have $3/14 + 3/14 = 3/7 \notin S$.
         \item The set
               $$S = \left\{\frac{a}{b} \in \Q :
                     \left|\frac{a}{b}\right| < 1\right\},$$
               is not a group under addition because it is not closed. Indeed,
               for $9/10 \in S$, we have $9/10 + 9/10 = 18/10 \notin S$.
         \item The set
               $$S = \left\{\frac{a}{b} \in \Q : a = 0 \text{ or }
                     \left|\frac{a}{b}\right| \ge 1\right\},$$
               is not a group under addition because it is not closed. Indeed,
               for $-11/10, 10/10 \in S$, we have
               $-11/10 + 10/10 = -1/10 \notin S$.
         \item We claim that the set
               $$S = \left\{\frac{a}{b} \in \Q : b = 1 \text{ or }
                 b = 2\right\},$$
               is a group under addition.

               \textbf{Proof.} It is clear that 0 is the identity for $S$ under
               addition, that $S$ is associative under addition (because
               $S \subset \Q$ and $\Q$ is associative under addition, and that
               the inverse of an element in $S$ is its additive inverse in $\Q$.
               So to complete the proof, we need only show that $S$ is closed
               under addition. Let $a_1/b_1, a_2/b_2 \in \Q$. By observation, we
               note that $a_1/b_1 + a_2/b_2$ must have a denominator of 1 or 2,
               so that it is in $S$. Thus $S$ is closed under addition. \qed
         \item The set
               $$S = \left\{\frac{a}{b} \in \Q : b \in {1, 2, 3} \right\},$$
               is not a group under addition because it is not closed. Indeed,
               for $1/2, 1/3 \in S$, we have $1/2 + 1/3 = 5/6 \notin S$.
      \end{enumerate}
%%%%%%%%%%%%%%%%%%%%%%%%%%%%%%%%%%%%%%1.7%%%%%%%%%%%%%%%%%%%%%%%%%%%%%%%%%%%%%%%
   \item[1.1.7]   Let $G = \{x \in \R : 0 \le x < 1\}$ and for $x, y \in G$ let
                  $x * y$ be the fractional part of $x + y$ (i.e.,
                  $x * y = x + y = [x + y]$ where $[a]$ is the greatest integer
                  less than or equal to $a$). Prove that $*$ is a well defined
                  binary operation on $G$ and that $G$ is an abelian group under
                  $*$ (called the \textit{real numbers mod }1).
                  
      \textbf{Proof.} The set $G$ is clearly non-empty, so consider
      $x, y, z \in G$. To show that $G$ is a group, we shall now prove that it 
      is well defined, associative, has an identity, and is closed under taking
      inverses.

      \textbf{Well Defined:} To show that $*$ is well defined is tantamount to
      showing that $G$ is closed under $*$.  By definition, we have that
      $0 \le x < 1$ and $0 \le y < 1$, so that $0 \le x + y < 2$. If
      $0 \le x + y < 1$, so that $[x + y] = 0$, then we have that
      $$0 \le x + y = x + y - [x + y] = x * y = x + y < 1.$$
      However if $1 \le x + y < 2$, so that $[x + y] = 1$ and
      $0 \le x + y - 1 < 1$, we must have that
      $$0 \le x + y - 1 = x + y - [x + y] = x * y = x + y - 1 < 1.$$
      In either case, we have $0 \le x * y < 1$; i.e. $x * y \in G$, so that $G$ 
      is closed under $*$. Also we have that
      $$x * y = x + y - [x + y] = y + x - [y + x],$$
      so that $G$ is abelian.

      \textbf{Associativity:} We have that
      \begin{align*}
         x * (y * z) &= x * (y + z - [y + z]) \\
              &= x + y + z - [y + z] - [x + y + z - [y + z]], \text{ and} \\ \\
         (x * y) * z &= (x + y - [x + y]) * z \\
                     &=  x + y + z - [x + y] - [x + y + z - [x + y]].
      \end{align*}
      By definition, we have that $0 \le x < 1$, $0 \le y < 1$, and
      $0 \le z < 1$, so that $0 \le x + y < 2$ and $0 \le y + z < 2$. Let us 
      now investigate the following possible cases:

      \textit{Case 1:} \textit{$0 \le x + y  < 1$ and $0 \le y + z < 1$}. That
      is $[x + y] = [y + z] = 0$. It then follows that
      $$x * (y * z) = (x * y) * z = x + y + z - [x + y + z].$$

      \textit{Case 2:} \textit{$1 \le x + y  < 2$ and $1 \le y + z < 2$}. That
      is $[x + y] = [y + z] = 1$. It then follows that
      $$x * (y * z) = (x * y) * z = x + y + z - 1 - [x + y + z - 1].$$

      \textit{Case 3:} \textit{$0 \le x + y  < 1$ and $1 \le y + z < 2$}. That
      is $[x + y] = 0$, and $[y + z] = 1$. It then follows that
      $$(x * y) * z = x + y + z - [x + y + z].$$
      Since $0 \le x + y < 1$ and $0 \le z < 1$, we must have that
      $0 \le x + y + z < 2$. Similarly, since $1 \le y + z < 2$ and
      $0 \le x < 1$, we must have that $1 \le x + y + z < 3$, and since we 
      already showed that $0 \le x + y + z < 2$, it follows that
      $1 \le x + y + z < 2$. Hence $[x + y + z] = 1$. We can then conclude that 
      $(x * y) * z = x +y + z - 1$. Now we have that
      $$x * (y * z) = x + y + z - 1 - [x + y + z - 1].$$
      We already showed that $1 \le x + y + z < 2$; thus,
      $0 \le x + y + z - 1 < 1$, so that $[x + y + z - 1] = 0$; that is,
      $$x * (y * z) = x + y + z - 1 = (x * y) * z.$$
   
      \textit{Case 4:} \textit{$1 \le x + y  < 2$ and $0 \le y + z < 1$}. Apply
      Case 3, with the roles of $x + y$ and $y + z$ interchanged.

      We have thus shown that in all possible cases, we have
      $$x * (y * z) = (x * y) * z,$$
      so that $G$ is associative under $*$.


      \textbf{Identity:} We observe that $0 \in G$ is the identity element since
      $$x * 0 = x + 0 - [x + 0] = x - [x] = x - 0 = x.$$

      \textbf{Inverse:} Suppose $x \neq 0$, so that $0 < x < 1$, and thus
      $0 < 1 - x < 1$; that is $1 - x \in G$. It follows that
      $$x * (1 - x) = x + (1 - x) + [x + (1 - x)] = 1 - 1 = 0,$$
      so that $1 - x$ is the inverse of $x \in G$, with $x \neq 0$. Clearly, the 
      inverse of 0 is 0. \\

      We can now conclude that $(G, *)$ is a group. \qed
%%%%%%%%%%%%%%%%%%%%%%%%%%%%%%%%%%%%%%1.8%%%%%%%%%%%%%%%%%%%%%%%%%%%%%%%%%%%%%%%
   \item[1.1.8]   Let $G = \{z \in \C : z^n = 1 \text{ for some } n \in \Z^+\}$.
                  \begin{enumerate}
                     \item Prove that $G$ is a group under multiplication
                           (called the group of \textit{roots of unity} in
                           $\C$).
                     \item Prove that $G$ is not a group under addition.
                  \end{enumerate}
                  
      \textbf{Proof.}
      
      \begin{enumerate}
         \item We observe that 1 is the identity element of $G$, so that $G$ is
               not empty. So let $x, y, z \in G$.
               
               \textbf{Closure:} By definition, there exist positive integers
               $m$ and $n$ such that $x^m = y^n = 1$. Thus
               $(xy)^{mn} = (x^m)^n(y^n)^m = 1^n1^m = 1$. This says that $G$ is
               closed under multiplication.
               
               \textbf{Associativity:} Since $\C$ is associative under
               multiplication and since $G \subseteq \C$, it follows that $G$ is
               associative under multiplication.
               
               \textbf{Identity:} As state above, the identity of $G$ is clearly
               1.
               
               \textbf{Inverse:} Notice that since
               $(x^{m - 1})^m = (x^m)^{m - 1} = 1$, we must have that
               $x^{m - 1} \in G$. Thus we have $x^{m - 1}x = x^m = 1$; i.e., the
               inverse of $x$ is $x^{m - 1}$.
               
               We have thus shown that $G$ is a group under multiplication. \qed
         \item $G$ is not a group under addition because it is not closed under
               addition. In particular, we have $1 \in G$, but
               $1 + 1 = 2 \notin G$ because $2^n \neq 1$ for any positive
               integer.
      \end{enumerate}
%%%%%%%%%%%%%%%%%%%%%%%%%%%%%%%%%%%%%%1.9%%%%%%%%%%%%%%%%%%%%%%%%%%%%%%%%%%%%%%%
   \item[1.1.9]   Let $G = \{a + b\sqrt{2} \in \R : a, b \in \Q\}$.
                  \begin{enumerate}
                     \item Prove that $G$ is a group under addition.
                     \item Prove that the nonzero elements of $G$ are a group 
                           under multiplication. [``Rationalize the
                           denominators" to find multiplicative inverse.]
                  \end{enumerate}
                  
      \textbf{Proof.}
      
      \begin{enumerate}
         \item \textbf{Closure:} $G$ is clearly nonempty, so let $x, y \in G$.
               By definition of $G$, it follows that $x = a_1 + b_1\sqrt{2}$ and
               $y = a_2 + b_2\sqrt{2}$ for some rational numbers $a_1$, $b_1$,
               $a_2$, and $b_2$. Thus
               $$x + y = (a_1 + a_2) + (b_1 + b_2)\sqrt{2} \in G,$$
               so that $G$ is closed under addition.
               
               \textbf{Associativity:} Since $\R$ is associative under addition
               and since $G \subseteq \R$, it follows that $G$ is associative
               under addition.
               
               \textbf{Identity:} The identity of $G$ is 0.
               
               \textbf{Inverse:} For an element $x = a_1 + b_1\sqrt{2} \in G$,
               the additive inverse of $x$ is $-a_1 - b_1\sqrt{2} \in G$.
               
               We have thus shown that $G$ is a group under addition. \qed
         \item Let $G^{\times}$ denote the set of nonzero elements of $G$.
         
               \textbf{Closure:} Let $x, y \in G^{\times}$. By definition of
               $G$, it follows that $x = a_1 + b_1\sqrt{2}$ and
               $y = a_2 + b_2\sqrt{2}$ for some rational numbers $a_1$, $b_1$,
               $a_2$, and $b_2$, with $a_1$ and $b_1$ not both zero and $a_2$
               and $b_2$ not both zero. Thus
               $$xy = (a_1a_2 + 2b_1b_2) + (a_1b_2 + a_2b_1)\sqrt{2}.$$
               Since neither $x$ nor $y$ is zero, it must be the case that $xy$
               is not zero, so that $G^{\times}$ is closed under multiplication.
               
               \textbf{Associativity:} Since $\R$ is associative under
               multiplication and since $G^{\times} \subseteq \R$, it follows
               that $G^{\times}$ is associative under multiplication.
               
               \textbf{Identity:} The element $1 = 1 + 0\sqrt{2} \in G^{\times}$
               is the identity of $G^{\times}$.
               
               \textbf{Inverse:} Let $x = a_1 + b_1\sqrt{2} \in G^{\times}$.
               Since $x \neq 0$, the real number $1/x$ exists, and we have that
               $$\frac{1}{x} = \frac{1}{a_1 + b_1\sqrt{2}}
                 \frac{a_1 - b_1\sqrt{2}}{a_1 - b_1\sqrt{2}} =
                 \left(\frac{a_1}{{a_1}^2 - 2{b_1}^2} -
                 \frac{b_1}{{a_1}^2 - 2{b_1}^2}\sqrt{2}\right) \in G^{\times}.
               $$
               
               Since $1/x \in G^{\times}$ and since $x \cdot 1/x = 1$, we have
               that $1/x$ is the multiplicative inverse of $x$.
               
               We have thus shown that $G^{\times}$ is a group under
               multiplication. \qed
      \end{enumerate}
%%%%%%%%%%%%%%%%%%%%%%%%%%%%%%%%%%%%%%1.10%%%%%%%%%%%%%%%%%%%%%%%%%%%%%%%%%%%%%%
   \item[1.1.10]  Prove that a finite group is abelian if and only if its group
                  table is a symmetric matrix.
                  
      \textbf{Proof.} Let $G$ be a group such that $|G| = n \in \Z^+$, and let
      $(a_{ij})$ denote the matrix of the group table of $G$. Since $G$ is
      finite, we can enumerate the elements of $G$ like so:
      $$G = \{g_1, g_2, \ldots, g_n\}.$$      
      $(\Leftarrow)$ Suppose that $(a_{ij})$ is a symmetric matrix. Let
      $a, b \in G$. Then we have that $a = g_r$ and $b = g_s$ for some
      $r, s \in \{1, 2, \ldots, n\}$. Since $(a_{ij})$ is symmetric, we must
      have that
      $$ab = g_rg_s = a_{rs} = a_{sr} = g_sg_r = ba,$$
      so that $G$ is abelian.
      
      $(\Rightarrow)$ Now suppose that $G$ is abelian. Consider
      $a_{rs} \in (a_{ij})$. It follows that
      $$a_{rs} = g_rg_s = g_sg_r = a_{sr},$$
      so that $(a_{ij})$ is symmetric. \qed      
%%%%%%%%%%%%%%%%%%%%%%%%%%%%%%%%%%%%%%1.11%%%%%%%%%%%%%%%%%%%%%%%%%%%%%%%%%%%%%%
   \item[1.1.11]  Find the orders of each element of the additive group
                  $\Z/12\Z$.
                  
      \textbf{Solution.} The orders of the elements $\overline{0}$,
      $\overline{1}$, $\overline{2}$, $\overline{3}$, $\overline{4}$,
      $\overline{5}$, $\overline{6}$, $\overline{7}$, $\overline{8}$,
      $\overline{9}$, $\overline{10}$, and $\overline{11}$ in $\Z/12\Z$ are
      1, 12, 6, 4, 3, 12, 2, 12, 3, 4, 6, and 12.
%%%%%%%%%%%%%%%%%%%%%%%%%%%%%%%%%%%%%%1.12%%%%%%%%%%%%%%%%%%%%%%%%%%%%%%%%%%%%%%
   \item[1.1.12]  Find the orders of the following elements of the
                  multiplicative group $(\Z/12\Z)^\times: \overline{1},
                  \overline{-1}, \overline{5}, \overline{7}, \overline{-7}, 
                  \overline{13}$.
                  
      \textbf{Solution.} The orders of the elements $\overline{1}$,
      $\overline{-1}$, $\overline{5}$, $\overline{7}$, $\overline{-7}$,
      $\overline{13}$ in $(\Z/12\Z)^\times$ are 1, 11, 5, 7, 5, and 13.
%%%%%%%%%%%%%%%%%%%%%%%%%%%%%%%%%%%%%%1.13%%%%%%%%%%%%%%%%%%%%%%%%%%%%%%%%%%%%%%
   \item[1.1.13]  Find the orders of the following elements of the additive
                  group $\Z/36\Z: \overline{1}, \overline{2}, \overline{6}, 
                  \overline{9}, \overline{10}, \overline{12}, \overline{-1}, 
                  \overline{-10}, \overline{-18}$.
                  
      \textbf{Solution.} The orders of the elements $\overline{1}$,
      $\overline{2}$, $\overline{6}$, $\overline{9}$, $\overline{10}$,
      $\overline{12}$, $\overline{-1}$, $\overline{-10}$, and $\overline{-18}$
      in $\Z/36\Z$ are 1, 18, 6, 4, 18, 3, 36, 18, and 2.
%%%%%%%%%%%%%%%%%%%%%%%%%%%%%%%%%%%%%%1.14%%%%%%%%%%%%%%%%%%%%%%%%%%%%%%%%%%%%%%
   \item[1.1.14]  Find the orders of the following elements of the
                  multiplicative group $(\Z/36\Z)^\times: \overline{1},
                  \overline{-1}, \overline{5}, \overline{13}, \overline{-13},
                  \overline{17}$.
                  
      \textbf{Solution.} The orders of the elements $\overline{1}$,
      $\overline{-1}$, $\overline{5}$, $\overline{13}$, $\overline{-13}$,
      $\overline{17}$ in $(\Z/36\Z)^\times$ are 1, 35, 29, 25, 11, and 17.
%%%%%%%%%%%%%%%%%%%%%%%%%%%%%%%%%%%%%%1.15%%%%%%%%%%%%%%%%%%%%%%%%%%%%%%%%%%%%%%
   \item[1.1.15]  Prove that $(a_1a_2\cdots a_n)^{-1} =
                  {a_n}^{-1}{a_{n-1}}^{-1}\cdots {a_1}^{-1}$ for all
                  $a_1, a_2, \ldots, a_n \in G$.
                  
      \textbf{Proof.} We shall proceed by induction on $n$. The statement is
      trivial for $n = 1$. So assume that it also holds for some positive
      integer $k$. Let $b = a_1a_2\cdots a_k$. It then follows that
      \begin{align*}
         (a_1a_2\cdots a_ka_{k+1})^{-1} &= (b \cdot a_{k+1})^{-1} \\
            &= {a_{k+1}}^{-1}b^{-1} &[\text{By Proposition 1 (4)}] \\
            &= {a_{k+1}}^{-1}{a_k}^{-1}\cdots {a_1}^{-1}.
                  &[\text{Inductive hypothesis}]
      \end{align*}
      That is, our statement holds for $k + 1$, so that, by the Principle of
      Mathematical Induction, it holds for each positive integer $n$. \qed
%%%%%%%%%%%%%%%%%%%%%%%%%%%%%%%%%%%%%%1.16%%%%%%%%%%%%%%%%%%%%%%%%%%%%%%%%%%%%%%
   \item[1.1.16]  Let $x$ be an element of $G$. Prove that $x^2 = 1$ if and only
                  if $|x|$ is either 1 or 2.
                  
      \textbf{Proof.}
      
      $(\Leftarrow)$ Suppose that $x^2 = 1$. Now if $|x| > 2$, then by
      definition, $x^2 \neq 1$. The only remaining possibilities are $|x| = 1$
      or $|x| = 2$.
      
      $(\Rightarrow)$ Suppose that $|x| = 1$ or $|x| = 2$. It immediately
      follows that $x^2 = 1$. \qed
%%%%%%%%%%%%%%%%%%%%%%%%%%%%%%%%%%%%%%1.17%%%%%%%%%%%%%%%%%%%%%%%%%%%%%%%%%%%%%%
   \item[1.1.17]  Let $x$ be an element of $G$. Prove that if $|x| = n$ for some
                  positive integer $n$ then $x^{-1} = x^{n-1}$.
                  
      \textbf{Proof.} Suppose that $|x| = n \in \Z^+$. By Exercise 1.1.18, it
      follows that $x^{n-1}x^1 = x^{n-1+1} = x^n = 1$, so that
      $x^{-1} = x^{n-1}$. \qed      
%%%%%%%%%%%%%%%%%%%%%%%%%%%%%%%%%%%%%%1.18%%%%%%%%%%%%%%%%%%%%%%%%%%%%%%%%%%%%%%
   \item[1.1.18]  Let $x$ and $y$ be elements of $G$. Prove that $xy = yx$ if
                  and only if $y^{-1}xy =x$ if and only if $x^{-1}y^{-1}xy = 1$.
                  
      \textbf{Proof.} First assume that $xy = yx$. We then have that
      $yx = xy = 1xy = yy^{-1}xy$, so that $x = y^{-1}xy$ by left cancellation.
      Now assume that $y^{-1}xy = x$. Thus
      $x1 = x = y^{-1}xy = 1y^{-1}xy = xx^{-1}y^{-1}xy$, so that
      $1 = x^{-1}y^{-1}xy$ by left cancellation. Finally assume that
      $x^{-1}y^{-1}xy = 1$. Multiplying on the left by $yx$ will yield the
      equation $xy = yx$. \qed
%%%%%%%%%%%%%%%%%%%%%%%%%%%%%%%%%%%%%%1.19%%%%%%%%%%%%%%%%%%%%%%%%%%%%%%%%%%%%%%
   \item[1.1.19]  Let $x \in G$ and let $a, b \in \Z^+$.
                  \begin{enumerate}
                     \item Prove that $x^{a+b} = x^ax^b$.
                     \item Prove that $(x^a)^b = x^{ab}$.
                     \item Prove that $(x^a)^{-1} = x^{-a}$.
                     \item Establish part (a) for arbitrary integers $a$ and $b$
                           (positive, negative or zero).
                     \item Establish part (b) for arbitrary integers $a$ and $b$
                           (positive, negative or zero).
                  \end{enumerate}
               
      \textbf{Proof.}
      
      \begin{enumerate}
         \item We have that
               \begin{align*}
                  x^{a+b} &= \underbrace{xx\cdot x}_{a+b \text{ factors}} \\
                          &= \underbrace{xx\cdot x}_{a \text{ factors}}\mbox{ }
                             \underbrace{xx\cdot x}_{b \text{ factors}} \\
                          &= x^ax^b.
               \end{align*} \qed
         \item We have that
               \begin{align*}
                  (x^a)^b &= (\underbrace{xx\cdot x}_{a \text{ factors}})^b \\
                          &= \underbrace{xx\cdot x}_{ab \text{ factors}} \\
                          &= x^{ab}.
               \end{align*} \qed
         \item We have
               \begin{align*}
                  (x^a)^{-1}
                     &= (\underbrace{xx\cdot x}_{a \text{ factors}})^{-1} \\
                     &= \underbrace{x^{-1}x^{-1}\cdot x^{-1}}_{
                           a \text{ factors}} &[\text{Exercise 1.1.15}] \\
                     &= x^{-a}.
               \end{align*} \qed
         \item Now suppose that $a$ is an integer and $b$ is a positive integer.
               We shall induct on $b$ to show that
               \begin{equation}
                  x^{a+b} = x^ax^b. \label{1_1_19_1}
               \end{equation}
               By Lemma 1.1.1, \eqref{1_1_19_1} holds if $b$ equals 1. So assume
               that it also holds for some positive integer $k$. We now have
               that
               \begin{align*}
                  x^ax^{k+1} &= x^ax^kx^1 &[\text{Lemma 1.1.1}] \\
                             &= (x^ax^k)x^1 \\
                             &= x^{a+k}x^1 &[\text{Inductive hypothesis}] \\
                             &= x^{(a+k)+1} &[\text{Lemma 1.3.2}] \\
                             &= x^{a+(k+1)}, &[\text{Associativity of addition}]
               \end{align*}
               so that \eqref{1_1_19_1} holds for $k + 1$, and hence, by the 
               Principle of Mathematical Induction, it holds for each positive
               integer $n$. \\

               If $a$ is 0 or $b$ is 0, then Lemma 1.1.1 tells us that
               \eqref{1_1_19_1} holds, so the only remaining possibility is $a$ 
               and $b$ are negative.\footnote{If $a$ is positive and $b$ is
               negative, then interchange the roles of $a$ and $b$ in the 
               induction proof.} Now suppose that $a$ and $b$ are negative.
               Hence
               \begin{align*}
                  x^ax^b &= x^{-(-a)}x^{-(-b)} \\
                     &= (x^{-1})^{-a}(x^{-1})^{-b} &[\text{Definition}] \\
                     &= (x^{-1})^{(-a + (-b))} &[\text{Part (a)}] \\
                     &= x^{-(-a + (-b))} &[\text{Definition}] \\
                     &= x^{a+b}.
               \end{align*}

               Combining this result with part (a), we thus shown that
               \eqref{1_1_19_1} holds for all integers $a$ and $b$. \qed
         \item It is clear that part (b) holds if $a$ is 0 or $b$ is 0, so let
               us complete the proof for arbritrary integers $a$ and $b$.

               \textbf{Case 1:} \textit{$a$ is positive and $b$ is negative}. 
               Hence
               \begin{align*}
                  (x^a)^b &= (x^a)^{-(-b)} \\
                          &= [(x^a)^{-1}]^{-b} &[\text{Definition}] \\
                          &= (x^{-a})^{-b} &[\text{Part (c)}] \\
                          &= [(x^{-1})^a]^{-b} &[\text{Definition}] \\
                          &= (x^{-1})^{-ab} &[\text{Part (b)}] \\
                          &= x^{-(-ab)} &[\text{Definition}] \\
                          &= x^{ab}.
               \end{align*}

               \textbf{Case 2:} \textit{$a$ and $b$ are negative}. Thus
               \begin{align*}
                  (x^a)^b &= [x^{-(-a)}]^b \\
                          &= [(x^{-1})^{-a}]^b &[\text{Definition}] \\
                          &= (x^{-1})^{-ab} &[\text{Case 1}] \\
                          &= [(x^{-1})^{-1}]^{ab}. &[\text{Definition}] \\
                          &= x^{ab}. &[\text{Proposition 1 (3)}]
               \end{align*}

               \textbf{Case 3:} \textit{$a$ is negative and $b$ is positive}. 
               Thus
               \begin{align*}
                  (x^a)^b &= [x^{-(-a)}]^b \\
                          &= [(x^{-1})^{-a}]^b &[\text{Definition}] \\
                          &= (x^{-1})^{-ab} &[\text{Case 1}] \\
                          &= x^{-(-ab)} &[\text{Definition}] \\
                          &= x^{ab}.
               \end{align*}

               Combining these results with part (a), we can conclude that
               $(x^a)^b = x^{ab}$ holds for all integers $a$ and $b$ and
               $x \in G$. \qed
      \end{enumerate}
%%%%%%%%%%%%%%%%%%%%%%%%%%%%%%%%%%%%%%1.20%%%%%%%%%%%%%%%%%%%%%%%%%%%%%%%%%%%%%%
   \item[1.1.20]  For $x$ an element in $G$ show that $x$ and $x^{-1}$ have the
                  same order.

      \textbf{Proof.}

      \textbf{Case 1:} \textit{$|x| = n \in \Z^+$}. Since
      $(x^{-1})^n = (x^n)^{-1} = 1^{-1} = 1$, it follows that $|x^{-1}| \le n$,
      so suppose to the contrary that $|x^{-1}| = m < n$. Then we have that
      $$x^m = [(x^{-1})^{-1}]^m = [(x^{-1})^m]^{-1} = 1^{-1} = 1,$$
      a contradiction, so that $|x^{-1}| = n = |x|$.

      \textbf{Case 2:} \textit{$|x| = +\infty$}. Suppose to the contrary that
      $|x^{-1}| = n \in \Z^+$. As we argued in Case 1, it must be the case that
      $x^n = 1$, a contradiction. Thus $|x| = +\infty = |x^{-1}|$. \qed
%%%%%%%%%%%%%%%%%%%%%%%%%%%%%%%%%%%%%%1.20%%%%%%%%%%%%%%%%%%%%%%%%%%%%%%%%%%%%%%
   \item[1.1.21]  Let $G$ be a finite group and let $x$ be an element of $G$ of
                  order $n$. Prove that if $n$ is odd, then $x = (x^2)^k$ for
                  some $k$.

      \textbf{Proof.} Suppose that $n$ is odd. We can then write $n = 2k + 1$
      for some nonnegative integer $k$. By supposition, we have that
      $xx^{2k} = x^{2k+1} = 1 = x^{-2k}x^{2k}$, so that by right cancellation,
      we can conclude that $x = x^{-2k} = (x^2)^{-k}$. \qed
%%%%%%%%%%%%%%%%%%%%%%%%%%%%%%%%%%%%%%1.22%%%%%%%%%%%%%%%%%%%%%%%%%%%%%%%%%%%%%%
   \item[1.1.22]  If $x$ and $g$ are elements of the group $G$, prove that
                  $|x| = |g^{-1}xg|$. Deduce that $|ab| = |ba|$ for all
                  $a, b \in G$.

      \textbf{Proof.} Let $x, g \in G$.

      \textbf{Case 1:} \textit{$|x| = n \in \Z^+$}. By Lemma 1.1.2, it follows
      that $(g^{-1}xg)^n = g^{-1}x^ng = g^{-1}g =1$, so that $|g^{-1}xg| \le n$,
      so suppose to the contrary that $|g^{-1}xg| = m < n$. Then we have that
      $$g^{-1}1g = 1 = (g^{-1}xg)^m = g^{-1}x^mg,$$
      so that $x^m = 1$ by left and right cancellations, a contradiction; thus,  
      $|g^{-1}xg| = n = |x|$.

      \textbf{Case 2:} \textit{$|x| = +\infty$}. Suppose to the contrary that
      $|g^{-1}xg| = n \in \Z^+$. As we argued in Case 1, it must then be the 
      case that $x^n = 1$, a contradiction. Thus $|x| = +\infty = |g^{-1}xg|$.

      Now consider $a, b \in G$. Set $x = ab$ and $g = a$. Since 
      $|x| = |g^{-1}xg|$, it follows that $|ab| = |a^{-1}aba| = |ba|$. \qed
%%%%%%%%%%%%%%%%%%%%%%%%%%%%%%%%%%%%%%1.23%%%%%%%%%%%%%%%%%%%%%%%%%%%%%%%%%%%%%%
   \item[1.1.23]  Suppose $x \in G$ and $|x| = n < \infty$. If $n = st$ for some
                  positive integers $s$ and $t$, prove that $|x^s| = t$.

      \textbf{Proof.} Suppose $n = st$ for some positive integers $s$ and $t$.
      By supposition, we have that $1 = x^n = x^{st} = (x^s)^t$; i.e.,
      $|x^s| \le t$. Suppose to the contrary that $|x^s| = m < t$. Then we have
      that $1 = (x^s)^m = x^{sm}$. Since $0 < m < t$, it follows that
      $0 < sm < st = n$. However $|x| = n$ and we just showed that $x^{sm} = 1$, 
      so that we have a contradiction. Hence we can conclude that $|x^s| = |t|$.
      \qed
%%%%%%%%%%%%%%%%%%%%%%%%%%%%%%%%%%%%%%1.24%%%%%%%%%%%%%%%%%%%%%%%%%%%%%%%%%%%%%%
   \item[1.1.24]  If $a$ and $b$ are \textit{commuting} elements of $G$, prove 
                  that $(ab)^n = a^nb^n$ for all $n \in \Z$. [Do this by 
                  induction for positive $n$ first.]

      \textbf{Proof.} Let $R(n)$ be the statement that $(ab)^n = a^nb^n$, for
      commuting elements $a$ and $b$.
               
      We now want to show using induction that $R(n)$ holds for every positive 
      integer $n$. It is clear that $R(1)$ is true. So suppose that $R(k)$ is 
      true for some positive integer $k$. We must now show that $R(k + 1)$ is 
      also true. Now we have that
      \begin{align*}
         (ab)^{k+1} &= (ab)^k(ab)^1 &[\text{Exercise 1.1.19}] \\
                    &= a^kb^k(ab)^1 &[\text{Since }R(k) \text{ is true}] \\
                    &= a^kb^k(ba)^1 &[ab = ba] \\
                    &= a^kb^kba \\
                    &= a^kb^{k+1}a \\
                    &= a^kab^{k+1} &[\text{$a$ commutes with $b$}] \\
                    &= a^{k+1}b^{k+1}, \\
      \end{align*}
      so that $R(k + 1)$ holds. It follows by the Principle of Mathematical 
      Induction that $R(n)$ holds for every positive integer $n$. By inpsection 
      we can see that $R(0)$ also holds. To complete the proof, we must now show 
      that $(ab)^{m} = a^mb^m$, where $m$ is a negative integer. First we notice 
      that
      \begin{equation}
         a^{-1}b^{-1} = (ba)^{-1} = (ab)^{-1} = b^{-1}a^{-1},
         \label{1_1_24_1}
      \end{equation}
      so that $a^{-1}$ and $b^{-1}$ are commuting elements. Thus it follows that
      \begin{align*}
         (ab)^m &= (ab)^{-(-m)} \\
                &= [(ab)^{-1}]^{-m} &[\text{Definition}] \\
                &= (a^{-1}b^{-1})^{-m} &[\eqref{1_1_24_1}] \\
                &= (a^{-1})^{-m}(b^{-1})^{-m} &[\text{$R(-m)$ holds}] \\
                &= a^mb^m,
      \end{align*}
      as desired. \qed
%%%%%%%%%%%%%%%%%%%%%%%%%%%%%%%%%%%%%%1.25%%%%%%%%%%%%%%%%%%%%%%%%%%%%%%%%%%%%%%
   \item[1.1.25]  Prove that if $x^2 = 1$ for all $x \in G$ then $G$ is abelian.

      \textbf{Proof.} Let $G$ be a group. Suppose that $x^2 = 1$ for all
      $x \in G$. We want to show that $G$ is abelian; that is, we want to show 
      that $xy = yx$ for all $x, y \in G$. So let $x, y \in G$. By hypothesis, 
      we have that $x^2 = e$, $y^2 = e$, and $(xy)^2 = e$, so that according to 
      Proposition 2, we must have that $x = x^{-1}$, $y = y^{-1}$, and
      $xy = (xy)^{-1}$. Thus
      \begin{align*}
         xy &= (xy)^{-1}      &[\text{By Hypothesis}] \\
            &= y^{-1}x^{-1}   &[\text{Proposition 1}] \\
            &= yx.
      \end{align*}
      Thus $G$ is abelian. \qed
%%%%%%%%%%%%%%%%%%%%%%%%%%%%%%%%%%%%%%1.26%%%%%%%%%%%%%%%%%%%%%%%%%%%%%%%%%%%%%%
   \item[1.1.26]  Assume $H$ is a nonempty subset of $(G, *)$ which is closed 
                  under the binary operation on $G$ and is closed under
                  inverses, i.e., for all $h$ and
                  $k \in H$, $hk$ and $h^{-1} \in H$. Prove that $H$ is a group 
                  under the operation $*$ restricted to $H$ (such a subset $H$
                  is called a subgroup of $G$).

      \textbf{Proof.} We know that $H$ is closed under $*$ and under inverses, 
      so it suffices to show that $*$ is associative on $H$ and that $H$ has an 
      identity under $*$. The associativity of $H$ under $*$ follows because $H$ 
      is a subset of $G$ and $G$ is associative under $*$. Since $H$ is nonempty
      we pick an $h \in H$. Then by hypothesis, we have that
      $1 = hh^{-1} \in H$, so that $H$ contains the identity. (Note that
      $hh^{-1} = h^{-1}h = 1$ and $h1 = 1h = h$ because these equalities hold in
      $G$.) \qed
%%%%%%%%%%%%%%%%%%%%%%%%%%%%%%%%%%%%%%1.27%%%%%%%%%%%%%%%%%%%%%%%%%%%%%%%%%%%%%%
   \item[1.1.27]  Prove that if $x$ is an element of the group $G$ then
                  $\{x^n : n \in \Z\}$ is a subgroup of $G$ (called the
                  \textit{cyclic subgroup} of $G$ generated by $x$).

      \textbf{Proof.} Consider the set
      $$H = \{x^n : n \in \Z\}.$$
      $H$ is nonempty because it contains $1 = x^0$. So let $h_1, h_2 \in H$.
      Thus we have $h_1 = x^a$ and $h_2 = x^b$ for some integers $a$ and $b$, so
      that $h_1h_2 = x^ax^b = x^{a+b} \in H$; in other words, $H$ is closed
      under the operation of $G$. Since $h_1^{-1} = (x^a)^{-1} = x^{-a} \in H$, 
      it follows that $H$ is also closed under inverses, so that $H$ is a
      subgroup of $G$ by Exercise 1.1.26.
%%%%%%%%%%%%%%%%%%%%%%%%%%%%%%%%%%%%%%1.28%%%%%%%%%%%%%%%%%%%%%%%%%%%%%%%%%%%%%%
   \item[1.1.28]  Let $(A, *)$ and $(B, \diamond)$ be groups and let
                  $A \times B$ be their direct product (as defined in Example
                  6). Verify all the group axioms for $A \times B$.
                  \begin{enumerate}
                     \item prove that the associative law holds: for all
                           $(a_i, b_i) \in A \times B, i = 1, 2, 3$
                           $$(a_1, b_1)[(a_2, b_2)(a_3, b_3)] =
                            [(a_1, b_1)(a_2, b_2)](a_3, b_3),$$
                     \item prove that (1, 1) is the identity of $A \times B$,
                           and
                     \item prove that the inverse of $(a, b)$ is
                           $(a^{-1}, b^{-1})$.
                  \end{enumerate}

      \textbf{Proof.} Let $(a_1, b_1)$, $(a_2, b_2)$, and
      $(a_3, b_3) \in A \times B$.

      \begin{enumerate}
         \item The set $A \times B$ is associative under the component wise
               operations of $A$ and $B$ because
               \begin{align*}
                  (a_1, b_1)[(a_2, b_2)(a_3, b_3)]
                     &= (a_1, b_1)(a_2a_3, b_2b_3) \\
                     &= (a_1a_2a_3, b_1b_2b_3) \\
                     &= [(a_1a_2)a_3, (b_1b_2)b_3] &[\text{Associativity}] \\
                     &= (a_1a_2, b_1b_2)(a_3, b_3) \\
                     &= [(a_1, b_1)(a_2, b_2)](a_3, b_3).
               \end{align*}
         \item Consider $(1, 1) \in A \times B$. It follows that
               \begin{align*}
                  (1, 1)(a_1, b_1) &= (1a_1, 1b_1) \\
                                   &= (a_1, b_1) \\
                                   &= (a_11, b_11) \\
                                   &= (a_1, b_1)(1, 1),
               \end{align*}
               so that $(1, 1)$ is the identity of $A \times B$.
         \item Consider $(a, b) \in A \times B$. It 
               follows that
               \begin{align*}
                  (a, b)(a^{-1}, b^{-1}) &= (aa^{-1}, bb^{-1}) \\
                                   &= (1, 1) \\
                                   &= (a^{-1}a, b^{-1}b) \\
                                   &= (a^{-1}, b^{-1})(a, b),
               \end{align*}
               so that $(a^{-1}, b^{-1})$ is the inverse of $(a, b)$.
      \end{enumerate}
%%%%%%%%%%%%%%%%%%%%%%%%%%%%%%%%%%%%%%1.29%%%%%%%%%%%%%%%%%%%%%%%%%%%%%%%%%%%%%%
   \item[1.1.29]  Prove that $A \times B$ is an abelian group if and only if
                  both $A$ and $B$ are abelian.

      \textbf{Proof.} 

      $(\Leftarrow)$ Suppose that $A$ and $B$ are abelian. Let $(a_1, b_1)$ and
      $(a_2, b_2) \in A \times B$. It follows that $A \times B$ is abelian
      because
      \begin{align*}
         (a_1, b_1)(a_2, b_2) &= (a_1a_2, b_1b_2) \\
            &= (a_2a_1, b_2b_1) &[\text{$A$ and $B$ are abelian}] \\
            &= (a_2, b_2)(a_1, b_1).
      \end{align*}

      $(\Rightarrow)$ Now suppose that $A \times B$ is abelian. Let $a_1$ and
      $a_2$ be members of $A$ and let $b_1$ and $b_2$ be members of $B$. Then
      we have that
      \begin{align*}
         (a_1a_2, b_1b_2) = (a_1, b_1)(a_2, b_2) \\
            &= (a_2, b_2)(a_1, b_1) &[\text{$A \times B$ is abelian}] \\
            &= (a_2a_1, b_2b_1),
      \end{align*}
      so that $(a_1a_2, b_1b_2) = (a_2a_1, b_2b_1)$; i.e., $a_1a_2 = a_2a_1$ and
      $b_1b_2 = b_2b_1$. We can now conclude that $A$ and $B$ are both abelian.
      \qed
%%%%%%%%%%%%%%%%%%%%%%%%%%%%%%%%%%%%%%1.30%%%%%%%%%%%%%%%%%%%%%%%%%%%%%%%%%%%%%%
   \item[1.1.30]  Prove that the elements $(a, 1)$ and $(1, b)$ of $A \times B$
                  commute and deduce that the order of $(a, b)$ is the least 
                  common multiple of $|a|$ and $|b|$.

      \textbf{Proof.} Let $A$ and $B$ be groups, and let $a \in A$, $b \in B$.
      We shall be assuming that there exist positive integers $m$ and $n$ such 
      that $|a| = m$ and $|b| = n$, for the problem does not make sense if the
      order of $a$ or $b$ is not finite. Consider $(a, 1)$,
      $(1, b) \in A \times B$. We have that
      \begin{align*}
         (a, 1)(1, b) &= (a1, 1b) \\
                      &= (a, b) \\
                      &= (1a, b1) \\
                      &= (1, b)(a, 1),
      \end{align*}
      so that $(a, 1)$ and $(b, 1)$ commute. To complete the proof, we let
      $s = \text{lcm}(m, n)$. Thus we can write $s = mx = ny$ for positive 
      integers $x$ and $y$. Thus we have that
      \begin{align*}
         (a, b)^s &= (a^s, b^s) &[\text{Lemma 1.1.4}] \\
                  &= (a^{mx}, a^{ny}) \\
                  &= [(a^m)^x, (a^n)^y] \\
                  &= (1^x, 1^y) \\
                  &= (1, 1).
      \end{align*}
      This say that $|(a, b)| \le s$, so there exists a positive integer $q$ 
      such that $|(a, b)| = q$. By Lemma 1.1.4, we have that
      $(a, b)^q = (a^q, b^q) = (1, 1)$, so that $a^q = 1$ and $b^q = 1$. Thus by 
      Lemma 1.1.3, it follows that $m \mid q$ and $n \mid q$, so that $s \mid q$ 
      by definition of the lcm. Since $s \mid q$, we must have that $s \le q$.
      But we previously showed that $q \le s$. Thus we can conclude that
      $s = q$, as desired. \qed
%%%%%%%%%%%%%%%%%%%%%%%%%%%%%%%%%%%%%%1.31%%%%%%%%%%%%%%%%%%%%%%%%%%%%%%%%%%%%%%
   \item[1.1.31]  Prove that any finite group $G$ of even order contains an
                  element of order 2. [Let $t(G)$ be the set
                  $\{g \in G : g \neq g^{-1}\}$. Show that $t(G)$ has an even 
                  number of elements and every nonidentity element of $G - t(G)$ 
                  has order 2.]

      \textbf{Proof.} Let $G$ be a finite group of even order. We wish to show
      that there exists some $g \in G$ such that $|g| = 2$. Consider this subset
      of $G$:
      $$S = \{g \in G: g \neq g^{-1}\}.$$

      If $|S| = 0$, then the proof is done, so assume that $|S| > 0$. Now $|S|$ 
      is even, for if this were not the case, then if we pair up every element
      of $S$ with its inverse, then one element must be without an inverse, a 
      contradiction. Now let $S' = G\backslash S$. It follows that
      $|G| = |S| + |S'|$. Notice that $S'$ is not empty because $e \in S'$. 
      Since $G$ and $S$ are both even, it follows that $|S'|$ must also be even. 
      Since we already showed that $|S'| \ge 1$, we can conclude that
      $|S'| \ge 2$, so that $S'$ contains a non-identity $a$, where
      $a = a^{-1}$. That is, $|a| = 2$. \qed
%%%%%%%%%%%%%%%%%%%%%%%%%%%%%%%%%%%%%%1.32%%%%%%%%%%%%%%%%%%%%%%%%%%%%%%%%%%%%%%
   \item[1.1.32]  If $x$ is an element of finite order $n$ in $G$, prove that
                  the elements 1, $x$, $x^2$, $\ldots$, $x^{n-1}$ are all 
                  distinct. Deduce that $|x| \le |G|$.

      \textbf{Proof.} Suppose that $|x| = n \in \Z^+$ for some $x \in G$. 
      Suppose to the contrary that the elements $x^0$, $x^1$, $x^2$, $\ldots$, 
      $x^{n-1}$ are not distinct. Then we must have that $x^i = x^j$ for some
      integer $i$ and $j$ where $0 \le i < j \le n - 1$. That is, $x^{j-i} = 1$,
      a contradiction because $j - i$ is a positive integer less thatn $n$. It
      follows that the elements $x^0$, $x$, $x^2$, $\ldots$, $x^{n-1}$ are all 
      distinct. Since there are clearly $n$ of these elements and since they are
      all members of $G$, we can conclude that $|x| = n \le |G|$. \qed
%%%%%%%%%%%%%%%%%%%%%%%%%%%%%%%%%%%%%%1.33%%%%%%%%%%%%%%%%%%%%%%%%%%%%%%%%%%%%%%
   \item[1.1.33]  Let $x$ be an element of finite order $n$ in $G$.
                  \begin{enumerate}
                     \item Prove that if $n$ is odd then $x^i \neq x^{-i}$ for
                           all $i = 1, 2, \ldots, n - 1$,
                     \item Prove that if $n = 2k$ and $1 \le i < n$ then
                           $x^i = x^{-i}$ if and only if $i = k$.
                  \end{enumerate}

      \textbf{Proof.}

      \begin{enumerate}
         \item Suppose that $n$ is odd. Now we shall suppose to the contrary
               that $x^i = x^{-i}$ for some integer $1 \le i \le n - 1$. Since
               $x^i = x^{-i}$, it follows that $x^{2i} = 1$. By Lemma 1.1.3, we
               must have that $n \mid 2i$, a contradiction because an odd
               number cannot divide a positive even number, so we conclude that
               $x^i \neq x^{-i}$ for all $i = 1, 2, \ldots, n - 1$. \qed
         \item Suppose that $n$ is even and $1 \le i < n$. Write $n = 2k$ for
               some positive integer $k$.

               $(\Leftarrow)$ Suppose that $i = k$. Then we have that
               $1 = x^{2k} = x^{2i} = x^ix^i$, so that $x^i = x^{-i}$.

               $(\Rightarrow)$ Conversely suppose that $x^i = x^{-i}$, so that
               $x^{2i} =1$. Thus, by Lemma 1.1.3, $2k \mid 2i$, or equivalently,
               $k \mid i$, so that $i = mk$ for some positive integer $m$. 
               Recall that $i < n = 2k$ by hypothesis, so that $mk < 2k$. That 
               is $m < 2$. But $m$ is a positive integer and so the only
               possibility is therefore $m = 1$, so that $i = k$. \qed
      \end{enumerate}
%%%%%%%%%%%%%%%%%%%%%%%%%%%%%%%%%%%%%%1.34%%%%%%%%%%%%%%%%%%%%%%%%%%%%%%%%%%%%%%
   \item[1.1.34]  If $x$ is an element of infinite order in $G$, prove that the
                  elements $x^n$, $n \in \Z$ are all distinct.

      \textbf{Proof.} Assume that $x$ is an element of infinite order in $G$.
      Now suppose to the contrary that $x^i = x^j$ for some unequal integers
      $i$ and $j$. We can further assume without loss of generality that
      $i < j$. Thus $x^{j-i} = 1$, a contradiction because this says that
      $|x| \le j - i$. It follows that distinct integral powers of $x$ yield 
      distinct elements of $G$. \qed
%%%%%%%%%%%%%%%%%%%%%%%%%%%%%%%%%%%%%%1.35%%%%%%%%%%%%%%%%%%%%%%%%%%%%%%%%%%%%%%
   \item[1.1.35]  If $x$ is an element of finite order $n$ in $G$, use the 
                  Division Algorithm to show that any integral power of $x$ 
                  equals one of the elements in the set
                  $\{1, x, x^2, \ldots, x^{n-1}\}$ (so these are all the
                  distinct elements of the cyclic subgroup of $G$ generated by
                  $x$).

      \textbf{Proof.} Assume that $x$ is an element of finite order $n$ in $G$.
      Let $z \in \Z$. By the Division Algorithm, there exist unique integers
      $q$ and $r$ such that $z = qn + r$ and $0 \le r < n$. That is
      $$x^z = x^{qn+r} = x^{qn}x^r = (x^n)^qx^r = 1^qx^r = x^r.$$
      Since $r \in \{0, 1, \ldots, n - 1\}$ and since $x^z = x^r$, it follows
      that $x^z \in \{x^0, x^1, \ldots, x^{n-1}\}$. \qed
%%%%%%%%%%%%%%%%%%%%%%%%%%%%%%%%%%%%%%1.36%%%%%%%%%%%%%%%%%%%%%%%%%%%%%%%%%%%%%%
   \item[1.1.36]  Assume $G = \{1, a, b, c\}$ is a group of order 4 with
                  identity 1. Assume also that $G$ has no elements of order 4
                  (so by Exercise 32, every element has order $\le$ 3). Use the
                  cancellation laws to show that there is a unique group table
                  for $G$. Deduce that $G$ is abelian.

      \textbf{Proof.} Assume $G = \{1, a, b, c\}$. We can tentatively fill out 
      the group table for $G$ like so:      
      $$
         \begin{tabular}{@{}c | c | c | c | c@{}} 
                & $1$ & $a$ & $b$ & $c$ \\ \hline
            $1$ & $1$ & $a$ & $b$ & $c$ \\ \hline
            $a$ & $a$ & $ $ & $ $ & $ $ \\ \hline
            $b$ & $b$ & $ $ & $ $ & $ $ \\ \hline
            $c$ & $c$ & $ $ & $ $ & $ $
         \end{tabular}
      $$
      By the left cancellation law, the equality $ab = a$ will result in $b = e$
      and the equality $ab = b$ will result in $a = e$, both of which are
      contradictions. The only remaining possiblities are $ab = c$ or $ab = 1$.

      \textbf{Case 1:} $ab = c$. For the same reason as above, we cannot have
      $ac = a$ or $ac = c$, so that $ac = 1$ or $ac = b$. So suppose first that
      $ac = b$. Then our table will now look like so:
      $$
         \begin{tabular}{@{}c | c | c | c | c@{}} 
                & $1$ & $a$ & $b$ & $c$ \\ \hline
            $1$ & $1$ & $a$ & $b$ & $c$ \\ \hline
            $a$ & $a$ & $ $ & $c$ & $b$ \\ \hline
            $b$ & $b$ & $ $ & $ $ & $ $ \\ \hline
            $c$ & $c$ & $ $ & $ $ & $ $
         \end{tabular}
      $$
      From the table above, we see that $aa$ must be equal to 1, since that is
      the only remaining possibility. The cancellation laws tell us that every
      element in a column and row of a group table must be unique, so we must
      have that:
      $$
         \begin{tabular}{@{}c | c | c | c | c@{}} 
                & $1$ & $a$ & $b$ & $c$ \\ \hline
            $1$ & $1$ & $a$ & $b$ & $c$ \\ \hline
            $a$ & $a$ & $1$ & $c$ & $b$ \\ \hline
            $b$ & $b$ & $c$ & $ $ & $ $ \\ \hline
            $c$ & $c$ & $b$ & $ $ & $ $
         \end{tabular}
      $$
      Note that we cannot have $bb = a$ because that would imply that $bbb = c$,
      so that $|b| > 3$, contradicting our hypothesis. Thus we must have that
      $bb = 1$. The remaining positions are thus completely determined, so that
      we have
      $$
         \begin{tabular}{@{}c | c | c | c | c@{}} 
                & $1$ & $a$ & $b$ & $c$ \\ \hline
            $1$ & $1$ & $a$ & $b$ & $c$ \\ \hline
            $a$ & $a$ & $1$ & $c$ & $b$ \\ \hline
            $b$ & $b$ & $c$ & $1$ & $a$ \\ \hline
            $c$ & $c$ & $b$ & $a$ & $1$
         \end{tabular}
      $$
      Now suppose that $ac = 1$, then we would be forced to fill in the table
      like so:
      $$
         \begin{tabular}{@{}c | c | c | c | c@{}} 
                & $1$ & $a$ & $b$ & $c$ \\ \hline
            $1$ & $1$ & $a$ & $b$ & $c$ \\ \hline
            $a$ & $a$ & $b$ & $c$ & $1$ \\ \hline
            $b$ & $b$ & $c$ & $ $ & $ $ \\ \hline
            $c$ & $c$ & $1$ & $ $ & $ $
         \end{tabular}
      $$
      Since $a^2 = b$ and $a^3 = c$, we have that $|a| > 3$, contradicting our
      hypothesis, so this is a dead end.

      \textbf{Case 2:} $ab = 1$. For the same reason as above, we cannot have
      $ac = a$ or $ac = c$, so that $ac = 1$ or $ac = b$. So suppose first that
      $ac = b$. Then our table will now look like so:
      $$
         \begin{tabular}{@{}c | c | c | c | c@{}} 
                & $1$ & $a$ & $b$ & $c$ \\ \hline
            $1$ & $1$ & $a$ & $b$ & $c$ \\ \hline
            $a$ & $a$ & $c$ & $1$ & $b$ \\ \hline
            $b$ & $b$ & $1$ & $c$ & $a$ \\ \hline
            $c$ & $c$ & $b$ & $a$ & $1$
         \end{tabular}
      $$
      Since $a^2 = c$ and $a^3 = b$, we have that $|a| > 3$, contradicting our
      hypothesis, so this is another dead end. From our arguments above, we see
      that the only viable and legal table is thus:
      $$
         \begin{tabular}{@{}c | c | c | c | c@{}} 
                & $1$ & $a$ & $b$ & $c$ \\ \hline
            $1$ & $1$ & $a$ & $b$ & $c$ \\ \hline
            $a$ & $a$ & $1$ & $c$ & $b$ \\ \hline
            $b$ & $b$ & $c$ & $1$ & $a$ \\ \hline
            $c$ & $c$ & $b$ & $a$ & $1$
         \end{tabular}
      $$
      This table is unique, and since it is symmeteric it follows that $G$ is
      abelian. \qed
\end{enumerate}

      \section{Examples: Polynomial Rings, Matrix Rings, And Group Rings}
         \begin{enumerate}
%%%%%%%%%%%%%%%%%%%%%%%%%%%%%%%%%%%Prob1.2_1%%%%%%%%%%%%%%%%%%%%%%%%%%%%%%%%%%%%
   \item[1.2.1]   For each of the following statements, determine whether it is 
                  true or false and justify your answer.
                  \begin{enumerate}
                     \item The set $\Z$ of integers is dense in $\R$.
                     \item The set of positive real numbers is dense in $\R$.
                     \item The set $\Q\backslash \Z$ of rational numbers that 
                           are not integers is dense in $\R$.
                  \end{enumerate}  

      \textbf{Solution:} 

      \begin{enumerate}
         \item False. Proposition 1.6 states that there is no integer in the
               interval (0, 1).
         \item False. The interval $(-1, 0)$ contains no positive real number.
         \item True. Let $a$ and $b$ be real numbers. Then we shall investigate
               the following two cases:
               
               \textbf{Case I:} $a < a + 1 \le b$. Theorem 1.8 says that there
               exists a unique integer $k$ in $[a, a + 1)$. Thus there is no
               integer in the interval $(k, a + 1)$. By the density of $\Q$ in
               $\R$, there exists a rational $q \in (k, a + 1)$. Since
               $(k, a + 1)$ contains no integer, then $q$ must be a member of
               $\Q\backslash\Z$. We observe that $q \in (a, b)$.
               
               \textbf{Case II:} $a < b < a + 1$. Theorem 1.8 says that there
               exists a unique integer $k$ in $[a, a + 1)$. If $k  \le b$, then
               $(a, k)$ has no integer, so there exists a noninteger rational
               in $(a, k) \subseteq (a, b)$ by the density of $\Q$ in $\R$. If,
               however, $k > b$, then the interval $(a, b)$ contains no integer,
               so that there exists a noninteger rational in $(a, b)$ by the
               density of $\Q$ in $\R$.
      \end{enumerate}
%%%%%%%%%%%%%%%%%%%%%%%%%%%%%%%%%%Prob1.2_2%%%%%%%%%%%%%%%%%%%%%%%%%%%%%%%%%%%%%
   \item[1.2.2]   Suppose that $S$ is a nonempty set of integers that is bounded
                  below. Show that $S$ has a minimum. In particular, conclude
                  that every nonempty set of natural numbers has a minimum.  

      \textbf{Proof:}

      Let $S$ be a nonempty set of integers bounded below. Then there exists
      some $r \in \R$ such that for every $a \in S$, we have that $r \le a$.
      Consider the set $S' = \{-s: s \in S\}$, the set of the additive inverses
      of the elements of $S$. Note that $S'$ is also a nonempty set of integers.
      So let $-d \in S'$ where $d \in S$. Hence $r \le d$, so that $-d \le -r$;
      that is $S'$ is bounded above. By Proposition 1.7 $S'$ has a maximum, say
      $-b$, where $b \in S$. It suffices to show that $b$ is the minimum in $S$.
      Let $c \in S$. Then we have that $-c \le -b$, so that $b \le c$; that is,
      $b$ is the minimum element in $S$. In paritcular, we can see that the
      Well Ordering Principle follows. \qed
%%%%%%%%%%%%%%%%%%%%%%%%%%%%%%%%%%Prob1.2_3%%%%%%%%%%%%%%%%%%%%%%%%%%%%%%%%%%%%%
   \item[1.2.3]   Let $S$ be a nonempty set of real numbers that is bounded
                  below. Prove that the set $S$ has a minimum if and only if the
                  number $\inf S$ belongs to $S$.
			
		\textbf{Proof:} Let $S$ be a nonempty set of real numbers that is bounded
      below.

      $(\Leftarrow)$ Suppose $\inf S$ belongs in $S$; then it immediately
      follows by definition that $\inf S$ is the minimum element of $S$. \\
      $(\Rightarrow)$ Now suppose that $S$ has a minimum, say $s$. By the
      Completeness Axiom, we have that $\inf S$ exists; since $s \in S$, we must
      have that $\inf S \le s$. But $s$ is also a lower bound for $S$ and since
      every lower bound of $S$ cannot exceed $\inf S$, we must have that
      $s \le \sup S$; we have shown that $\inf S \le s$ and $s \le \inf S$ so
      that $s = \inf S$. \qed
%%%%%%%%%%%%%%%%%%%%%%%%%%%%%%%%%%Prob1.2_4%%%%%%%%%%%%%%%%%%%%%%%%%%%%%%%%%%%%%
   \item[1.2.4]   For each of the following two sets, find the maximum, minimum,
                  infimum, and supremum if they are defined. Justify your
                  conclusions.
                  \begin{enumerate}
                     \item $S = \{1/n : n \in \N\}$.
                     \item $T = \{x \in \R : x^2 < 2\}$.
                  \end{enumerate}

      \textbf{Solution:}

      \begin{enumerate}
         \item The \textbf{maximum} is 1. To show this consider any natural
               number $n$; then we have $n \ge 1$. Multiply this inequality by
               the positive number $1/n$ to give us $1/n \le 1$. Since
               $1 = 1/1 \in S$, we are done. $S$ has no \textbf{minimum}. Assume
               by way of contradiction that $\min S$ exists. Then by definition
               of $S$, we know that $\min S$ must be positive. So by the
               Archimedean Property, there exists a natural number $n_1$(so that
               $1/n_1 \in S$) such that $1/n_1 < \min S$, a contradiction. So
               $\min S$ doesn't exist. Since the Archimedean Property enables us
               to find a member of $S$ that is less than any positive number, no
               positive number can be a lower bound for $S$. Thus $S$ can only 
               be bounded below by negative numbers and 0. It follows that the
               \textbf{infimum} of  $S$ is 0. Since $S$ has a maximum, this
               maximum. By Problem 1.1.15, we have that the \textbf{supremum} of
               $S = 1$. If we consider $-T$, the set of the additive inverses of
               the elements of $T$.
         \item It is trivial to show that%%%%%%%%%%%%%%%%%%%%%%%%%%%%%%%%%%%%%%%%%%%%%%%%%%%%Show true
               $T = \{x \in \R: -\sqrt{2} < x < \sqrt{2}\}$. We claim that the
               \textbf{infimum} and \textbf{supremum} of $T$ are $-\sqrt{2}$ and
               $\sqrt{2}$. Suppose by contradiction that this is false; then 
               there exist $a > -\sqrt{2}$ and $b < \sqrt{2}$ such that $a$ and 
               $b$ are the infimum and supremum of $T$. Then by the density of
               $\Q$ in $\R$, there exist rationals $p$ and $q$ such that
               $-\sqrt{2} < p < a$ and $b < q < \sqrt{2}$; that is, $p$ and $q$ 
               are members of $T$. But since $p$ is less than $a$ and $q > b$, 
               we have contradictions. Thus our claim holds. Since the infimum 
               and supremum are not members of $T$, it follows that $T$ has 
               neither a \textbf{maximum} nor a \textbf{minimum}.
      \end{enumerate}
%%%%%%%%%%%%%%%%%%%%%%%%%%%%%%%%%%Prob1.2_5%%%%%%%%%%%%%%%%%%%%%%%%%%%%%%%%%%%%%
   \item[1.2.5]   Suppose that the number $a$ has the property that for every
                  natural number $n$, $a \le 1/n$. Prove that $a \le 0$.

      \textbf{Proof:} Assume by way of contradiction that $a > 0$. By The
      Archimedean Property there exists a natural number $k$ such that
      $a > 1/k$, a contradiction. Thus $a \le 0$. \qed

%%%%%%%%%%%%%%%%%%%%%%%%%%%%%%%%%%Prob1.2_6%%%%%%%%%%%%%%%%%%%%%%%%%%%%%%%%%%%%%
   \item[1.2.6]   Given a real number $a$, define
                  $S \equiv \{x : x \in \Q, x < a\}$. Prove that $a = \sup S$.

      \textbf{Proof:} By the density of $\Q$ in $\R$, there exists a rational
      $q \in (a - 1, a)$, so that $q \in S$. Thus $S$ is nonempty. By 
      definition, $S$ is bounded above by $a$; since $S$ is also nonempty, the
      Completeness Axiom says that $\sup S$ exists. So we must have that
      $\sup S \le a$. Now suppose that $\sup S < a$, then the density of $\Q$ in
      $\R$ guarantees that we have a rational $q$ in $(\sup S, a)$, so that $q$
      is also a member of $S$, a contradiction since we cannot have a member of
      $S$ that is greater than $\sup S$. Thus $a = \sup S$. \qed

%%%%%%%%%%%%%%%%%%%%%%%%%%%%%%%%%%Prob1.2_7%%%%%%%%%%%%%%%%%%%%%%%%%%%%%%%%%%%%%
   \item[1.2.7]   Show that for any real number $c$, there is exactly one 
                  integer in the interval $(c, c+1]$.

      \textbf{Proof:} Let $c$ be a real number. According to Theorem 1.8, there
      exists a unique integer $k$ in the interval $[-(c + 1), -c)$. So we have
      $-(c + 1) \le k < -c$, so that $c < -k \le c + 1$. Hence we have an
      integer $-k$ in the interval $(c, c + 1]$. We can see that $-k$ is unique
      because if another integer $h$ exists in $(c, c + 1]$, then $-h$ would
      also be in $[-(c + 1), -c)$, and since $k$ is unique, we must have
      $-h = k$, so that $h = -k$. \qed
   
%%%%%%%%%%%%%%%%%%%%%%%%%%%%%%%%%%Prob1.2_8%%%%%%%%%%%%%%%%%%%%%%%%%%%%%%%%%%%%%
   \item[1.2.8]   Show that the Archimedean Property is a consequence of the
                  assertion that for any real number $c$, there is an integer in
                  the interval $[c, c + 1)$.

      \textbf{Proof:} Let $\epsilon$ be a positive real number. It suffices to
      show that there exists a natural number greater than $\epsilon$. By our
      assertion, there exists an integer $k$ in the interval
      $[\epsilon+ 1, \epsilon + 2)$. So we have $\epsilon < \epsilon + 1 \le k$,
      so that $k$ is a positive integer. \qed
%%%%%%%%%%%%%%%%%%%%%%%%%%%%%%%%%%Prob1.2_9%%%%%%%%%%%%%%%%%%%%%%%%%%%%%%%%%%%%%
   \item[1.2.9]   Show that the Archimedean Property is a consequence of the
                  assertion that every interval $(a, b)$ contains a rational
                  number.

      \textbf{Proof:} Let $\epsilon$ be a positive real number. It suffices to
      show that there exists a natural number greater than $\epsilon$. By our
      assertion, there exist positive integers $p$ and $q$ such that
      $p/q \in (\epsilon, \epsilon + 1)$. Since $p$ and $q$ are positive, we 
      have that $q \ge 1$ so that $pq \ge p$; that is $p/q \le p$. We have now
      shown that $\varepsilon < p/q \le p$. Particularly $p > \varepsilon$,
      which is what we wanted to prove. \qed

      
\end{enumerate}

      \section{Ring Homomorphisms and Quotient Rings}
         \begin{enumerate}
%%%%%%%%%%%%%%%%%%%%%%%%%%%%%%%%%%%%%2.3.1%%%%%%%%%%%%%%%%%%%%%%%%%%%%%%%%%%%%%%
   \item[2.3.1]   Find all subgroups of $Z_{45} = \cyc{x}$, giving a generator
                  for each. Describe the containments between these subgroups.
                  
      \textbf{Solution.} Since the positive divisors of 45 are: 1, 3, 5, 9, 15,
      and 45, it follows that the subgroups of $Z_{45}$ are
      $$\cyc{x}, \cyc{x^3}, \cyc{x^5}, \cyc{x^9}, \cyc{x^{15}}, \text{ and }
        \cyc{x^{45}}.$$
        
      We have the following containments:
      $$
         \begin{tabular}{>{$}c<{$}>{$}c<{$}>{$}c<{$}>{$}c<{$}>{$}c<{$}>{$}c<{$}>{$}c<{$}}
            \cyc{x^{45}} & \le & \cyc{x^{15}} & \le & \cyc{x^5} & \le & \cyc{x} \\
            \cyc{x^{15}} & \le &  \cyc{x^3} & \le & \cyc{x} \\
            \cyc{x^9} & \le &  \cyc{x^3} & \le & \cyc{x}
         \end{tabular}
      $$
%%%%%%%%%%%%%%%%%%%%%%%%%%%%%%%%%%%%%2.3.2%%%%%%%%%%%%%%%%%%%%%%%%%%%%%%%%%%%%%%
   \item[2.3.2]   If $x$ is an element of the finite group $G$ and $|x| = |G|$,
                  prove that $G = \cyc{x}$. Give an explicit example to show 
                  that this result need not be true if $G$ is an infinite group.
                  
      \textbf{Proof.} Let $G$ be a finite group, so that $|G| = n \in \Z^+$.
      Suppose that there exists $x \in G$ such that $|x| = n$. Clearly
      $\cyc{x} \subseteq G$. But $|\cyc{x}| = n$ since $|x| = n$; thus
      $G \subseteq \cyc{x}$ so that $G = \cyc{x}$. Now let $G = \Z$. We have
      that $|\cyc{2}| = |G|$ but $G \neq \cyc{2}$. \qed
%%%%%%%%%%%%%%%%%%%%%%%%%%%%%%%%%%%%%2.3.3%%%%%%%%%%%%%%%%%%%%%%%%%%%%%%%%%%%%%%
   \item[2.3.3]   Find all generators for $\Z/48\Z$.
   
      \textbf{Solution.} The generators for $\Z/48\Z$ are: $\cyc{\overline{1}}$,
      $\cyc{\overline{5}}$, $\cyc{\overline{7}}$, $\cyc{\overline{11}}$,
      $\cyc{\overline{13}}$, $\cyc{\overline{17}}$, $\cyc{\overline{19}}$,
      $\cyc{\overline{23}}$, $\cyc{\overline{25}}$, $\cyc{\overline{29}}$,
      $\cyc{\overline{31}}$, $\cyc{\overline{35}}$, $\cyc{\overline{37}}$,
      $\cyc{\overline{41}}$, $\cyc{\overline{43}}$, and $\cyc{\overline{47}}$.
%%%%%%%%%%%%%%%%%%%%%%%%%%%%%%%%%%%%%2.3.4%%%%%%%%%%%%%%%%%%%%%%%%%%%%%%%%%%%%%%
   \item[2.3.4]   Find all generators for $\Z/202\Z$.
   
      \textbf{Solution.} Let $S$ be the set of generators for $\Z/202\Z$. Then
      $|S| = 100$ since
      $$S = \{\cyc{x} : x \text{ is odd and positive}, x \neq 101, \text{ and } x < 202\}.$$
%%%%%%%%%%%%%%%%%%%%%%%%%%%%%%%%%%%%%2.3.5%%%%%%%%%%%%%%%%%%%%%%%%%%%%%%%%%%%%%%
   \item[2.3.5]   Find the number of generators for $\Z/49000\Z$.
   
      \textbf{Solution.} For a positive integer $n$ let $\varphi(n)$ be the
      number of positive integers---less than or equal to $n$---that are
      relatively prime to $n$. Then the number of generators for $\Z/49000\Z$ is
      $\varphi(49000) = \varphi(2^35^37^2) =
      \varphi(2^3)\varphi(5^3)\varphi(7^2) = 16800$. 
%%%%%%%%%%%%%%%%%%%%%%%%%%%%%%%%%%%%%2.3.6%%%%%%%%%%%%%%%%%%%%%%%%%%%%%%%%%%%%%%
   \item[2.3.6]   In $\Z/48\Z$ write out all elements of $\cyc{\overline{a}}$
                  for every $\overline{a}$. Find all inclusions between
                  subgroups in $\Z/48\Z$.
      
      \textbf{Solution.}
      $$
         \begin{tabular}{|c|c|} \hline
            \textbf{Generators} & \textbf{Subgroups in} $\Z/48\Z$ \\ \hline
            0 & $\{0\}$ \\ \hline
            24 & $\{0, 24\}$ \\ \hline
            16, 32 & $\{0, 16, 32\}$ \\ \hline
            12, 36 & $\{0, 12, 24, 36\}$ \\ \hline
            8, 40 & $\{0, 8, 16, 24, 32, 40\}$ \\ \hline
            6, 18, 30, 42 & $\{0, 6, 12, 18, 24, 30, 36, 42\}$ \\ \hline
            4,20,28,44 & $\{0,4,8,12,16, 20, 24, 28, 32, 36, 40, 44\}$ \\ \hline
            3, 9, 15, 21, 27, 33, 39, 45 & $\{0, 3, 6, 9, 12, 15, 18, 21, 24,
            27, 30, 33, 36, 39, 42, 45\}$ \\ \hline            
            2, 10, 14, 22, 26, 34, 38, 46 & $\{x : 0 \le x \le 46,
            x \text{ is even}\}$ \\ \hline
            \text{See Exercise } 2.3.3 & $\Z/48\Z$ \\ \hline
         \end{tabular}
      $$
%%%%%%%%%%%%%%%%%%%%%%%%%%%%%%%%%%%%%2.3.7%%%%%%%%%%%%%%%%%%%%%%%%%%%%%%%%%%%%%%
   \item[2.3.7]   Let $Z_{48} = \cyc{x}$ and use the isomorphism
                  $\Z/48\Z \cong Z_{48}$ given by $\overline{1} \mapsto x$ to
                  list all subgroups of $Z_{48}$ as computed in the preceding
                  exercise.
                  
      \textbf{Solution.}
      $$
         \begin{tabular}{|c|} \hline
            \textbf{Subgroups in} $Z_{48}$ \\ \hline
            $\{1\}$ \\ \hline
            $\{1, x^{24}\}$ \\ \hline
            $\{1, x^{16}, x^{32}\}$ \\ \hline
            $\{1, x^{12}, x^{24}, x^{36}\}$ \\ \hline
            $\{1, x^8, x^{16}, x^{24}, x^{32}, x^{40}\}$ \\ \hline
            $\{1, x^6, x^{12}, x^{18}, x^{24}, x^{30},x^{36},x^{42}\}$ \\ \hline
            $\{1,x^4,x^8,x^{12},x^{16}, x^{20}, x^{24}, x^{28}, x^{32}, x^{36},
               x^{40}, x^{44}\}$ \\ \hline
            $\{1, x^3, x^6, x^9, x^{12}, x^{15}, x^{18}, x^{21}, x^{24},
            x^{27}, x^{30}, x^{33}, x^{36}, x^{39}, x^{42}, x^{45}\}$ \\ \hline
            $\{x^y : 0 \le y \le 46, y \text{ is even}\}$ \\ \hline
            $Z_{48}$ \\ \hline
         \end{tabular}
      $$
%%%%%%%%%%%%%%%%%%%%%%%%%%%%%%%%%%%%%2.3.8%%%%%%%%%%%%%%%%%%%%%%%%%%%%%%%%%%%%%%
   \item[2.3.8]   Let $Z_{48} = \cyc{x}$. For which integers $a$ does the map
                  $\varphi_a$ defined by $\varphi_a : \overline{1} \mapsto x^a$
                  extend to an \textit{isomorphism} from $\Z/48\Z$ onto
                  $Z_{48}$.
                  
      \textbf{Solution.} Suppose that $(a, 48) = 1$. Then it follows that $x^a$
      generates $Z_{48}$. Thus $\varphi_a$ is an isomorphism by Theorem 4 (Page
      56). Now suppose that $a$ is not relatively prime to 48. Then $x^a$ does
      not generate $Z_{48}$, so that the image of $\varphi_a$ is not $Z_{48}$.
      Hence $\varphi_a$ is an isomorphism if and only if $(a, 48) = 1$.
%%%%%%%%%%%%%%%%%%%%%%%%%%%%%%%%%%%%%2.3.9%%%%%%%%%%%%%%%%%%%%%%%%%%%%%%%%%%%%%%
   \item[2.3.9]   Let $Z_{36} = \cyc{x}$. For which integers $a$ does the map
                  $\psi_a$ defined by $\psi_a : \overline{1} \mapsto x^a$ extend
                  to a \textit{well defined homomorphism} from $\Z/48\Z$ into
                  $Z_{36}$. Can $\psi_a$ ever be a surjective homomorphism?
                  
      \textbf{Solution.} First we shall find the restriction(s) on $a$ such that
      $\psi_a$ is well defined. Suppose $b = c$ for some $b, c \in \Z/48\Z$. It
      suffices to show that $\psi_a(b) = \psi_a(c)$. Since $b = c$, there exists
      an integer $k$ such that $b = c + 48k$. Thus $\psi_a(b) = \psi_a(c+48k)$,
      so that
      $\psi_a(b)=(x^a)^{c+48k}=x^{ac + 48ak}= x^{ac}x^{48ak}=\psi_a(c)x^{12ak}$.
      So we must require $x^{12ak} = 1$ for all $k \in \Z$. Now $x^{12ak} = 1$
      for all $k \in \Z$ if and only if $3 \mid a$ if and only if $\psi_a$ is
      well defined. It follows immediately that
      $\psi_a$ is an homomorphism since
      \begin{align*}
         \psi_a(p + q) &= (x^a)^{p+q} \\
            &= x^{ap+aq} \\
            &= x^{ap}x^{aq} \\
            &= (x^a)^p(x^a)^q \\
            &= \psi_a(p)\psi_a(q)
      \end{align*}      
      for all $p, q \in \Z/48\Z$.
      
      \textit{Can $\psi_a$ ever be a surjective homomorphism?} No!
      
      \textbf{Proof.} Suppose to the contrary that $\psi_a$ is surjective. Then
      there exists $y \in \Z/48\Z$ such that $\psi_a(y) = x$. That is
      $x^{ay} = x$, so that $x^{ay-1} = 1$; thus $ay - 1 = 36m$ for some integer
      $m$. Rearrange the equality $ay - 1 = 36m$ to get $1 = ay - 36m$. Recall
      that $3 \mid a$; since $3$ also divides 36, it follows that 3 must divide
      1, a contradiction. Thus $\psi_a$ can never be surjective. \qed
%%%%%%%%%%%%%%%%%%%%%%%%%%%%%%%%%%%%%2.3.10%%%%%%%%%%%%%%%%%%%%%%%%%%%%%%%%%%%%%
   \item[2.3.10]  What is the order of $\overline{30}$ in $\Z/54\Z$? Write out
                  all the elements and their orders in $\cyc{\overline{30}}$.
                  
      \textbf{Solution.} The order of $30$ in $\Z/54\Z$ is
      $$\frac{54}{(30, 54)} = 9.$$
      The elements of $\cyc{30}$ and their respective orders are:
      $$
         \begin{tabular}{|c|c|} \hline
            Element of $\cyc{30}$ & Order \\ \hline
            30 & 9 \\ \hline
             6 & 9 \\ \hline
            36 & 3 \\ \hline
            12 & 9 \\ \hline
            42 & 9 \\ \hline
            18 & 3 \\ \hline
            48 & 9 \\ \hline
            24 & 9 \\ \hline
             0 & 1 \\ \hline
         \end{tabular}
      $$
%%%%%%%%%%%%%%%%%%%%%%%%%%%%%%%%%%%%%2.3.11%%%%%%%%%%%%%%%%%%%%%%%%%%%%%%%%%%%%%
   \item[2.3.11]  Find all cyclic subgroups of $D_8$. Find a proper subgroup of
                  $D_8$ which is not cyclic.
                  
      \textbf{Solution.} In $D_8$, only $r$ and $r^4$ have order 4. Thus
      $\{1, r, r^2, r^3\}$ is the only cyclic subgroup of order 4. The trivial
      subgroup is the only cyclic subgroup of order 1. Finally there are 5
      cyclic subgroups of order 2 and they are of the form $\{1, x\}$ where
      $x \in \{r^2, s, sr, sr^2, sr^3\}$. The set $\{1, s, r^2, sr^2\}$ is a
      non-cyclic proper subgroup of $D_8$.
%%%%%%%%%%%%%%%%%%%%%%%%%%%%%%%%%%%%%2.3.12%%%%%%%%%%%%%%%%%%%%%%%%%%%%%%%%%%%%%
   \item[2.3.12]  Prove that the following groups are \textit{not} cyclic:
                  \begin{enumerate}
                     \item $Z_2 \times Z_2$
                     \item $Z_2 \times \Z$
                     \item $\Z \times \Z$.
                  \end{enumerate}
      
      \textbf{Proof.}
      \begin{enumerate}
         \item The order of $Z_2 \times Z_2$ is 4, but no element in this group
               has order 4; thus $Z_2 \times Z_2$ is not cyclic.
         \item Let $Z_2 = \cyc{x}$. Observe that $Z_2 \times \Z$ is not finite,
               so in order for it to be cyclic it must be isomorphic to $\Z$.
               But this is not the case since $Z_2 \times \Z$ has two elements
               of finite order(namely $(1, 0)$ and $(x, 0)$) while $\Z$ has
               exactly 1 element of finite order.
         \item Suppose to the contrary that $\Z \times \Z$ is cyclic. Then there
               exist nonzero integers $a$ and $b$ such that
               $$\Z \times \Z = \cyc{(a,b)} = \{(na, nb) : n \in \Z\}.$$
               Thus there exists an integer $m$ such that
               $(ma, mb) = (0, 1)$. That is, $ma = 0$ and $mb = 1$. Since
               $ma = 0$, we must have $m = 0$ or $a = 0$. If $m$ is 0, then
               $(ma, mb) = (0, 0) \neq (0, 1)$, a contradiction; thus we must
               have $a = 0$, contradicting our assumption that $a$ is nonzero.
               Thus $\Z \times \Z$ is not cyclic.
      \end{enumerate} \qed
%%%%%%%%%%%%%%%%%%%%%%%%%%%%%%%%%%%%%2.3.13%%%%%%%%%%%%%%%%%%%%%%%%%%%%%%%%%%%%%
   \item[2.3.13]  Prove that the following pairs of groups are \textit{not}
                  isomorphic:
                  \begin{enumerate}
                     \item $\Z \times Z_2$ and $\Z$
                     \item $\Q \times Z_2$ and $\Q$.
                  \end{enumerate}
      
      \textbf{Proof.}
      \begin{enumerate}
         \item By Exercise 1.6.11, we know that $\Z \times Z_2$ is isomorphic to
               $Z_2 \times \Z$. By Exercise 2.3.12, $Z_2 \times \Z$ is not
               cyclic; thus $\Z \times Z_2$ is not cyclic. That is,
               $\Z \times Z_2$ is not isomorphic to $\Z$.
         \item Let $Z_2 = \cyc{x}$. It immediately follows that
               $\Q \times Z_2$ and $\Q$ are not isomorphic since $\Q \times Z_2$
               has two elements of finite order(namely $(0, 1)$ and $(0, x)$)
               while $\Q$ has exactly 1 element of finite order.
      \end{enumerate} \qed
%%%%%%%%%%%%%%%%%%%%%%%%%%%%%%%%%%%%%2.3.14%%%%%%%%%%%%%%%%%%%%%%%%%%%%%%%%%%%%%
   \item[2.3.14]  Let $\sigma =$ (1 2 3 4 5 6 7 8 9 10 11 12). For each of the
                  following integers $a$ compute $\sigma^a$:
                  $$a = 13, 65, 626, 1195, -6, -81, -570,\text{ and } {-1211}.$$
                  
      \textbf{Solution.}
      
      \begin{alignat*}{4}
         &\sigma^{13}   &&= \sigma &&\text{ } \\
         &\sigma^{65}   &&= \sigma^5 &&=
            (1\;6\;11\;4\;9\;2\;7\;12\;5\;10\;3\;8) \\
         &\sigma^{626}  &&= \sigma^2 &&= (1\;3\;5\;7\;9\;11) \\
         &\sigma^{1195} &&= \sigma^7 &&=
            (1\;8\;3\;10\;5\;12\;7\;2\;9\;4\;11\;6\;13) \\
         &\sigma^{-6} &&= \sigma^6 &&= (1\;7)
            (1\;8\;3\;10\;5\;12\;7\;2\;9\;4\;11\;6\;13) \\
         &\sigma^{-81} &&= \sigma^3 &&= (1\;4\;7\;10) \\
         &\sigma^{-570} &&= \sigma^6 &&= (1\;7) \\
         &\sigma^{-1211} &&= \sigma
      \end{alignat*}
%%%%%%%%%%%%%%%%%%%%%%%%%%%%%%%%%%%%%2.3.15%%%%%%%%%%%%%%%%%%%%%%%%%%%%%%%%%%%%%
   \item[2.3.15]  Prove that $\Q \times \Q$ is not cyclic.
   
      \textbf{Proof.} Since $\Q$ is infinite and, by Exercise 1.6.6, $\Q$ is not
      isomorphic to $\Z$, it follows that $\Q$ is not cyclic. We know that the
      subgroup of every cyclic group is cyclic; since $\Q \times\{1\} \cong \Q$,
      it follows that $\Q \times \{1\}$ is not cyclic; thus $\Q \times \Q$ is
      not cyclic because it has a noncyclic subgroup, namely $\Q \times \{1\}$.
      \qed
%%%%%%%%%%%%%%%%%%%%%%%%%%%%%%%%%%%%%2.3.16%%%%%%%%%%%%%%%%%%%%%%%%%%%%%%%%%%%%%
   \item[2.3.16]  Assume $|x| = n$ and $|y| = m$. Suppose that $x$ and $y$
                  \textit{commute}: $xy = yx$. Prove that $|xy|$ divides the
                  least common multiple of $m$ and $n$. Need this be true if $x$
                  and $y$ do \textit{not} commute? Give an example of commuting
                  elements $x$, $y$ such that the order of $xy$ is not equal to
                  the least common multiple of $|x|$ and $|y|$.
                  
      \textbf{Proof.} Let $l = \text{lcm}(m, n)$. So there exist integers
      $m'$ and $n'$ such that $mm' = nn' = l$. So we have that
      $$(xy)^l = x^ly^l = x^{nn'}y^{mm'} = (x^n)^{n'}(y^m)^{m'} = 1.$$
      That is $|xy|$ divides $l$ (by Proposition 3, Page 55).
      
      \textit{Need this be true if $x$ and $y$ do not commute?} No! Let
      $$
         A = \left(\begin{tabular}{@{}cc@{}}
            0 & 1/2 \\
            2 & 0
         \end{tabular}\right) \text{ and }
         B = \left(\begin{tabular}{@{}cc@{}}
            0 & 1 \\
            1 & 0
         \end{tabular}\right).
      $$
      A simple computation will show us that although $|A| = |B| = 2$, we have
      that $|AB| = \infty$.
      
      \textbf{Example.} Consider $\Z/2\Z = \{0, 1\}$. Let $x = y = 1$. Then we
      have $|x| = |y| = 2$, so that lcm($|x|, |y|) = 2 \neq |x + y| = |0| = 1$.
      \qed
%%%%%%%%%%%%%%%%%%%%%%%%%%%%%%%%%%%%%2.3.17%%%%%%%%%%%%%%%%%%%%%%%%%%%%%%%%%%%%%
   \item[2.3.17]  Find a presentation for $Z_n$ with one generator.
   
      \textbf{Solution.} $Z_n = \cyc{x : x^n = 1}$.
%%%%%%%%%%%%%%%%%%%%%%%%%%%%%%%%%%%%%2.3.18%%%%%%%%%%%%%%%%%%%%%%%%%%%%%%%%%%%%%
   \item[2.3.18]  Show that if $H$ is any group and $h$ is an element of $H$
                  with $h^n = 1$, then there is a unique homomorphism from
                  $Z_n = \cyc{x}$ to $H$ such that $x \mapsto h$.
                  
      \textbf{Proof.} Let $n \in \Z^+$, $Z_n = \cyc{x}$, $H$ a group, and
      $h^n  = 1$ for some $h \in H$. First we shall show the existence of a
      homomorphism from $Z_n$ to $H$ such that $x \mapsto h$. So consider the
      map $\alpha : \cyc{x} \rightarrow H$ defined by $\alpha(x^a) = h^a$.
      Clearly $\alpha(x) = h$. Now we will show that $\alpha$ is well defined.
      Suppose $x^w = x^y$ for some $x^w, x^y \in Z_n$. Thus $w = y + nk$ for
      some integer $k$. Thus
      $$\alpha(x^w) = \alpha(x^{y+nk})=h^{y+nk}=h^{y}{h^n}^k =h^y=\alpha(x^y),$$
      so that $\alpha$ is well defined. Now we have that
      $$\alpha(x^px^q)=\alpha(x^{p+q})=h^{p+q}=h^ph^q=\alpha(x^p)\alpha(x^q),$$
      so that $\alpha$ is an homomorphism. Now to show uniqueness, we suppose
      that $\phi : \cyc{x} \rightarrow H$ is an homommorphism such that
      $\phi(x) = h$. Since $\phi$ is a homomorphism, it follows that
      $\phi(x^a) = h^a$. Thus $\phi = \alpha$, as desired. \qed
%%%%%%%%%%%%%%%%%%%%%%%%%%%%%%%%%%%%%2.3.19%%%%%%%%%%%%%%%%%%%%%%%%%%%%%%%%%%%%%
   \item[2.3.19]  Show that if $H$ is any group and $h$ is an element of $H$,
                  then there is a unique homomorphism from $\Z$ to $H$ such that
                  $1 \mapsto h$.
                  
      \textbf{Proof.} Let $H$ be a group and let $h \in H$. First we shall show
      that there exists a homomorphism from $\Z$ to $H$ such that $1 \mapsto h$.
      So consider the map $\alpha : \Z \rightarrow H$ defined by
      $n \mapsto h^n$. Clearly $\alpha(1) = h$ and
      $$\alpha(x+y) = h^{x+y} = h^xh^y = \alpha(x)\alpha(y) \text{ for all }
        x, y \in \Z^+,$$
      so that $\alpha$ is a homomorphism. To show uniqueness, suppose that
      $\alpha' : \Z \rightarrow H$ is an homomorphism such that
      $\alpha'(1) = h$. Then according to Exercise 1.6.1, we have that
      $\alpha'(n) = \alpha'(n\cdot1) = \alpha'(1)^n = h^n$ for all $n \in \Z$;
      that is, $\alpha' = \alpha$, as desired. \qed
%%%%%%%%%%%%%%%%%%%%%%%%%%%%%%%%%%%%%2.3.20%%%%%%%%%%%%%%%%%%%%%%%%%%%%%%%%%%%%%
   \item[2.3.20]  Let $p$ be a prime and let $n$ be a positive integer. Show
                  that if $x$ is an element of the group $G$ such that
                  $x^{p^n} = 1$ then $|x| = p^m$ for some $m \le n$.
                  
      \textbf{Proof.} Suppose that $x \in G$ such that $x^{p^n} = 1$. Then it
      follows by Proposition 3 (Page 55) that $|x|$ divides $p^n$. Since $p$ is
      a prime, its factors are $p^i$, $0 \le i \le n$. Thus $|x| = p^m$ for
      some nonnegative $m$ not greater than $n$. \qed
%%%%%%%%%%%%%%%%%%%%%%%%%%%%%%%%%%%%%2.3.21%%%%%%%%%%%%%%%%%%%%%%%%%%%%%%%%%%%%%
   \item[2.3.21]  Let $p$ be an odd prime and let $n$ be a positive integer
                  $\ge 2$. Use the Binomial Theorem to show that
                  $(1+p)^{p^{n-1}} \equiv 1$ (mod $p^n$) but
                  $(1+p)^{p^{n-2}} \not\equiv 1$ (mod $p^n$). Deduce that $1+p$
                  is an element of order $p^{n-1}$ in the multiplicative group
                  $(\Z/p^n\Z)^\times$.

      \textbf{Lemma 2.3.1.} \textit{For an integer $n \ge 2$ and an odd prime
      $p$, let $f_p(n)$ be the number of $p$ factors of $n!$ (i.e., the greatest
      nonnegative integer $j$ such that $p^j \mid i!$), then it follows that
      $f_p(n) < \D\frac{n}{2}$}.

      \textbf{Proof.} Let $n \ge 2$ be an integer and $p$ an odd prime. For a
      a positive integer $r$, let $g_p(n, r)$ be the number of positive
      integers, less than or equal to $n$, that have at least $r$ number of $p$ 
      factors. It follows that $g_p(n, r) = \D\gint{\frac{n}{p^r}}$, where
      $\gint{x}$ is the greatest integer less than or equal to $x$. Finally let
      $k_n$ be the maximum nonnegative integer such that $p^{k_n}$ is a multiple
      of some positive integer not greater than $n$. Thus we have that
      \begin{align*}
         f_p(n) &= g_p(n, 1) + g_p(n, 2) + \cdots + g_p(n, k_n) \\
            &= \sum_{i=1}^{k_n} g_p(n, i)
            = \sum_{i=1}^{k_n} \gint{\frac{n}{p^i}} \\
            &\le \sum_{i=1}^{k_n} \frac{n}{p^i}
            < \sum_{i=1}^\infty \frac{n}{p^i} \\
            &= \frac{n}{p-1} &[\text{Sum of Geometric Series}] \\
            &< \frac{n}{2}. &[\text{Since }p \ge 3]
      \end{align*}

      So we can write $n! = p^{f_p(n)} h_n$ for some $h_n \in \Z^+$, so that
      $(h_n, p) = 1$.

      Now we are ready to commence the proof of the problem. By the Binomial
      Theorem, we have that
      \begin{align*}
         (1+p)^{p^{n-1}} &= \sum_{i=0}^{p^{n-1}}\binom{p^{n-1}}{i}p^i \\
            &= \sum_{i=0}^{p^{n-1}}p^i\frac{p^{n-1}(p^{n-1}-1)(p^{n-1}-2)
               \cdots(p^{n-1}-i+1)}{i!} \\
            &= \sum_{i=0}^{p^{n-1}}p^i\frac{p^{n-1}(p^{n-1}-1)(p^{n-1}-2)
               \cdots(p^{n-1}-i+1)}{p^{f_p(i)} h_i} \\
            &= 1 + p^n + p^n\sum_{i=2}^{p^{n-1}}\frac{p^{i-1}(p^{n-1}-1)
               (p^{n-1}-2) \cdots(p^{n-1}-i+1)}{p^{f_p(i)} h_i}.
      \end{align*}
      Now $f_p(i) < i / 2 \le i - 1$ for $i \ge 2$. Thus $i - 1 - f_p(i) \ge 0$
      (so that $p^{i - 1 - f_p(i)}$ is an integer) if $i \ge 2$. We then have
      \begin{equation} \label{2_3_21_1}
         (1+p)^{p^{n-1}} = 1 + p^n + p^n\sum_{i=2}^{p^{n-1}}\frac{p^{i-1-f_p(i)}
        (p^{n-1}-1)(p^{n-1}-2) \cdots(p^{n-1}-i+1)}{h_i}
      \end{equation}
      Since $(h_i, p) = 1$, it follows that $h_i$ must divide
      $p^{i-1}(p^{n-1}-1)(p^{n-1}-2) \cdots(p^{n-1}-i+1)$. Hence
      $$\sum_{i=2}^{p^{n-1}}\frac{p^{i-1-f_p(i)}
        (p^{n-1}-1)(p^{n-1}-2) \cdots(p^{n-1}-i+1)}{h_i}$$
      is an integer and we can conclude from \eqref{2_3_21_1} that
      $(1+p)^{p^{n-1}} \equiv 1$ (mod $p^n$). Now we have that
      \begin{align*}
         (1+p)^{p^{n-2}} &= \sum_{i=0}^{p^{n-2}}\binom{p^{n-2}}{i}p^i \\
            &= \sum_{i=0}^{p^{n-2}}p^i\frac{p^{n-2}(p^{n-2}-1)(p^{n-2}-2)
               \cdots(p^{n-2}-i+1)}{i!} \\
            &= 1 + p^{n-1} + p^n\frac{p^{n-2}-1}{2} + p^n\frac{p(p^{n-2}-1)(p^{n-2}-2)}{3!} +\sum_{i=4}^{p^{n-1}}p^i\frac{p^{n-2}(p^{n-2}-1)(p^{n-2}-2)
               \cdots(p^{n-2}-i+1)}{p^{f_p(i)} h_i} \\
            &= 1 + p^n + p^n\sum_{i=2}^{p^{n-1}}\frac{p^{i-1}(p^{n-1}-1)
               (p^{n-1}-2) \cdots(p^{n-1}-i+1)}{p^{f_p(i)} h_i}.
      \end{align*}
      
%%%%%%%%%%%%%%%%%%%%%%%%%%%%%%%%%%%%%2.3.22%%%%%%%%%%%%%%%%%%%%%%%%%%%%%%%%%%%%%
   \item[2.3.22]  Let $n$ be an integer $\ge 3$. Use the Binomial Theorem to
                  show that $(1+2^2)^{2^{n-2}} \equiv 1$ (mod $2^n$) but
                  $(1+2^2)^{2^{n-3}} \not\equiv 1$ (mod $2^n$). Deduce that 5 is
                  an element of order $2^{n-2}$ in the multiplicative group
                  $(\Z/2^n\Z)^\times$.

      \textbf{Proof.}
%%%%%%%%%%%%%%%%%%%%%%%%%%%%%%%%%%%%%2.3.23%%%%%%%%%%%%%%%%%%%%%%%%%%%%%%%%%%%%%
   \item[2.3.23]  Show that $(\Z/2^n\Z)^\times$ is not cyclic for any $n \ge 3$.
                  [Find two distinct subgroups of order 2.]
%%%%%%%%%%%%%%%%%%%%%%%%%%%%%%%%%%%%%2.3.24%%%%%%%%%%%%%%%%%%%%%%%%%%%%%%%%%%%%%
   \item[2.3.24]  Let $G$ be a finite group and let $x \in G$.
                  \begin{enumerate}
                     \item Prove that if $g \in N_G(\cyc{x})$ then
                           $gxg^{-1} = x^a$ for some $a \in \Z$. 
                     \item Prove conversely that if $gxg^{-1} = x^a$ for some
                           $a \in \Z$ then $g \in N_G(\cyc{x})$. [Show first
                           that $gx^kg^{-1} = (gxg^{-1})^k = x^{ak}$ for any
                           integer $k$, so that $g\cyc{x}g^{-1} \le \cyc{x}$.
                           If $x$ has order $n$, show the elements $gx^ig^{-1}$,
                           $i = 0, 1, \ldots, n-1$ are distinct, so that
                           $|g\cyc{x}g^{-1}| = |\cyc{x}| = n$ and conclude that
                           $g\cyc{x}g^{-1} = \cyc{x}$.]
                  \end{enumerate}
                  Note that this cuts down some of the work in computing
                  normalizers of cyclic subgroups since one does not have to
                  check $ghg^{-1} \in \cyc{x}$ for every $h \in \cyc{x}$.
%%%%%%%%%%%%%%%%%%%%%%%%%%%%%%%%%%%%%2.3.25%%%%%%%%%%%%%%%%%%%%%%%%%%%%%%%%%%%%%
   \item[2.3.25]  Let $G$ be a cyclic group of order $n$ and let $k$ be an
                  integer relatively prime to $n$. Prove that the map
                  $x \mapsto x^k$ is surjective. Use Lagrange's Theorem
                  (Exercise 1.7.19) to prove the same is true for any finite
                  group of order $n$. (For such $k$ each element has a
                  $k^{\text{th}}$ root in $G$. It follows from Cauchy's Theorem
                  in Section 3.2 that if $k$ is not relatively prime to the
                  order of $G$ then the map $x \mapsto x^k$ is not surjective.)
%%%%%%%%%%%%%%%%%%%%%%%%%%%%%%%%%%%%%2.3.26%%%%%%%%%%%%%%%%%%%%%%%%%%%%%%%%%%%%%
   \item[2.3.26]  Let $Z_n$ be a cyclic group of order $n$ and for each integer
                  $a$ let
                  $$\sigma_a : Z_n \mapsto Z_n \qquad by \qquad \sigma_a(x) =
                  x^a \quad \text{for all } x \in Z_n.$$
                  \begin{enumerate}
                     \item Prove that $\sigma_a$ is an automorphism of $Z_n$ if
                           and only if $a$ and $n$ are relatively prime(
                           automorphisms were introduced in Exercise 1.6.20).
                     \item Prove that $\sigma_a = \sigma_b$ if and only if
                           $a \equiv b$ (mod $n$).
                     \item Prove that \textit{every} automorphism of $Z_n$ is
                           equal to $\sigma_a$ for some integer $a$.
                     \item Prove that $\sigma_a\circ\sigma_b=\sigma_{ab}$.
                           Deduce that the map $\overline{a} \mapsto \sigma_a$
                           is an isomorphism of $(\Z/n\Z)^\times$ onto the
                           automorphism group of $Z_n$ (so Aut($Z_n$) is an
                           abelian group of order $\varphi(n)$).
                  \end{enumerate}
                  %%%%%MISSING CONTAINMENT%%%%%%%%
\end{enumerate}


































      \section{Properties of Ideals}
         Let $F$ be a field and let $n \in \Z^+$.
\begin{enumerate}
%%%%%%%%%%%%%%%%%%%%%%%%%%%%%%%%%%%%%1.4.1%%%%%%%%%%%%%%%%%%%%%%%%%%%%%%%%%%%%%%
   \item[1.4.1]   Prove that $|GL_2(\F_2)| = 6$.
%%%%%%%%%%%%%%%%%%%%%%%%%%%%%%%%%%%%%1.4.2%%%%%%%%%%%%%%%%%%%%%%%%%%%%%%%%%%%%%%
   \item[1.4.2]   Write out all the elements of $GL_2(\F_2)$ and compute the
                  order of each element.
%%%%%%%%%%%%%%%%%%%%%%%%%%%%%%%%%%%%%1.4.3%%%%%%%%%%%%%%%%%%%%%%%%%%%%%%%%%%%%%%
   \item[1.4.3]   Show that $GL_2(\F_2)$ is non-abelian.
%%%%%%%%%%%%%%%%%%%%%%%%%%%%%%%%%%%%%1.4.4%%%%%%%%%%%%%%%%%%%%%%%%%%%%%%%%%%%%%%
   \item[1.4.4]   Show that if $n$ is not prime then $\Z/n\Z$ is not a field.
%%%%%%%%%%%%%%%%%%%%%%%%%%%%%%%%%%%%%1.4.5%%%%%%%%%%%%%%%%%%%%%%%%%%%%%%%%%%%%%%
   \item[1.4.5]   Show that $GL_n(F)$ is a finite group if and only if $F$ has a
                  finite number of elements.
%%%%%%%%%%%%%%%%%%%%%%%%%%%%%%%%%%%%%1.4.6%%%%%%%%%%%%%%%%%%%%%%%%%%%%%%%%%%%%%%
   \item[1.4.6]   If $|F| = q$ is finite prove that $|GL_n(F)| < q^{n^2}$.
%%%%%%%%%%%%%%%%%%%%%%%%%%%%%%%%%%%%%1.4.7%%%%%%%%%%%%%%%%%%%%%%%%%%%%%%%%%%%%%%
   \item[1.4.7]   Let $p$ be a prime. Prove that the order of $GL_2(\F_p)$ is
                  $p^4 - p^3 - p^2 + p$.
%%%%%%%%%%%%%%%%%%%%%%%%%%%%%%%%%%%%%1.4.8%%%%%%%%%%%%%%%%%%%%%%%%%%%%%%%%%%%%%%
   \item[1.4.8]   Show that $GL_n(F)$ is non-abelian for any $n \ge 2$ and any
                  $F$.
%%%%%%%%%%%%%%%%%%%%%%%%%%%%%%%%%%%%%1.4.9%%%%%%%%%%%%%%%%%%%%%%%%%%%%%%%%%%%%%%
   \item[1.4.9]   Prove that the binary operation of matrix multiplication of
                  $2 \times 2$ matrices with real number entries is associative.
%%%%%%%%%%%%%%%%%%%%%%%%%%%%%%%%%%%%%1.4.10%%%%%%%%%%%%%%%%%%%%%%%%%%%%%%%%%%%%%
   \item[1.4.10]  Let $\left\{\left(\begin{tabular}{@{}cc@{}}
                     $a$ & $b$ \\
                      0  & $c$
                  \end{tabular}\right) : a, b, c \in \R, a \neq 0, c \neq 0
                  \right\}$.

                  \begin{enumerate}
                     \item Compute the product of
                           $\left(\begin{tabular}{@{}cc@{}}
                              $a_1$ & $b_1$ \\
                              0  & $c_1$
                           \end{tabular}\right)$ and
                           $\left(\begin{tabular}{@{}cc@{}}
                              $a_2$ & $b_2$ \\
                              0  & $c_2$
                           \end{tabular}\right)$ to show that $G$ is closed under
                           matrix multiplication.
                     \item Find the matrix inverse of
                           $\left(\begin{tabular}{@{}cc@{}}
                              $a$ & $b$ \\
                              0  & $c$
                           \end{tabular}\right)$ and deduce that $G$ is closed 
                           under inverses.
                     \item Deduce that $G$ is a subgroup of $GL_2(\R)$.
                     \item Prove that the set of elements of $G$ whose two
                           diagonal entries are equal is also a subgroup of
                           $GL_2(\R)$.
                  \end{enumerate}
\end{enumerate}

The next exercise introduces the \textit{Heisenberg group} over the field $F$
and develops some of its basic properties. When $F = \R$ this groups plays an
important role in quantum mechanics and signal theory by giving a group
theoretic interpretation (due to H. Weyl) of Heisenberg's Uncertainty Principle.
Note also that the Heisenberg group may be defined more generally---for example,
with entries in $\Z$.

\begin{enumerate}
%%%%%%%%%%%%%%%%%%%%%%%%%%%%%%%%%%%%%1.4.11%%%%%%%%%%%%%%%%%%%%%%%%%%%%%%%%%%%%%
   \item[1.4.11]  Let $H(F) = \left\{\left(\begin{tabular}{@{}ccc@{}}
                     1 & $a$ & $b$ \\
                     0 & 1 & $c$ \\
                     0 & 0 & 1
                  \end{tabular}\right) : a, b, c \in F\right\}$---called the
                  \textit{Heisenberg group} over $F$. Let
                  $X = \left(\begin{tabular}{@{}ccc@{}}
                     1 & $a$ & $b$ \\
                     0 & 1 & $c$ \\
                     0 & 0 & 1
                  \end{tabular}\right)$ and $Y =\left(\begin{tabular}{@{}ccc@{}}
                     1 & $d$ & $e$ \\
                     0 & 1 & $f$ \\
                     0 & 0 & 1
                  \end{tabular}\right)$ be elements of $H(F)$.

                  \begin{enumerate}
                     \item Compute the matrix product $XY$ and deduce that
                           $H(F)$ is closed under matrix multiplication. Exhibit
                           explicit matrices such that $XY \neq YX$ (so that
                           $H(F)$ is always non-abelian).
                     \item Find an explicit formula for the matrix inverse
                           $X^{-1}$ and deduce that $H(F)$ is closed under
                           inverses.
                     \item Prove the associative law for $H(F)$ and deduce that
                           $H(F)$ is a group of order $|F|^3$. 

                           (Do not assume that matrix multiplication is 
                           associative).
                     \item Find the order of each element of the finite group
                           $H(\Z/2\Z)$.
                     \item Prove that every nonidentity element of the group
                           $H(\R)$ has infinite order.
                  \end{enumerate}
\end{enumerate}

\end{comment}
\end{document}
