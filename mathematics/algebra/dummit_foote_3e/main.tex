\documentclass[9pt]{book}

\usepackage{amssymb}
\usepackage{amsmath}
\usepackage{amsfonts}
\usepackage{cancel}
\usepackage{comment}
\usepackage{fancyhdr}
\usepackage{mathrsfs}
\usepackage{color}

\usepackage{array}
%\usepackage[retainorgcmds]{IEEEtrantools}


\usepackage[utf8]{inputenc}

%\usepackage{hyperref}
%\hypersetup{
%    colorlinks=true,
%    linkcolor=blue,
%    filecolor=magenta,      
%    urlcolor=cyan,
%    pdftitle={Sharelatex Example},
%    bookmarks=true,
%    pdfpagemode=FullScreen,
%}

%\voffset = -50pt
%\textheight = 700pt
\addtolength{\textwidth}{60pt}
\addtolength{\evensidemargin}{-30pt}
\addtolength{\oddsidemargin}{-30pt}
%\setlength{\headheight}{44pt}

\newcommand{\qed}{\hfill \ensuremath{\Box}}

\newcommand{\N}{\mathbb{N}}
\newcommand{\Z}{\mathbb{Z}}
\newcommand{\Q}{\mathbb{Q}}
\newcommand{\R}{\mathbb{R}}
\newcommand{\C}{\mathbb{C}}
\newcommand{\F}{\mathbb{F}}
\newcommand{\uline}{\underline{\hspace{3mm}}}
\newcommand{\cyc}[1]{\langle #1 \rangle}
\newcommand{\CYC}[1]{\left\langle #1 \right\rangle}
\newcommand{\gint}[1]{\left\lfloor #1 \right\rfloor}
\newcommand{\D}{\displaystyle}
\setcounter{chapter}{-1}
\title{\vspace{-14cm}Abstract Algebra by Dummit and Foote \\Solutions Manual, By}
\author{Joseph Okonoboh}
\date{\today}

\begin{document}

   \maketitle
\begin{comment}   
   \tableofcontents

   \chapter{Preliminaries}
      \section{Basics}
         Let
   $$\mathcal{A} \mbox{ be the set of real } 2 \times 2
     \mbox{ matrices}, M = \left(
      \begin{tabular}{@{}l r@{}} 
         1 & 1\\ 
         0 & 1
      \end{tabular}\right)$$
and let
   $$\mathcal{B} = \{X \in \mathcal{A} : MX = XM\}.$$
\begin{enumerate}
%%%%%%%%%%%%%%%%%%%%%%%%%%%%%%%%%%%Prob0.1.1%%%%%%%%%%%%%%%%%%%%%%%%%%%%%%%%%%%%
   \item[0.1.1] Determine which of the following elements of $\mathcal{A}$ lie
                in $\mathcal{B}$:

                $$M = \left(
                   \begin{tabular}{@{}l r@{}} 
                      1 & 1 \\ 
                      0 & 1
                   \end{tabular}\right), \left(
                   \begin{tabular}{@{}l r@{}} 
                      1 & 1 \\ 
                      1 & 1
                   \end{tabular}\right), \left(
                   \begin{tabular}{@{}l r@{}} 
                      0 & 0 \\ 
                      0 & 0
                   \end{tabular}\right), \left(
                   \begin{tabular}{@{}l r@{}} 
                      1 & 1 \\ 
                      1 & 0
                   \end{tabular}\right), \left(
                   \begin{tabular}{@{}l r@{}} 
                      1 & 0 \\ 
                      0 & 1
                   \end{tabular}\right), \left(
                   \begin{tabular}{@{}l r@{}} 
                      0 & 1 \\ 
                      1 & 1
                   \end{tabular}\right).$$
      \textbf{Solution:} A quick computation will show us only these matrices
      lie in $\mathcal{B}$.

                $$\left(
                   \begin{tabular}{@{}l r@{}} 
                      1 & 1 \\ 
                      0 & 1
                   \end{tabular}\right), \left(
                   \begin{tabular}{@{}l r@{}} 
                      0 & 0 \\ 
                      0 & 0
                   \end{tabular}\right), \left(
                   \begin{tabular}{@{}l r@{}} 
                      1 & 0 \\ 
                      0 & 1
                   \end{tabular}\right).$$
%%%%%%%%%%%%%%%%%%%%%%%%%%%%%%%%%%%Prob0.1.2%%%%%%%%%%%%%%%%%%%%%%%%%%%%%%%%%%%%
   \item[0.1.2] Prove that if $P, Q \in \mathcal{B}$, then
                $P + Q \in \mathcal{B}$.

      \textbf{Proof:} Suppose that $P, Q \in \mathcal{B}$. We want to show that
      $P + Q \in \mathcal{B}$. That is, we must show that $M(P + Q) = (P + Q)M$.
      Thus
      \begin{align*}
         M(P + Q) &= MP + MQ   &[\text{Distributive Law}]\\
                  &= PM + QM   &[P, Q \in \mathcal{B}] \\
                  &= (P + Q)M. &[\text{Distributive Law}]
      \end{align*} \qed
%%%%%%%%%%%%%%%%%%%%%%%%%%%%%%%%%%%Prob0.1.3%%%%%%%%%%%%%%%%%%%%%%%%%%%%%%%%%%%%
   \item[0.1.3] Prove that if $P, Q \in \mathcal{B}$, then
                $P \cdot Q \in \mathcal{B}$.

      \textbf{Proof:} Suppose that $P, Q \in \mathcal{B}$. We want to show that
      $PQ \in \mathcal{B}$. That is, we must show that $M(PQ) = (PQ)M$.
      Then
      \begin{align*}
         M(PQ) &= (MP)Q  &[\text{Associative Law}]\\
               &= (PM)Q  &[P \in \mathcal{B}] \\
               &= P(MQ)  &[\text{Associative Law}] \\
               &= P(QM)  &[Q \in \mathcal{B}] \\
               &= (PQ)M.  &[\text{Associative Law}]
      \end{align*} \qed
%%%%%%%%%%%%%%%%%%%%%%%%%%%%%%%%%%%Prob0.1.4%%%%%%%%%%%%%%%%%%%%%%%%%%%%%%%%%%%%
   \item[0.1.4] Find conditions on $p, q, r, s$ which determine precisely when
                $$\left(
                     \begin{tabular}{@{}l r@{}} 
                        $p$ & $q$ \\ 
                        $r$ & $s$
                     \end{tabular}\right) \in \mathcal{B}.$$

      \textbf{Proof:} Suppose $A = \left(\begin{tabular}{@{}l r@{}} 
      $p$ & $q$ \\ 
      $r$ & $s$
      \end{tabular}\right) \in \mathcal{B}$. Then it must be the case that
      $AM = MA$, so that
      $$ \left(
         \begin{tabular}{@{}l r@{}} 
            $p$ & $q$ \\ 
            $r$ & $s$
         \end{tabular}\right)\left(
         \begin{tabular}{@{}l r@{}} 
            1 & 1 \\ 
            0 & 1
         \end{tabular}\right) = \left(
         \begin{tabular}{@{}l r@{}} 
            1 & 1 \\ 
            0 & 1
         \end{tabular}\right)\left(
         \begin{tabular}{@{}l r@{}} 
            $p$ & $q$ \\ 
            $r$ & $s$
         \end{tabular}\right).
      $$

      That is,
      $$ \left(
         \begin{tabular}{@{}l r@{}} 
            $p$ & $p + q$ \\ 
            $r$ & $r + s$
         \end{tabular}\right) = \left(
         \begin{tabular}{@{}c c@{}} 
            $p + r$ & $q + s$ \\ 
            $r$ & $s$
         \end{tabular}\right).
      $$

      Equating corresponding entries tells us that
      $r = 0$ and $p = s$. Thus
      $$\mathcal{B} = \left\{\left(
         \begin{tabular}{@{}l r@{}} 
            $p$ & $q$ \\ 
            $0$ & $p$
         \end{tabular}\right) \in \mathcal{A}\right\}.$$
%%%%%%%%%%%%%%%%%%%%%%%%%%%%%%%%%%%Prob0.1.5%%%%%%%%%%%%%%%%%%%%%%%%%%%%%%%%%%%%
   \item[0.1.5] Determine whether the following functions $f$ are well defined:
                \begin{enumerate}
                   \item $f : \Q \rightarrow \Z$ defined by $f(a/b) = a$.
                   \item $f : \Q \rightarrow \Q$ defined by $f(a/b) = a^2/b^2$.
                \end{enumerate}

      \textbf{Solution:}
         \begin{enumerate}
            \item $f$ is not well defined because $4/1 = 8/2$, but
                  $f(4/1) = 4 \neq 8 = f(8/2)$.
            \item We claim that $f$ is well defined.

                  \textbf{Proof:} We want to show that all representatives for 
                  an element in $\Q$ have the same output under $f$. So let
                  $a/b \in \Q$ with $\gcd(a, b) = 1$. Suppose $m \in \Q$ such 
                  that $m = a/b$. Then we must have that $m = ka/kb$ where $k$ 
                  is some nonzero integer. It follows that
                  $f(ka/kb) = k^2a^2/k^2b^2 = a^2/b^2 = f(a/b)$. \qed
         \end{enumerate}
%%%%%%%%%%%%%%%%%%%%%%%%%%%%%%%%%%%Prob0.1.6%%%%%%%%%%%%%%%%%%%%%%%%%%%%%%%%%%%%
   \item[0.1.6] Determine whether the function $f: \R^+ \rightarrow \Z$ defined
                by mapping a real number $r$ to the first digit to the right of
                the decimal point in a decimal expansion of $r$ is well defined.

      \textbf{Proof:} The map $f$ is not well defined because we have
      $0.\overline{9} = 1.\overline{0}$, but
      $f(0.\overline{9}) = 9 \neq 0 = f(1.\overline{0})$. \qed
%%%%%%%%%%%%%%%%%%%%%%%%%%%%%%%%%%%Prob0.1.7%%%%%%%%%%%%%%%%%%%%%%%%%%%%%%%%%%%%
   \item[0.1.7] Let $f : A \rightarrow B$ be a surjective map of sets. Prove
                that the relation
                $$a \sim b \mbox{ if and only if } f(a) = f(b)$$
                is an equivalence relation whose equivalence classes are the
                fibers of $f$.

      \textbf{Proof:} We want to show that the relation $\sim$ is an equivalence
      relation on $A$. That is, we want to show that $\sim$ is reflexive,
      symmetric, and transitive on $A$. Thus
      \begin{enumerate}
         \item[\textit{Reflexivity}:]  Let $a \in A$. Since $f(a) = f(a)$, it
                                       follows that $a \sim a$, so that $\sim$ 
                                       is reflexive on $A$.
         \item[\textit{Symmetry}:]     Let $a, b \in A$ with $a \sim b$. Since
                                       $a \sim b$, we must have that
                                       $f(a) = f(b)$; it immediately follows    
                                       that $f(b) = f(a)$, so that $b \sim a$;
                                       that is, $\sim$ is symmetric on $A$.
         \item[\textit{Transitivity}:] Let $a, b, c \in A$ with $a \sim b$ and
                                       $b \sim c$. Thus by definition, we have
                                       that $f(a) = f(b)$ and $f(b) = f(a)$. By
                                       transitivity of the equality relation it
                                       follows that $f(a) = f(c)$, so that
                                       $a \sim c$. Thus $\sim$ is transitive on
                                       $A$.
      \end{enumerate}

      Now we must show that the fibers of $f$ are the equivalence classes of
      $\sim$. So let $\mathcal{C}$ be the set of equivalence classes of $\sim$ 
      and let  $\mathcal{F}$ be the set of fibers of $f$. First we want to show
      that $\mathcal{F} \subseteq \mathcal{C}$. So consider
      $F \in \mathcal{F}$. Let $h \in F$. Then we see that $\overline{h} = F$
      (because if $h' \in \overline{h}$ then $f(h') = f(h)$, so that $h' \in F$,
      and $\overline{h} \subseteq F$. Also if $g \in F$, then $f(h) = f(g)$ so  
      that $f \in \overline{h}$ and $F \subseteq \overline{h}$), so that $F \in
      \mathcal{C}$ and $\mathcal{F} \subseteq \mathcal{C}$. Now we want to show
      that $\mathcal{C} \subseteq \mathcal{F}$ to complete the proof. Suppose
      $C \in \mathcal{C}$. Let $c \in C$. Then $C$ is the fiber of $f(c)$. Thus
      $C \in \mathcal{F}$ so that $\mathcal{C} \subseteq \mathcal{F}$. Since
      $\mathcal{F} \subseteq \mathcal{C}$ and
      $\mathcal{C} \subseteq \mathcal{F}$, we must have that
      $\mathcal{C} = \mathcal{F}$. \qed
%%%%%%%%%%%%%%%%%%%%%%%%%%%%%%%%%%%Prob0.1.8%%%%%%%%%%%%%%%%%%%%%%%%%%%%%%%%%%%%
   \item[0.1.8] \textbf{Proposition 1:}
                \begin{enumerate}
                   \item The map $f$ is injective if and only if $f$ has a left
                         inverse.
                   \item The map $f$ is surjective if and only if $f$ has a
                         right inverse.
                   \item The map $f$ is a bijection if and only if there exists
                         $g : B \rightarrow A$ such that $f \circ g$ is the
                         identity map on $B$ and $g \circ f$ is the identity map
                         on $A$.
                   \item If $A$ and $B$ are finite sets with the same number of
                         elements, then $f : A \rightarrow B$ is bijective if
                         and only if $f$ is injective if and only if $f$ is
                         surjective.
                \end{enumerate}

      \textbf{Proof:}
      \begin{enumerate}
         \item ($\Leftarrow$) Suppose first that the map $f : A \rightarrow B$ 
               has a left inverse, say $g : B \rightarrow A$. We want to show 
               that $f$ has is injective. So consider $a, b \in A$ such that
               $f(a) = f(b)$. Then we have that $g(f(a)) = g(f(b))$ so that
               $a = b$ since $g$ is a left inverse of $f$. Thus $f$ is
               injective. \\
               ($\Rightarrow$) Now suppose that $f$ is injective. Then we must 
               show that $f$ has	a left inverse. Notice that since $f$ is 
               injective, it must be the case that for every $b'' \in f(A)$ 
               there exists a unique $a'' \in A$ such that $f(a'') = b''$. Let 
               $a'$ be any member of $A$ and consider the map
	            $$g : B \rightarrow A,$$
	            where for every $b \in B$, we have
	            \begin{equation*}
		            g(b) = \left\{
			            \begin{array}{ll}
				            a & \text{if } b \in f(A) \text{ and where } f(a) = b,\\
                        a' & \text{if } b \in B\setminus f(A).
                     \end{array} \right.
               \end{equation*}
   
               So $(g \circ f)(a_1) = g(f(a_1)) = a_1$ for all $a_1 \in A$.
               Hence $g \circ f$ is the identity function on $A$, so that $g$ is
               a left inverse of $f$.

         \item $(\Leftarrow)$ Suppose that $f$ has a right inverse, say
               $g : B \rightarrow A$. We must now show that $f$ is also 
               surjective; so let $b \in B$. Then, by supposition, we have that 
               $f(g(b)) = b$. Thus $f$ maps $g(b)$---a member of $A$---to $b$, 
               so that $f$ is onto.
   
               $(\Rightarrow)$ Now suppose that $f$ is surjective. We now want 
               to show that $f$ has a right inverse. Let $b \in B$. Since the 
               surjectivity of $f$ guarantees that the fiber of $\{b\}$ over $f$
               is not empty, we let $h : B \rightarrow A$ be a function that 
               maps an element $b \in B$ to some element in the fiber of $\{b\}$
               over $f$. Clearly if $c \in B$, then $f(h(c)) = c$, so that $h$ 
               is a right inverse of $f$.
         \item $(\Leftarrow)$ Suppose that $g$ is a left and right inverse of
               $f$. Then by 1 and 2, $f$ is an injection and a surjection, so 
               that $f$ is a bijection.
   
               $(\Rightarrow)$ Now suppose that $f$ is a bijection. Let
               $b \in B$. We notice that the fiber of $\{b\}$ under $f$ is not 
               empty since $f$ is surjecitive, and this fiber contains exactly 
               one element of $A$. The latter is so since if $a_1, a_2$ are in 
               the fiber of $\{b\}$ over $f$, then $f(a_1) = f(a_2)$ so that 
               $a_1 = a_2$ by the injectivity of $f$. So let $g$ be the map
               $g : B \rightarrow A$ that maps $c \in B$ to the only element in 
               the fiber of $\{c\}$ over $f$. It is trivial to show that $g$ is 
               both a left and right inverse of $f$.

         \item Let $|A| = |B| = n \in \Z^+$. First we shall assume that $f$ is
               bijective. It immediately follows that $f$ is injective. Now 
               assume that $f$ is injective. Since $f$ is one to one, no two 
               elements of $A$ map to the same element in $B$. This implies that
               $f(A)$ must contain exactly $n$ elements since $A$ contains $n$ 
               elements. But $|B| = n$, so that $f(A) = B$. That is, $f$ is 
               surjective. Now suppose that $f$ is surjective. Since $f$ is
               surjective, none of its fibers is empty; thus, the number of
               fibers of $f$ must equal $|B| = n$. We shall argue by    
               contradiction that $f$ is injective. So suppose that $f$ is not
               injective. Let $f_1$, $f_2$, $\ldots$, $f_n$ be the $n$ fibers of
               $f$. Since $f$ is not injective, one of this fibers must contain
               more than 1 element. so assume without loss of generality that
               $|f_1| \ge 2$. Since these fibers are a partition of $A$ and
               since each fiber contains at least one element it follows that
               \begin{align*}
                  |A| &= |f_1| + |f_2| + \cdots + |f_n| \\
                      &\ge 2 + (n - 1) \\
                      &\ge n + 1,
               \end{align*}
               a contradiction since $|A| = n$. Thus $f$ is injective.	
      \end{enumerate}
\end{enumerate}

      \section{Properties Of The Integers}
         \begin{enumerate}
%%%%%%%%%%%%%%%%%%%%%%%%%%%%%%%%%%%%%0.2.1%%%%%%%%%%%%%%%%%%%%%%%%%%%%%%%%%%%%%%
   \item[0.2.1]   For each of the following pairs of integers $a$ and $b$,
                  determine their greatest common divisor, ther least common
                  multiple, and write their greatest common divisor in the form
                  $ax + by$ for some integers $x$ and $y$.
                  \begin{enumerate}
                     \item $a = 20$, $b = 13$.
                     \item $a = 69$, $b = 372$.
                     \item $a = 792$, $b = 275$.
                     \item $a = 11391$, $b = 5673$.
                     \item $a = 1761$, $b = 1567$.
                     \item $a = 507885$, $b = 60808$.
                  \end{enumerate}

      \textbf{Solution.}

      \begin{enumerate}
         \item Using the Euclidean Algorithm we have
               \begin{align*}
                  20 &= 1 \cdot 13 + 7 \\
                  13 &= 1 \cdot 7 + 6 \\
                  7  &= 1 \cdot 6 + 1, \text{ so that } \\ \\
                  1  &= 7 - 1 \cdot 6 \\
                     &= 7 - 1 \cdot (13 - 1 \cdot 7) \\
                     &= 2 \cdot 7 - 1 \cdot 13 \\
                     &= 2 \cdot (20 - 1 \cdot 13) - 1 \cdot 13 \\
                     &= 2 \cdot 20 - 3 \cdot 13.
               \end{align*}

               Thus $\gcd(20, 13) = 1$ and we have $x = 2$ and $y = -3$.
         \item Using the Euclidean Algorithm we have
               \begin{align*}
                  69  &= 0 \cdot 372 + 69 \\
                  372 &= 5 \cdot 69 + 27 \\
                  69  &= 2 \cdot 27 + 15 \\
                  27  &= 1 \cdot 15 + 12 \\
                  15  &= 1 \cdot 12 + 3 \\
                  12  &= 4 \cdot 3 + 0, \text{ so that } \\ \\
                   3 &= 15 - 1 \cdot 12 \\
                     &= 15 - 1 \cdot (27 - 1 \cdot 15) \\
                     &= 2 \cdot 15 - 1 \cdot 27 \\
                     &= 2 \cdot (69 - 2 \cdot 27) - 1 \cdot 27 \\
                     &= 2 \cdot 69 - 5 \cdot 27 \\
                     &= 2 \cdot 69 - 5 \cdot (372 - 5 \cdot 69) \\
                     &= 27 \cdot 69 - 5 \cdot 372 \\
                     &= 27 \cdot 69 - 5 \cdot 372.
               \end{align*}

               Thus $\gcd(69, 372) = 3$ and we have $x = 27$ and $y = -5$.
         \item Using the Euclidean Algorithm we have
               \begin{align*}
                  792 &= 2 \cdot 275 + 242 \\
                  275 &= 1 \cdot 242 + 33 \\
                  242 &= 7 \cdot 33 + 11 \\
                  33  &= 3 \cdot 11 + 0, \text{ so that } \\ \\
                  11  &= 242 - 7 \cdot 33 \\
                      &= 242 - 7 \cdot (275 - 1 \cdot 242) \\
                      &= 8 \cdot 242 - 7 \cdot 275 \\
                      &= 8 \cdot (792 - 2 \cdot 275) - 7 \cdot 275 \\
                      &= 8 \cdot 792 - 23 \cdot 275.
               \end{align*}

               Thus $\gcd(792, 275) = 11$ and we have $x = 8$ and $y = -23$.
         \item Using the Euclidean Algorithm we have
               \begin{align*}
                  11391 &= 2 \cdot 5673 + 45 \\
                  5673  &= 126 \cdot 45 + 3 \\
                  45    &= 3 \cdot 15 + 0, \text{ so that } \\ \\
                  3     &= 5673 - 126 \cdot 45 \\
                        &= 5673 - 126 \cdot (11391 - 2 \cdot 5673) \\
                        &= -126 \cdot 11391 + 253 \cdot 5673.
               \end{align*}

               Thus $\gcd(11391, 5673) = 3$ and we have $x = -126$ and
               $y = 253$.
         \item Using the Euclidean Algorithm we have
               \begin{align*}
                  1761 &= 1 \cdot 1567 + 194 \\
                  1567 &= 8 \cdot 194 + 15 \\
                  194  &= 12 \cdot 15 + 14 \\
                  15   &= 1 \cdot 14 + 1, \text{ so that } \\ \\
                   1   &= 15 - 1 \cdot 14 \\
                       &= 15 - 1 \cdot (194 - 12 \cdot 15) \\
                       &= 13 \cdot 15 - 1 \cdot 194 \\
                       &= 13 \cdot (1567 - 8 \cdot 194) - 1 \cdot 194 \\
                       &= 13 \cdot 1567 - 105 \cdot 194 \\
                       &= 13 \cdot 1567 - 105 \cdot (1761 - 1 \cdot 1567) \\
                       &= 118 \cdot 1567 - 105 \cdot 1761.
               \end{align*}

               Thus $\gcd(1761, 1567) = 1$ and we have $x = -105$ and $y = 118$.
         \item Using the Euclidean Algorithm we have
               \begin{align*}
                  507885 &= 8 \cdot 60808 + 21421 \\
                  60808  &= 2 \cdot 21421 + 17966 \\
                  21421  &= 1 \cdot 17966 + 3455 \\
                  17966  &= 5 \cdot 3455 +  691 \\
                  3455   &= 5 \cdot 691 + 0, \text{ so that } \\ \\
                  691    &= 17966 - 5 \cdot 3455 \\
                         &= 17966 - 5 \cdot (21421 - 1 \cdot 17966) \\
                         &= 6 \cdot 17966 - 5 \cdot 21421 \\
                         &= 6 \cdot (60808 - 2 \cdot 21421) - 5 \cdot 21421 \\
                         &= 6 \cdot 60808 - 17 \cdot 21421 \\
                         &= 6 \cdot 60808 - 17 \cdot (507885 - 8 \cdot 60808) \\
                         &= 142 \cdot 60808 - 17 \cdot 507885.
               \end{align*}

               Thus $\gcd(507885, 60808) = 691$ and we have $x = -17$ and
               $y = 142$.
      \end{enumerate}
%%%%%%%%%%%%%%%%%%%%%%%%%%%%%%%%%%%%%0.2.2%%%%%%%%%%%%%%%%%%%%%%%%%%%%%%%%%%%%%%
   \item[0.2.2]   Prove that if the integer $k$ divides the integers $a$ and $b$
                  then $k$ divides $as + bt$ for every pair of integers $s$ and
                  $t$.

      \textbf{Proof.} Let $a$ and $b$ be integers. Assume that $k$ divides
      $a$ and $b$. Consider any pair of integers $s$ and $t$. We want to show
      that $k$ also divides $as + bt$; that is, we must show that there exists
      some integer $m_1$ such that $as + bt = km_1$. Since $k$ divides $a$ and
      $b$, we must have that $a = km_2$ and $b = km_3$ for some integers $m_2$
      and $m_3$. Thus
      \begin{align*}
         as + bt &= km_2s + km_3t \\
                 &= k(m_2s + m_3t). 
      \end{align*}

      So take $m_1 = m_2s + m_3t$. \qed
%%%%%%%%%%%%%%%%%%%%%%%%%%%%%%%%%%%%%0.2.3%%%%%%%%%%%%%%%%%%%%%%%%%%%%%%%%%%%%%%
   \item[0.2.3]   Prove that if $n$ is composite then there are integers $a$ and
                  $b$ such that $n$ divides $ab$ but $n$ does not divide either
                  $a$ or $b$.

      \textbf{Proof.} Let $n > 1$ be a composite integer. Then $n = cd$, where
      $c$ and $d$ are integers greater than 1. Clearly, $n \mid cd$ because
      $n = cd$, but $n$ divides neither $c$ nor $d$, since they are both less
      than $n$. \qed
%%%%%%%%%%%%%%%%%%%%%%%%%%%%%%%%%%%%%0.2.4%%%%%%%%%%%%%%%%%%%%%%%%%%%%%%%%%%%%%%
   \item[0.2.4]   Let $a$, $b$ and $N$ be fixed integers with $a$ and $b$ 
                  nonzero and let $d = (a, b)$ be the greatest common divisor of
                  $a$ and $b$. Suppose $x_0$ and $y_0$ are particular solutions
                  to $ax + by = N$. Prove for any integer $t$ that the integers
                  $$x = x_0 + \frac{b}{d}t \qquad y = y_0 - \frac{a}{d}t$$
                  are also solutions to $ax + by = N$.

      \textbf{Proof.} Let $t$ be an integer, and let
		$$x = x_0 + \frac{b}{d}t \text{ and } y = y_0 - \frac{a}{d}t.$$
		Then we have
		\begin{align*}
			ax + by &= a\left(x_0 + \frac{b}{d}t\right) +
						  b\left(y_0 - \frac{a}{d}t\right) \\
					  &= ax_0 + by_0 \\
					  &= N,					  
		\end{align*}
		so that $x = x_0 + \frac{b}{d}t$ and $y = y_0 - \frac{a}{d}t$ are
		solutions to the equation $ax + by = N$.
%%%%%%%%%%%%%%%%%%%%%%%%%%%%%%%%%%%%%0.2.5%%%%%%%%%%%%%%%%%%%%%%%%%%%%%%%%%%%%%%
   \item[0.2.5]   Determine the value $\varphi(n)$ for each integer $n \le 30$
                  where $\varphi$ denotes the Euler $\varphi-$function.  

      \textbf{Solution.}

      We shall be making use of the multiplicative property of the Euler
		$\varphi-$function. So 

      \begin{center}
         \begin{tabular}{@{}l c r c l c l c r@{}}
            $\varphi(1)$ & = & 1, & & $\varphi(2)$ & = & 1, & & \\
            $\varphi(3)$ & = & 2, & & $\varphi(4)$ & = &
            $\varphi(2^2)$ & = & 2, \\
            $\varphi(5)$ & = & 4, & & $\varphi(6)$ & = &
            $\varphi(2)\varphi(3)$ & = & 2, \\
            $\varphi(7)$ & = & 6, & & $\varphi(8)$ & = &
            $\varphi(2^3)$ & = & 4, \\
            $\varphi(9) = \varphi(3^2)$ & = & 6, & & $\varphi(10)$ & = &
            $\varphi(2)\varphi(5)$ & = & 4, \\
            $\varphi(11)$ & = & 10, & & $\varphi(12)$ & = &
            $\varphi(3)\varphi(4)$ & = & 4, \\
            $\varphi(13)$ & = & 12, & & $\varphi(14)$ & = &
            $\varphi(2)\varphi(7)$ & = & 6, \\
            $\varphi(15) = \varphi(3)\varphi(5)$ & = & 8, & &
            $\varphi(16)$ & = &  $\varphi(2^4)$ & = & 8, \\
            $\varphi(17)$ & = & 16, & & $\varphi(18)$ & = &
            $\varphi(2)\varphi(9)$ & = & 6, \\
            $\varphi(19)$ & = & 18, & & $\varphi(20)$ & = &
            $\varphi(4)\varphi(5)$ & = & 8, \\
            $\varphi(21) = \varphi(3)\varphi(7)$ & = & 12, & &
            $\varphi(22)$ & = & $\varphi(2)\varphi(11)$ & = & 10, \\
            $\varphi(23)$ & = & 22, & &
            $\varphi(24)$ & = & $\varphi(3)\varphi(8)$ & = & 8, \\
            $\varphi(25) = \varphi(5^2)$ & = & 20, & &
            $\varphi(26)$ & = & $\varphi(2)\varphi(13)$ & = & 12, \\
            $\varphi(27) = \varphi(3^3)$ & = & 18, & &
            $\varphi(28)$ & = & $\varphi(4)\varphi(7)$ & = & 12, \\
            $\varphi(29)$ & = & 28, & &
            $\varphi(30)$ & = & $\varphi(2)\varphi(15)$ & = & 8. \\
         \end{tabular}
      \end{center}
%%%%%%%%%%%%%%%%%%%%%%%%%%%%%%%%%%%%%0.2.6%%%%%%%%%%%%%%%%%%%%%%%%%%%%%%%%%%%%%%
   \item[0.2.6]   Prove the Well Ordering Principle of $\Z$ by induction and
                  prove the minimal element is unique.

      \textbf{Proof.} Let $P$ be a nonempty subset of $\Z^+$. We want to show 
      that $P$ has a minimal element. So suppose by way of contradiction that
      $P$ does not have a minimal element. For a natural number $n$, let $S(n)$ 
      be the statement that $n$ is not a member of $P$. We now want to show that
      by Strong Induction that $S(n)$ holds for every natural number $n$. If 1
      is in $P$, then it would be the smallest member of $P$, contradicting our
      assumption that $P$ has no minimal element, so $1 \notin P$; hence $S(1)$ 
      is true. Now suppose that $S(j)$ is true for every natural number $j < k$,
      where $k$ is a natural number greater than 1. By our supposition, we know
      that every integer less than $k$ is not in $P$, so if $k$ is in $P$, it
      would be the minimal element of $P$, contradicting our assumption that $P$
      has no minimal element. Thus $S(k)$ is true. It follows by Mathematical
      Induction that $S(n)$ holds for every positive integer $n$. That is, $P$
      is empty, a contradiction. We can now conclude that $P$ has a minimal 
      element, say $p$. To show that $p$ is unique assume that $q$ is also a
      minimal element of $P$. By virtue of $p$ as a minimal element of $P$, we 
      have $p \le q$ and, by virtue of $q$ as a minimal element of $P$, we have
      $q \le p$, so that $p = q$. Hence the minimal element of $P$ is
      unique. \qed
%%%%%%%%%%%%%%%%%%%%%%%%%%%%%%%%%%%%%0.2.7%%%%%%%%%%%%%%%%%%%%%%%%%%%%%%%%%%%%%%
   \item[0.2.7]   If $p$ is a prime prove that there do not exist nonzero
                  integers $a$ and $b$ such that $a^2 = pb^2$.

      \textbf{Proof.} Let $p$ be a prime number. Suppose by contradiction that 
      there exist nonzero integers $a$ and $b$ such that $a^2 = pb^2$. We can
      further suppose that $a$ and $b$ are relatively prime. Since $a^2 = pb^2$,
      it follows that $a^2$ has $p$ as one of its prime factors, so that $a$
      also has $p$ as one of its prime factors. We can then write $a = pm$
      for some integer $m$. Substituting $a = pm$ in the equation $a^2 = pb^2$,
      will give us the equation $pm^2 = b^2$. We can similarly conclude that
      $b$ has $p$ as one of its prime factors, so that $\gcd(a, b) \ge p$, a
      contradiction. Thus there do not exist nonzero integers $a$ and $b$ such 
      that $a^2 = pb^2$. \qed
%%%%%%%%%%%%%%%%%%%%%%%%%%%%%%%%%%%%%0.2.8%%%%%%%%%%%%%%%%%%%%%%%%%%%%%%%%%%%%%%
   \item[0.2.8]   Let $p$ be a prime, $n \in \Z^+$. Find a formula for the
                  largest power of $p$ which divides
                  $n! = n(n - 1)(n - 2)\cdots2 \cdot 1$ (it involves the
                  greatest integer function).

      \textbf{Proof.} Let $p$ be a prime and let $n$ be a positive integer. The
      largest power of $p$ that divides $n!$, say $k$, is simply the number of 
      multiples of $p$ in the set $\{1, 2, \ldots, n\}$. Thus
      $k = \lfloor{n/p}\rfloor$, where $\lfloor{x}\rfloor$ is the greatest
      integer less than the real number $x$.
%%%%%%%%%%%%%%%%%%%%%%%%%%%%%%%%%%%%%0.2.9%%%%%%%%%%%%%%%%%%%%%%%%%%%%%%%%%%%%%%
   \item[0.2.9]   Write a computer program to determine the greatest common
                  divisor $(a, b)$ of two integers $a$ and $b$ and to express
                  $(a, b)$ in the form $ax + by$ for some integers $x$ and $y$.

   \begin{verbatim}
# Python
# For positive integers a and b gcd(a, b) returns
# a tuple (r, x, y) where r = gcd(a, b) and xa + yb = r
def gcd(a, b, x1 = 1, y1 = 0, x2 = 0, y2 = 1):
   q = a // b
   r = a % b

   if r == 0:
      return (b, 0, 1)

   x1 = x1 - q * x2
   y1 = y1 - q * y2

   if b % r == 0:
      return (r, x1, y1)

   return gcd(b, r, x2, y2, x1, y1)
   \end{verbatim}
%%%%%%%%%%%%%%%%%%%%%%%%%%%%%%%%%%%%%0.2.10%%%%%%%%%%%%%%%%%%%%%%%%%%%%%%%%%%%%%
   \item[0.2.10]  Prove for any given positive integer $N$ there exist only
                  finitely many integers $n$ with $\varphi(n) = N$ where
                  $\varphi$ denotes Euler's $\varphi$-function. Conclude in 
                  particular that $\varphi(n)$ tends to infinity as $n$ tends to
                  infinity.

      \textbf{Proof.} Let $N \in \N$ and define
      $$S_N = \{n \in \N : \varphi(n) = N\}.$$
      
      We want to show that $S_N$ is finite. So it suffices to show that $S_N$ is
      bounded; i.e., there exists a positive integer $K$ such that $n < K$ for
      all $n \in S_N$. If $S_N$ is empty, then we are done, so assume that $S_N$
      is nonempty. Let $m \in S_N$. If $m = 1$, then $N = 1$, and thus
      $S_1 = \{1, 2\}$ is finite. So assume $m > 1$. By the Fundamental 
      Theorem of Arithmetic, it follows that
      $$m = {p_1}^{c_1}{p_2}^{c_2}\cdots{p_s}^{c_s},$$
      where the $p_i$s are mutually distinct primes, $s$ and the $c_i$s are
      positive integers, and $s < N$ (the number of distinct prime factors a
      positive integer has is less than the integer). Applying $\varphi$ to $m$ 
      gives us 
      \begin{align}
         \varphi(m) &= \varphi({p_1}^{c_1}{p_2}^{c_2}
            \cdots{p_s}^{c_s}) \nonumber \\
            &= \varphi({p_1}^{c_1})\varphi({p_2}^{c_2})\cdots
               \varphi({p_s}^{c_s})  \nonumber \\
            &= (p_1 - 1)(p_2 - 1)\cdots(p_s-1){p_1}^{c_1-1}{p_2}^{c_2-1}
               \cdots{p_s}^{c_s-1} \label{0_2_1} \\
            &= N. \nonumber
      \end{align}

      Let $q$ be the least prime greater than $N + 1$. Thus each $p_i - 1 < q$,
      for otherwise $\varphi(m) > N$ by $\eqref{0_2_1}$. Similarly, since
      ${p_i}^{N} > N$, it must be the case that each $c_i < N + 1$. So
      $$m = {p_1}^{c_1}{p_2}^{c_2}\cdots{p_s}^{c_s} <
        \underbrace{q^{N+1}q^{N+1}\cdots q^{N+1}}_{N \text{ times}} 
        = q^{(N+1)N}.$$
      Thus
      $$S_N \subset \{1, 2, \ldots, q^{(N+1)N}\},$$ 
      so that $S_N$ is finite. Let $\varepsilon$ be a positive number. To show 
      that $\varphi(n)$ tends to infinity, we must find a natural number $K$
      such that $\varphi(n) > \varepsilon$ for all $n > K$. Define
      $M_j := \max(S_j)$ for all $j \in \N$. Choose
      $K = \max\{M_1, M_2, \ldots, M_{\lceil\varepsilon \rceil}\}$,
      ($\lceil\varepsilon\rceil$ is the least integer greater than
      $\varepsilon$). Consider $n > K$ and assume to the contrary that
      $\varphi(n) = l \in \{1, 2, \ldots, \lceil\varepsilon\rceil\}$; that is,
      $n \in S_l$, so that $M_l \ge n$. By definition, $K \ge M_l$, and thus,
      $K \ge n$, a contradiction. Hence $\varphi(n) \notin \{1, 2, \ldots, 
      \lceil\varepsilon\rceil\}$; that is,
      $\varphi(n)  > \lceil\varepsilon\rceil \ge \varepsilon$. \qed
%%%%%%%%%%%%%%%%%%%%%%%%%%%%%%%%%%%%%0.2.11%%%%%%%%%%%%%%%%%%%%%%%%%%%%%%%%%%%%%
   \item[0.2.11]  Prove that if $d$ divides $n$ then $\varphi(d)$ divides
                  $\varphi(n)$ where $\varphi$ denotes Euler's
                  $\varphi-$function.

      \textbf{Proof.} Let $d$ and $n$ be positive integers such that $d \mid n$.
      We want to show that $\varphi(d) \mid \varphi(n)$. Let
      ${d_1}^{a_1}{d_2}^{a_2}\cdots{d_k}^{a_k}$ be the prime factorization of
      $d$, where each $a_i$ is a positive integer and each $d_i$ is a unique
      prime. Since $d \mid n$, it follows that there exists an integer $m$ such
      that
      $$n = ({d_1}^{a_1}{d_2}^{a_2}\cdots{d_k}^{a_k})m.$$
      Now we shall factor out the maximum powers of each $d_i$ in $m$, so that 
      we can write
      $$m = ({d_1}^{c_1}{d_2}^{c_2}\cdots{d_k}^{c_k})m'$$
      where each $c_i$ is a nonnegative integer and $m'$ is an integer that is
      prime to ${d_1}^{c_1}{d_2}^{c_2}\cdots{d_k}^{c_k}$. Thus we have that
      $$n = ({d_1}^{b_1}{d_2}^{b_2}\cdots{d_k}^{b_k})m', \quad b_i = a_i + c_i$$
      so that
      \begin{align*}
         \varphi(n) &= \varphi({d_1}^{b_1}{d_2}^{b_2}\cdots{d_k}^{b_k}m') \\
                    &= \varphi({d_1}^{b_1}{d_2}^{b_2}\cdots{d_k}^{b_k})
                       \varphi(m') \\
                    &= {d_1}^{b_1 - 1}(d_1 - 1){d_2}^{b_2 - 1}(d_2 - 1)\cdots
                       {d_k}^{b_k - 1}(d_k - 1)\varphi(m') \\
                    &= {d_1}^{c_1}{d_1}^{a_1 - 1}(d_1 - 1)
                       {d_2}^{c_2}{d_2}^{a_2 - 1}(d_2 - 1)\cdots
                       {d_k}^{c_k}{d_k}^{a_k - 1}(d_k - 1)\varphi(m')\\
                    &= {d_1}^{c_1}{d_2}^{c_2}\cdots{d_k}^{c_k}\varphi(m')
                       {d_1}^{a_1 - 1}(d_1 - 1)
                       {d_2}^{a_2 - 1}(d_2 - 1)\cdots
                       {d_k}^{a_k - 1}(d_k - 1) \\
                    &= ({d_1}^{c_1}{d_2}^{c_2}\cdots{d_k}^{c_k}
                        \varphi(m'))\varphi(d),
      \end{align*}
      so that $\varphi(d) \mid \varphi(n)$. \qed
      
\end{enumerate}

      \section{$\Z/n\Z$ : The Integers Modulo $n$}
         \begin{enumerate}
%%%%%%%%%%%%%%%%%%%%%%%%%%%%%%%%%%%%%2.3.1%%%%%%%%%%%%%%%%%%%%%%%%%%%%%%%%%%%%%%
   \item[2.3.1]   Find all subgroups of $Z_{45} = \cyc{x}$, giving a generator
                  for each. Describe the containments between these subgroups.
                  
      \textbf{Solution.} Since the positive divisors of 45 are: 1, 3, 5, 9, 15,
      and 45, it follows that the subgroups of $Z_{45}$ are
      $$\cyc{x}, \cyc{x^3}, \cyc{x^5}, \cyc{x^9}, \cyc{x^{15}}, \text{ and }
        \cyc{x^{45}}.$$
        
      We have the following containments:
      $$
         \begin{tabular}{>{$}c<{$}>{$}c<{$}>{$}c<{$}>{$}c<{$}>{$}c<{$}>{$}c<{$}>{$}c<{$}}
            \cyc{x^{45}} & \le & \cyc{x^{15}} & \le & \cyc{x^5} & \le & \cyc{x} \\
            \cyc{x^{15}} & \le &  \cyc{x^3} & \le & \cyc{x} \\
            \cyc{x^9} & \le &  \cyc{x^3} & \le & \cyc{x}
         \end{tabular}
      $$
%%%%%%%%%%%%%%%%%%%%%%%%%%%%%%%%%%%%%2.3.2%%%%%%%%%%%%%%%%%%%%%%%%%%%%%%%%%%%%%%
   \item[2.3.2]   If $x$ is an element of the finite group $G$ and $|x| = |G|$,
                  prove that $G = \cyc{x}$. Give an explicit example to show 
                  that this result need not be true if $G$ is an infinite group.
                  
      \textbf{Proof.} Let $G$ be a finite group, so that $|G| = n \in \Z^+$.
      Suppose that there exists $x \in G$ such that $|x| = n$. Clearly
      $\cyc{x} \subseteq G$. But $|\cyc{x}| = n$ since $|x| = n$; thus
      $G \subseteq \cyc{x}$ so that $G = \cyc{x}$. Now let $G = \Z$. We have
      that $|\cyc{2}| = |G|$ but $G \neq \cyc{2}$. \qed
%%%%%%%%%%%%%%%%%%%%%%%%%%%%%%%%%%%%%2.3.3%%%%%%%%%%%%%%%%%%%%%%%%%%%%%%%%%%%%%%
   \item[2.3.3]   Find all generators for $\Z/48\Z$.
   
      \textbf{Solution.} The generators for $\Z/48\Z$ are: $\cyc{\overline{1}}$,
      $\cyc{\overline{5}}$, $\cyc{\overline{7}}$, $\cyc{\overline{11}}$,
      $\cyc{\overline{13}}$, $\cyc{\overline{17}}$, $\cyc{\overline{19}}$,
      $\cyc{\overline{23}}$, $\cyc{\overline{25}}$, $\cyc{\overline{29}}$,
      $\cyc{\overline{31}}$, $\cyc{\overline{35}}$, $\cyc{\overline{37}}$,
      $\cyc{\overline{41}}$, $\cyc{\overline{43}}$, and $\cyc{\overline{47}}$.
%%%%%%%%%%%%%%%%%%%%%%%%%%%%%%%%%%%%%2.3.4%%%%%%%%%%%%%%%%%%%%%%%%%%%%%%%%%%%%%%
   \item[2.3.4]   Find all generators for $\Z/202\Z$.
   
      \textbf{Solution.} Let $S$ be the set of generators for $\Z/202\Z$. Then
      $|S| = 100$ since
      $$S = \{\cyc{x} : x \text{ is odd and positive}, x \neq 101, \text{ and } x < 202\}.$$
%%%%%%%%%%%%%%%%%%%%%%%%%%%%%%%%%%%%%2.3.5%%%%%%%%%%%%%%%%%%%%%%%%%%%%%%%%%%%%%%
   \item[2.3.5]   Find the number of generators for $\Z/49000\Z$.
   
      \textbf{Solution.} For a positive integer $n$ let $\varphi(n)$ be the
      number of positive integers---less than or equal to $n$---that are
      relatively prime to $n$. Then the number of generators for $\Z/49000\Z$ is
      $\varphi(49000) = \varphi(2^35^37^2) =
      \varphi(2^3)\varphi(5^3)\varphi(7^2) = 16800$. 
%%%%%%%%%%%%%%%%%%%%%%%%%%%%%%%%%%%%%2.3.6%%%%%%%%%%%%%%%%%%%%%%%%%%%%%%%%%%%%%%
   \item[2.3.6]   In $\Z/48\Z$ write out all elements of $\cyc{\overline{a}}$
                  for every $\overline{a}$. Find all inclusions between
                  subgroups in $\Z/48\Z$.
      
      \textbf{Solution.}
      $$
         \begin{tabular}{|c|c|} \hline
            \textbf{Generators} & \textbf{Subgroups in} $\Z/48\Z$ \\ \hline
            0 & $\{0\}$ \\ \hline
            24 & $\{0, 24\}$ \\ \hline
            16, 32 & $\{0, 16, 32\}$ \\ \hline
            12, 36 & $\{0, 12, 24, 36\}$ \\ \hline
            8, 40 & $\{0, 8, 16, 24, 32, 40\}$ \\ \hline
            6, 18, 30, 42 & $\{0, 6, 12, 18, 24, 30, 36, 42\}$ \\ \hline
            4,20,28,44 & $\{0,4,8,12,16, 20, 24, 28, 32, 36, 40, 44\}$ \\ \hline
            3, 9, 15, 21, 27, 33, 39, 45 & $\{0, 3, 6, 9, 12, 15, 18, 21, 24,
            27, 30, 33, 36, 39, 42, 45\}$ \\ \hline            
            2, 10, 14, 22, 26, 34, 38, 46 & $\{x : 0 \le x \le 46,
            x \text{ is even}\}$ \\ \hline
            \text{See Exercise } 2.3.3 & $\Z/48\Z$ \\ \hline
         \end{tabular}
      $$
%%%%%%%%%%%%%%%%%%%%%%%%%%%%%%%%%%%%%2.3.7%%%%%%%%%%%%%%%%%%%%%%%%%%%%%%%%%%%%%%
   \item[2.3.7]   Let $Z_{48} = \cyc{x}$ and use the isomorphism
                  $\Z/48\Z \cong Z_{48}$ given by $\overline{1} \mapsto x$ to
                  list all subgroups of $Z_{48}$ as computed in the preceding
                  exercise.
                  
      \textbf{Solution.}
      $$
         \begin{tabular}{|c|} \hline
            \textbf{Subgroups in} $Z_{48}$ \\ \hline
            $\{1\}$ \\ \hline
            $\{1, x^{24}\}$ \\ \hline
            $\{1, x^{16}, x^{32}\}$ \\ \hline
            $\{1, x^{12}, x^{24}, x^{36}\}$ \\ \hline
            $\{1, x^8, x^{16}, x^{24}, x^{32}, x^{40}\}$ \\ \hline
            $\{1, x^6, x^{12}, x^{18}, x^{24}, x^{30},x^{36},x^{42}\}$ \\ \hline
            $\{1,x^4,x^8,x^{12},x^{16}, x^{20}, x^{24}, x^{28}, x^{32}, x^{36},
               x^{40}, x^{44}\}$ \\ \hline
            $\{1, x^3, x^6, x^9, x^{12}, x^{15}, x^{18}, x^{21}, x^{24},
            x^{27}, x^{30}, x^{33}, x^{36}, x^{39}, x^{42}, x^{45}\}$ \\ \hline
            $\{x^y : 0 \le y \le 46, y \text{ is even}\}$ \\ \hline
            $Z_{48}$ \\ \hline
         \end{tabular}
      $$
%%%%%%%%%%%%%%%%%%%%%%%%%%%%%%%%%%%%%2.3.8%%%%%%%%%%%%%%%%%%%%%%%%%%%%%%%%%%%%%%
   \item[2.3.8]   Let $Z_{48} = \cyc{x}$. For which integers $a$ does the map
                  $\varphi_a$ defined by $\varphi_a : \overline{1} \mapsto x^a$
                  extend to an \textit{isomorphism} from $\Z/48\Z$ onto
                  $Z_{48}$.
                  
      \textbf{Solution.} Suppose that $(a, 48) = 1$. Then it follows that $x^a$
      generates $Z_{48}$. Thus $\varphi_a$ is an isomorphism by Theorem 4 (Page
      56). Now suppose that $a$ is not relatively prime to 48. Then $x^a$ does
      not generate $Z_{48}$, so that the image of $\varphi_a$ is not $Z_{48}$.
      Hence $\varphi_a$ is an isomorphism if and only if $(a, 48) = 1$.
%%%%%%%%%%%%%%%%%%%%%%%%%%%%%%%%%%%%%2.3.9%%%%%%%%%%%%%%%%%%%%%%%%%%%%%%%%%%%%%%
   \item[2.3.9]   Let $Z_{36} = \cyc{x}$. For which integers $a$ does the map
                  $\psi_a$ defined by $\psi_a : \overline{1} \mapsto x^a$ extend
                  to a \textit{well defined homomorphism} from $\Z/48\Z$ into
                  $Z_{36}$. Can $\psi_a$ ever be a surjective homomorphism?
                  
      \textbf{Solution.} First we shall find the restriction(s) on $a$ such that
      $\psi_a$ is well defined. Suppose $b = c$ for some $b, c \in \Z/48\Z$. It
      suffices to show that $\psi_a(b) = \psi_a(c)$. Since $b = c$, there exists
      an integer $k$ such that $b = c + 48k$. Thus $\psi_a(b) = \psi_a(c+48k)$,
      so that
      $\psi_a(b)=(x^a)^{c+48k}=x^{ac + 48ak}= x^{ac}x^{48ak}=\psi_a(c)x^{12ak}$.
      So we must require $x^{12ak} = 1$ for all $k \in \Z$. Now $x^{12ak} = 1$
      for all $k \in \Z$ if and only if $3 \mid a$ if and only if $\psi_a$ is
      well defined. It follows immediately that
      $\psi_a$ is an homomorphism since
      \begin{align*}
         \psi_a(p + q) &= (x^a)^{p+q} \\
            &= x^{ap+aq} \\
            &= x^{ap}x^{aq} \\
            &= (x^a)^p(x^a)^q \\
            &= \psi_a(p)\psi_a(q)
      \end{align*}      
      for all $p, q \in \Z/48\Z$.
      
      \textit{Can $\psi_a$ ever be a surjective homomorphism?} No!
      
      \textbf{Proof.} Suppose to the contrary that $\psi_a$ is surjective. Then
      there exists $y \in \Z/48\Z$ such that $\psi_a(y) = x$. That is
      $x^{ay} = x$, so that $x^{ay-1} = 1$; thus $ay - 1 = 36m$ for some integer
      $m$. Rearrange the equality $ay - 1 = 36m$ to get $1 = ay - 36m$. Recall
      that $3 \mid a$; since $3$ also divides 36, it follows that 3 must divide
      1, a contradiction. Thus $\psi_a$ can never be surjective. \qed
%%%%%%%%%%%%%%%%%%%%%%%%%%%%%%%%%%%%%2.3.10%%%%%%%%%%%%%%%%%%%%%%%%%%%%%%%%%%%%%
   \item[2.3.10]  What is the order of $\overline{30}$ in $\Z/54\Z$? Write out
                  all the elements and their orders in $\cyc{\overline{30}}$.
                  
      \textbf{Solution.} The order of $30$ in $\Z/54\Z$ is
      $$\frac{54}{(30, 54)} = 9.$$
      The elements of $\cyc{30}$ and their respective orders are:
      $$
         \begin{tabular}{|c|c|} \hline
            Element of $\cyc{30}$ & Order \\ \hline
            30 & 9 \\ \hline
             6 & 9 \\ \hline
            36 & 3 \\ \hline
            12 & 9 \\ \hline
            42 & 9 \\ \hline
            18 & 3 \\ \hline
            48 & 9 \\ \hline
            24 & 9 \\ \hline
             0 & 1 \\ \hline
         \end{tabular}
      $$
%%%%%%%%%%%%%%%%%%%%%%%%%%%%%%%%%%%%%2.3.11%%%%%%%%%%%%%%%%%%%%%%%%%%%%%%%%%%%%%
   \item[2.3.11]  Find all cyclic subgroups of $D_8$. Find a proper subgroup of
                  $D_8$ which is not cyclic.
                  
      \textbf{Solution.} In $D_8$, only $r$ and $r^4$ have order 4. Thus
      $\{1, r, r^2, r^3\}$ is the only cyclic subgroup of order 4. The trivial
      subgroup is the only cyclic subgroup of order 1. Finally there are 5
      cyclic subgroups of order 2 and they are of the form $\{1, x\}$ where
      $x \in \{r^2, s, sr, sr^2, sr^3\}$. The set $\{1, s, r^2, sr^2\}$ is a
      non-cyclic proper subgroup of $D_8$.
%%%%%%%%%%%%%%%%%%%%%%%%%%%%%%%%%%%%%2.3.12%%%%%%%%%%%%%%%%%%%%%%%%%%%%%%%%%%%%%
   \item[2.3.12]  Prove that the following groups are \textit{not} cyclic:
                  \begin{enumerate}
                     \item $Z_2 \times Z_2$
                     \item $Z_2 \times \Z$
                     \item $\Z \times \Z$.
                  \end{enumerate}
      
      \textbf{Proof.}
      \begin{enumerate}
         \item The order of $Z_2 \times Z_2$ is 4, but no element in this group
               has order 4; thus $Z_2 \times Z_2$ is not cyclic.
         \item Let $Z_2 = \cyc{x}$. Observe that $Z_2 \times \Z$ is not finite,
               so in order for it to be cyclic it must be isomorphic to $\Z$.
               But this is not the case since $Z_2 \times \Z$ has two elements
               of finite order(namely $(1, 0)$ and $(x, 0)$) while $\Z$ has
               exactly 1 element of finite order.
         \item Suppose to the contrary that $\Z \times \Z$ is cyclic. Then there
               exist nonzero integers $a$ and $b$ such that
               $$\Z \times \Z = \cyc{(a,b)} = \{(na, nb) : n \in \Z\}.$$
               Thus there exists an integer $m$ such that
               $(ma, mb) = (0, 1)$. That is, $ma = 0$ and $mb = 1$. Since
               $ma = 0$, we must have $m = 0$ or $a = 0$. If $m$ is 0, then
               $(ma, mb) = (0, 0) \neq (0, 1)$, a contradiction; thus we must
               have $a = 0$, contradicting our assumption that $a$ is nonzero.
               Thus $\Z \times \Z$ is not cyclic.
      \end{enumerate} \qed
%%%%%%%%%%%%%%%%%%%%%%%%%%%%%%%%%%%%%2.3.13%%%%%%%%%%%%%%%%%%%%%%%%%%%%%%%%%%%%%
   \item[2.3.13]  Prove that the following pairs of groups are \textit{not}
                  isomorphic:
                  \begin{enumerate}
                     \item $\Z \times Z_2$ and $\Z$
                     \item $\Q \times Z_2$ and $\Q$.
                  \end{enumerate}
      
      \textbf{Proof.}
      \begin{enumerate}
         \item By Exercise 1.6.11, we know that $\Z \times Z_2$ is isomorphic to
               $Z_2 \times \Z$. By Exercise 2.3.12, $Z_2 \times \Z$ is not
               cyclic; thus $\Z \times Z_2$ is not cyclic. That is,
               $\Z \times Z_2$ is not isomorphic to $\Z$.
         \item Let $Z_2 = \cyc{x}$. It immediately follows that
               $\Q \times Z_2$ and $\Q$ are not isomorphic since $\Q \times Z_2$
               has two elements of finite order(namely $(0, 1)$ and $(0, x)$)
               while $\Q$ has exactly 1 element of finite order.
      \end{enumerate} \qed
%%%%%%%%%%%%%%%%%%%%%%%%%%%%%%%%%%%%%2.3.14%%%%%%%%%%%%%%%%%%%%%%%%%%%%%%%%%%%%%
   \item[2.3.14]  Let $\sigma =$ (1 2 3 4 5 6 7 8 9 10 11 12). For each of the
                  following integers $a$ compute $\sigma^a$:
                  $$a = 13, 65, 626, 1195, -6, -81, -570,\text{ and } {-1211}.$$
                  
      \textbf{Solution.}
      
      \begin{alignat*}{4}
         &\sigma^{13}   &&= \sigma &&\text{ } \\
         &\sigma^{65}   &&= \sigma^5 &&=
            (1\;6\;11\;4\;9\;2\;7\;12\;5\;10\;3\;8) \\
         &\sigma^{626}  &&= \sigma^2 &&= (1\;3\;5\;7\;9\;11) \\
         &\sigma^{1195} &&= \sigma^7 &&=
            (1\;8\;3\;10\;5\;12\;7\;2\;9\;4\;11\;6\;13) \\
         &\sigma^{-6} &&= \sigma^6 &&= (1\;7)
            (1\;8\;3\;10\;5\;12\;7\;2\;9\;4\;11\;6\;13) \\
         &\sigma^{-81} &&= \sigma^3 &&= (1\;4\;7\;10) \\
         &\sigma^{-570} &&= \sigma^6 &&= (1\;7) \\
         &\sigma^{-1211} &&= \sigma
      \end{alignat*}
%%%%%%%%%%%%%%%%%%%%%%%%%%%%%%%%%%%%%2.3.15%%%%%%%%%%%%%%%%%%%%%%%%%%%%%%%%%%%%%
   \item[2.3.15]  Prove that $\Q \times \Q$ is not cyclic.
   
      \textbf{Proof.} Since $\Q$ is infinite and, by Exercise 1.6.6, $\Q$ is not
      isomorphic to $\Z$, it follows that $\Q$ is not cyclic. We know that the
      subgroup of every cyclic group is cyclic; since $\Q \times\{1\} \cong \Q$,
      it follows that $\Q \times \{1\}$ is not cyclic; thus $\Q \times \Q$ is
      not cyclic because it has a noncyclic subgroup, namely $\Q \times \{1\}$.
      \qed
%%%%%%%%%%%%%%%%%%%%%%%%%%%%%%%%%%%%%2.3.16%%%%%%%%%%%%%%%%%%%%%%%%%%%%%%%%%%%%%
   \item[2.3.16]  Assume $|x| = n$ and $|y| = m$. Suppose that $x$ and $y$
                  \textit{commute}: $xy = yx$. Prove that $|xy|$ divides the
                  least common multiple of $m$ and $n$. Need this be true if $x$
                  and $y$ do \textit{not} commute? Give an example of commuting
                  elements $x$, $y$ such that the order of $xy$ is not equal to
                  the least common multiple of $|x|$ and $|y|$.
                  
      \textbf{Proof.} Let $l = \text{lcm}(m, n)$. So there exist integers
      $m'$ and $n'$ such that $mm' = nn' = l$. So we have that
      $$(xy)^l = x^ly^l = x^{nn'}y^{mm'} = (x^n)^{n'}(y^m)^{m'} = 1.$$
      That is $|xy|$ divides $l$ (by Proposition 3, Page 55).
      
      \textit{Need this be true if $x$ and $y$ do not commute?} No! Let
      $$
         A = \left(\begin{tabular}{@{}cc@{}}
            0 & 1/2 \\
            2 & 0
         \end{tabular}\right) \text{ and }
         B = \left(\begin{tabular}{@{}cc@{}}
            0 & 1 \\
            1 & 0
         \end{tabular}\right).
      $$
      A simple computation will show us that although $|A| = |B| = 2$, we have
      that $|AB| = \infty$.
      
      \textbf{Example.} Consider $\Z/2\Z = \{0, 1\}$. Let $x = y = 1$. Then we
      have $|x| = |y| = 2$, so that lcm($|x|, |y|) = 2 \neq |x + y| = |0| = 1$.
      \qed
%%%%%%%%%%%%%%%%%%%%%%%%%%%%%%%%%%%%%2.3.17%%%%%%%%%%%%%%%%%%%%%%%%%%%%%%%%%%%%%
   \item[2.3.17]  Find a presentation for $Z_n$ with one generator.
   
      \textbf{Solution.} $Z_n = \cyc{x : x^n = 1}$.
%%%%%%%%%%%%%%%%%%%%%%%%%%%%%%%%%%%%%2.3.18%%%%%%%%%%%%%%%%%%%%%%%%%%%%%%%%%%%%%
   \item[2.3.18]  Show that if $H$ is any group and $h$ is an element of $H$
                  with $h^n = 1$, then there is a unique homomorphism from
                  $Z_n = \cyc{x}$ to $H$ such that $x \mapsto h$.
                  
      \textbf{Proof.} Let $n \in \Z^+$, $Z_n = \cyc{x}$, $H$ a group, and
      $h^n  = 1$ for some $h \in H$. First we shall show the existence of a
      homomorphism from $Z_n$ to $H$ such that $x \mapsto h$. So consider the
      map $\alpha : \cyc{x} \rightarrow H$ defined by $\alpha(x^a) = h^a$.
      Clearly $\alpha(x) = h$. Now we will show that $\alpha$ is well defined.
      Suppose $x^w = x^y$ for some $x^w, x^y \in Z_n$. Thus $w = y + nk$ for
      some integer $k$. Thus
      $$\alpha(x^w) = \alpha(x^{y+nk})=h^{y+nk}=h^{y}{h^n}^k =h^y=\alpha(x^y),$$
      so that $\alpha$ is well defined. Now we have that
      $$\alpha(x^px^q)=\alpha(x^{p+q})=h^{p+q}=h^ph^q=\alpha(x^p)\alpha(x^q),$$
      so that $\alpha$ is an homomorphism. Now to show uniqueness, we suppose
      that $\phi : \cyc{x} \rightarrow H$ is an homommorphism such that
      $\phi(x) = h$. Since $\phi$ is a homomorphism, it follows that
      $\phi(x^a) = h^a$. Thus $\phi = \alpha$, as desired. \qed
%%%%%%%%%%%%%%%%%%%%%%%%%%%%%%%%%%%%%2.3.19%%%%%%%%%%%%%%%%%%%%%%%%%%%%%%%%%%%%%
   \item[2.3.19]  Show that if $H$ is any group and $h$ is an element of $H$,
                  then there is a unique homomorphism from $\Z$ to $H$ such that
                  $1 \mapsto h$.
                  
      \textbf{Proof.} Let $H$ be a group and let $h \in H$. First we shall show
      that there exists a homomorphism from $\Z$ to $H$ such that $1 \mapsto h$.
      So consider the map $\alpha : \Z \rightarrow H$ defined by
      $n \mapsto h^n$. Clearly $\alpha(1) = h$ and
      $$\alpha(x+y) = h^{x+y} = h^xh^y = \alpha(x)\alpha(y) \text{ for all }
        x, y \in \Z^+,$$
      so that $\alpha$ is a homomorphism. To show uniqueness, suppose that
      $\alpha' : \Z \rightarrow H$ is an homomorphism such that
      $\alpha'(1) = h$. Then according to Exercise 1.6.1, we have that
      $\alpha'(n) = \alpha'(n\cdot1) = \alpha'(1)^n = h^n$ for all $n \in \Z$;
      that is, $\alpha' = \alpha$, as desired. \qed
%%%%%%%%%%%%%%%%%%%%%%%%%%%%%%%%%%%%%2.3.20%%%%%%%%%%%%%%%%%%%%%%%%%%%%%%%%%%%%%
   \item[2.3.20]  Let $p$ be a prime and let $n$ be a positive integer. Show
                  that if $x$ is an element of the group $G$ such that
                  $x^{p^n} = 1$ then $|x| = p^m$ for some $m \le n$.
                  
      \textbf{Proof.} Suppose that $x \in G$ such that $x^{p^n} = 1$. Then it
      follows by Proposition 3 (Page 55) that $|x|$ divides $p^n$. Since $p$ is
      a prime, its factors are $p^i$, $0 \le i \le n$. Thus $|x| = p^m$ for
      some nonnegative $m$ not greater than $n$. \qed
%%%%%%%%%%%%%%%%%%%%%%%%%%%%%%%%%%%%%2.3.21%%%%%%%%%%%%%%%%%%%%%%%%%%%%%%%%%%%%%
   \item[2.3.21]  Let $p$ be an odd prime and let $n$ be a positive integer
                  $\ge 2$. Use the Binomial Theorem to show that
                  $(1+p)^{p^{n-1}} \equiv 1$ (mod $p^n$) but
                  $(1+p)^{p^{n-2}} \not\equiv 1$ (mod $p^n$). Deduce that $1+p$
                  is an element of order $p^{n-1}$ in the multiplicative group
                  $(\Z/p^n\Z)^\times$.

      \textbf{Lemma 2.3.1.} \textit{For an integer $n \ge 2$ and an odd prime
      $p$, let $f_p(n)$ be the number of $p$ factors of $n!$ (i.e., the greatest
      nonnegative integer $j$ such that $p^j \mid i!$), then it follows that
      $f_p(n) < \D\frac{n}{2}$}.

      \textbf{Proof.} Let $n \ge 2$ be an integer and $p$ an odd prime. For a
      a positive integer $r$, let $g_p(n, r)$ be the number of positive
      integers, less than or equal to $n$, that have at least $r$ number of $p$ 
      factors. It follows that $g_p(n, r) = \D\gint{\frac{n}{p^r}}$, where
      $\gint{x}$ is the greatest integer less than or equal to $x$. Finally let
      $k_n$ be the maximum nonnegative integer such that $p^{k_n}$ is a multiple
      of some positive integer not greater than $n$. Thus we have that
      \begin{align*}
         f_p(n) &= g_p(n, 1) + g_p(n, 2) + \cdots + g_p(n, k_n) \\
            &= \sum_{i=1}^{k_n} g_p(n, i)
            = \sum_{i=1}^{k_n} \gint{\frac{n}{p^i}} \\
            &\le \sum_{i=1}^{k_n} \frac{n}{p^i}
            < \sum_{i=1}^\infty \frac{n}{p^i} \\
            &= \frac{n}{p-1} &[\text{Sum of Geometric Series}] \\
            &< \frac{n}{2}. &[\text{Since }p \ge 3]
      \end{align*}

      So we can write $n! = p^{f_p(n)} h_n$ for some $h_n \in \Z^+$, so that
      $(h_n, p) = 1$.

      Now we are ready to commence the proof of the problem. By the Binomial
      Theorem, we have that
      \begin{align*}
         (1+p)^{p^{n-1}} &= \sum_{i=0}^{p^{n-1}}\binom{p^{n-1}}{i}p^i \\
            &= \sum_{i=0}^{p^{n-1}}p^i\frac{p^{n-1}(p^{n-1}-1)(p^{n-1}-2)
               \cdots(p^{n-1}-i+1)}{i!} \\
            &= \sum_{i=0}^{p^{n-1}}p^i\frac{p^{n-1}(p^{n-1}-1)(p^{n-1}-2)
               \cdots(p^{n-1}-i+1)}{p^{f_p(i)} h_i} \\
            &= 1 + p^n + p^n\sum_{i=2}^{p^{n-1}}\frac{p^{i-1}(p^{n-1}-1)
               (p^{n-1}-2) \cdots(p^{n-1}-i+1)}{p^{f_p(i)} h_i}.
      \end{align*}
      Now $f_p(i) < i / 2 \le i - 1$ for $i \ge 2$. Thus $i - 1 - f_p(i) \ge 0$
      (so that $p^{i - 1 - f_p(i)}$ is an integer) if $i \ge 2$. We then have
      \begin{equation} \label{2_3_21_1}
         (1+p)^{p^{n-1}} = 1 + p^n + p^n\sum_{i=2}^{p^{n-1}}\frac{p^{i-1-f_p(i)}
        (p^{n-1}-1)(p^{n-1}-2) \cdots(p^{n-1}-i+1)}{h_i}
      \end{equation}
      Since $(h_i, p) = 1$, it follows that $h_i$ must divide
      $p^{i-1}(p^{n-1}-1)(p^{n-1}-2) \cdots(p^{n-1}-i+1)$. Hence
      $$\sum_{i=2}^{p^{n-1}}\frac{p^{i-1-f_p(i)}
        (p^{n-1}-1)(p^{n-1}-2) \cdots(p^{n-1}-i+1)}{h_i}$$
      is an integer and we can conclude from \eqref{2_3_21_1} that
      $(1+p)^{p^{n-1}} \equiv 1$ (mod $p^n$). Now we have that
      \begin{align*}
         (1+p)^{p^{n-2}} &= \sum_{i=0}^{p^{n-2}}\binom{p^{n-2}}{i}p^i \\
            &= \sum_{i=0}^{p^{n-2}}p^i\frac{p^{n-2}(p^{n-2}-1)(p^{n-2}-2)
               \cdots(p^{n-2}-i+1)}{i!} \\
            &= 1 + p^{n-1} + p^n\frac{p^{n-2}-1}{2} + p^n\frac{p(p^{n-2}-1)(p^{n-2}-2)}{3!} +\sum_{i=4}^{p^{n-1}}p^i\frac{p^{n-2}(p^{n-2}-1)(p^{n-2}-2)
               \cdots(p^{n-2}-i+1)}{p^{f_p(i)} h_i} \\
            &= 1 + p^n + p^n\sum_{i=2}^{p^{n-1}}\frac{p^{i-1}(p^{n-1}-1)
               (p^{n-1}-2) \cdots(p^{n-1}-i+1)}{p^{f_p(i)} h_i}.
      \end{align*}
      
%%%%%%%%%%%%%%%%%%%%%%%%%%%%%%%%%%%%%2.3.22%%%%%%%%%%%%%%%%%%%%%%%%%%%%%%%%%%%%%
   \item[2.3.22]  Let $n$ be an integer $\ge 3$. Use the Binomial Theorem to
                  show that $(1+2^2)^{2^{n-2}} \equiv 1$ (mod $2^n$) but
                  $(1+2^2)^{2^{n-3}} \not\equiv 1$ (mod $2^n$). Deduce that 5 is
                  an element of order $2^{n-2}$ in the multiplicative group
                  $(\Z/2^n\Z)^\times$.

      \textbf{Proof.}
%%%%%%%%%%%%%%%%%%%%%%%%%%%%%%%%%%%%%2.3.23%%%%%%%%%%%%%%%%%%%%%%%%%%%%%%%%%%%%%
   \item[2.3.23]  Show that $(\Z/2^n\Z)^\times$ is not cyclic for any $n \ge 3$.
                  [Find two distinct subgroups of order 2.]
%%%%%%%%%%%%%%%%%%%%%%%%%%%%%%%%%%%%%2.3.24%%%%%%%%%%%%%%%%%%%%%%%%%%%%%%%%%%%%%
   \item[2.3.24]  Let $G$ be a finite group and let $x \in G$.
                  \begin{enumerate}
                     \item Prove that if $g \in N_G(\cyc{x})$ then
                           $gxg^{-1} = x^a$ for some $a \in \Z$. 
                     \item Prove conversely that if $gxg^{-1} = x^a$ for some
                           $a \in \Z$ then $g \in N_G(\cyc{x})$. [Show first
                           that $gx^kg^{-1} = (gxg^{-1})^k = x^{ak}$ for any
                           integer $k$, so that $g\cyc{x}g^{-1} \le \cyc{x}$.
                           If $x$ has order $n$, show the elements $gx^ig^{-1}$,
                           $i = 0, 1, \ldots, n-1$ are distinct, so that
                           $|g\cyc{x}g^{-1}| = |\cyc{x}| = n$ and conclude that
                           $g\cyc{x}g^{-1} = \cyc{x}$.]
                  \end{enumerate}
                  Note that this cuts down some of the work in computing
                  normalizers of cyclic subgroups since one does not have to
                  check $ghg^{-1} \in \cyc{x}$ for every $h \in \cyc{x}$.
%%%%%%%%%%%%%%%%%%%%%%%%%%%%%%%%%%%%%2.3.25%%%%%%%%%%%%%%%%%%%%%%%%%%%%%%%%%%%%%
   \item[2.3.25]  Let $G$ be a cyclic group of order $n$ and let $k$ be an
                  integer relatively prime to $n$. Prove that the map
                  $x \mapsto x^k$ is surjective. Use Lagrange's Theorem
                  (Exercise 1.7.19) to prove the same is true for any finite
                  group of order $n$. (For such $k$ each element has a
                  $k^{\text{th}}$ root in $G$. It follows from Cauchy's Theorem
                  in Section 3.2 that if $k$ is not relatively prime to the
                  order of $G$ then the map $x \mapsto x^k$ is not surjective.)
%%%%%%%%%%%%%%%%%%%%%%%%%%%%%%%%%%%%%2.3.26%%%%%%%%%%%%%%%%%%%%%%%%%%%%%%%%%%%%%
   \item[2.3.26]  Let $Z_n$ be a cyclic group of order $n$ and for each integer
                  $a$ let
                  $$\sigma_a : Z_n \mapsto Z_n \qquad by \qquad \sigma_a(x) =
                  x^a \quad \text{for all } x \in Z_n.$$
                  \begin{enumerate}
                     \item Prove that $\sigma_a$ is an automorphism of $Z_n$ if
                           and only if $a$ and $n$ are relatively prime(
                           automorphisms were introduced in Exercise 1.6.20).
                     \item Prove that $\sigma_a = \sigma_b$ if and only if
                           $a \equiv b$ (mod $n$).
                     \item Prove that \textit{every} automorphism of $Z_n$ is
                           equal to $\sigma_a$ for some integer $a$.
                     \item Prove that $\sigma_a\circ\sigma_b=\sigma_{ab}$.
                           Deduce that the map $\overline{a} \mapsto \sigma_a$
                           is an isomorphism of $(\Z/n\Z)^\times$ onto the
                           automorphism group of $Z_n$ (so Aut($Z_n$) is an
                           abelian group of order $\varphi(n)$).
                  \end{enumerate}
                  %%%%%MISSING CONTAINMENT%%%%%%%%
\end{enumerate}


































         
   \part{}

   \chapter{Introduction To Groups}
      \section{Basic Axioms And Examples}
         Let
   $$\mathcal{A} \mbox{ be the set of real } 2 \times 2
     \mbox{ matrices}, M = \left(
      \begin{tabular}{@{}l r@{}} 
         1 & 1\\ 
         0 & 1
      \end{tabular}\right)$$
and let
   $$\mathcal{B} = \{X \in \mathcal{A} : MX = XM\}.$$
\begin{enumerate}
%%%%%%%%%%%%%%%%%%%%%%%%%%%%%%%%%%%Prob0.1.1%%%%%%%%%%%%%%%%%%%%%%%%%%%%%%%%%%%%
   \item[0.1.1] Determine which of the following elements of $\mathcal{A}$ lie
                in $\mathcal{B}$:

                $$M = \left(
                   \begin{tabular}{@{}l r@{}} 
                      1 & 1 \\ 
                      0 & 1
                   \end{tabular}\right), \left(
                   \begin{tabular}{@{}l r@{}} 
                      1 & 1 \\ 
                      1 & 1
                   \end{tabular}\right), \left(
                   \begin{tabular}{@{}l r@{}} 
                      0 & 0 \\ 
                      0 & 0
                   \end{tabular}\right), \left(
                   \begin{tabular}{@{}l r@{}} 
                      1 & 1 \\ 
                      1 & 0
                   \end{tabular}\right), \left(
                   \begin{tabular}{@{}l r@{}} 
                      1 & 0 \\ 
                      0 & 1
                   \end{tabular}\right), \left(
                   \begin{tabular}{@{}l r@{}} 
                      0 & 1 \\ 
                      1 & 1
                   \end{tabular}\right).$$
      \textbf{Solution:} A quick computation will show us only these matrices
      lie in $\mathcal{B}$.

                $$\left(
                   \begin{tabular}{@{}l r@{}} 
                      1 & 1 \\ 
                      0 & 1
                   \end{tabular}\right), \left(
                   \begin{tabular}{@{}l r@{}} 
                      0 & 0 \\ 
                      0 & 0
                   \end{tabular}\right), \left(
                   \begin{tabular}{@{}l r@{}} 
                      1 & 0 \\ 
                      0 & 1
                   \end{tabular}\right).$$
%%%%%%%%%%%%%%%%%%%%%%%%%%%%%%%%%%%Prob0.1.2%%%%%%%%%%%%%%%%%%%%%%%%%%%%%%%%%%%%
   \item[0.1.2] Prove that if $P, Q \in \mathcal{B}$, then
                $P + Q \in \mathcal{B}$.

      \textbf{Proof:} Suppose that $P, Q \in \mathcal{B}$. We want to show that
      $P + Q \in \mathcal{B}$. That is, we must show that $M(P + Q) = (P + Q)M$.
      Thus
      \begin{align*}
         M(P + Q) &= MP + MQ   &[\text{Distributive Law}]\\
                  &= PM + QM   &[P, Q \in \mathcal{B}] \\
                  &= (P + Q)M. &[\text{Distributive Law}]
      \end{align*} \qed
%%%%%%%%%%%%%%%%%%%%%%%%%%%%%%%%%%%Prob0.1.3%%%%%%%%%%%%%%%%%%%%%%%%%%%%%%%%%%%%
   \item[0.1.3] Prove that if $P, Q \in \mathcal{B}$, then
                $P \cdot Q \in \mathcal{B}$.

      \textbf{Proof:} Suppose that $P, Q \in \mathcal{B}$. We want to show that
      $PQ \in \mathcal{B}$. That is, we must show that $M(PQ) = (PQ)M$.
      Then
      \begin{align*}
         M(PQ) &= (MP)Q  &[\text{Associative Law}]\\
               &= (PM)Q  &[P \in \mathcal{B}] \\
               &= P(MQ)  &[\text{Associative Law}] \\
               &= P(QM)  &[Q \in \mathcal{B}] \\
               &= (PQ)M.  &[\text{Associative Law}]
      \end{align*} \qed
%%%%%%%%%%%%%%%%%%%%%%%%%%%%%%%%%%%Prob0.1.4%%%%%%%%%%%%%%%%%%%%%%%%%%%%%%%%%%%%
   \item[0.1.4] Find conditions on $p, q, r, s$ which determine precisely when
                $$\left(
                     \begin{tabular}{@{}l r@{}} 
                        $p$ & $q$ \\ 
                        $r$ & $s$
                     \end{tabular}\right) \in \mathcal{B}.$$

      \textbf{Proof:} Suppose $A = \left(\begin{tabular}{@{}l r@{}} 
      $p$ & $q$ \\ 
      $r$ & $s$
      \end{tabular}\right) \in \mathcal{B}$. Then it must be the case that
      $AM = MA$, so that
      $$ \left(
         \begin{tabular}{@{}l r@{}} 
            $p$ & $q$ \\ 
            $r$ & $s$
         \end{tabular}\right)\left(
         \begin{tabular}{@{}l r@{}} 
            1 & 1 \\ 
            0 & 1
         \end{tabular}\right) = \left(
         \begin{tabular}{@{}l r@{}} 
            1 & 1 \\ 
            0 & 1
         \end{tabular}\right)\left(
         \begin{tabular}{@{}l r@{}} 
            $p$ & $q$ \\ 
            $r$ & $s$
         \end{tabular}\right).
      $$

      That is,
      $$ \left(
         \begin{tabular}{@{}l r@{}} 
            $p$ & $p + q$ \\ 
            $r$ & $r + s$
         \end{tabular}\right) = \left(
         \begin{tabular}{@{}c c@{}} 
            $p + r$ & $q + s$ \\ 
            $r$ & $s$
         \end{tabular}\right).
      $$

      Equating corresponding entries tells us that
      $r = 0$ and $p = s$. Thus
      $$\mathcal{B} = \left\{\left(
         \begin{tabular}{@{}l r@{}} 
            $p$ & $q$ \\ 
            $0$ & $p$
         \end{tabular}\right) \in \mathcal{A}\right\}.$$
%%%%%%%%%%%%%%%%%%%%%%%%%%%%%%%%%%%Prob0.1.5%%%%%%%%%%%%%%%%%%%%%%%%%%%%%%%%%%%%
   \item[0.1.5] Determine whether the following functions $f$ are well defined:
                \begin{enumerate}
                   \item $f : \Q \rightarrow \Z$ defined by $f(a/b) = a$.
                   \item $f : \Q \rightarrow \Q$ defined by $f(a/b) = a^2/b^2$.
                \end{enumerate}

      \textbf{Solution:}
         \begin{enumerate}
            \item $f$ is not well defined because $4/1 = 8/2$, but
                  $f(4/1) = 4 \neq 8 = f(8/2)$.
            \item We claim that $f$ is well defined.

                  \textbf{Proof:} We want to show that all representatives for 
                  an element in $\Q$ have the same output under $f$. So let
                  $a/b \in \Q$ with $\gcd(a, b) = 1$. Suppose $m \in \Q$ such 
                  that $m = a/b$. Then we must have that $m = ka/kb$ where $k$ 
                  is some nonzero integer. It follows that
                  $f(ka/kb) = k^2a^2/k^2b^2 = a^2/b^2 = f(a/b)$. \qed
         \end{enumerate}
%%%%%%%%%%%%%%%%%%%%%%%%%%%%%%%%%%%Prob0.1.6%%%%%%%%%%%%%%%%%%%%%%%%%%%%%%%%%%%%
   \item[0.1.6] Determine whether the function $f: \R^+ \rightarrow \Z$ defined
                by mapping a real number $r$ to the first digit to the right of
                the decimal point in a decimal expansion of $r$ is well defined.

      \textbf{Proof:} The map $f$ is not well defined because we have
      $0.\overline{9} = 1.\overline{0}$, but
      $f(0.\overline{9}) = 9 \neq 0 = f(1.\overline{0})$. \qed
%%%%%%%%%%%%%%%%%%%%%%%%%%%%%%%%%%%Prob0.1.7%%%%%%%%%%%%%%%%%%%%%%%%%%%%%%%%%%%%
   \item[0.1.7] Let $f : A \rightarrow B$ be a surjective map of sets. Prove
                that the relation
                $$a \sim b \mbox{ if and only if } f(a) = f(b)$$
                is an equivalence relation whose equivalence classes are the
                fibers of $f$.

      \textbf{Proof:} We want to show that the relation $\sim$ is an equivalence
      relation on $A$. That is, we want to show that $\sim$ is reflexive,
      symmetric, and transitive on $A$. Thus
      \begin{enumerate}
         \item[\textit{Reflexivity}:]  Let $a \in A$. Since $f(a) = f(a)$, it
                                       follows that $a \sim a$, so that $\sim$ 
                                       is reflexive on $A$.
         \item[\textit{Symmetry}:]     Let $a, b \in A$ with $a \sim b$. Since
                                       $a \sim b$, we must have that
                                       $f(a) = f(b)$; it immediately follows    
                                       that $f(b) = f(a)$, so that $b \sim a$;
                                       that is, $\sim$ is symmetric on $A$.
         \item[\textit{Transitivity}:] Let $a, b, c \in A$ with $a \sim b$ and
                                       $b \sim c$. Thus by definition, we have
                                       that $f(a) = f(b)$ and $f(b) = f(a)$. By
                                       transitivity of the equality relation it
                                       follows that $f(a) = f(c)$, so that
                                       $a \sim c$. Thus $\sim$ is transitive on
                                       $A$.
      \end{enumerate}

      Now we must show that the fibers of $f$ are the equivalence classes of
      $\sim$. So let $\mathcal{C}$ be the set of equivalence classes of $\sim$ 
      and let  $\mathcal{F}$ be the set of fibers of $f$. First we want to show
      that $\mathcal{F} \subseteq \mathcal{C}$. So consider
      $F \in \mathcal{F}$. Let $h \in F$. Then we see that $\overline{h} = F$
      (because if $h' \in \overline{h}$ then $f(h') = f(h)$, so that $h' \in F$,
      and $\overline{h} \subseteq F$. Also if $g \in F$, then $f(h) = f(g)$ so  
      that $f \in \overline{h}$ and $F \subseteq \overline{h}$), so that $F \in
      \mathcal{C}$ and $\mathcal{F} \subseteq \mathcal{C}$. Now we want to show
      that $\mathcal{C} \subseteq \mathcal{F}$ to complete the proof. Suppose
      $C \in \mathcal{C}$. Let $c \in C$. Then $C$ is the fiber of $f(c)$. Thus
      $C \in \mathcal{F}$ so that $\mathcal{C} \subseteq \mathcal{F}$. Since
      $\mathcal{F} \subseteq \mathcal{C}$ and
      $\mathcal{C} \subseteq \mathcal{F}$, we must have that
      $\mathcal{C} = \mathcal{F}$. \qed
%%%%%%%%%%%%%%%%%%%%%%%%%%%%%%%%%%%Prob0.1.8%%%%%%%%%%%%%%%%%%%%%%%%%%%%%%%%%%%%
   \item[0.1.8] \textbf{Proposition 1:}
                \begin{enumerate}
                   \item The map $f$ is injective if and only if $f$ has a left
                         inverse.
                   \item The map $f$ is surjective if and only if $f$ has a
                         right inverse.
                   \item The map $f$ is a bijection if and only if there exists
                         $g : B \rightarrow A$ such that $f \circ g$ is the
                         identity map on $B$ and $g \circ f$ is the identity map
                         on $A$.
                   \item If $A$ and $B$ are finite sets with the same number of
                         elements, then $f : A \rightarrow B$ is bijective if
                         and only if $f$ is injective if and only if $f$ is
                         surjective.
                \end{enumerate}

      \textbf{Proof:}
      \begin{enumerate}
         \item ($\Leftarrow$) Suppose first that the map $f : A \rightarrow B$ 
               has a left inverse, say $g : B \rightarrow A$. We want to show 
               that $f$ has is injective. So consider $a, b \in A$ such that
               $f(a) = f(b)$. Then we have that $g(f(a)) = g(f(b))$ so that
               $a = b$ since $g$ is a left inverse of $f$. Thus $f$ is
               injective. \\
               ($\Rightarrow$) Now suppose that $f$ is injective. Then we must 
               show that $f$ has	a left inverse. Notice that since $f$ is 
               injective, it must be the case that for every $b'' \in f(A)$ 
               there exists a unique $a'' \in A$ such that $f(a'') = b''$. Let 
               $a'$ be any member of $A$ and consider the map
	            $$g : B \rightarrow A,$$
	            where for every $b \in B$, we have
	            \begin{equation*}
		            g(b) = \left\{
			            \begin{array}{ll}
				            a & \text{if } b \in f(A) \text{ and where } f(a) = b,\\
                        a' & \text{if } b \in B\setminus f(A).
                     \end{array} \right.
               \end{equation*}
   
               So $(g \circ f)(a_1) = g(f(a_1)) = a_1$ for all $a_1 \in A$.
               Hence $g \circ f$ is the identity function on $A$, so that $g$ is
               a left inverse of $f$.

         \item $(\Leftarrow)$ Suppose that $f$ has a right inverse, say
               $g : B \rightarrow A$. We must now show that $f$ is also 
               surjective; so let $b \in B$. Then, by supposition, we have that 
               $f(g(b)) = b$. Thus $f$ maps $g(b)$---a member of $A$---to $b$, 
               so that $f$ is onto.
   
               $(\Rightarrow)$ Now suppose that $f$ is surjective. We now want 
               to show that $f$ has a right inverse. Let $b \in B$. Since the 
               surjectivity of $f$ guarantees that the fiber of $\{b\}$ over $f$
               is not empty, we let $h : B \rightarrow A$ be a function that 
               maps an element $b \in B$ to some element in the fiber of $\{b\}$
               over $f$. Clearly if $c \in B$, then $f(h(c)) = c$, so that $h$ 
               is a right inverse of $f$.
         \item $(\Leftarrow)$ Suppose that $g$ is a left and right inverse of
               $f$. Then by 1 and 2, $f$ is an injection and a surjection, so 
               that $f$ is a bijection.
   
               $(\Rightarrow)$ Now suppose that $f$ is a bijection. Let
               $b \in B$. We notice that the fiber of $\{b\}$ under $f$ is not 
               empty since $f$ is surjecitive, and this fiber contains exactly 
               one element of $A$. The latter is so since if $a_1, a_2$ are in 
               the fiber of $\{b\}$ over $f$, then $f(a_1) = f(a_2)$ so that 
               $a_1 = a_2$ by the injectivity of $f$. So let $g$ be the map
               $g : B \rightarrow A$ that maps $c \in B$ to the only element in 
               the fiber of $\{c\}$ over $f$. It is trivial to show that $g$ is 
               both a left and right inverse of $f$.

         \item Let $|A| = |B| = n \in \Z^+$. First we shall assume that $f$ is
               bijective. It immediately follows that $f$ is injective. Now 
               assume that $f$ is injective. Since $f$ is one to one, no two 
               elements of $A$ map to the same element in $B$. This implies that
               $f(A)$ must contain exactly $n$ elements since $A$ contains $n$ 
               elements. But $|B| = n$, so that $f(A) = B$. That is, $f$ is 
               surjective. Now suppose that $f$ is surjective. Since $f$ is
               surjective, none of its fibers is empty; thus, the number of
               fibers of $f$ must equal $|B| = n$. We shall argue by    
               contradiction that $f$ is injective. So suppose that $f$ is not
               injective. Let $f_1$, $f_2$, $\ldots$, $f_n$ be the $n$ fibers of
               $f$. Since $f$ is not injective, one of this fibers must contain
               more than 1 element. so assume without loss of generality that
               $|f_1| \ge 2$. Since these fibers are a partition of $A$ and
               since each fiber contains at least one element it follows that
               \begin{align*}
                  |A| &= |f_1| + |f_2| + \cdots + |f_n| \\
                      &\ge 2 + (n - 1) \\
                      &\ge n + 1,
               \end{align*}
               a contradiction since $|A| = n$. Thus $f$ is injective.	
      \end{enumerate}
\end{enumerate}

      \section{Dihedral Groups}
         \begin{enumerate}
%%%%%%%%%%%%%%%%%%%%%%%%%%%%%%%%%%%%%0.2.1%%%%%%%%%%%%%%%%%%%%%%%%%%%%%%%%%%%%%%
   \item[0.2.1]   For each of the following pairs of integers $a$ and $b$,
                  determine their greatest common divisor, ther least common
                  multiple, and write their greatest common divisor in the form
                  $ax + by$ for some integers $x$ and $y$.
                  \begin{enumerate}
                     \item $a = 20$, $b = 13$.
                     \item $a = 69$, $b = 372$.
                     \item $a = 792$, $b = 275$.
                     \item $a = 11391$, $b = 5673$.
                     \item $a = 1761$, $b = 1567$.
                     \item $a = 507885$, $b = 60808$.
                  \end{enumerate}

      \textbf{Solution.}

      \begin{enumerate}
         \item Using the Euclidean Algorithm we have
               \begin{align*}
                  20 &= 1 \cdot 13 + 7 \\
                  13 &= 1 \cdot 7 + 6 \\
                  7  &= 1 \cdot 6 + 1, \text{ so that } \\ \\
                  1  &= 7 - 1 \cdot 6 \\
                     &= 7 - 1 \cdot (13 - 1 \cdot 7) \\
                     &= 2 \cdot 7 - 1 \cdot 13 \\
                     &= 2 \cdot (20 - 1 \cdot 13) - 1 \cdot 13 \\
                     &= 2 \cdot 20 - 3 \cdot 13.
               \end{align*}

               Thus $\gcd(20, 13) = 1$ and we have $x = 2$ and $y = -3$.
         \item Using the Euclidean Algorithm we have
               \begin{align*}
                  69  &= 0 \cdot 372 + 69 \\
                  372 &= 5 \cdot 69 + 27 \\
                  69  &= 2 \cdot 27 + 15 \\
                  27  &= 1 \cdot 15 + 12 \\
                  15  &= 1 \cdot 12 + 3 \\
                  12  &= 4 \cdot 3 + 0, \text{ so that } \\ \\
                   3 &= 15 - 1 \cdot 12 \\
                     &= 15 - 1 \cdot (27 - 1 \cdot 15) \\
                     &= 2 \cdot 15 - 1 \cdot 27 \\
                     &= 2 \cdot (69 - 2 \cdot 27) - 1 \cdot 27 \\
                     &= 2 \cdot 69 - 5 \cdot 27 \\
                     &= 2 \cdot 69 - 5 \cdot (372 - 5 \cdot 69) \\
                     &= 27 \cdot 69 - 5 \cdot 372 \\
                     &= 27 \cdot 69 - 5 \cdot 372.
               \end{align*}

               Thus $\gcd(69, 372) = 3$ and we have $x = 27$ and $y = -5$.
         \item Using the Euclidean Algorithm we have
               \begin{align*}
                  792 &= 2 \cdot 275 + 242 \\
                  275 &= 1 \cdot 242 + 33 \\
                  242 &= 7 \cdot 33 + 11 \\
                  33  &= 3 \cdot 11 + 0, \text{ so that } \\ \\
                  11  &= 242 - 7 \cdot 33 \\
                      &= 242 - 7 \cdot (275 - 1 \cdot 242) \\
                      &= 8 \cdot 242 - 7 \cdot 275 \\
                      &= 8 \cdot (792 - 2 \cdot 275) - 7 \cdot 275 \\
                      &= 8 \cdot 792 - 23 \cdot 275.
               \end{align*}

               Thus $\gcd(792, 275) = 11$ and we have $x = 8$ and $y = -23$.
         \item Using the Euclidean Algorithm we have
               \begin{align*}
                  11391 &= 2 \cdot 5673 + 45 \\
                  5673  &= 126 \cdot 45 + 3 \\
                  45    &= 3 \cdot 15 + 0, \text{ so that } \\ \\
                  3     &= 5673 - 126 \cdot 45 \\
                        &= 5673 - 126 \cdot (11391 - 2 \cdot 5673) \\
                        &= -126 \cdot 11391 + 253 \cdot 5673.
               \end{align*}

               Thus $\gcd(11391, 5673) = 3$ and we have $x = -126$ and
               $y = 253$.
         \item Using the Euclidean Algorithm we have
               \begin{align*}
                  1761 &= 1 \cdot 1567 + 194 \\
                  1567 &= 8 \cdot 194 + 15 \\
                  194  &= 12 \cdot 15 + 14 \\
                  15   &= 1 \cdot 14 + 1, \text{ so that } \\ \\
                   1   &= 15 - 1 \cdot 14 \\
                       &= 15 - 1 \cdot (194 - 12 \cdot 15) \\
                       &= 13 \cdot 15 - 1 \cdot 194 \\
                       &= 13 \cdot (1567 - 8 \cdot 194) - 1 \cdot 194 \\
                       &= 13 \cdot 1567 - 105 \cdot 194 \\
                       &= 13 \cdot 1567 - 105 \cdot (1761 - 1 \cdot 1567) \\
                       &= 118 \cdot 1567 - 105 \cdot 1761.
               \end{align*}

               Thus $\gcd(1761, 1567) = 1$ and we have $x = -105$ and $y = 118$.
         \item Using the Euclidean Algorithm we have
               \begin{align*}
                  507885 &= 8 \cdot 60808 + 21421 \\
                  60808  &= 2 \cdot 21421 + 17966 \\
                  21421  &= 1 \cdot 17966 + 3455 \\
                  17966  &= 5 \cdot 3455 +  691 \\
                  3455   &= 5 \cdot 691 + 0, \text{ so that } \\ \\
                  691    &= 17966 - 5 \cdot 3455 \\
                         &= 17966 - 5 \cdot (21421 - 1 \cdot 17966) \\
                         &= 6 \cdot 17966 - 5 \cdot 21421 \\
                         &= 6 \cdot (60808 - 2 \cdot 21421) - 5 \cdot 21421 \\
                         &= 6 \cdot 60808 - 17 \cdot 21421 \\
                         &= 6 \cdot 60808 - 17 \cdot (507885 - 8 \cdot 60808) \\
                         &= 142 \cdot 60808 - 17 \cdot 507885.
               \end{align*}

               Thus $\gcd(507885, 60808) = 691$ and we have $x = -17$ and
               $y = 142$.
      \end{enumerate}
%%%%%%%%%%%%%%%%%%%%%%%%%%%%%%%%%%%%%0.2.2%%%%%%%%%%%%%%%%%%%%%%%%%%%%%%%%%%%%%%
   \item[0.2.2]   Prove that if the integer $k$ divides the integers $a$ and $b$
                  then $k$ divides $as + bt$ for every pair of integers $s$ and
                  $t$.

      \textbf{Proof.} Let $a$ and $b$ be integers. Assume that $k$ divides
      $a$ and $b$. Consider any pair of integers $s$ and $t$. We want to show
      that $k$ also divides $as + bt$; that is, we must show that there exists
      some integer $m_1$ such that $as + bt = km_1$. Since $k$ divides $a$ and
      $b$, we must have that $a = km_2$ and $b = km_3$ for some integers $m_2$
      and $m_3$. Thus
      \begin{align*}
         as + bt &= km_2s + km_3t \\
                 &= k(m_2s + m_3t). 
      \end{align*}

      So take $m_1 = m_2s + m_3t$. \qed
%%%%%%%%%%%%%%%%%%%%%%%%%%%%%%%%%%%%%0.2.3%%%%%%%%%%%%%%%%%%%%%%%%%%%%%%%%%%%%%%
   \item[0.2.3]   Prove that if $n$ is composite then there are integers $a$ and
                  $b$ such that $n$ divides $ab$ but $n$ does not divide either
                  $a$ or $b$.

      \textbf{Proof.} Let $n > 1$ be a composite integer. Then $n = cd$, where
      $c$ and $d$ are integers greater than 1. Clearly, $n \mid cd$ because
      $n = cd$, but $n$ divides neither $c$ nor $d$, since they are both less
      than $n$. \qed
%%%%%%%%%%%%%%%%%%%%%%%%%%%%%%%%%%%%%0.2.4%%%%%%%%%%%%%%%%%%%%%%%%%%%%%%%%%%%%%%
   \item[0.2.4]   Let $a$, $b$ and $N$ be fixed integers with $a$ and $b$ 
                  nonzero and let $d = (a, b)$ be the greatest common divisor of
                  $a$ and $b$. Suppose $x_0$ and $y_0$ are particular solutions
                  to $ax + by = N$. Prove for any integer $t$ that the integers
                  $$x = x_0 + \frac{b}{d}t \qquad y = y_0 - \frac{a}{d}t$$
                  are also solutions to $ax + by = N$.

      \textbf{Proof.} Let $t$ be an integer, and let
		$$x = x_0 + \frac{b}{d}t \text{ and } y = y_0 - \frac{a}{d}t.$$
		Then we have
		\begin{align*}
			ax + by &= a\left(x_0 + \frac{b}{d}t\right) +
						  b\left(y_0 - \frac{a}{d}t\right) \\
					  &= ax_0 + by_0 \\
					  &= N,					  
		\end{align*}
		so that $x = x_0 + \frac{b}{d}t$ and $y = y_0 - \frac{a}{d}t$ are
		solutions to the equation $ax + by = N$.
%%%%%%%%%%%%%%%%%%%%%%%%%%%%%%%%%%%%%0.2.5%%%%%%%%%%%%%%%%%%%%%%%%%%%%%%%%%%%%%%
   \item[0.2.5]   Determine the value $\varphi(n)$ for each integer $n \le 30$
                  where $\varphi$ denotes the Euler $\varphi-$function.  

      \textbf{Solution.}

      We shall be making use of the multiplicative property of the Euler
		$\varphi-$function. So 

      \begin{center}
         \begin{tabular}{@{}l c r c l c l c r@{}}
            $\varphi(1)$ & = & 1, & & $\varphi(2)$ & = & 1, & & \\
            $\varphi(3)$ & = & 2, & & $\varphi(4)$ & = &
            $\varphi(2^2)$ & = & 2, \\
            $\varphi(5)$ & = & 4, & & $\varphi(6)$ & = &
            $\varphi(2)\varphi(3)$ & = & 2, \\
            $\varphi(7)$ & = & 6, & & $\varphi(8)$ & = &
            $\varphi(2^3)$ & = & 4, \\
            $\varphi(9) = \varphi(3^2)$ & = & 6, & & $\varphi(10)$ & = &
            $\varphi(2)\varphi(5)$ & = & 4, \\
            $\varphi(11)$ & = & 10, & & $\varphi(12)$ & = &
            $\varphi(3)\varphi(4)$ & = & 4, \\
            $\varphi(13)$ & = & 12, & & $\varphi(14)$ & = &
            $\varphi(2)\varphi(7)$ & = & 6, \\
            $\varphi(15) = \varphi(3)\varphi(5)$ & = & 8, & &
            $\varphi(16)$ & = &  $\varphi(2^4)$ & = & 8, \\
            $\varphi(17)$ & = & 16, & & $\varphi(18)$ & = &
            $\varphi(2)\varphi(9)$ & = & 6, \\
            $\varphi(19)$ & = & 18, & & $\varphi(20)$ & = &
            $\varphi(4)\varphi(5)$ & = & 8, \\
            $\varphi(21) = \varphi(3)\varphi(7)$ & = & 12, & &
            $\varphi(22)$ & = & $\varphi(2)\varphi(11)$ & = & 10, \\
            $\varphi(23)$ & = & 22, & &
            $\varphi(24)$ & = & $\varphi(3)\varphi(8)$ & = & 8, \\
            $\varphi(25) = \varphi(5^2)$ & = & 20, & &
            $\varphi(26)$ & = & $\varphi(2)\varphi(13)$ & = & 12, \\
            $\varphi(27) = \varphi(3^3)$ & = & 18, & &
            $\varphi(28)$ & = & $\varphi(4)\varphi(7)$ & = & 12, \\
            $\varphi(29)$ & = & 28, & &
            $\varphi(30)$ & = & $\varphi(2)\varphi(15)$ & = & 8. \\
         \end{tabular}
      \end{center}
%%%%%%%%%%%%%%%%%%%%%%%%%%%%%%%%%%%%%0.2.6%%%%%%%%%%%%%%%%%%%%%%%%%%%%%%%%%%%%%%
   \item[0.2.6]   Prove the Well Ordering Principle of $\Z$ by induction and
                  prove the minimal element is unique.

      \textbf{Proof.} Let $P$ be a nonempty subset of $\Z^+$. We want to show 
      that $P$ has a minimal element. So suppose by way of contradiction that
      $P$ does not have a minimal element. For a natural number $n$, let $S(n)$ 
      be the statement that $n$ is not a member of $P$. We now want to show that
      by Strong Induction that $S(n)$ holds for every natural number $n$. If 1
      is in $P$, then it would be the smallest member of $P$, contradicting our
      assumption that $P$ has no minimal element, so $1 \notin P$; hence $S(1)$ 
      is true. Now suppose that $S(j)$ is true for every natural number $j < k$,
      where $k$ is a natural number greater than 1. By our supposition, we know
      that every integer less than $k$ is not in $P$, so if $k$ is in $P$, it
      would be the minimal element of $P$, contradicting our assumption that $P$
      has no minimal element. Thus $S(k)$ is true. It follows by Mathematical
      Induction that $S(n)$ holds for every positive integer $n$. That is, $P$
      is empty, a contradiction. We can now conclude that $P$ has a minimal 
      element, say $p$. To show that $p$ is unique assume that $q$ is also a
      minimal element of $P$. By virtue of $p$ as a minimal element of $P$, we 
      have $p \le q$ and, by virtue of $q$ as a minimal element of $P$, we have
      $q \le p$, so that $p = q$. Hence the minimal element of $P$ is
      unique. \qed
%%%%%%%%%%%%%%%%%%%%%%%%%%%%%%%%%%%%%0.2.7%%%%%%%%%%%%%%%%%%%%%%%%%%%%%%%%%%%%%%
   \item[0.2.7]   If $p$ is a prime prove that there do not exist nonzero
                  integers $a$ and $b$ such that $a^2 = pb^2$.

      \textbf{Proof.} Let $p$ be a prime number. Suppose by contradiction that 
      there exist nonzero integers $a$ and $b$ such that $a^2 = pb^2$. We can
      further suppose that $a$ and $b$ are relatively prime. Since $a^2 = pb^2$,
      it follows that $a^2$ has $p$ as one of its prime factors, so that $a$
      also has $p$ as one of its prime factors. We can then write $a = pm$
      for some integer $m$. Substituting $a = pm$ in the equation $a^2 = pb^2$,
      will give us the equation $pm^2 = b^2$. We can similarly conclude that
      $b$ has $p$ as one of its prime factors, so that $\gcd(a, b) \ge p$, a
      contradiction. Thus there do not exist nonzero integers $a$ and $b$ such 
      that $a^2 = pb^2$. \qed
%%%%%%%%%%%%%%%%%%%%%%%%%%%%%%%%%%%%%0.2.8%%%%%%%%%%%%%%%%%%%%%%%%%%%%%%%%%%%%%%
   \item[0.2.8]   Let $p$ be a prime, $n \in \Z^+$. Find a formula for the
                  largest power of $p$ which divides
                  $n! = n(n - 1)(n - 2)\cdots2 \cdot 1$ (it involves the
                  greatest integer function).

      \textbf{Proof.} Let $p$ be a prime and let $n$ be a positive integer. The
      largest power of $p$ that divides $n!$, say $k$, is simply the number of 
      multiples of $p$ in the set $\{1, 2, \ldots, n\}$. Thus
      $k = \lfloor{n/p}\rfloor$, where $\lfloor{x}\rfloor$ is the greatest
      integer less than the real number $x$.
%%%%%%%%%%%%%%%%%%%%%%%%%%%%%%%%%%%%%0.2.9%%%%%%%%%%%%%%%%%%%%%%%%%%%%%%%%%%%%%%
   \item[0.2.9]   Write a computer program to determine the greatest common
                  divisor $(a, b)$ of two integers $a$ and $b$ and to express
                  $(a, b)$ in the form $ax + by$ for some integers $x$ and $y$.

   \begin{verbatim}
# Python
# For positive integers a and b gcd(a, b) returns
# a tuple (r, x, y) where r = gcd(a, b) and xa + yb = r
def gcd(a, b, x1 = 1, y1 = 0, x2 = 0, y2 = 1):
   q = a // b
   r = a % b

   if r == 0:
      return (b, 0, 1)

   x1 = x1 - q * x2
   y1 = y1 - q * y2

   if b % r == 0:
      return (r, x1, y1)

   return gcd(b, r, x2, y2, x1, y1)
   \end{verbatim}
%%%%%%%%%%%%%%%%%%%%%%%%%%%%%%%%%%%%%0.2.10%%%%%%%%%%%%%%%%%%%%%%%%%%%%%%%%%%%%%
   \item[0.2.10]  Prove for any given positive integer $N$ there exist only
                  finitely many integers $n$ with $\varphi(n) = N$ where
                  $\varphi$ denotes Euler's $\varphi$-function. Conclude in 
                  particular that $\varphi(n)$ tends to infinity as $n$ tends to
                  infinity.

      \textbf{Proof.} Let $N \in \N$ and define
      $$S_N = \{n \in \N : \varphi(n) = N\}.$$
      
      We want to show that $S_N$ is finite. So it suffices to show that $S_N$ is
      bounded; i.e., there exists a positive integer $K$ such that $n < K$ for
      all $n \in S_N$. If $S_N$ is empty, then we are done, so assume that $S_N$
      is nonempty. Let $m \in S_N$. If $m = 1$, then $N = 1$, and thus
      $S_1 = \{1, 2\}$ is finite. So assume $m > 1$. By the Fundamental 
      Theorem of Arithmetic, it follows that
      $$m = {p_1}^{c_1}{p_2}^{c_2}\cdots{p_s}^{c_s},$$
      where the $p_i$s are mutually distinct primes, $s$ and the $c_i$s are
      positive integers, and $s < N$ (the number of distinct prime factors a
      positive integer has is less than the integer). Applying $\varphi$ to $m$ 
      gives us 
      \begin{align}
         \varphi(m) &= \varphi({p_1}^{c_1}{p_2}^{c_2}
            \cdots{p_s}^{c_s}) \nonumber \\
            &= \varphi({p_1}^{c_1})\varphi({p_2}^{c_2})\cdots
               \varphi({p_s}^{c_s})  \nonumber \\
            &= (p_1 - 1)(p_2 - 1)\cdots(p_s-1){p_1}^{c_1-1}{p_2}^{c_2-1}
               \cdots{p_s}^{c_s-1} \label{0_2_1} \\
            &= N. \nonumber
      \end{align}

      Let $q$ be the least prime greater than $N + 1$. Thus each $p_i - 1 < q$,
      for otherwise $\varphi(m) > N$ by $\eqref{0_2_1}$. Similarly, since
      ${p_i}^{N} > N$, it must be the case that each $c_i < N + 1$. So
      $$m = {p_1}^{c_1}{p_2}^{c_2}\cdots{p_s}^{c_s} <
        \underbrace{q^{N+1}q^{N+1}\cdots q^{N+1}}_{N \text{ times}} 
        = q^{(N+1)N}.$$
      Thus
      $$S_N \subset \{1, 2, \ldots, q^{(N+1)N}\},$$ 
      so that $S_N$ is finite. Let $\varepsilon$ be a positive number. To show 
      that $\varphi(n)$ tends to infinity, we must find a natural number $K$
      such that $\varphi(n) > \varepsilon$ for all $n > K$. Define
      $M_j := \max(S_j)$ for all $j \in \N$. Choose
      $K = \max\{M_1, M_2, \ldots, M_{\lceil\varepsilon \rceil}\}$,
      ($\lceil\varepsilon\rceil$ is the least integer greater than
      $\varepsilon$). Consider $n > K$ and assume to the contrary that
      $\varphi(n) = l \in \{1, 2, \ldots, \lceil\varepsilon\rceil\}$; that is,
      $n \in S_l$, so that $M_l \ge n$. By definition, $K \ge M_l$, and thus,
      $K \ge n$, a contradiction. Hence $\varphi(n) \notin \{1, 2, \ldots, 
      \lceil\varepsilon\rceil\}$; that is,
      $\varphi(n)  > \lceil\varepsilon\rceil \ge \varepsilon$. \qed
%%%%%%%%%%%%%%%%%%%%%%%%%%%%%%%%%%%%%0.2.11%%%%%%%%%%%%%%%%%%%%%%%%%%%%%%%%%%%%%
   \item[0.2.11]  Prove that if $d$ divides $n$ then $\varphi(d)$ divides
                  $\varphi(n)$ where $\varphi$ denotes Euler's
                  $\varphi-$function.

      \textbf{Proof.} Let $d$ and $n$ be positive integers such that $d \mid n$.
      We want to show that $\varphi(d) \mid \varphi(n)$. Let
      ${d_1}^{a_1}{d_2}^{a_2}\cdots{d_k}^{a_k}$ be the prime factorization of
      $d$, where each $a_i$ is a positive integer and each $d_i$ is a unique
      prime. Since $d \mid n$, it follows that there exists an integer $m$ such
      that
      $$n = ({d_1}^{a_1}{d_2}^{a_2}\cdots{d_k}^{a_k})m.$$
      Now we shall factor out the maximum powers of each $d_i$ in $m$, so that 
      we can write
      $$m = ({d_1}^{c_1}{d_2}^{c_2}\cdots{d_k}^{c_k})m'$$
      where each $c_i$ is a nonnegative integer and $m'$ is an integer that is
      prime to ${d_1}^{c_1}{d_2}^{c_2}\cdots{d_k}^{c_k}$. Thus we have that
      $$n = ({d_1}^{b_1}{d_2}^{b_2}\cdots{d_k}^{b_k})m', \quad b_i = a_i + c_i$$
      so that
      \begin{align*}
         \varphi(n) &= \varphi({d_1}^{b_1}{d_2}^{b_2}\cdots{d_k}^{b_k}m') \\
                    &= \varphi({d_1}^{b_1}{d_2}^{b_2}\cdots{d_k}^{b_k})
                       \varphi(m') \\
                    &= {d_1}^{b_1 - 1}(d_1 - 1){d_2}^{b_2 - 1}(d_2 - 1)\cdots
                       {d_k}^{b_k - 1}(d_k - 1)\varphi(m') \\
                    &= {d_1}^{c_1}{d_1}^{a_1 - 1}(d_1 - 1)
                       {d_2}^{c_2}{d_2}^{a_2 - 1}(d_2 - 1)\cdots
                       {d_k}^{c_k}{d_k}^{a_k - 1}(d_k - 1)\varphi(m')\\
                    &= {d_1}^{c_1}{d_2}^{c_2}\cdots{d_k}^{c_k}\varphi(m')
                       {d_1}^{a_1 - 1}(d_1 - 1)
                       {d_2}^{a_2 - 1}(d_2 - 1)\cdots
                       {d_k}^{a_k - 1}(d_k - 1) \\
                    &= ({d_1}^{c_1}{d_2}^{c_2}\cdots{d_k}^{c_k}
                        \varphi(m'))\varphi(d),
      \end{align*}
      so that $\varphi(d) \mid \varphi(n)$. \qed
      
\end{enumerate}

      \section{Symmetric Groups}
         \begin{enumerate}
%%%%%%%%%%%%%%%%%%%%%%%%%%%%%%%%%%%%%2.3.1%%%%%%%%%%%%%%%%%%%%%%%%%%%%%%%%%%%%%%
   \item[2.3.1]   Find all subgroups of $Z_{45} = \cyc{x}$, giving a generator
                  for each. Describe the containments between these subgroups.
                  
      \textbf{Solution.} Since the positive divisors of 45 are: 1, 3, 5, 9, 15,
      and 45, it follows that the subgroups of $Z_{45}$ are
      $$\cyc{x}, \cyc{x^3}, \cyc{x^5}, \cyc{x^9}, \cyc{x^{15}}, \text{ and }
        \cyc{x^{45}}.$$
        
      We have the following containments:
      $$
         \begin{tabular}{>{$}c<{$}>{$}c<{$}>{$}c<{$}>{$}c<{$}>{$}c<{$}>{$}c<{$}>{$}c<{$}}
            \cyc{x^{45}} & \le & \cyc{x^{15}} & \le & \cyc{x^5} & \le & \cyc{x} \\
            \cyc{x^{15}} & \le &  \cyc{x^3} & \le & \cyc{x} \\
            \cyc{x^9} & \le &  \cyc{x^3} & \le & \cyc{x}
         \end{tabular}
      $$
%%%%%%%%%%%%%%%%%%%%%%%%%%%%%%%%%%%%%2.3.2%%%%%%%%%%%%%%%%%%%%%%%%%%%%%%%%%%%%%%
   \item[2.3.2]   If $x$ is an element of the finite group $G$ and $|x| = |G|$,
                  prove that $G = \cyc{x}$. Give an explicit example to show 
                  that this result need not be true if $G$ is an infinite group.
                  
      \textbf{Proof.} Let $G$ be a finite group, so that $|G| = n \in \Z^+$.
      Suppose that there exists $x \in G$ such that $|x| = n$. Clearly
      $\cyc{x} \subseteq G$. But $|\cyc{x}| = n$ since $|x| = n$; thus
      $G \subseteq \cyc{x}$ so that $G = \cyc{x}$. Now let $G = \Z$. We have
      that $|\cyc{2}| = |G|$ but $G \neq \cyc{2}$. \qed
%%%%%%%%%%%%%%%%%%%%%%%%%%%%%%%%%%%%%2.3.3%%%%%%%%%%%%%%%%%%%%%%%%%%%%%%%%%%%%%%
   \item[2.3.3]   Find all generators for $\Z/48\Z$.
   
      \textbf{Solution.} The generators for $\Z/48\Z$ are: $\cyc{\overline{1}}$,
      $\cyc{\overline{5}}$, $\cyc{\overline{7}}$, $\cyc{\overline{11}}$,
      $\cyc{\overline{13}}$, $\cyc{\overline{17}}$, $\cyc{\overline{19}}$,
      $\cyc{\overline{23}}$, $\cyc{\overline{25}}$, $\cyc{\overline{29}}$,
      $\cyc{\overline{31}}$, $\cyc{\overline{35}}$, $\cyc{\overline{37}}$,
      $\cyc{\overline{41}}$, $\cyc{\overline{43}}$, and $\cyc{\overline{47}}$.
%%%%%%%%%%%%%%%%%%%%%%%%%%%%%%%%%%%%%2.3.4%%%%%%%%%%%%%%%%%%%%%%%%%%%%%%%%%%%%%%
   \item[2.3.4]   Find all generators for $\Z/202\Z$.
   
      \textbf{Solution.} Let $S$ be the set of generators for $\Z/202\Z$. Then
      $|S| = 100$ since
      $$S = \{\cyc{x} : x \text{ is odd and positive}, x \neq 101, \text{ and } x < 202\}.$$
%%%%%%%%%%%%%%%%%%%%%%%%%%%%%%%%%%%%%2.3.5%%%%%%%%%%%%%%%%%%%%%%%%%%%%%%%%%%%%%%
   \item[2.3.5]   Find the number of generators for $\Z/49000\Z$.
   
      \textbf{Solution.} For a positive integer $n$ let $\varphi(n)$ be the
      number of positive integers---less than or equal to $n$---that are
      relatively prime to $n$. Then the number of generators for $\Z/49000\Z$ is
      $\varphi(49000) = \varphi(2^35^37^2) =
      \varphi(2^3)\varphi(5^3)\varphi(7^2) = 16800$. 
%%%%%%%%%%%%%%%%%%%%%%%%%%%%%%%%%%%%%2.3.6%%%%%%%%%%%%%%%%%%%%%%%%%%%%%%%%%%%%%%
   \item[2.3.6]   In $\Z/48\Z$ write out all elements of $\cyc{\overline{a}}$
                  for every $\overline{a}$. Find all inclusions between
                  subgroups in $\Z/48\Z$.
      
      \textbf{Solution.}
      $$
         \begin{tabular}{|c|c|} \hline
            \textbf{Generators} & \textbf{Subgroups in} $\Z/48\Z$ \\ \hline
            0 & $\{0\}$ \\ \hline
            24 & $\{0, 24\}$ \\ \hline
            16, 32 & $\{0, 16, 32\}$ \\ \hline
            12, 36 & $\{0, 12, 24, 36\}$ \\ \hline
            8, 40 & $\{0, 8, 16, 24, 32, 40\}$ \\ \hline
            6, 18, 30, 42 & $\{0, 6, 12, 18, 24, 30, 36, 42\}$ \\ \hline
            4,20,28,44 & $\{0,4,8,12,16, 20, 24, 28, 32, 36, 40, 44\}$ \\ \hline
            3, 9, 15, 21, 27, 33, 39, 45 & $\{0, 3, 6, 9, 12, 15, 18, 21, 24,
            27, 30, 33, 36, 39, 42, 45\}$ \\ \hline            
            2, 10, 14, 22, 26, 34, 38, 46 & $\{x : 0 \le x \le 46,
            x \text{ is even}\}$ \\ \hline
            \text{See Exercise } 2.3.3 & $\Z/48\Z$ \\ \hline
         \end{tabular}
      $$
%%%%%%%%%%%%%%%%%%%%%%%%%%%%%%%%%%%%%2.3.7%%%%%%%%%%%%%%%%%%%%%%%%%%%%%%%%%%%%%%
   \item[2.3.7]   Let $Z_{48} = \cyc{x}$ and use the isomorphism
                  $\Z/48\Z \cong Z_{48}$ given by $\overline{1} \mapsto x$ to
                  list all subgroups of $Z_{48}$ as computed in the preceding
                  exercise.
                  
      \textbf{Solution.}
      $$
         \begin{tabular}{|c|} \hline
            \textbf{Subgroups in} $Z_{48}$ \\ \hline
            $\{1\}$ \\ \hline
            $\{1, x^{24}\}$ \\ \hline
            $\{1, x^{16}, x^{32}\}$ \\ \hline
            $\{1, x^{12}, x^{24}, x^{36}\}$ \\ \hline
            $\{1, x^8, x^{16}, x^{24}, x^{32}, x^{40}\}$ \\ \hline
            $\{1, x^6, x^{12}, x^{18}, x^{24}, x^{30},x^{36},x^{42}\}$ \\ \hline
            $\{1,x^4,x^8,x^{12},x^{16}, x^{20}, x^{24}, x^{28}, x^{32}, x^{36},
               x^{40}, x^{44}\}$ \\ \hline
            $\{1, x^3, x^6, x^9, x^{12}, x^{15}, x^{18}, x^{21}, x^{24},
            x^{27}, x^{30}, x^{33}, x^{36}, x^{39}, x^{42}, x^{45}\}$ \\ \hline
            $\{x^y : 0 \le y \le 46, y \text{ is even}\}$ \\ \hline
            $Z_{48}$ \\ \hline
         \end{tabular}
      $$
%%%%%%%%%%%%%%%%%%%%%%%%%%%%%%%%%%%%%2.3.8%%%%%%%%%%%%%%%%%%%%%%%%%%%%%%%%%%%%%%
   \item[2.3.8]   Let $Z_{48} = \cyc{x}$. For which integers $a$ does the map
                  $\varphi_a$ defined by $\varphi_a : \overline{1} \mapsto x^a$
                  extend to an \textit{isomorphism} from $\Z/48\Z$ onto
                  $Z_{48}$.
                  
      \textbf{Solution.} Suppose that $(a, 48) = 1$. Then it follows that $x^a$
      generates $Z_{48}$. Thus $\varphi_a$ is an isomorphism by Theorem 4 (Page
      56). Now suppose that $a$ is not relatively prime to 48. Then $x^a$ does
      not generate $Z_{48}$, so that the image of $\varphi_a$ is not $Z_{48}$.
      Hence $\varphi_a$ is an isomorphism if and only if $(a, 48) = 1$.
%%%%%%%%%%%%%%%%%%%%%%%%%%%%%%%%%%%%%2.3.9%%%%%%%%%%%%%%%%%%%%%%%%%%%%%%%%%%%%%%
   \item[2.3.9]   Let $Z_{36} = \cyc{x}$. For which integers $a$ does the map
                  $\psi_a$ defined by $\psi_a : \overline{1} \mapsto x^a$ extend
                  to a \textit{well defined homomorphism} from $\Z/48\Z$ into
                  $Z_{36}$. Can $\psi_a$ ever be a surjective homomorphism?
                  
      \textbf{Solution.} First we shall find the restriction(s) on $a$ such that
      $\psi_a$ is well defined. Suppose $b = c$ for some $b, c \in \Z/48\Z$. It
      suffices to show that $\psi_a(b) = \psi_a(c)$. Since $b = c$, there exists
      an integer $k$ such that $b = c + 48k$. Thus $\psi_a(b) = \psi_a(c+48k)$,
      so that
      $\psi_a(b)=(x^a)^{c+48k}=x^{ac + 48ak}= x^{ac}x^{48ak}=\psi_a(c)x^{12ak}$.
      So we must require $x^{12ak} = 1$ for all $k \in \Z$. Now $x^{12ak} = 1$
      for all $k \in \Z$ if and only if $3 \mid a$ if and only if $\psi_a$ is
      well defined. It follows immediately that
      $\psi_a$ is an homomorphism since
      \begin{align*}
         \psi_a(p + q) &= (x^a)^{p+q} \\
            &= x^{ap+aq} \\
            &= x^{ap}x^{aq} \\
            &= (x^a)^p(x^a)^q \\
            &= \psi_a(p)\psi_a(q)
      \end{align*}      
      for all $p, q \in \Z/48\Z$.
      
      \textit{Can $\psi_a$ ever be a surjective homomorphism?} No!
      
      \textbf{Proof.} Suppose to the contrary that $\psi_a$ is surjective. Then
      there exists $y \in \Z/48\Z$ such that $\psi_a(y) = x$. That is
      $x^{ay} = x$, so that $x^{ay-1} = 1$; thus $ay - 1 = 36m$ for some integer
      $m$. Rearrange the equality $ay - 1 = 36m$ to get $1 = ay - 36m$. Recall
      that $3 \mid a$; since $3$ also divides 36, it follows that 3 must divide
      1, a contradiction. Thus $\psi_a$ can never be surjective. \qed
%%%%%%%%%%%%%%%%%%%%%%%%%%%%%%%%%%%%%2.3.10%%%%%%%%%%%%%%%%%%%%%%%%%%%%%%%%%%%%%
   \item[2.3.10]  What is the order of $\overline{30}$ in $\Z/54\Z$? Write out
                  all the elements and their orders in $\cyc{\overline{30}}$.
                  
      \textbf{Solution.} The order of $30$ in $\Z/54\Z$ is
      $$\frac{54}{(30, 54)} = 9.$$
      The elements of $\cyc{30}$ and their respective orders are:
      $$
         \begin{tabular}{|c|c|} \hline
            Element of $\cyc{30}$ & Order \\ \hline
            30 & 9 \\ \hline
             6 & 9 \\ \hline
            36 & 3 \\ \hline
            12 & 9 \\ \hline
            42 & 9 \\ \hline
            18 & 3 \\ \hline
            48 & 9 \\ \hline
            24 & 9 \\ \hline
             0 & 1 \\ \hline
         \end{tabular}
      $$
%%%%%%%%%%%%%%%%%%%%%%%%%%%%%%%%%%%%%2.3.11%%%%%%%%%%%%%%%%%%%%%%%%%%%%%%%%%%%%%
   \item[2.3.11]  Find all cyclic subgroups of $D_8$. Find a proper subgroup of
                  $D_8$ which is not cyclic.
                  
      \textbf{Solution.} In $D_8$, only $r$ and $r^4$ have order 4. Thus
      $\{1, r, r^2, r^3\}$ is the only cyclic subgroup of order 4. The trivial
      subgroup is the only cyclic subgroup of order 1. Finally there are 5
      cyclic subgroups of order 2 and they are of the form $\{1, x\}$ where
      $x \in \{r^2, s, sr, sr^2, sr^3\}$. The set $\{1, s, r^2, sr^2\}$ is a
      non-cyclic proper subgroup of $D_8$.
%%%%%%%%%%%%%%%%%%%%%%%%%%%%%%%%%%%%%2.3.12%%%%%%%%%%%%%%%%%%%%%%%%%%%%%%%%%%%%%
   \item[2.3.12]  Prove that the following groups are \textit{not} cyclic:
                  \begin{enumerate}
                     \item $Z_2 \times Z_2$
                     \item $Z_2 \times \Z$
                     \item $\Z \times \Z$.
                  \end{enumerate}
      
      \textbf{Proof.}
      \begin{enumerate}
         \item The order of $Z_2 \times Z_2$ is 4, but no element in this group
               has order 4; thus $Z_2 \times Z_2$ is not cyclic.
         \item Let $Z_2 = \cyc{x}$. Observe that $Z_2 \times \Z$ is not finite,
               so in order for it to be cyclic it must be isomorphic to $\Z$.
               But this is not the case since $Z_2 \times \Z$ has two elements
               of finite order(namely $(1, 0)$ and $(x, 0)$) while $\Z$ has
               exactly 1 element of finite order.
         \item Suppose to the contrary that $\Z \times \Z$ is cyclic. Then there
               exist nonzero integers $a$ and $b$ such that
               $$\Z \times \Z = \cyc{(a,b)} = \{(na, nb) : n \in \Z\}.$$
               Thus there exists an integer $m$ such that
               $(ma, mb) = (0, 1)$. That is, $ma = 0$ and $mb = 1$. Since
               $ma = 0$, we must have $m = 0$ or $a = 0$. If $m$ is 0, then
               $(ma, mb) = (0, 0) \neq (0, 1)$, a contradiction; thus we must
               have $a = 0$, contradicting our assumption that $a$ is nonzero.
               Thus $\Z \times \Z$ is not cyclic.
      \end{enumerate} \qed
%%%%%%%%%%%%%%%%%%%%%%%%%%%%%%%%%%%%%2.3.13%%%%%%%%%%%%%%%%%%%%%%%%%%%%%%%%%%%%%
   \item[2.3.13]  Prove that the following pairs of groups are \textit{not}
                  isomorphic:
                  \begin{enumerate}
                     \item $\Z \times Z_2$ and $\Z$
                     \item $\Q \times Z_2$ and $\Q$.
                  \end{enumerate}
      
      \textbf{Proof.}
      \begin{enumerate}
         \item By Exercise 1.6.11, we know that $\Z \times Z_2$ is isomorphic to
               $Z_2 \times \Z$. By Exercise 2.3.12, $Z_2 \times \Z$ is not
               cyclic; thus $\Z \times Z_2$ is not cyclic. That is,
               $\Z \times Z_2$ is not isomorphic to $\Z$.
         \item Let $Z_2 = \cyc{x}$. It immediately follows that
               $\Q \times Z_2$ and $\Q$ are not isomorphic since $\Q \times Z_2$
               has two elements of finite order(namely $(0, 1)$ and $(0, x)$)
               while $\Q$ has exactly 1 element of finite order.
      \end{enumerate} \qed
%%%%%%%%%%%%%%%%%%%%%%%%%%%%%%%%%%%%%2.3.14%%%%%%%%%%%%%%%%%%%%%%%%%%%%%%%%%%%%%
   \item[2.3.14]  Let $\sigma =$ (1 2 3 4 5 6 7 8 9 10 11 12). For each of the
                  following integers $a$ compute $\sigma^a$:
                  $$a = 13, 65, 626, 1195, -6, -81, -570,\text{ and } {-1211}.$$
                  
      \textbf{Solution.}
      
      \begin{alignat*}{4}
         &\sigma^{13}   &&= \sigma &&\text{ } \\
         &\sigma^{65}   &&= \sigma^5 &&=
            (1\;6\;11\;4\;9\;2\;7\;12\;5\;10\;3\;8) \\
         &\sigma^{626}  &&= \sigma^2 &&= (1\;3\;5\;7\;9\;11) \\
         &\sigma^{1195} &&= \sigma^7 &&=
            (1\;8\;3\;10\;5\;12\;7\;2\;9\;4\;11\;6\;13) \\
         &\sigma^{-6} &&= \sigma^6 &&= (1\;7)
            (1\;8\;3\;10\;5\;12\;7\;2\;9\;4\;11\;6\;13) \\
         &\sigma^{-81} &&= \sigma^3 &&= (1\;4\;7\;10) \\
         &\sigma^{-570} &&= \sigma^6 &&= (1\;7) \\
         &\sigma^{-1211} &&= \sigma
      \end{alignat*}
%%%%%%%%%%%%%%%%%%%%%%%%%%%%%%%%%%%%%2.3.15%%%%%%%%%%%%%%%%%%%%%%%%%%%%%%%%%%%%%
   \item[2.3.15]  Prove that $\Q \times \Q$ is not cyclic.
   
      \textbf{Proof.} Since $\Q$ is infinite and, by Exercise 1.6.6, $\Q$ is not
      isomorphic to $\Z$, it follows that $\Q$ is not cyclic. We know that the
      subgroup of every cyclic group is cyclic; since $\Q \times\{1\} \cong \Q$,
      it follows that $\Q \times \{1\}$ is not cyclic; thus $\Q \times \Q$ is
      not cyclic because it has a noncyclic subgroup, namely $\Q \times \{1\}$.
      \qed
%%%%%%%%%%%%%%%%%%%%%%%%%%%%%%%%%%%%%2.3.16%%%%%%%%%%%%%%%%%%%%%%%%%%%%%%%%%%%%%
   \item[2.3.16]  Assume $|x| = n$ and $|y| = m$. Suppose that $x$ and $y$
                  \textit{commute}: $xy = yx$. Prove that $|xy|$ divides the
                  least common multiple of $m$ and $n$. Need this be true if $x$
                  and $y$ do \textit{not} commute? Give an example of commuting
                  elements $x$, $y$ such that the order of $xy$ is not equal to
                  the least common multiple of $|x|$ and $|y|$.
                  
      \textbf{Proof.} Let $l = \text{lcm}(m, n)$. So there exist integers
      $m'$ and $n'$ such that $mm' = nn' = l$. So we have that
      $$(xy)^l = x^ly^l = x^{nn'}y^{mm'} = (x^n)^{n'}(y^m)^{m'} = 1.$$
      That is $|xy|$ divides $l$ (by Proposition 3, Page 55).
      
      \textit{Need this be true if $x$ and $y$ do not commute?} No! Let
      $$
         A = \left(\begin{tabular}{@{}cc@{}}
            0 & 1/2 \\
            2 & 0
         \end{tabular}\right) \text{ and }
         B = \left(\begin{tabular}{@{}cc@{}}
            0 & 1 \\
            1 & 0
         \end{tabular}\right).
      $$
      A simple computation will show us that although $|A| = |B| = 2$, we have
      that $|AB| = \infty$.
      
      \textbf{Example.} Consider $\Z/2\Z = \{0, 1\}$. Let $x = y = 1$. Then we
      have $|x| = |y| = 2$, so that lcm($|x|, |y|) = 2 \neq |x + y| = |0| = 1$.
      \qed
%%%%%%%%%%%%%%%%%%%%%%%%%%%%%%%%%%%%%2.3.17%%%%%%%%%%%%%%%%%%%%%%%%%%%%%%%%%%%%%
   \item[2.3.17]  Find a presentation for $Z_n$ with one generator.
   
      \textbf{Solution.} $Z_n = \cyc{x : x^n = 1}$.
%%%%%%%%%%%%%%%%%%%%%%%%%%%%%%%%%%%%%2.3.18%%%%%%%%%%%%%%%%%%%%%%%%%%%%%%%%%%%%%
   \item[2.3.18]  Show that if $H$ is any group and $h$ is an element of $H$
                  with $h^n = 1$, then there is a unique homomorphism from
                  $Z_n = \cyc{x}$ to $H$ such that $x \mapsto h$.
                  
      \textbf{Proof.} Let $n \in \Z^+$, $Z_n = \cyc{x}$, $H$ a group, and
      $h^n  = 1$ for some $h \in H$. First we shall show the existence of a
      homomorphism from $Z_n$ to $H$ such that $x \mapsto h$. So consider the
      map $\alpha : \cyc{x} \rightarrow H$ defined by $\alpha(x^a) = h^a$.
      Clearly $\alpha(x) = h$. Now we will show that $\alpha$ is well defined.
      Suppose $x^w = x^y$ for some $x^w, x^y \in Z_n$. Thus $w = y + nk$ for
      some integer $k$. Thus
      $$\alpha(x^w) = \alpha(x^{y+nk})=h^{y+nk}=h^{y}{h^n}^k =h^y=\alpha(x^y),$$
      so that $\alpha$ is well defined. Now we have that
      $$\alpha(x^px^q)=\alpha(x^{p+q})=h^{p+q}=h^ph^q=\alpha(x^p)\alpha(x^q),$$
      so that $\alpha$ is an homomorphism. Now to show uniqueness, we suppose
      that $\phi : \cyc{x} \rightarrow H$ is an homommorphism such that
      $\phi(x) = h$. Since $\phi$ is a homomorphism, it follows that
      $\phi(x^a) = h^a$. Thus $\phi = \alpha$, as desired. \qed
%%%%%%%%%%%%%%%%%%%%%%%%%%%%%%%%%%%%%2.3.19%%%%%%%%%%%%%%%%%%%%%%%%%%%%%%%%%%%%%
   \item[2.3.19]  Show that if $H$ is any group and $h$ is an element of $H$,
                  then there is a unique homomorphism from $\Z$ to $H$ such that
                  $1 \mapsto h$.
                  
      \textbf{Proof.} Let $H$ be a group and let $h \in H$. First we shall show
      that there exists a homomorphism from $\Z$ to $H$ such that $1 \mapsto h$.
      So consider the map $\alpha : \Z \rightarrow H$ defined by
      $n \mapsto h^n$. Clearly $\alpha(1) = h$ and
      $$\alpha(x+y) = h^{x+y} = h^xh^y = \alpha(x)\alpha(y) \text{ for all }
        x, y \in \Z^+,$$
      so that $\alpha$ is a homomorphism. To show uniqueness, suppose that
      $\alpha' : \Z \rightarrow H$ is an homomorphism such that
      $\alpha'(1) = h$. Then according to Exercise 1.6.1, we have that
      $\alpha'(n) = \alpha'(n\cdot1) = \alpha'(1)^n = h^n$ for all $n \in \Z$;
      that is, $\alpha' = \alpha$, as desired. \qed
%%%%%%%%%%%%%%%%%%%%%%%%%%%%%%%%%%%%%2.3.20%%%%%%%%%%%%%%%%%%%%%%%%%%%%%%%%%%%%%
   \item[2.3.20]  Let $p$ be a prime and let $n$ be a positive integer. Show
                  that if $x$ is an element of the group $G$ such that
                  $x^{p^n} = 1$ then $|x| = p^m$ for some $m \le n$.
                  
      \textbf{Proof.} Suppose that $x \in G$ such that $x^{p^n} = 1$. Then it
      follows by Proposition 3 (Page 55) that $|x|$ divides $p^n$. Since $p$ is
      a prime, its factors are $p^i$, $0 \le i \le n$. Thus $|x| = p^m$ for
      some nonnegative $m$ not greater than $n$. \qed
%%%%%%%%%%%%%%%%%%%%%%%%%%%%%%%%%%%%%2.3.21%%%%%%%%%%%%%%%%%%%%%%%%%%%%%%%%%%%%%
   \item[2.3.21]  Let $p$ be an odd prime and let $n$ be a positive integer
                  $\ge 2$. Use the Binomial Theorem to show that
                  $(1+p)^{p^{n-1}} \equiv 1$ (mod $p^n$) but
                  $(1+p)^{p^{n-2}} \not\equiv 1$ (mod $p^n$). Deduce that $1+p$
                  is an element of order $p^{n-1}$ in the multiplicative group
                  $(\Z/p^n\Z)^\times$.

      \textbf{Lemma 2.3.1.} \textit{For an integer $n \ge 2$ and an odd prime
      $p$, let $f_p(n)$ be the number of $p$ factors of $n!$ (i.e., the greatest
      nonnegative integer $j$ such that $p^j \mid i!$), then it follows that
      $f_p(n) < \D\frac{n}{2}$}.

      \textbf{Proof.} Let $n \ge 2$ be an integer and $p$ an odd prime. For a
      a positive integer $r$, let $g_p(n, r)$ be the number of positive
      integers, less than or equal to $n$, that have at least $r$ number of $p$ 
      factors. It follows that $g_p(n, r) = \D\gint{\frac{n}{p^r}}$, where
      $\gint{x}$ is the greatest integer less than or equal to $x$. Finally let
      $k_n$ be the maximum nonnegative integer such that $p^{k_n}$ is a multiple
      of some positive integer not greater than $n$. Thus we have that
      \begin{align*}
         f_p(n) &= g_p(n, 1) + g_p(n, 2) + \cdots + g_p(n, k_n) \\
            &= \sum_{i=1}^{k_n} g_p(n, i)
            = \sum_{i=1}^{k_n} \gint{\frac{n}{p^i}} \\
            &\le \sum_{i=1}^{k_n} \frac{n}{p^i}
            < \sum_{i=1}^\infty \frac{n}{p^i} \\
            &= \frac{n}{p-1} &[\text{Sum of Geometric Series}] \\
            &< \frac{n}{2}. &[\text{Since }p \ge 3]
      \end{align*}

      So we can write $n! = p^{f_p(n)} h_n$ for some $h_n \in \Z^+$, so that
      $(h_n, p) = 1$.

      Now we are ready to commence the proof of the problem. By the Binomial
      Theorem, we have that
      \begin{align*}
         (1+p)^{p^{n-1}} &= \sum_{i=0}^{p^{n-1}}\binom{p^{n-1}}{i}p^i \\
            &= \sum_{i=0}^{p^{n-1}}p^i\frac{p^{n-1}(p^{n-1}-1)(p^{n-1}-2)
               \cdots(p^{n-1}-i+1)}{i!} \\
            &= \sum_{i=0}^{p^{n-1}}p^i\frac{p^{n-1}(p^{n-1}-1)(p^{n-1}-2)
               \cdots(p^{n-1}-i+1)}{p^{f_p(i)} h_i} \\
            &= 1 + p^n + p^n\sum_{i=2}^{p^{n-1}}\frac{p^{i-1}(p^{n-1}-1)
               (p^{n-1}-2) \cdots(p^{n-1}-i+1)}{p^{f_p(i)} h_i}.
      \end{align*}
      Now $f_p(i) < i / 2 \le i - 1$ for $i \ge 2$. Thus $i - 1 - f_p(i) \ge 0$
      (so that $p^{i - 1 - f_p(i)}$ is an integer) if $i \ge 2$. We then have
      \begin{equation} \label{2_3_21_1}
         (1+p)^{p^{n-1}} = 1 + p^n + p^n\sum_{i=2}^{p^{n-1}}\frac{p^{i-1-f_p(i)}
        (p^{n-1}-1)(p^{n-1}-2) \cdots(p^{n-1}-i+1)}{h_i}
      \end{equation}
      Since $(h_i, p) = 1$, it follows that $h_i$ must divide
      $p^{i-1}(p^{n-1}-1)(p^{n-1}-2) \cdots(p^{n-1}-i+1)$. Hence
      $$\sum_{i=2}^{p^{n-1}}\frac{p^{i-1-f_p(i)}
        (p^{n-1}-1)(p^{n-1}-2) \cdots(p^{n-1}-i+1)}{h_i}$$
      is an integer and we can conclude from \eqref{2_3_21_1} that
      $(1+p)^{p^{n-1}} \equiv 1$ (mod $p^n$). Now we have that
      \begin{align*}
         (1+p)^{p^{n-2}} &= \sum_{i=0}^{p^{n-2}}\binom{p^{n-2}}{i}p^i \\
            &= \sum_{i=0}^{p^{n-2}}p^i\frac{p^{n-2}(p^{n-2}-1)(p^{n-2}-2)
               \cdots(p^{n-2}-i+1)}{i!} \\
            &= 1 + p^{n-1} + p^n\frac{p^{n-2}-1}{2} + p^n\frac{p(p^{n-2}-1)(p^{n-2}-2)}{3!} +\sum_{i=4}^{p^{n-1}}p^i\frac{p^{n-2}(p^{n-2}-1)(p^{n-2}-2)
               \cdots(p^{n-2}-i+1)}{p^{f_p(i)} h_i} \\
            &= 1 + p^n + p^n\sum_{i=2}^{p^{n-1}}\frac{p^{i-1}(p^{n-1}-1)
               (p^{n-1}-2) \cdots(p^{n-1}-i+1)}{p^{f_p(i)} h_i}.
      \end{align*}
      
%%%%%%%%%%%%%%%%%%%%%%%%%%%%%%%%%%%%%2.3.22%%%%%%%%%%%%%%%%%%%%%%%%%%%%%%%%%%%%%
   \item[2.3.22]  Let $n$ be an integer $\ge 3$. Use the Binomial Theorem to
                  show that $(1+2^2)^{2^{n-2}} \equiv 1$ (mod $2^n$) but
                  $(1+2^2)^{2^{n-3}} \not\equiv 1$ (mod $2^n$). Deduce that 5 is
                  an element of order $2^{n-2}$ in the multiplicative group
                  $(\Z/2^n\Z)^\times$.

      \textbf{Proof.}
%%%%%%%%%%%%%%%%%%%%%%%%%%%%%%%%%%%%%2.3.23%%%%%%%%%%%%%%%%%%%%%%%%%%%%%%%%%%%%%
   \item[2.3.23]  Show that $(\Z/2^n\Z)^\times$ is not cyclic for any $n \ge 3$.
                  [Find two distinct subgroups of order 2.]
%%%%%%%%%%%%%%%%%%%%%%%%%%%%%%%%%%%%%2.3.24%%%%%%%%%%%%%%%%%%%%%%%%%%%%%%%%%%%%%
   \item[2.3.24]  Let $G$ be a finite group and let $x \in G$.
                  \begin{enumerate}
                     \item Prove that if $g \in N_G(\cyc{x})$ then
                           $gxg^{-1} = x^a$ for some $a \in \Z$. 
                     \item Prove conversely that if $gxg^{-1} = x^a$ for some
                           $a \in \Z$ then $g \in N_G(\cyc{x})$. [Show first
                           that $gx^kg^{-1} = (gxg^{-1})^k = x^{ak}$ for any
                           integer $k$, so that $g\cyc{x}g^{-1} \le \cyc{x}$.
                           If $x$ has order $n$, show the elements $gx^ig^{-1}$,
                           $i = 0, 1, \ldots, n-1$ are distinct, so that
                           $|g\cyc{x}g^{-1}| = |\cyc{x}| = n$ and conclude that
                           $g\cyc{x}g^{-1} = \cyc{x}$.]
                  \end{enumerate}
                  Note that this cuts down some of the work in computing
                  normalizers of cyclic subgroups since one does not have to
                  check $ghg^{-1} \in \cyc{x}$ for every $h \in \cyc{x}$.
%%%%%%%%%%%%%%%%%%%%%%%%%%%%%%%%%%%%%2.3.25%%%%%%%%%%%%%%%%%%%%%%%%%%%%%%%%%%%%%
   \item[2.3.25]  Let $G$ be a cyclic group of order $n$ and let $k$ be an
                  integer relatively prime to $n$. Prove that the map
                  $x \mapsto x^k$ is surjective. Use Lagrange's Theorem
                  (Exercise 1.7.19) to prove the same is true for any finite
                  group of order $n$. (For such $k$ each element has a
                  $k^{\text{th}}$ root in $G$. It follows from Cauchy's Theorem
                  in Section 3.2 that if $k$ is not relatively prime to the
                  order of $G$ then the map $x \mapsto x^k$ is not surjective.)
%%%%%%%%%%%%%%%%%%%%%%%%%%%%%%%%%%%%%2.3.26%%%%%%%%%%%%%%%%%%%%%%%%%%%%%%%%%%%%%
   \item[2.3.26]  Let $Z_n$ be a cyclic group of order $n$ and for each integer
                  $a$ let
                  $$\sigma_a : Z_n \mapsto Z_n \qquad by \qquad \sigma_a(x) =
                  x^a \quad \text{for all } x \in Z_n.$$
                  \begin{enumerate}
                     \item Prove that $\sigma_a$ is an automorphism of $Z_n$ if
                           and only if $a$ and $n$ are relatively prime(
                           automorphisms were introduced in Exercise 1.6.20).
                     \item Prove that $\sigma_a = \sigma_b$ if and only if
                           $a \equiv b$ (mod $n$).
                     \item Prove that \textit{every} automorphism of $Z_n$ is
                           equal to $\sigma_a$ for some integer $a$.
                     \item Prove that $\sigma_a\circ\sigma_b=\sigma_{ab}$.
                           Deduce that the map $\overline{a} \mapsto \sigma_a$
                           is an isomorphism of $(\Z/n\Z)^\times$ onto the
                           automorphism group of $Z_n$ (so Aut($Z_n$) is an
                           abelian group of order $\varphi(n)$).
                  \end{enumerate}
                  %%%%%MISSING CONTAINMENT%%%%%%%%
\end{enumerate}


































      \section{Matrix Groups}
         \begin{enumerate}
%%%%%%%%%%%%%%%%%%%%%%%%%%%%%%%%%%Prob4.1%%%%%%%%%%%%%%%%%%%%%%%%%%%%%%%%%%%%%%%
   \item[4.1]  Mark each statement True or False. Justify each answer.
               \begin{enumerate}
                  \item To prove a universal statement $\forall$ $x$, $p(x)$, we 
                        let $x$ represent an arbitrary member from the system 
                        under consideration and show that $p(x)$ is true.
                  \item To prove an existential statement
                        $\exists$ $x \ni p(x)$, we must find a particular $x$ in 
                        the system for which $p(x)$ is true.
                  \item In writing a proof, it is important to include all the 
                        logical steps.
               \end{enumerate}

      \textbf{Solution:} 

      \begin{enumerate}
         \item True. Since the $x$ is arbitrary, then if $p(x)$ is true, it must
               be the case that the statement is true for all $x$.
         \item False. We must find at least one $x$ in the system for which
               $p(x)$ is true.
         \item False. Only ``relevant" steps should be included.
      \end{enumerate}
%%%%%%%%%%%%%%%%%%%%%%%%%%%%%%%%%%Prob4.2%%%%%%%%%%%%%%%%%%%%%%%%%%%%%%%%%%%%%%%
   \item[4.2]  Mark each statement True or False. Justify each answer.
               \begin{enumerate}
                  \item A proof by contradiction may use the tautology
                        $({\sim}p \Rightarrow c) \Leftrightarrow p$.
                  \item A proof by contradiction may use the tautology
                        $[(p \lor {\sim}q) \Rightarrow c] \Leftrightarrow
                        (p \Rightarrow q)$.
                  \item Definitions often play an important role in proofs.
               \end{enumerate}

      \textbf{Solution:}

      \begin{enumerate}
         \item True. [See Text]
         \item False. The left side of the tautology should be
               $[(p \land {\sim}q) \Rightarrow c]$.
         \item True. [See Text]
      \end{enumerate}
%%%%%%%%%%%%%%%%%%%%%%%%%%%%%%%%%%Prob4.3%%%%%%%%%%%%%%%%%%%%%%%%%%%%%%%%%%%%%%%
   \item[4.3]  Prove: There exists an integer $n$ such that $n^2 + 3n/2 = 1$. Is
               this integer unique?
			
		\textbf{Proof:} Let $n = -2$. Then we have $(-2)^2 + 3(-2)/2 = 1$, so that
		the integer $-2$ is a solution to the equation $n^2 + 3n/2 = 1$. We claim
		that this integer is unique. To show this, suppose that $n' \in \Z$ is
		also a solution; that is, $n'^2 + 3n'/2 = 1$. Multiplying the latter
		equation by 2 and factoring will give us $(n' + 2)(n' - 1/2)= 0$, so that
		$n' = -2$. \qed
%%%%%%%%%%%%%%%%%%%%%%%%%%%%%%%%%%Prob4.4%%%%%%%%%%%%%%%%%%%%%%%%%%%%%%%%%%%%%%%
   \item[4.4]  Prove: There exists a rational number $x$ such that
               $x^2 + 3x/2 = 1$. Is this rational number unique?
			
		\textbf{Proof:} Let $x = 1/2$. Then we have $(1/2)^2 + 3(1/2)/2 = 1$, so
		that the rational number $1/2$ is a solution to the equation
		$n^2 + 3n/2 = 1$. This rational number is not unique since the rational
		number $-2/1$ also solves the equation. \qed
%%%%%%%%%%%%%%%%%%%%%%%%%%%%%%%%%%Prob4.5%%%%%%%%%%%%%%%%%%%%%%%%%%%%%%%%%%%%%%%
   \item[4.5]  Prove: For every real number $x > 3$, there exists a real number
               $y < 0$ such that $x = 3y/(2 + y)$.
					
		\textbf{Proof:} Choose $y = 2x/(3-x)$ for some real number greater $x$
		than 3. First we must show that $y$ is negative and that $x = 3y/(2 + y)$.
		Since $x$ is greater than 3, we have that $2x$ is positive and $3 - x$ is
		negative, so that $y$ is negative. Finally, we have that
		\begin{align*}
			\frac{3y}{2 + y} &= \frac{\frac{6x}{3 - x}}{2 + \frac{2x}{3 - x}} \\
								  &= \frac{6x}{2(3 - x) + 2x} \\
								  &= \frac{6x}{6 - 2x + 2x} \\
								  &= x.
		\end{align*} \qed
%%%%%%%%%%%%%%%%%%%%%%%%%%%%%%%%%%Prob4.6%%%%%%%%%%%%%%%%%%%%%%%%%%%%%%%%%%%%%%%
   \item[4.6]  Prove: For every real number $x > 1$, there exist two distinct
               positive real numbers $y$ and $z$ such that
               $$x = \frac{y^2 + 9}{6y} = \frac{z^2 + 9}{6z}.$$
					
		\textbf{Proof:} Let $x$ be a real number greater than 1. So choose
		$$y = 3x - 3\sqrt{x^2 - 1} \text{ and } z = 3x + 3\sqrt{x^2 - 1}.$$
		It is routine to check that $y$ and $z$ satisfy the required equations.
		\qed
%%%%%%%%%%%%%%%%%%%%%%%%%%%%%%%%%%Prob4.7%%%%%%%%%%%%%%%%%%%%%%%%%%%%%%%%%%%%%%%
   \item[4.7] Prove: If $x^2 + x - 6 \ge 0$, then $x \le -3$ or $x \ge 2$.
	
		\textbf{Proof:} Suppose that $x^2 + x - 6 \ge 0$ for some real number $x$.
		Then we have that $x^2 + x - 6 = (x + 3)(x - 2)\ge 0$. Solving this
		inequality will give us $x \le -3$ or $x \ge 2$. \qed
%%%%%%%%%%%%%%%%%%%%%%%%%%%%%%%%%%Prob4.8%%%%%%%%%%%%%%%%%%%%%%%%%%%%%%%%%%%%%%%
   \item[4.8] Prove: If $x/(x - 1) \le 2$, then $x < 1$ or $x \ge 2$.
	
		\textbf{Proof:} Suppose $x/(x - 1) \le 2$ for some real number $x$.
		
		\textbf{Case I:} $x > 1$. \\
		Multiply the inequality $x/(x - 1) \le 2$ by the positive number $x - 1$
		to give us $x \le 2(x - 1)$. Solving the latter inequality results in
		$x \ge 2$.
		
		\textbf{Case II:} $x < 1$. \\
		Multiply the inequality $x/(x - 1) \le 2$ by the negative number $x - 1$
		to give us $x \ge 2(x - 1)$. Solving the latter inequality results in
		$x \le 2$. By assumption, $x$ must also satisfy $x < 1$. To satisfy
		$x < 1$ and $x \le 2$, we must have that $x < 1$.
		
		The proof is done. \qed
%%%%%%%%%%%%%%%%%%%%%%%%%%%%%%%%%%Prob4.9%%%%%%%%%%%%%%%%%%%%%%%%%%%%%%%%%%%%%%%
   \item[4.9] Prove: $\log_27$ is irrational.
	
		\textbf{Proof:} Suppose by way of contradiction that $\log_27$ is
		rational. We note that the $\log$ function is strictly increasing, and
		since $\log_22 = 1$, we must have $\log_27 > \log_22 > 1$, so that
		$\log_27$ is positive. Thus there exist positive integers $p$ and $q$
		such $\log_27 = p/q$. That is $2^{p/q} = 7$, so that $2^p = 7^q$. By the
		Fundamental Theorem of Arithmetic, it follows that $2^p$ and $7^q$ have
		the same prime factors. But this is false, since $2^p$'s only prime factor
		is 2 and $7^q$'s only prime factor is 7. Thus $\log_27$ is irrational.
		\qed
%%%%%%%%%%%%%%%%%%%%%%%%%%%%%%%%%%Prob4.10%%%%%%%%%%%%%%%%%%%%%%%%%%%%%%%%%%%%%%
   \item[4.10] Prove: If $x$ is a real number, then $|x - 2| \le 3$ implies that
               $-1 \le x \le 5$.
					
		\textbf{Proof:} Let $x$ be a real number. Suppose that $|x - 2| \le 3$.
		Then by definition we have that $-3 \le x - 2 \le 3$. Adding 2 to these
		inequalities results in $-1 \le x \le 5$. \qed
%%%%%%%%%%%%%%%%%%%%%%%%%%%%%%%%%%Prob4.11%%%%%%%%%%%%%%%%%%%%%%%%%%%%%%%%%%%%%%
   \item[4.11] Consider the following theorem:
               ``If $m^2$ is odd, then $m$ is odd." Indicate what, if anything, 
               is wrong with each of the following ``proofs."
               \begin{enumerate}
                  \item Suppose $m$ is odd. Then $m = 2k + 1$ for some integer
                        $k$. Thus $m^2 = (2k + 1)^2 = 4k^2 + 4k + 1 =
                        2(2k^2 + 2k) + 1$, which is odd. Thus if $m^2$ is odd,
                        then $m$ is odd.
                  \item Suppose $m$ is not odd. Then $m$ is even and $m = 2k$ 
                        for some integer $k$. Thus
                        $m^2 = (2k)^2 = 4k^2 = 2(2k^2)$, which is even. Thus if
                        $m$ is not odd, then $m^2$ is not odd. It follows that
                        if $m^2$ is odd, then $m$ is odd.
               \end{enumerate}
					
		\textbf{Solution:}
		
		\begin{enumerate}
			\item	The proof proved the converse of the original implication,
					instead of proving the implication; and as we know, the converse
					of an implication is not necessarily equivalent to the original
					implication.
			\item	The proof proved the contrapositive of the original statement,
					but it did not explicitly state so.
      \end{enumerate} 
%%%%%%%%%%%%%%%%%%%%%%%%%%%%%%%%%%Prob4.12%%%%%%%%%%%%%%%%%%%%%%%%%%%%%%%%%%%%%%
   \item[4.12] Consider the following theorem: ``If $xy = 0$, then $x = 0$ or
               $y = 0$." Indicate what, if anything, is wrong with each of the
               following ``proofs."
               \begin{enumerate}
                  \item Suppose $xy = 0$ and $x \neq 0$. Then dividing both
                        sides of the first equation by $x$ we have $y = 0$. Thus
                        if $xy = 0$, then $x = 0$ or $y = 0$.
                  \item There are two cases to consider. First suppose that
                        $x = 0$. Then $x \cdot y = x \cdot 0 = 0$. Similarly,
                        suppose that $y = 0$. Then $x \cdot y = x \cdot 0 = 0$.
                        In either case, $x \cdot y = 0$. Thus if $xy = 0$, then
                        $x = 0$ or $y = 0$.
               \end{enumerate}
					
		\textbf{Solution:}
		
		\begin{enumerate}
			\item	The proof is OK.
			\item	The proof proved the converse of the original implication,
					instead of proving the implication.
      \end{enumerate} 
%%%%%%%%%%%%%%%%%%%%%%%%%%%%%%%%%%Prob4.13%%%%%%%%%%%%%%%%%%%%%%%%%%%%%%%%%%%%%%
   \item[4.13] Suppose $x$ and $y$ are real numbers. Recall that a real number
               $m$ is rational iff $m = p/q$ where $p$ and $q$ are integers and
               $q \neq 0$. If a real number is not rational, then it is
               irrational. Prove the following. [You may use the fact that the
               sum and product of integers is again an integer.]
               \begin{enumerate}
                  \item If $x$ is rational and $y$ is rational, then $x + y$ is
                        rational.
                  \item If $x$ is rational and $y$ is rational, then $xy$ is 
                        rational.
                  \item If $x$ is rational and $y$ is irrational, then $x + y$
                        is irrational.
               \end{enumerate}
					
		\textbf{Proof:}
		
		\begin{enumerate}
			\item	Let $x$ and $y$ be rational numbers. Then there exist integers
					$p$ and $r$ and nonzero integers $q$ and $s$ such that $x = p/q$
					and $y = r/s$. Then it follows that
					\begin{align*}
						x + y &= \frac{p}{q} + \frac{r}{s} \\
								&= \frac{ps + qr}{qs}.
					\end{align*}
					
					Since the sum and product of integers are also integers we have
					that $x + y$ is also a rational number. \qed
			\item	Let $x$ and $y$ be rational numbers. Then there exist integers
					$p$ and $r$ and nonzero integers $q$ and $s$ such that $x = p/q$
					and $y = r/s$. Then it follows that $xy = pr/qs$. Since the
					product of integers is also an integer we have that $xy$ is also
					a rational number. \qed
			\item	Let $x$ be a rational number and let $y$ be an irrational number.
					Then we can write $x = p/q$ for some integer $p$ and nonzero
					integer $q$. Assume by way of contradiction that $x + y$ is
					rational; hence $x + y = r/s$ for some integers $r$ and $s$, with
					$s \neq 0$. Thus we have that $y = (r/s) + (-x)$. By (b) it
					follows that $y$ is a rational number, a contradiction. Thus
					$x + y$ must be irrational. \qed
      \end{enumerate} 
%%%%%%%%%%%%%%%%%%%%%%%%%%%%%%%%%%Prob4.14%%%%%%%%%%%%%%%%%%%%%%%%%%%%%%%%%%%%%%
   \item[4.14] Suppose $x$ and $y$ are real numbers. Prove or give a 
               counterexample.
               \begin{enumerate}
                  \item If $x$ is irrational and $y$ is irrational, then $x + y$
                        is irrational.
                  \item If $x + y$ is irrational, then $x$ is irrational or
                        $y$ is irrational.
                  \item If $x$ is irrational and $y$ is irrational, then $xy$
                        is irrational.
                  \item If $xy$ is irrational, then $x$ is irrational or $y$ is
                        irrational.
               \end{enumerate}
					
		\textbf{Proof:}
		
		\begin{enumerate}
			\item	False. Let $x = \sqrt{2}$ and let $y = -\sqrt{2}$. Then
					$x + y = 0$, a rational number. \qed
			\item	True. The contrapositive of this statement was proved in
					4.13(a). \qed
			\item	False. Let $x = y = \sqrt{2}$. Then $xy = 2$, a rational
					number. \qed
			\item	True. The contrapositive of this statement was proved in
					4.13(b). \qed
      \end{enumerate} 
%%%%%%%%%%%%%%%%%%%%%%%%%%%%%%%%%%Prob4.15%%%%%%%%%%%%%%%%%%%%%%%%%%%%%%%%%%%%%%
   \item[4.15] Consider the following theorem and proof.
               \begin{quote}
                  \textbf{Theorem}: If $x$ is rational and $y$ is irrational,
                  then $xy$ is irrational.
                  \begin{quote}
                     \textbf{Proof}: Suppose $x$ is rational and $y$ is
                     irrational. If $xy$ is rational, then we have $x = p/q$ and
                     $xy = m/n$ for some integers $p$, $q$, $m$ and $n$, with
                     $q \neq 0$ and $n \neq 0$. It follows that
                     $$y = \frac{xy}{x} = \frac{m/n}{p/q} = \frac{mq}{np}.$$
                     This implies that $y$ is rational, a contradiction. We
                     conclude that $xy$ must be irrational.
                  \end{quote}
               \end{quote}
               \begin{enumerate}
                  \item Find a specific counterexample to show that the theorem
                        is false.
                  \item Explain what is wrong with the proof.
                  \item What additional condition on $x$ in the hypothesis would
                        make the conclusion true?
               \end{enumerate}
					
		\textbf{Solution:}
		
		\begin{enumerate}
			\item	Let $x = 0$ and let $y$ be any irrational number. Then $xy = 0$,
					a rational number.
         \item The proof did not take into consideration that $x$ could be 0.
               If $x$ were zero, then the expression $y = xy/x$ would be
               illegal.
         \item In addition to being rational, $x$ must also be nonzero.

      \end{enumerate} 
%%%%%%%%%%%%%%%%%%%%%%%%%%%%%%%%%%Prob4.16%%%%%%%%%%%%%%%%%%%%%%%%%%%%%%%%%%%%%%
   \item[4.16] Prove or give a counterexample: If $x$ is irrational, then
               $\sqrt{x}$ is irrational.

      \textbf{Proof:} Assume by way of contradiction that for some irrational
      number $x$ that $\sqrt{x}$ is rational. Then we have $\sqrt{x} = p/q$ for
      some integers $p$ and $q$, with $q \neq 0$. Squaring both sides of this
      equality results in $x = p^2/q^2$; that is $x$ is rational, a
      contradiction. Thus $\sqrt{x}$ is irrational. \qed
%%%%%%%%%%%%%%%%%%%%%%%%%%%%%%%%%%Prob4.17%%%%%%%%%%%%%%%%%%%%%%%%%%%%%%%%%%%%%%
   \item[4.17] Prove or give a counterexample: There do not exist three
               consecutive even integers $a$, $b$, and $c$ such that
               $a^2 + b^2 = c^2$.

      \textbf{Counterexample:} Consider the triple $(-2, 0, 2)$ or $(6, 8, 10)$.
%%%%%%%%%%%%%%%%%%%%%%%%%%%%%%%%%%Prob4.18%%%%%%%%%%%%%%%%%%%%%%%%%%%%%%%%%%%%%%
   \item[4.18] Consider the following theorem: There do not exist three
               consecutive odd integers $a$, $b$, and $c$ such that
               $a^2 + b^2 = c^2$.
               \begin{enumerate}
                  \item Complete the following restatement of the theorem: For
                        every three consecutive odd integers $a$, $b$, and $c$,
                        we have that $a^2 + b^2 \neq c^2$.
                  \item Change the sentence in part (a) into an implication
                        $p \Rightarrow q$: If $a$, $b$, and $c$ are consecutive
                        odd integers, then $a^2 + b^2 \neq c^2$.
                  \item Fill in the blanks in the following proof of the
                        theorem.
                        \begin{quote}
                           \textbf{Proof}: Let $a$, $b$, and $c$ be consecutive
                           odd integers. Then $a = 2k + 1$, $b = 2k + 3$, and
                           $c = 2k + 5$ for some integer $k$. Suppose
                           $a^2 + b^2 = c^2$. Then
                           $(2k + 1)^2 + (2k + 3)^2 = (2k + 5)^2$.

                           \quad It follows that
                           $8k^2 + 16k + 10 = 4k^2 + 20k + 25$ and
                           $4k^2 - 4k - 15 = 0$. Thus $k = 5/2$ or $k = -3/2$.
                           This contradicts $k$ being an integer. Therefore,
                           there do not exist three consecutive odd integers
                           $a$, $b$, and $c$ such that $a^2 + b^2 = c^2$.
                        \end{quote}
                  \item Which of the tautologies in Example 3.12 best describes
                        the structure of the proof? 3.12 (g)
               \end{enumerate}
%%%%%%%%%%%%%%%%%%%%%%%%%%%%%%%%%%Prob4.19%%%%%%%%%%%%%%%%%%%%%%%%%%%%%%%%%%%%%%
   \item[4.19] Prove or give a counterexample: The sum of any five consecutive
               integers is divisible by five.

      \textbf{Proof:} Let $x$ be the least integer amongst a list of five
      consecutive integers. Then the integers are: $x$, $x + 1$, $x + 2$,
      $x + 3$, $x + 4$, and their sum is $5x + 10 = 5(x + 2)$; that is, their
      sum is divisible by 5. \qed
%%%%%%%%%%%%%%%%%%%%%%%%%%%%%%%%%%Prob4.20%%%%%%%%%%%%%%%%%%%%%%%%%%%%%%%%%%%%%%
   \item[4.20] Prove or give a counterexample: The sum of any four consecutive
               integers is never divisible by four.

      \textbf{Proof:} Let $x$ be the least integer amongst a list of four
      consecutive integers. Then the integers are: $x$, $x + 1$, $x + 2$,
      $x + 3$, and their sum is $4x + 6 = 4(x + 2) + 2$. This says that the
      sum of any four consecutive integers always a remainder of 2 when divided
      by 4, so that this sum is not divisible by four. \qed
%%%%%%%%%%%%%%%%%%%%%%%%%%%%%%%%%%Prob4.21%%%%%%%%%%%%%%%%%%%%%%%%%%%%%%%%%%%%%%
   \item[4.21] Prove or give a counterexample: For every positive integer $n$,
               $n^2 + 3n + 8$ is even.

      \textbf{Proof:} Let $n$ be a positive integer. We shall investigate the
      following two cases:

      \textbf{Cases I:} \textit{$n$ is even}. Thus we can write $n = 2k$ for
      some natural number $k$, so that
      $n^2 + 3n + 8 = 4k^2 + 6k + 8 = 2(2k^2 + 3k + 4)$, an even number.

      \textbf{Cases II:} \textit{$n$ is odd}. Thus we can write $n = 2k + 1$ for
      some natural number $k$, so that
      $n^2 + 3n + 8 = 4k^2 + 10k + 12 = 2(2k^2 + 5k + 6)$, an even number.

      Thus if $n$ is a positive integer, then $n^2 + 3n + 8$ is even. \qed
%%%%%%%%%%%%%%%%%%%%%%%%%%%%%%%%%%Prob4.22%%%%%%%%%%%%%%%%%%%%%%%%%%%%%%%%%%%%%%
   \item[4.22] Prove or give a counterexample: For every positive integer $n$,
               $n^2 + 4n + 8$ is even.
   
      \textbf{Counterexample:} If $n = 1$, then $n^2 + 4n + 8 = 13$, an odd
      number.
%%%%%%%%%%%%%%%%%%%%%%%%%%%%%%%%%%Prob4.23%%%%%%%%%%%%%%%%%%%%%%%%%%%%%%%%%%%%%%
   \item[4.23] Prove or give a counterexample: there do not exist irrational
               numbers $x$ and $y$ such that $x^y$ is rational.

      \textbf{Counterexample:} We do not know whether or not the number
      $\sqrt{2}^{\sqrt{2}}$ is rational, so we shall investigate the following
      two possibilities.

      \textbf{Cases I:} \textit{$\sqrt{2}^{\sqrt{2}}$ is irrational}. Then let
      $x = \sqrt{2}^{\sqrt{2}}$ and $y = \sqrt{2}$. Then we have $x^y = 2$.

      \textbf{Cases II:} \textit{$\sqrt{2}^{\sqrt{2}}$ is rational}. Then let
      $x = y = \sqrt{2}$, so that $x^y$ is a rational number.       
%%%%%%%%%%%%%%%%%%%%%%%%%%%%%%%%%%Prob4.24%%%%%%%%%%%%%%%%%%%%%%%%%%%%%%%%%%%%%%
   \item[4.24] Prove or give a counterexample: there do not exist rational
               numbers $x$ and $y$ such that $x^y$ is a positive integer and
               $y^x$ is a negative integer.

      \textbf{Counterexample:} Let $x = 1$ and $y = -2$. Then we have
      $x^y = 1$ and $y^x = -2$.
%%%%%%%%%%%%%%%%%%%%%%%%%%%%%%%%%%Prob4.25%%%%%%%%%%%%%%%%%%%%%%%%%%%%%%%%%%%%%%
   \item[4.25] Prove or give a counterexample: for all $x > 0$ we have
               $x^2 + 1 < (x + 1)^2 \le 2(x^2 + 1)$.

      \textbf{Proof:} Let $x$ be a positive real number. First we want to show
      that $x^2 + 1 < (x + 1)^2$. We assumed that $x > 0$ so that $2x > 0$;
      that is $-2x < 0$. Adding $(x + 1)^2$ to the inequality $-2x < 0$ results
      in $(x + 1)^2 - 2x < (x + 1)^2$, so that $x^2 + 1 < (x + 1)^2$. Now we
      must show that $(x + 1)^2 \le 2(x^2 + 1)$. We know that for any real
      number $y$ we have that $y^2 \ge 0$; thus we have that
      $(x - 1)^2 = x^2 - 2x + 1 \ge 0$, so that $-x^2 + 2x - 1 \le 0$. Adding
      $2x^2 + 2$ to the inequality $-x^2 + 2x - 1 \le 0$ results in
      $x^2 + 2x + 1 \le 2x^2 + 2$; that is, $(x + 1)^2 \le 2(x^2 + 1)$. The
      proof is done. \qed
\end{enumerate}

      \section{The Quaternion Group}
         \begin{enumerate}
%%%%%%%%%%%%%%%%%%%%%%%%%%%%%%%%%%%%%2.5.1%%%%%%%%%%%%%%%%%%%%%%%%%%%%%%%%%%%%%%
   \item[2.5.1]   Let $H$ and $K$ be subgroups of $G$. Exhibit all possible
                  sublattices which show only $G$, 1, $H$, $K$, and their joins
                  and intersections. What distinguishes the different drawings?
%%%%%%%%%%%%%%%%%%%%%%%%%%%%%%%%%%%%%2.5.2%%%%%%%%%%%%%%%%%%%%%%%%%%%%%%%%%%%%%%
   \item[2.5.2]   In each of (a) to (d) list all subgroups of $D_{16}$ that
                  satisfy the given condition.
                  \begin{enumerate}
                     \item Subgroups that are contained in $\cyc{sr^2, r^4}$
                     \item Subgroups that are contained in $\cyc{sr^7, r^4}$
                     \item Subgroups that contain $\cyc{r^4}$
                     \item Subgroups that contain $\cyc{s}$.
                  \end{enumerate}
                  
      \textbf{Solution.}
      
      \begin{enumerate}
         \item The subgroups that are contained in $\cyc{sr^2, r^4}$ are those
               that have an upward path to the join of $sr^2$ and $r^4$. Thus
               these subgroups are: $1$, $\cyc{r^4}$, $\cyc{sr^2}$,
               $\cyc{sr^6}$, and $\cyc{sr^2, r^4}$.
         \item The subgroups that are contained in $\cyc{sr^7, r^4}$ are those
               that have an upward path to the join of $sr^3$ and $r^4$. Since
               the join of $sr^7$ and $r^4$ is $\cyc{sr^3, r^4}$. It follows
               that these subgroups are: $1$, $\cyc{r^4}$, $\cyc{sr^3}$,
               $\cyc{sr^7}$, and $\cyc{sr^7, r^4} = \cyc{sr^3, r^4}$.
         \item The subgroups that contain $r^4$ are: $\cyc{r^4}$,
               $\cyc{sr^2, r^4}$, $\cyc{s, r^4}$, $\cyc{r^2}$,
               $\cyc{sr^3, r^4}$, $\cyc{sr^5, r^4}$, $\cyc{s, r^2}$, $\cyc{r}$,
               $\cyc{sr, r^2}$, and $D_{16}$.
         \item The subgroups that contain $s$ are: $\cyc{s}$, $\cyc{s, r^4}$,
               $\cyc{s, r^2}$, and $D_{16}$.
      \end{enumerate}
%%%%%%%%%%%%%%%%%%%%%%%%%%%%%%%%%%%%%2.5.3%%%%%%%%%%%%%%%%%%%%%%%%%%%%%%%%%%%%%%
   \item[2.5.3]   Show that the subgroup $\cyc{s, r^2}$ of $D_8$ is isomorphic
                  to $V_4$.
                  
      \textbf{Proof.} By Exercise 1.1.36, there is exactly one group, say $K$,
      of order 4 that has no element of order 4. Since
      $\cyc{s, r^2} = \{1, s, r^2, sr^2\}$, it follows that every non-identity
      element of $\cyc{s, r^2}$ has order 2, so that $\cyc{s, r^2} \cong K$.
      Similarly, the Klein-4 group has no element of order 4. Thus
      $V_4 \cong K$, and we conclude that $V_4 \cong \cyc{s, r^2}$. \qed
%%%%%%%%%%%%%%%%%%%%%%%%%%%%%%%%%%%%%2.5.4%%%%%%%%%%%%%%%%%%%%%%%%%%%%%%%%%%%%%%
   \item[2.5.4]   Use the given lattice to find all pairs of elements that
                  generate $D_8$ (there are 12 pairs).
                  
      \textbf{Solution.} It suffices to find all pairs of elements whose join is
      $D_8$. They are: $\cyc{s, r}$, $\cyc{s, rs}$, $\cyc{s, r^3s}$,
      $\cyc{r^2s, rs}$, $\cyc{r^2s, r^3s}$, $\cyc{r^2s, r}$, $\cyc{r^2s, r^3}$,
      $\cyc{r, rs}$, $\cyc{r, r^3s}$,  $\cyc{r^3, s}$, $\cyc{r^3, rs}$, and
      $\cyc{r^3, r^3s}$,
%%%%%%%%%%%%%%%%%%%%%%%%%%%%%%%%%%%%%2.5.5%%%%%%%%%%%%%%%%%%%%%%%%%%%%%%%%%%%%%%
   \item[2.5.5]   Use the given lattice to find all elements $x \in D_{16}$
                  such that $D_{16} = \cyc{x, s}$ (there are 8 such elements
                  $x$).
                  
      \textbf{Solution.} By observing the given lattice of $D_{16}$, we find
      that      
      $$x \in \{r, r^3, r^5, r^7, sr^3, sr^7, sr^5, sr\}.$$
%%%%%%%%%%%%%%%%%%%%%%%%%%%%%%%%%%%%%2.5.6%%%%%%%%%%%%%%%%%%%%%%%%%%%%%%%%%%%%%%
   \item[2.5.6]   Use the given lattices to help find the centralizers of every
                  element in the following groups:

                  (a) $D_8$ \qquad (b) $Q_8$ \qquad
                  (c) $S_3$ \qquad (d) $D_{16}$.
                  
      \begin{enumerate}
         \item $$
                  \begin{tabular}{@{}|c|c|@{}} \hline
                     Elements in $D_8$ & Centralizer \\ \hline
                     1, $r^2$ & $D_8$ \\ \hline
                     $r, r^3$ & $\cyc{r}$ \\ \hline
                     $s, r^2s$ & $\cyc{s, r^2}$ \\ \hline
                     $rs, r^3s$ & $\cyc{rs, r^2}$ \\ \hline
                  \end{tabular}
               $$
         \item $$
                  \begin{tabular}{@{}|c|c|@{}} \hline
                     Elements in $Q_8$ & Centralizer \\ \hline
                     $\pm1$ & $Q_8$ \\ \hline
                     $\pm i$ & $\cyc{i}$ \\ \hline
                     $\pm j$ & $\cyc{j}$ \\ \hline
                     $\pm k$ & $\cyc{k}$ \\ \hline
                  \end{tabular}
               $$
         \item $$
                  \begin{tabular}{@{}|c|c|@{}} \hline
                     Element(s) in $Q_8$ & Centralizer \\ \hline
                     1 & $S_3$ \\ \hline
                     (1 2) & $\cyc{(1\;2)}$ \\ \hline
                     (1 3) & $\cyc{(1\;3)}$ \\ \hline
                     (2 3) & $\cyc{(2\;3)}$ \\ \hline
                     (1 2 3), (1 3 2) & $\cyc{(1\;2\;3)}$ \\ \hline
                  \end{tabular}
               $$
         \item $$
                  \begin{tabular}{@{}|c|c|@{}} \hline
                     Elements in $D_{16}$ & Centralizer \\ \hline
                     1, $r^4$ & $D_{16}$ \\ \hline
                     $r, r^2, r^3, r^5, r^6, r^7$ & $\cyc{r}$ \\ \hline
                     $s, sr^4$ & $\cyc{s, r^4}$ \\ \hline
                     $sr, sr^5$ & $\cyc{sr^5, r^4}$ \\ \hline
                     $sr^2, sr^6$ & $\cyc{sr^2, r^4}$ \\ \hline
                     $sr^3, sr^7$ & $\cyc{sr^3, r^4}$ \\ \hline
                  \end{tabular}
               $$
      \end{enumerate}
%%%%%%%%%%%%%%%%%%%%%%%%%%%%%%%%%%%%%2.5.7%%%%%%%%%%%%%%%%%%%%%%%%%%%%%%%%%%%%%%
   \item[2.5.7]   Find the center of $D_{16}$.
   
      \textbf{Solution.} From Exercise 2.5.6(d), we see that only 1 and $r^4$
      are in the all the centralizers of the elements of $D_{16}$. Thus
      $Z(D_{16}) = \cyc{r_4}$.
%%%%%%%%%%%%%%%%%%%%%%%%%%%%%%%%%%%%%2.5.8%%%%%%%%%%%%%%%%%%%%%%%%%%%%%%%%%%%%%%
   \item[2.5.8]   In each of the following groups find the normalizer of each
                  subgroup:

                  (a) $S_3$ \qquad (b) $Q_8$.
%%%%%%%%%%%%%%%%%%%%%%%%%%%%%%%%%%%%%2.5.9%%%%%%%%%%%%%%%%%%%%%%%%%%%%%%%%%%%%%%
   \item[2.5.9]   Draw the lattices of subgroups of the following groups:

                  (a) $\Z/16\Z$ \qquad (b) $\Z/24\Z$ \qquad
                  (c) $\Z/48\Z$. [See Exercise 6 in Section 3.]
%%%%%%%%%%%%%%%%%%%%%%%%%%%%%%%%%%%%%2.5.10%%%%%%%%%%%%%%%%%%%%%%%%%%%%%%%%%%%%%
   \item[2.5.10]  Classify groups of order 4 by proving that if $|G| = 4$ then
                  $G \cong Z_4$ or $G\cong V_4$. [See Exercise 36, Section 1.1.]
%%%%%%%%%%%%%%%%%%%%%%%%%%%%%%%%%%%%%2.5.11%%%%%%%%%%%%%%%%%%%%%%%%%%%%%%%%%%%%%
   \item[2.5.11]  Consider the group of order 16 with the following
                  presentation:

                  $$QD_{16} = \cyc{\sigma, \tau : \sigma^8 = \tau^2 = 1,
                    \sigma\tau = \tau\sigma^3}$$
                  (called the \textit{quasidihedral} or \textit{semidihedral}
                  group of order 16). This group has three subgroups of order 8:
                  $\cyc{\tau, \sigma^2} \cong D_8, \cyc{\sigma} \cong Z_8$ and
                  $\cyc{\sigma^2, \sigma\tau} \cong Q_8$ and every proper
                  subgroup is contained in one of these three subgroups. Fill in
                  the missing subgroups in the lattice of all subgroups of the 
                  quasidiheral group on the following page, exhibiting each
                  subgroup with at most two generators. (This is another example
                  of a nonplanar lattice.)
\end{enumerate}

\noindent The next three examples lead to two nonisomorphic groups that have the 
          same lattice of subgroups.

\begin{enumerate}
%%%%%%%%%%%%%%%%%%%%%%%%%%%%%%%%%%%%%2.5.12%%%%%%%%%%%%%%%%%%%%%%%%%%%%%%%%%%%%%
   \item[2.5.12]  The group
                  $A = Z_2 \times Z_4 = \cyc{a, b : a^2 = b^4 = 1, ab = ba}$ has
                  order 8 and has three subgroups of order 4:
                  $\cyc{a, b^2} \cong V_4$, $\cyc{b} \cong Z_4$ and
                  \begin{verbatim}
                     *
                     *
                     *
                     *
                     *
                     *
                     *
                     *
                     *
                  \end{verbatim}
                  $\cyc{ab} \cong Z_4$ and every proper subgroup is contained in
                  one of these three. Draw the lattice of all subgroups of $A$,
                  giving each subgroup in terms of at most two generators.
%%%%%%%%%%%%%%%%%%%%%%%%%%%%%%%%%%%%%2.5.13%%%%%%%%%%%%%%%%%%%%%%%%%%%%%%%%%%%%%
   \item[2.5.13]  The group
                  $G = Z_2 \times Z_8 = \cyc{x, y : x^2 = y^8 = 1, xy = yx}$ has
                  order 16 and has three subgroups of order 8:
                  $\cyc{x, y^2} \cong Z_2 \times Z_4$, $\cyc{y} \cong Z_8$ and
                  $\cyc{xy} \cong Z_8$ and every proper subgroup is contained in
                  one of these three. Draw the lattice of all subgroups of $G$,
                  giving each subgroup in terms of at most two generators.
%%%%%%%%%%%%%%%%%%%%%%%%%%%%%%%%%%%%%2.5.14%%%%%%%%%%%%%%%%%%%%%%%%%%%%%%%%%%%%%
   \item[2.5.14]  Let $M$ be the group of order 16 with the following 
                  presentation:
                  $$\cyc{u, v : u^2 v^8 = 1, vu = uv^5}$$
                  (sometimes called the \textit{modular} group of order 16). It
                  has three subgroups of order 8: $\cyc{u, v^2}$, $\cyc{v}$, and
                  $\cyc{uv}$ and every proper subgroup is contained in one of
                  these three. Prove that $\cyc{u, v^2} \cong Z_2 \times Z_4$,
                  $\cyc{v} \cong Z_8$ and $\cyc{uv} \cong Z_8$. Show that the
                  lattice of subgroups of $M$ is the same as the lattice of
                  subgroups of $Z_2 \times Z_8$ (cf. Exercise 13) but that these
                  two groups are not isomorphic.
%%%%%%%%%%%%%%%%%%%%%%%%%%%%%%%%%%%%%2.5.15%%%%%%%%%%%%%%%%%%%%%%%%%%%%%%%%%%%%%
   \item[2.5.15]  Describe the isomorphism type of each of the three subgroups
                  of $D_{16}$ of order 8.
%%%%%%%%%%%%%%%%%%%%%%%%%%%%%%%%%%%%%2.5.16%%%%%%%%%%%%%%%%%%%%%%%%%%%%%%%%%%%%%
   \item[2.5.16]  Use the lattice of subgroups of the quasidihedral group of
                  order 16 to show that every element of order 2 is contained in
                  the proper subgroup $\cyc{\tau, \sigma^2}$.
%%%%%%%%%%%%%%%%%%%%%%%%%%%%%%%%%%%%%2.5.17%%%%%%%%%%%%%%%%%%%%%%%%%%%%%%%%%%%%%
   \item[2.5.17]  Use the lattice of subgroups of the modular group $M$ of order
                  16 to show that the set $\{x \in M : x^2 = 1\}$ is a subgroup
                  of $M$ isomorphic to the Klein 4-group.
%%%%%%%%%%%%%%%%%%%%%%%%%%%%%%%%%%%%%2.5.18%%%%%%%%%%%%%%%%%%%%%%%%%%%%%%%%%%%%%
   \item[2.5.18]  Use the lattice to help find the centralizer of every element
                  of $QD_{16}$.
%%%%%%%%%%%%%%%%%%%%%%%%%%%%%%%%%%%%%2.5.19%%%%%%%%%%%%%%%%%%%%%%%%%%%%%%%%%%%%%
   \item[2.5.19]  Use the lattice to help find $N_{D_{16}}(\cyc{s, r^4})$.
%%%%%%%%%%%%%%%%%%%%%%%%%%%%%%%%%%%%%2.5.20%%%%%%%%%%%%%%%%%%%%%%%%%%%%%%%%%%%%%
   \item[2.5.20]  Use the lattice of subgroups of $QD_{16}$ to help find the
                  normalizers.

                  (a) $N_{QD_{16}}(\cyc{\tau\sigma})$ \qquad
                  (b) $N_{QD_{16}}(\cyc{\tau, \sigma^4})$.
\end{enumerate}

      \section{Homomorphisms And Isomorphisms}
         Let $G$ and $H$ be groups.
\begin{enumerate}
%%%%%%%%%%%%%%%%%%%%%%%%%%%%%%%%%%%%%1.6.1%%%%%%%%%%%%%%%%%%%%%%%%%%%%%%%%%%%%%%
   \item[1.6.1]   Let $\varphi : G \rightarrow H$ be a homomorphism.
                  \begin{enumerate}
                     \item Prove that $\varphi(x^n) = \varphi(x)^n$ for all
                           $n \in \Z^+$.
                     \item Do part (a) for $n = -1$ and deduce that
                           $\varphi(x^n) = \varphi(x)^n$ for all $n \in \Z$.
                  \end{enumerate}

      \textbf{Solution.}

      \begin{enumerate}
         \item \textbf{Proof.} We shall proceed by induction on $n$. It is clear
               that $\varphi(x^1) = \varphi(x)^1$. Now suppose that
               $\varphi(x^k) = \varphi(x)^k$ for some integer $k$. Thus
               \begin{align*}
                  \varphi(x^{k+1}) &= \varphi(x^kx) \\
                     &= \varphi(x^k)\varphi(x)
                        &[\varphi\text{ is a homomorphism}] \\
                     &= \varphi(x)^k\varphi(x) &[\text{Inductive hypothesis}] \\
                     &= \varphi(x)^{k+1},
               \end{align*}
               so that, by Mathematical Induction, $\varphi(x^n) = \varphi(x)^n$ 
               for all $n \in \Z^+$. \qed
         \item Since
               $$1 \cdot \varphi(1) = \varphi(1) = \varphi(1 \cdot 1) =
                 \varphi(1)\cdot\varphi(1),$$
               it follows by cancellation that $\varphi(1) = 1$. Thus
               $$\varphi(x)\varphi(x^{-1}) = \varphi(xx^{-1}) =\varphi(1) = 1,$$
               so that $\varphi(x^{-1}) = \varphi(x)^{-1}$. Now let $n$ be a
               positive integer. Then it follows that
               \begin{align*}
                  \varphi(x^{-n}) &= \varphi((x^{-1})^n) \\
                     &= \varphi(x^{-1})^n &[\text{1.6.1(a)}] \\
                     &= (\varphi(x)^{-1})^n \\
                     &= \varphi(x)^{-n}.
               \end{align*}
               Moreover $\varphi(x^0) = 1 = \varphi(x)^0$; thus we can conclude 
               that $\varphi(x^n) = \varphi(x)^n$ for all $n \in \Z$.
      \end{enumerate}
%%%%%%%%%%%%%%%%%%%%%%%%%%%%%%%%%%%%%1.6.2%%%%%%%%%%%%%%%%%%%%%%%%%%%%%%%%%%%%%%
   \item[1.6.2]   If $\varphi : G \rightarrow H$ is an isomorphism, prove that
                  $|\varphi(x)| = |x|$ for all $x \in G$. Deduce that any two
                  isomorphic groups have the same number of elements of order
                  $n$ for each $n \in \Z^+$. Is the result true if $\varphi$ is
                  only assumed to be a homomorphism?

      \textbf{Proof.} Assume that $\varphi : G \rightarrow H$ is a group
      isomorphism. Let $x \in G$. Suppose $|x| = n$. By the preceding exercise, 
      we have that $\varphi(x)^n = \varphi(x^n) = \varphi(1) = 1$, so that
      $|\varphi(x)| \le n$. Now suppose that $|\varphi(x)| = m < n$. Then we
      must have that $\varphi(1) = 1 = \varphi(x)^m = \varphi(x^m)$. That is,
      $x^m = 1$ (since $\varphi$ is injective), a contradiction since $|x| = n$.
      Thus $|\varphi(x)| = |x| = n$. Finally suppose that $|x| = \infty$ and
      $|\varphi(x)| = r < \infty$. Then, as previously argued, we must have
      that $x^r = 1$, a contradiction. Thus if $y \in G$, it must follow that
      $|y| = |\varphi(y)|$. \qed

      For a positive integer $n$, we can now exhibit a bijection (using
      $\varphi$) between the elements of $G$ of order $n$ and the elements of
      $H$ of order $n$. Thus any two isomorphic groups must have the same number 
      of elements of order $n$ for each $n \in \Z^+$. If $\varphi$ is only 
      assumed to be a homomorphism then the result is not generally true. 
      Consider the homomorphism
      $$\alpha : S_3 \rightarrow \{1\}.$$
      Although $S_3$ has an element of order 2, the trivial group $\{1\}$ has no 
      element of order 2.
%%%%%%%%%%%%%%%%%%%%%%%%%%%%%%%%%%%%%1.6.3%%%%%%%%%%%%%%%%%%%%%%%%%%%%%%%%%%%%%%
   \item[1.6.3]   If $\varphi : G \rightarrow H$ is an isomorphism, prove that
                  $G$ is abelian if and only if $H$ is abelian. If
                  $\varphi : G \rightarrow H$ is a homomorphism, what additional
                  conditions on $\varphi$ (if any) are sufficient to ensure that
                  if $G$ is abelian, then so is $H$?

      \textbf{Proof.} Suppose $\varphi : G \rightarrow H$ is an isomorphism.

      ($\Rightarrow$) Assume $G$ is abelian. Consider $h_1$ and $h_2$ in $H$.
      Since $\varphi$ is surjective, there exist $g_1$, $g_2 \in G$ such that
      $\varphi(g_1) = h_1$ and $\varphi(g_2) = h_2$. Since $G$ is abelian, we
      have that
      $$h_1h_2 = \varphi(g_1)\varphi(g_2) = \varphi(g_1g_2) = \varphi(g_2g_1) = 
        \varphi(g_2)\varphi(g_1) = h_2h_1,$$
      so that $H$ is also abelian.

      ($\Rightarrow$) Now assume that $H$ is abelian. Consider $g_3$ and $g_4$ 
      in $G$. Since $H$ is abelian, we have that
      $$\varphi(g_3g_4) = \varphi(g_3)\varphi(g_4) = \varphi(g_4)\varphi(g_3) =
        \varphi(g_4g_3).$$
      We thus conclude that $g_3g_4 = g_4g_3$ since $\varphi$ is injective. That
      is, $G$ is abelian. \qed

      Finally suppose $\varphi$ is an homomorphism and $G$ is abelian. Looking
      at the first direction of our proof above, we see that restricting
      $\varphi$ to be surjective is sufficient to make $H$ abelian.
%%%%%%%%%%%%%%%%%%%%%%%%%%%%%%%%%%%%%1.6.4%%%%%%%%%%%%%%%%%%%%%%%%%%%%%%%%%%%%%%
   \item[1.6.4]   Prove that the multiplicative groups $\R - \{0\}$ and
                  $\C - \{0\}$ are not isomorphic.

      \textbf{Proof.} Since $\R - \{0\}$ has no element of order 4, and since
      $\C - \{0\}$ has 2 elements ($i$ and $-i$) of order 4, it follows that
      these two multiplicative groups are not isomorphic. \qed
%%%%%%%%%%%%%%%%%%%%%%%%%%%%%%%%%%%%%1.6.5%%%%%%%%%%%%%%%%%%%%%%%%%%%%%%%%%%%%%%
   \item[1.6.5]   Prove that the additive groups of $\R$ and $\Q$ are not
                  isomorphic.

      \textbf{Proof.} Since $|\R| \neq |\Q|$, it follows that $(\R, +)$ and
      $(\Q, +)$ are not isomorphic. \qed
%%%%%%%%%%%%%%%%%%%%%%%%%%%%%%%%%%%%%1.6.6%%%%%%%%%%%%%%%%%%%%%%%%%%%%%%%%%%%%%%
   \item[1.6.6]   Prove that the additive groups of $\Z$ and $\Q$ are not
                  isomorphic.

      \textbf{Proof.} First we shall show that $(\Q, +)$ is not cyclic. So
      suppose that $(\Q, +) = \cyc{c}$. Thus there exists an integer $n$ such 
      that $nc = \D\frac{c}{2}$, so that $\D n = \frac{1}{2}$, a contradiction. 
      That is, $(\Q, +)$ is not cyclic. Now suppose to the contrary that
      $\varphi : \Z \rightarrow \Q$ is a homomorphism from $(\Z, +)$ to
      $(\Q, +)$. Let $q \in \Q$ and let $z$ be the unique preimage of $q$ under
      $\varphi$. Thus
      \begin{align*}
         q &= \varphi(z) \\
           &= \varphi(z \cdot 1) \\
           &= z \cdot \varphi(1),  &[1.6.1(a)]
      \end{align*}
      so that $\Q = \cyc{\varphi(1)}$, a contradiction. Thus $(\Z, +)$ and
      $(\Q, +)$ are not isomorphic. \qed
%%%%%%%%%%%%%%%%%%%%%%%%%%%%%%%%%%%%%1.6.7%%%%%%%%%%%%%%%%%%%%%%%%%%%%%%%%%%%%%%
   \item[1.6.7]   Prove that $D_8$ and $Q_8$ are not isomorphic.

      \textbf{Proof.} Since $D_8$ has exactly 5 elements of order 2 and $Q_8$
      has exactly 1 element of order 2, it follows that these two groups are not
      isomorphic. \qed
%%%%%%%%%%%%%%%%%%%%%%%%%%%%%%%%%%%%%1.6.8%%%%%%%%%%%%%%%%%%%%%%%%%%%%%%%%%%%%%%
   \item[1.6.8]   Prove that if $n \neq m$, $S_n$ and $S_m$ are not isomorphic.

      \textbf{Proof.} Let $n$ and $m$ be unequal positive integers so that
      $n! \neq m!$; i.e $n! = |S_n| \neq |S_m| = m!$. Thus no bijective map can
      exist between $S_n$ and $S_m$, so that these two groups are not
      isomorphic. \qed
%%%%%%%%%%%%%%%%%%%%%%%%%%%%%%%%%%%%%1.6.9%%%%%%%%%%%%%%%%%%%%%%%%%%%%%%%%%%%%%%
   \item[1.6.9]   Prove that $D_{24}$ and $S_4$ are not isomorphic.

      \textbf{Proof.} Since $D_{24}$ has exactly 2 elements of order 4 and $S_4$
      has exactly 6 elements of order 4, it follows that these two groups are
      not isomorphic. \qed
%%%%%%%%%%%%%%%%%%%%%%%%%%%%%%%%%%%%%1.6.10%%%%%%%%%%%%%%%%%%%%%%%%%%%%%%%%%%%%%
   \item[1.6.10]  Fill in the details of the proof that the symmetric groups
                  $S_\triangle$ and $S_\Omega$ are isomorphic if
                  $|\triangle| = |\Omega|$ as follows: let
                  $\theta : \triangle \rightarrow \Omega$ be a bijection. Define
                  $$\varphi : S_\triangle \rightarrow S_\Omega \qquad
                    \text{by} \qquad \varphi(\sigma) = \theta \circ \sigma \circ 
                    \theta^{-1} \text{ for all } \sigma \in S_\triangle$$
                  and prove the following:
                  \begin{enumerate}
                     \item $\varphi$ is well defined, that is, if $\sigma$ is a
                           permutation of $\triangle$ then
                           $\theta\circ\sigma\circ\theta^{-1}$ is a permutation
                           of $\Omega$.
                     \item $\varphi$ is a bijection from $S_\triangle$ onto
                           $S_\Omega$. [Find a 2-sided inverse for $\varphi$.]
                     \item $\varphi$ is a homomorphism, that is,
                           $\varphi(\sigma\circ\tau) =
                            \varphi(\sigma)\circ\varphi(\tau)$.
                  \end{enumerate}
                  Note the similarity to the \textit{change of basis} or
                  \textit{similarity} transformations for matrices.

      \textbf{Proof.}

      \begin{enumerate}
         \item Let $\sigma \in S_\triangle$. Notice that
               $\varphi(\sigma) = \theta\circ\sigma\circ\theta^{-1}$ maps
               $\Omega$ into $\Omega$, and is also a bijecton since it is a
               composition of bijective maps. Thus
               $\varphi(\sigma) \in S_\Omega$, so that $\varphi$ is well
               defined.
         \item Consider the map
               $$\alpha : S_\Omega \rightarrow S_\triangle \qquad
                 \text{by} \qquad \alpha(\sigma) = \theta \circ \sigma^{-1} 
                 \circ \theta^{-1} \text{ for all } \sigma \in S_\Omega.$$
               A trivial computation will show us that $\varphi\circ\alpha$ is
               the identity map on $S_\Omega$ and $\alpha\circ\varphi$ is the
               identity map on $S_\triangle$. Thus $\varphi$ is a bijection.
         \item Let $\sigma, \tau \in S_\triangle$. So
               \begin{align*}
                  \varphi(\sigma\circ\tau) &=
                     \theta\circ\sigma\circ\tau\circ\theta^{-1} \\
                     &= \theta\circ\sigma\circ\theta^{-1}\circ\theta\circ
                        \tau\circ\theta^{-1} \\
                     &= \varphi(\sigma) \circ \varphi(\tau),
               \end{align*}
               as desired.
      \end{enumerate} \qed
%%%%%%%%%%%%%%%%%%%%%%%%%%%%%%%%%%%%%1.6.11%%%%%%%%%%%%%%%%%%%%%%%%%%%%%%%%%%%%%
   \item[1.6.11]  Let $A$ and $B$ be groups. Prove that
                  $A \times B \cong B\times A$.

      \textbf{Proof.} Consider the map $f : A \times B \rightarrow B \times A$,
      $(a, b) \mapsto (b, a)$. Now define
      $$g : B \times A\rightarrow A\times B, \quad (b, a) \mapsto (a, b).$$
      So $f$ is bijective since $g$ is its two-sided inverse. Now $f$ is a
      homomorphism because
      \begin{align*}
         f((a_1, b_1)(a_2, b_2)) &= f((a_1a_2, b_1b_2)) \\
            &= (b_1b_2, a_1a_2) \\
            &= (b_1, a_1)(b_2, a_2) \\
            &= f((a_1, b_1))f((a_2, b_2)).
      \end{align*} \qed
%%%%%%%%%%%%%%%%%%%%%%%%%%%%%%%%%%%%%1.6.12%%%%%%%%%%%%%%%%%%%%%%%%%%%%%%%%%%%%%
   \item[1.6.12]  Let $A$, $B$, and $C$ be groups and let $G = A \times B$ and
                  $H = B \times C$. Prove that $G \times C \cong A \times H$.

      \textbf{Proof.} Proceed as we did in Exercise 1.6.11 with the following
      modification:
      $$f : G \times C \rightarrow A \times H,
        \quad((a, b), c) \mapsto (a, (b, c))$$ and 
      $$g : A \times H\rightarrow G\times C,
        \quad (a, (b, c)) \mapsto ((a, b), c).$$ \qed
%%%%%%%%%%%%%%%%%%%%%%%%%%%%%%%%%%%%%1.6.13%%%%%%%%%%%%%%%%%%%%%%%%%%%%%%%%%%%%%
   \item[1.6.13]  Let $G$ and $H$ be groups and let $\varphi : G \rightarrow H$
                  be a homomorphism. Prove that the image of $\varphi$,
                  $\varphi(G)$, is a subgroup of $H$. Prove that if $\varphi$ is
                  injective then $G \cong \varphi(G)$.

      \textbf{Proof.} The set $\varphi(G)$ is nonempty since it contains
      $\varphi(1) = 1$. So let $h_1, h_2 \in \varphi(G)$. Thus there exist
      $g_1, g_2 \in G$ such that $\varphi(g_1) = h_1$ and $\varphi(g_2) = h_2$. 
      So
      $$\varphi(g_1g_2^{-1}) = \varphi(g_1)\varphi(g_2)^{-1} = h_1h_2^{-1} \in
        \varphi(G).$$
      That is $\varphi(G)$ is a subgroup of $H$. Now suppose that $\varphi$ is
      injective and consider the map $\alpha : G \rightarrow \varphi(G)$,
      $g \mapsto \varphi(g)$. Since $\varphi$ is an injective homomorphism, it
      follows that $\alpha$ is also an injective homomorphism. Also it is clear
      that $\alpha$ is onto. Thus $\alpha$ is an isomorphism and we have that
      $G \cong \varphi(G)$. \qed
%%%%%%%%%%%%%%%%%%%%%%%%%%%%%%%%%%%%%1.6.14%%%%%%%%%%%%%%%%%%%%%%%%%%%%%%%%%%%%%
   \item[1.6.14]  Let $G$ and $H$ be groups and let $\varphi : G \rightarrow H$
                  be a homomorphism. Define the \text{kernel} of $\varphi$ to be
                  $\{g \in G : \varphi(g) = 1_H\}$. Prove that the kernel of
                  $\varphi$ is a subgroup of $G$. Prove that $\varphi$ is
                  injective if and only if the kernel of $\varphi$ is the
                  identity subgroup of $G$.

      \textbf{Proof 1.} Let ker($\varphi$) be the kernel of $\varphi$. The set
      ker($\varphi$) is not empty because $1_G \in \text{ker}(\varphi)$. Now let
      $x, y \in \text{ker}(\varphi)$. Then we have that
      $$\varphi(xy^{-1}) = \varphi(x)\varphi(y)^{-1} = 1_H{1_H}^{-1} = 1_H,$$
      so that $xy^{-1} \in \text{ker}(\varphi)$. That is ker($\varphi$) is a
      subgroup of $G$. \qed

      \textbf{Proof 2.} ($\Rightarrow$) Assume that $\varphi$ is injective. Let
      $x \in \text{ker}(\varphi)$. Thus we have that
      $\varphi(1_G) = \varphi(x) = 1_H$, so that $x = 1_G$ since $\varphi$ is
      one to one. That is ker($\varphi$) = $\{1_G\}$, the identity subgroup of
      $G$.

      ($\Leftarrow$) Assume that ker($\varphi$) = $\{1_G\}$. Suppose that
      $\varphi(g_1) = \varphi(g_2)$ for some $g_1, g_2 \in G$. Then it follows
      that $\varphi(g_1)\varphi(g_2)^{-1} = 1_H$, so that
      $\varphi(g_1{g_2}^{-1}) = 1_H$. Since ker($\varphi$) = $\{1_G\}$, we can
      conclude that $g_1{g_2}^{-1} = 1_G$, so that $g_1 = g_2$; i.e., $\varphi$
      is injective. \qed      
%%%%%%%%%%%%%%%%%%%%%%%%%%%%%%%%%%%%%1.6.15%%%%%%%%%%%%%%%%%%%%%%%%%%%%%%%%%%%%%
   \item[1.6.15]  Define a map $\pi : \R^2 \rightarrow \R$ by $\pi((x, y)) = x$.
                  Prove that $\pi$ is a homomorphism and find the kernel of
                  $\pi$.

      \textbf{Proof.} Let $(a, b), (c, d) \in \R^2$. Then it follows immediately
      that $\pi$ is a homomorphism since
      $$\pi((a, b) + (c, d)) = \pi((a + c, b + d)) =
         a + c = \pi(a, b) + \pi(c, d).$$
      The kernel of $\pi$ is the set $\{(0, y) : y \in \R\}$. \qed
%%%%%%%%%%%%%%%%%%%%%%%%%%%%%%%%%%%%%1.6.16%%%%%%%%%%%%%%%%%%%%%%%%%%%%%%%%%%%%%
   \item[1.6.16]  Let $A$ and $B$ be groups and let $G$ be their direct product,
                  $A \times B$. Prove that the maps $\pi_1 : G \rightarrow A$
                  and $\pi_2 : G \rightarrow B$ defined by $\pi_1((a, b)) = a$
                  and $\pi_2((a, b)) = b$ are homomorphisms and find their
                  kernels.

      \textbf{Proof.} Let $(a_1, b_1), (a_2, b_2) \in G$. Then it follows 
      immediately that $\pi_1$ and $\pi_2$ are homomorphisms since
      $$\pi_1((a_1, b_1)(a_2, b_2)) = \pi_1((a_1a_2, b_1b_2)) =
         a_1a_2 = \pi_1(a_1, b_1)\pi(a_2, b_2)$$
      and
      $$\pi_2((a_1, b_1)(a_2, b_2)) = \pi_2((a_1a_2, b_1b_2)) =
         b_1b_2 = \pi_1(a_1, b_1)\pi(a_2, b_2).$$
      The kernel of $\pi_1 = \{(1, b) : b \in B\}$ and
      the kernel of $\pi_2 = \{(a, 1) : a \in A\}$. \qed
%%%%%%%%%%%%%%%%%%%%%%%%%%%%%%%%%%%%%1.6.17%%%%%%%%%%%%%%%%%%%%%%%%%%%%%%%%%%%%%
   \item[1.6.17]  Let $G$ be any group. Prove that the map from $G$ to itself
                  defined by $g \mapsto g^{-1}$ is a homomorphism if and only if
                  $G$ is abelian.

      \textbf{Proof.} Let $x, y \in G$. Consider the map
      $\alpha : G \rightarrow G$, $g \mapsto g^{-1}$. 

      ($\Leftarrow$) Assume that $G$ is abelian. Then it
      follows that
      \begin{align*}
         \alpha(xy) &= (xy)^{-1} \\
            &= y^{-1}x^{-1} \\
            &= x^{-1}y^{-1} &[G \text{ is abelian}] \\
            &= \alpha(x)\alpha(y),
      \end{align*}
      so that $\alpha$ is a homomorphism.

      ($\Rightarrow$) Assume that $\alpha$ is a homomorphism. Then it
      follows that
      \begin{align*}
         xy &= \alpha(x^{-1})\alpha(y^{-1}) \\
            &= \alpha(x^{-1}y^{-1}) \\
            &= \alpha((yx)^{-1}) \\
            &= yx,
      \end{align*}
      so that $G$ is abelian. \qed
%%%%%%%%%%%%%%%%%%%%%%%%%%%%%%%%%%%%%1.6.18%%%%%%%%%%%%%%%%%%%%%%%%%%%%%%%%%%%%%
   \item[1.6.18]  Let $G$ be any group. Prove that the map from $G$ to itself
                  defined by $g \mapsto g^2$ is a homomorphism if and only if
                  $G$ is abelian.

      \textbf{Proof.} Let $x, y \in G$. Consider the map
      $\alpha : G \rightarrow G$, $g \mapsto g^2$. 

      ($\Leftarrow$) Assume that $G$ is abelian. Then it
      follows that
      \begin{align*}
         \alpha(xy) &= (xy)^2 \\
            &= x^2y^2 &[G \text{ is abelian}] \\
            &= \alpha(x)\alpha(y),
      \end{align*}
      so that $\alpha$ is a homomorphism.

      ($\Rightarrow$) Assume that $\alpha$ is a homomorphism. Then it
      follows that
      \begin{align*}
         x^2y^2 &= \alpha(x)\alpha(y) \\
            &= \alpha(xy) \\
            &= (xy)^2 \\
            &= xyxy,
      \end{align*}
      so that $xxyy = xyxy$. By cancellation we thus have $xy = yx$; i.e, $G$ is
      abelian. \qed
%%%%%%%%%%%%%%%%%%%%%%%%%%%%%%%%%%%%%1.6.19%%%%%%%%%%%%%%%%%%%%%%%%%%%%%%%%%%%%%
   \item[1.6.19]  Let $G = \{z \in \C : z^n = 1 \text{ for some }n \in \Z^+\}$.
                  Prove that for any fixed integer $k > 1$ the map from $G$ to
                  itself defined by $z \mapsto z^k$ is a surjective homomorphism
                  but is not an isomorphism.

      \textbf{Proof.} Consider an integer $k > 1$ and the map
      $\alpha : G \rightarrow G$, $g \mapsto g^k$. Let $x, y \in G$. The map
      $\alpha$ is a homomorphism since
      $\alpha(xy) = (xy)^k = x^ky^k = \alpha(x)\alpha(y)$. Since $x \in G$, it
      follows that $x^m = 1$ for some positive integer $m$. Notice that
      $x^{1/k} \in G$ since $(x^{1/k})^{km} = 1$. Thus $\alpha$ is onto because
      $\alpha(x^{1/k}) = x$. Now consider the complex number
      $e^{2\pi/k} = \cos2\pi/k + i \sin2\pi/k$. Notice that $e^{2\pi/k} \in G$
      since $(e^{2\pi/k})^k = 1$. Also notice that $e^{2\pi/k} \neq 1$ (since
      $2\pi/k$ is not a multiple of $2\pi$), but we have that
      $\alpha(e^{2\pi/k}) = \alpha(1) = 1$, so that $\alpha$ is not injective; 
      i.e, $\alpha$ is not an isomorphism. \qed
%%%%%%%%%%%%%%%%%%%%%%%%%%%%%%%%%%%%%1.6.20%%%%%%%%%%%%%%%%%%%%%%%%%%%%%%%%%%%%%
   \item[1.6.20]  Let $G$ be a group and let Aut($G$) be the set of all
                  isomorphisms from $G$ onto $G$. Prove that Aut($G$) is a
                  group under function composition (called the
                  \text{automorphism group} of $G$ and the elements of Aut($G$)
                  are called \text{automorphisms} of $G$).

      \textbf{Proof.}

      \textbf{Closure.} Let $\alpha, \gamma \in \text{Aut}(G)$. Since the
      composition of two bijective functions is also bijective, it follows that
      $\alpha \circ \gamma$ is bijective. Now let $x, y \in G$. It follows that
      $$(\alpha\circ\gamma)(xy) = \alpha(\gamma(xy)) =
         \alpha(\gamma(x)\gamma(y)) = \alpha(\gamma(x))\alpha(\gamma(y)) =
         ((\alpha\circ\gamma)(x))((\alpha\circ\gamma)(y)),$$
      so that $\alpha\circ\gamma)$ is also an isomorphism on $G$, and thus,
      Aut($G$) is closed.

      \textbf{Associativity.} This follows from the associativity of functions.

      \textbf{Identity.} The identity map is the identity of Aut($G$).

      \textbf{Inverse.} Since every map in Aut($G$) is bijective, it follows 
      that every map has a 2-sided inverse.

      Thus we have shown that Aut($G$) is a group under composition. \qed      
%%%%%%%%%%%%%%%%%%%%%%%%%%%%%%%%%%%%%1.6.21%%%%%%%%%%%%%%%%%%%%%%%%%%%%%%%%%%%%%
   \item[1.6.21]  Prove that for each fixed nonzero $k \in \Q$ the map from $\Q$
                  to itself defined by $q \mapsto kq$ is an automorphism of
                  $\Q$.

      \textbf{Proof.} Let $k$ be a nonzero rational number. Consider the map
      $f : \Q \rightarrow \Q$, $q \mapsto kq$. Let $x, y \in Q$. We have that
      $f(x + y) = k(x + y) = kx + ky = f(x) + f(y)$, so that $f$ is a 
      homomorphism. Now suppose $f(x) = f(y)$, so that $kx = ky$. Since
      $k \neq 0$, we shall multiply the equality $kx = ky$ by $1/k$ to get
      $x = y$; i.e, $f$ is injective. Since $f(x/k) = x$, it follows that $f$ is
      onto, so that $f$ is an automorphism of $\Q$. \qed
%%%%%%%%%%%%%%%%%%%%%%%%%%%%%%%%%%%%%1.6.22%%%%%%%%%%%%%%%%%%%%%%%%%%%%%%%%%%%%%
   \item[1.6.22]  Let $A$ be an abelian group and fix some $k \in \Z$. Prove
                  that the map $a \mapsto a^k$ is a homomorphism from $A$ to 
                  itself. If $k = -1$ prove that this homomorphism is an
                  isomorphism.

      \textbf{Proof.} Let $k$ be an integer, $x, y \in A$. Consider the map
      $f : A \rightarrow A$, $a \mapsto a^k$. Since $A$ is abelian we have that
      $f(xy) = (xy)^k = x^ky^k = f(x)f(y)$, so that $f$ is a homomorphism. Now 
      assume that $k = -1$. In this case, notice that the map $f$ is also the
      2-sided inverse of $f$; thus $f$ is bijective, so that $f$ is an 
      isomorphism. \qed
%%%%%%%%%%%%%%%%%%%%%%%%%%%%%%%%%%%%%1.6.23%%%%%%%%%%%%%%%%%%%%%%%%%%%%%%%%%%%%%
   \item[1.6.23]  Let $G$ be a finite group which possesses an automorphism
                  $\sigma$ such that $\sigma(g) = g$ if and only if $g = 1$. If
                  $\sigma^2$ is the identity map from $G$ to $G$, prove that $G$
                  is abelian (such an automorphism $\sigma$ is called
                  \text{fixed point free} of order 2). [Hint. Show that every
                  element of $G$ can be written in the form $x^{-1}\sigma(x)$
                  and apply $\sigma$ to such an expression.]

      \textbf{Proof.} Consider the map $\alpha : G \rightarrow G$,
      $g \mapsto g^{-1}\sigma(g)$. Suppose that for some $x, y \in G$, we have
      that $\alpha(x) = \alpha(y)$. Then it follows that
      $x^{-1}\sigma(x) = y^{-1}\sigma(y)$, so that
      $yx^{-1} =\nobreak \sigma(y)\sigma(x)^{-1} = \sigma(yx^{-1})$. Since
      $\sigma(g) = g$ if and only if $g = 1$, we must then have that
      $yx^{-1} = 1$; thus $y = x$, so that $\alpha$ is injective; $\alpha$ is
      also surjective since $G$ is finite. So let $z \in G$. Then we must have 
      that $z = h^{-1}\sigma(h)$ for some $h \in G$. So
      $\sigma(z) = \sigma(h^{-1}\sigma(h)) = \sigma(h)^{-1}\sigma^2(h) =
       \sigma(h)^{-1}h = z^{-1}$. We can then conclude by Exercise 1.6.17 that
      $G$ is abelian. \qed
%%%%%%%%%%%%%%%%%%%%%%%%%%%%%%%%%%%%%1.6.24%%%%%%%%%%%%%%%%%%%%%%%%%%%%%%%%%%%%%
   \item[1.6.24]  Let $G$ be a finite group and let $x$ and $y$ be distinct
                  elements of order 2 in $G$ that generate $G$. Prove that
                  $G \cong D_{2n}$, where $n = |xy|$. [See Exercise 1.2.6]
                  
      \textbf{Proof.} Let $a = xy$. Since $G$ is finite, $|a|$ must also be
      finite. So write $|a| = n$. By Exercise 1.2.6, we have that
      $ax = xa^{-1}$. Also note since $y = x^2y = x(xy) = xa$, the elements
      $a$ and $x$ generate $G$. Thus we have that
      $$G = \cyc{a, x : a^n = x^2 = 1, ax = xa^{-1}}.$$
      By the discussion on Page 38-39 of the Textbook, the map
      $\varphi : D_{2n} \rightarrow G$, given by $\varphi(r) = a$ and
      $\varphi(s) = x$ is an isomorphism. Hence $G \cong D_{2n}$. \qed      
%%%%%%%%%%%%%%%%%%%%%%%%%%%%%%%%%%%%%1.6.25%%%%%%%%%%%%%%%%%%%%%%%%%%%%%%%%%%%%%
   \item[1.6.25]  Let $n \in \Z^+$, let $r$ and $s$ be the usual generators of
                  $D_{2n}$ and let $\theta = 2\pi/n$.
                  \begin{enumerate}
                     \item Prove that the matrix
                           $\left(\begin{tabular}{@{}cr@{}}
                              $\cos\theta$ & $-\sin\theta$ \\
                              $\sin\theta$ & $\cos\theta$
                           \end{tabular}\right)$ is the matrix of the linear
                           transformation which rotates the $x$, $y$ plane about
                           the origin in a counterclockwise direction by
                           $\theta$ radians.
                     \item Prove that the map
                           $\varphi : D_{2n} \rightarrow GL_2(\R)$ defined on
                           generators by
                           $$\varphi(r) = \left(\begin{tabular}{@{}cr@{}}
                              $\cos\theta$ & $-\sin\theta$ \\
                              $\sin\theta$ & $\cos\theta$
                           \end{tabular}\right) \quad\text{and}\quad\varphi(s) =
                           \left(\begin{tabular}{@{}cc@{}}
                              0 & 1 \\
                              1 & 0
                           \end{tabular}\right)$$
                           extends to a homomorphism of $D_{2n}$ into
                           $GL_2(\R)$.
                     \item Prove that the homomorphism $\varphi$ in part (b) is
                           injective.
                  \end{enumerate}
                  
      \textbf{Proof.}
      
      \begin{enumerate}
         \item Consider $T_\theta : \R^2 \rightarrow \R^2$,
               $(x, y) \mapsto (x\cos\theta - y\sin\theta,
               x\sin\theta + y\cos\theta)$, the linear transformation which
               rotates the cartesian plane about the origin in a
               counterclockwise direction by $\theta$ radians. Consider the
               standard basis $\{(1, 0), (0, 1)\}$ for $\R^2$. A simple
               calculation will show us that the matrix of $T_\theta$ with
               respect to the standard basis is $\left(\begin{tabular}{@{}cr@{}}
                  $\cos\theta$ & $-\sin\theta$ \\
                  $\sin\theta$ & $\cos\theta$
               \end{tabular}\right)$. 
         \item It suffices to show that $\varphi(r)$ and $\varphi(s)$ satisfy
               (in $GL_2(\R)$) the relations satisfied by $r$ and $s$
               (in $D_{2n}$). It is clear that $\varphi(s)^2$ is the identity
               matrix. Also $\varphi(r)^n$ is the identity matrix because the
               least (positive) number of rotations we need to get back to the
               starting point is $n$. Finally we have that
               $$\varphi(r)\varphi(s) = \left(\begin{tabular}{@{}rc@{}}
                  $-\sin\theta$ & $\cos\theta$ \\
                  $\cos\theta$ & $\sin\theta$
               \end{tabular}\right) = \varphi(s)\varphi(r)^{-1},$$
               as desired.
         \item First note that (using induction)
               $$\varphi(r^p) = \left(\begin{tabular}{@{}cr@{}}
                  $\cos p\theta$ & $-\sin p\theta$ \\
                  $\sin p\theta$ & $\cos p\theta$
               \end{tabular}\right) \qquad \text{for all } p \in \Z.$$
               Now suppose that $\varphi(r^is^j) = \varphi(r^xs^y)$. That is,
               $\varphi(r^i)\varphi(s^j) = \varphi(r^x)\varphi(s^y)$, so that
               $\varphi(r^{i-x}) = \varphi(s^{y-j})$.

               \textbf{Case 1.} $y - j \equiv 0$ mod 2. That is $s^y = s^j$,
               so that $$\left(\begin{tabular}{@{}cr@{}}
                  $\cos [(i-x)\theta]$ & $-\sin[(i-x)\theta]$ \\
                  $\sin[(i-x)\theta]$ & $\cos[(i-x)\theta]$
               \end{tabular}\right) = \varphi(r^{i-x}) = \varphi(s^{y-j}) =
               \varphi(1) = \left(\begin{tabular}{@{}cc@{}}
                  1 & 0 \\
                  0 & 1
               \end{tabular}\right).$$
               It follows that $(i - x)\theta = 2\pi k$ for some integer $k$.
               Recall that $\theta = 2\pi/n$; thus $i - x = nk$ and we have that
               $r^{i-x} = r^{nk} = 1$, so we can conclude that $r^i = r^x$. We
               have thus shown that $r^is^j = r^xs^y$.

               \textbf{Case 2.} $y - j \equiv 1$ mod 2. Now
               $$\left(\begin{tabular}{@{}cr@{}}
                  $\cos [(i-x)\theta]$ & $-\sin[(i-x)\theta]$ \\
                  $\sin[(i-x)\theta]$ & $\cos[(i-x)\theta]$
               \end{tabular}\right) = \varphi(r^{i-x}) = \varphi(s^{y-j}) =
               \varphi(s) = \left(\begin{tabular}{@{}cc@{}}
                  0 & 1 \\
                  1 & 0
               \end{tabular}\right),$$
               a contradiction because $-1 = \sin[(i-x)\theta] = 1$. Hence the
               only possibility is Case 1, wherein we showed that
               $r^is^j = r^xs^y$, so that $\varphi$ is injective.
      \end{enumerate} \qed
%%%%%%%%%%%%%%%%%%%%%%%%%%%%%%%%%%%%%1.6.26%%%%%%%%%%%%%%%%%%%%%%%%%%%%%%%%%%%%%
   \item[1.6.26]  Let $i$ and $j$ be the generators of $Q_8$ described in
                  Section 5. Prove that the map $\varphi$ from $Q_8$ to
                  $GL_2(\C)$ defined on generators by
                  $$\varphi(i) = \left(\begin{tabular}{@{}cc@{}}
                        $\sqrt{-1}$ & 0 \\
                        0         &  $-\sqrt{-1}$
                    \end{tabular}\right) \qquad \text{and} \qquad
                    \varphi(j) = \left(\begin{tabular}{@{}cr@{}}
                        0 & $-1$ \\
                        1 & 0
                    \end{tabular}\right)$$
                  extends to a homomorphism. Prove that $\varphi$ is injective.

      \textbf{Proof.} To show that $\varphi$ extends to an homomorphism, it
      suffices to show that $\varphi(i)$ and $\varphi(j)$ satisfy
      (in $GL_2(\C)$) the relations satisfied by $i$ and $j$ (in $Q_8$). Using
      Exercise 1.5.3, we must then show that $\varphi(i)^2 = \varphi(j)^2$ and
      $\varphi(i)^4 = 1$. These equalities immediately follow using matrix
      multiplication. Now suppose that $\varphi(i^xj^y) = \varphi(i^mj^n$). Then
      it follows that $\varphi(i)^{x-m} = \varphi(j)^{n-y}$. Notice that
      $|\varphi(i)| = |\varphi(j)| = 4$ and that $\varphi(i)^p = \varphi(j)^q$
      if and only if $p \equiv q \equiv 2$ (or 0) mod 4. So we must have that
      $x - m \equiv n - y \equiv 2 $ (or 0) mod 4. In either case, it follows
      that $i^{x - m} = j^{n - y}$. That is $i^xj^y = i^mj^n$, so that
      $\varphi$ is injective. \qed
\end{enumerate}

      \section{Group Actions}
         \begin{enumerate}
%%%%%%%%%%%%%%%%%%%%%%%%%%%%%%%%%%%%%1.7.1%%%%%%%%%%%%%%%%%%%%%%%%%%%%%%%%%%%%%%
   \item[1.7.1]   Let $F$ be a field. Show that the multiplicative group of
                  nonzero elements of $F$ (denoted by $F^\times$) acts on the
                  set $F$ by $g \cdot a = ga$, where $g \in F^\times$,
                  $a \in F$ and $ga$ is the usual product in $F$ of the two
                  field elements (state clearly which axioms in the definition
                  of a field are used).

      \textbf{Proof.} Let $f_1, f_2 \in F^\times$ and $f_3 \in F$. It is clear
      that $f_1f_3 \in F$ because $F$ is closed under multiplication. Since 1 is
      is the identity for $(F^\times, \cdot)$, we have that $1 \cdot f_3 = f_3$.
      By the associativity of multiplication in $F$, it follows that
      $f_1(f_2f_3) = (f_1f_2)f_3$. Thus $F^\times$ acts on $F$ by
      multiplication. \qed
%%%%%%%%%%%%%%%%%%%%%%%%%%%%%%%%%%%%%1.7.2%%%%%%%%%%%%%%%%%%%%%%%%%%%%%%%%%%%%%%
   \item[1.7.2]   Show that the additive group $\Z$ acts on itself by
                  $z \cdot a = z + a$ for all $z, a \in \Z$.

      \textbf{Proof.} Let $z_1$, $z_2$, and $z_3$ be integers. Clearly
      $z_1 \cdot z_3 = z_1 + z_3 \in \Z$. Notice that the identity of $(\Z, +)$ 
      is 0; thus $z_1 \cdot 0 = z_1 + 0 = z_1$. Using the associativity of $\Z$ 
      under +, we have that
      $$z_1 \cdot (z_2 \cdot z_3) = z_1 + (z_2 + z_3) = (z_1 + z_2) + z_3 =
      (z_1 \cdot z_2) \cdot z_3.$$
      Thus $\Z$ acts on itself by the given operation. \qed
%%%%%%%%%%%%%%%%%%%%%%%%%%%%%%%%%%%%%1.7.3%%%%%%%%%%%%%%%%%%%%%%%%%%%%%%%%%%%%%%
   \item[1.7.3]   Show that the additive group $\R$ acts on the $x, y$ plane
                  $\R \times \R$ by $r \cdot (x, y) = (x + ry, y)$.

      \textbf{Proof.} Let $r_1$, $r_2$, $r_3$, and $r_4$ be real numbers. Since
      $\R$ is closed under addition, we have that
      $r_1 \cdot (r_2, r_3) = (r_2 + r_1r_3, r_3) \in \R \times \R$. We also
      have that
      $$0 \cdot (r_1, r_2) = (r_1 + 0 \cdot r_2, r_2) = (r_1, r_2).$$
      Finally we have that
      \begin{align*}
         r_1 \cdot (r_2 \cdot (r_3, r_4)) &= r_1 \cdot (r_3 + r_2r_4, r_4) \\
            &= (r_3 + r_2r_4 + r_1r_4, r_4) \\
            &= (r_3 + (r_2 + r_1)r_4, r_4) \\
            &= (r_2 + r_1) \cdot (r_3, r_4), \\
      \end{align*}
      so that $\R$ acts on the cartesian plane by the given operation. \qed
%%%%%%%%%%%%%%%%%%%%%%%%%%%%%%%%%%%%%1.7.4%%%%%%%%%%%%%%%%%%%%%%%%%%%%%%%%%%%%%%
   \item[1.7.4]   Let $G$ be a group acting on a set $A$ and fix some $a \in A$.
                  Show that the following sets are subgroups of $G$.
                  \begin{enumerate}
                     \item the kernel of the action,
                     \item $\{g \in G : ga = a\}$---this subgroup is called the
                           \textit{stablizer} of $a$ in $G$.
                  \end{enumerate}

      \textbf{Proof.}

      \begin{enumerate}
         \item Let $K$ be the kernel of this action. The set $K$ is not empty
               because $1 \in K$. Now let $g, h \in K$ and $a_1 \in A$. So we
               have that
               \begin{align*}
                  (gh^{-1})a_1 &= g(h^{-1}a_1) \\
                     &= g(h^{-1}(ha_1)) &[ha_1 = a_1] \\
                     &= g((h^{-1}h)a_1) \\
                     & = g(1a_1) = ga_1 = a_1.
               \end{align*}

               Thus $gh^{-1} \in K$, so that $K$ is a subgroup of $G$.

         \item Let $G_a$ be the stabilizer of $a$. Argue similarly as in (a),
               replacing $a_1$ with $a$, to conclude that $G_a$ is a subgroup of
               $G$.
      \end{enumerate}  \qed
%%%%%%%%%%%%%%%%%%%%%%%%%%%%%%%%%%%%%1.7.5%%%%%%%%%%%%%%%%%%%%%%%%%%%%%%%%%%%%%%
   \item[1.7.5]   Prove that the kernel of an action of the group $G$ on the set
                  $A$ is the same as the kernel of the corresponding permutation
                  representation $G \rightarrow S_A$.

      \textbf{Proof.} Let $K_1$ be the kernel of the action of $G$ on $A$ and
      let $K_2$ be the kernel of the permutation representation
      $\varphi : G \rightarrow S_A$. The proof follows since
      \begin{align*}
         g \in K_1 &\Longleftrightarrow ga = a \quad\text{ for all } a \in A \\
            &\Longleftrightarrow \varphi(g)(a)=a\quad\text{ for all } a\in A \\
            &\Longleftrightarrow \varphi(g) = 1 \\
            &\Longleftrightarrow g \in K_2.
      \end{align*} \qed
%%%%%%%%%%%%%%%%%%%%%%%%%%%%%%%%%%%%%1.7.6%%%%%%%%%%%%%%%%%%%%%%%%%%%%%%%%%%%%%%
   \item[1.7.6]   Prove that a group $G$ acts faithfully on a set $A$ if and
                  only if the kernel of the action is the set consisting only of
                  the identity.

      \textbf{Proof.} Let $\varphi : G \rightarrow S_A$ be the corresponding 
      permutation representation.

      ($\Rightarrow$) Assume that $G$ acts faithfully on $A$. Let $g$ be an
      element of the kernel of this action. Then it follows that
      $\varphi(g) = 1$; but $\varphi$ is injective (since this action is
      faithul), so that $g = 1$, and it follows that the kernel of the action is
      trivial.

      ($\Leftarrow$) Conversely, suppose the kernel of this action is trivial.
      Then it follows by Exercise 1.7.5 that the kernel of $\varphi$ is also
      trivial. So suppose that $\varphi(g_2) = \varphi(g_3)$. Then we have
      that $\varphi(g_2{g_3}^{-1}) = 1$, and since the kernel of $\varphi$ is 
      trivial, it follows that $g_2{g_3}^{-1} = 1$. Thus $g_2 = g_3$ and we
      have that $\varphi$ is injective; that is, $G$ acts faithfully on $A$.
      \qed

%%%%%%%%%%%%%%%%%%%%%%%%%%%%%%%%%%%%%1.7.7%%%%%%%%%%%%%%%%%%%%%%%%%%%%%%%%%%%%%%
   \item[1.7.7]   Prove that in Example 2 in this section the action is
                  faithful when the vector space is nonzero.

      \textbf{Proof.} Consider a nonzero vector space $V$ over $F$. Let
      $F^\times$ act on $V$ by left multiplication. Suppose to the contrary that
      two unequal members of $F^\times$---say $f_1$ and $f_2$---induce the same 
      permutation on $V$. Let $v$ be a nonzero element in $V$. Then we must 
      particularly have that $f_1v = f_2v$, so that $(f_1 - f_2)v = 0$. Since
      $f_1 \neq f_2$, it follows that $f_1 - f_2 \neq 0$, so that $f_1 - f_2$
      has a multiplicative inverse in $F^\times$. Multiply $(f_1 - f_2)v = 0$
      by this inverse to get $v = 0$, a contradiction. Thus no two members of
      $F^\times$ induce the same permutation, so that the action of $F^\times$
      on $V$ is faithful. \qed
%%%%%%%%%%%%%%%%%%%%%%%%%%%%%%%%%%%%%1.7.8%%%%%%%%%%%%%%%%%%%%%%%%%%%%%%%%%%%%%%
   \item[1.7.8]   Let $A$ be a nonempty set and let $k$ be a positive integer
                  with $k \le |A|$. The symmetric group $S_A$ acts on the set
                  $B$ consisting of all subsets of $A$ of cardinality $k$ by
                  $\sigma \cdot \{a_1, \ldots, a_k\} = \{\sigma(a_1), \ldots,
                   \sigma(a_k)\}$.
                  \begin{enumerate}
                     \item Prove that this is a group action.
                     \item Describe explicitly how the elements (1 2) and
                           (1 2 3) act on the six 2-element subsets of
                           $\{1, 2, 3, 4\}$.
                  \end{enumerate}

      \textbf{Solution.}
   
      \begin{enumerate}
         \item \textbf{Proof.} Let $\sigma, \alpha \in S_A$ and
               $\{a_1, \ldots, a_k\} \in B$. Since $\sigma$ is injective, it 
               follows that $\sigma(a_1)$, $\ldots$, $\sigma(a_k)$ are all 
               distinct, so that $\{\sigma(a_1), \ldots, \sigma(a_k)\} \in B$.
               Now $1 \cdot\nobreak \{a_1, \ldots, a_k\} = \{a_1, \ldots, a_k\}$
               and
               \begin{align*}
                  \sigma \cdot (\alpha \cdot \{a_1, \ldots, a_k\}) &=
                     \sigma \cdot \{\alpha(a_1), \ldots, \alpha(a_k)\} \\
                     &= \{\sigma(\alpha(a_1)), \ldots, \sigma(\alpha(a_k))\} \\
                     &= \{(\sigma\circ\alpha)(a_1), \ldots,
                          (\sigma\circ\alpha)(a_k)\} \\
                     &= (\sigma\circ\alpha) \cdot \{a_1, \ldots, a_k\}.
               \end{align*}
               
               Thus the operation is a group action. \qed
         \item $$
               \begin{tabular}{@{}|c|c|c|c|c|c|c|@{}} \hline
                  $\circ$ & $\{1\;2\}$ & $\{1\;3\}$ & $\{1\;4\}$ & $\{2\;3\}$ &
                            $\{2\;4\}$ & $\{3\;4\}$ \\ \hline
                  (1 2)   & $\{1\;2\}$ & $\{2\;3\}$ & $\{2\;4\}$ & $\{1\;3\}$ &
                            $\{1\;4\}$ & $\{3\;4\}$ \\ \hline
                  (1 2 3) & $\{2\;3\}$ & $\{1\;2\}$ & $\{2\;4\}$ & $\{1\;3\}$ &
                            $\{3\;4\}$ & $\{1\;4\}$ \\ \hline
               \end{tabular}
               $$
      \end{enumerate}
%%%%%%%%%%%%%%%%%%%%%%%%%%%%%%%%%%%%%1.7.9%%%%%%%%%%%%%%%%%%%%%%%%%%%%%%%%%%%%%%
   \item[1.7.9]   Do both parts of the preceding exercise with ``ordered
                  $k$-tuples" in place of ``$k$-element subsets," where the
                  action on $k$-tuples is defined as above but with set braces
                  replaced by parentheses (note that, for example, the 2-tuples
                  (1, 2) and (2, 1) are different even though the sets
                  $\{1, 2\}$ and $\{2, 1\}$ are the same, so the sets being
                  acted upon are different).

      \textbf{Solution.}
   
      \begin{enumerate}
         \item The proof is similar to Exercise 1.7.8 (a).
         \item $$
               \begin{tabular}{@{}|c|c|c|c|c|c|c|c|c|c|c|c|c|@{}} \hline
                  $\circ$ & (1 2) & (2 1) & (1 4) & (4 1) &
                            (2 4) & (4 2) \\ \hline
                  (1 2)   & (2 1) & (1 2) & (2 4) & (4 2) &
                            (1 4) & (4 1) \\ \hline
                  (1 2 3) & (2 3) & (2 1) & (2 4) & (3 1) &
                            (3 4) & (1 4) \\ \hline
               \end{tabular}
               $$
               and
               $$
               \begin{tabular}{@{}|c|c|c|c|c|c|c|c|c|c|c|c|c|@{}} \hline
                  $\circ$ & (1 3) & (3 1) & (2 3) & (3 2) &
                            (3 4) & (4 3) \\ \hline
                  (1 2)   & (2 3) & (3 2) & (1 3) & (3 1) &
                            (3 4) & (4 3) \\ \hline
                  (1 2 3) & (2 1) & (1 2) & (3 1) & (1 3) &
                            (1 4) & (4 1) \\ \hline
               \end{tabular}
               $$
      \end{enumerate}
%%%%%%%%%%%%%%%%%%%%%%%%%%%%%%%%%%%%%1.7.10%%%%%%%%%%%%%%%%%%%%%%%%%%%%%%%%%%%%%
   \item[1.7.10]  With reference to the preceding two exercises determine:
                  \begin{enumerate}
                     \item for which values of $k$ the action of $S_n$ on
                           $k$-element subsets is faithful, and
                     \item for which values of $k$ the action of $S_n$ on
                           ordered $k$-tuples is faithful.
                  \end{enumerate}

      \textbf{Solution.}
   
      \begin{enumerate}
         \item \textit{If $k < n$, then this action is faithful}.

               \textbf{Proof.} By Exercise 1.7.6, it suffices to show that
               kernel of this action is trivial (Exercise 1.7.6). Let $\sigma$ 
               be a nonidentity element in $S_n$, so that $\sigma(i) = j$ for 
               some $i \neq j$. Let $B$ be any $k$-subset such that $i \in B$
               and $j \neq B$. Notice that $B \neq \sigma B$ since
               $j \in \sigma B$. Thus $\sigma$ is not in the kernel of this 
               action. Since $\sigma$ was an arbitrary nonidentity element, it 
               follows that the kernel of this action is trivial, so that this 
               action is faithful by Exercise 1.7.6. \qed \\
   
               \textit{If $k = n$, then this action is not faithful}.

               \textbf{Proof.} Observe that the set of all $n$-element subsets 
               contains a single element, $I_n = \{1, \ldots, n\}$. Thus every 
               element of $S_n$ must map $I_n$ to itself, so that every element 
               of $S_n$ induces the same permutation. \qed
         \item \textit{This action is faithful for all $k \le n$}.

               \textbf{Proof.} Let $\sigma$ be a nonidentity element in $S_n$.
               Then $\sigma(i) = j$ for some $i \neq j$. Let $B$ be any
               $k$-tuple with $i$ as its first element. Then the first element
               of $\sigma A$ is $j$; thus $A \neq \sigma A$; that is, the kernel 
               of this action is trivial, so that this action of $S_n$ on
               ordered $k$-tuples is faithful. \qed
      \end{enumerate}
%%%%%%%%%%%%%%%%%%%%%%%%%%%%%%%%%%%%%1.7.11%%%%%%%%%%%%%%%%%%%%%%%%%%%%%%%%%%%%%
   \item[1.7.11]  Write out the cycle decomposition of the eight permutations in
                  $S_4$ corresponding to the elements of $D_8$ given by the
                  action of $D_8$ on the vertices of a square (where the
                  vertices of the square are labelled as in Section 2).

      \textbf{Solution.}

      $$
         \begin{tabular}{@{}|c|c|@{}} \hline
            Element in $D_8$ & Corresponding element in $S_4$ \\ \hline         
            1      & (1) \\ \hline
            $r$    & (1 2 3 4) \\ \hline
            $r^2$  & (1 3)(2 4) \\ \hline
            $r^3$  & (1 4 3 2) \\ \hline        
            $s$    & (2 4) \\ \hline
            $sr$   & (1 4)(2 3) \\ \hline
            $sr^2$ & (1 3) \\ \hline
            $sr^3$ & (1 2)(3 4)\\ \hline
         \end{tabular}
      $$
%%%%%%%%%%%%%%%%%%%%%%%%%%%%%%%%%%%%%1.7.12%%%%%%%%%%%%%%%%%%%%%%%%%%%%%%%%%%%%%
   \item[1.7.12]  Assume $n$ is an even positive integer and show that $D_{2n}$
                  acts on the set consisting of pairs of opposite vertices of a
                  regular $n$-gon. Find the kernel of this action (label
                  vertices as usual).

      \textbf{Proof.} Assume that $n$ is an even integer greater than 3. So
      write $n = 2k$. Label the vertices of a regular $n$-gon as was done on
      Page 24 of the text. Let $I_n = \{1, 2, \ldots, n\}$ be the set of
      vertices of the $n$-gon. Also for the sake of convenience, we shall assume
      that two vertices $x$ and $y$ are equal if and only if $x \equiv y$ mod
      $n$; thus, for example, vertex 0 = vertex $n$ = vertex $-n$, and so on.
      A pair of vertices $\{a, b\}$ is said to be an opposite pair if and only 
      if $|a - b| = k$ mod $n$. Thus the set of opposite pairs, $P$, is
      $$\{\{x, x + k\} : 1 \le x \le k\} =
        \{\{1, 1 + k\}, \{2, 2 + k\}, \ldots, \{k, 2k\}\}.$$
      Let $r$, $s$ be the usual generators of $D_{2n}$. Viewing these generators
      as members of $S_n$, it follows that (in cycle notation)
      \begin{itemize}
         \item $r =$(1, 2, $\ldots$, $n$) and
         \item $s = \D\prod_{i=2}^k(i, 2k+2-i) = (2, 2k)(3, 2k-1)\cdots(k, k+2)$.
      \end{itemize}

      Now consider a pair of vertices $\{x, x + k\} \in P$. It follows that
      $$r \cdot \{x, x + k\} = \{(x + 1, x + k + 1\} \in P$$
      because $x + k + 1 - (x + 1) = k$. Assume $x = 1$. Since $s$ fixes 1 and
      $k + 1$, it must also fix $\{x, x + 1\}$. Now assume $2 \le x \le k$. Thus
      $$s \cdot \{x, x + k\} = \{2k+2 - x, 2k+2-x-k\} \in P.$$
      We have thus shown that $r$ and $s$ send every opposite pair to another
      opposite pair; since $r$ and $s$ generate it follows that every member of
      $D_{2n}$ maps an opposite pair to another opposite pair. Since we are
      regarding $r$ and $s$ as permutations in $S_n$, axiom 2 immediately
      follows by the composition of permutations. The kernel of this action is
      $\{1, r^k\}$. \qed
%%%%%%%%%%%%%%%%%%%%%%%%%%%%%%%%%%%%%1.7.13%%%%%%%%%%%%%%%%%%%%%%%%%%%%%%%%%%%%%
   \item[1.7.13]  Find the kernel of the left regular action.

      \textbf{Proof.} Let $(G, *)$ be a group. Recall that the left regular 
      action of $G$ on itself is defined by $g \cdot a = g * a$ for all
      $g, a \in G$. Let $x$ be in the kernel of this action. Then we must
      particularly have that $x = x * 1 = x \cdot 1 = 1$. Thus the kernel of the
      regular action of a group on itself is its trivial subgroup.
      \qed
%%%%%%%%%%%%%%%%%%%%%%%%%%%%%%%%%%%%%1.7.14%%%%%%%%%%%%%%%%%%%%%%%%%%%%%%%%%%%%%
   \item[1.7.14]  Let $G$ be a group and let $A = G$. Show that if $G$ is
                  non-abelian then the maps defined by $g \cdot a = ag$ for all
                  $g, a \in G$ \textit{do not} satisfy the axioms of a (left)
                  group action of $G$ on itself.

      \textbf{Proof.} Assume that $G$ is non-abelian. Suppose to the contrary 
      that the map defined by $g \cdot a = ag$ for all $g, a \in G$ satisfies
      the axioms of a (left) group action. Let $x, y, z \in G$. Then it follows
      by axiom 2 of a group action that $x \cdot (y \cdot z) = (xy) \cdot z$.
      Now
      \begin{align*}
         x \cdot (y \cdot z) &= (xy)\cdot z &\Longleftrightarrow \\
         x \cdot (zy) &= z(xy) &\Longleftrightarrow \\
         (zy)x &= z(xy) &\Longleftrightarrow \\
         z(yx) &= z(xy) &\Longleftrightarrow \\
         yx &= xy &\Longleftrightarrow,
      \end{align*}
      so that $G$ is abelian, a contradiction. Thus this map does not define a
      group action. \qed
%%%%%%%%%%%%%%%%%%%%%%%%%%%%%%%%%%%%%1.7.15%%%%%%%%%%%%%%%%%%%%%%%%%%%%%%%%%%%%%
   \item[1.7.15]  Let $G$ be any group and let $A = G$. Show that the maps
                  defined by $g \cdot a = ag^{-1}$ for all $g, a \in G$
                  \textit{do} satisfy the axioms of a (left) group action of $G$
                  on itself.

      \textbf{Proof.} We shall only verify the second axiom since the first one
      is trivial. Let $x, y, z \in G$. It follows that
      \begin{align*}
         x \cdot (y \cdot z) &= x \cdot (zy^{-1}) \\
            &= zy^{-1}x^{-1} \\
            &= z(xy)^{-1} \\
            &= (xy) \cdot z,
      \end{align*}
      so that axiom 2 holds. Thus this map defines a group action. \qed
%%%%%%%%%%%%%%%%%%%%%%%%%%%%%%%%%%%%%1.7.16%%%%%%%%%%%%%%%%%%%%%%%%%%%%%%%%%%%%%
   \item[1.7.16]  Let $G$ be any group and let $A = G$. Show that the maps
                  defined by $g \cdot a = gag^{-1}$ for all $g, a \in G$
                  \textit{do} satisfy the axioms of a (left) group action(this
                  action of $G$ on itself is called \textit{conjugation}).

      \textbf{Proof.} We shall only verify the second axiom since the first one
      is trivial. Let $x, y, z \in G$. It follows that
      \begin{align*}
         x \cdot (y \cdot z) &= x \cdot (yzy^{-1}) \\
            &= xyzy^{-1}x^{-1} \\
            &= xyz(xy)^{-1} \\
            &= (xy) \cdot z,
      \end{align*}
      so that axiom 2 holds. Thus this map defines a group action. \qed
%%%%%%%%%%%%%%%%%%%%%%%%%%%%%%%%%%%%%1.7.17%%%%%%%%%%%%%%%%%%%%%%%%%%%%%%%%%%%%%
   \item[1.7.17]  Let $G$ be a group and let $G$ act on itself by left 
                  conjugation, so each $g \in G$ maps $G$ to $G$ by
                  $$x \mapsto gxg^{-1}.$$
                  For fixed $g \in G$, prove that conjugation by $g$ is an
                  isomorphism from $G$ onto itself. Deduce that $x$ and
                  $gxg^{-1}$ have the same order for all $x \in G$ and that for
                  any subset $A$ of $G$, $|A| = |gAg^{-1}|$ (here
                  $gAg^{-1} = \{gag^{-1} : a \in A\})$.

      \textbf{Proof.} Let $g \in G$. Define $\sigma_g : G \rightarrow G$ by
      $x \mapsto gxg^{-1}$. This map is bijective since it has a 2-sided
      inverse, namely $\sigma_{g^{-1}}$. So it remains to show that $\sigma_g$
      is a homomorphism. This follows immediately because
      \begin{align*}
         \sigma_g(xy) = gxyg^{-1} = gxg^{-1}gyg^{-1} = \sigma_g(x)\sigma_g(y)
      \end{align*}
      for all $x, y \in G$. By Exercise 1.6.2, it follows that
      $|x| = |\sigma_g(x)| = |gxg^{-1}|$ for all $x \in G$. Let $A \subseteq G$.
      Then the map $\alpha_g : A \rightarrow gAg^{-1}$ defined by
      $a \mapsto gag^{-1}$ is also bijective (its 2-sided inverse is
      $\alpha_{g^{-1}}$); thus $|A| = |gAg^{-1}|$. \qed      
%%%%%%%%%%%%%%%%%%%%%%%%%%%%%%%%%%%%%1.7.18%%%%%%%%%%%%%%%%%%%%%%%%%%%%%%%%%%%%%
   \item[1.7.18]  Let $H$ be a group acting on a set $A$. Prove that the
                  relation $\sim$ on $A$ defined by
                  $$a \sim b \quad \text{if and only if} \quad
                    a = hb \quad \text{for some }h \in H$$
                  is an equivalence relation. (For each $x \in A$ the
                  equivalence class of $x$ under $\sim$ is called the
                  \textit{orbit} of $x$ under the action of $H$. The orbits
                  under the action of $H$ partition the set $A$.)

      \textbf{Proof.} Let $a, b, c \in H$.

      \textbf{Reflexivity.} Since $1a = a$, it follows that $a \sim a$, so that
      $\sim$ is reflexive on $A$.

      \textbf{Symmetry.} Suppose $a \sim b$. Then there exists $h \in H$ such
      that $a = hb$. Now $h^{-1}a = h^{-1}hb = 1b = b$, so that $b \sim a$; thus
      $\sim$ is symmetric on $A$.

      \textbf{Transitivity.} Suppose $a \sim b$ and $b \sim c$. Then there exist
      $h_1, h_2 \in H$ such that $a = h_1b$ and $b = h_2c$. Thus
      $a = h_1b = h_1(h_2c) = (h_1h_2)c$, so that $a \sim c$; thus $\sim$ is 
      transitive on $A$.

      We can thus conclude that $\sim$ is an equivalence relation on $A$. \qed
%%%%%%%%%%%%%%%%%%%%%%%%%%%%%%%%%%%%%1.7.19%%%%%%%%%%%%%%%%%%%%%%%%%%%%%%%%%%%%%
   \item[1.7.19]  Let $H$ be a subgroup of the finite group $G$ and let $H$ act
                  on $G$ (here $A = G$) by left multiplication. Let $x \in G$
                  and let $\mathcal{O}$ be the orbit of $x$ under the action of
                  $H$. Prove that the map
                  $$H \rightarrow \mathcal{O}\quad \text{defined by} \quad
                    h \mapsto hx$$
                  is a bijection (hence all orbits have cardinality $|H|$). From
                  this and the preceding exercise deduce
                  $\textit{Lagrange's Theorem}:$
                  \begin{center}
                     \textit{if $G$ is a finite group and $H$ is a subgroup of
                     $G$ then $|H|$ divides $|G|$}.
                  \end{center}
                  
      \textbf{Proof.} Let $x \in G$ and let $\mathcal{O}$ be the orbit of $x$
      under the action of $H$ on $G$. Consider the map
      $\alpha : H \rightarrow \mathcal{O}$, $h \mapsto hx$. Suppose
      $\alpha(a) = \alpha(b)$. Then it follows that $ax = bx$, so that $a = b$,
      by cancellation. Thus $\alpha$ is injective. Since $G$ is finite, $H$
      must also be finite; thus $\alpha$ is necessarily surjective, and we 
      conclude that $\alpha$ is bijective. That is $|H| = |\mathcal{O}|$. Let
      $|G| = n$. By Exercise 1.7.18, the action of $H$ on $G$ partitions $G$
      into $m$ distinct orbits (there is a finite number of orbits because $G$
      is finite). Since $x$ was arbitrary, it follows that each orbit of this
      action has size $|H|$, so that $m\cdot|H| = |G|$. That is $|H| \mid |G|$.
      \qed
%%%%%%%%%%%%%%%%%%%%%%%%%%%%%%%%%%%%%1.7.20%%%%%%%%%%%%%%%%%%%%%%%%%%%%%%%%%%%%%
   \item[1.7.20]  Show that the group of rigid motions of a tetrahedron is
                  isomorphic to a subgroup of $S_4$.
%%%%%%%%%%%%%%%%%%%%%%%%%%%%%%%%%%%%%1.7.21%%%%%%%%%%%%%%%%%%%%%%%%%%%%%%%%%%%%%
   \item[1.7.21]  Show that the group of rigid motions of a cube is isomorphic
                  to $S_4$. [This group acts on the set of four pairs of
                  opposite vertices.]
%%%%%%%%%%%%%%%%%%%%%%%%%%%%%%%%%%%%%1.7.22%%%%%%%%%%%%%%%%%%%%%%%%%%%%%%%%%%%%%
   \item[1.7.22]  Show that the group of rigid motions of an octahedron is
                  isomorphic to $S_4$. [This group acts on the set
                  of four pairs of opposite faces.] Deduce that the groups of
                  rigid motions of a cube and an octahedron are isomorphic.
                  (These groups are isomorphic because these solids are ``dual"
                  ---see \textit{Introduction to Geometry} by H.Coxeter, Wiley,
                  1961. We shall see later that the groups of rigid motions of
                  the dodecahedron and icosahedron are isomorphic as well---
                  these solids are also dual.)
%%%%%%%%%%%%%%%%%%%%%%%%%%%%%%%%%%%%%1.7.23%%%%%%%%%%%%%%%%%%%%%%%%%%%%%%%%%%%%%
   \item[1.7.23]  Explain why the action of the group of rigid motions of a cube
                  on the set of three pairs of opposite faces is not faithful.
                  Find the kernel of this action.
\end{enumerate}

         
   \chapter{Subgroups}
      \section{Definition And Examples}
         Let
   $$\mathcal{A} \mbox{ be the set of real } 2 \times 2
     \mbox{ matrices}, M = \left(
      \begin{tabular}{@{}l r@{}} 
         1 & 1\\ 
         0 & 1
      \end{tabular}\right)$$
and let
   $$\mathcal{B} = \{X \in \mathcal{A} : MX = XM\}.$$
\begin{enumerate}
%%%%%%%%%%%%%%%%%%%%%%%%%%%%%%%%%%%Prob0.1.1%%%%%%%%%%%%%%%%%%%%%%%%%%%%%%%%%%%%
   \item[0.1.1] Determine which of the following elements of $\mathcal{A}$ lie
                in $\mathcal{B}$:

                $$M = \left(
                   \begin{tabular}{@{}l r@{}} 
                      1 & 1 \\ 
                      0 & 1
                   \end{tabular}\right), \left(
                   \begin{tabular}{@{}l r@{}} 
                      1 & 1 \\ 
                      1 & 1
                   \end{tabular}\right), \left(
                   \begin{tabular}{@{}l r@{}} 
                      0 & 0 \\ 
                      0 & 0
                   \end{tabular}\right), \left(
                   \begin{tabular}{@{}l r@{}} 
                      1 & 1 \\ 
                      1 & 0
                   \end{tabular}\right), \left(
                   \begin{tabular}{@{}l r@{}} 
                      1 & 0 \\ 
                      0 & 1
                   \end{tabular}\right), \left(
                   \begin{tabular}{@{}l r@{}} 
                      0 & 1 \\ 
                      1 & 1
                   \end{tabular}\right).$$
      \textbf{Solution:} A quick computation will show us only these matrices
      lie in $\mathcal{B}$.

                $$\left(
                   \begin{tabular}{@{}l r@{}} 
                      1 & 1 \\ 
                      0 & 1
                   \end{tabular}\right), \left(
                   \begin{tabular}{@{}l r@{}} 
                      0 & 0 \\ 
                      0 & 0
                   \end{tabular}\right), \left(
                   \begin{tabular}{@{}l r@{}} 
                      1 & 0 \\ 
                      0 & 1
                   \end{tabular}\right).$$
%%%%%%%%%%%%%%%%%%%%%%%%%%%%%%%%%%%Prob0.1.2%%%%%%%%%%%%%%%%%%%%%%%%%%%%%%%%%%%%
   \item[0.1.2] Prove that if $P, Q \in \mathcal{B}$, then
                $P + Q \in \mathcal{B}$.

      \textbf{Proof:} Suppose that $P, Q \in \mathcal{B}$. We want to show that
      $P + Q \in \mathcal{B}$. That is, we must show that $M(P + Q) = (P + Q)M$.
      Thus
      \begin{align*}
         M(P + Q) &= MP + MQ   &[\text{Distributive Law}]\\
                  &= PM + QM   &[P, Q \in \mathcal{B}] \\
                  &= (P + Q)M. &[\text{Distributive Law}]
      \end{align*} \qed
%%%%%%%%%%%%%%%%%%%%%%%%%%%%%%%%%%%Prob0.1.3%%%%%%%%%%%%%%%%%%%%%%%%%%%%%%%%%%%%
   \item[0.1.3] Prove that if $P, Q \in \mathcal{B}$, then
                $P \cdot Q \in \mathcal{B}$.

      \textbf{Proof:} Suppose that $P, Q \in \mathcal{B}$. We want to show that
      $PQ \in \mathcal{B}$. That is, we must show that $M(PQ) = (PQ)M$.
      Then
      \begin{align*}
         M(PQ) &= (MP)Q  &[\text{Associative Law}]\\
               &= (PM)Q  &[P \in \mathcal{B}] \\
               &= P(MQ)  &[\text{Associative Law}] \\
               &= P(QM)  &[Q \in \mathcal{B}] \\
               &= (PQ)M.  &[\text{Associative Law}]
      \end{align*} \qed
%%%%%%%%%%%%%%%%%%%%%%%%%%%%%%%%%%%Prob0.1.4%%%%%%%%%%%%%%%%%%%%%%%%%%%%%%%%%%%%
   \item[0.1.4] Find conditions on $p, q, r, s$ which determine precisely when
                $$\left(
                     \begin{tabular}{@{}l r@{}} 
                        $p$ & $q$ \\ 
                        $r$ & $s$
                     \end{tabular}\right) \in \mathcal{B}.$$

      \textbf{Proof:} Suppose $A = \left(\begin{tabular}{@{}l r@{}} 
      $p$ & $q$ \\ 
      $r$ & $s$
      \end{tabular}\right) \in \mathcal{B}$. Then it must be the case that
      $AM = MA$, so that
      $$ \left(
         \begin{tabular}{@{}l r@{}} 
            $p$ & $q$ \\ 
            $r$ & $s$
         \end{tabular}\right)\left(
         \begin{tabular}{@{}l r@{}} 
            1 & 1 \\ 
            0 & 1
         \end{tabular}\right) = \left(
         \begin{tabular}{@{}l r@{}} 
            1 & 1 \\ 
            0 & 1
         \end{tabular}\right)\left(
         \begin{tabular}{@{}l r@{}} 
            $p$ & $q$ \\ 
            $r$ & $s$
         \end{tabular}\right).
      $$

      That is,
      $$ \left(
         \begin{tabular}{@{}l r@{}} 
            $p$ & $p + q$ \\ 
            $r$ & $r + s$
         \end{tabular}\right) = \left(
         \begin{tabular}{@{}c c@{}} 
            $p + r$ & $q + s$ \\ 
            $r$ & $s$
         \end{tabular}\right).
      $$

      Equating corresponding entries tells us that
      $r = 0$ and $p = s$. Thus
      $$\mathcal{B} = \left\{\left(
         \begin{tabular}{@{}l r@{}} 
            $p$ & $q$ \\ 
            $0$ & $p$
         \end{tabular}\right) \in \mathcal{A}\right\}.$$
%%%%%%%%%%%%%%%%%%%%%%%%%%%%%%%%%%%Prob0.1.5%%%%%%%%%%%%%%%%%%%%%%%%%%%%%%%%%%%%
   \item[0.1.5] Determine whether the following functions $f$ are well defined:
                \begin{enumerate}
                   \item $f : \Q \rightarrow \Z$ defined by $f(a/b) = a$.
                   \item $f : \Q \rightarrow \Q$ defined by $f(a/b) = a^2/b^2$.
                \end{enumerate}

      \textbf{Solution:}
         \begin{enumerate}
            \item $f$ is not well defined because $4/1 = 8/2$, but
                  $f(4/1) = 4 \neq 8 = f(8/2)$.
            \item We claim that $f$ is well defined.

                  \textbf{Proof:} We want to show that all representatives for 
                  an element in $\Q$ have the same output under $f$. So let
                  $a/b \in \Q$ with $\gcd(a, b) = 1$. Suppose $m \in \Q$ such 
                  that $m = a/b$. Then we must have that $m = ka/kb$ where $k$ 
                  is some nonzero integer. It follows that
                  $f(ka/kb) = k^2a^2/k^2b^2 = a^2/b^2 = f(a/b)$. \qed
         \end{enumerate}
%%%%%%%%%%%%%%%%%%%%%%%%%%%%%%%%%%%Prob0.1.6%%%%%%%%%%%%%%%%%%%%%%%%%%%%%%%%%%%%
   \item[0.1.6] Determine whether the function $f: \R^+ \rightarrow \Z$ defined
                by mapping a real number $r$ to the first digit to the right of
                the decimal point in a decimal expansion of $r$ is well defined.

      \textbf{Proof:} The map $f$ is not well defined because we have
      $0.\overline{9} = 1.\overline{0}$, but
      $f(0.\overline{9}) = 9 \neq 0 = f(1.\overline{0})$. \qed
%%%%%%%%%%%%%%%%%%%%%%%%%%%%%%%%%%%Prob0.1.7%%%%%%%%%%%%%%%%%%%%%%%%%%%%%%%%%%%%
   \item[0.1.7] Let $f : A \rightarrow B$ be a surjective map of sets. Prove
                that the relation
                $$a \sim b \mbox{ if and only if } f(a) = f(b)$$
                is an equivalence relation whose equivalence classes are the
                fibers of $f$.

      \textbf{Proof:} We want to show that the relation $\sim$ is an equivalence
      relation on $A$. That is, we want to show that $\sim$ is reflexive,
      symmetric, and transitive on $A$. Thus
      \begin{enumerate}
         \item[\textit{Reflexivity}:]  Let $a \in A$. Since $f(a) = f(a)$, it
                                       follows that $a \sim a$, so that $\sim$ 
                                       is reflexive on $A$.
         \item[\textit{Symmetry}:]     Let $a, b \in A$ with $a \sim b$. Since
                                       $a \sim b$, we must have that
                                       $f(a) = f(b)$; it immediately follows    
                                       that $f(b) = f(a)$, so that $b \sim a$;
                                       that is, $\sim$ is symmetric on $A$.
         \item[\textit{Transitivity}:] Let $a, b, c \in A$ with $a \sim b$ and
                                       $b \sim c$. Thus by definition, we have
                                       that $f(a) = f(b)$ and $f(b) = f(a)$. By
                                       transitivity of the equality relation it
                                       follows that $f(a) = f(c)$, so that
                                       $a \sim c$. Thus $\sim$ is transitive on
                                       $A$.
      \end{enumerate}

      Now we must show that the fibers of $f$ are the equivalence classes of
      $\sim$. So let $\mathcal{C}$ be the set of equivalence classes of $\sim$ 
      and let  $\mathcal{F}$ be the set of fibers of $f$. First we want to show
      that $\mathcal{F} \subseteq \mathcal{C}$. So consider
      $F \in \mathcal{F}$. Let $h \in F$. Then we see that $\overline{h} = F$
      (because if $h' \in \overline{h}$ then $f(h') = f(h)$, so that $h' \in F$,
      and $\overline{h} \subseteq F$. Also if $g \in F$, then $f(h) = f(g)$ so  
      that $f \in \overline{h}$ and $F \subseteq \overline{h}$), so that $F \in
      \mathcal{C}$ and $\mathcal{F} \subseteq \mathcal{C}$. Now we want to show
      that $\mathcal{C} \subseteq \mathcal{F}$ to complete the proof. Suppose
      $C \in \mathcal{C}$. Let $c \in C$. Then $C$ is the fiber of $f(c)$. Thus
      $C \in \mathcal{F}$ so that $\mathcal{C} \subseteq \mathcal{F}$. Since
      $\mathcal{F} \subseteq \mathcal{C}$ and
      $\mathcal{C} \subseteq \mathcal{F}$, we must have that
      $\mathcal{C} = \mathcal{F}$. \qed
%%%%%%%%%%%%%%%%%%%%%%%%%%%%%%%%%%%Prob0.1.8%%%%%%%%%%%%%%%%%%%%%%%%%%%%%%%%%%%%
   \item[0.1.8] \textbf{Proposition 1:}
                \begin{enumerate}
                   \item The map $f$ is injective if and only if $f$ has a left
                         inverse.
                   \item The map $f$ is surjective if and only if $f$ has a
                         right inverse.
                   \item The map $f$ is a bijection if and only if there exists
                         $g : B \rightarrow A$ such that $f \circ g$ is the
                         identity map on $B$ and $g \circ f$ is the identity map
                         on $A$.
                   \item If $A$ and $B$ are finite sets with the same number of
                         elements, then $f : A \rightarrow B$ is bijective if
                         and only if $f$ is injective if and only if $f$ is
                         surjective.
                \end{enumerate}

      \textbf{Proof:}
      \begin{enumerate}
         \item ($\Leftarrow$) Suppose first that the map $f : A \rightarrow B$ 
               has a left inverse, say $g : B \rightarrow A$. We want to show 
               that $f$ has is injective. So consider $a, b \in A$ such that
               $f(a) = f(b)$. Then we have that $g(f(a)) = g(f(b))$ so that
               $a = b$ since $g$ is a left inverse of $f$. Thus $f$ is
               injective. \\
               ($\Rightarrow$) Now suppose that $f$ is injective. Then we must 
               show that $f$ has	a left inverse. Notice that since $f$ is 
               injective, it must be the case that for every $b'' \in f(A)$ 
               there exists a unique $a'' \in A$ such that $f(a'') = b''$. Let 
               $a'$ be any member of $A$ and consider the map
	            $$g : B \rightarrow A,$$
	            where for every $b \in B$, we have
	            \begin{equation*}
		            g(b) = \left\{
			            \begin{array}{ll}
				            a & \text{if } b \in f(A) \text{ and where } f(a) = b,\\
                        a' & \text{if } b \in B\setminus f(A).
                     \end{array} \right.
               \end{equation*}
   
               So $(g \circ f)(a_1) = g(f(a_1)) = a_1$ for all $a_1 \in A$.
               Hence $g \circ f$ is the identity function on $A$, so that $g$ is
               a left inverse of $f$.

         \item $(\Leftarrow)$ Suppose that $f$ has a right inverse, say
               $g : B \rightarrow A$. We must now show that $f$ is also 
               surjective; so let $b \in B$. Then, by supposition, we have that 
               $f(g(b)) = b$. Thus $f$ maps $g(b)$---a member of $A$---to $b$, 
               so that $f$ is onto.
   
               $(\Rightarrow)$ Now suppose that $f$ is surjective. We now want 
               to show that $f$ has a right inverse. Let $b \in B$. Since the 
               surjectivity of $f$ guarantees that the fiber of $\{b\}$ over $f$
               is not empty, we let $h : B \rightarrow A$ be a function that 
               maps an element $b \in B$ to some element in the fiber of $\{b\}$
               over $f$. Clearly if $c \in B$, then $f(h(c)) = c$, so that $h$ 
               is a right inverse of $f$.
         \item $(\Leftarrow)$ Suppose that $g$ is a left and right inverse of
               $f$. Then by 1 and 2, $f$ is an injection and a surjection, so 
               that $f$ is a bijection.
   
               $(\Rightarrow)$ Now suppose that $f$ is a bijection. Let
               $b \in B$. We notice that the fiber of $\{b\}$ under $f$ is not 
               empty since $f$ is surjecitive, and this fiber contains exactly 
               one element of $A$. The latter is so since if $a_1, a_2$ are in 
               the fiber of $\{b\}$ over $f$, then $f(a_1) = f(a_2)$ so that 
               $a_1 = a_2$ by the injectivity of $f$. So let $g$ be the map
               $g : B \rightarrow A$ that maps $c \in B$ to the only element in 
               the fiber of $\{c\}$ over $f$. It is trivial to show that $g$ is 
               both a left and right inverse of $f$.

         \item Let $|A| = |B| = n \in \Z^+$. First we shall assume that $f$ is
               bijective. It immediately follows that $f$ is injective. Now 
               assume that $f$ is injective. Since $f$ is one to one, no two 
               elements of $A$ map to the same element in $B$. This implies that
               $f(A)$ must contain exactly $n$ elements since $A$ contains $n$ 
               elements. But $|B| = n$, so that $f(A) = B$. That is, $f$ is 
               surjective. Now suppose that $f$ is surjective. Since $f$ is
               surjective, none of its fibers is empty; thus, the number of
               fibers of $f$ must equal $|B| = n$. We shall argue by    
               contradiction that $f$ is injective. So suppose that $f$ is not
               injective. Let $f_1$, $f_2$, $\ldots$, $f_n$ be the $n$ fibers of
               $f$. Since $f$ is not injective, one of this fibers must contain
               more than 1 element. so assume without loss of generality that
               $|f_1| \ge 2$. Since these fibers are a partition of $A$ and
               since each fiber contains at least one element it follows that
               \begin{align*}
                  |A| &= |f_1| + |f_2| + \cdots + |f_n| \\
                      &\ge 2 + (n - 1) \\
                      &\ge n + 1,
               \end{align*}
               a contradiction since $|A| = n$. Thus $f$ is injective.	
      \end{enumerate}
\end{enumerate}

      \section{Centralizers And Normalizers, Stabilizes And Kernels}
         \begin{enumerate}
%%%%%%%%%%%%%%%%%%%%%%%%%%%%%%%%%%%%%0.2.1%%%%%%%%%%%%%%%%%%%%%%%%%%%%%%%%%%%%%%
   \item[0.2.1]   For each of the following pairs of integers $a$ and $b$,
                  determine their greatest common divisor, ther least common
                  multiple, and write their greatest common divisor in the form
                  $ax + by$ for some integers $x$ and $y$.
                  \begin{enumerate}
                     \item $a = 20$, $b = 13$.
                     \item $a = 69$, $b = 372$.
                     \item $a = 792$, $b = 275$.
                     \item $a = 11391$, $b = 5673$.
                     \item $a = 1761$, $b = 1567$.
                     \item $a = 507885$, $b = 60808$.
                  \end{enumerate}

      \textbf{Solution.}

      \begin{enumerate}
         \item Using the Euclidean Algorithm we have
               \begin{align*}
                  20 &= 1 \cdot 13 + 7 \\
                  13 &= 1 \cdot 7 + 6 \\
                  7  &= 1 \cdot 6 + 1, \text{ so that } \\ \\
                  1  &= 7 - 1 \cdot 6 \\
                     &= 7 - 1 \cdot (13 - 1 \cdot 7) \\
                     &= 2 \cdot 7 - 1 \cdot 13 \\
                     &= 2 \cdot (20 - 1 \cdot 13) - 1 \cdot 13 \\
                     &= 2 \cdot 20 - 3 \cdot 13.
               \end{align*}

               Thus $\gcd(20, 13) = 1$ and we have $x = 2$ and $y = -3$.
         \item Using the Euclidean Algorithm we have
               \begin{align*}
                  69  &= 0 \cdot 372 + 69 \\
                  372 &= 5 \cdot 69 + 27 \\
                  69  &= 2 \cdot 27 + 15 \\
                  27  &= 1 \cdot 15 + 12 \\
                  15  &= 1 \cdot 12 + 3 \\
                  12  &= 4 \cdot 3 + 0, \text{ so that } \\ \\
                   3 &= 15 - 1 \cdot 12 \\
                     &= 15 - 1 \cdot (27 - 1 \cdot 15) \\
                     &= 2 \cdot 15 - 1 \cdot 27 \\
                     &= 2 \cdot (69 - 2 \cdot 27) - 1 \cdot 27 \\
                     &= 2 \cdot 69 - 5 \cdot 27 \\
                     &= 2 \cdot 69 - 5 \cdot (372 - 5 \cdot 69) \\
                     &= 27 \cdot 69 - 5 \cdot 372 \\
                     &= 27 \cdot 69 - 5 \cdot 372.
               \end{align*}

               Thus $\gcd(69, 372) = 3$ and we have $x = 27$ and $y = -5$.
         \item Using the Euclidean Algorithm we have
               \begin{align*}
                  792 &= 2 \cdot 275 + 242 \\
                  275 &= 1 \cdot 242 + 33 \\
                  242 &= 7 \cdot 33 + 11 \\
                  33  &= 3 \cdot 11 + 0, \text{ so that } \\ \\
                  11  &= 242 - 7 \cdot 33 \\
                      &= 242 - 7 \cdot (275 - 1 \cdot 242) \\
                      &= 8 \cdot 242 - 7 \cdot 275 \\
                      &= 8 \cdot (792 - 2 \cdot 275) - 7 \cdot 275 \\
                      &= 8 \cdot 792 - 23 \cdot 275.
               \end{align*}

               Thus $\gcd(792, 275) = 11$ and we have $x = 8$ and $y = -23$.
         \item Using the Euclidean Algorithm we have
               \begin{align*}
                  11391 &= 2 \cdot 5673 + 45 \\
                  5673  &= 126 \cdot 45 + 3 \\
                  45    &= 3 \cdot 15 + 0, \text{ so that } \\ \\
                  3     &= 5673 - 126 \cdot 45 \\
                        &= 5673 - 126 \cdot (11391 - 2 \cdot 5673) \\
                        &= -126 \cdot 11391 + 253 \cdot 5673.
               \end{align*}

               Thus $\gcd(11391, 5673) = 3$ and we have $x = -126$ and
               $y = 253$.
         \item Using the Euclidean Algorithm we have
               \begin{align*}
                  1761 &= 1 \cdot 1567 + 194 \\
                  1567 &= 8 \cdot 194 + 15 \\
                  194  &= 12 \cdot 15 + 14 \\
                  15   &= 1 \cdot 14 + 1, \text{ so that } \\ \\
                   1   &= 15 - 1 \cdot 14 \\
                       &= 15 - 1 \cdot (194 - 12 \cdot 15) \\
                       &= 13 \cdot 15 - 1 \cdot 194 \\
                       &= 13 \cdot (1567 - 8 \cdot 194) - 1 \cdot 194 \\
                       &= 13 \cdot 1567 - 105 \cdot 194 \\
                       &= 13 \cdot 1567 - 105 \cdot (1761 - 1 \cdot 1567) \\
                       &= 118 \cdot 1567 - 105 \cdot 1761.
               \end{align*}

               Thus $\gcd(1761, 1567) = 1$ and we have $x = -105$ and $y = 118$.
         \item Using the Euclidean Algorithm we have
               \begin{align*}
                  507885 &= 8 \cdot 60808 + 21421 \\
                  60808  &= 2 \cdot 21421 + 17966 \\
                  21421  &= 1 \cdot 17966 + 3455 \\
                  17966  &= 5 \cdot 3455 +  691 \\
                  3455   &= 5 \cdot 691 + 0, \text{ so that } \\ \\
                  691    &= 17966 - 5 \cdot 3455 \\
                         &= 17966 - 5 \cdot (21421 - 1 \cdot 17966) \\
                         &= 6 \cdot 17966 - 5 \cdot 21421 \\
                         &= 6 \cdot (60808 - 2 \cdot 21421) - 5 \cdot 21421 \\
                         &= 6 \cdot 60808 - 17 \cdot 21421 \\
                         &= 6 \cdot 60808 - 17 \cdot (507885 - 8 \cdot 60808) \\
                         &= 142 \cdot 60808 - 17 \cdot 507885.
               \end{align*}

               Thus $\gcd(507885, 60808) = 691$ and we have $x = -17$ and
               $y = 142$.
      \end{enumerate}
%%%%%%%%%%%%%%%%%%%%%%%%%%%%%%%%%%%%%0.2.2%%%%%%%%%%%%%%%%%%%%%%%%%%%%%%%%%%%%%%
   \item[0.2.2]   Prove that if the integer $k$ divides the integers $a$ and $b$
                  then $k$ divides $as + bt$ for every pair of integers $s$ and
                  $t$.

      \textbf{Proof.} Let $a$ and $b$ be integers. Assume that $k$ divides
      $a$ and $b$. Consider any pair of integers $s$ and $t$. We want to show
      that $k$ also divides $as + bt$; that is, we must show that there exists
      some integer $m_1$ such that $as + bt = km_1$. Since $k$ divides $a$ and
      $b$, we must have that $a = km_2$ and $b = km_3$ for some integers $m_2$
      and $m_3$. Thus
      \begin{align*}
         as + bt &= km_2s + km_3t \\
                 &= k(m_2s + m_3t). 
      \end{align*}

      So take $m_1 = m_2s + m_3t$. \qed
%%%%%%%%%%%%%%%%%%%%%%%%%%%%%%%%%%%%%0.2.3%%%%%%%%%%%%%%%%%%%%%%%%%%%%%%%%%%%%%%
   \item[0.2.3]   Prove that if $n$ is composite then there are integers $a$ and
                  $b$ such that $n$ divides $ab$ but $n$ does not divide either
                  $a$ or $b$.

      \textbf{Proof.} Let $n > 1$ be a composite integer. Then $n = cd$, where
      $c$ and $d$ are integers greater than 1. Clearly, $n \mid cd$ because
      $n = cd$, but $n$ divides neither $c$ nor $d$, since they are both less
      than $n$. \qed
%%%%%%%%%%%%%%%%%%%%%%%%%%%%%%%%%%%%%0.2.4%%%%%%%%%%%%%%%%%%%%%%%%%%%%%%%%%%%%%%
   \item[0.2.4]   Let $a$, $b$ and $N$ be fixed integers with $a$ and $b$ 
                  nonzero and let $d = (a, b)$ be the greatest common divisor of
                  $a$ and $b$. Suppose $x_0$ and $y_0$ are particular solutions
                  to $ax + by = N$. Prove for any integer $t$ that the integers
                  $$x = x_0 + \frac{b}{d}t \qquad y = y_0 - \frac{a}{d}t$$
                  are also solutions to $ax + by = N$.

      \textbf{Proof.} Let $t$ be an integer, and let
		$$x = x_0 + \frac{b}{d}t \text{ and } y = y_0 - \frac{a}{d}t.$$
		Then we have
		\begin{align*}
			ax + by &= a\left(x_0 + \frac{b}{d}t\right) +
						  b\left(y_0 - \frac{a}{d}t\right) \\
					  &= ax_0 + by_0 \\
					  &= N,					  
		\end{align*}
		so that $x = x_0 + \frac{b}{d}t$ and $y = y_0 - \frac{a}{d}t$ are
		solutions to the equation $ax + by = N$.
%%%%%%%%%%%%%%%%%%%%%%%%%%%%%%%%%%%%%0.2.5%%%%%%%%%%%%%%%%%%%%%%%%%%%%%%%%%%%%%%
   \item[0.2.5]   Determine the value $\varphi(n)$ for each integer $n \le 30$
                  where $\varphi$ denotes the Euler $\varphi-$function.  

      \textbf{Solution.}

      We shall be making use of the multiplicative property of the Euler
		$\varphi-$function. So 

      \begin{center}
         \begin{tabular}{@{}l c r c l c l c r@{}}
            $\varphi(1)$ & = & 1, & & $\varphi(2)$ & = & 1, & & \\
            $\varphi(3)$ & = & 2, & & $\varphi(4)$ & = &
            $\varphi(2^2)$ & = & 2, \\
            $\varphi(5)$ & = & 4, & & $\varphi(6)$ & = &
            $\varphi(2)\varphi(3)$ & = & 2, \\
            $\varphi(7)$ & = & 6, & & $\varphi(8)$ & = &
            $\varphi(2^3)$ & = & 4, \\
            $\varphi(9) = \varphi(3^2)$ & = & 6, & & $\varphi(10)$ & = &
            $\varphi(2)\varphi(5)$ & = & 4, \\
            $\varphi(11)$ & = & 10, & & $\varphi(12)$ & = &
            $\varphi(3)\varphi(4)$ & = & 4, \\
            $\varphi(13)$ & = & 12, & & $\varphi(14)$ & = &
            $\varphi(2)\varphi(7)$ & = & 6, \\
            $\varphi(15) = \varphi(3)\varphi(5)$ & = & 8, & &
            $\varphi(16)$ & = &  $\varphi(2^4)$ & = & 8, \\
            $\varphi(17)$ & = & 16, & & $\varphi(18)$ & = &
            $\varphi(2)\varphi(9)$ & = & 6, \\
            $\varphi(19)$ & = & 18, & & $\varphi(20)$ & = &
            $\varphi(4)\varphi(5)$ & = & 8, \\
            $\varphi(21) = \varphi(3)\varphi(7)$ & = & 12, & &
            $\varphi(22)$ & = & $\varphi(2)\varphi(11)$ & = & 10, \\
            $\varphi(23)$ & = & 22, & &
            $\varphi(24)$ & = & $\varphi(3)\varphi(8)$ & = & 8, \\
            $\varphi(25) = \varphi(5^2)$ & = & 20, & &
            $\varphi(26)$ & = & $\varphi(2)\varphi(13)$ & = & 12, \\
            $\varphi(27) = \varphi(3^3)$ & = & 18, & &
            $\varphi(28)$ & = & $\varphi(4)\varphi(7)$ & = & 12, \\
            $\varphi(29)$ & = & 28, & &
            $\varphi(30)$ & = & $\varphi(2)\varphi(15)$ & = & 8. \\
         \end{tabular}
      \end{center}
%%%%%%%%%%%%%%%%%%%%%%%%%%%%%%%%%%%%%0.2.6%%%%%%%%%%%%%%%%%%%%%%%%%%%%%%%%%%%%%%
   \item[0.2.6]   Prove the Well Ordering Principle of $\Z$ by induction and
                  prove the minimal element is unique.

      \textbf{Proof.} Let $P$ be a nonempty subset of $\Z^+$. We want to show 
      that $P$ has a minimal element. So suppose by way of contradiction that
      $P$ does not have a minimal element. For a natural number $n$, let $S(n)$ 
      be the statement that $n$ is not a member of $P$. We now want to show that
      by Strong Induction that $S(n)$ holds for every natural number $n$. If 1
      is in $P$, then it would be the smallest member of $P$, contradicting our
      assumption that $P$ has no minimal element, so $1 \notin P$; hence $S(1)$ 
      is true. Now suppose that $S(j)$ is true for every natural number $j < k$,
      where $k$ is a natural number greater than 1. By our supposition, we know
      that every integer less than $k$ is not in $P$, so if $k$ is in $P$, it
      would be the minimal element of $P$, contradicting our assumption that $P$
      has no minimal element. Thus $S(k)$ is true. It follows by Mathematical
      Induction that $S(n)$ holds for every positive integer $n$. That is, $P$
      is empty, a contradiction. We can now conclude that $P$ has a minimal 
      element, say $p$. To show that $p$ is unique assume that $q$ is also a
      minimal element of $P$. By virtue of $p$ as a minimal element of $P$, we 
      have $p \le q$ and, by virtue of $q$ as a minimal element of $P$, we have
      $q \le p$, so that $p = q$. Hence the minimal element of $P$ is
      unique. \qed
%%%%%%%%%%%%%%%%%%%%%%%%%%%%%%%%%%%%%0.2.7%%%%%%%%%%%%%%%%%%%%%%%%%%%%%%%%%%%%%%
   \item[0.2.7]   If $p$ is a prime prove that there do not exist nonzero
                  integers $a$ and $b$ such that $a^2 = pb^2$.

      \textbf{Proof.} Let $p$ be a prime number. Suppose by contradiction that 
      there exist nonzero integers $a$ and $b$ such that $a^2 = pb^2$. We can
      further suppose that $a$ and $b$ are relatively prime. Since $a^2 = pb^2$,
      it follows that $a^2$ has $p$ as one of its prime factors, so that $a$
      also has $p$ as one of its prime factors. We can then write $a = pm$
      for some integer $m$. Substituting $a = pm$ in the equation $a^2 = pb^2$,
      will give us the equation $pm^2 = b^2$. We can similarly conclude that
      $b$ has $p$ as one of its prime factors, so that $\gcd(a, b) \ge p$, a
      contradiction. Thus there do not exist nonzero integers $a$ and $b$ such 
      that $a^2 = pb^2$. \qed
%%%%%%%%%%%%%%%%%%%%%%%%%%%%%%%%%%%%%0.2.8%%%%%%%%%%%%%%%%%%%%%%%%%%%%%%%%%%%%%%
   \item[0.2.8]   Let $p$ be a prime, $n \in \Z^+$. Find a formula for the
                  largest power of $p$ which divides
                  $n! = n(n - 1)(n - 2)\cdots2 \cdot 1$ (it involves the
                  greatest integer function).

      \textbf{Proof.} Let $p$ be a prime and let $n$ be a positive integer. The
      largest power of $p$ that divides $n!$, say $k$, is simply the number of 
      multiples of $p$ in the set $\{1, 2, \ldots, n\}$. Thus
      $k = \lfloor{n/p}\rfloor$, where $\lfloor{x}\rfloor$ is the greatest
      integer less than the real number $x$.
%%%%%%%%%%%%%%%%%%%%%%%%%%%%%%%%%%%%%0.2.9%%%%%%%%%%%%%%%%%%%%%%%%%%%%%%%%%%%%%%
   \item[0.2.9]   Write a computer program to determine the greatest common
                  divisor $(a, b)$ of two integers $a$ and $b$ and to express
                  $(a, b)$ in the form $ax + by$ for some integers $x$ and $y$.

   \begin{verbatim}
# Python
# For positive integers a and b gcd(a, b) returns
# a tuple (r, x, y) where r = gcd(a, b) and xa + yb = r
def gcd(a, b, x1 = 1, y1 = 0, x2 = 0, y2 = 1):
   q = a // b
   r = a % b

   if r == 0:
      return (b, 0, 1)

   x1 = x1 - q * x2
   y1 = y1 - q * y2

   if b % r == 0:
      return (r, x1, y1)

   return gcd(b, r, x2, y2, x1, y1)
   \end{verbatim}
%%%%%%%%%%%%%%%%%%%%%%%%%%%%%%%%%%%%%0.2.10%%%%%%%%%%%%%%%%%%%%%%%%%%%%%%%%%%%%%
   \item[0.2.10]  Prove for any given positive integer $N$ there exist only
                  finitely many integers $n$ with $\varphi(n) = N$ where
                  $\varphi$ denotes Euler's $\varphi$-function. Conclude in 
                  particular that $\varphi(n)$ tends to infinity as $n$ tends to
                  infinity.

      \textbf{Proof.} Let $N \in \N$ and define
      $$S_N = \{n \in \N : \varphi(n) = N\}.$$
      
      We want to show that $S_N$ is finite. So it suffices to show that $S_N$ is
      bounded; i.e., there exists a positive integer $K$ such that $n < K$ for
      all $n \in S_N$. If $S_N$ is empty, then we are done, so assume that $S_N$
      is nonempty. Let $m \in S_N$. If $m = 1$, then $N = 1$, and thus
      $S_1 = \{1, 2\}$ is finite. So assume $m > 1$. By the Fundamental 
      Theorem of Arithmetic, it follows that
      $$m = {p_1}^{c_1}{p_2}^{c_2}\cdots{p_s}^{c_s},$$
      where the $p_i$s are mutually distinct primes, $s$ and the $c_i$s are
      positive integers, and $s < N$ (the number of distinct prime factors a
      positive integer has is less than the integer). Applying $\varphi$ to $m$ 
      gives us 
      \begin{align}
         \varphi(m) &= \varphi({p_1}^{c_1}{p_2}^{c_2}
            \cdots{p_s}^{c_s}) \nonumber \\
            &= \varphi({p_1}^{c_1})\varphi({p_2}^{c_2})\cdots
               \varphi({p_s}^{c_s})  \nonumber \\
            &= (p_1 - 1)(p_2 - 1)\cdots(p_s-1){p_1}^{c_1-1}{p_2}^{c_2-1}
               \cdots{p_s}^{c_s-1} \label{0_2_1} \\
            &= N. \nonumber
      \end{align}

      Let $q$ be the least prime greater than $N + 1$. Thus each $p_i - 1 < q$,
      for otherwise $\varphi(m) > N$ by $\eqref{0_2_1}$. Similarly, since
      ${p_i}^{N} > N$, it must be the case that each $c_i < N + 1$. So
      $$m = {p_1}^{c_1}{p_2}^{c_2}\cdots{p_s}^{c_s} <
        \underbrace{q^{N+1}q^{N+1}\cdots q^{N+1}}_{N \text{ times}} 
        = q^{(N+1)N}.$$
      Thus
      $$S_N \subset \{1, 2, \ldots, q^{(N+1)N}\},$$ 
      so that $S_N$ is finite. Let $\varepsilon$ be a positive number. To show 
      that $\varphi(n)$ tends to infinity, we must find a natural number $K$
      such that $\varphi(n) > \varepsilon$ for all $n > K$. Define
      $M_j := \max(S_j)$ for all $j \in \N$. Choose
      $K = \max\{M_1, M_2, \ldots, M_{\lceil\varepsilon \rceil}\}$,
      ($\lceil\varepsilon\rceil$ is the least integer greater than
      $\varepsilon$). Consider $n > K$ and assume to the contrary that
      $\varphi(n) = l \in \{1, 2, \ldots, \lceil\varepsilon\rceil\}$; that is,
      $n \in S_l$, so that $M_l \ge n$. By definition, $K \ge M_l$, and thus,
      $K \ge n$, a contradiction. Hence $\varphi(n) \notin \{1, 2, \ldots, 
      \lceil\varepsilon\rceil\}$; that is,
      $\varphi(n)  > \lceil\varepsilon\rceil \ge \varepsilon$. \qed
%%%%%%%%%%%%%%%%%%%%%%%%%%%%%%%%%%%%%0.2.11%%%%%%%%%%%%%%%%%%%%%%%%%%%%%%%%%%%%%
   \item[0.2.11]  Prove that if $d$ divides $n$ then $\varphi(d)$ divides
                  $\varphi(n)$ where $\varphi$ denotes Euler's
                  $\varphi-$function.

      \textbf{Proof.} Let $d$ and $n$ be positive integers such that $d \mid n$.
      We want to show that $\varphi(d) \mid \varphi(n)$. Let
      ${d_1}^{a_1}{d_2}^{a_2}\cdots{d_k}^{a_k}$ be the prime factorization of
      $d$, where each $a_i$ is a positive integer and each $d_i$ is a unique
      prime. Since $d \mid n$, it follows that there exists an integer $m$ such
      that
      $$n = ({d_1}^{a_1}{d_2}^{a_2}\cdots{d_k}^{a_k})m.$$
      Now we shall factor out the maximum powers of each $d_i$ in $m$, so that 
      we can write
      $$m = ({d_1}^{c_1}{d_2}^{c_2}\cdots{d_k}^{c_k})m'$$
      where each $c_i$ is a nonnegative integer and $m'$ is an integer that is
      prime to ${d_1}^{c_1}{d_2}^{c_2}\cdots{d_k}^{c_k}$. Thus we have that
      $$n = ({d_1}^{b_1}{d_2}^{b_2}\cdots{d_k}^{b_k})m', \quad b_i = a_i + c_i$$
      so that
      \begin{align*}
         \varphi(n) &= \varphi({d_1}^{b_1}{d_2}^{b_2}\cdots{d_k}^{b_k}m') \\
                    &= \varphi({d_1}^{b_1}{d_2}^{b_2}\cdots{d_k}^{b_k})
                       \varphi(m') \\
                    &= {d_1}^{b_1 - 1}(d_1 - 1){d_2}^{b_2 - 1}(d_2 - 1)\cdots
                       {d_k}^{b_k - 1}(d_k - 1)\varphi(m') \\
                    &= {d_1}^{c_1}{d_1}^{a_1 - 1}(d_1 - 1)
                       {d_2}^{c_2}{d_2}^{a_2 - 1}(d_2 - 1)\cdots
                       {d_k}^{c_k}{d_k}^{a_k - 1}(d_k - 1)\varphi(m')\\
                    &= {d_1}^{c_1}{d_2}^{c_2}\cdots{d_k}^{c_k}\varphi(m')
                       {d_1}^{a_1 - 1}(d_1 - 1)
                       {d_2}^{a_2 - 1}(d_2 - 1)\cdots
                       {d_k}^{a_k - 1}(d_k - 1) \\
                    &= ({d_1}^{c_1}{d_2}^{c_2}\cdots{d_k}^{c_k}
                        \varphi(m'))\varphi(d),
      \end{align*}
      so that $\varphi(d) \mid \varphi(n)$. \qed
      
\end{enumerate}

      \section{Cyclic Groups And Cyclic Subgroups}
         \begin{enumerate}
%%%%%%%%%%%%%%%%%%%%%%%%%%%%%%%%%%%%%2.3.1%%%%%%%%%%%%%%%%%%%%%%%%%%%%%%%%%%%%%%
   \item[2.3.1]   Find all subgroups of $Z_{45} = \cyc{x}$, giving a generator
                  for each. Describe the containments between these subgroups.
                  
      \textbf{Solution.} Since the positive divisors of 45 are: 1, 3, 5, 9, 15,
      and 45, it follows that the subgroups of $Z_{45}$ are
      $$\cyc{x}, \cyc{x^3}, \cyc{x^5}, \cyc{x^9}, \cyc{x^{15}}, \text{ and }
        \cyc{x^{45}}.$$
        
      We have the following containments:
      $$
         \begin{tabular}{>{$}c<{$}>{$}c<{$}>{$}c<{$}>{$}c<{$}>{$}c<{$}>{$}c<{$}>{$}c<{$}}
            \cyc{x^{45}} & \le & \cyc{x^{15}} & \le & \cyc{x^5} & \le & \cyc{x} \\
            \cyc{x^{15}} & \le &  \cyc{x^3} & \le & \cyc{x} \\
            \cyc{x^9} & \le &  \cyc{x^3} & \le & \cyc{x}
         \end{tabular}
      $$
%%%%%%%%%%%%%%%%%%%%%%%%%%%%%%%%%%%%%2.3.2%%%%%%%%%%%%%%%%%%%%%%%%%%%%%%%%%%%%%%
   \item[2.3.2]   If $x$ is an element of the finite group $G$ and $|x| = |G|$,
                  prove that $G = \cyc{x}$. Give an explicit example to show 
                  that this result need not be true if $G$ is an infinite group.
                  
      \textbf{Proof.} Let $G$ be a finite group, so that $|G| = n \in \Z^+$.
      Suppose that there exists $x \in G$ such that $|x| = n$. Clearly
      $\cyc{x} \subseteq G$. But $|\cyc{x}| = n$ since $|x| = n$; thus
      $G \subseteq \cyc{x}$ so that $G = \cyc{x}$. Now let $G = \Z$. We have
      that $|\cyc{2}| = |G|$ but $G \neq \cyc{2}$. \qed
%%%%%%%%%%%%%%%%%%%%%%%%%%%%%%%%%%%%%2.3.3%%%%%%%%%%%%%%%%%%%%%%%%%%%%%%%%%%%%%%
   \item[2.3.3]   Find all generators for $\Z/48\Z$.
   
      \textbf{Solution.} The generators for $\Z/48\Z$ are: $\cyc{\overline{1}}$,
      $\cyc{\overline{5}}$, $\cyc{\overline{7}}$, $\cyc{\overline{11}}$,
      $\cyc{\overline{13}}$, $\cyc{\overline{17}}$, $\cyc{\overline{19}}$,
      $\cyc{\overline{23}}$, $\cyc{\overline{25}}$, $\cyc{\overline{29}}$,
      $\cyc{\overline{31}}$, $\cyc{\overline{35}}$, $\cyc{\overline{37}}$,
      $\cyc{\overline{41}}$, $\cyc{\overline{43}}$, and $\cyc{\overline{47}}$.
%%%%%%%%%%%%%%%%%%%%%%%%%%%%%%%%%%%%%2.3.4%%%%%%%%%%%%%%%%%%%%%%%%%%%%%%%%%%%%%%
   \item[2.3.4]   Find all generators for $\Z/202\Z$.
   
      \textbf{Solution.} Let $S$ be the set of generators for $\Z/202\Z$. Then
      $|S| = 100$ since
      $$S = \{\cyc{x} : x \text{ is odd and positive}, x \neq 101, \text{ and } x < 202\}.$$
%%%%%%%%%%%%%%%%%%%%%%%%%%%%%%%%%%%%%2.3.5%%%%%%%%%%%%%%%%%%%%%%%%%%%%%%%%%%%%%%
   \item[2.3.5]   Find the number of generators for $\Z/49000\Z$.
   
      \textbf{Solution.} For a positive integer $n$ let $\varphi(n)$ be the
      number of positive integers---less than or equal to $n$---that are
      relatively prime to $n$. Then the number of generators for $\Z/49000\Z$ is
      $\varphi(49000) = \varphi(2^35^37^2) =
      \varphi(2^3)\varphi(5^3)\varphi(7^2) = 16800$. 
%%%%%%%%%%%%%%%%%%%%%%%%%%%%%%%%%%%%%2.3.6%%%%%%%%%%%%%%%%%%%%%%%%%%%%%%%%%%%%%%
   \item[2.3.6]   In $\Z/48\Z$ write out all elements of $\cyc{\overline{a}}$
                  for every $\overline{a}$. Find all inclusions between
                  subgroups in $\Z/48\Z$.
      
      \textbf{Solution.}
      $$
         \begin{tabular}{|c|c|} \hline
            \textbf{Generators} & \textbf{Subgroups in} $\Z/48\Z$ \\ \hline
            0 & $\{0\}$ \\ \hline
            24 & $\{0, 24\}$ \\ \hline
            16, 32 & $\{0, 16, 32\}$ \\ \hline
            12, 36 & $\{0, 12, 24, 36\}$ \\ \hline
            8, 40 & $\{0, 8, 16, 24, 32, 40\}$ \\ \hline
            6, 18, 30, 42 & $\{0, 6, 12, 18, 24, 30, 36, 42\}$ \\ \hline
            4,20,28,44 & $\{0,4,8,12,16, 20, 24, 28, 32, 36, 40, 44\}$ \\ \hline
            3, 9, 15, 21, 27, 33, 39, 45 & $\{0, 3, 6, 9, 12, 15, 18, 21, 24,
            27, 30, 33, 36, 39, 42, 45\}$ \\ \hline            
            2, 10, 14, 22, 26, 34, 38, 46 & $\{x : 0 \le x \le 46,
            x \text{ is even}\}$ \\ \hline
            \text{See Exercise } 2.3.3 & $\Z/48\Z$ \\ \hline
         \end{tabular}
      $$
%%%%%%%%%%%%%%%%%%%%%%%%%%%%%%%%%%%%%2.3.7%%%%%%%%%%%%%%%%%%%%%%%%%%%%%%%%%%%%%%
   \item[2.3.7]   Let $Z_{48} = \cyc{x}$ and use the isomorphism
                  $\Z/48\Z \cong Z_{48}$ given by $\overline{1} \mapsto x$ to
                  list all subgroups of $Z_{48}$ as computed in the preceding
                  exercise.
                  
      \textbf{Solution.}
      $$
         \begin{tabular}{|c|} \hline
            \textbf{Subgroups in} $Z_{48}$ \\ \hline
            $\{1\}$ \\ \hline
            $\{1, x^{24}\}$ \\ \hline
            $\{1, x^{16}, x^{32}\}$ \\ \hline
            $\{1, x^{12}, x^{24}, x^{36}\}$ \\ \hline
            $\{1, x^8, x^{16}, x^{24}, x^{32}, x^{40}\}$ \\ \hline
            $\{1, x^6, x^{12}, x^{18}, x^{24}, x^{30},x^{36},x^{42}\}$ \\ \hline
            $\{1,x^4,x^8,x^{12},x^{16}, x^{20}, x^{24}, x^{28}, x^{32}, x^{36},
               x^{40}, x^{44}\}$ \\ \hline
            $\{1, x^3, x^6, x^9, x^{12}, x^{15}, x^{18}, x^{21}, x^{24},
            x^{27}, x^{30}, x^{33}, x^{36}, x^{39}, x^{42}, x^{45}\}$ \\ \hline
            $\{x^y : 0 \le y \le 46, y \text{ is even}\}$ \\ \hline
            $Z_{48}$ \\ \hline
         \end{tabular}
      $$
%%%%%%%%%%%%%%%%%%%%%%%%%%%%%%%%%%%%%2.3.8%%%%%%%%%%%%%%%%%%%%%%%%%%%%%%%%%%%%%%
   \item[2.3.8]   Let $Z_{48} = \cyc{x}$. For which integers $a$ does the map
                  $\varphi_a$ defined by $\varphi_a : \overline{1} \mapsto x^a$
                  extend to an \textit{isomorphism} from $\Z/48\Z$ onto
                  $Z_{48}$.
                  
      \textbf{Solution.} Suppose that $(a, 48) = 1$. Then it follows that $x^a$
      generates $Z_{48}$. Thus $\varphi_a$ is an isomorphism by Theorem 4 (Page
      56). Now suppose that $a$ is not relatively prime to 48. Then $x^a$ does
      not generate $Z_{48}$, so that the image of $\varphi_a$ is not $Z_{48}$.
      Hence $\varphi_a$ is an isomorphism if and only if $(a, 48) = 1$.
%%%%%%%%%%%%%%%%%%%%%%%%%%%%%%%%%%%%%2.3.9%%%%%%%%%%%%%%%%%%%%%%%%%%%%%%%%%%%%%%
   \item[2.3.9]   Let $Z_{36} = \cyc{x}$. For which integers $a$ does the map
                  $\psi_a$ defined by $\psi_a : \overline{1} \mapsto x^a$ extend
                  to a \textit{well defined homomorphism} from $\Z/48\Z$ into
                  $Z_{36}$. Can $\psi_a$ ever be a surjective homomorphism?
                  
      \textbf{Solution.} First we shall find the restriction(s) on $a$ such that
      $\psi_a$ is well defined. Suppose $b = c$ for some $b, c \in \Z/48\Z$. It
      suffices to show that $\psi_a(b) = \psi_a(c)$. Since $b = c$, there exists
      an integer $k$ such that $b = c + 48k$. Thus $\psi_a(b) = \psi_a(c+48k)$,
      so that
      $\psi_a(b)=(x^a)^{c+48k}=x^{ac + 48ak}= x^{ac}x^{48ak}=\psi_a(c)x^{12ak}$.
      So we must require $x^{12ak} = 1$ for all $k \in \Z$. Now $x^{12ak} = 1$
      for all $k \in \Z$ if and only if $3 \mid a$ if and only if $\psi_a$ is
      well defined. It follows immediately that
      $\psi_a$ is an homomorphism since
      \begin{align*}
         \psi_a(p + q) &= (x^a)^{p+q} \\
            &= x^{ap+aq} \\
            &= x^{ap}x^{aq} \\
            &= (x^a)^p(x^a)^q \\
            &= \psi_a(p)\psi_a(q)
      \end{align*}      
      for all $p, q \in \Z/48\Z$.
      
      \textit{Can $\psi_a$ ever be a surjective homomorphism?} No!
      
      \textbf{Proof.} Suppose to the contrary that $\psi_a$ is surjective. Then
      there exists $y \in \Z/48\Z$ such that $\psi_a(y) = x$. That is
      $x^{ay} = x$, so that $x^{ay-1} = 1$; thus $ay - 1 = 36m$ for some integer
      $m$. Rearrange the equality $ay - 1 = 36m$ to get $1 = ay - 36m$. Recall
      that $3 \mid a$; since $3$ also divides 36, it follows that 3 must divide
      1, a contradiction. Thus $\psi_a$ can never be surjective. \qed
%%%%%%%%%%%%%%%%%%%%%%%%%%%%%%%%%%%%%2.3.10%%%%%%%%%%%%%%%%%%%%%%%%%%%%%%%%%%%%%
   \item[2.3.10]  What is the order of $\overline{30}$ in $\Z/54\Z$? Write out
                  all the elements and their orders in $\cyc{\overline{30}}$.
                  
      \textbf{Solution.} The order of $30$ in $\Z/54\Z$ is
      $$\frac{54}{(30, 54)} = 9.$$
      The elements of $\cyc{30}$ and their respective orders are:
      $$
         \begin{tabular}{|c|c|} \hline
            Element of $\cyc{30}$ & Order \\ \hline
            30 & 9 \\ \hline
             6 & 9 \\ \hline
            36 & 3 \\ \hline
            12 & 9 \\ \hline
            42 & 9 \\ \hline
            18 & 3 \\ \hline
            48 & 9 \\ \hline
            24 & 9 \\ \hline
             0 & 1 \\ \hline
         \end{tabular}
      $$
%%%%%%%%%%%%%%%%%%%%%%%%%%%%%%%%%%%%%2.3.11%%%%%%%%%%%%%%%%%%%%%%%%%%%%%%%%%%%%%
   \item[2.3.11]  Find all cyclic subgroups of $D_8$. Find a proper subgroup of
                  $D_8$ which is not cyclic.
                  
      \textbf{Solution.} In $D_8$, only $r$ and $r^4$ have order 4. Thus
      $\{1, r, r^2, r^3\}$ is the only cyclic subgroup of order 4. The trivial
      subgroup is the only cyclic subgroup of order 1. Finally there are 5
      cyclic subgroups of order 2 and they are of the form $\{1, x\}$ where
      $x \in \{r^2, s, sr, sr^2, sr^3\}$. The set $\{1, s, r^2, sr^2\}$ is a
      non-cyclic proper subgroup of $D_8$.
%%%%%%%%%%%%%%%%%%%%%%%%%%%%%%%%%%%%%2.3.12%%%%%%%%%%%%%%%%%%%%%%%%%%%%%%%%%%%%%
   \item[2.3.12]  Prove that the following groups are \textit{not} cyclic:
                  \begin{enumerate}
                     \item $Z_2 \times Z_2$
                     \item $Z_2 \times \Z$
                     \item $\Z \times \Z$.
                  \end{enumerate}
      
      \textbf{Proof.}
      \begin{enumerate}
         \item The order of $Z_2 \times Z_2$ is 4, but no element in this group
               has order 4; thus $Z_2 \times Z_2$ is not cyclic.
         \item Let $Z_2 = \cyc{x}$. Observe that $Z_2 \times \Z$ is not finite,
               so in order for it to be cyclic it must be isomorphic to $\Z$.
               But this is not the case since $Z_2 \times \Z$ has two elements
               of finite order(namely $(1, 0)$ and $(x, 0)$) while $\Z$ has
               exactly 1 element of finite order.
         \item Suppose to the contrary that $\Z \times \Z$ is cyclic. Then there
               exist nonzero integers $a$ and $b$ such that
               $$\Z \times \Z = \cyc{(a,b)} = \{(na, nb) : n \in \Z\}.$$
               Thus there exists an integer $m$ such that
               $(ma, mb) = (0, 1)$. That is, $ma = 0$ and $mb = 1$. Since
               $ma = 0$, we must have $m = 0$ or $a = 0$. If $m$ is 0, then
               $(ma, mb) = (0, 0) \neq (0, 1)$, a contradiction; thus we must
               have $a = 0$, contradicting our assumption that $a$ is nonzero.
               Thus $\Z \times \Z$ is not cyclic.
      \end{enumerate} \qed
%%%%%%%%%%%%%%%%%%%%%%%%%%%%%%%%%%%%%2.3.13%%%%%%%%%%%%%%%%%%%%%%%%%%%%%%%%%%%%%
   \item[2.3.13]  Prove that the following pairs of groups are \textit{not}
                  isomorphic:
                  \begin{enumerate}
                     \item $\Z \times Z_2$ and $\Z$
                     \item $\Q \times Z_2$ and $\Q$.
                  \end{enumerate}
      
      \textbf{Proof.}
      \begin{enumerate}
         \item By Exercise 1.6.11, we know that $\Z \times Z_2$ is isomorphic to
               $Z_2 \times \Z$. By Exercise 2.3.12, $Z_2 \times \Z$ is not
               cyclic; thus $\Z \times Z_2$ is not cyclic. That is,
               $\Z \times Z_2$ is not isomorphic to $\Z$.
         \item Let $Z_2 = \cyc{x}$. It immediately follows that
               $\Q \times Z_2$ and $\Q$ are not isomorphic since $\Q \times Z_2$
               has two elements of finite order(namely $(0, 1)$ and $(0, x)$)
               while $\Q$ has exactly 1 element of finite order.
      \end{enumerate} \qed
%%%%%%%%%%%%%%%%%%%%%%%%%%%%%%%%%%%%%2.3.14%%%%%%%%%%%%%%%%%%%%%%%%%%%%%%%%%%%%%
   \item[2.3.14]  Let $\sigma =$ (1 2 3 4 5 6 7 8 9 10 11 12). For each of the
                  following integers $a$ compute $\sigma^a$:
                  $$a = 13, 65, 626, 1195, -6, -81, -570,\text{ and } {-1211}.$$
                  
      \textbf{Solution.}
      
      \begin{alignat*}{4}
         &\sigma^{13}   &&= \sigma &&\text{ } \\
         &\sigma^{65}   &&= \sigma^5 &&=
            (1\;6\;11\;4\;9\;2\;7\;12\;5\;10\;3\;8) \\
         &\sigma^{626}  &&= \sigma^2 &&= (1\;3\;5\;7\;9\;11) \\
         &\sigma^{1195} &&= \sigma^7 &&=
            (1\;8\;3\;10\;5\;12\;7\;2\;9\;4\;11\;6\;13) \\
         &\sigma^{-6} &&= \sigma^6 &&= (1\;7)
            (1\;8\;3\;10\;5\;12\;7\;2\;9\;4\;11\;6\;13) \\
         &\sigma^{-81} &&= \sigma^3 &&= (1\;4\;7\;10) \\
         &\sigma^{-570} &&= \sigma^6 &&= (1\;7) \\
         &\sigma^{-1211} &&= \sigma
      \end{alignat*}
%%%%%%%%%%%%%%%%%%%%%%%%%%%%%%%%%%%%%2.3.15%%%%%%%%%%%%%%%%%%%%%%%%%%%%%%%%%%%%%
   \item[2.3.15]  Prove that $\Q \times \Q$ is not cyclic.
   
      \textbf{Proof.} Since $\Q$ is infinite and, by Exercise 1.6.6, $\Q$ is not
      isomorphic to $\Z$, it follows that $\Q$ is not cyclic. We know that the
      subgroup of every cyclic group is cyclic; since $\Q \times\{1\} \cong \Q$,
      it follows that $\Q \times \{1\}$ is not cyclic; thus $\Q \times \Q$ is
      not cyclic because it has a noncyclic subgroup, namely $\Q \times \{1\}$.
      \qed
%%%%%%%%%%%%%%%%%%%%%%%%%%%%%%%%%%%%%2.3.16%%%%%%%%%%%%%%%%%%%%%%%%%%%%%%%%%%%%%
   \item[2.3.16]  Assume $|x| = n$ and $|y| = m$. Suppose that $x$ and $y$
                  \textit{commute}: $xy = yx$. Prove that $|xy|$ divides the
                  least common multiple of $m$ and $n$. Need this be true if $x$
                  and $y$ do \textit{not} commute? Give an example of commuting
                  elements $x$, $y$ such that the order of $xy$ is not equal to
                  the least common multiple of $|x|$ and $|y|$.
                  
      \textbf{Proof.} Let $l = \text{lcm}(m, n)$. So there exist integers
      $m'$ and $n'$ such that $mm' = nn' = l$. So we have that
      $$(xy)^l = x^ly^l = x^{nn'}y^{mm'} = (x^n)^{n'}(y^m)^{m'} = 1.$$
      That is $|xy|$ divides $l$ (by Proposition 3, Page 55).
      
      \textit{Need this be true if $x$ and $y$ do not commute?} No! Let
      $$
         A = \left(\begin{tabular}{@{}cc@{}}
            0 & 1/2 \\
            2 & 0
         \end{tabular}\right) \text{ and }
         B = \left(\begin{tabular}{@{}cc@{}}
            0 & 1 \\
            1 & 0
         \end{tabular}\right).
      $$
      A simple computation will show us that although $|A| = |B| = 2$, we have
      that $|AB| = \infty$.
      
      \textbf{Example.} Consider $\Z/2\Z = \{0, 1\}$. Let $x = y = 1$. Then we
      have $|x| = |y| = 2$, so that lcm($|x|, |y|) = 2 \neq |x + y| = |0| = 1$.
      \qed
%%%%%%%%%%%%%%%%%%%%%%%%%%%%%%%%%%%%%2.3.17%%%%%%%%%%%%%%%%%%%%%%%%%%%%%%%%%%%%%
   \item[2.3.17]  Find a presentation for $Z_n$ with one generator.
   
      \textbf{Solution.} $Z_n = \cyc{x : x^n = 1}$.
%%%%%%%%%%%%%%%%%%%%%%%%%%%%%%%%%%%%%2.3.18%%%%%%%%%%%%%%%%%%%%%%%%%%%%%%%%%%%%%
   \item[2.3.18]  Show that if $H$ is any group and $h$ is an element of $H$
                  with $h^n = 1$, then there is a unique homomorphism from
                  $Z_n = \cyc{x}$ to $H$ such that $x \mapsto h$.
                  
      \textbf{Proof.} Let $n \in \Z^+$, $Z_n = \cyc{x}$, $H$ a group, and
      $h^n  = 1$ for some $h \in H$. First we shall show the existence of a
      homomorphism from $Z_n$ to $H$ such that $x \mapsto h$. So consider the
      map $\alpha : \cyc{x} \rightarrow H$ defined by $\alpha(x^a) = h^a$.
      Clearly $\alpha(x) = h$. Now we will show that $\alpha$ is well defined.
      Suppose $x^w = x^y$ for some $x^w, x^y \in Z_n$. Thus $w = y + nk$ for
      some integer $k$. Thus
      $$\alpha(x^w) = \alpha(x^{y+nk})=h^{y+nk}=h^{y}{h^n}^k =h^y=\alpha(x^y),$$
      so that $\alpha$ is well defined. Now we have that
      $$\alpha(x^px^q)=\alpha(x^{p+q})=h^{p+q}=h^ph^q=\alpha(x^p)\alpha(x^q),$$
      so that $\alpha$ is an homomorphism. Now to show uniqueness, we suppose
      that $\phi : \cyc{x} \rightarrow H$ is an homommorphism such that
      $\phi(x) = h$. Since $\phi$ is a homomorphism, it follows that
      $\phi(x^a) = h^a$. Thus $\phi = \alpha$, as desired. \qed
%%%%%%%%%%%%%%%%%%%%%%%%%%%%%%%%%%%%%2.3.19%%%%%%%%%%%%%%%%%%%%%%%%%%%%%%%%%%%%%
   \item[2.3.19]  Show that if $H$ is any group and $h$ is an element of $H$,
                  then there is a unique homomorphism from $\Z$ to $H$ such that
                  $1 \mapsto h$.
                  
      \textbf{Proof.} Let $H$ be a group and let $h \in H$. First we shall show
      that there exists a homomorphism from $\Z$ to $H$ such that $1 \mapsto h$.
      So consider the map $\alpha : \Z \rightarrow H$ defined by
      $n \mapsto h^n$. Clearly $\alpha(1) = h$ and
      $$\alpha(x+y) = h^{x+y} = h^xh^y = \alpha(x)\alpha(y) \text{ for all }
        x, y \in \Z^+,$$
      so that $\alpha$ is a homomorphism. To show uniqueness, suppose that
      $\alpha' : \Z \rightarrow H$ is an homomorphism such that
      $\alpha'(1) = h$. Then according to Exercise 1.6.1, we have that
      $\alpha'(n) = \alpha'(n\cdot1) = \alpha'(1)^n = h^n$ for all $n \in \Z$;
      that is, $\alpha' = \alpha$, as desired. \qed
%%%%%%%%%%%%%%%%%%%%%%%%%%%%%%%%%%%%%2.3.20%%%%%%%%%%%%%%%%%%%%%%%%%%%%%%%%%%%%%
   \item[2.3.20]  Let $p$ be a prime and let $n$ be a positive integer. Show
                  that if $x$ is an element of the group $G$ such that
                  $x^{p^n} = 1$ then $|x| = p^m$ for some $m \le n$.
                  
      \textbf{Proof.} Suppose that $x \in G$ such that $x^{p^n} = 1$. Then it
      follows by Proposition 3 (Page 55) that $|x|$ divides $p^n$. Since $p$ is
      a prime, its factors are $p^i$, $0 \le i \le n$. Thus $|x| = p^m$ for
      some nonnegative $m$ not greater than $n$. \qed
%%%%%%%%%%%%%%%%%%%%%%%%%%%%%%%%%%%%%2.3.21%%%%%%%%%%%%%%%%%%%%%%%%%%%%%%%%%%%%%
   \item[2.3.21]  Let $p$ be an odd prime and let $n$ be a positive integer
                  $\ge 2$. Use the Binomial Theorem to show that
                  $(1+p)^{p^{n-1}} \equiv 1$ (mod $p^n$) but
                  $(1+p)^{p^{n-2}} \not\equiv 1$ (mod $p^n$). Deduce that $1+p$
                  is an element of order $p^{n-1}$ in the multiplicative group
                  $(\Z/p^n\Z)^\times$.

      \textbf{Lemma 2.3.1.} \textit{For an integer $n \ge 2$ and an odd prime
      $p$, let $f_p(n)$ be the number of $p$ factors of $n!$ (i.e., the greatest
      nonnegative integer $j$ such that $p^j \mid i!$), then it follows that
      $f_p(n) < \D\frac{n}{2}$}.

      \textbf{Proof.} Let $n \ge 2$ be an integer and $p$ an odd prime. For a
      a positive integer $r$, let $g_p(n, r)$ be the number of positive
      integers, less than or equal to $n$, that have at least $r$ number of $p$ 
      factors. It follows that $g_p(n, r) = \D\gint{\frac{n}{p^r}}$, where
      $\gint{x}$ is the greatest integer less than or equal to $x$. Finally let
      $k_n$ be the maximum nonnegative integer such that $p^{k_n}$ is a multiple
      of some positive integer not greater than $n$. Thus we have that
      \begin{align*}
         f_p(n) &= g_p(n, 1) + g_p(n, 2) + \cdots + g_p(n, k_n) \\
            &= \sum_{i=1}^{k_n} g_p(n, i)
            = \sum_{i=1}^{k_n} \gint{\frac{n}{p^i}} \\
            &\le \sum_{i=1}^{k_n} \frac{n}{p^i}
            < \sum_{i=1}^\infty \frac{n}{p^i} \\
            &= \frac{n}{p-1} &[\text{Sum of Geometric Series}] \\
            &< \frac{n}{2}. &[\text{Since }p \ge 3]
      \end{align*}

      So we can write $n! = p^{f_p(n)} h_n$ for some $h_n \in \Z^+$, so that
      $(h_n, p) = 1$.

      Now we are ready to commence the proof of the problem. By the Binomial
      Theorem, we have that
      \begin{align*}
         (1+p)^{p^{n-1}} &= \sum_{i=0}^{p^{n-1}}\binom{p^{n-1}}{i}p^i \\
            &= \sum_{i=0}^{p^{n-1}}p^i\frac{p^{n-1}(p^{n-1}-1)(p^{n-1}-2)
               \cdots(p^{n-1}-i+1)}{i!} \\
            &= \sum_{i=0}^{p^{n-1}}p^i\frac{p^{n-1}(p^{n-1}-1)(p^{n-1}-2)
               \cdots(p^{n-1}-i+1)}{p^{f_p(i)} h_i} \\
            &= 1 + p^n + p^n\sum_{i=2}^{p^{n-1}}\frac{p^{i-1}(p^{n-1}-1)
               (p^{n-1}-2) \cdots(p^{n-1}-i+1)}{p^{f_p(i)} h_i}.
      \end{align*}
      Now $f_p(i) < i / 2 \le i - 1$ for $i \ge 2$. Thus $i - 1 - f_p(i) \ge 0$
      (so that $p^{i - 1 - f_p(i)}$ is an integer) if $i \ge 2$. We then have
      \begin{equation} \label{2_3_21_1}
         (1+p)^{p^{n-1}} = 1 + p^n + p^n\sum_{i=2}^{p^{n-1}}\frac{p^{i-1-f_p(i)}
        (p^{n-1}-1)(p^{n-1}-2) \cdots(p^{n-1}-i+1)}{h_i}
      \end{equation}
      Since $(h_i, p) = 1$, it follows that $h_i$ must divide
      $p^{i-1}(p^{n-1}-1)(p^{n-1}-2) \cdots(p^{n-1}-i+1)$. Hence
      $$\sum_{i=2}^{p^{n-1}}\frac{p^{i-1-f_p(i)}
        (p^{n-1}-1)(p^{n-1}-2) \cdots(p^{n-1}-i+1)}{h_i}$$
      is an integer and we can conclude from \eqref{2_3_21_1} that
      $(1+p)^{p^{n-1}} \equiv 1$ (mod $p^n$). Now we have that
      \begin{align*}
         (1+p)^{p^{n-2}} &= \sum_{i=0}^{p^{n-2}}\binom{p^{n-2}}{i}p^i \\
            &= \sum_{i=0}^{p^{n-2}}p^i\frac{p^{n-2}(p^{n-2}-1)(p^{n-2}-2)
               \cdots(p^{n-2}-i+1)}{i!} \\
            &= 1 + p^{n-1} + p^n\frac{p^{n-2}-1}{2} + p^n\frac{p(p^{n-2}-1)(p^{n-2}-2)}{3!} +\sum_{i=4}^{p^{n-1}}p^i\frac{p^{n-2}(p^{n-2}-1)(p^{n-2}-2)
               \cdots(p^{n-2}-i+1)}{p^{f_p(i)} h_i} \\
            &= 1 + p^n + p^n\sum_{i=2}^{p^{n-1}}\frac{p^{i-1}(p^{n-1}-1)
               (p^{n-1}-2) \cdots(p^{n-1}-i+1)}{p^{f_p(i)} h_i}.
      \end{align*}
      
%%%%%%%%%%%%%%%%%%%%%%%%%%%%%%%%%%%%%2.3.22%%%%%%%%%%%%%%%%%%%%%%%%%%%%%%%%%%%%%
   \item[2.3.22]  Let $n$ be an integer $\ge 3$. Use the Binomial Theorem to
                  show that $(1+2^2)^{2^{n-2}} \equiv 1$ (mod $2^n$) but
                  $(1+2^2)^{2^{n-3}} \not\equiv 1$ (mod $2^n$). Deduce that 5 is
                  an element of order $2^{n-2}$ in the multiplicative group
                  $(\Z/2^n\Z)^\times$.

      \textbf{Proof.}
%%%%%%%%%%%%%%%%%%%%%%%%%%%%%%%%%%%%%2.3.23%%%%%%%%%%%%%%%%%%%%%%%%%%%%%%%%%%%%%
   \item[2.3.23]  Show that $(\Z/2^n\Z)^\times$ is not cyclic for any $n \ge 3$.
                  [Find two distinct subgroups of order 2.]
%%%%%%%%%%%%%%%%%%%%%%%%%%%%%%%%%%%%%2.3.24%%%%%%%%%%%%%%%%%%%%%%%%%%%%%%%%%%%%%
   \item[2.3.24]  Let $G$ be a finite group and let $x \in G$.
                  \begin{enumerate}
                     \item Prove that if $g \in N_G(\cyc{x})$ then
                           $gxg^{-1} = x^a$ for some $a \in \Z$. 
                     \item Prove conversely that if $gxg^{-1} = x^a$ for some
                           $a \in \Z$ then $g \in N_G(\cyc{x})$. [Show first
                           that $gx^kg^{-1} = (gxg^{-1})^k = x^{ak}$ for any
                           integer $k$, so that $g\cyc{x}g^{-1} \le \cyc{x}$.
                           If $x$ has order $n$, show the elements $gx^ig^{-1}$,
                           $i = 0, 1, \ldots, n-1$ are distinct, so that
                           $|g\cyc{x}g^{-1}| = |\cyc{x}| = n$ and conclude that
                           $g\cyc{x}g^{-1} = \cyc{x}$.]
                  \end{enumerate}
                  Note that this cuts down some of the work in computing
                  normalizers of cyclic subgroups since one does not have to
                  check $ghg^{-1} \in \cyc{x}$ for every $h \in \cyc{x}$.
%%%%%%%%%%%%%%%%%%%%%%%%%%%%%%%%%%%%%2.3.25%%%%%%%%%%%%%%%%%%%%%%%%%%%%%%%%%%%%%
   \item[2.3.25]  Let $G$ be a cyclic group of order $n$ and let $k$ be an
                  integer relatively prime to $n$. Prove that the map
                  $x \mapsto x^k$ is surjective. Use Lagrange's Theorem
                  (Exercise 1.7.19) to prove the same is true for any finite
                  group of order $n$. (For such $k$ each element has a
                  $k^{\text{th}}$ root in $G$. It follows from Cauchy's Theorem
                  in Section 3.2 that if $k$ is not relatively prime to the
                  order of $G$ then the map $x \mapsto x^k$ is not surjective.)
%%%%%%%%%%%%%%%%%%%%%%%%%%%%%%%%%%%%%2.3.26%%%%%%%%%%%%%%%%%%%%%%%%%%%%%%%%%%%%%
   \item[2.3.26]  Let $Z_n$ be a cyclic group of order $n$ and for each integer
                  $a$ let
                  $$\sigma_a : Z_n \mapsto Z_n \qquad by \qquad \sigma_a(x) =
                  x^a \quad \text{for all } x \in Z_n.$$
                  \begin{enumerate}
                     \item Prove that $\sigma_a$ is an automorphism of $Z_n$ if
                           and only if $a$ and $n$ are relatively prime(
                           automorphisms were introduced in Exercise 1.6.20).
                     \item Prove that $\sigma_a = \sigma_b$ if and only if
                           $a \equiv b$ (mod $n$).
                     \item Prove that \textit{every} automorphism of $Z_n$ is
                           equal to $\sigma_a$ for some integer $a$.
                     \item Prove that $\sigma_a\circ\sigma_b=\sigma_{ab}$.
                           Deduce that the map $\overline{a} \mapsto \sigma_a$
                           is an isomorphism of $(\Z/n\Z)^\times$ onto the
                           automorphism group of $Z_n$ (so Aut($Z_n$) is an
                           abelian group of order $\varphi(n)$).
                  \end{enumerate}
                  %%%%%MISSING CONTAINMENT%%%%%%%%
\end{enumerate}


































      \section{Subgroups Generated By Subsets Of A Group}
         \begin{enumerate}
%%%%%%%%%%%%%%%%%%%%%%%%%%%%%%%%%%Prob4.1%%%%%%%%%%%%%%%%%%%%%%%%%%%%%%%%%%%%%%%
   \item[4.1]  Mark each statement True or False. Justify each answer.
               \begin{enumerate}
                  \item To prove a universal statement $\forall$ $x$, $p(x)$, we 
                        let $x$ represent an arbitrary member from the system 
                        under consideration and show that $p(x)$ is true.
                  \item To prove an existential statement
                        $\exists$ $x \ni p(x)$, we must find a particular $x$ in 
                        the system for which $p(x)$ is true.
                  \item In writing a proof, it is important to include all the 
                        logical steps.
               \end{enumerate}

      \textbf{Solution:} 

      \begin{enumerate}
         \item True. Since the $x$ is arbitrary, then if $p(x)$ is true, it must
               be the case that the statement is true for all $x$.
         \item False. We must find at least one $x$ in the system for which
               $p(x)$ is true.
         \item False. Only ``relevant" steps should be included.
      \end{enumerate}
%%%%%%%%%%%%%%%%%%%%%%%%%%%%%%%%%%Prob4.2%%%%%%%%%%%%%%%%%%%%%%%%%%%%%%%%%%%%%%%
   \item[4.2]  Mark each statement True or False. Justify each answer.
               \begin{enumerate}
                  \item A proof by contradiction may use the tautology
                        $({\sim}p \Rightarrow c) \Leftrightarrow p$.
                  \item A proof by contradiction may use the tautology
                        $[(p \lor {\sim}q) \Rightarrow c] \Leftrightarrow
                        (p \Rightarrow q)$.
                  \item Definitions often play an important role in proofs.
               \end{enumerate}

      \textbf{Solution:}

      \begin{enumerate}
         \item True. [See Text]
         \item False. The left side of the tautology should be
               $[(p \land {\sim}q) \Rightarrow c]$.
         \item True. [See Text]
      \end{enumerate}
%%%%%%%%%%%%%%%%%%%%%%%%%%%%%%%%%%Prob4.3%%%%%%%%%%%%%%%%%%%%%%%%%%%%%%%%%%%%%%%
   \item[4.3]  Prove: There exists an integer $n$ such that $n^2 + 3n/2 = 1$. Is
               this integer unique?
			
		\textbf{Proof:} Let $n = -2$. Then we have $(-2)^2 + 3(-2)/2 = 1$, so that
		the integer $-2$ is a solution to the equation $n^2 + 3n/2 = 1$. We claim
		that this integer is unique. To show this, suppose that $n' \in \Z$ is
		also a solution; that is, $n'^2 + 3n'/2 = 1$. Multiplying the latter
		equation by 2 and factoring will give us $(n' + 2)(n' - 1/2)= 0$, so that
		$n' = -2$. \qed
%%%%%%%%%%%%%%%%%%%%%%%%%%%%%%%%%%Prob4.4%%%%%%%%%%%%%%%%%%%%%%%%%%%%%%%%%%%%%%%
   \item[4.4]  Prove: There exists a rational number $x$ such that
               $x^2 + 3x/2 = 1$. Is this rational number unique?
			
		\textbf{Proof:} Let $x = 1/2$. Then we have $(1/2)^2 + 3(1/2)/2 = 1$, so
		that the rational number $1/2$ is a solution to the equation
		$n^2 + 3n/2 = 1$. This rational number is not unique since the rational
		number $-2/1$ also solves the equation. \qed
%%%%%%%%%%%%%%%%%%%%%%%%%%%%%%%%%%Prob4.5%%%%%%%%%%%%%%%%%%%%%%%%%%%%%%%%%%%%%%%
   \item[4.5]  Prove: For every real number $x > 3$, there exists a real number
               $y < 0$ such that $x = 3y/(2 + y)$.
					
		\textbf{Proof:} Choose $y = 2x/(3-x)$ for some real number greater $x$
		than 3. First we must show that $y$ is negative and that $x = 3y/(2 + y)$.
		Since $x$ is greater than 3, we have that $2x$ is positive and $3 - x$ is
		negative, so that $y$ is negative. Finally, we have that
		\begin{align*}
			\frac{3y}{2 + y} &= \frac{\frac{6x}{3 - x}}{2 + \frac{2x}{3 - x}} \\
								  &= \frac{6x}{2(3 - x) + 2x} \\
								  &= \frac{6x}{6 - 2x + 2x} \\
								  &= x.
		\end{align*} \qed
%%%%%%%%%%%%%%%%%%%%%%%%%%%%%%%%%%Prob4.6%%%%%%%%%%%%%%%%%%%%%%%%%%%%%%%%%%%%%%%
   \item[4.6]  Prove: For every real number $x > 1$, there exist two distinct
               positive real numbers $y$ and $z$ such that
               $$x = \frac{y^2 + 9}{6y} = \frac{z^2 + 9}{6z}.$$
					
		\textbf{Proof:} Let $x$ be a real number greater than 1. So choose
		$$y = 3x - 3\sqrt{x^2 - 1} \text{ and } z = 3x + 3\sqrt{x^2 - 1}.$$
		It is routine to check that $y$ and $z$ satisfy the required equations.
		\qed
%%%%%%%%%%%%%%%%%%%%%%%%%%%%%%%%%%Prob4.7%%%%%%%%%%%%%%%%%%%%%%%%%%%%%%%%%%%%%%%
   \item[4.7] Prove: If $x^2 + x - 6 \ge 0$, then $x \le -3$ or $x \ge 2$.
	
		\textbf{Proof:} Suppose that $x^2 + x - 6 \ge 0$ for some real number $x$.
		Then we have that $x^2 + x - 6 = (x + 3)(x - 2)\ge 0$. Solving this
		inequality will give us $x \le -3$ or $x \ge 2$. \qed
%%%%%%%%%%%%%%%%%%%%%%%%%%%%%%%%%%Prob4.8%%%%%%%%%%%%%%%%%%%%%%%%%%%%%%%%%%%%%%%
   \item[4.8] Prove: If $x/(x - 1) \le 2$, then $x < 1$ or $x \ge 2$.
	
		\textbf{Proof:} Suppose $x/(x - 1) \le 2$ for some real number $x$.
		
		\textbf{Case I:} $x > 1$. \\
		Multiply the inequality $x/(x - 1) \le 2$ by the positive number $x - 1$
		to give us $x \le 2(x - 1)$. Solving the latter inequality results in
		$x \ge 2$.
		
		\textbf{Case II:} $x < 1$. \\
		Multiply the inequality $x/(x - 1) \le 2$ by the negative number $x - 1$
		to give us $x \ge 2(x - 1)$. Solving the latter inequality results in
		$x \le 2$. By assumption, $x$ must also satisfy $x < 1$. To satisfy
		$x < 1$ and $x \le 2$, we must have that $x < 1$.
		
		The proof is done. \qed
%%%%%%%%%%%%%%%%%%%%%%%%%%%%%%%%%%Prob4.9%%%%%%%%%%%%%%%%%%%%%%%%%%%%%%%%%%%%%%%
   \item[4.9] Prove: $\log_27$ is irrational.
	
		\textbf{Proof:} Suppose by way of contradiction that $\log_27$ is
		rational. We note that the $\log$ function is strictly increasing, and
		since $\log_22 = 1$, we must have $\log_27 > \log_22 > 1$, so that
		$\log_27$ is positive. Thus there exist positive integers $p$ and $q$
		such $\log_27 = p/q$. That is $2^{p/q} = 7$, so that $2^p = 7^q$. By the
		Fundamental Theorem of Arithmetic, it follows that $2^p$ and $7^q$ have
		the same prime factors. But this is false, since $2^p$'s only prime factor
		is 2 and $7^q$'s only prime factor is 7. Thus $\log_27$ is irrational.
		\qed
%%%%%%%%%%%%%%%%%%%%%%%%%%%%%%%%%%Prob4.10%%%%%%%%%%%%%%%%%%%%%%%%%%%%%%%%%%%%%%
   \item[4.10] Prove: If $x$ is a real number, then $|x - 2| \le 3$ implies that
               $-1 \le x \le 5$.
					
		\textbf{Proof:} Let $x$ be a real number. Suppose that $|x - 2| \le 3$.
		Then by definition we have that $-3 \le x - 2 \le 3$. Adding 2 to these
		inequalities results in $-1 \le x \le 5$. \qed
%%%%%%%%%%%%%%%%%%%%%%%%%%%%%%%%%%Prob4.11%%%%%%%%%%%%%%%%%%%%%%%%%%%%%%%%%%%%%%
   \item[4.11] Consider the following theorem:
               ``If $m^2$ is odd, then $m$ is odd." Indicate what, if anything, 
               is wrong with each of the following ``proofs."
               \begin{enumerate}
                  \item Suppose $m$ is odd. Then $m = 2k + 1$ for some integer
                        $k$. Thus $m^2 = (2k + 1)^2 = 4k^2 + 4k + 1 =
                        2(2k^2 + 2k) + 1$, which is odd. Thus if $m^2$ is odd,
                        then $m$ is odd.
                  \item Suppose $m$ is not odd. Then $m$ is even and $m = 2k$ 
                        for some integer $k$. Thus
                        $m^2 = (2k)^2 = 4k^2 = 2(2k^2)$, which is even. Thus if
                        $m$ is not odd, then $m^2$ is not odd. It follows that
                        if $m^2$ is odd, then $m$ is odd.
               \end{enumerate}
					
		\textbf{Solution:}
		
		\begin{enumerate}
			\item	The proof proved the converse of the original implication,
					instead of proving the implication; and as we know, the converse
					of an implication is not necessarily equivalent to the original
					implication.
			\item	The proof proved the contrapositive of the original statement,
					but it did not explicitly state so.
      \end{enumerate} 
%%%%%%%%%%%%%%%%%%%%%%%%%%%%%%%%%%Prob4.12%%%%%%%%%%%%%%%%%%%%%%%%%%%%%%%%%%%%%%
   \item[4.12] Consider the following theorem: ``If $xy = 0$, then $x = 0$ or
               $y = 0$." Indicate what, if anything, is wrong with each of the
               following ``proofs."
               \begin{enumerate}
                  \item Suppose $xy = 0$ and $x \neq 0$. Then dividing both
                        sides of the first equation by $x$ we have $y = 0$. Thus
                        if $xy = 0$, then $x = 0$ or $y = 0$.
                  \item There are two cases to consider. First suppose that
                        $x = 0$. Then $x \cdot y = x \cdot 0 = 0$. Similarly,
                        suppose that $y = 0$. Then $x \cdot y = x \cdot 0 = 0$.
                        In either case, $x \cdot y = 0$. Thus if $xy = 0$, then
                        $x = 0$ or $y = 0$.
               \end{enumerate}
					
		\textbf{Solution:}
		
		\begin{enumerate}
			\item	The proof is OK.
			\item	The proof proved the converse of the original implication,
					instead of proving the implication.
      \end{enumerate} 
%%%%%%%%%%%%%%%%%%%%%%%%%%%%%%%%%%Prob4.13%%%%%%%%%%%%%%%%%%%%%%%%%%%%%%%%%%%%%%
   \item[4.13] Suppose $x$ and $y$ are real numbers. Recall that a real number
               $m$ is rational iff $m = p/q$ where $p$ and $q$ are integers and
               $q \neq 0$. If a real number is not rational, then it is
               irrational. Prove the following. [You may use the fact that the
               sum and product of integers is again an integer.]
               \begin{enumerate}
                  \item If $x$ is rational and $y$ is rational, then $x + y$ is
                        rational.
                  \item If $x$ is rational and $y$ is rational, then $xy$ is 
                        rational.
                  \item If $x$ is rational and $y$ is irrational, then $x + y$
                        is irrational.
               \end{enumerate}
					
		\textbf{Proof:}
		
		\begin{enumerate}
			\item	Let $x$ and $y$ be rational numbers. Then there exist integers
					$p$ and $r$ and nonzero integers $q$ and $s$ such that $x = p/q$
					and $y = r/s$. Then it follows that
					\begin{align*}
						x + y &= \frac{p}{q} + \frac{r}{s} \\
								&= \frac{ps + qr}{qs}.
					\end{align*}
					
					Since the sum and product of integers are also integers we have
					that $x + y$ is also a rational number. \qed
			\item	Let $x$ and $y$ be rational numbers. Then there exist integers
					$p$ and $r$ and nonzero integers $q$ and $s$ such that $x = p/q$
					and $y = r/s$. Then it follows that $xy = pr/qs$. Since the
					product of integers is also an integer we have that $xy$ is also
					a rational number. \qed
			\item	Let $x$ be a rational number and let $y$ be an irrational number.
					Then we can write $x = p/q$ for some integer $p$ and nonzero
					integer $q$. Assume by way of contradiction that $x + y$ is
					rational; hence $x + y = r/s$ for some integers $r$ and $s$, with
					$s \neq 0$. Thus we have that $y = (r/s) + (-x)$. By (b) it
					follows that $y$ is a rational number, a contradiction. Thus
					$x + y$ must be irrational. \qed
      \end{enumerate} 
%%%%%%%%%%%%%%%%%%%%%%%%%%%%%%%%%%Prob4.14%%%%%%%%%%%%%%%%%%%%%%%%%%%%%%%%%%%%%%
   \item[4.14] Suppose $x$ and $y$ are real numbers. Prove or give a 
               counterexample.
               \begin{enumerate}
                  \item If $x$ is irrational and $y$ is irrational, then $x + y$
                        is irrational.
                  \item If $x + y$ is irrational, then $x$ is irrational or
                        $y$ is irrational.
                  \item If $x$ is irrational and $y$ is irrational, then $xy$
                        is irrational.
                  \item If $xy$ is irrational, then $x$ is irrational or $y$ is
                        irrational.
               \end{enumerate}
					
		\textbf{Proof:}
		
		\begin{enumerate}
			\item	False. Let $x = \sqrt{2}$ and let $y = -\sqrt{2}$. Then
					$x + y = 0$, a rational number. \qed
			\item	True. The contrapositive of this statement was proved in
					4.13(a). \qed
			\item	False. Let $x = y = \sqrt{2}$. Then $xy = 2$, a rational
					number. \qed
			\item	True. The contrapositive of this statement was proved in
					4.13(b). \qed
      \end{enumerate} 
%%%%%%%%%%%%%%%%%%%%%%%%%%%%%%%%%%Prob4.15%%%%%%%%%%%%%%%%%%%%%%%%%%%%%%%%%%%%%%
   \item[4.15] Consider the following theorem and proof.
               \begin{quote}
                  \textbf{Theorem}: If $x$ is rational and $y$ is irrational,
                  then $xy$ is irrational.
                  \begin{quote}
                     \textbf{Proof}: Suppose $x$ is rational and $y$ is
                     irrational. If $xy$ is rational, then we have $x = p/q$ and
                     $xy = m/n$ for some integers $p$, $q$, $m$ and $n$, with
                     $q \neq 0$ and $n \neq 0$. It follows that
                     $$y = \frac{xy}{x} = \frac{m/n}{p/q} = \frac{mq}{np}.$$
                     This implies that $y$ is rational, a contradiction. We
                     conclude that $xy$ must be irrational.
                  \end{quote}
               \end{quote}
               \begin{enumerate}
                  \item Find a specific counterexample to show that the theorem
                        is false.
                  \item Explain what is wrong with the proof.
                  \item What additional condition on $x$ in the hypothesis would
                        make the conclusion true?
               \end{enumerate}
					
		\textbf{Solution:}
		
		\begin{enumerate}
			\item	Let $x = 0$ and let $y$ be any irrational number. Then $xy = 0$,
					a rational number.
         \item The proof did not take into consideration that $x$ could be 0.
               If $x$ were zero, then the expression $y = xy/x$ would be
               illegal.
         \item In addition to being rational, $x$ must also be nonzero.

      \end{enumerate} 
%%%%%%%%%%%%%%%%%%%%%%%%%%%%%%%%%%Prob4.16%%%%%%%%%%%%%%%%%%%%%%%%%%%%%%%%%%%%%%
   \item[4.16] Prove or give a counterexample: If $x$ is irrational, then
               $\sqrt{x}$ is irrational.

      \textbf{Proof:} Assume by way of contradiction that for some irrational
      number $x$ that $\sqrt{x}$ is rational. Then we have $\sqrt{x} = p/q$ for
      some integers $p$ and $q$, with $q \neq 0$. Squaring both sides of this
      equality results in $x = p^2/q^2$; that is $x$ is rational, a
      contradiction. Thus $\sqrt{x}$ is irrational. \qed
%%%%%%%%%%%%%%%%%%%%%%%%%%%%%%%%%%Prob4.17%%%%%%%%%%%%%%%%%%%%%%%%%%%%%%%%%%%%%%
   \item[4.17] Prove or give a counterexample: There do not exist three
               consecutive even integers $a$, $b$, and $c$ such that
               $a^2 + b^2 = c^2$.

      \textbf{Counterexample:} Consider the triple $(-2, 0, 2)$ or $(6, 8, 10)$.
%%%%%%%%%%%%%%%%%%%%%%%%%%%%%%%%%%Prob4.18%%%%%%%%%%%%%%%%%%%%%%%%%%%%%%%%%%%%%%
   \item[4.18] Consider the following theorem: There do not exist three
               consecutive odd integers $a$, $b$, and $c$ such that
               $a^2 + b^2 = c^2$.
               \begin{enumerate}
                  \item Complete the following restatement of the theorem: For
                        every three consecutive odd integers $a$, $b$, and $c$,
                        we have that $a^2 + b^2 \neq c^2$.
                  \item Change the sentence in part (a) into an implication
                        $p \Rightarrow q$: If $a$, $b$, and $c$ are consecutive
                        odd integers, then $a^2 + b^2 \neq c^2$.
                  \item Fill in the blanks in the following proof of the
                        theorem.
                        \begin{quote}
                           \textbf{Proof}: Let $a$, $b$, and $c$ be consecutive
                           odd integers. Then $a = 2k + 1$, $b = 2k + 3$, and
                           $c = 2k + 5$ for some integer $k$. Suppose
                           $a^2 + b^2 = c^2$. Then
                           $(2k + 1)^2 + (2k + 3)^2 = (2k + 5)^2$.

                           \quad It follows that
                           $8k^2 + 16k + 10 = 4k^2 + 20k + 25$ and
                           $4k^2 - 4k - 15 = 0$. Thus $k = 5/2$ or $k = -3/2$.
                           This contradicts $k$ being an integer. Therefore,
                           there do not exist three consecutive odd integers
                           $a$, $b$, and $c$ such that $a^2 + b^2 = c^2$.
                        \end{quote}
                  \item Which of the tautologies in Example 3.12 best describes
                        the structure of the proof? 3.12 (g)
               \end{enumerate}
%%%%%%%%%%%%%%%%%%%%%%%%%%%%%%%%%%Prob4.19%%%%%%%%%%%%%%%%%%%%%%%%%%%%%%%%%%%%%%
   \item[4.19] Prove or give a counterexample: The sum of any five consecutive
               integers is divisible by five.

      \textbf{Proof:} Let $x$ be the least integer amongst a list of five
      consecutive integers. Then the integers are: $x$, $x + 1$, $x + 2$,
      $x + 3$, $x + 4$, and their sum is $5x + 10 = 5(x + 2)$; that is, their
      sum is divisible by 5. \qed
%%%%%%%%%%%%%%%%%%%%%%%%%%%%%%%%%%Prob4.20%%%%%%%%%%%%%%%%%%%%%%%%%%%%%%%%%%%%%%
   \item[4.20] Prove or give a counterexample: The sum of any four consecutive
               integers is never divisible by four.

      \textbf{Proof:} Let $x$ be the least integer amongst a list of four
      consecutive integers. Then the integers are: $x$, $x + 1$, $x + 2$,
      $x + 3$, and their sum is $4x + 6 = 4(x + 2) + 2$. This says that the
      sum of any four consecutive integers always a remainder of 2 when divided
      by 4, so that this sum is not divisible by four. \qed
%%%%%%%%%%%%%%%%%%%%%%%%%%%%%%%%%%Prob4.21%%%%%%%%%%%%%%%%%%%%%%%%%%%%%%%%%%%%%%
   \item[4.21] Prove or give a counterexample: For every positive integer $n$,
               $n^2 + 3n + 8$ is even.

      \textbf{Proof:} Let $n$ be a positive integer. We shall investigate the
      following two cases:

      \textbf{Cases I:} \textit{$n$ is even}. Thus we can write $n = 2k$ for
      some natural number $k$, so that
      $n^2 + 3n + 8 = 4k^2 + 6k + 8 = 2(2k^2 + 3k + 4)$, an even number.

      \textbf{Cases II:} \textit{$n$ is odd}. Thus we can write $n = 2k + 1$ for
      some natural number $k$, so that
      $n^2 + 3n + 8 = 4k^2 + 10k + 12 = 2(2k^2 + 5k + 6)$, an even number.

      Thus if $n$ is a positive integer, then $n^2 + 3n + 8$ is even. \qed
%%%%%%%%%%%%%%%%%%%%%%%%%%%%%%%%%%Prob4.22%%%%%%%%%%%%%%%%%%%%%%%%%%%%%%%%%%%%%%
   \item[4.22] Prove or give a counterexample: For every positive integer $n$,
               $n^2 + 4n + 8$ is even.
   
      \textbf{Counterexample:} If $n = 1$, then $n^2 + 4n + 8 = 13$, an odd
      number.
%%%%%%%%%%%%%%%%%%%%%%%%%%%%%%%%%%Prob4.23%%%%%%%%%%%%%%%%%%%%%%%%%%%%%%%%%%%%%%
   \item[4.23] Prove or give a counterexample: there do not exist irrational
               numbers $x$ and $y$ such that $x^y$ is rational.

      \textbf{Counterexample:} We do not know whether or not the number
      $\sqrt{2}^{\sqrt{2}}$ is rational, so we shall investigate the following
      two possibilities.

      \textbf{Cases I:} \textit{$\sqrt{2}^{\sqrt{2}}$ is irrational}. Then let
      $x = \sqrt{2}^{\sqrt{2}}$ and $y = \sqrt{2}$. Then we have $x^y = 2$.

      \textbf{Cases II:} \textit{$\sqrt{2}^{\sqrt{2}}$ is rational}. Then let
      $x = y = \sqrt{2}$, so that $x^y$ is a rational number.       
%%%%%%%%%%%%%%%%%%%%%%%%%%%%%%%%%%Prob4.24%%%%%%%%%%%%%%%%%%%%%%%%%%%%%%%%%%%%%%
   \item[4.24] Prove or give a counterexample: there do not exist rational
               numbers $x$ and $y$ such that $x^y$ is a positive integer and
               $y^x$ is a negative integer.

      \textbf{Counterexample:} Let $x = 1$ and $y = -2$. Then we have
      $x^y = 1$ and $y^x = -2$.
%%%%%%%%%%%%%%%%%%%%%%%%%%%%%%%%%%Prob4.25%%%%%%%%%%%%%%%%%%%%%%%%%%%%%%%%%%%%%%
   \item[4.25] Prove or give a counterexample: for all $x > 0$ we have
               $x^2 + 1 < (x + 1)^2 \le 2(x^2 + 1)$.

      \textbf{Proof:} Let $x$ be a positive real number. First we want to show
      that $x^2 + 1 < (x + 1)^2$. We assumed that $x > 0$ so that $2x > 0$;
      that is $-2x < 0$. Adding $(x + 1)^2$ to the inequality $-2x < 0$ results
      in $(x + 1)^2 - 2x < (x + 1)^2$, so that $x^2 + 1 < (x + 1)^2$. Now we
      must show that $(x + 1)^2 \le 2(x^2 + 1)$. We know that for any real
      number $y$ we have that $y^2 \ge 0$; thus we have that
      $(x - 1)^2 = x^2 - 2x + 1 \ge 0$, so that $-x^2 + 2x - 1 \le 0$. Adding
      $2x^2 + 2$ to the inequality $-x^2 + 2x - 1 \le 0$ results in
      $x^2 + 2x + 1 \le 2x^2 + 2$; that is, $(x + 1)^2 \le 2(x^2 + 1)$. The
      proof is done. \qed
\end{enumerate}

      \section{The Lattice Of Subgroups Of A Subgroup}
         \begin{enumerate}
%%%%%%%%%%%%%%%%%%%%%%%%%%%%%%%%%%%%%2.5.1%%%%%%%%%%%%%%%%%%%%%%%%%%%%%%%%%%%%%%
   \item[2.5.1]   Let $H$ and $K$ be subgroups of $G$. Exhibit all possible
                  sublattices which show only $G$, 1, $H$, $K$, and their joins
                  and intersections. What distinguishes the different drawings?
%%%%%%%%%%%%%%%%%%%%%%%%%%%%%%%%%%%%%2.5.2%%%%%%%%%%%%%%%%%%%%%%%%%%%%%%%%%%%%%%
   \item[2.5.2]   In each of (a) to (d) list all subgroups of $D_{16}$ that
                  satisfy the given condition.
                  \begin{enumerate}
                     \item Subgroups that are contained in $\cyc{sr^2, r^4}$
                     \item Subgroups that are contained in $\cyc{sr^7, r^4}$
                     \item Subgroups that contain $\cyc{r^4}$
                     \item Subgroups that contain $\cyc{s}$.
                  \end{enumerate}
                  
      \textbf{Solution.}
      
      \begin{enumerate}
         \item The subgroups that are contained in $\cyc{sr^2, r^4}$ are those
               that have an upward path to the join of $sr^2$ and $r^4$. Thus
               these subgroups are: $1$, $\cyc{r^4}$, $\cyc{sr^2}$,
               $\cyc{sr^6}$, and $\cyc{sr^2, r^4}$.
         \item The subgroups that are contained in $\cyc{sr^7, r^4}$ are those
               that have an upward path to the join of $sr^3$ and $r^4$. Since
               the join of $sr^7$ and $r^4$ is $\cyc{sr^3, r^4}$. It follows
               that these subgroups are: $1$, $\cyc{r^4}$, $\cyc{sr^3}$,
               $\cyc{sr^7}$, and $\cyc{sr^7, r^4} = \cyc{sr^3, r^4}$.
         \item The subgroups that contain $r^4$ are: $\cyc{r^4}$,
               $\cyc{sr^2, r^4}$, $\cyc{s, r^4}$, $\cyc{r^2}$,
               $\cyc{sr^3, r^4}$, $\cyc{sr^5, r^4}$, $\cyc{s, r^2}$, $\cyc{r}$,
               $\cyc{sr, r^2}$, and $D_{16}$.
         \item The subgroups that contain $s$ are: $\cyc{s}$, $\cyc{s, r^4}$,
               $\cyc{s, r^2}$, and $D_{16}$.
      \end{enumerate}
%%%%%%%%%%%%%%%%%%%%%%%%%%%%%%%%%%%%%2.5.3%%%%%%%%%%%%%%%%%%%%%%%%%%%%%%%%%%%%%%
   \item[2.5.3]   Show that the subgroup $\cyc{s, r^2}$ of $D_8$ is isomorphic
                  to $V_4$.
                  
      \textbf{Proof.} By Exercise 1.1.36, there is exactly one group, say $K$,
      of order 4 that has no element of order 4. Since
      $\cyc{s, r^2} = \{1, s, r^2, sr^2\}$, it follows that every non-identity
      element of $\cyc{s, r^2}$ has order 2, so that $\cyc{s, r^2} \cong K$.
      Similarly, the Klein-4 group has no element of order 4. Thus
      $V_4 \cong K$, and we conclude that $V_4 \cong \cyc{s, r^2}$. \qed
%%%%%%%%%%%%%%%%%%%%%%%%%%%%%%%%%%%%%2.5.4%%%%%%%%%%%%%%%%%%%%%%%%%%%%%%%%%%%%%%
   \item[2.5.4]   Use the given lattice to find all pairs of elements that
                  generate $D_8$ (there are 12 pairs).
                  
      \textbf{Solution.} It suffices to find all pairs of elements whose join is
      $D_8$. They are: $\cyc{s, r}$, $\cyc{s, rs}$, $\cyc{s, r^3s}$,
      $\cyc{r^2s, rs}$, $\cyc{r^2s, r^3s}$, $\cyc{r^2s, r}$, $\cyc{r^2s, r^3}$,
      $\cyc{r, rs}$, $\cyc{r, r^3s}$,  $\cyc{r^3, s}$, $\cyc{r^3, rs}$, and
      $\cyc{r^3, r^3s}$,
%%%%%%%%%%%%%%%%%%%%%%%%%%%%%%%%%%%%%2.5.5%%%%%%%%%%%%%%%%%%%%%%%%%%%%%%%%%%%%%%
   \item[2.5.5]   Use the given lattice to find all elements $x \in D_{16}$
                  such that $D_{16} = \cyc{x, s}$ (there are 8 such elements
                  $x$).
                  
      \textbf{Solution.} By observing the given lattice of $D_{16}$, we find
      that      
      $$x \in \{r, r^3, r^5, r^7, sr^3, sr^7, sr^5, sr\}.$$
%%%%%%%%%%%%%%%%%%%%%%%%%%%%%%%%%%%%%2.5.6%%%%%%%%%%%%%%%%%%%%%%%%%%%%%%%%%%%%%%
   \item[2.5.6]   Use the given lattices to help find the centralizers of every
                  element in the following groups:

                  (a) $D_8$ \qquad (b) $Q_8$ \qquad
                  (c) $S_3$ \qquad (d) $D_{16}$.
                  
      \begin{enumerate}
         \item $$
                  \begin{tabular}{@{}|c|c|@{}} \hline
                     Elements in $D_8$ & Centralizer \\ \hline
                     1, $r^2$ & $D_8$ \\ \hline
                     $r, r^3$ & $\cyc{r}$ \\ \hline
                     $s, r^2s$ & $\cyc{s, r^2}$ \\ \hline
                     $rs, r^3s$ & $\cyc{rs, r^2}$ \\ \hline
                  \end{tabular}
               $$
         \item $$
                  \begin{tabular}{@{}|c|c|@{}} \hline
                     Elements in $Q_8$ & Centralizer \\ \hline
                     $\pm1$ & $Q_8$ \\ \hline
                     $\pm i$ & $\cyc{i}$ \\ \hline
                     $\pm j$ & $\cyc{j}$ \\ \hline
                     $\pm k$ & $\cyc{k}$ \\ \hline
                  \end{tabular}
               $$
         \item $$
                  \begin{tabular}{@{}|c|c|@{}} \hline
                     Element(s) in $Q_8$ & Centralizer \\ \hline
                     1 & $S_3$ \\ \hline
                     (1 2) & $\cyc{(1\;2)}$ \\ \hline
                     (1 3) & $\cyc{(1\;3)}$ \\ \hline
                     (2 3) & $\cyc{(2\;3)}$ \\ \hline
                     (1 2 3), (1 3 2) & $\cyc{(1\;2\;3)}$ \\ \hline
                  \end{tabular}
               $$
         \item $$
                  \begin{tabular}{@{}|c|c|@{}} \hline
                     Elements in $D_{16}$ & Centralizer \\ \hline
                     1, $r^4$ & $D_{16}$ \\ \hline
                     $r, r^2, r^3, r^5, r^6, r^7$ & $\cyc{r}$ \\ \hline
                     $s, sr^4$ & $\cyc{s, r^4}$ \\ \hline
                     $sr, sr^5$ & $\cyc{sr^5, r^4}$ \\ \hline
                     $sr^2, sr^6$ & $\cyc{sr^2, r^4}$ \\ \hline
                     $sr^3, sr^7$ & $\cyc{sr^3, r^4}$ \\ \hline
                  \end{tabular}
               $$
      \end{enumerate}
%%%%%%%%%%%%%%%%%%%%%%%%%%%%%%%%%%%%%2.5.7%%%%%%%%%%%%%%%%%%%%%%%%%%%%%%%%%%%%%%
   \item[2.5.7]   Find the center of $D_{16}$.
   
      \textbf{Solution.} From Exercise 2.5.6(d), we see that only 1 and $r^4$
      are in the all the centralizers of the elements of $D_{16}$. Thus
      $Z(D_{16}) = \cyc{r_4}$.
%%%%%%%%%%%%%%%%%%%%%%%%%%%%%%%%%%%%%2.5.8%%%%%%%%%%%%%%%%%%%%%%%%%%%%%%%%%%%%%%
   \item[2.5.8]   In each of the following groups find the normalizer of each
                  subgroup:

                  (a) $S_3$ \qquad (b) $Q_8$.
%%%%%%%%%%%%%%%%%%%%%%%%%%%%%%%%%%%%%2.5.9%%%%%%%%%%%%%%%%%%%%%%%%%%%%%%%%%%%%%%
   \item[2.5.9]   Draw the lattices of subgroups of the following groups:

                  (a) $\Z/16\Z$ \qquad (b) $\Z/24\Z$ \qquad
                  (c) $\Z/48\Z$. [See Exercise 6 in Section 3.]
%%%%%%%%%%%%%%%%%%%%%%%%%%%%%%%%%%%%%2.5.10%%%%%%%%%%%%%%%%%%%%%%%%%%%%%%%%%%%%%
   \item[2.5.10]  Classify groups of order 4 by proving that if $|G| = 4$ then
                  $G \cong Z_4$ or $G\cong V_4$. [See Exercise 36, Section 1.1.]
%%%%%%%%%%%%%%%%%%%%%%%%%%%%%%%%%%%%%2.5.11%%%%%%%%%%%%%%%%%%%%%%%%%%%%%%%%%%%%%
   \item[2.5.11]  Consider the group of order 16 with the following
                  presentation:

                  $$QD_{16} = \cyc{\sigma, \tau : \sigma^8 = \tau^2 = 1,
                    \sigma\tau = \tau\sigma^3}$$
                  (called the \textit{quasidihedral} or \textit{semidihedral}
                  group of order 16). This group has three subgroups of order 8:
                  $\cyc{\tau, \sigma^2} \cong D_8, \cyc{\sigma} \cong Z_8$ and
                  $\cyc{\sigma^2, \sigma\tau} \cong Q_8$ and every proper
                  subgroup is contained in one of these three subgroups. Fill in
                  the missing subgroups in the lattice of all subgroups of the 
                  quasidiheral group on the following page, exhibiting each
                  subgroup with at most two generators. (This is another example
                  of a nonplanar lattice.)
\end{enumerate}

\noindent The next three examples lead to two nonisomorphic groups that have the 
          same lattice of subgroups.

\begin{enumerate}
%%%%%%%%%%%%%%%%%%%%%%%%%%%%%%%%%%%%%2.5.12%%%%%%%%%%%%%%%%%%%%%%%%%%%%%%%%%%%%%
   \item[2.5.12]  The group
                  $A = Z_2 \times Z_4 = \cyc{a, b : a^2 = b^4 = 1, ab = ba}$ has
                  order 8 and has three subgroups of order 4:
                  $\cyc{a, b^2} \cong V_4$, $\cyc{b} \cong Z_4$ and
                  \begin{verbatim}
                     *
                     *
                     *
                     *
                     *
                     *
                     *
                     *
                     *
                  \end{verbatim}
                  $\cyc{ab} \cong Z_4$ and every proper subgroup is contained in
                  one of these three. Draw the lattice of all subgroups of $A$,
                  giving each subgroup in terms of at most two generators.
%%%%%%%%%%%%%%%%%%%%%%%%%%%%%%%%%%%%%2.5.13%%%%%%%%%%%%%%%%%%%%%%%%%%%%%%%%%%%%%
   \item[2.5.13]  The group
                  $G = Z_2 \times Z_8 = \cyc{x, y : x^2 = y^8 = 1, xy = yx}$ has
                  order 16 and has three subgroups of order 8:
                  $\cyc{x, y^2} \cong Z_2 \times Z_4$, $\cyc{y} \cong Z_8$ and
                  $\cyc{xy} \cong Z_8$ and every proper subgroup is contained in
                  one of these three. Draw the lattice of all subgroups of $G$,
                  giving each subgroup in terms of at most two generators.
%%%%%%%%%%%%%%%%%%%%%%%%%%%%%%%%%%%%%2.5.14%%%%%%%%%%%%%%%%%%%%%%%%%%%%%%%%%%%%%
   \item[2.5.14]  Let $M$ be the group of order 16 with the following 
                  presentation:
                  $$\cyc{u, v : u^2 v^8 = 1, vu = uv^5}$$
                  (sometimes called the \textit{modular} group of order 16). It
                  has three subgroups of order 8: $\cyc{u, v^2}$, $\cyc{v}$, and
                  $\cyc{uv}$ and every proper subgroup is contained in one of
                  these three. Prove that $\cyc{u, v^2} \cong Z_2 \times Z_4$,
                  $\cyc{v} \cong Z_8$ and $\cyc{uv} \cong Z_8$. Show that the
                  lattice of subgroups of $M$ is the same as the lattice of
                  subgroups of $Z_2 \times Z_8$ (cf. Exercise 13) but that these
                  two groups are not isomorphic.
%%%%%%%%%%%%%%%%%%%%%%%%%%%%%%%%%%%%%2.5.15%%%%%%%%%%%%%%%%%%%%%%%%%%%%%%%%%%%%%
   \item[2.5.15]  Describe the isomorphism type of each of the three subgroups
                  of $D_{16}$ of order 8.
%%%%%%%%%%%%%%%%%%%%%%%%%%%%%%%%%%%%%2.5.16%%%%%%%%%%%%%%%%%%%%%%%%%%%%%%%%%%%%%
   \item[2.5.16]  Use the lattice of subgroups of the quasidihedral group of
                  order 16 to show that every element of order 2 is contained in
                  the proper subgroup $\cyc{\tau, \sigma^2}$.
%%%%%%%%%%%%%%%%%%%%%%%%%%%%%%%%%%%%%2.5.17%%%%%%%%%%%%%%%%%%%%%%%%%%%%%%%%%%%%%
   \item[2.5.17]  Use the lattice of subgroups of the modular group $M$ of order
                  16 to show that the set $\{x \in M : x^2 = 1\}$ is a subgroup
                  of $M$ isomorphic to the Klein 4-group.
%%%%%%%%%%%%%%%%%%%%%%%%%%%%%%%%%%%%%2.5.18%%%%%%%%%%%%%%%%%%%%%%%%%%%%%%%%%%%%%
   \item[2.5.18]  Use the lattice to help find the centralizer of every element
                  of $QD_{16}$.
%%%%%%%%%%%%%%%%%%%%%%%%%%%%%%%%%%%%%2.5.19%%%%%%%%%%%%%%%%%%%%%%%%%%%%%%%%%%%%%
   \item[2.5.19]  Use the lattice to help find $N_{D_{16}}(\cyc{s, r^4})$.
%%%%%%%%%%%%%%%%%%%%%%%%%%%%%%%%%%%%%2.5.20%%%%%%%%%%%%%%%%%%%%%%%%%%%%%%%%%%%%%
   \item[2.5.20]  Use the lattice of subgroups of $QD_{16}$ to help find the
                  normalizers.

                  (a) $N_{QD_{16}}(\cyc{\tau\sigma})$ \qquad
                  (b) $N_{QD_{16}}(\cyc{\tau, \sigma^4})$.
\end{enumerate}

\end{comment}
      \section{The Lattice Of Subgroups Of A Subgroup}
         \begin{enumerate}
%%%%%%%%%%%%%%%%%%%%%%%%%%%%%%%%%%%%%2.5.1%%%%%%%%%%%%%%%%%%%%%%%%%%%%%%%%%%%%%%
   \item[2.5.1]   Let $H$ and $K$ be subgroups of $G$. Exhibit all possible
                  sublattices which show only $G$, 1, $H$, $K$, and their joins
                  and intersections. What distinguishes the different drawings?
%%%%%%%%%%%%%%%%%%%%%%%%%%%%%%%%%%%%%2.5.2%%%%%%%%%%%%%%%%%%%%%%%%%%%%%%%%%%%%%%
   \item[2.5.2]   In each of (a) to (d) list all subgroups of $D_{16}$ that
                  satisfy the given condition.
                  \begin{enumerate}
                     \item Subgroups that are contained in $\cyc{sr^2, r^4}$
                     \item Subgroups that are contained in $\cyc{sr^7, r^4}$
                     \item Subgroups that contain $\cyc{r^4}$
                     \item Subgroups that contain $\cyc{s}$.
                  \end{enumerate}
                  
      \textbf{Solution.}
      
      \begin{enumerate}
         \item The subgroups that are contained in $\cyc{sr^2, r^4}$ are those
               that have an upward path to the join of $sr^2$ and $r^4$. Thus
               these subgroups are: $1$, $\cyc{r^4}$, $\cyc{sr^2}$,
               $\cyc{sr^6}$, and $\cyc{sr^2, r^4}$.
         \item The subgroups that are contained in $\cyc{sr^7, r^4}$ are those
               that have an upward path to the join of $sr^3$ and $r^4$. Since
               the join of $sr^7$ and $r^4$ is $\cyc{sr^3, r^4}$. It follows
               that these subgroups are: $1$, $\cyc{r^4}$, $\cyc{sr^3}$,
               $\cyc{sr^7}$, and $\cyc{sr^7, r^4} = \cyc{sr^3, r^4}$.
         \item The subgroups that contain $r^4$ are: $\cyc{r^4}$,
               $\cyc{sr^2, r^4}$, $\cyc{s, r^4}$, $\cyc{r^2}$,
               $\cyc{sr^3, r^4}$, $\cyc{sr^5, r^4}$, $\cyc{s, r^2}$, $\cyc{r}$,
               $\cyc{sr, r^2}$, and $D_{16}$.
         \item The subgroups that contain $s$ are: $\cyc{s}$, $\cyc{s, r^4}$,
               $\cyc{s, r^2}$, and $D_{16}$.
      \end{enumerate}
%%%%%%%%%%%%%%%%%%%%%%%%%%%%%%%%%%%%%2.5.3%%%%%%%%%%%%%%%%%%%%%%%%%%%%%%%%%%%%%%
   \item[2.5.3]   Show that the subgroup $\cyc{s, r^2}$ of $D_8$ is isomorphic
                  to $V_4$.
                  
      \textbf{Proof.} By Exercise 1.1.36, there is exactly one group, say $K$,
      of order 4 that has no element of order 4. Since
      $\cyc{s, r^2} = \{1, s, r^2, sr^2\}$, it follows that every non-identity
      element of $\cyc{s, r^2}$ has order 2, so that $\cyc{s, r^2} \cong K$.
      Similarly, the Klein-4 group has no element of order 4. Thus
      $V_4 \cong K$, and we conclude that $V_4 \cong \cyc{s, r^2}$. \qed
%%%%%%%%%%%%%%%%%%%%%%%%%%%%%%%%%%%%%2.5.4%%%%%%%%%%%%%%%%%%%%%%%%%%%%%%%%%%%%%%
   \item[2.5.4]   Use the given lattice to find all pairs of elements that
                  generate $D_8$ (there are 12 pairs).
                  
      \textbf{Solution.} It suffices to find all pairs of elements whose join is
      $D_8$. They are: $\cyc{s, r}$, $\cyc{s, rs}$, $\cyc{s, r^3s}$,
      $\cyc{r^2s, rs}$, $\cyc{r^2s, r^3s}$, $\cyc{r^2s, r}$, $\cyc{r^2s, r^3}$,
      $\cyc{r, rs}$, $\cyc{r, r^3s}$,  $\cyc{r^3, s}$, $\cyc{r^3, rs}$, and
      $\cyc{r^3, r^3s}$,
%%%%%%%%%%%%%%%%%%%%%%%%%%%%%%%%%%%%%2.5.5%%%%%%%%%%%%%%%%%%%%%%%%%%%%%%%%%%%%%%
   \item[2.5.5]   Use the given lattice to find all elements $x \in D_{16}$
                  such that $D_{16} = \cyc{x, s}$ (there are 8 such elements
                  $x$).
                  
      \textbf{Solution.} By observing the given lattice of $D_{16}$, we find
      that      
      $$x \in \{r, r^3, r^5, r^7, sr^3, sr^7, sr^5, sr\}.$$
%%%%%%%%%%%%%%%%%%%%%%%%%%%%%%%%%%%%%2.5.6%%%%%%%%%%%%%%%%%%%%%%%%%%%%%%%%%%%%%%
   \item[2.5.6]   Use the given lattices to help find the centralizers of every
                  element in the following groups:

                  (a) $D_8$ \qquad (b) $Q_8$ \qquad
                  (c) $S_3$ \qquad (d) $D_{16}$.
                  
      \begin{enumerate}
         \item $$
                  \begin{tabular}{@{}|c|c|@{}} \hline
                     Elements in $D_8$ & Centralizer \\ \hline
                     1, $r^2$ & $D_8$ \\ \hline
                     $r, r^3$ & $\cyc{r}$ \\ \hline
                     $s, r^2s$ & $\cyc{s, r^2}$ \\ \hline
                     $rs, r^3s$ & $\cyc{rs, r^2}$ \\ \hline
                  \end{tabular}
               $$
         \item $$
                  \begin{tabular}{@{}|c|c|@{}} \hline
                     Elements in $Q_8$ & Centralizer \\ \hline
                     $\pm1$ & $Q_8$ \\ \hline
                     $\pm i$ & $\cyc{i}$ \\ \hline
                     $\pm j$ & $\cyc{j}$ \\ \hline
                     $\pm k$ & $\cyc{k}$ \\ \hline
                  \end{tabular}
               $$
         \item $$
                  \begin{tabular}{@{}|c|c|@{}} \hline
                     Element(s) in $Q_8$ & Centralizer \\ \hline
                     1 & $S_3$ \\ \hline
                     (1 2) & $\cyc{(1\;2)}$ \\ \hline
                     (1 3) & $\cyc{(1\;3)}$ \\ \hline
                     (2 3) & $\cyc{(2\;3)}$ \\ \hline
                     (1 2 3), (1 3 2) & $\cyc{(1\;2\;3)}$ \\ \hline
                  \end{tabular}
               $$
         \item $$
                  \begin{tabular}{@{}|c|c|@{}} \hline
                     Elements in $D_{16}$ & Centralizer \\ \hline
                     1, $r^4$ & $D_{16}$ \\ \hline
                     $r, r^2, r^3, r^5, r^6, r^7$ & $\cyc{r}$ \\ \hline
                     $s, sr^4$ & $\cyc{s, r^4}$ \\ \hline
                     $sr, sr^5$ & $\cyc{sr^5, r^4}$ \\ \hline
                     $sr^2, sr^6$ & $\cyc{sr^2, r^4}$ \\ \hline
                     $sr^3, sr^7$ & $\cyc{sr^3, r^4}$ \\ \hline
                  \end{tabular}
               $$
      \end{enumerate}
%%%%%%%%%%%%%%%%%%%%%%%%%%%%%%%%%%%%%2.5.7%%%%%%%%%%%%%%%%%%%%%%%%%%%%%%%%%%%%%%
   \item[2.5.7]   Find the center of $D_{16}$.
   
      \textbf{Solution.} From Exercise 2.5.6(d), we see that only 1 and $r^4$
      are in the all the centralizers of the elements of $D_{16}$. Thus
      $Z(D_{16}) = \cyc{r_4}$.
%%%%%%%%%%%%%%%%%%%%%%%%%%%%%%%%%%%%%2.5.8%%%%%%%%%%%%%%%%%%%%%%%%%%%%%%%%%%%%%%
   \item[2.5.8]   In each of the following groups find the normalizer of each
                  subgroup:

                  (a) $S_3$ \qquad (b) $Q_8$.
%%%%%%%%%%%%%%%%%%%%%%%%%%%%%%%%%%%%%2.5.9%%%%%%%%%%%%%%%%%%%%%%%%%%%%%%%%%%%%%%
   \item[2.5.9]   Draw the lattices of subgroups of the following groups:

                  (a) $\Z/16\Z$ \qquad (b) $\Z/24\Z$ \qquad
                  (c) $\Z/48\Z$. [See Exercise 6 in Section 3.]
%%%%%%%%%%%%%%%%%%%%%%%%%%%%%%%%%%%%%2.5.10%%%%%%%%%%%%%%%%%%%%%%%%%%%%%%%%%%%%%
   \item[2.5.10]  Classify groups of order 4 by proving that if $|G| = 4$ then
                  $G \cong Z_4$ or $G\cong V_4$. [See Exercise 36, Section 1.1.]
%%%%%%%%%%%%%%%%%%%%%%%%%%%%%%%%%%%%%2.5.11%%%%%%%%%%%%%%%%%%%%%%%%%%%%%%%%%%%%%
   \item[2.5.11]  Consider the group of order 16 with the following
                  presentation:

                  $$QD_{16} = \cyc{\sigma, \tau : \sigma^8 = \tau^2 = 1,
                    \sigma\tau = \tau\sigma^3}$$
                  (called the \textit{quasidihedral} or \textit{semidihedral}
                  group of order 16). This group has three subgroups of order 8:
                  $\cyc{\tau, \sigma^2} \cong D_8, \cyc{\sigma} \cong Z_8$ and
                  $\cyc{\sigma^2, \sigma\tau} \cong Q_8$ and every proper
                  subgroup is contained in one of these three subgroups. Fill in
                  the missing subgroups in the lattice of all subgroups of the 
                  quasidiheral group on the following page, exhibiting each
                  subgroup with at most two generators. (This is another example
                  of a nonplanar lattice.)
\end{enumerate}

\noindent The next three examples lead to two nonisomorphic groups that have the 
          same lattice of subgroups.

\begin{enumerate}
%%%%%%%%%%%%%%%%%%%%%%%%%%%%%%%%%%%%%2.5.12%%%%%%%%%%%%%%%%%%%%%%%%%%%%%%%%%%%%%
   \item[2.5.12]  The group
                  $A = Z_2 \times Z_4 = \cyc{a, b : a^2 = b^4 = 1, ab = ba}$ has
                  order 8 and has three subgroups of order 4:
                  $\cyc{a, b^2} \cong V_4$, $\cyc{b} \cong Z_4$ and
                  \begin{verbatim}
                     *
                     *
                     *
                     *
                     *
                     *
                     *
                     *
                     *
                  \end{verbatim}
                  $\cyc{ab} \cong Z_4$ and every proper subgroup is contained in
                  one of these three. Draw the lattice of all subgroups of $A$,
                  giving each subgroup in terms of at most two generators.
%%%%%%%%%%%%%%%%%%%%%%%%%%%%%%%%%%%%%2.5.13%%%%%%%%%%%%%%%%%%%%%%%%%%%%%%%%%%%%%
   \item[2.5.13]  The group
                  $G = Z_2 \times Z_8 = \cyc{x, y : x^2 = y^8 = 1, xy = yx}$ has
                  order 16 and has three subgroups of order 8:
                  $\cyc{x, y^2} \cong Z_2 \times Z_4$, $\cyc{y} \cong Z_8$ and
                  $\cyc{xy} \cong Z_8$ and every proper subgroup is contained in
                  one of these three. Draw the lattice of all subgroups of $G$,
                  giving each subgroup in terms of at most two generators.
%%%%%%%%%%%%%%%%%%%%%%%%%%%%%%%%%%%%%2.5.14%%%%%%%%%%%%%%%%%%%%%%%%%%%%%%%%%%%%%
   \item[2.5.14]  Let $M$ be the group of order 16 with the following 
                  presentation:
                  $$\cyc{u, v : u^2 v^8 = 1, vu = uv^5}$$
                  (sometimes called the \textit{modular} group of order 16). It
                  has three subgroups of order 8: $\cyc{u, v^2}$, $\cyc{v}$, and
                  $\cyc{uv}$ and every proper subgroup is contained in one of
                  these three. Prove that $\cyc{u, v^2} \cong Z_2 \times Z_4$,
                  $\cyc{v} \cong Z_8$ and $\cyc{uv} \cong Z_8$. Show that the
                  lattice of subgroups of $M$ is the same as the lattice of
                  subgroups of $Z_2 \times Z_8$ (cf. Exercise 13) but that these
                  two groups are not isomorphic.
%%%%%%%%%%%%%%%%%%%%%%%%%%%%%%%%%%%%%2.5.15%%%%%%%%%%%%%%%%%%%%%%%%%%%%%%%%%%%%%
   \item[2.5.15]  Describe the isomorphism type of each of the three subgroups
                  of $D_{16}$ of order 8.
%%%%%%%%%%%%%%%%%%%%%%%%%%%%%%%%%%%%%2.5.16%%%%%%%%%%%%%%%%%%%%%%%%%%%%%%%%%%%%%
   \item[2.5.16]  Use the lattice of subgroups of the quasidihedral group of
                  order 16 to show that every element of order 2 is contained in
                  the proper subgroup $\cyc{\tau, \sigma^2}$.
%%%%%%%%%%%%%%%%%%%%%%%%%%%%%%%%%%%%%2.5.17%%%%%%%%%%%%%%%%%%%%%%%%%%%%%%%%%%%%%
   \item[2.5.17]  Use the lattice of subgroups of the modular group $M$ of order
                  16 to show that the set $\{x \in M : x^2 = 1\}$ is a subgroup
                  of $M$ isomorphic to the Klein 4-group.
%%%%%%%%%%%%%%%%%%%%%%%%%%%%%%%%%%%%%2.5.18%%%%%%%%%%%%%%%%%%%%%%%%%%%%%%%%%%%%%
   \item[2.5.18]  Use the lattice to help find the centralizer of every element
                  of $QD_{16}$.
%%%%%%%%%%%%%%%%%%%%%%%%%%%%%%%%%%%%%2.5.19%%%%%%%%%%%%%%%%%%%%%%%%%%%%%%%%%%%%%
   \item[2.5.19]  Use the lattice to help find $N_{D_{16}}(\cyc{s, r^4})$.
%%%%%%%%%%%%%%%%%%%%%%%%%%%%%%%%%%%%%2.5.20%%%%%%%%%%%%%%%%%%%%%%%%%%%%%%%%%%%%%
   \item[2.5.20]  Use the lattice of subgroups of $QD_{16}$ to help find the
                  normalizers.

                  (a) $N_{QD_{16}}(\cyc{\tau\sigma})$ \qquad
                  (b) $N_{QD_{16}}(\cyc{\tau, \sigma^4})$.
\end{enumerate}

\end{document}
