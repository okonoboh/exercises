\begin{enumerate}
%%%%%%%%%%%%%%%%%%%%%%%%%%%%%%%%%%Prob1.1%%%%%%%%%%%%%%%%%%%%%%%%%%%%%%%%%%%%%%%
   \item[1.1]  Suppose $a$ and $b$ are real numbers, not both 0. Find real
               numbers $c$ and $d$ such that
               $$1/(a + bi) = c + di.$$

      \textbf{Solution:} Let $a$ and $b$ real numbers, not both 0. Then we have
      \begin{align*}
         \frac{1}{a + bi} &= \frac{a - bi}{(a + bi)(a - bi)} \\
                          &= \frac{a - bi}{a^2 + b^2} \\
                          &= \frac{a}{a^2 + b^2} + \frac{-b}{a^2 + b^2}i.
      \end{align*}

      The above suggests that we take $c = a/(a^2 + b^2)$ and
      $d = -b/(a^2 + b^2)$.
%%%%%%%%%%%%%%%%%%%%%%%%%%%%%%%%%%Prob1.2%%%%%%%%%%%%%%%%%%%%%%%%%%%%%%%%%%%%%%%
   \item[1.2]  Show that
               $$\frac{-1 + \sqrt{3}i}{2}$$
               is a cube root of 1 (meaning that its cube equals 1).

      \textbf{Proof:} It suffices to show that the statement
      $$\left(\frac{-1 + \sqrt{3}i}{2}\right)^3 = 1$$
      is true. So
      \begin{align*}
         \left(\frac{-1 + \sqrt{3}i}{2}\right)^3 &=
            \left(\frac{-1 + \sqrt{3}i}{2}\right)
            \left(\frac{-1 + \sqrt{3}i}{2}\right)
            \left(\frac{-1 + \sqrt{3}i}{2}\right) \\
                                                 &=
            \left(\frac{-2 - 2\sqrt{3}i}{4}\right)
            \left(\frac{-1 + \sqrt{3}i}{2}\right) \\
                                                 &= 1.
      \end{align*} \qed
%%%%%%%%%%%%%%%%%%%%%%%%%%%%%%%%%%Prob1.3%%%%%%%%%%%%%%%%%%%%%%%%%%%%%%%%%%%%%%%
   \item[1.3]  Prove that $-(-v) = v$ for every $v \in V$.

      \textbf{Proof:} Let $v \in V$. It suffices to show that the additive
      inverse of $-v$ is $v$. This immediately follows because we have
      $v + (-v) = 0$, so that $-(-v) = v$. \qed
%%%%%%%%%%%%%%%%%%%%%%%%%%%%%%%%%%Prob1.4%%%%%%%%%%%%%%%%%%%%%%%%%%%%%%%%%%%%%%%
   \item[1.4]  Prove that if $a \in \F$, $v \in V$, and $av = 0$, then $a = 0$
               or $v = 0$.

      \textbf{Proof:} Let $a \in \F$ and let $v \in V$. Assume that $av = 0$. If
      $a = 0$, then the proof is done. So assume that $a \neq 0$. Since $a$ is
      not 0, it has a multiplicative inverse $a^{-1} \in \F$. So we shall
      multiply both sides of the equation $av = 0$ by $a^{-1}$ to get
      $1v = a^{-1}0 = 0$. Since $1v = v$, it follows that $v = 0$. \qed
%%%%%%%%%%%%%%%%%%%%%%%%%%%%%%%%%%Prob1.5%%%%%%%%%%%%%%%%%%%%%%%%%%%%%%%%%%%%%%%
   \item[1.5]  For each of the following subsets of $\F^3$, determine whether it
               is a subspace of $\F^3$:
               \begin{enumerate}
                  \item $\{(x_1, x_2, x_3) \in \F^3 : x_1 + 2x_2 + 3x_3 = 0\}$;
                  \item $\{(x_1, x_2, x_3) \in \F^3 : x_1 + 2x_2 + 3x_3 = 4\}$;
                  \item $\{(x_1, x_2, x_3) \in \F^3 : x_1x_2x_3 = 0\}$;
                  \item $\{(x_1, x_2, x_3) \in \F^3 : x_1 = 5x_3\}$;
               \end{enumerate}
      
      \textbf{Solution:} (We observe that all the subsets above are not empty.)

      \begin{enumerate}
         \item Let $F_1 = \{(x_1, x_2, x_3) \in \F^3 : x_1 + 2x_2 + 3x_3 = 0\}$.
               We claim that $F_1$ is a subspace of $\F^3$. So let
               $u = (a, b, c)$ and $v = (a', b', c')$ be members of $F_1$(so 
               that $a + 2b + 3c = a' + 2b' + 3c' = 0$), and let $r \in \F$. We 
               wish to show that $u - v$ and $ru$ are members of $F_1$. So
               $u - v = (a - a', b - b', c - c')$ and
               \begin{align*}
                  a - a' + 2(b - b') + 3(c - c') &=
                     (a + 2b + 3c) - (a' + 2b' + 3c') \\
                                                 &=
                     0 - 0 = 0,
               \end{align*}
               so that $u - v \in F_1$. Also $ru = (ra, rb, rc)$ and
               $ra + 2rb + 3rc = r(a + 2b + 3c) = r0 = 0$, so that
               $ru \in F_1$. \qed
         \item Let $F_2 = \{(x_1, x_2, x_3) \in \F^3 : x_1 + 2x_2 + 3x_3 = 4\}$.
               We claim that $F_2$ is not a subspace of $\F^3$. This is evident
               because $(0, 0, 0) \notin F_2$.
         \item Let $F_3 = \{(x_1, x_2, x_3) \in \F^3 : x_1x_2x_3 = 0\}$. We
               claim that $F_3$ is not a subspace of $\F^3$ because $F_3$ is not
               closed under addition. For example: although $(1, 1, 0)$ and
               $(0, 0, 1)$ are members of $F_3$, their sum $(1, 1, 1)$ is not
               in $F_3$.
         \item Let $F_4 = \{(x_1, x_2, x_3) \in \F^3 : x_1 = 5x_3\}$.
               We claim that $F_4$ is a subspace of $\F^3$. So let
               $u = (a, b, c)$ and $v = (a', b', c')$ be members of $F_4$(so 
               that $a = 5c$ and $a' = 5c'$), and let $r \in \F$. We 
               wish to show that $u - v$ and $ru$ are members of $F_4$. So
               $u - v = (a - a', b - b', c - c')$ and
               $a - a' = 5c - 5c' = 5(c - c')$ so that $u - v \in F_4$. Also
               $ru = (ra, rb, rc)$ and $ra = r(5c) = 5(rc)$, so that
               $ru \in F_4$. \qed      
      \end{enumerate}
%%%%%%%%%%%%%%%%%%%%%%%%%%%%%%%%%%Prob1.6%%%%%%%%%%%%%%%%%%%%%%%%%%%%%%%%%%%%%%%
   \item[1.6]  Give an example of a nonempty subset $U$ of $\R^2$ such that $U$
               is closed under addition and under taking additive inverses
               (meaning $-u \in U$ whenever $u \in U$), but $U$ is not a
               subspace of $\R^2$.

      \textbf{Solution:} Let $U = \{(x, y) \in \Z^2\}$. Clearly $U$ is nonempty
      and closed under addition and taking inverses. However for $(1, 1) \in U$,
      $0.5(1, 1) \notin U$ so that $U$ is not a subspace of $\R^2$.
%%%%%%%%%%%%%%%%%%%%%%%%%%%%%%%%%%Prob1.7%%%%%%%%%%%%%%%%%%%%%%%%%%%%%%%%%%%%%%%
   \item[1.7]  Give an example of a nonempty subset $U$ of $\R^2$ such that $U$
               is closed under scalar multiplication, but $U$ is not a subspace
               of $\R^2$.

      \textbf{Solution:} Let
      $U = \{(x, y) \in \R^2 : x = 3y \mbox{ or } x = 5y\}$. Let $r \in R$ and
      let $(u_1, u_2) \in U$. Then $u_1 = 3u_2$ or $u_1 = 5u_2$, so that
      $ru_1 = 3ru_2$ or $ru_1 = 5ru_2$; hence $r(u_1, u_2)$ is in $U$. We have
      shown that $U$ is closed under scalar multiplication. However, $U$ is not
      a subspace of $\R^2$ because it is not closed under addition. For example,
      $(1, 3)$ and $(1, 5)$ are members of $U$ but their sum, $(2, 8)$, is not.
%%%%%%%%%%%%%%%%%%%%%%%%%%%%%%%%%%Prob1.8%%%%%%%%%%%%%%%%%%%%%%%%%%%%%%%%%%%%%%%
   \item[1.8]  Prove that the intersection of any collection of subspaces of $V$
               is a subspace of $V$.

      \textbf{Proof:} Let $\{V_i : i \in I\}$ be a collection of subspaces of
      some vector space $V$, where $I$ is some indexing set. We want to show
      that
      $$T = \bigcap_{i \in I} V_i$$
      is a subspace of $V$. The set $T$ is not empty since each $V_i$ contains
      0, so that $T$ contains 0. So let $a, b \in T$, and let $r \in \F$. It
      suffices to show that $a - b$ and $ra$ are also members of $T$. Since
      $a$ and $b$ are in $T$, then $a$ and $b$ must be in each $V_i$. Since each
      $V_i$ is a vector space, it must contain $a - b$ and $ra$. Hence $T$ must
      contain $a - b$ and $ra$. \qed
%%%%%%%%%%%%%%%%%%%%%%%%%%%%%%%%%%Prob1.9%%%%%%%%%%%%%%%%%%%%%%%%%%%%%%%%%%%%%%%
   \item[1.9]  Prove that the union of two subspaces of $V$ is a subspace of $V$
               if and only if one of the subspaces is contained in the other.

      \textbf{Proof:} Let $T$ and $W$ be subspaces of some vector space $V$.

      $(\Leftarrow)$ Assume that $T \subseteq W$. Then $T \cup W = W$, a
      subspace of $V$ as hypothesized.

      $(\Rightarrow)$ Assume that $T$ is not contained in $W$ and that $W$ is 
      not contained in $T$. That is, there exist $t \in T$ and $w \in W$ such
      that $t \notin W$ and $w \notin T$. We shall complete the proof by way of
      contradiction. So suppose that $T \cup W$ is a vector space of $V$. Then
      since $T \cup W$ is closed under addition, we must have
      $t + w \in (T \cup W)$. So we now investigate the following two cases:

      \textbf{Case I:} $t + w \in T$. Then since $-t \in T$, we must have that
      $w = (t + w) + (-t) \in T$, a contradiction.

      \textbf{Case II:} $t + w \in W$. Then since $-w \in W$, we must have that
      $t = (t + w) + (-w) \in W$, a contradiction.

      Thus $T \cup W$ is not a subspace of $V$. \qed
%%%%%%%%%%%%%%%%%%%%%%%%%%%%%%%%%%Prob1.10%%%%%%%%%%%%%%%%%%%%%%%%%%%%%%%%%%%%%%
   \item[1.10] Suppose that $U$ is a subspace of $V$. What is $U + U$?

      \textbf{Proof:} Let $U$ be a subspace of some vector space $V$. We claim
      that $U + U = U$. So we must show that each set is a subset of the other.
      Let $u \in U$. Since $u = u + 0$ and since $0 \in U$, it follows that
      $u \in U + U$, so that $U \subseteq U + U$. Now let $u_1 + u_2 \in U + U$.
      Since $u_1$ and $u_2$ are members of $U$ and since $U$ is closed under
      addition, we must have that $u_1 + u_2 \in U$, so that
      $U + U \subseteq U$. Thus $U = U + U$. \qed
%%%%%%%%%%%%%%%%%%%%%%%%%%%%%%%%%%Prob1.11%%%%%%%%%%%%%%%%%%%%%%%%%%%%%%%%%%%%%%
   \item[1.11] Is the operation of addition on the subspaces of $V$ commutative?
               Associative? (In other words, if $U_1$, $U_2$, $U_3$ are 
               subspaces of $V$, is $U_1 + U_2 = U_2 + U_1$? Is
               $(U_1 + U_2) + U_3 = U_1 + (U_2 + U_3)$?)

      \textbf{Solution:} Let $V$ be a vector space. We can easily show that the
      operation of addition on the subspaces of $V$ is commutative and 
      associative. Let $U_1$, $U_2$, and $U_3$ be subspaces of $V$. We want to
      show that addition on the subspaces is commutative; that is,
      $U_1 + U_2 = U_2 + U_1$. So consider some element $u \in U_1 + U_2$. Then
      $u = v + w$ for some $v \in U_1$ and $w \in U_2$. Since addition is
      commutative in $V$, we must have $u = v + w = w + v$, a member of
      $U_2 + U_1$; thus $U_1 + U_2 \subseteq U_2 + U_1$. We can similarly show
      that $U_2 + U_1 \subseteq U_1 + U_2$ to conclude that
      $U_1 + U_2 = U_2 + U_1$. Using this same procedure and the associativity
      of addition in $V$ we can also conclude that
      $(U_1 + U_2) + U_3 = U_1 + (U_2 + U_3)$, so that addition on the subspaces
      of $V$ is commutative and associative.
%%%%%%%%%%%%%%%%%%%%%%%%%%%%%%%%%%Prob1.12%%%%%%%%%%%%%%%%%%%%%%%%%%%%%%%%%%%%%%
   \item[1.12] Does the operation of addition on the subspaces of $V$ have an
               additive identity? Which subspaces have additive inverses?

      \textbf{Solution} Let $V$ be a vector space and let $U$ be any subspace of
      $V$. Then $U + \{0\} = U$, so that the zero subspace is the additive
      identity of the operation of addition on the subspaces of $V$. Suppose
      that $U$ has an additive inverse $W$. Then we must have that
      $U + W = \{0\}$. But this would imply that $U \subseteq \{0\}$ and
      $W \subseteq \{0\}$, so that $U = W = \{0\}$. So only the zero subspace
      has an additive inverse.
%%%%%%%%%%%%%%%%%%%%%%%%%%%%%%%%%%Prob1.13%%%%%%%%%%%%%%%%%%%%%%%%%%%%%%%%%%%%%%
   \item[1.13] Prove or give a counterexample: if $U_1$, $U_2$, $W$ are 
               subspaces of $V$ such that
               $$U_1 + W = U_2 + W,$$
               then $U_1 = U_2$.

      \textbf{Counterexample:} Let $V$ be a nontrivial vector space. We showed
      that in Problem 1.10 that $V + V = V$. Thus we have
      $$\{0\} + V = V + V,$$
      but $V \neq \{0\}$.
%%%%%%%%%%%%%%%%%%%%%%%%%%%%%%%%%%Prob4.14%%%%%%%%%%%%%%%%%%%%%%%%%%%%%%%%%%%%%%
   \item[1.14] Suppose $U$ is the subspace of $\mathcal{P}(\F)$ consisting of
               all polynomials $p$ of the form
               $$p(z) = az^2 + bz^5,$$
               where $a, b \in \F$. Find a subspace $W$ of $\mathcal{P}(\F)$
               such that $\mathcal{P}(\F) = U \oplus W$.

      \textbf{Solution:} Let $W$ be the set of all polynomials such that for
      for each nonnegative integer $m$ there exists a $p \in W$ of the form if
      and only if
      $$p(z) = a_0 + a_1z + a_2z^3 + a_3z^4 + a_4z^6 + \cdots + a_mz^{m + 2}.$$
      We observe that $W$ is a subspace of $\mathcal{P}(\F)$, that
      $\mathcal{P}(\F) = U + W$, and that $U \cap W = \{0\}$. Hence by
      Proposition 1.9, we must have $\mathcal{P}(\F) = U \oplus W$.
%%%%%%%%%%%%%%%%%%%%%%%%%%%%%%%%%%Prob1.15%%%%%%%%%%%%%%%%%%%%%%%%%%%%%%%%%%%%%%
   \item[1.15] Prove or give a counterexample: if $U_1$, $U_2$, $W$ are
               subspaces of $V$ such that
               $$V = U_1 \oplus W \quad \text{ and } \quad V = U_2 \oplus W,$$
               then $U_1 = U_2$.

      \textbf{Counterexample:} Let $V = \R^2$, $U_1 = \{(x, 0) : x \in \R\}$,
      $U_2 = \{(s, 5s) : s \in \R\}$, and $W = \{(0, y) : y \in \R\}$. One can
      easily check that
      $$V = U_1 \oplus W \quad \text{ and } \quad V = U_2 \oplus W,$$
      but $U_1 \neq U_2$.
\end{enumerate}
