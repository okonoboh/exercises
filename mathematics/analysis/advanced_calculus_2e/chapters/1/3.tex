\begin{enumerate}
%%%%%%%%%%%%%%%%%%%%%%%%%%%%%%%%%%%Lemm1.3.1%%%%%%%%%%%%%%%%%%%%%%%%%%%%%%%%%%%%
   \item[]        \textbf{Lemma 1.3.1} Let $a$ and $b$ positive real numbers.
                  Then we have that
                  $$\frac{a}{b} + \frac{b}{a} \ge 2.$$

      \textbf{Proof.} We must have that $(a - b)^2 \ge 0$, so that
      $a^2 + b^2 \ge 2ab$. So we can divide the inequality $a^2 + b^2 \ge 2ab$
      by the positive number $ab$ to conclude that
      $$\frac{a}{b} + \frac{b}{a} \ge 2.$$ \qed
%%%%%%%%%%%%%%%%%%%%%%%%%%%%%%%%%%%Lemm1.3.2%%%%%%%%%%%%%%%%%%%%%%%%%%%%%%%%%%%%
   \item[]        \textbf{Lemma 1.3.2} Let $a$ be a nonzero real number, and let
                  $m$ be an integer. Then we have that
                  $$a^{m+1} = a^ma^1.$$

      \textbf{Proof.}  For a natural number $s$, we shall be using the following 
      facts:
      $$a^s = \underbrace{a \cdot a \cdots a}_{s \text{ factors}} \quad
        \text{and} \quad a^{-s} = \underbrace{a^{-1} \cdot a^{-1}\cdots
        a^{-1}}_{s \text{ factors}}.$$

      \textbf{Case 1:} \textit{$m = 0$}. It follows that
      $$a^{m+1} = a^{0+1} = a^1 = 1\cdot a^1 = a^0 \cdot a^1 = a^m \cdot a^1.$$

      \textbf{Case 2:} \textit{$m$ is positive.} Then it follows that $m + 1$ is 
      positive, so that
      \begin{align*}
         a^{m + 1} &= \underbrace{a \cdot a \cdots a}_{m+1\text{ factors}} \\
            &= \underbrace{a \cdot a \cdots a}_{m\text{ factors}} \cdot a^1 \\
            &= a^ma^1.
      \end{align*}

      \textbf{Case 3:} \textit{$m$ is negative.} If $m = -1$, then we have that
      $$a^{m+1} = a^{-1+1} = a^0 = 1 = a^{-1}a^1 = a^ma^1.$$
      If $m < -1$, then $m + 1$ is negative, so that $-(m + 1) = -m - 1$ is 
      positive. Thus
      \begin{align*}
         a^ma^1 &= a^{-(-m)}a^1 \\
                &= \underbrace{a^{-1} \cdot a^{-1} \cdots a^{-1}}_{
                   -m\text{ factors}} \cdot a^1 \\
                &= \underbrace{a^{-1} \cdot a^{-1} \cdots a^{-1}}_{
                   -m-1\text{ factors}} \cdot (a^{-1} \cdot a^1) \\
                &= \underbrace{a^{-1} \cdot a^{-1} \cdots a^{-1}}_{
                   -m-1\text{ factors}} \\
                &= a^{-(-m-1)} \\
                &= a^{m+1}.
      \end{align*}

      In all cases, we can see that our assertion holds. \qed
%%%%%%%%%%%%%%%%%%%%%%%%%%%%%%%%%%%Lemm1.3.3%%%%%%%%%%%%%%%%%%%%%%%%%%%%%%%%%%%%
   \item[]        \textbf{Lemma 1.3.3} Let $a$ and $b$ be nonzero real numbers.
                  Then we have that
                  $$(ab)^{-1} = a^{-1}b^{-1}.$$

      \textbf{Proof.} By the commutativity of multiplication on the reals, we
      have that      
      \begin{equation}
         \left(\frac{1}{a} \cdot \frac{1}{b}\right)(ab) =
         \left(\frac{1}{a} \cdot a\right)\cdot \left(\frac{1}{b}
         \cdot b\right) = 1 \cdot 1 = 1, \label{l1_3_3_1}
      \end{equation}
      so that
      \begin{align*}
         a^{-1}b^{-1} &= \frac{1}{a} \cdot \frac{1}{b} &[\text{By definition}] \\
                      &= (ab)^{-1}. &[\eqref{l1_3_3_1}]
      \end{align*}
%%%%%%%%%%%%%%%%%%%%%%%%%%%%%%%%%%%Prob1.3.1%%%%%%%%%%%%%%%%%%%%%%%%%%%%%%%%%%%%
   \item[1.3.1]   Write out the Difference of Powers Formula explicitly for
                  $n = 4$ and 5.
                  
      \textbf{Solution.} For $n = 4$, we have
      $$a^4 - b^4 = (a - b)(a^3 + a^2b + ab^2 + b^3),$$
      and for $n = 5$, we have
      $$a^5 - b^5 = (a - b)(a^4 + a^3b + a^2b^2 + ab^3 + b^4).$$
%%%%%%%%%%%%%%%%%%%%%%%%%%%%%%%%%%Prob1.3.2%%%%%%%%%%%%%%%%%%%%%%%%%%%%%%%%%%%%%
   \item[1.3.2]   Write out the Binomial Formula explicitly for $n = 2, 3$, and
                  4.
                  
      \textbf{Solution.} 
      \begin{align*}
         (a + b)^2 &= \binom{2}{0}a^2b^0 + \binom{2}{1}a^1b^1 +
                      \binom{2}{2}a^0b^2 \\
                   &= a^2 + 2ab + b^2, \\
         (a + b)^3 &= \binom{3}{0}a^3b^0 + \binom{3}{1}a^2b^1 +
                      \binom{3}{2}a^1b^2 + \binom{3}{3}a^0b^3 \\
                   &= a^3 + 3a^2b + 3ab^2 + b^3, \text{ and } \\
         (a + b)^4 &= \binom{4}{0}a^4b^0 + \binom{4}{1}a^3b^1 +
                      \binom{4}{2}a^2b^2 + \binom{4}{3}a^1b^3 +
                      \binom{4}{4}a^0b^4 \\
                   &= a^4 + 3a^3b + 6a^2b^2 + 4ab^3 + b^4.
      \end{align*}
%%%%%%%%%%%%%%%%%%%%%%%%%%%%%%%%%%Prob1.3.3%%%%%%%%%%%%%%%%%%%%%%%%%%%%%%%%%%%%%
   \item[1.3.3]   Show that the Triangle Inequality becomes an equality if $a$
                  and $b$ are of the same sign.

      \textbf{Proof.} Let $a$ and $b$ be real numbers. Suppose first that $a$
      and $b$ are positive. By the Positivity Axioms, it follows that $a + b$ 
      must also be positive, so that we have $|a + b| = a + b = |a| + |b|$. Now 
      suppose $a$ and $b$ are both negative. Then it follows that $-a$ and $-b$
      are positive, so that $|a| + |b| = -a + (-b) = -(a + b) = |a + b|$. \qed
%%%%%%%%%%%%%%%%%%%%%%%%%%%%%%%%%%Prob1.3.4%%%%%%%%%%%%%%%%%%%%%%%%%%%%%%%%%%%%%
   \item[1.3.4]   Let $a > 0$. Prove that if $x$ is a number such that
                  $|x - a| < a/2$, then $x > a/2$. 

      \textbf{Proof.} Suppose $x$ is a real number such that $|x - a| < a/2$.
      From Equation 1.6 (in the text), we have that $-a/2 < x - a < a/2$.
      Particularly, we have that $-a/2 < x - a$, so that adding $a$ to both
      sides of the inequality $-a/2 < x - a$ will yield $x > a/2$. \qed
         
%%%%%%%%%%%%%%%%%%%%%%%%%%%%%%%%%%Prob1.3.5%%%%%%%%%%%%%%%%%%%%%%%%%%%%%%%%%%%%%
   \item[1.3.5]   Let $b < 0$. Prove that if $x$ is a number such that
                  $|x - b| < |b|/2$, then $x < b/2$.

      \textbf{Proof.} Suppose $x$ is a real number such that $|x - b| < |b|/2$.
      From Equation 1.6 (in the text), we have that $-|b|/2 < x - b < |b|/2$.
      Particularly, we have that $x - b < |b|/2 = -b/2$, so that adding $b$ to 
      both sides of the inequality $x - b < -b/2$ will yield $x < b/2$. \qed 
%%%%%%%%%%%%%%%%%%%%%%%%%%%%%%%%%%Prob1.3.6%%%%%%%%%%%%%%%%%%%%%%%%%%%%%%%%%%%%%
   \item[1.3.6]   Which of the following inequalities hold for all numbers $a$
                  and $b$? Justify your conclusions.
                  \begin{enumerate}
                     \item $|a + b| \ge |a| + |b|$.
                     \item $|a + b| \le |a| - |b|$.
                  \end{enumerate}

      \textbf{Solution.} Neither holds.
      
      \begin{enumerate}
         \item \textbf{Counterexample:} Let $a = 7$ and $b = -1$. Then
               $|7 + (-1)| = 6 \not\ge 8 = |7| + |-1|$.
         \item \textbf{Counterexample:} Let $a = 7$ and $b = 2$. Then
               $|7 + 2| = 9 \not\le 5 = |7| - |2|$.
      \end{enumerate}
%%%%%%%%%%%%%%%%%%%%%%%%%%%%%%%%%%Prob1.3.7%%%%%%%%%%%%%%%%%%%%%%%%%%%%%%%%%%%%%
   \item[1.3.7]   By writing $a = (a + b) + (-b)$ use the Triangle Inequality to
                  obtain $|a| - |b| \le |a + b|$. Then interchange $a$ and $b$
                  to show that
                  $$||a| - |b|| \le |a + b|.$$
                  Then replace $b$ by $-b$ to obtain
                  $$||a| - |b|| \le |a - b|.$$

      \textbf{Proof.} Let $a$ and $b$ be real numbers. By the Triangle
      Inequality, we have that
      $$|a| = |(a + b) + (-b)| \le |a + b| + |-b|,$$
      so that
      \begin{equation}
         |a| - |b| = |a| - |-b| \le |a + b|, \label{1_3_7_1}
      \end{equation}
      and by multiplying inequality \eqref{1_3_7_1} by $-1$, we have that
      $$-|a + b| \le |b| - |a|,$$
      and interchanging $a$ and $b$ in inequality \eqref{1_3_7_1} will also give 
      us
      $$|b| - |a| \le |a + b|,$$
      so that
      $$-|a + b| \le |b| - |a| \le |a + b|.$$
      That is,
      \begin{equation}
         ||a| - |b|| = ||b| - |a|| \le |a + b|. \label{1_3_7_2}
      \end{equation}
      Finally we obtain the inequality $||a| - |b|| \le |a - b|$ by replacing
      $b$ with $-b$ in inequality \eqref{1_3_7_2}. \qed
   
%%%%%%%%%%%%%%%%%%%%%%%%%%%%%%%%%%Prob1.3.8%%%%%%%%%%%%%%%%%%%%%%%%%%%%%%%%%%%%%
   \item[1.3.8]   Let $a$ and $b$ be numbers such that $|a - b| \le 1$. Prove
                  that $|a| \le |b| + 1$.

      \textbf{Proof.} Since $|a - b| \le 1$, it follows that
      $-1 \le a - b \le 1$, so that
      $$b - 1 \le a \le b + 1.$$

      Let us now investigate the following two cases:

      \textbf{Case 1.} \textit{$b$ is negative.} Thus
      $$-(|b| + 1) = -|b| - 1 = b - 1 \le a \le b + 1 < -b + 1 = |b| + 1.$$

      \textbf{Case 2.} \textit{$b$ is nonnegative.} Thus
      $$-(|b| + 1) = -|b| - 1 = -b - 1 \le b - 1 \le a \le b + 1 = |b| + 1.$$
      In either case, we observe that $-(|b| + 1) \le a \le |b| + 1$, so that
      $|a| \le |b| + 1$. \qed
%%%%%%%%%%%%%%%%%%%%%%%%%%%%%%%%%%Prob1.3.9%%%%%%%%%%%%%%%%%%%%%%%%%%%%%%%%%%%%%
   \item[1.3.9]   For a natural number $n$ and any two nonnegative numbers $a$
                  and $b$, use the Difference of Powers Formula to prove that
                  $$a \le b \qquad \text{if and only if }a^n \le b^n.$$

      \textbf{Proof.} Let $n$ be a natural number, and let $a$ and $b$ be 
      nonnegative real numbers. It is clear that the statement of our problem
      holds if $a = b$ or $a = 0$ or $b = 0$ or $n = 1$. So assume that
      $a \neq b$, $a$ and $b$ are positive, and $n > 1$. Lemma 1 from Section
      1.1 tells us that if $a^n < b^n$ then $a < b$, so it suffices to prove the 
      other direction. Suppose that $a < b$. Since $a$ and $b$ are positive, it
      follows that $(a^{n-1} + a^{n-2}b + \cdots + ab^{n-2} + b^{n-1})$ is also
      positive, and since $a < b$, it follows that $a - b$ is negative. We can
      then conclude that
      $$a^n - b^n = (a - b)(a^{n-1} + a^{n-2}b + \cdots + ab^{n-2} + b^{n-1})$$
      is negative; that is, $a^n < b^n$. \qed
%%%%%%%%%%%%%%%%%%%%%%%%%%%%%%%%%%%Prob1.3.10%%%%%%%%%%%%%%%%%%%%%%%%%%%%%%%%%%%
   \item[1.3.10]  For a natural number $n$ and numbers $a$ and $b$ such that
                  $a \ge b \ge 0$, prove that
                  $$a^n - b^n \ge nb^{n - 1}(a - b).$$

      \textbf{Proof.} Consider the inequality
      \begin{equation}
         a^n - b^n \ge nb^{n - 1}(a - b) \label{1_3_10_1}
      \end{equation}
      where $a$ and $b$ are nonnegative real numbers such that $a \ge b \ge 0$, 
      and $n$ is a natural number. It is clear that inequality \eqref{1_3_10_1}
      holds whenever $a = b$ or $a = 0$ or $b = 0$. So assume that $a > b > 0$.
      First we want to prove by induction on $n$ that
      \begin{equation}
         n \le \sum_{j = 0}^{n - 1} \left(\frac{a}{b}\right)^{n - 1 - j}
         \text{ for all natural } n. \label{1_3_10_2}
      \end{equation}

      It is clear that inequality \eqref{1_3_10_2} holds for $n = 1$, so assume
      that it holds for some natural number $k$. We must now show that it also
      holds for $k + 1$. Thus
      \begin{align*}
         \sum_{j = 0}^{k} \left(\frac{a}{b}\right)^{k - j}
            &= 1 + \sum_{j = 0}^{k - 1} \left(\frac{a}{b}\right)^{k - j} \\
            &= 1 + \frac{a}{b}\sum_{j = 0}^{k - 1}
               \left(\frac{a}{b}\right)^{k - j - 1} \\
            &\ge 1 + \frac{a}{b}k &[\text{Inductive Hypothesis}] \\
            &> 1 + k. &\left[a > b > 0, \text{ so that }
                                   \frac{a}{b} > 1\right]
      \end{align*}

      Hence inequality \eqref{1_3_10_2} holds for $k + 1$, so that by the
      Principle of Mathematical Induction, it holds for all natural $n$. Since
      $b$ is positive, it follows that $b^{n-1}$ is also positive, so multiply
      inequality \eqref{1_3_10_2} by the positive number $(a - b)b^{n-1}$ to get
      \begin{align*}
         nb^{n-1}(a-b) &\le (a-b)b^{n-1}\sum_{j = 0}^{n-1}
            \left(\frac{a}{b}\right)^{n-1-j} \\
            &= (a-b)\sum_{j = 0}^{n-1} a^{n-1-j}b^j \\
            &= a^n - b^n. &[\text{Difference Formula}]
      \end{align*}
      \qed
%%%%%%%%%%%%%%%%%%%%%%%%%%%%%%%%%%%Prob1.3.11%%%%%%%%%%%%%%%%%%%%%%%%%%%%%%%%%%%
   \item[1.3.11]  (Bernoulli's Inequality) Show that for a natural number $n$
                  and a nonnegative number $b$,
                  $$(1 + b)^n \ge 1 + nb.$$
                  (\textit{Hint}: In the Binomial Formula, set $a = 1$.)

      \textbf{Proof.} Let $n$ be a natural number and let $b$ be a nonnegative
      number. If $n$ equals 1, then we have that $(1 + b)^n = 1 + nb$, so
      suppose that $n \ge 2$. Then we have by the Binomial Formula that
      $$(1 + b)^n = 1 + nb + \sum_{k=2}^n\binom{n}{k}b^k \ge 1 + nb,$$
      so that $(1 + b)^n \ge 1 + nb$ for all natural number $n$. \qed
%%%%%%%%%%%%%%%%%%%%%%%%%%%%%%%%%%%Prob1.3.12%%%%%%%%%%%%%%%%%%%%%%%%%%%%%%%%%%%
   \item[1.3.12]  Use the Principle of Mathematical Induction to provide a
                  direct proof of Bernoulli's Inequality for all $b > -1$, not
                  just for the case where $b \ge 0$ which, as outlined in
                  Exercise 11 follows from the Binomial Formula.

      \textbf{Proof.} Let $b$ be a real number such that $b > -1$. We want to
      show by induction that
      \begin{equation}
         (1 + b)^n \ge 1 + nb, \label{1_3_12_1}
      \end{equation}
      for all natural number $n$. Inequality \eqref{1_3_12_1} trivially holds
      when $n$ equals 1. So assume that it also holds for some natural number
      $k$. To complete the proof, we have to now show that \eqref{1_3_12_1}
      holds for $k + 1$. By our inductive hypothesis, it follows that
      $(1 + b)^k \ge 1 + kb$. Since $b > -1$, we have that $1 + b > 0$, so that
      \begin{align*}
         (1 + b)^{k + 1} = (1 + b)^k(1 + b) &\ge (1 + kb)(1 + b) \\
                          &= 1 + (k + 1)b + kb^2 \\
                          &\ge 1 + (k + 1)b.
      \end{align*}

      Thus \eqref{1_3_12_1} holds for $k + 1$; and hence, by the Principle of 
      Mathematical Induction, it holds for all natural number $n$. \qed
%%%%%%%%%%%%%%%%%%%%%%%%%%%%%%%%%%%Prob1.3.13%%%%%%%%%%%%%%%%%%%%%%%%%%%%%%%%%%%
   \item[1.3.13]  For a natural number $n$ and a nonnegative number $b$ show
                  that
                  $$(1 + b)^n \ge 1 + nb + \frac{n(n - 1)}{2}b^2.$$

      \textbf{Proof.} Let $n$ be a natural number and let $b$ be a nonnegative
      number. Trivial computations will show us that the equation
      \begin{equation}
         (1 + b)^n \ge 1 + nb + \frac{n(n - 1)}{2}b^2 \label{1_3_13_1}
      \end{equation}
      holds for $n = 1$ and $n = 2$. So assume that $n \ge 3$. By the Binomial
      Theorem, we have that
      $$(1 + b)^n = 1 + nb + \frac{n(n - 1)}{2}b^2 +
         \sum_{k=3}^n\binom{n}{k}b^k \ge 1 + nb + \frac{n(n - 1)}{2}b^2,$$
      so that
      $$(1 + b)^n \ge 1 + nb + \frac{n(n - 1)}{2}b^2$$
      for all natural number $n$. \qed
%%%%%%%%%%%%%%%%%%%%%%%%%%%%%%%%%%%Prob1.3.14%%%%%%%%%%%%%%%%%%%%%%%%%%%%%%%%%%%
   \item[1.3.14]  (Cauchy's Inequality) Using the fact that the square of a real
                  number is nonnegative, prove that for any numbers $a$ and $b$,
                  $$ab \le \frac{1}{2}(a^2 + b^2).$$

      \textbf{Proof.} Let $a$ and $b$ be real numbers. Then it follows that
      $0 \le (a - b)^2 = a^2 + b^2 - 2ab$. That is, $2ab \le a^2 + b^2$, so that
      $$ab \le \frac{1}{2}(a^2 + b^2).$$ \qed
%%%%%%%%%%%%%%%%%%%%%%%%%%%%%%%%%%%Prob1.3.15%%%%%%%%%%%%%%%%%%%%%%%%%%%%%%%%%%%
   \item[1.3.15]  Use Cauchy's Inequality to prove that if $a \ge 0$ and
                  $b \ge 0$, then
                  $$\sqrt{ab} \le \frac{1}{2}(a + b).$$

      \textbf{Proof.} Let $a$ and $b$ be nonnegative real numbers, so that
      $\sqrt{a}$ and $\sqrt{b}$ exist. Applying Cauchy's Inequality(from
      Exercise 1.3.15) to $\sqrt{a}$ and $\sqrt{b}$ will lead us to conclude
      that
      $$\sqrt{ab} \le \frac{1}{2}(a + b).$$ \qed
%%%%%%%%%%%%%%%%%%%%%%%%%%%%%%%%%%%Prob1.3.16%%%%%%%%%%%%%%%%%%%%%%%%%%%%%%%%%%%
   \item[1.3.16]  Use Cauchy's Inequality to show that for any numbers $a$ and
                  $b$ and a natural number $n$,
                  $$ab \le \frac{1}{2}(na^2 + \frac{1}{n}b^2).$$
                  (\textit{Hint:} Replace $a$ by $\sqrt{n}a$ and $b$ by
                   $b/\sqrt{n}$ in Cauchy's Inequality.)

      \textbf{Proof.} Let $a$ and $b$ be real numbers and let $n$ be a natural
      number. Applying Cauchy's Inequality(from Exercise 1.3.15) to $\sqrt{n}a$ 
      and $b/\sqrt{n}$ will lead us to conclude
      $$ab \le \frac{1}{2}(na^2 + \frac{1}{n}b^2).$$ \qed
%%%%%%%%%%%%%%%%%%%%%%%%%%%%%%%%%%%Prob1.3.17%%%%%%%%%%%%%%%%%%%%%%%%%%%%%%%%%%%
   \item[1.3.17]  Let $a$, $b$, and $c$ be nonnegative numbers. Prove the
                  following inequalities:
                  \begin{enumerate}
                     \item $ab + bc + ca \le a^2 + b^2 + c^2$.
                     \item $8abc \le (a + b)(b + c)(c + a)$.
                     \item $abc(a + b + c) \le a^2b^2 + b^2c^2 + c^2a^2$.
                  \end{enumerate}

      \textbf{Proof.}

      \begin{enumerate}
         \item By Exercise 1.3.14, we have that
               \begin{align*}
                  ab &\le \frac{1}{2}(a^2 + b^2), \\
                  bc &\le \frac{1}{2}(b^2 + c^2), \text{ and } \\
                  ca &\le \frac{1}{2}(c^2 + a^2), \text{ so that} \\
                  ab + bc + ca &\le \frac{1}{2}(a^2 + b^2) +
                                    \frac{1}{2}(b^2 + c^2) +
                                    \frac{1}{2}(c^2 + a^2) \\
                               &= a^2 + b^2 + c^2.
               \end{align*}
         \item Since $a$, $b$, and $c$ are nonnegative, it follows that
               $c(a - b)^2 \ge 0$, $a(b - c)^2 \ge 0$, and $b(a - c)^2 \ge 0$,
               so that
               \begin{align*}
                  a^2c + b^2c &\ge 2abc, \\
                  ab^2 + ac^2 &\ge 2abc, \text{ and } \\
                  a^2b + bc^2 &\ge 2abc. \\
               \end{align*}
               Thus
               \begin{align*}
                  (a + b)(b + c)(c + a) &= 2abc + (a^2c + b^2c) +
                                          (ab^2 + ac^2) + (a^2b + bc^2) \\
                                        &\ge 2abc + 2abc + 2abc + 2abc \\
                                        &= 8abc.
               \end{align*}
         \item Since $a$, $b$, and $c$ are nonnegative, it follows that $ab$,
               $bc$, and $ac$ are nonnegative. Hence
               \begin{align*}
                  abc(a + b + c) &= (ab)(bc) + (bc)(ac) + (ac)(ab) \\
                     &\le (ab)^2 + (bc)^2 + (ac)^2 &[\text{By part (a)}] \\
                     &= a^2b^2 + b^2c^2 + a^2c^2.
               \end{align*}
                 
      \end{enumerate} \qed
%%%%%%%%%%%%%%%%%%%%%%%%%%%%%%%%%%%Prob1.3.18%%%%%%%%%%%%%%%%%%%%%%%%%%%%%%%%%%%
   \item[1.3.18]  A function $f: \R \rightarrow \R$ is called
                  \textit{strictly increasing} provided that $f(u) > f(v)$ for
                  all numbers $u$ and $v$ such that $u > v$.
                  \begin{enumerate}
                     \item Define $p(x) = x^3$ for all $x$. Prove that the
                           polynomial $p : \R \rightarrow \R$ is strictly
                           increasing.
                     \item Fix a number $c$ and define $q(x) = x^3 + cx$ for all
                           $x$. Prove that the polynomial
                           $q : \R \rightarrow \R$ is strictly increasing if and
                           only if $c \ge 0$. (\textit{Hint}: For $c < 0$,
                           consider the graph to understand why it is not
                           strictly increasing and then prove it is not
                           increasing.)
                  \end{enumerate}

      \textbf{Proof.}

      \begin{enumerate}
         \item Let $u$ and $v$ be real numbers such that $u > v$. By using the
               Difference Formula and then completing the square, we get
               $$p(u) - p(v) = u^3 - v^3 = (u - v)(u^2 + uv + v^2) =
                 (u - v)\left[\left(u + \frac{v}{2}\right)^2 +
                 \frac{3v^2}{4}\right],$$
               so that $p(u) - p(v)$ is positive. That is $p(u) > p(v)$, so that
               $p$ is strictly increasing.
         \item $(\Leftarrow)$ Suppose that $c \ge 0$, and let $u$ and $v$ be
               real numbers such that $u > v$.  By using the Difference Formula 
               and then completing the square, we get
               $$q(u) - q(v) = (u - v)(u^2 + uv + v^2 + c) =
                 (u - v)\left[\left(u + \frac{v}{2}\right)^2 +
                 \frac{3v^2 + 4c}{4}\right],$$
               so that $q(u) - q(v)$ is positive; that is, $q$ is strictly
               increasing.

               $(\Rightarrow)$ Conversely suppose that $q$ is strictly
               increasing. Now suppose to the contrary that $c < 0$. Then it
               follows that $\sqrt{-c} > -\sqrt{-c}$. However
               $q(\sqrt{-c}) \not> q(-\sqrt{-c})$ because
               $q(\sqrt{-c}) = q(-\sqrt{-c}) = 0$, a contradiction since $q$ is
               strictly increasing. Thus we must have that $c \ge 0$. \qed
      \end{enumerate}
%%%%%%%%%%%%%%%%%%%%%%%%%%%%%%%%%%%Prob1.3.19%%%%%%%%%%%%%%%%%%%%%%%%%%%%%%%%%%%
   \item[1.3.19]  Let $n$ be a natural number and $a_1$, $a_2$, $\ldots$, $a_n$
                  be positive numbers. Prove that
                  $$(1 + a_1)(1 + a_2)\cdots(1 + a_n) \ge
                     1 + a_1 + a_2 + \cdots + a_n$$
                  and that
                  $$(a_1 + a_2 + \cdots + a_n)({a_1}^{-1} + {a_2}^{-1} + \cdots
                     + {a_n}^{-1}) \ge n^2.$$

      \textbf{Proof.} We want to first show that
      \begin{equation}
         (1 + a_1)(1 + a_2)\cdots(1 + a_n) \ge
          1 + a_1 + a_2 + \cdots + a_n, \label{1_3_19_1}
      \end{equation}
      by inducting on $n$. It is clear that \eqref{1_3_19_1} holds for $n = 1$,
      so assume that it also holds for some positive integer $k$. Letting
      $m = a_{k + 1}(a_1 + a_2 + \cdots + a_k)$ and using our inductive 
      hypothesis, we have
      \begin{align*}
         (1 + a_1)(1 + a_2)\cdots(1 + a_k)(1 + a_{k + 1})
            &\ge (1 + a_1 + a_2 + \cdots + a_k)(1 + a_{k + 1}) \\
            &= (1 + a_1 + a_2 + \cdots + a_k + a_{k + 1}) + m \\
            &> (1 + a_1 + a_2 + \cdots + a_k + a_{k + 1}), 
      \end{align*}
      so that \eqref{1_3_19_1} holds for $k + 1$, and hence, by the Principle of
      Mathematical Induction, it holds for all natural $n$.

      Now we want show that
      \begin{equation}
         \left(\sum_{i = 1}^ka_i\right)
         \left(\sum_{j = 1}^k{a_j}^{-1}\right) \ge n^2, \label{1_3_19_2}
      \end{equation}
      by induction on $n$. Again it is clear that \eqref{1_3_19_2} holds for
      $n = 1$. So assume that it also holds for some natural number $k$. Then
      we have that
      \begin{align*}
         \left(a_{k+1} + \sum_{i = 1}^ka_i\right)
         \left({a_{k+1}}^{-1} + \sum_{j = 1}^k{a_j}^{-1}\right)
            &= 1 + \sum_{j = 1}^k\left(\frac{a_j}{a_{k+1}} +
               \frac{a_{k+1}}{a_j}\right) + \sum_{i = 1}^ka_i
               \sum_{j = 1}^k{a_j}^{-1} \\
            &\ge 1 + 2k + \sum_{i = 1}^ka_i
               \sum_{j = 1}^k{a_j}^{-1} &[\text{Lemma 1.3.1}] \\
            &\ge 1 + 2k + k^2 &[\text{Inductive hypothesis}] \\
            &= (k + 1)^2,
      \end{align*}
      so that \eqref{1_3_19_2} holds for $k + 1$, and hence, by the Principle of
      Mathematical Induction, it holds for all natural $n$. \qed
%%%%%%%%%%%%%%%%%%%%%%%%%%%%%%%%%%%Prob1.3.20%%%%%%%%%%%%%%%%%%%%%%%%%%%%%%%%%%%
   \item[1.3.20]  Use the Geometric Sum Formula to find a formula for
                  \begin{enumerate}
                     \item $\displaystyle\frac{1}{1 + x^2} +
                            \frac{1}{(1 + x^2)^2} + \cdots +
                            \frac{1}{(1 + x^2)^n}$.
                     \item Also, show that if $a \neq 0$, then
                           $$\frac{1}{a} = 1 + (1 - a) + (1 - a)^2 +
                             \frac{(1 - a)^3}{a}.$$
                  \end{enumerate}

      \textbf{Proof.}

      \begin{enumerate}
         \item Let $x$ be a real number and let $n$ be a positive integer. The
               sum
               $$\frac{1}{1 + x^2} + \frac{1}{(1 + x^2)^2} + \cdots +
                 \frac{1}{(1 + x^2)^n}$$
               is $n$ if $x = 0$, so assume $x \neq 0$. Then we have that
               \begin{align*}
                  \frac{1}{1 + x^2} + \frac{1}{(1 + x^2)^2} + \cdots +
                  \frac{1}{(1 + x^2)^n}
                     &= \frac{(1 + x^2)^{n-1}}{(1 + x^2)^n} +
                        \frac{(1 + x^2)^{n-2}}{(1 + x^2)^n} + \cdots +
                        \frac{1}{(1 + x^2)^n} \\
                     &= \frac{(1 + x^2)^{n-1} + (1 + x^2)^{n-2} + \cdots + 1}
                             {(1 + x^2)^n} \\
                     &= \frac{x^2[(1 + x^2)^{n-1} + (1 + x^2)^{n-2} + \cdots+1]}
                             {x^2(1 + x^2)^n} \\
                     &= \frac{((1 + x^2) - 1)[(1 + x^2)^{n-1} +
                              (1 + x^2)^{n-2} + \cdots+1]}{x^2(1 + x^2)^n} \\
                     &= \frac{(1 + x^2)^n - 1^n}{x^2(1 + x^2)^n} \\
                     &= \frac{(1 + x^2)^n - 1}{x^2(1 + x^2)^n}.
               \end{align*}
         \item Let $a$ be a nonzero real number. Thus
               \begin{equation}
                  1 = a + (1 - a), \label{1_3_20_1}
               \end{equation}
               so that multiplying both sides of \eqref{1_3_20_1} by $(1 - a)$
               will give us
               $$1-a = (1-a)(a + (1 - a)).$$
               That is,
               \begin{equation}
                  1 = a + (1-a)(a + (1 - a)), \label{1_3_20_2}
               \end{equation}
               so that multiplying both sides of \eqref{1_3_20_2} by $(1 - a)$
               will give us
               $$1-a = (1-a)(a + (1-a)(a + (1 - a))).$$

               That is,
               $$1 = a + a(1 - a) + a(1 - a)^2 + (1 - a)^3.$$
               Divide the last equation above to get the desired result. \qed
      \end{enumerate}
%%%%%%%%%%%%%%%%%%%%%%%%%%%%%%%%%%%Prob1.3.21%%%%%%%%%%%%%%%%%%%%%%%%%%%%%%%%%%%
   \item[1.3.21]  Prove that if $n$ and $k$ are natural numbers such that
                  $k \le n$, then
                  $$\binom{n + 1}{k} = \binom{n}{k - 1} + \binom{n}{k}.$$

      \textbf{Proof.} Let $n$ and $k$ be natural numbers with $k \le n$. Then
      it follows by definition that
      \begin{align*}
         \binom{n}{k - 1} + \binom{n}{k} &= \frac{n!}{(n-k+1)!(k-1)!} +
            \frac{n!}{(n-k)!k!} \\
            &= \frac{kn! + (n - k + 1)n!}{(n - k + 1)!k!} \\
            &= \frac{(k + n - k + 1)n!}{(n + 1 - k)!k!} \\
            &= \frac{(n + 1)n!}{(n + 1 - k)!k!} \\
            &= \frac{(n + 1)!}{(n + 1 - k)!k!} \\
            &= \binom{n + 1}{k}.
      \end{align*} \qed
%%%%%%%%%%%%%%%%%%%%%%%%%%%%%%%%%%%Prob1.3.22%%%%%%%%%%%%%%%%%%%%%%%%%%%%%%%%%%%
   \item[1.3.22]  Use the formula in Exercise 21 to provide an inductive proof
                  of the Binomial Formula.

      \textbf{Proof.} Let $a$ and $b$ be real numbers and let $n$ be a natural
      number. The Binomial Formula says that
      $$(a + b)^n = \sum_{i = 0}^n\binom{n}{i}a^{n-i}b^i.$$
      We want to show by induction on $n$ that the Binomial Formula holds for
      each natural number $n$. So
      $$(a+b)^1 = a^1b^0 + a^0b^1 = \binom{1}{0}a^1b^0 + \binom{1}{1}a^0b^1.$$
      That is, the Binomial Formula holds whenever $n$ is 1. Now assume that it
      holds for some natural number $k$. To complete the proof, we have to now
      show that it must also hold for $k + 1$. Hence
      \begin{align*}
         \sum_{i = 0}^{k+1}\binom{k+1}{i}a^{k+1-i}b^i
            &= b^{k + 1} + \sum_{i = 0}^k\binom{k+1}{i}a^{k+1-i}b^i \\
            &= b^{k + 1} + \sum_{i = 0}^k
                  \left[\binom{k}{i-1}+\binom{k}{i}\right]a^{k+1-i}b^i
                  &[\text{Exercise 1.3.21}] \\
            &= b^{k + 1} + \sum_{i = 0}^k\binom{k}{i-1}a^{k+1-i}b^i +
               \sum_{i=0}^k\binom{k}{i}a^{k+1-i}b^i \\
            &= b^{k + 1} + \sum_{i = 1}^k\binom{k}{i-1}a^{k+1-i}b^i +
               a\sum_{i=0}^k\binom{k}{i}a^{k-i}b^i \\
            &= b^{k + 1} + b\sum_{j = 0}^{k-1}\binom{k}{j}a^{k-j}b^j +
               a\sum_{i=0}^k\binom{k}{i}a^{k-i}b^i \\
            &= b^{k + 1} + b\left(-b^k + \sum_{j = 0}^k
               \binom{k}{j}a^{k-j}b^j\right) + a\sum_{i=0}^k
               \binom{k}{i}a^{k-i}b^i \\
            &= b^{k + 1} + b\left(-b^k + (a+b)^k\right) + a(a+b)^k
                  &[\text{Inductive hypothesis}] \\
            &= (a+b)^{k+1},
      \end{align*}      
      so that the Binomial Formula holds for $k + 1$, and hence, by the
      Principle of Mathematical Induction, it holds for all natural $n$. \qed
%%%%%%%%%%%%%%%%%%%%%%%%%%%%%%%%%%%Prob1.3.23%%%%%%%%%%%%%%%%%%%%%%%%%%%%%%%%%%%
   \item[1.3.23]  Let $a$ and $b$ be nonzero numbers and $m$ and $n$ be
                  integers. Prove
                  the following equalities:
                  \begin{enumerate}
                     \item $a^{m + n} = a^ma^n$.
                     \item $(ab)^n = a^nb^n$.
                  \end{enumerate}

      \textbf{Proof.}

      \begin{enumerate}
         \item We want to show that, for every integer $m$ and $n$, we have that
               \begin{equation}
                  a^{m + n} = a^ma^n. \label{1_3_23_1}
               \end{equation}
               Assume first that $n$ is positive. Now let's induct on $n$. By
               Lemma 1.3.2, we see that \eqref{1_3_23_1} holds for $n = 1$, so 
               suppose that it holds for some natural number $k$. Now we must 
               show that it also holds for $k + 1$. Hence
               \begin{align*}
                  a^ma^{k+1} &= a^ma^ka^1 &[\text{Lemma 1.3.2}] \\
                             &= (a^ma^k)a^1 \\
                             &= a^{m+k}a^1 &[\text{Inductive hypothesis}] \\
                             &= a^{(m+k)+1} &[\text{Lemma 1.3.2}] \\
                             &= a^{m+(k+1)}, &[\text{Associativity of addition}]
               \end{align*}
               so that \eqref{1_3_23_1} holds for $k + 1$, and hence, by the 
               Principle of Mathematical Induction, it holds for all natural
               $n$. \\
         
               If $m$ is 0 or $n$ is 0, then Lemma 1.3.2 tells us that
               \eqref{1_3_23_1} holds, so the only remaining possibility is $m$ 
               and $n$ are negative.\footnote{If $m$ is positive and $n$ is
               negative, then interchange the roles of $m$ and $n$ in the 
               induction proof.} Now suppose that $m$ and $n$ are negative.
               Hence
               \begin{align*}
                  a^ma^n &= a^{-(-m)}a^{-(-n)} \\
                     &= \underbrace{a^{-1}\cdots a^{-1}}_{-m \text{ factors}}
                        \underbrace{a^{-1}\cdots a^{-1}}_{-n \text{ factors}} \\
                     &= \underbrace{a^{-1}\cdots a^{-1}}_{-m-n\text{ factors}}\\
                     &= a^{-(-m-n)} = a^{m+n}.
               \end{align*}

               We have thus shown that \eqref{1_3_23_1} holds for all integers
               $m$ and $n$.
         \item  We want to how that, for every intger $n$, we have that
               \begin{equation}
                  (ab)^n = a^nb^n. \label{1_3_23_2}
               \end{equation}

               Clearly \eqref{1_3_23_2} holds for $n = 0$, so let us first
               suppose that $n$ is a natural number. Now let us induct on $n$.
               It is clear that \eqref{1_3_23_2} holds for $n = 1$, so assume
               that it also holds for some positive integer $k$. Thus
               \begin{align*}
                  (ab)^{k+1} &= (ab)(ab)^k \\
                     &= (ab)a^kb^k &[\text{Inductive hypothesis}] \\
                     &= a^{k+1}b^{k+1}, &[\text{Multiplication is commutative}]
               \end{align*}
               so that \eqref{1_3_23_2} holds for $k + 1$, and thus, by the
               Principle of Mathematical Induction, holds for every positive
               integer $n$. Now suppose that $n$ is negative. Then it follows
               that $n = -m$, where $m$ is a positive integer. Thus
               \begin{align*}
                  (ab)^n &= (ab)^{-(-n)} \\
                     &= ((ab)^{-1})^m \\
                     &= (a^{-1}b^{-1})^m &[\text{Lemma 1.3.3}] \\
                     &= \underbrace{(a^{-1}b^{-1})\cdots (a^{-1}b^{-1})}_{
                        m \text{ factors}} \\
                     &= \underbrace{a^{-1}\cdots a^{-1}}_{m\text{ factors}}\cdot
                        \underbrace{b^{-1}\cdots b^{-1}}_{m\text{ factors}} \\
                     &= (a^{-1})^m(b^{-1})^m \\
                     &= a^{-m}b^{-m} \\
                     &= a^nb^n.
               \end{align*}
      \end{enumerate}
%%%%%%%%%%%%%%%%%%%%%%%%%%%%%%%%%%%Prob1.3.24%%%%%%%%%%%%%%%%%%%%%%%%%%%%%%%%%%%
   \item[1.3.24]  A natural number $n$ is \textit{even} if it can be written as
                  $n = 2k$ for some other natural number $k$, and is called
                  \textit{odd} if either $n = 1$ or $n = 2k + 1$ for some other
                  natural number $k$.
                  \begin{enumerate}
                     \item Prove that each natural number $n$ is either odd or
                           even.
                     \item Prove that if $m$ is a natural number, then $2m > 1$.
                     \item Prove that a natural number $n$ cannot be both odd
                           and even. (\textit{Hint}: Use part (b).)
                     \item Suppose that $k_1$, $k_2$, $l_1$, and $l_2$ are
                           natural numbers such that $l_1$ and $l_2$ are odd.
                           Prove that if $2^{k_1}l_1 = 2^{k_2}l_2$, then
                           $k_1 = k_2$ and $l_1 = l_2$.
                  \end{enumerate}

      \textbf{Proof.}

      \begin{enumerate}
         \item We proceed by induction on $n$. By definition $n$ is odd whenever
               it equals 1. So assume that some positive integer $k$ is either
               odd or even. If $k$ is even, then we have $k = 2i$ for some
               natural number $i$, so that $k + 1 = 2i + 1$; that is, $k + 1$ is
               odd. However if $k$ is odd, then we must have two choices. First,
               $k$ is 1, so that $k + 1 = 2$; that is $k + 1$ is even. Second,
               $k = 2j + 1$ for some natural number $j$, so that
               $k + 1 = 2(j+1)$; that is $k + 1$ is even. In any case, we have
               shown that $k + 1$ is either odd or even, so that by the
               Principle of Mathematical Induction every natural number $n$ is
               either even or odd. \qed
         \item Let $m$ be a natural number. Then we have that $m \ge 1$, so that
               $2m \ge 2 > 1$. \qed
         \item Let $n$ be a natural number. Suppose to the contrary that $n$ is
               both even and odd. Then since $n$ is even, it follows that
               $n = 2k$ for some natural number $k$. We also know that $n$ is
               odd, but it cannot be 1 since (b) says that $n = 2k > 1$. Thus
               the only remaining possibility is that $n = 2i + 1$ for some
               natural number $i$. Thus we can conclude that $n = 2k = 2i + 1$,
               so that $2(k - i) = 1$; that is $k > i$, so that $k - i$ is a
               natural number, but this contradicts (b). Thus $n$ cannot be both
               even and odd. \qed
         \item Suppose to the contrary that $k_1 \neq k_2$. We can assume 
               without loss of generality that $k_1 < k_2$, so that
               $k_1 + m = k_2$ for some natural number $m$. Thus since
               $2^{k_1}l_1 = 2^{k_2}l_2$, we must have that
               $2^{k_1}l_1 = 2^{k_1}2^ml_2$, so that $l_1 = 2^ml_2$.; that is, 
               $l_1 = 2((2^{m-1})l_2)$. This says that $l_1$ is even, a
               contradiction. We can then conclude that $k_1 = k_2$, and by
               dividing the equation $2^{k_1}l_1 = 2^{k_2}l_2$ by $2^{k_1}$, we
               shall get $l_1 = l_2$. \qed
      \end{enumerate}
%%%%%%%%%%%%%%%%%%%%%%%%%%%%%%%%%%%Prob1.3.25%%%%%%%%%%%%%%%%%%%%%%%%%%%%%%%%%%%
   \item[1.3.25]  \begin{enumerate}
                     \item Prove that if $n$ is a natural number, then
                           $2^n > n$.
                     \item Prove that if $n$ is a natural number, then
                           $$n = 2^{k_0}l_0$$
                           for some odd natural number $l_0$ and some
                           nonnegative integer $k_0$. (\textit{Hint}: If $n$ is
                           odd, let $k = 0$ and $l = n$; if $n$ is even, let
                           $A = \{k \in \N : n = 2^kl$ for some $l \in \N\}$.
                           By (a), $A \subseteq \{1, 2, \ldots, n\}$. Choose
                           $k_0$ to be the maximum of $A$.)
                  \end{enumerate}

      \textbf{Proof.}

      \begin{enumerate}
         \item We shall proceed by induction on $n$. Since $2^1 > 1$, it follows
               that our assertion holds for $n = 1$, so assume that it holds for
               some positive integer $k$. Thus we must have that
               \begin{align*}
                  2^{k+1} = 2 \cdot 2^k &> 2 \cdot k
                     &[\text{Inductive hypothesis}] \\
                     &= k + k \\
                     &\ge k + 1,
               \end{align*} 
               so that the our assertion holds for $k + 1$, and hence, by the
               Principle of Mathematical Induction, it holds for all natural
               $n$. \qed
         \item Let $n$ be a natural number. If $n$ is odd, then we have
               $$n = 2^0n.$$
               Now suppose $n$ is even. Thus we have that $n = 2i$ for some
               natural number $i$. Let
               $A = \{k \in \N : n = 2^kl$ for some $l \in \N\}$. The set $A$
               is not empty because $n = 2^1i$, so that $1 \in A$. So let
               $k \in A$. Thus $n = 2^kl$ for some natural number $n$. Thus we
               have by (a) that $n = 2^kl \ge 2^k > k$, so that $A$ is bounded
               above by $n$. Hence, by Proposition 1.7, $A$ has a maximum, which
               we shall denote by $k'$. By definition of $A$, there exists a
               natural number $l'$ such that $n = 2^{k'}l'$. If $l'$ were even
               then we would have $l' = 2j$ for some natural number $j$, so that
               $n = 2^{k'+1}j$. That is, $k' + 1 \in A$, a contradiction, since
               $k'$ is the maximum element of $A$. We can now conclude that $l'$
               is odd. \qed
      \end{enumerate}
%%%%%%%%%%%%%%%%%%%%%%%%%%%%%%%%%%%Prob1.3.26%%%%%%%%%%%%%%%%%%%%%%%%%%%%%%%%%%%
   \item[1.3.26]  Prove that Exercises 24 and 25 are sufficient to prove the
                  assertions (i) and (ii) about the natural numbers that
                  preceded the proof of the irrationality of $\sqrt{2}$.

      \textbf{Proof.}

      \begin{enumerate}
         \item[(i)]  Let $x = m/n$ be a positive rational number, where $m$ and
                     $n$ are positive integers. By Exercise 1.3.25(b) we have
                     that
                     $$m = 2^{k_1}l_1 \text{ and } n = 2^{k_2}l_2$$
                     such that $k_1$ and $k_2$ are nonnegative integers and
                     $l_1$ and $l_2$ are odd natural numbers. If $k_1 = k_2$,
                     then we have that $x = l_1/l_2$. If $k_1 < k_2$, then
                     $\displaystyle x = \frac{l_1}{2^{k_2-k_1}l_2}$. Finally if
                     $k_1 > k_2$, then
                     $\displaystyle x = \frac{2^{k_1-k_2}l_1}{l_2}$. In any
                     case, we have shown that assertion (i) is true for positive
                     rational numbers. Now if $x$ is 0, then we can write
                     $x = 0/1$. The only remaining possiblity is a negative $x$.
                     We can write a negative rational number $x$ as $-y$ where
                     $y$ is a positive rational number. We previously showed
                     that $y$ can be written as a fraction $r/s$ where $r$ or
                     $s$ is odd. If $r$ is odd, then write $x = r/-s$. If $s$ is
                     odd, then write $x = -r/s$. \qed
         \item[(ii)] Let $n$ be a natural number such that $n^2$ is even. Now 
                     suppose to the contrary that $n$ is odd. Clearly $n$ cannot
                     be 1, since $1^2 = 1$ is odd. So $n = 2k + 1$ for some
                     natural number $k$. Then we have that
                     $n^2 = 4k^2 + 4k + 1 = 2(2k^2 + 2k) + 1$; that is $n^2$ is
                     odd, a contradiction by Exercise 1.3.24(c). Thus $n$ must
                     be even. \qed
      \end{enumerate}
%%%%%%%%%%%%%%%%%%%%%%%%%%%%%%%%%%%Prob1.3.27%%%%%%%%%%%%%%%%%%%%%%%%%%%%%%%%%%%
   \item[1.3.27]  A real number of the form $m/2^n$, where $m$ and $n$ are
                  integers, is called a \textit{dyadic rational}. Prove that the
                  set of dyadic rationals is dense in $\R$.

      \textbf{Proof.} Let $(a, b)$, with $a < b$, be an interval in $\R$. It
      suffices to show that there exists a $c \in (a, b)$, such that $c = m/2^n$
      for some integers $m$ and $n$. By the Archimedean Principle, there exists
      an $n$ such that $(1/n) < b - a$. Thus by Exercise 1.3.25, we have that
      $(1/2^n) < (1/n) < b - a$, so that $2^na < 2^nb - 1$. By Theorem 1.8,
      there exists an integer $m$ such that $2^na < 2^b - 1 \le m < 2^nb$, so
      that $a < m/2^n < b$. \qed
      
\end{enumerate}
