\begin{enumerate}
   \item[]        \textbf{Lemma 1.1.1} Let $a$ and $b$ positive real numbers and
                  let $n$ be a natural number. If $a^n < b^n$, then we must have 
                  that $a < b$.
   
      \textbf{Proof.} Lemma 1.1.1 trivially holds if $n$ is 1. So assume that
      $n > 1$. The inequality $a^n < b^n$ holds if and only if $a^n - b^n < 0$. 
      Thus
      $$a^n - b^n = (a - b)(a^{n-1} + a^{n-2}b + \cdots + ab^{n-2} + b^{n-1})
                  < 0.$$
      Since $a$ and $b$ are positive, we must have that
      $a^{n-1} + a^{n-2}b + \cdots + ab^{n-2} + b^{n-1}$ is also
      positive. Now if $a - b$ were positive, then $a^n- b^n$ would be positive,
      a contradiction; hence $a - b$ must be negative, so that $a < b$. \qed

      If $a = 0$, then the inequality trivially holds.

   \item[]        \textbf{Lemma 1.1.2} Let $a$ and $b$ be real numbers. Then we
                  have the following:
                  \begin{enumerate}
                     \item $-b - a = -(a + b)$.
                     \item $ab = -(a(-b))$.
                     \item $-((-a)b) = ab$.
                     \item $(-a)(-b) = ab$.
                  \end{enumerate}
   
      \textbf{Proof.} 
      \begin{enumerate}
         \item By definition, we have that $-b - a = (-b) + (-a)$. So
               \begin{align*}
                  (-b - a) + (a + b) &= (-b) + (-a) + a + b \\
                                 &= (-b) + b + (-a) + a \\
                                 &= 0.
               \end{align*}
               Since $(-b - a) + (a + b) = 0$, it follows that
               $-b - a = -(a + b)$.
         \item We have
               \begin{align*}
                  ab + a(-b) &= a(b + (-b)) \\
                             &= a \cdot 0 = 0.
               \end{align*}
               Since $ab + a(-b) = 0$, it follows that $ab = -(a(-b))$.
         \item We have
               \begin{align*}
                  ab + (-a)b &= (a + (-a))b \\
                             &= 0 \cdot b = 0.
               \end{align*}
               Since $ab + (-a)b = 0$, it follows that $-((-a)b) = ab$.
         \item We have
               \begin{align*}
                  (-a)(-b) + (-a)b &= (-a)((-b) + b) \\
                                   &= (-a) \cdot 0 = 0.
               \end{align*}
               Since $(-a)(-b) + (-a)b = 0$, it follows that
               $(-a)(-b) = -((-a)b) = ab$.
      \end{enumerate} \qed
%%%%%%%%%%%%%%%%%%%%%%%%%%%%%%%%%%%Prob1.1_1%%%%%%%%%%%%%%%%%%%%%%%%%%%%%%%%%%%%
   \item[1.1.1]   For each of the following statements, determine whether it is 
                  true or false and justify your answer.
                  \begin{enumerate}
                     \item The set of irrational numbers is inductive.
                     \item The set of squares of rational numbers is inductive.
                     \item The sum of irrational numbers is irrational.
                     \item The product of irrational numbers is irrational.
                     \item If $n$ is a natural number and $n^2$ is odd, then $n$
                           is odd.
                  \end{enumerate}

      \textbf{Solution.} 

      \begin{enumerate}
         \item False. Since 1 = 1/1 is rational, it is not in the set of 
               irrational numbers.
         \item False. Although $1 = 1^2/1^2$ is in the set of squares of 
               rational numbers, $1 + 1 = 2$ is not, since 2 is irrational.
         \item False. $\sqrt{2}$ and $-\sqrt{2}$ are irrational numbers, but
               their sum, 0, is a rational number.
         \item False. $\sqrt{2}$ and $-\sqrt{2}$ are irrational numbers, but
               their product, $-2$, is a rational number.
         \item True. The contrapositive of this implication says that if a
               natural number $n$ is even, then $n^2$ is even. So suppose that
               $n = 2k$ for some integer $k$, then $n^2 = 2(2k^2 )$, so that
               $n^2$ is even.
      \end{enumerate}

%%%%%%%%%%%%%%%%%%%%%%%%%%%%%%%%%%Prob1.1_2%%%%%%%%%%%%%%%%%%%%%%%%%%%%%%%%%%%%%
   \item[1.1.2]   For each of the following statements, determine whether it is 
                  true or false and justify your answer.
                  \begin{enumerate}
                     \item Every nonempty set of real numbers that is bounded 
                           above has a largest member.
                     \item If $S$ is a nonempty set of positive real numbers, 
                           then $0 \le \inf S$.
                     \item If $S$ is a set of real numbers that is bounded above
                           and $B$ is a nonempty subset of $S$, then
                           $\sup B \le \sup S$.
                  \end{enumerate}

      \textbf{Solution.}

      \begin{enumerate}
         \item False. The set $[0, 1)$ is bounded above by 1 but has no maximum.
         \item True. Since $S$ is a set of positive real numbers, it follows 
               that it is bounded below by $0$. Thus, $0 \le \inf S$.
         \item True. Let $b \in B$. Since $B \subseteq S$, we have that $b$ is
               also in $S$, so that $b \le \sup S$. Thus $B$ is bounded above
               by $\sup S$. By the Completeness Axiom, we must have that
               $\sup B \le \sup S$.
      \end{enumerate}
%%%%%%%%%%%%%%%%%%%%%%%%%%%%%%%%%%Prob1.1_3%%%%%%%%%%%%%%%%%%%%%%%%%%%%%%%%%%%%%
   \item[1.1.3]   Use the Principle of Mathematical Induction to prove that for 
                  a natural number $n$,

                  \begin{equation}
                     \sum_{j=1}^nj^2 = \frac{n(n+1)(2n+1)}{6}.\label{1_1_3_1}
                  \end{equation}

      \textbf{Proof.} We shall prove that \eqref{1_1_3_1} holds by Mathematical 
      Induction. Let $S(n)$ be the statement that \eqref{1_1_3_1} holds. It is 
      evident that $S(1)$ holds. So suppose that for some natural number $k$ 
      that $S(k)$ is true.

      Now
      \begin{align*}
         S(k+1) &= \sum_{j=1}^{k+1}j^2 \\
                &= (k + 1)^2 + \sum_{j=1}^kj^2 \\
                &= (k + 1)^2 + \frac{k(k+1)(2k+1)}{6} \qquad
                   \text{[Inductive Hypothesis]}\\
                &= \frac{6(k + 1)^2 + k(k+1)(2k+1)}{6} \\
                &= \frac{(k + 1)(6(k + 1) + k(2k+1))}{6} \\
                &= \frac{(k + 1)(2k^2 + 7k + 6)}{6} \\
                &= \frac{(k + 1)((k + 1) + 1)(2(k + 1) + 1)}{6}.
      \end{align*}

      That is, $S(k + 1)$ holds. Thus $S(n)$ holds for all natural numbers. \qed
%%%%%%%%%%%%%%%%%%%%%%%%%%%%%%%%%%Prob1.1_4%%%%%%%%%%%%%%%%%%%%%%%%%%%%%%%%%%%%%
   \item[1.1.4]   Let $n$ be a natural number. Find a formula for
                  $\sum_{j = 1}^n j(j + 1)$.

      \textbf{Proof.} Claim that
      \begin{equation}
         \sum_{j=1}^nj(j + 1) = \frac{n(n+1)(n+2)}{3}.\label{1_1_4_1}
      \end{equation}

      We shall prove Equation \eqref{1_1_4_1} using Mathematical Induction. It
      is easy to see that Equation \eqref{1_1_4_1} holds when $n$ is 1. So
      suppose that it also holds for some natural number $k$. Thus

      \begin{align*}
         \sum_{j=1}^{k+1}j(j + 1) &= (k + 1)(k + 2) + \sum_{j=1}^kj(j + 2) \\
                &= (k + 1)(k + 2) + \frac{k(k+1)(k+2)}{3}  \qquad
                   \text{[Inductive Hypothesis]} \\
                &= \frac{3(k + 1)(k + 2) + k(k+1)(k+2)}{3} \\
                &= \frac{(k + 1)(3(k + 2) + k(k+2))}{3} \\
                &= \frac{(k + 1)(k + 2)(k + 3)}{3} \\
                &= \frac{(k + 1)((k + 1)+1)((k + 1)+2)}{3}, \\
      \end{align*}

      so that Equation \eqref{1_1_4_1} holds for $k + 1$. Thus, it holds for all
      natural numbers. \qed      
%%%%%%%%%%%%%%%%%%%%%%%%%%%%%%%%%%Prob1.1_5%%%%%%%%%%%%%%%%%%%%%%%%%%%%%%%%%%%%%
   \item[1.1.5]   Let $n$ be a natural number. Prove that
                  \begin{equation}
                     1^3 + 2^3 + \cdots + n^3 = (1 + 2 + \cdots + n)^2.
                     \label{1_1_5_1}
                  \end{equation}

      \textbf{Proof.} Equation \eqref{1_1_5_1} trivially holds for when $n$ is
      1, so assume it holds for some natural number $k$. Then we have that

      \begin{align*}
         1^3 + 2^3 + \cdots + (k + 1)^3 &=
         1^3 + 2^3 + \cdots + k^3 + (k + 1)^3 \\
         &= (1 + 2 + \cdots + k)^2 + (k+1)^3  \qquad
            &\text{[Inductive Hypothesis]} \\
         &= \frac{k^2(k+1)^2}{4} + (k+1)^3 \qquad &\text{(Example 1.1)} \\
         &= \frac{k^2(k+1)^2+4(k+1)^3}{4} \\
         &= \frac{(k+1)^2(k^2 + 4k+4)}{4} \\
         &= \left(\frac{(k+1)((k+1)+1)}{2}\right)^2 \\
         &= (1 + 2 + \cdots + (k+1))^2, \qquad &\text{(Example 1.1)}
      \end{align*}

      so that Equation \eqref{1_1_5_1} holds for $k + 1$. Thus, it holds for all
      natural numbers. \qed
%%%%%%%%%%%%%%%%%%%%%%%%%%%%%%%%%%Prob1.1_6%%%%%%%%%%%%%%%%%%%%%%%%%%%%%%%%%%%%%
   \item[1.1.6]   Let $m$ and $n$ be natural numbers.
                  \begin{enumerate}
                     \item Prove that the sum, $m + n$, also is a natural 
                           number.
                     \item Prove that the product, $mn$, also is a natural 
                           number.
                  \end{enumerate}

      \textbf{Proof.} We shall prove both parts by Mathematical Induction on
      $n$.

      \begin{enumerate}
         \item Since the set of Natural numbers is an inductive set and since
               $m$ is a natural, it must be the case that $m + 1$ is a natural
               number. So assume that for some natural number $k$ that $m + k$ 
               is also a natural number. As we argued above, $(m + k) + 1$
               must also be a natural number. Thus $m + n$ is a natural number
               for all natural numbers $n$. \qed
         \item We have $m \cdot 1 = m$, a natural number by assumption. So 
               assume that for some natural number $k$ that $mk$ is a natural 
               number. Then $m(k+1) = mk + m$. Part (a) tells us that $mk + m$
               is a natural number. Thus $mk$ is a natural number for all
               natural numbers $n$. \qed
      \end{enumerate}
%%%%%%%%%%%%%%%%%%%%%%%%%%%%%%%%%%Prob1.1_7%%%%%%%%%%%%%%%%%%%%%%%%%%%%%%%%%%%%%
   \item[1.1.7]   Prove that if $n$ is a natural number greater than 1, then
                  $n - 1$ is also a natural number.

      \textbf{Proof.} Consider the sets
      $$B = \{n \in \N \mbox{ and } n - 1 \in \N\} \mbox{ and } 
        S = B \cup \{1\}.$$

      It suffices to show that $B = \N\backslash\{1\}$. Since
      $B = S\backslash\{1\}$, it then suffices to show that $S = \N$. We can
      accomplish this by showing that $S$ is inductive. It is obvious that
      $1 \in S$. Suppose $k$ is a member of $S$. Then we have that $k \in \N$ 
      because $S \subseteq \N$; and since $\N$ is an inductive set, we must have
      that $k + 1 \in \N$. Thus since $k + 1 \in \N$ and $k \in \N$, it follows 
      that $k + 1 \in S$, so that $S$ is inductive. Hence $S = \N$, by (1.1) 
      [Textbook Page 6].  \qed
%%%%%%%%%%%%%%%%%%%%%%%%%%%%%%%%%%Prob1.1_8%%%%%%%%%%%%%%%%%%%%%%%%%%%%%%%%%%%%%
   \item[1.1.8]   Prove that if $n$ and $m$ are natural numbers such that
                  $n > m$, then $n - m$ is also a natural number.

      \textbf{Proof.} We shall proceed by Mathematical Induction on $m$. By 
      Problem 1.1.7, the statement of the problem holds for $m = 1$. So assume 
      it also holds for some natural number $k$. Suppose that $n > k + 1$. 
      Clearly $n > k$, so by our inductive hypothesis, $n - k$ is a natural 
      number. Since $n > k + 1$ we must have that $n - k > 1$. It follows by 
      Problem 1.1.7 that $(n - k) - 1 = n - (k + 1)$ is also a natural number, 
      so that the statement of the problem holds for $m = k + 1$. Thus it holds
      for all natural number $m$. That is, if $n > m$, then $n - m$ is also a 
      natural number. \qed
%%%%%%%%%%%%%%%%%%%%%%%%%%%%%%%%%%Prob1.1_9%%%%%%%%%%%%%%%%%%%%%%%%%%%%%%%%%%%%%
   \item[1.1.9]   Use Exercise 8 to prove that the sum, difference, and product 
                  of integers also are integers.
      
      \textbf{Proof.} We shall be using results from Lemma 1.1.2 in this
      problem. Let $a$ and $b$ be integers. If either $a$ or $b$ is 0, then it
      is clear that $a + b$, $a - b$, $ab$ are all integers. So we can assume
      that $a \neq 0$ and $b \neq 0$. We shall now divide the problem into
      cases:
      
      \textbf{Case 1:} $a$ is positive and $b$ is negative. \\
      First we want to show that $a + b$ is also an integer. 
      By the positivity axioms, exactly one of the following is true:
      $$a + b = 0, \quad a + b > 0, \quad a + b < 0.$$
      If $a + b = 0$, then we are done. If $a + b > 0$, then we must have that
      $a > -b$. But since $b$ is negative, it must follow that $-b$ is positive.
      Thus by Problem 1.1.8, $a - (-b) = a + (-(-b)) = a + b$ is a positive 
      integer. Now the only remaining possiblity is $a + b < 0$, or, 
      equivalently, $-b > a$. By Problem 1.1.8, we have that
      $-b - a = -(b + a) = -(a + b)$ is a positive integer; that is $a + b$ is
      a negative integer. To show that $a - b$ is an integer, recall that
      $a - b = a + (-b)$. But $a$ and $-b$ are both positive integers, so that 
      $a - b$ is a positive integer by Problem 1.1.6(a). It now remains to show
      $ab$ is an integer. Problem 1.1.6(b) tells us that $a(-b)$ is a positive 
      integer, so that $-(a(-b)) = ab$ is a negative integer.
      
      \textbf{Case 2:} $a$ is negative and $b$ is negative. \\
      First we want to show that $a + b$ is also an integer. By Problem 1.1.6
      (a), we have that $(-a) + (-b)$ is a positive integer, so that
      $-((-a) + (-b)) = a + b$ is a negative integer. To show that $a - b$ is 
      an integer, we observe that $a - b = a + (-b)$. Since $a$ is negative and 
      $-b$ is positive, it follows from Case 1 that $a - b$ is an integer. It 
      now remains to show $ab$ is an integer. This follows from Problem 1.1.6(b)
      because $(-a)(-b)$ is a multiplication involving positive integers, so
      that $(-a)(-b) = ab$ is also a positive integer.

      \textbf{Case 3:} $a$ is positive and $b$ is positive. Problem 1.1.6 tells
      us that $ab$ and $a + b$ are both positive integers. Case 1 tells us that
      $a - b = a + (-b)$ is an integer.

      In the case where $a$ is negative and $b$ is positive, we can interchange
      the roles of $a$ and $b$ and use Case 1. \qed
%%%%%%%%%%%%%%%%%%%%%%%%%%%%%%%%Prob1.1_10%%%%%%%%%%%%%%%%%%%%%%%%%%%%%%%%%%%%%%
   \item[1.1.10]  Use Exercise 9 to prove that the rationals numbers satisfy the
                  Field Axioms.

      \textbf{Proof.} Let $p$ and $q$ be rational numbers. Then we have
      $p = a/b$ and $q = c/d$ for some integers $a$, $b$, $c$, and $d$, with
      $c \neq 0$ and $d \neq 0$. First we want to show that the set of rationals
      is closed under addition and multiplication. Since
      $\displaystyle p + q = \frac{ad + bc}{bd}$ and $pq= ac/bd$, it follows by
      Problem 1.1.9 that $p + q$ and $pq$ are also rational numbers. Now since
      $\Q$ is a subset of $\R$, then $\Q$ is commutative and associative under 
      addition and multiplication. The additive inverse of $p$ is $-a/b$; if $p$
      is not zero, then its multiplicative inverse is $b / a$. The rational
      numbers $1/1 = 1$ and $0/1 = 0$ serve as the multiplicative and additive
      inverses for $\Q$. The distributive law also holds in $\Q$ since it holds
      in $\R$. Thus $\Q$ is a Field. \qed
%%%%%%%%%%%%%%%%%%%%%%%%%%%%%%%%Prob1.1_11%%%%%%%%%%%%%%%%%%%%%%%%%%%%%%%%%%%%%%
   \item[1.1.11]
                  \begin{enumerate}
                     \item Prove that the sum of a rational number and an 
                           irrational number must be irrational.
                     \item Prove that the product of two nonzero numbers, one 
                           rational and one irrational is irrational.
                  \end{enumerate}
      
      \textbf{Proof.}
      
      \begin{enumerate}
         \item Let $a$ be a rational number number, and let $b$ be an irrational
               number. We shall show by contradiction that $a + b$ is
               irrational. Suppose that $a + b = p/q$ for some integers $p$ and
               $q$, where $q$ is nonzero. We then have that $b = (p - qa)/q$.
               This says that $b$ is rational, contradicting our initial
               assumption that $b$ was irrational. Thus $a + b$ is irrational.
         \item Let $a$ be a nonzero rational number and let $b$ be a nonzero
               irrational number. We shall show by contradiction that $ab$ is
               irrational. So say that $ab = p/q$ for some integers $p$ and $q$
               where $q$ is nonzero. We then have that $b = p/aq$. This says
               that $b$ is rational, contradicting our initial assumption that
               $b$ was irrational. Thus $ab$ is irrational.
      \end{enumerate}
      
      
%%%%%%%%%%%%%%%%%%%%%%%%%%%%%%%%Prob1.1_12%%%%%%%%%%%%%%%%%%%%%%%%%%%%%%%%%%%%%%
   \item[1.1.12]  Use Proposition 1.2 to show that there is no rational number 
                  whose square equals 2/9.
      
      \textbf{Proof.} It suffices to show that $\sqrt{2/9}$ is irrational. Since
      $\sqrt{2/9} = \sqrt{2} * (1/3)$, it follows by Proposition 1.2 and Problem
      11 (b) that $\sqrt{2} * (1/3)$ is irrational.
%%%%%%%%%%%%%%%%%%%%%%%%%%%%%%%%Prob1.1_13%%%%%%%%%%%%%%%%%%%%%%%%%%%%%%%%%%%%%%
   \item[1.1.13]  Suppose that $S$ is a nonempty set of real numbers that is
                  bounded. Prove that $\inf S \le \sup S$.
              
      \textbf{Proof.} Since $S$ is nonempty, let $s$ be any member of $S$. We 
      thus have by the Completeness Axiom that $\inf S \le s \le \sup S$, so
      that $\inf S \le \sup S$.
%%%%%%%%%%%%%%%%%%%%%%%%%%%%%%%%Prob1.1_14%%%%%%%%%%%%%%%%%%%%%%%%%%%%%%%%%%%%%%
   \item[1.1.14]  Suppose that $S$ is a nonempty set of real numbers that is 
                  bounded and that $\inf S = \sup S$. Prove that the set $S$ 
                  consists of exactly one number.
              
      \textbf{Proof.} Since $S$ is nonempty, let $s$ be any member of $S$. We 
      thus have by the Completeness Axiom that $\inf S \le s \le \sup S$. Thus
      $\inf S \le s \le \inf S$, so that $s = \inf S$. We have shown that if
      $s \in S$, then $s = \inf S$. Thus $S$ has only one element.
%%%%%%%%%%%%%%%%%%%%%%%%%%%%%%%%Prob1.1_15%%%%%%%%%%%%%%%%%%%%%%%%%%%%%%%%%%%%%%
   \item[1.1.15]  For a set $S$ of numbers, a member $c$ of $S$ is called the
                  \textit{maximum} of $S$ provided that it is an upper bound for
                  $S$. Prove that a set $S$ of numbers has a maximum if and only
                  if it is bounded above and $\sup S$ belongs to $S$. Give an 
                  example of a set $S$ of numbers that is nonempty and bounded 
                  above but has no maximum.
              
      \textbf{Proof.} First suppose that $S$ has a maximum, $\max S$. By 
      definition, $\max S$ is an upper bound for $S$. By the Completeness Axiom,
      it follows that $\sup S$ exists; since $\max S$ is in $S$, we must have 
      that $\max S \le \sup S$. But since $\max S$ is also an upper bound for
      $S$ and since no other upper bound can be less than $\sup S$, we must have
      that $\max S = \sup S$, so that $\sup S$ is in $S$. Now suppose that $S$
      is bounded above and that $\sup S$ belongs to $S$. Then it immediately
      follows that $\sup S$ is the maximum of $S$. The set $[0, 1)$ is nonempty
      and bounded above, but it has no maximum since
      $\sup$ $[0, 1) = 1 \notin [0, 1)$.
%%%%%%%%%%%%%%%%%%%%%%%%%%%%%%%%Prob1.1_16%%%%%%%%%%%%%%%%%%%%%%%%%%%%%%%%%%%%%%
   \item[1.1.16]  Prove that $\sqrt{3}$ is not a rational number.
              
      \textbf{Proof.} We shall prove by contradiction that $\sqrt{3}$ is not a 
      rational number. So suppose that $\sqrt{3} = p/q$ for integers $p$ and $q$
      where $q$ is nonzero and $p$ and $q$ have no common positive factor 
      greater than 1. So we have that $p^2 = 3q^2$. This says that $p^2$ has a  
      prime factor of 3. Thus $p$ must also have a prime factor of 3, so we can 
      write $p = 3m$ so that $q^2 = 3m^2$. Similarly, $q$ must also have a prime
      factor of 3, a contradiction since we assumed that $p$ and $q$ are
      relatively prime. Thus, $\sqrt{3}$ is irrational. \qed
%%%%%%%%%%%%%%%%%%%%%%%%%%%%%%%%Prob1.1_17%%%%%%%%%%%%%%%%%%%%%%%%%%%%%%%%%%%%%%
   \item[1.1.17]  (Outline of the proof of Proposition 1.3) Define
                  $$S \equiv \{x : x \in \R, x \ge 0, x^2 < c\}$$
                  \begin{enumerate}
                     \item Show that $c + 1$ is an upper bound for $S$ and
                           therefore, by the Completeness Axiom, $S$ has a least
                           upper bound that we denote by $b$.
                     \item Show that if $b^2 > c$, then we can choose a suitably
                           small positive number $r$ such that $b - r$ is also 
                           an upper bound for $S$, thus contradicting the choice
                           of $b$ as the \textit{least} upper bound of $S$.
                     \item Show that if $b^2 < c$, then we can choose a suitably
                           small positive number $r$ such that $b + r$ belongs 
                           to $S$, thus contradicting the choice of $b$ as an 
                           upper bound of $S$.
                     \item Use parts (b) and (c) and the Positivity Axioms for
                           $\R$ to conclude that $b^2 = c$.
                  \end{enumerate}
	  
	   \textbf{Proof.}
      
      \begin{enumerate}
         \item First notice that $S$ is nonempty because 0 is in $S$. Let $y$
               be any member of $S$. Then we have
               $y^2 < c < 2c < 2c + c^2 + 1 = (c + 1)^2$, so that
               $y^2 < (c + 1)^2$; that is, $y < c + 1$ by Lemma 1.1.1. This says 
               that $S$ is bounded above by $c + 1$, so that $\sup S$, say $b$, 
               exists by The Completeness Axiom.
         \item Suppose that $b^2 > c$. We want to choose a positive number
               $r < b$ so that $b - r$ is also an upper bound for $S$.
               Choose $r = (b^2 - c)/4b$. This choice of $r$ is clearly
               positive. To show that $r < b$, we notice that $3b^2 > -c^2$, so
               that $4b^2 > b^2 - c^2$. Multiplying the inequality
               $4b^2 > b^2 - c^2$ by the positive number $\frac{1}{4b}$ results
               in $b > (b^2 - c^2)/4b = r$, so that $b - r$ is positive. Now
               we have that
               \begin{align*}
                  (b - r)^2 &= b^2 + r^2 - 2br > b^2 - 2br = \\
                            &= b^2 - 2b \cdot \frac{b^2 - c}{4b} \\
                            &= \frac{b^2}{2} + \frac{c}{2} >
                            \frac{c}{2} + \frac{c}{2} = c.
               \end{align*}
               
               So for any $y \in S$, we have $y^2 < c < (b - r)^2$, so that
               $y < b - r$ by Lemma 1.1.1. Thus $S$ has an upper bound, $b - r$, 
               that is less than $b$, a contradiction. Thus $b^2$ cannot be 
               greater than $c$.
         \item Now suppose that $b^2 < c$. Let $r$ be
               $$\min\left\{\frac{b}{2}, \frac{c - b^2}{4b}\right\}.$$
               
               We observe that, as chosen, $0 < r < b$, so that $r^2 < br$.
               Thus
               \begin{align*}
                  (b + r)^2 = b^2 + r^2 + 2br &< b^2 + 3br \\
                  &\le b^2 + 3b \cdot \frac{c - b^2}{4b} \\
                  &= \frac{b^2}{4} + \frac{3c}{4} \\
                  &< \frac{c}{4} + \frac{3c}{4} = c.
               \end{align*}
               
               Since $b + r$ is positive, and since $(b + r)^2 < c$, it follows
               by the definition of $S$ that $b + r$ is in $S$, and since
               $b + r > b$, it must be an upper bound for $S$. Then by Problem
               1.1.15, it must be the supremum of $S$, a contradiction since
               $b = \sup S$. Thus $b^2$ cannot be less than $c$.
         \item In (b) we showed that $b^2 > c$ is not true; that is $b^2 - c$ is
               not positive. Also, we showed in (c) that $c > b^2$ is also not
               true; that is $c - b^2 = -(b^2 - c)$ is also not positive. Thus,
               by the Positivity Axioms, we must have that $b^2 - c = 0$ so that
               $b^2 = c$.
                  
      \end{enumerate}
%%%%%%%%%%%%%%%%%%%%%%%%%%%%%%%%Prob1.1_18%%%%%%%%%%%%%%%%%%%%%%%%%%%%%%%%%%%%%%
   \item[1.1.18]  Prove that there is a positive number $x$ such that $x^3 = 5$.
 
      \textbf{Proof.} We shall proceed as we did in Problem 1.1.17. So define
      $$S \equiv \{x : x \in \R, x \ge 0, x^3 < 5\}.$$ $S$ is nonempty because
      $0 \in S$. Now we must find an upper bound for $S$. So let $s \in S$. Then
      it follows that $0 \le s^3 < 5$. Particularly we must also have that
      $s^3 < 8$, so that $s < 2$ by Lemma 1.1.1. That is $S$ is bounded above 
      denote by 2; then by the Completeness Axiom, $S$ has a supremum, which we 
      shall call $b$. First suppose that $b^3 > 5$. Consider the number $b - r$ 
      where $r = \frac{b^3 - 5}{3b^2}$. We want to show that $b - r$ is an upper 
      bound for $S$. Clearly $r$ is positive. Since $b$ is positive, we must
      have that $2b^3 > -5$ so that $3b^3 > b^3 - 5$; thus
      $\frac{b^3 - 5}{3b^2} < b$; that is $r < b$, so that $b - r$ is positive. 
      We see that $5 < (b - r)^3$ because
      \begin{align*}
         (b - r)^3 &= (b - r)^2(b - r) \\
                   &= (b^2 - 2br + r^2)(b - r) \\
                   &> (b^2 - 2br)(b - r) \\
                   &= b^3 - 3b^2r + 2br^2 \\
                   &> b^3 - 3b^2r \\
                   &= b^3 - 3b^2\frac{b^3 - 5}{3b^2} = 5.
      \end{align*}

      Now if $y \in S$, then we have $0 \le y^3 < 5 < (b - r)^3$. By
      Lemma 1.1.1, we have that $y < b - r$. That is $S$ is bounded above by
      $b - r$, a contradiction because $b = \sup S$. \\

      Now suppose $b^3 < 5$. Let $r$ be
      $$\min\left\{\frac{b}{2}, \frac{5 - b^3}{7b^2}\right\}.$$
               
      We observe that, as chosen, $0 < r < b$, so that $r^2 < br$. Also,
      multiplying the inequalities $0 < r < b$ by the positive number $3br$ will
      give us $0 < 3br^2 < 3b^2r$. Armed with this information we have that    
      \begin{align*}
         (b + r)^3 &= (b + r)^2(b + r) \\
                   &= (b^2 + r^2 + 2br)(b + r) \\
                   &< (b^2 + br + 2br)(b + r) \\
                   &= b^3 + 4b^2r + 3br^2 \\
                   &< b^3 + 4b^2r + 3b^2r \\
                   &= b^3 + 7b^2r \\
                   &\le b^3 + 7b^2\frac{5 - b^3}{7b^2} = 5.
      \end{align*}
               
      We have thus shown that $0 < (b + r)^3 < 5$. Since $b + r$ is positive, 
      and since $(b + r)^3 < 5$, it follows by the definition of $S$ that
      $b + r$ is in $S$, and since $b + r > b$, it must be an upper bound for
      $S$. Then by Problem 1.1.15, it must be the supremum of $S$, a
      contradiction since  $b = \sup S$. Thus $b^3$ cannot be less than $c$.

      We have shown above that the statements $b^3 < 5$ and $b^3 > 5$ are false,
      so by the Positivity Axioms, it must be the case that $b^3 = 5$. \qed     
%%%%%%%%%%%%%%%%%%%%%%%%%%%%%%%%Prob1.1_19%%%%%%%%%%%%%%%%%%%%%%%%%%%%%%%%%%%%%%
   \item[1.1.19]  Define $S \equiv \{x \in \R : x^2 < x\}$. Prove that
                  $\sup S = 1$.

      \textbf{Proof.} Define $I = \{x \in \R: 0 < x < 1\}$. We want to first 
      show that $S = I$. Notice that $S$ is nonempty because $1/2 \in S$. So let
      $s \in S$. Then we have that $s^2 < s$, so that $s^2 - s < 0$. That is
      $s(s - 1) < 0$. We notice that $s$ and $s - 1$ can neither both be   
      positive nor can they both be negative, since that would make
      $s(s - 1) > 0$, a contradiction. Now if $s$ is negative and $s - 1$ is 
      positive then we would have $s < 0$ and $s > 1$, another contradiction. So
      the only possibilility is $0 < s$ and $s < 1$; thus we have shown that if
      $s \in S$, then $0 < s < 1$, so that $S \subseteq I$. Now let $y \in I$.
      Then we have that $0 < y < 1$. Particularly we have that $y < 1$. Since
      $y$ is positive, we can multiply the inequality $y < 1$ by $y$ to get
      $y^2 < y$. That is, $y \in S$, so that $I \subseteq S$. We have thus
      shown that $S = I$. We immediately see that $S$ is bounded above by 1 and
      since it is also nonempty, it follows by the Completeness Axiom that $S$
      has a least upper bound, say $b$. Since 1 is also an upper bound for $S$,
      we must have that $b \le 1$. Now suppose that $b < 1$. Then $b$ must
      be a member of $S$. Then by Problem 1.15, $b$ is the maximum member of
      $S$, but this is a contradiction since $\frac{b + 1}{2} \in S$ but
      $\frac{b + 1}{2} > b$. Thus it follows that $b = 1$. \qed
%%%%%%%%%%%%%%%%%%%%%%%%%%%%%%%%Prob1.1_20%%%%%%%%%%%%%%%%%%%%%%%%%%%%%%%%%%%%%%
   \item[1.1.20] 
                  \begin{enumerate}
                     \item For real numbers $a$ and $b$, suppose that the number
                           $x$ is a solution to the equation
                           $$(x - a)(x - b) = 0.$$
                           Prove that either $x = a$ or $x = b$.
                     \item For a positive number $c$, show that if $x$ is any 
                           number such that $x^2 = c$, then either $x=\sqrt{c}$ 
                           or $x = -\sqrt{c}$.
                     \item Let $a$, $b$, and $c$ be real numbers such that
                           $a \neq 0$, and consider the quadratic equation
                           $$ax^2 + bx + c = 0.$$
                           Prove that a number $x$ is a solution of this
                           equation if and only if
                           $$(2ax + b)^2 = b^2 - 4ac.$$
                           Suppose that $b^2 - 4ac > 0$. Prove that the 
                           quadratic equation has exactly two solutions, given 
                           by
                           $$x = \frac{-b  + \sqrt{b^2 - 4ac}}{2a} \mbox{ and }
                             x = \frac{-b  - \sqrt{b^2 - 4ac}}{2a}.$$
                     \item In part (c) now suppose that $b^2 - 4ac < 0$. Prove 
                           that there is no real number that is a solution of 
                           the quadratic equation.
                  \end{enumerate}
	  
      \textbf{Proof.}
      
      \begin{enumerate}
         \item From the Preliminaries we know that if
               $$(x - a)(x - b) = 0,$$
               then either $x - a = 0$ or $x - b = 0$ so that $x = a$ or
               $x = b$.
         \item From Problem 17, we know that there exists a unique $b$ such that
               $b^2 = c$. We denote this $b$ as $\sqrt{c}$. Since $x^2 = c$ if
               and only if $x^2 - c = 0$, we have that
               $x^2 - c = (x - \sqrt{c})(x + \sqrt{c}) = 0$, so that
               $x = \sqrt{c}$ or $x = -\sqrt{c}$ by (a).
         \item First suppose that $y$ is a solution of the quadratic equation
               \begin{equation}
                  ax^2 + bx + c = 0, \label{1_1_20_0}
               \end{equation}
               then it follows that
               \begin{equation}
                  ay^2 + by + c = 0. \label{1_1_20_1}
               \end{equation}

               Since $a \neq 0$, we can multiplty \eqref{1_1_20_1} by $a^{-1}$
               to get
               $$y^2 + \frac{b}{a}y + \frac{c}{a} = 0.$$

               Then
               $$y^2 + \frac{b}{a}y + \frac{b^2}{4a^2} = \frac{b^2}{4a^2} - 
                 \frac{c}{a},$$
               so that
               $$4a^2y^2 + 4aby + b^2 = b^2 - 4ac,$$
               or
               \begin{equation}
                  (2ay + b)^2 = b^2 - 4ac. \label{1_1_20_2}
               \end{equation}

               Now if we assume that \eqref{1_1_20_2} holds, then we can retrace
               our steps to get $ay^2 + by + c = 0$, so that $y$ is a solution
               of Equation \eqref{1_1_20_0}.

               Now suppose $b^2 - 4ac > 0$. Thus, $\sqrt{b^2 - 4ac}$ exists.
               We can rewrite Equation \eqref{1_1_20_2} to get
               $$(2ay + b + \sqrt{b^2 - 4ac})(2ay + b - \sqrt{b^2 - 4ac}) = 0.$$
               From (a) it follows that
               $$y = \frac{-b  + \sqrt{b^2 - 4ac}}{2a} \mbox{ and }
                 y = \frac{-b  - \sqrt{b^2 - 4ac}}{2a}.$$
         \item Suppose $b^2 - 4ac < 0$. Let $y$ be any real number. It follows
               that $(2ay + b)^2 \ge 0 \neq b^2 - 4ac$. Since
               $(2ay + b)^2 \neq b^2 - 4ac$, it follows by (c) that no real 
               solution exists.
      \end{enumerate}      
\end{enumerate}
