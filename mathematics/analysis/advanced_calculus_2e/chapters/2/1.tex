\begin{enumerate}
%%%%%%%%%%%%%%%%%%%%%%%%%%%%%%%%%%%%%2.1.1%%%%%%%%%%%%%%%%%%%%%%%%%%%%%%%%%%%%%%
   \item[2.1.1]   For each of the following statements, determine whether it is
                  true or false and justify your answer.
                  \begin{enumerate}
                     \item If the sequence $\{{a_n}^2\}$ converges, then the 
                           sequence $\{a_n\}$ also converges.
                     \item If the sequence $\{a_n + b_n\}$ converges, then the
                           sequences $\{a_n\}$ and $\{b_n\}$ also converge.
                     \item If the sequences $\{a_n + b_n\}$ and $\{a_n\}$
                           converge, then the sequence $\{b_n\}$ also converges.
                     \item If the sequence $\{|a_n|\}$ converges, then the
                           sequence $\{a_n\}$ also converges.
                  \end{enumerate}
%%%%%%%%%%%%%%%%%%%%%%%%%%%%%%%%%%%%%2.1.2%%%%%%%%%%%%%%%%%%%%%%%%%%%%%%%%%%%%%%
   \item[2.1.2]   Using only the Archimedean Property of $\R$, give a direct
                  $\epsilon$-$N$ verfication of the following limits:
                  \begin{enumerate}
                     \item $\displaystyle\lim_{n \rightarrow \infty}
                            \frac{1}{\sqrt{n}} = 0$.
                     \item $\displaystyle\lim_{n \rightarrow \infty}
                            \frac{1}{n + 5} = 0$.
                  \end{enumerate}
%%%%%%%%%%%%%%%%%%%%%%%%%%%%%%%%%%%%%2.1.3%%%%%%%%%%%%%%%%%%%%%%%%%%%%%%%%%%%%%%
   \item[2.1.3]   Using only the Archimedean Property of $\R$, give a direct
                  $\epsilon$-$N$ verfication of the convergence of the following 
                  sequences:
                  \begin{enumerate}
                     \item $\displaystyle\left\{\frac{2}{\sqrt{n}} +
                           \frac{1}{n} + 3\right\}$.
                     \item $\displaystyle\left\{\frac{n^2}{n^2 + n}\right\}$.
                  \end{enumerate}
%%%%%%%%%%%%%%%%%%%%%%%%%%%%%%%%%%Prob2.1.4%%%%%%%%%%%%%%%%%%%%%%%%%%%%%%%%%%%%%
   \item[2.1.4]   For the sequence $\{a_n\}$ defined in Example 2.3:
                  \begin{enumerate}
                     \item What are the terms $a_{10}$, $a_{20}$, $a_{30}$?
                     \item Find the second index $n$ for which $a_n = 1/4$ and
                           the fourth index $n$ for which $a_n = 1$.
                     \item For $j$ an odd natural number, set
                           $$n = \frac{j(j+1)}{2} + \frac{j+1}{2}$$
                           and show that $a_n = 1/2$.
                     \item Show that $\{a_n\}$ does not converge.
                  \end{enumerate}
%%%%%%%%%%%%%%%%%%%%%%%%%%%%%%%%%%%%%2.1.5%%%%%%%%%%%%%%%%%%%%%%%%%%%%%%%%%%%%%%
   \item[2.1.5]   For the sequence $\{a_n\}$ defined in Example 2.3 and any
                  rational number $x$ in the interval $(0, 1]$, show that there
                  are infinitely many indices $n$ such that $a_n = x$.
%%%%%%%%%%%%%%%%%%%%%%%%%%%%%%%%%%%%%2.1.6%%%%%%%%%%%%%%%%%%%%%%%%%%%%%%%%%%%%%%
   \item[2.1.6]   Suppose that the sequence $\{a_n\}$ converges to $a$ and that
                  $a > 0$. Show that there is an index $N$ such that $a_n > 0$
                  for all indices $n \ge N$.
%%%%%%%%%%%%%%%%%%%%%%%%%%%%%%%%%%%%%2.1.7%%%%%%%%%%%%%%%%%%%%%%%%%%%%%%%%%%%%%%
   \item[2.1.7]   Suppose that the sequence $\{a_n\}$ converges to $l$ and that
                  the sequence $\{b_n\}$ has the property that there is an index
                  $N$ such that
                  $$a_n = b_n \quad \text{for all indices }n \ge N.$$
                  Show that $\{b_n\}$ also converges to $l$.
                  (\textit{Suggestion}: Use the Comparison Lemma for a quick
                  proof.)
%%%%%%%%%%%%%%%%%%%%%%%%%%%%%%%%%%%%%2.1.8%%%%%%%%%%%%%%%%%%%%%%%%%%%%%%%%%%%%%%
   \item[2.1.8]   Prove that the sequence $\{c_n\}$ converges to $c$ if and only
                  if the sequence $\{c_n - c\}$ converges to 0.
%%%%%%%%%%%%%%%%%%%%%%%%%%%%%%%%%%%%%2.1.9%%%%%%%%%%%%%%%%%%%%%%%%%%%%%%%%%%%%%%
   \item[2.1.9]   Prove that the Archimedean Property of $\R$ is equivalent to
                  the fact that $\lim_{n \rightarrow \infty} 1/n = 0$.
%%%%%%%%%%%%%%%%%%%%%%%%%%%%%%%%%%%%%2.1.10%%%%%%%%%%%%%%%%%%%%%%%%%%%%%%%%%%%%%
   \item[2.1.10]  Prove that
                  $$\lim_{n \rightarrow \infty} n^{1/n} = 1.$$
                  \textit{Hint}: Define $\alpha_n = n^{1/n} - 1$ and use the
                  Binomial Formula to show that for each index $n$,
                  $$n = (1 + \alpha_n)^n \ge 1 + [n(n - 1)/2]{\alpha_n}^2.$$
%%%%%%%%%%%%%%%%%%%%%%%%%%%%%%%%%%%%%2.1.11%%%%%%%%%%%%%%%%%%%%%%%%%%%%%%%%%%%%%
   \item[2.1.11]  We have proven that the sequence $\{1/n\}$ converges to 0 and
                  that it does not converge to any other number. Use this to
                  prove that none of the following assertions is equivalent to
                  the definition of convergence of a sequence $\{a_n\}$ to the
                  number $a$.
                  \begin{enumerate}
                     \item For some $\epsilon > 0$ there is an index $N$ such
                           that
                           $$|a_n - a| < \epsilon \quad\text{for all indices }
                             n \ge N.$$
                     \item For each $\epsilon > 0$ and each index $N$,
                           $$|a_n - a| < \epsilon \quad\text{for all indices }
                             n \ge N.$$
                     \item There is an index $N$ such that for every number
                           $\epsilon > 0$,
                           $$|a_n - a| < \epsilon \quad\text{for all indices }
                             n \ge N.$$
                  \end{enumerate}
%%%%%%%%%%%%%%%%%%%%%%%%%%%%%%%%%%%%%2.1.12%%%%%%%%%%%%%%%%%%%%%%%%%%%%%%%%%%%%%
   \item[2.1.12]  For the sequence defined in Example 2.2, show that for every
                  index $n$, $|a_n - \sqrt{2}| < 2/n$. Use this property to show
                  that the sequence converges to $\sqrt{2}$.
%%%%%%%%%%%%%%%%%%%%%%%%%%%%%%%%%%%%%2.1.13%%%%%%%%%%%%%%%%%%%%%%%%%%%%%%%%%%%%%
   \item[2.1.13]  Prove the Polynomial Property for convergent sequences by
                  using an inductive argument based on the degree of the
                  polynomial.
%%%%%%%%%%%%%%%%%%%%%%%%%%%%%%%%%%%%%2.1.14%%%%%%%%%%%%%%%%%%%%%%%%%%%%%%%%%%%%%
   \item[2.1.14]  Define the sequence $\{s_n\}$ by
                  $$s_n = \frac{1}{2 \cdot 1} + \frac{1}{3 \cdot 2} + \cdots +
                    \frac{1}{(n + 1)(n)} \quad \text{for every index } n.$$
                  Prove that
                  $$\lim_{n \rightarrow \infty} s_n = 1.$$
%%%%%%%%%%%%%%%%%%%%%%%%%%%%%%%%%%%%%2.1.15%%%%%%%%%%%%%%%%%%%%%%%%%%%%%%%%%%%%%
   \item[2.1.15]  Let $\{a_n\}$ be a sequence of real numbers. Suppose that for
                  each positive number $c$ there is an index $N$ such that
                  $$a_n > c \quad\text{for all indices } n \ge N.$$
                  When this is so, the sequence $\{a_n\}$ is said to
                  \textit{converge to infinity}, and we write
                  $$\lim_{n \rightarrow \infty} a_n = \infty.$$
                  Prove the following:
                  \begin{enumerate}
                     \item $\displaystyle\lim_{n \rightarrow \infty}
                           [n^3 - 4n^2 - 100n] = \infty$.
                     \item $\displaystyle\lim_{n \rightarrow \infty}
                           \left[\sqrt{n} - \frac{1}{n^2} + 4\right] = \infty$.
                  \end{enumerate}
%%%%%%%%%%%%%%%%%%%%%%%%%%%%%%%%%%%%%2.1.16%%%%%%%%%%%%%%%%%%%%%%%%%%%%%%%%%%%%%
   \item[2.1.16]  Discuss the convergence to infinity of each of the following
                  sequences:
                  \begin{enumerate}
                     \item $\{\sqrt{n + 1} - \sqrt{n}\}$.
                     \item $\{(\sqrt{n + 1} - \sqrt{n})\sqrt{n}\}$.
                     \item $\{(\sqrt{n + 1} - \sqrt{n})n\}$.
                  \end{enumerate}
%%%%%%%%%%%%%%%%%%%%%%%%%%%%%%%%%%%%%2.1.17%%%%%%%%%%%%%%%%%%%%%%%%%%%%%%%%%%%%%
   \item[2.1.17]  For a sequence $\{a_n\}$ of positive numbers show that
                  $$\lim_{n \rightarrow \infty} a_n = \infty \quad\text{if and
                    only if } \lim_{n \rightarrow \infty}
                    \left[\frac{1}{a_n}\right] = 0.$$
%%%%%%%%%%%%%%%%%%%%%%%%%%%%%%%%%%%%%2.1.18%%%%%%%%%%%%%%%%%%%%%%%%%%%%%%%%%%%%%
   \item[2.1.18]  (The Convergence of Cesaro Averages.) Suppose that the
                  sequence $\{a_n\}$ converges to $a$. Define the sequence
                  $\{\sigma_n\}$ by
                  $$\sigma_n = \frac{a_1 + a_2 + \cdots + a_n}{n}
                    \quad\text{for every index } n.$$
                  Prove that the sequence $\{\sigma_n\}$ also converges to $a$.
\end{enumerate}
