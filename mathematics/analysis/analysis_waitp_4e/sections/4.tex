\begin{enumerate}
%%%%%%%%%%%%%%%%%%%%%%%%%%%%%%%%%%Prob4.1%%%%%%%%%%%%%%%%%%%%%%%%%%%%%%%%%%%%%%%
   \item[4.1]  Mark each statement True or False. Justify each answer.
               \begin{enumerate}
                  \item To prove a universal statement $\forall$ $x$, $p(x)$, we 
                        let $x$ represent an arbitrary member from the system 
                        under consideration and show that $p(x)$ is true.
                  \item To prove an existential statement
                        $\exists$ $x \ni p(x)$, we must find a particular $x$ in 
                        the system for which $p(x)$ is true.
                  \item In writing a proof, it is important to include all the 
                        logical steps.
               \end{enumerate}

      \textbf{Solution:} 

      \begin{enumerate}
         \item True. Since the $x$ is arbitrary, then if $p(x)$ is true, it must
               be the case that the statement is true for all $x$.
         \item False. We must find at least one $x$ in the system for which
               $p(x)$ is true.
         \item False. Only ``relevant" steps should be included.
      \end{enumerate}
%%%%%%%%%%%%%%%%%%%%%%%%%%%%%%%%%%Prob4.2%%%%%%%%%%%%%%%%%%%%%%%%%%%%%%%%%%%%%%%
   \item[4.2]  Mark each statement True or False. Justify each answer.
               \begin{enumerate}
                  \item A proof by contradiction may use the tautology
                        $({\sim}p \Rightarrow c) \Leftrightarrow p$.
                  \item A proof by contradiction may use the tautology
                        $[(p \lor {\sim}q) \Rightarrow c] \Leftrightarrow
                        (p \Rightarrow q)$.
                  \item Definitions often play an important role in proofs.
               \end{enumerate}

      \textbf{Solution:}

      \begin{enumerate}
         \item True. [See Text]
         \item False. The left side of the tautology should be
               $[(p \land {\sim}q) \Rightarrow c]$.
         \item True. [See Text]
      \end{enumerate}
%%%%%%%%%%%%%%%%%%%%%%%%%%%%%%%%%%Prob4.3%%%%%%%%%%%%%%%%%%%%%%%%%%%%%%%%%%%%%%%
   \item[4.3]  Prove: There exists an integer $n$ such that $n^2 + 3n/2 = 1$. Is
               this integer unique?
			
		\textbf{Proof:} Let $n = -2$. Then we have $(-2)^2 + 3(-2)/2 = 1$, so that
		the integer $-2$ is a solution to the equation $n^2 + 3n/2 = 1$. We claim
		that this integer is unique. To show this, suppose that $n' \in \Z$ is
		also a solution; that is, $n'^2 + 3n'/2 = 1$. Multiplying the latter
		equation by 2 and factoring will give us $(n' + 2)(n' - 1/2)= 0$, so that
		$n' = -2$. \qed
%%%%%%%%%%%%%%%%%%%%%%%%%%%%%%%%%%Prob4.4%%%%%%%%%%%%%%%%%%%%%%%%%%%%%%%%%%%%%%%
   \item[4.4]  Prove: There exists a rational number $x$ such that
               $x^2 + 3x/2 = 1$. Is this rational number unique?
			
		\textbf{Proof:} Let $x = 1/2$. Then we have $(1/2)^2 + 3(1/2)/2 = 1$, so
		that the rational number $1/2$ is a solution to the equation
		$n^2 + 3n/2 = 1$. This rational number is not unique since the rational
		number $-2/1$ also solves the equation. \qed
%%%%%%%%%%%%%%%%%%%%%%%%%%%%%%%%%%Prob4.5%%%%%%%%%%%%%%%%%%%%%%%%%%%%%%%%%%%%%%%
   \item[4.5]  Prove: For every real number $x > 3$, there exists a real number
               $y < 0$ such that $x = 3y/(2 + y)$.
					
		\textbf{Proof:} Choose $y = 2x/(3-x)$ for some real number greater $x$
		than 3. First we must show that $y$ is negative and that $x = 3y/(2 + y)$.
		Since $x$ is greater than 3, we have that $2x$ is positive and $3 - x$ is
		negative, so that $y$ is negative. Finally, we have that
		\begin{align*}
			\frac{3y}{2 + y} &= \frac{\frac{6x}{3 - x}}{2 + \frac{2x}{3 - x}} \\
								  &= \frac{6x}{2(3 - x) + 2x} \\
								  &= \frac{6x}{6 - 2x + 2x} \\
								  &= x.
		\end{align*} \qed
%%%%%%%%%%%%%%%%%%%%%%%%%%%%%%%%%%Prob4.6%%%%%%%%%%%%%%%%%%%%%%%%%%%%%%%%%%%%%%%
   \item[4.6]  Prove: For every real number $x > 1$, there exist two distinct
               positive real numbers $y$ and $z$ such that
               $$x = \frac{y^2 + 9}{6y} = \frac{z^2 + 9}{6z}.$$
					
		\textbf{Proof:} Let $x$ be a real number greater than 1. So choose
		$$y = 3x - 3\sqrt{x^2 - 1} \text{ and } z = 3x + 3\sqrt{x^2 - 1}.$$
		It is routine to check that $y$ and $z$ satisfy the required equations.
		\qed
%%%%%%%%%%%%%%%%%%%%%%%%%%%%%%%%%%Prob4.7%%%%%%%%%%%%%%%%%%%%%%%%%%%%%%%%%%%%%%%
   \item[4.7] Prove: If $x^2 + x - 6 \ge 0$, then $x \le -3$ or $x \ge 2$.
	
		\textbf{Proof:} Suppose that $x^2 + x - 6 \ge 0$ for some real number $x$.
		Then we have that $x^2 + x - 6 = (x + 3)(x - 2)\ge 0$. Solving this
		inequality will give us $x \le -3$ or $x \ge 2$. \qed
%%%%%%%%%%%%%%%%%%%%%%%%%%%%%%%%%%Prob4.8%%%%%%%%%%%%%%%%%%%%%%%%%%%%%%%%%%%%%%%
   \item[4.8] Prove: If $x/(x - 1) \le 2$, then $x < 1$ or $x \ge 2$.
	
		\textbf{Proof:} Suppose $x/(x - 1) \le 2$ for some real number $x$.
		
		\textbf{Case I:} $x > 1$. \\
		Multiply the inequality $x/(x - 1) \le 2$ by the positive number $x - 1$
		to give us $x \le 2(x - 1)$. Solving the latter inequality results in
		$x \ge 2$.
		
		\textbf{Case II:} $x < 1$. \\
		Multiply the inequality $x/(x - 1) \le 2$ by the negative number $x - 1$
		to give us $x \ge 2(x - 1)$. Solving the latter inequality results in
		$x \le 2$. By assumption, $x$ must also satisfy $x < 1$. To satisfy
		$x < 1$ and $x \le 2$, we must have that $x < 1$.
		
		The proof is done. \qed
%%%%%%%%%%%%%%%%%%%%%%%%%%%%%%%%%%Prob4.9%%%%%%%%%%%%%%%%%%%%%%%%%%%%%%%%%%%%%%%
   \item[4.9] Prove: $\log_27$ is irrational.
	
		\textbf{Proof:} Suppose by way of contradiction that $\log_27$ is
		rational. We note that the $\log$ function is strictly increasing, and
		since $\log_22 = 1$, we must have $\log_27 > \log_22 > 1$, so that
		$\log_27$ is positive. Thus there exist positive integers $p$ and $q$
		such $\log_27 = p/q$. That is $2^{p/q} = 7$, so that $2^p = 7^q$. By the
		Fundamental Theorem of Arithmetic, it follows that $2^p$ and $7^q$ have
		the same prime factors. But this is false, since $2^p$'s only prime factor
		is 2 and $7^q$'s only prime factor is 7. Thus $\log_27$ is irrational.
		\qed
%%%%%%%%%%%%%%%%%%%%%%%%%%%%%%%%%%Prob4.10%%%%%%%%%%%%%%%%%%%%%%%%%%%%%%%%%%%%%%
   \item[4.10] Prove: If $x$ is a real number, then $|x - 2| \le 3$ implies that
               $-1 \le x \le 5$.
					
		\textbf{Proof:} Let $x$ be a real number. Suppose that $|x - 2| \le 3$.
		Then by definition we have that $-3 \le x - 2 \le 3$. Adding 2 to these
		inequalities results in $-1 \le x \le 5$. \qed
%%%%%%%%%%%%%%%%%%%%%%%%%%%%%%%%%%Prob4.11%%%%%%%%%%%%%%%%%%%%%%%%%%%%%%%%%%%%%%
   \item[4.11] Consider the following theorem:
               ``If $m^2$ is odd, then $m$ is odd." Indicate what, if anything, 
               is wrong with each of the following ``proofs."
               \begin{enumerate}
                  \item Suppose $m$ is odd. Then $m = 2k + 1$ for some integer
                        $k$. Thus $m^2 = (2k + 1)^2 = 4k^2 + 4k + 1 =
                        2(2k^2 + 2k) + 1$, which is odd. Thus if $m^2$ is odd,
                        then $m$ is odd.
                  \item Suppose $m$ is not odd. Then $m$ is even and $m = 2k$ 
                        for some integer $k$. Thus
                        $m^2 = (2k)^2 = 4k^2 = 2(2k^2)$, which is even. Thus if
                        $m$ is not odd, then $m^2$ is not odd. It follows that
                        if $m^2$ is odd, then $m$ is odd.
               \end{enumerate}
					
		\textbf{Solution:}
		
		\begin{enumerate}
			\item	The proof proved the converse of the original implication,
					instead of proving the implication; and as we know, the converse
					of an implication is not necessarily equivalent to the original
					implication.
			\item	The proof proved the contrapositive of the original statement,
					but it did not explicitly state so.
      \end{enumerate} 
%%%%%%%%%%%%%%%%%%%%%%%%%%%%%%%%%%Prob4.12%%%%%%%%%%%%%%%%%%%%%%%%%%%%%%%%%%%%%%
   \item[4.12] Consider the following theorem: ``If $xy = 0$, then $x = 0$ or
               $y = 0$." Indicate what, if anything, is wrong with each of the
               following ``proofs."
               \begin{enumerate}
                  \item Suppose $xy = 0$ and $x \neq 0$. Then dividing both
                        sides of the first equation by $x$ we have $y = 0$. Thus
                        if $xy = 0$, then $x = 0$ or $y = 0$.
                  \item There are two cases to consider. First suppose that
                        $x = 0$. Then $x \cdot y = x \cdot 0 = 0$. Similarly,
                        suppose that $y = 0$. Then $x \cdot y = x \cdot 0 = 0$.
                        In either case, $x \cdot y = 0$. Thus if $xy = 0$, then
                        $x = 0$ or $y = 0$.
               \end{enumerate}
					
		\textbf{Solution:}
		
		\begin{enumerate}
			\item	The proof is OK.
			\item	The proof proved the converse of the original implication,
					instead of proving the implication.
      \end{enumerate} 
%%%%%%%%%%%%%%%%%%%%%%%%%%%%%%%%%%Prob4.13%%%%%%%%%%%%%%%%%%%%%%%%%%%%%%%%%%%%%%
   \item[4.13] Suppose $x$ and $y$ are real numbers. Recall that a real number
               $m$ is rational iff $m = p/q$ where $p$ and $q$ are integers and
               $q \neq 0$. If a real number is not rational, then it is
               irrational. Prove the following. [You may use the fact that the
               sum and product of integers is again an integer.]
               \begin{enumerate}
                  \item If $x$ is rational and $y$ is rational, then $x + y$ is
                        rational.
                  \item If $x$ is rational and $y$ is rational, then $xy$ is 
                        rational.
                  \item If $x$ is rational and $y$ is irrational, then $x + y$
                        is irrational.
               \end{enumerate}
					
		\textbf{Proof:}
		
		\begin{enumerate}
			\item	Let $x$ and $y$ be rational numbers. Then there exist integers
					$p$ and $r$ and nonzero integers $q$ and $s$ such that $x = p/q$
					and $y = r/s$. Then it follows that
					\begin{align*}
						x + y &= \frac{p}{q} + \frac{r}{s} \\
								&= \frac{ps + qr}{qs}.
					\end{align*}
					
					Since the sum and product of integers are also integers we have
					that $x + y$ is also a rational number. \qed
			\item	Let $x$ and $y$ be rational numbers. Then there exist integers
					$p$ and $r$ and nonzero integers $q$ and $s$ such that $x = p/q$
					and $y = r/s$. Then it follows that $xy = pr/qs$. Since the
					product of integers is also an integer we have that $xy$ is also
					a rational number. \qed
			\item	Let $x$ be a rational number and let $y$ be an irrational number.
					Then we can write $x = p/q$ for some integer $p$ and nonzero
					integer $q$. Assume by way of contradiction that $x + y$ is
					rational; hence $x + y = r/s$ for some integers $r$ and $s$, with
					$s \neq 0$. Thus we have that $y = (r/s) + (-x)$. By (b) it
					follows that $y$ is a rational number, a contradiction. Thus
					$x + y$ must be irrational. \qed
      \end{enumerate} 
%%%%%%%%%%%%%%%%%%%%%%%%%%%%%%%%%%Prob4.14%%%%%%%%%%%%%%%%%%%%%%%%%%%%%%%%%%%%%%
   \item[4.14] Suppose $x$ and $y$ are real numbers. Prove or give a 
               counterexample.
               \begin{enumerate}
                  \item If $x$ is irrational and $y$ is irrational, then $x + y$
                        is irrational.
                  \item If $x + y$ is irrational, then $x$ is irrational or
                        $y$ is irrational.
                  \item If $x$ is irrational and $y$ is irrational, then $xy$
                        is irrational.
                  \item If $xy$ is irrational, then $x$ is irrational or $y$ is
                        irrational.
               \end{enumerate}
					
		\textbf{Proof:}
		
		\begin{enumerate}
			\item	False. Let $x = \sqrt{2}$ and let $y = -\sqrt{2}$. Then
					$x + y = 0$, a rational number. \qed
			\item	True. The contrapositive of this statement was proved in
					4.13(a). \qed
			\item	False. Let $x = y = \sqrt{2}$. Then $xy = 2$, a rational
					number. \qed
			\item	True. The contrapositive of this statement was proved in
					4.13(b). \qed
      \end{enumerate} 
%%%%%%%%%%%%%%%%%%%%%%%%%%%%%%%%%%Prob4.15%%%%%%%%%%%%%%%%%%%%%%%%%%%%%%%%%%%%%%
   \item[4.15] Consider the following theorem and proof.
               \begin{quote}
                  \textbf{Theorem}: If $x$ is rational and $y$ is irrational,
                  then $xy$ is irrational.
                  \begin{quote}
                     \textbf{Proof}: Suppose $x$ is rational and $y$ is
                     irrational. If $xy$ is rational, then we have $x = p/q$ and
                     $xy = m/n$ for some integers $p$, $q$, $m$ and $n$, with
                     $q \neq 0$ and $n \neq 0$. It follows that
                     $$y = \frac{xy}{x} = \frac{m/n}{p/q} = \frac{mq}{np}.$$
                     This implies that $y$ is rational, a contradiction. We
                     conclude that $xy$ must be irrational.
                  \end{quote}
               \end{quote}
               \begin{enumerate}
                  \item Find a specific counterexample to show that the theorem
                        is false.
                  \item Explain what is wrong with the proof.
                  \item What additional condition on $x$ in the hypothesis would
                        make the conclusion true?
               \end{enumerate}
					
		\textbf{Solution:}
		
		\begin{enumerate}
			\item	Let $x = 0$ and let $y$ be any irrational number. Then $xy = 0$,
					a rational number.
         \item The proof did not take into consideration that $x$ could be 0.
               If $x$ were zero, then the expression $y = xy/x$ would be
               illegal.
         \item In addition to being rational, $x$ must also be nonzero.

      \end{enumerate} 
%%%%%%%%%%%%%%%%%%%%%%%%%%%%%%%%%%Prob4.16%%%%%%%%%%%%%%%%%%%%%%%%%%%%%%%%%%%%%%
   \item[4.16] Prove or give a counterexample: If $x$ is irrational, then
               $\sqrt{x}$ is irrational.

      \textbf{Proof:} Assume by way of contradiction that for some irrational
      number $x$ that $\sqrt{x}$ is rational. Then we have $\sqrt{x} = p/q$ for
      some integers $p$ and $q$, with $q \neq 0$. Squaring both sides of this
      equality results in $x = p^2/q^2$; that is $x$ is rational, a
      contradiction. Thus $\sqrt{x}$ is irrational. \qed
%%%%%%%%%%%%%%%%%%%%%%%%%%%%%%%%%%Prob4.17%%%%%%%%%%%%%%%%%%%%%%%%%%%%%%%%%%%%%%
   \item[4.17] Prove or give a counterexample: There do not exist three
               consecutive even integers $a$, $b$, and $c$ such that
               $a^2 + b^2 = c^2$.

      \textbf{Counterexample:} Consider the triple $(-2, 0, 2)$ or $(6, 8, 10)$.
%%%%%%%%%%%%%%%%%%%%%%%%%%%%%%%%%%Prob4.18%%%%%%%%%%%%%%%%%%%%%%%%%%%%%%%%%%%%%%
   \item[4.18] Consider the following theorem: There do not exist three
               consecutive odd integers $a$, $b$, and $c$ such that
               $a^2 + b^2 = c^2$.
               \begin{enumerate}
                  \item Complete the following restatement of the theorem: For
                        every three consecutive odd integers $a$, $b$, and $c$,
                        we have that $a^2 + b^2 \neq c^2$.
                  \item Change the sentence in part (a) into an implication
                        $p \Rightarrow q$: If $a$, $b$, and $c$ are consecutive
                        odd integers, then $a^2 + b^2 \neq c^2$.
                  \item Fill in the blanks in the following proof of the
                        theorem.
                        \begin{quote}
                           \textbf{Proof}: Let $a$, $b$, and $c$ be consecutive
                           odd integers. Then $a = 2k + 1$, $b = 2k + 3$, and
                           $c = 2k + 5$ for some integer $k$. Suppose
                           $a^2 + b^2 = c^2$. Then
                           $(2k + 1)^2 + (2k + 3)^2 = (2k + 5)^2$.

                           \quad It follows that
                           $8k^2 + 16k + 10 = 4k^2 + 20k + 25$ and
                           $4k^2 - 4k - 15 = 0$. Thus $k = 5/2$ or $k = -3/2$.
                           This contradicts $k$ being an integer. Therefore,
                           there do not exist three consecutive odd integers
                           $a$, $b$, and $c$ such that $a^2 + b^2 = c^2$.
                        \end{quote}
                  \item Which of the tautologies in Example 3.12 best describes
                        the structure of the proof? 3.12 (g)
               \end{enumerate}
%%%%%%%%%%%%%%%%%%%%%%%%%%%%%%%%%%Prob4.19%%%%%%%%%%%%%%%%%%%%%%%%%%%%%%%%%%%%%%
   \item[4.19] Prove or give a counterexample: The sum of any five consecutive
               integers is divisible by five.

      \textbf{Proof:} Let $x$ be the least integer amongst a list of five
      consecutive integers. Then the integers are: $x$, $x + 1$, $x + 2$,
      $x + 3$, $x + 4$, and their sum is $5x + 10 = 5(x + 2)$; that is, their
      sum is divisible by 5. \qed
%%%%%%%%%%%%%%%%%%%%%%%%%%%%%%%%%%Prob4.20%%%%%%%%%%%%%%%%%%%%%%%%%%%%%%%%%%%%%%
   \item[4.20] Prove or give a counterexample: The sum of any four consecutive
               integers is never divisible by four.

      \textbf{Proof:} Let $x$ be the least integer amongst a list of four
      consecutive integers. Then the integers are: $x$, $x + 1$, $x + 2$,
      $x + 3$, and their sum is $4x + 6 = 4(x + 2) + 2$. This says that the
      sum of any four consecutive integers always a remainder of 2 when divided
      by 4, so that this sum is not divisible by four. \qed
%%%%%%%%%%%%%%%%%%%%%%%%%%%%%%%%%%Prob4.21%%%%%%%%%%%%%%%%%%%%%%%%%%%%%%%%%%%%%%
   \item[4.21] Prove or give a counterexample: For every positive integer $n$,
               $n^2 + 3n + 8$ is even.

      \textbf{Proof:} Let $n$ be a positive integer. We shall investigate the
      following two cases:

      \textbf{Cases I:} \textit{$n$ is even}. Thus we can write $n = 2k$ for
      some natural number $k$, so that
      $n^2 + 3n + 8 = 4k^2 + 6k + 8 = 2(2k^2 + 3k + 4)$, an even number.

      \textbf{Cases II:} \textit{$n$ is odd}. Thus we can write $n = 2k + 1$ for
      some natural number $k$, so that
      $n^2 + 3n + 8 = 4k^2 + 10k + 12 = 2(2k^2 + 5k + 6)$, an even number.

      Thus if $n$ is a positive integer, then $n^2 + 3n + 8$ is even. \qed
%%%%%%%%%%%%%%%%%%%%%%%%%%%%%%%%%%Prob4.22%%%%%%%%%%%%%%%%%%%%%%%%%%%%%%%%%%%%%%
   \item[4.22] Prove or give a counterexample: For every positive integer $n$,
               $n^2 + 4n + 8$ is even.
   
      \textbf{Counterexample:} If $n = 1$, then $n^2 + 4n + 8 = 13$, an odd
      number.
%%%%%%%%%%%%%%%%%%%%%%%%%%%%%%%%%%Prob4.23%%%%%%%%%%%%%%%%%%%%%%%%%%%%%%%%%%%%%%
   \item[4.23] Prove or give a counterexample: there do not exist irrational
               numbers $x$ and $y$ such that $x^y$ is rational.

      \textbf{Counterexample:} We do not know whether or not the number
      $\sqrt{2}^{\sqrt{2}}$ is rational, so we shall investigate the following
      two possibilities.

      \textbf{Cases I:} \textit{$\sqrt{2}^{\sqrt{2}}$ is irrational}. Then let
      $x = \sqrt{2}^{\sqrt{2}}$ and $y = \sqrt{2}$. Then we have $x^y = 2$.

      \textbf{Cases II:} \textit{$\sqrt{2}^{\sqrt{2}}$ is rational}. Then let
      $x = y = \sqrt{2}$, so that $x^y$ is a rational number.       
%%%%%%%%%%%%%%%%%%%%%%%%%%%%%%%%%%Prob4.24%%%%%%%%%%%%%%%%%%%%%%%%%%%%%%%%%%%%%%
   \item[4.24] Prove or give a counterexample: there do not exist rational
               numbers $x$ and $y$ such that $x^y$ is a positive integer and
               $y^x$ is a negative integer.

      \textbf{Counterexample:} Let $x = 1$ and $y = -2$. Then we have
      $x^y = 1$ and $y^x = -2$.
%%%%%%%%%%%%%%%%%%%%%%%%%%%%%%%%%%Prob4.25%%%%%%%%%%%%%%%%%%%%%%%%%%%%%%%%%%%%%%
   \item[4.25] Prove or give a counterexample: for all $x > 0$ we have
               $x^2 + 1 < (x + 1)^2 \le 2(x^2 + 1)$.

      \textbf{Proof:} Let $x$ be a positive real number. First we want to show
      that $x^2 + 1 < (x + 1)^2$. We assumed that $x > 0$ so that $2x > 0$;
      that is $-2x < 0$. Adding $(x + 1)^2$ to the inequality $-2x < 0$ results
      in $(x + 1)^2 - 2x < (x + 1)^2$, so that $x^2 + 1 < (x + 1)^2$. Now we
      must show that $(x + 1)^2 \le 2(x^2 + 1)$. We know that for any real
      number $y$ we have that $y^2 \ge 0$; thus we have that
      $(x - 1)^2 = x^2 - 2x + 1 \ge 0$, so that $-x^2 + 2x - 1 \le 0$. Adding
      $2x^2 + 2$ to the inequality $-x^2 + 2x - 1 \le 0$ results in
      $x^2 + 2x + 1 \le 2x^2 + 2$; that is, $(x + 1)^2 \le 2(x^2 + 1)$. The
      proof is done. \qed
\end{enumerate}
