\begin{enumerate}
%%%%%%%%%%%%%%%%%%%%%%%%%%%%%%%%%%Prob3.1%%%%%%%%%%%%%%%%%%%%%%%%%%%%%%%%%%%%%%%
   \item[3.1] Mark each statement True or False. Justify each answer.
      \begin{enumerate}
         \item When an implication $p \Rightarrow q$ is used as a theorem, we
               refer to $p$ as the antecedent.
         \item The contrapositive of $p \Rightarrow q$ is
               $\sim$$p \Rightarrow$ $\sim$$q$.
         \item The inverse of $p \Rightarrow q$ is
               $\sim$$q \Rightarrow$ $\sim$$p$.
         \item To prove ``$\forall$ $n$, $p(n)$" is true, it takes only one 
               example.
         \item To prove ``$\exists$ $n \ni p(n)$" is true, it takes only one 
               example.         
      \end{enumerate}

      \textbf{Solution:} 

      \begin{enumerate}
         \item True. We refer to $p$ as the antecedent and $q$ as the 
               consequent.
         \item False. The contrapositive of $p \Rightarrow q$ is
               $\sim$$q \Rightarrow$ $\sim$$p$.
         \item False. The inverse of $p \Rightarrow q$ is
               $\sim$$p \Rightarrow$ $\sim$$q$.
         \item False. We have to show that $p(n)$ holds for all $n$.
         \item True. To show existence, it is sufficient to provide one example.
      \end{enumerate}
%%%%%%%%%%%%%%%%%%%%%%%%%%%%%%%%%%Prob2.2%%%%%%%%%%%%%%%%%%%%%%%%%%%%%%%%%%%%%%%
   \item[3.2] Mark each statement True or False. Justify each answer.
      \begin{enumerate}
         \item When an implication $p \Rightarrow q$ is used as a theorem, we
               refer to $q$ as the conclusion.
         \item A statement that is always false is called a lie.
         \item The converse of $p \Rightarrow q$ is $q \Rightarrow p$.
         \item To prove ``$\forall$ $n$, $p(n)$" is false, it takes only one
               counterexample.
         \item To prove ``$\exists$ $n \ni p(n)$" is false, it takes only one
               counterexample.
      \end{enumerate}

      \textbf{Solution:} 

      \begin{enumerate}
         \item False. We refer to $q$ as the consequent.
         \item False. It is called a contradiction.
         \item True.
         \item True.
         \item False. We have to show that $p(n)$ holds for all $n$.
      \end{enumerate}
%%%%%%%%%%%%%%%%%%%%%%%%%%%%%%%%%%Prob3.3%%%%%%%%%%%%%%%%%%%%%%%%%%%%%%%%%%%%%%%
   \item[3.3] Write the contrapositive of each implication.
      \begin{enumerate}
         \item If all roses are red, then all violets are blue.
         \item $H$ is normal if $H$ is not regular.
         \item If $K$ is closed and bounded, then $K$ is compact.
      \end{enumerate}

      \textbf{Solution:} 

      \begin{enumerate}
         \item If all violets are not blue, then all roses are not red.
         \item If $H$ is regular, then $H$ is not normal.
         \item If $K$ is not compact, then $K$ is not closed or $K$ is not 
               bounded.
      \end{enumerate}
%%%%%%%%%%%%%%%%%%%%%%%%%%%%%%%%%%Prob3.4%%%%%%%%%%%%%%%%%%%%%%%%%%%%%%%%%%%%%%%
   \item[3.4] Write the converse of each implication in Exercise 3.3.

      \textbf{Solution:} 

      \begin{enumerate}
         \item If all violets are blue, then all roses are red.
         \item If $H$ is not regular, then $H$ is normal.
         \item If $K$ is compact, then $K$ is closed and bounded.
      \end{enumerate}
%%%%%%%%%%%%%%%%%%%%%%%%%%%%%%%%%%Prob3.5%%%%%%%%%%%%%%%%%%%%%%%%%%%%%%%%%%%%%%%
   \item[3.5] Write the inverse of each implication in Exercise 3.3.

      \textbf{Solution:} 

      \begin{enumerate}
         \item If all roses are not red, then all violets are not blue.
         \item If $H$ is not normal, then $H$ is regular.
         \item If $K$ is not closed or $K$ is unbounded, then $K$ is not
               compact.
      \end{enumerate}
%%%%%%%%%%%%%%%%%%%%%%%%%%%%%%%%%%Prob3.6%%%%%%%%%%%%%%%%%%%%%%%%%%%%%%%%%%%%%%%
   \item[3.6] Provide a countexample for each statement.
      \begin{enumerate}
         \item For every real number $x$, if $x^2 > 4$, then $x > 2$.
         \item For every positive integer $n$, $n^2 + n + 41$ is prime.
         \item Every triangle is a right triangle.
         \item No integer greater than 100 is prime.
         \item Every prime is an odd number.
         \item For every positive integer $n$, $3n$ is divisible by 6.
         \item If $x$ and $y$ are unequal positive integers and $xy$ is a
               perfect square, then $x$ and $y$ are perfect squares.
         \item Every real number has a reciprocal.
         \item For all real numbers $x > 0$, we have $x^2 \le x^3$.
         \item The reciprocal of a real number $x \ge 1$ is a real number $y$
               such that $0 < y < 1$.
         \item $3^n + 2$ is prime for all positive integers $n$.
         \item No rational number satisfies the equation
               $x^3 + (x - 1)^2 = x^2 + 1$.
         \item No rational number satisfies the equation
               $x^4 + (1/x) - \sqrt{x + 1} = 0$.
      \end{enumerate}

      \textbf{Solution:}

      \begin{enumerate}
         \item Let $x = -4$. We have $(-4)^2 > 4$, but $-4 < 2$.
         \item Let $n = 41$. Then $41^2 + 41 + 41 = 41 * 43$, so that
               $n^2 + n + 41$ is not prime for every positive integer $n$.
         \item Any triangle with lenghts 1, 2, and 3 is not a right triangle.
         \item The integer 103 is greater than 100, but it is prime.
         \item The number 2 is prime but it is even.
         \item For $n = 5$, $3n = 15$ is not divisible by 6.
         \item Let $x = 2$, and $y = 32$. We have $xy = 64 8^2$, so that $xy$
               is a perfect square but neither $x$ nor $y$ is a perfect square.
         \item The real number 0 does not have a reciprocal.
         \item Let $x = 0.5$. We have $0.5^2 = 0.25 > 0.125 = 0.5^3$.
         \item Let $x = 1$. The reciprocal of 1 is 1 but $1 \notin (0, 1)$.
         \item Let $n = 5$. Then $3^n + 2 = 245 = 5 * 49$, which is not prime.
         \item The rational number 0 satisfies the equation 
               $x^4 + (1/x) - \sqrt{x + 1} = 0$.
         \item The rational number $-1$ satisfies the equation 
               $x^4 + (1/x) - \sqrt{x + 1} = 0$.
      \end{enumerate}
%%%%%%%%%%%%%%%%%%%%%%%%%%%%%%%%%%Prob3.7%%%%%%%%%%%%%%%%%%%%%%%%%%%%%%%%%%%%%%%
   \item[3.7] Suppose $p$ and $q$ are integers. Prove the following.

      \begin{enumerate}
         \item If $p$ is odd and $q$ is odd, then $p + q$ is even.
         \item If $p$ is odd and $q$ is odd, then $pq$ is odd.
         \item If $p$ is odd and $q$ is even, then $p + q$ is odd.
         \item If $p$ is even and $q$ is even, then $p + q$ is even.
         \item If $p$ is even or $q$ is even, then $pq$ is even.
         \item If $pq$ is odd, then $p$ is odd and $q$ is odd.
         \item If $p^2$ is even, then $p$ is even.
         \item If $p^2$ is odd, then $p$ is odd.
      \end{enumerate}

      \textbf{Solution:}

      \begin{enumerate}
         \item Let $p = 2k_1 + 1$ and $q = 2k_2 + 1$ for some integers $k_1$ and
               $k_2$. Then $p + q = 2(k_1 + k_2 + 1)$, so that $p + q$ is even.
         \item Let $p = 2k_1 + 1$ and $q = 2k_2 + 1$ for some integers $k_1$ and
               $k_2$. Then $pq = 2(k_1k_2 + k_1 + k_2) + 1$, so that $pq$ is
               odd.
         \item Let $p = 2k_1 + 1$ and $q = 2k_2$ for some integers $k_1$ and
               $k_2$. Then $p + q = 2(k_1 + k_2) + 1$, so that $p + q$ is odd.
         \item Let $p = 2k_1$ and $q = 2k_2$ for some integers $k_1$ and $k_2$. 
               Then $p + q = 2(k_1 + k_2)$, so that $p + q$ is even.
         \item We can assume without loss of generality that $p$ is even. So
               write $p = 2k_1$ for some integer $k_1$. Thus $pq = 2(k_1q)$, so
               that $pq$ is even.
         \item This follows from (e), since (f) is the contrapositve of (e).
         \item We shall instead prove the contrapositive which says that,
               ``If $p$ is odd, then $p^2$ is odd." But this follows from (b).
         \item This follows from (f).
         
      \end{enumerate}
%%%%%%%%%%%%%%%%%%%%%%%%%%%%%%%%%%Prob3.8%%%%%%%%%%%%%%%%%%%%%%%%%%%%%%%%%%%%%%%
   \item[3.8] Let $f$ be the function given by $f(x) = 3x - 5$. Use the
              contrapositive implication to prove: If $x_1 \neq x_2$, then
              $f(x_1) \neq f(x_2)$.
      

      \textbf{Proof:} We wish to show that: If $f(x_1) = f(x_2)$, then
      $x_1 = x_2$. So suppose that $f(x_1) = f(x_2)$ for some real numbers
      $x_1$ and $x_2$. This implies that $3x_1 - 5 = 3x_2 - 5$, so that
      $3x_1 = 3x_2$. That is, $x_1 = x_2$.      
%%%%%%%%%%%%%%%%%%%%%%%%%%%%%%%%%%Prob3.9%%%%%%%%%%%%%%%%%%%%%%%%%%%%%%%%%%%%%%%
   \item[3.9] In each part, a list of hypotheses is given. These hypotheses are
              assumed to be true. Using tautologies from Example 3.12, you are
              to establish the desired conclusion. Indicate which tautology you
              are using to justify each step
      \begin{enumerate}
         \item Hypotheses: $r \Rightarrow$ $\sim$$s$, $t \Rightarrow s$ \\
               Conclusion: $r \Rightarrow$ $\sim$$t$
         \item Hypotheses: $r$, $\sim$$t, (r \land s) \Rightarrow t$ \\
               Conclusion: $\sim$$s$
         \item Hypotheses: $r \Rightarrow$ $\sim$$s$, $\sim$$r \Rightarrow$
                           $\sim$$t$, $\sim$$t \Rightarrow u, v \Rightarrow s$\\
               Conclusion: $\sim$$v \lor u$
      \end{enumerate}

      \textbf{Solution:}
      
      \begin{enumerate}
         \item By Example 3.12(c), we have $(t \Rightarrow s)$ $\Leftrightarrow$
               $(\sim$$s \Rightarrow$ $\sim$$t)$, so that the conclusion follows
               by Example 3.12(l).
         \item By Example 3.12(p), we have $[(r \land s) \Rightarrow t]$
               $\Leftrightarrow$ $[r \Rightarrow (\sim$$s \lor t)]$. The
               first and third hypotheses thus imply that $\sim$$s \lor t$.
               By Example 3.12(j), it follows that \\
               $[\sim$$t$ $\land$ $(\sim$$s \lor t)] \Rightarrow$ $\sim$$s$.
         \item If we take the conjuction of the contrapositive of the first
               hypothesis and the second and third hypothesis, we get
               $s \Rightarrow u$ by Example 3.12(m). The conjunction of the
               fourth hypothesis and $s \Rightarrow u$, says thats
               $v \Rightarrow u$, so that $\sim$$v \lor u$.
      \end{enumerate}
%%%%%%%%%%%%%%%%%%%%%%%%%%%%%%%%%%Prob3.10%%%%%%%%%%%%%%%%%%%%%%%%%%%%%%%%%%%%%%
   \item[3.10] Repeat Exercise 3.9 for the following hypotheses and conclusions.
      \begin{enumerate}
         \item Hypotheses: $\sim$$r$, $(\sim$$r \land s) \Rightarrow r$ \\
               Conclusion: $\sim$$s$
         \item Hypotheses: $\sim$$t$, $(r \lor s) \Rightarrow t$ \\
               Conclusion: $\sim$$s$
         \item Hypotheses: $r \Rightarrow$ $\sim$$s$, $t \Rightarrow u$,
                           $s \lor t$ \\
               Conclusion: $\sim$$r \lor u$
      \end{enumerate}

      \textbf{Solution:}
      
      \begin{enumerate}
         \item Example 3.12(n) says the second hypothesis is logically 
               equivalent to $\sim$$r \Rightarrow (s \Rightarrow r)$, which is
               also logically equivalent to $\sim$$r \Rightarrow (\sim$$r 
               \Rightarrow$ $\sim$$s$) which is also logically equivalent to
               $\sim$$r \Rightarrow$ $\sim$$s$. If we conjunct this with the
               first hypothesis, we can conclude $\sim$$s$ from Example 3.12(h).
         \item The second hypothesis is logically equivalent to its 
               contrapositive: $\sim$$t \Rightarrow$ $\sim$$r$
               $\land$ $\sim$$s$, so that we if we conjunct this with the
               first hypothesis, we can conclude that
               $\sim$$r$ $\land$ $\sim$$s$ by using Example 3.12(h). Then we
               can further conclude $\sim$$s$ by Example 3.12(k).
         \item The third hypothesis is logically equivalent to $\sim$$s$
               $\Rightarrow t$; call this statement $c'$. Now take the 
               conjunction of the first hypothesis, $c'$, and the second 
               hypothesis, and then conclude from Example 3.12(m) that
               $\sim$$r \lor u$.
               
      \end{enumerate}
%%%%%%%%%%%%%%%%%%%%%%%%%%%%%%%%%%Prob3.11%%%%%%%%%%%%%%%%%%%%%%%%%%%%%%%%%%%%%%
   \item[3.11] Assume that the following two hypotheses are true: (1) If the
               basketball center is healthy or the point guard is hot, then the
               team will win and the fans will be happy; and (2) if the fans are
               happy or the coach is a millionaire, then the college will 
               balance the budget. Derive the following conclusion: If the
               basketball center is healthy, then the college will balance the
               budget. Using letters to represent the simple statements, write
               out a formal proof in the format of Exercise 3.9.

      \textbf{Solution:}

      Let $p \equiv $ Basketball center is healthy. \\
      Let $q \equiv $ Point guard is hot. \\
      Let $r \equiv $ The team will win. \\
      Let $s \equiv $ The fans will be happy. \\
      Let $t \equiv $ The coach is a millionaire. \\
      Let $u \equiv $ The college will balance the budget. \\

      Hypotheses: $(p \lor q) \Rightarrow (r \land s)$,
                  $(s \lor t) \Rightarrow u$ \\
      Conclusion: $p \Rightarrow u$.

      \textbf{Proof:}
      \begin{align*}
         [(p \lor q) \Rightarrow (r \land s)] \land [(s \lor t) \Rightarrow u]
         &\Leftrightarrow [p \Rightarrow (r \land s)] \land
         [q \Rightarrow (r \land s)] \land [(s \lor t) \Rightarrow u] \\
         &\Rightarrow [p \Rightarrow (r \land s)] \land
         [(s \lor t) \Rightarrow u] \\
         &\Rightarrow (p \Rightarrow r) \land (p \Rightarrow s) \land
         [(s \lor t) \Rightarrow u] \\
         &\Rightarrow (p \Rightarrow s) \land [(s \lor t) \Rightarrow u] \\
         &\Rightarrow (p \Rightarrow s) \land (s \Rightarrow u) \land
         (t \Rightarrow u) \\
         &\Rightarrow (p \Rightarrow s) \land (s \Rightarrow u) \\
         &\Rightarrow (p \Rightarrow u)
      \end{align*}

      
\end{enumerate}
