\begin{enumerate}
%%%%%%%%%%%%%%%%%%%%%%%%%%%%%%%%%%Prob2.1%%%%%%%%%%%%%%%%%%%%%%%%%%%%%%%%%%%%%%%
   \item[2.1] Mark each statement True or False. Justify each answer.
      \begin{enumerate}
         \item The symbol ``$\forall$" means ``for every."
         \item The negation of a universal statement is another universal
               statement.
         \item The symbol ``$\ni$" is read ``such that."
      \end{enumerate}

      \textbf{Solution:} 

      \begin{enumerate}
         \item True. Definition.
         \item False. The negation of a universal statement is an existential
               statement.
         \item True. Definition.
      \end{enumerate}
%%%%%%%%%%%%%%%%%%%%%%%%%%%%%%%%%%Prob2.2%%%%%%%%%%%%%%%%%%%%%%%%%%%%%%%%%%%%%%%
   \item[2.2] Mark each statement True or False. Justify each answer.
      \begin{enumerate}
         \item The symbol ``$\exists$" means ``there exist several."
         \item If a variable is used in the antecedent of an implication without
               being quantified, then the universal quantifier is assumed to
               apply.
         \item The order in which quantifiers are used affects the truth table.
      \end{enumerate}

      \textbf{Solution:} 

      \begin{enumerate}
         \item False. It means ``there exists at least one."
         \item True. [See Text]
         \item True. [See Text]
      \end{enumerate}
%%%%%%%%%%%%%%%%%%%%%%%%%%%%%%%%%%Prob2.3%%%%%%%%%%%%%%%%%%%%%%%%%%%%%%%%%%%%%%%
   \item[2.3] Write the negation of each statement.
      \begin{enumerate}
         \item All the roads in Yellowstone are open.
         \item Some fish are green.
         \item No even integer is prime.
         \item $\exists$ $x < 3 \ni x^2 \ge 10.$
      \end{enumerate}

      \textbf{Solution:} 

      \begin{enumerate}
         \item There exists a road in Yellowstone that is not open.
         \item All fishes are not green.
         \item There exists an even prime integer.
         \item $\forall$ $x < 3$, $x^2 < 10$.
      \end{enumerate}
%%%%%%%%%%%%%%%%%%%%%%%%%%%%%%%%%%Prob2.4%%%%%%%%%%%%%%%%%%%%%%%%%%%%%%%%%%%%%%%
   \item[2.4] Write the negation of each statement.
      \begin{enumerate}
         \item Some basketball players at Central High are short.
         \item All of the lights are on.
         \item No bounded interval contains infinite many integers.
         \item $\exists$ $x$ in $S \ni x \ge 5$.
         \item $\forall$ $x \ni 0 < x < 1$, $f(x) < 2$ or $f(x) > 5$.
         \item If $x > 5$, then $\exists$ $y > 0 \ni x^2 > 25 + y$.
      \end{enumerate}

      \textbf{Solution:} 

      \begin{enumerate}
         \item All basketball players at Central High are tall.
         \item At least one of the lights is off.
         \item There exists a bounded interval that contains infinite many
               integers.
         \item $\forall$ $x$ in $S$, $x < 5$.
         \item $\exists$ $x \ni 0 < x < 1 \ni f(x) \ge 2$ and $f(x) \le 5$.
         \item $x > 5$ and $\forall$ $y > 0$, $x^2 \le 25 + y$.
      \end{enumerate}
%%%%%%%%%%%%%%%%%%%%%%%%%%%%%%%%%%Prob2.5%%%%%%%%%%%%%%%%%%%%%%%%%%%%%%%%%%%%%%%
   \item[2.5] Determine the truth value of each statement, assuming that $x$,
              $y$, and $z$ are real numbers.
      \begin{enumerate}
         \item $\exists$ $x \ni \forall$ $y$ $\exists$ $z \ni x + y = z$.
         \item $\exists$ $x \ni \forall$ $y$ and $\forall$ $z, x + y = z$.
         \item $\forall$ $x$ and $\forall$ $y$, $\exists$ $z \ni y - z = x$.
         \item $\forall$ $x$ and $\forall$ $y$, $\exists$ $z \ni xz = y$.
         \item $\exists$ $x \ni \forall$ $y$ and $\forall$ $z, z > y$ implies 
               that $z > x + y$.
         \item $\forall$ $x, \exists$ $y$ and $\exists$ $z \ni z > y$ implies 
               that $z > x + y$.
      \end{enumerate}

      \textbf{Solution:} 

      \begin{enumerate}
         \item True. Choose $x = 0$. Let $y$ be any real number and choose
               $z = y$.
         \item False. Letting $y = -x$ and $z = 1$ would imply that $1 = 0$.
         \item True. Let $x$ and $y$ be any real numbers. Choose $z = y - x$.
         \item False. Letting $x = 0$ and $y = 1$ would imply that $0 = 1$.
         \item True. Choose $x = -1$. Let $y$ and $z$ be real numbers with
               $z > y$. We have $0 > x$, so that $y > x + y$. Thus $z > x + y$.
         \item True.
      \end{enumerate}
%%%%%%%%%%%%%%%%%%%%%%%%%%%%%%%%%%Prob2.6%%%%%%%%%%%%%%%%%%%%%%%%%%%%%%%%%%%%%%%
   \item[2.6] Determine the truth value of each statement, assuming that $x$,
              $y$, and $z$ are real numbers.
      \begin{enumerate}
         \item $\forall$ $x$ and $\forall$ $y$, $\exists$ $z \ni x + y = z$.
         \item $\forall$ $x$ $\exists$ $y \ni$  $\forall$ $z, x + y = z$.
         \item $\exists$ $x \ni$ $\forall$ $y$, $\exists$ $z \ni xz = y$.
         \item $\forall$ $x$ and $\forall$ $y$, $\exists$ $z \ni yz = x$.
         \item $\forall$ $x$ $\exists$ $y \ni$ $\forall$ $z, z > y$ implies that
               $z > x + y$.
         \item $\forall$ $x$ and $\forall$ $y$, $\exists$ $z \ni z > y$ implies
               that $z > x + y$.
      \end{enumerate}

      \textbf{Solution:}

      \begin{enumerate}
         \item True. Let $x$ and $y$ be any real numbers. Choose $z = x + y$.
         \item False. Let $x$ be any real number. Choose $z = x + y + 1$.
         \item True. Choose $x = 1$.
         \item False. Let $x = 1$ and $y = 0$.
         \item False. Let $x = 1$. Choose $z = y + 1$.
         \item True. Choose $z \le y$ to make the antecedent false.
      \end{enumerate}
%%%%%%%%%%%%%%%%%%%%%%%%%%%%%%%%%%Prob2.7%%%%%%%%%%%%%%%%%%%%%%%%%%%%%%%%%%%%%%%
   \item[2.7] Below are two strategies for determining the truth value of a
              statement involving a positive number $x$ and another statment
              $P(x)$.

      \begin{quote}
         \begin{enumerate}
            \item[(\textit{i})]  Find some $x > 0$ such that $P(x)$ is true.
            \item[(\textit{ii})] Let $x$ be the name for any number greater 
                                 than 0 and show $P(x)$ is true.
         \end{enumerate}
      \end{quote}

      For each statement below, indicate which strategy is more appropriate.

      \begin{enumerate}
         \item $\forall$ $x > 0$, $P(x)$.
         \item $\exists$ $x > 0 \ni P(x)$.
         \item $\exists$ $x > 0 \ni$ $\sim$$P(x)$.
         \item $\forall$ $x > 0$, $\sim$$P(x)$.
      \end{enumerate}

      \textbf{Solution:}

      \begin{enumerate}
         \item \textit{ii}.
         \item \textit{i}.
         \item \textit{i}.
         \item \textit{ii}.
      \end{enumerate}
%%%%%%%%%%%%%%%%%%%%%%%%%%%%%%%%%%Prob2.8%%%%%%%%%%%%%%%%%%%%%%%%%%%%%%%%%%%%%%%
   \item[2.8] Which of the following best identifies $f$ as a constant 
              function, where $x$ and $y$ are real numbers.
      \begin{enumerate}
         \item $\exists$ $x \ni$ $\forall$ $y$, $f(x) = y$.
         \item $\forall$ $x$ $\exists$ $y \ni$ $f(x) = y$.
         \item $\exists$ $y \ni$ $\forall$ $x$, $f(x) = y$.
         \item $\forall$ $y$ $\exists$ $x \ni$ $f(x) = y$.
      \end{enumerate}

      \textbf{Solution:}

      \begin{enumerate}
         \item $f$ is not a function.
         \item This just says that the domain of $f$ is the real numbers.
         \item This says that $f$ is a constant function.
         \item This says that $f$ is surjective.
      \end{enumerate}
%%%%%%%%%%%%%%%%%%%%%%%%%%%%%%%%%%Prob2.9%%%%%%%%%%%%%%%%%%%%%%%%%%%%%%%%%%%%%%%
   \item[2.9] Determine the truth value of each statement, assuming $x$ is a 
              real number.
      \begin{enumerate}
         \item $\exists$ $x$ in $[2,4] \ni x < 7$.
         \item $\forall$ $x$ in $[2,4], x < 7$.
         \item $\exists$ $x \ni x^2 = 5$.
         \item $\forall$ $x$, $x^2 = 5$.
         \item $\exists$ $x \ni x^2 \neq -3$.
         \item $\forall$ $x$, $x^2 \neq -3$.
         \item $\exists$ $x \ni x \div x = 1$.
         \item $\forall$ $x$, $x \div x = 1$.
      \end{enumerate}

      \textbf{Solution:}
      
      \begin{enumerate}
         \item True. Choose $x = 2$.
         \item True, since $2 \le x \le 4$.
         \item True. Choose $x = \sqrt{5}$.
         \item False. We have $0^2 = 0 \neq 5$.
         \item True. We have $0^2 = 0 \neq -3$.
         \item True, since $x^2 \ge 0 > -3$.
         \item True. Choose $x = 1$.
         \item False, since $0 \div 0$ is undefined.
      \end{enumerate}
%%%%%%%%%%%%%%%%%%%%%%%%%%%%%%%%%%Prob2.10%%%%%%%%%%%%%%%%%%%%%%%%%%%%%%%%%%%%%%
   \item[2.10] Determine the truth value of each statement, assuming $x$ is a 
               real number.
      \begin{enumerate}
         \item $\exists$ $x$ in $[3,5] \ni x \ge 4$.
         \item $\forall$ $x$ in $[3,5], x \ge 4$.
         \item $\exists$ $x \ni x^2 \neq 3$.
         \item $\forall$ $x$, $x^2 \neq 3$.
         \item $\exists$ $x \ni x^2 \neq -5$.
         \item $\forall$ $x$, $x^2 \neq -5$.
         \item $\exists$ $x \ni x - x = 0$.
         \item $\forall$ $x$, $x - x = 0$.
      \end{enumerate}

      \textbf{Solution:}
      
      \begin{enumerate}
         \item True. Choose $x = 4$.
         \item False, since $x = 3 < 4$.
         \item True. Choose $x = 1$.
         \item False. $\sqrt{3}^2 = 3$.
         \item True. We have $0^2 = 0 \neq -5$.
         \item True, since $x^2 \ge 0 > -5$.
         \item True. Choose $x = 0$.
         \item True. 
      \end{enumerate}
\end{enumerate}

\noindent In exercises 2.11 to 2.19 you are to do two things: (a) rewrite the 
defining conditions in logical symbolism using $\forall$, $\exists$, $\ni$, and
$\Rightarrow$, as appropriate; and (b) write the negation of part (a) using the
same symbolism.

\begin{enumerate}
%%%%%%%%%%%%%%%%%%%%%%%%%%%%%%%%%%Prob2.11%%%%%%%%%%%%%%%%%%%%%%%%%%%%%%%%%%%%%%
   \item[2.11] A function $f$ is \textit{even} iff for every $x$,
               $f(-x) = f(x)$.

      \textbf{Solution:}
      
      \begin{enumerate}
         \item $\forall$ $x$, $f(-x) = f(x)$.
         \item $\exists$ $x \ni f(-x) \neq f(x)$.
      \end{enumerate}
%%%%%%%%%%%%%%%%%%%%%%%%%%%%%%%%%%Prob2.12%%%%%%%%%%%%%%%%%%%%%%%%%%%%%%%%%%%%%%
   \item[2.12] A function $f$ is \textit{periodic} iff there exists a $k > 0$
               such that for every $x$, $f(x + k) = f(x)$.

      \textbf{Solution:}
      
      \begin{enumerate}
         \item $\exists$ $k > 0 \ni$ $\forall$ $x$, $f(x + k) = f(x)$.
         \item $\forall$ $k > 0$ $\exists$ $x \ni f(x + k) \neq f(x)$. 
      \end{enumerate}
%%%%%%%%%%%%%%%%%%%%%%%%%%%%%%%%%%Prob2.13%%%%%%%%%%%%%%%%%%%%%%%%%%%%%%%%%%%%%%
   \item[2.13] A function $f$ is \textit{increasing} iff for every $x$ and for
               every $y$, if $x \le y$, then $f(x) \le f(y)$.

      \textbf{Solution:}
      
      \begin{enumerate}
         \item $\forall$ $x$ and $\forall$ $y$, $x \le y \Rightarrow
               f(x) \le f(y)$.
         \item $\exists$ $x$ and $\exists$ $y \ni x \le y$ and $f(x) > f(y)$.
      \end{enumerate}
%%%%%%%%%%%%%%%%%%%%%%%%%%%%%%%%%%Prob2.14%%%%%%%%%%%%%%%%%%%%%%%%%%%%%%%%%%%%%%
   \item[2.14] A function $f$ is \textit{strictly decreasing} iff for every $x$ 
               and for every $y$, if $x < y$, then $f(x) > f(y)$.

      \textbf{Solution:}
      
      \begin{enumerate}
         \item $\forall$ $x$ and $\forall$ $y$, $x < y \Rightarrow
               f(x) > f(y)$.
         \item $\exists$ $x$ and $\exists$ $y \ni x < y$ and $f(x) \le f(y)$.
      \end{enumerate}
%%%%%%%%%%%%%%%%%%%%%%%%%%%%%%%%%%Prob2.15%%%%%%%%%%%%%%%%%%%%%%%%%%%%%%%%%%%%%%
   \item[2.15] A function $f : A \rightarrow B$ is \textit{injective} iff for 
               every $x$ and $y$ in $A$, if $f(x) = f(y)$, then $x = y$.

      \textbf{Solution:}
      
      \begin{enumerate}
         \item $\forall$ $x$ and $y$ in $A$, $f(x) = f(y) \Rightarrow x = y$.
         \item $\exists$ $x$ and $y$ in $A \ni f(x) = f(y)$ and $x \neq y$.
      \end{enumerate}
%%%%%%%%%%%%%%%%%%%%%%%%%%%%%%%%%%Prob2.16%%%%%%%%%%%%%%%%%%%%%%%%%%%%%%%%%%%%%%
   \item[2.16] A function $f : A \rightarrow B$ is \textit{surjective} iff for 
               every $y$ in $B$ there exists an $x$ in $A$ such that $f(x) = y$.

      \textbf{Solution:}
      
      \begin{enumerate}
         \item $\forall$ $y$ in $B$ $\exists$ $x$ in $A \ni f(x) = y$.
         \item $\exists$ $y$ in $B \ni$ $\forall$ $x$ in $A$, $f(x) \neq y$.
      \end{enumerate}
%%%%%%%%%%%%%%%%%%%%%%%%%%%%%%%%%%Prob2.17%%%%%%%%%%%%%%%%%%%%%%%%%%%%%%%%%%%%%%
   \item[2.17] A function $f : D \rightarrow R$ is \textit{continuous} at
               $c \in D$ iff for every $\varepsilon > 0$ there is a $\delta > 0$
               such that $|f(x) - f(c)| < \varepsilon$ whenever
               $|x - c| < \delta$ and $x \in D$.

      \textbf{Solution:}
      
      \begin{enumerate}
         \item $\forall$ $\varepsilon > 0$, $\exists$ $\delta > 0 \ni$
               $|x - c| < \delta$ and $x$ in $D$
               $\Rightarrow |f(x) -f(c)| < \varepsilon$.
         \item $\exists$ $\varepsilon > 0 \ni$ $\forall$ $\delta > 0$,
               $|x - c| < \delta$ and $x$ in $D$ and
               $|f(x) - f(c)| \ge \varepsilon$.
      \end{enumerate}
%%%%%%%%%%%%%%%%%%%%%%%%%%%%%%%%%%Prob2.18%%%%%%%%%%%%%%%%%%%%%%%%%%%%%%%%%%%%%%
   \item[2.18] A function $f$ is \textit{uniformly continuous on a set} $S$ iff
               for every $\varepsilon > 0$ there is a $\delta > 0$ such that
               $|f(x) - f(y)| < \varepsilon$ whenever $x$ and $y$ are in $S$ and
               $|x - y| < \delta$.

      \textbf{Solution:}
      
      \begin{enumerate}
         \item $\forall$ $\varepsilon > 0$, $\exists$ $\delta > 0 \ni$
               $|x - y| < \delta$ and $x$ and $y$ are in $D$
               $\Rightarrow |f(x) - f(y)| < \varepsilon$.
         \item $\exists$ $\varepsilon > 0 \ni$ $\forall$ $\delta > 0$,
               $|x - y| < \delta$ and $x$ and $y$ are in $D$ and
               $|f(x) - f(y)| \ge \varepsilon$.
      \end{enumerate}
%%%%%%%%%%%%%%%%%%%%%%%%%%%%%%%%%%Prob2.19%%%%%%%%%%%%%%%%%%%%%%%%%%%%%%%%%%%%%%
   \item[2.19] The real number $L$ is the \textit{limit} of the function
               $f : D \rightarrow R$ at the point $c$ iff for each
               $\varepsilon > 0$ there exists a $\delta > 0$ such that
               $|f(x) - L| < \varepsilon$ whenever $x \in D$ and
               $0 < |x - c| < \delta$.

      \textbf{Solution:}
      
      \begin{enumerate}
         \item $\forall$ $\varepsilon > 0$, $\exists$ $\delta > 0 \ni$
               $0 < |x - c| < \delta$ and $x$ in $D$
               $\Rightarrow |f(x) - L| < \varepsilon$.
         \item $\exists$ $\varepsilon > 0 \ni$ $\forall$ $\delta > 0$,
               $0 < |x - c| < \delta$ and $x$ in $D$ and
               $|f(x) - L| \ge \varepsilon$.
      \end{enumerate}
\end{enumerate}
